% \iffalse meta-comment
%
% File: epstopdf.dtx
% Version: 2016/05/15 v2.6
% Info: Conversion with epstopdf on the fly
%
% Copyright (C) 2001, 2006-2010 by
%    Heiko Oberdiek <heiko.oberdiek at googlemail.com>
%
% This work may be distributed and/or modified under the
% conditions of the LaTeX Project Public License, either
% version 1.3c of this license or (at your option) any later
% version. This version of this license is in
%    http://www.latex-project.org/lppl/lppl-1-3c.txt
% and the latest version of this license is in
%    http://www.latex-project.org/lppl.txt
% and version 1.3 or later is part of all distributions of
% LaTeX version 2005/12/01 or later.
%
% This work has the LPPL maintenance status "maintained".
%
% This Current Maintainer of this work is Heiko Oberdiek.
%
% This work consists of the main source file epstopdf.dtx
% and the derived files
%    epstopdf.sty, epstopdf.pdf, epstopdf.ins, epstopdf.drv,
%    epstopdf-base.sty, epstopdf-test1.tex.
%
% Distribution:
%    CTAN:macros/latex/contrib/oberdiek/epstopdf.dtx
%    CTAN:macros/latex/contrib/oberdiek/epstopdf.pdf
%
% Unpacking:
%    (a) If epstopdf.ins is present:
%           tex epstopdf.ins
%    (b) Without epstopdf.ins:
%           tex epstopdf.dtx
%    (c) If you insist on using LaTeX
%           latex \let\install=y% \iffalse meta-comment
%
% File: epstopdf.dtx
% Version: 2016/05/15 v2.6
% Info: Conversion with epstopdf on the fly
%
% Copyright (C) 2001, 2006-2010 by
%    Heiko Oberdiek <heiko.oberdiek at googlemail.com>
%
% This work may be distributed and/or modified under the
% conditions of the LaTeX Project Public License, either
% version 1.3c of this license or (at your option) any later
% version. This version of this license is in
%    http://www.latex-project.org/lppl/lppl-1-3c.txt
% and the latest version of this license is in
%    http://www.latex-project.org/lppl.txt
% and version 1.3 or later is part of all distributions of
% LaTeX version 2005/12/01 or later.
%
% This work has the LPPL maintenance status "maintained".
%
% This Current Maintainer of this work is Heiko Oberdiek.
%
% This work consists of the main source file epstopdf.dtx
% and the derived files
%    epstopdf.sty, epstopdf.pdf, epstopdf.ins, epstopdf.drv,
%    epstopdf-base.sty, epstopdf-test1.tex.
%
% Distribution:
%    CTAN:macros/latex/contrib/oberdiek/epstopdf.dtx
%    CTAN:macros/latex/contrib/oberdiek/epstopdf.pdf
%
% Unpacking:
%    (a) If epstopdf.ins is present:
%           tex epstopdf.ins
%    (b) Without epstopdf.ins:
%           tex epstopdf.dtx
%    (c) If you insist on using LaTeX
%           latex \let\install=y% \iffalse meta-comment
%
% File: epstopdf.dtx
% Version: 2016/05/15 v2.6
% Info: Conversion with epstopdf on the fly
%
% Copyright (C) 2001, 2006-2010 by
%    Heiko Oberdiek <heiko.oberdiek at googlemail.com>
%
% This work may be distributed and/or modified under the
% conditions of the LaTeX Project Public License, either
% version 1.3c of this license or (at your option) any later
% version. This version of this license is in
%    http://www.latex-project.org/lppl/lppl-1-3c.txt
% and the latest version of this license is in
%    http://www.latex-project.org/lppl.txt
% and version 1.3 or later is part of all distributions of
% LaTeX version 2005/12/01 or later.
%
% This work has the LPPL maintenance status "maintained".
%
% This Current Maintainer of this work is Heiko Oberdiek.
%
% This work consists of the main source file epstopdf.dtx
% and the derived files
%    epstopdf.sty, epstopdf.pdf, epstopdf.ins, epstopdf.drv,
%    epstopdf-base.sty, epstopdf-test1.tex.
%
% Distribution:
%    CTAN:macros/latex/contrib/oberdiek/epstopdf.dtx
%    CTAN:macros/latex/contrib/oberdiek/epstopdf.pdf
%
% Unpacking:
%    (a) If epstopdf.ins is present:
%           tex epstopdf.ins
%    (b) Without epstopdf.ins:
%           tex epstopdf.dtx
%    (c) If you insist on using LaTeX
%           latex \let\install=y% \iffalse meta-comment
%
% File: epstopdf.dtx
% Version: 2016/05/15 v2.6
% Info: Conversion with epstopdf on the fly
%
% Copyright (C) 2001, 2006-2010 by
%    Heiko Oberdiek <heiko.oberdiek at googlemail.com>
%
% This work may be distributed and/or modified under the
% conditions of the LaTeX Project Public License, either
% version 1.3c of this license or (at your option) any later
% version. This version of this license is in
%    http://www.latex-project.org/lppl/lppl-1-3c.txt
% and the latest version of this license is in
%    http://www.latex-project.org/lppl.txt
% and version 1.3 or later is part of all distributions of
% LaTeX version 2005/12/01 or later.
%
% This work has the LPPL maintenance status "maintained".
%
% This Current Maintainer of this work is Heiko Oberdiek.
%
% This work consists of the main source file epstopdf.dtx
% and the derived files
%    epstopdf.sty, epstopdf.pdf, epstopdf.ins, epstopdf.drv,
%    epstopdf-base.sty, epstopdf-test1.tex.
%
% Distribution:
%    CTAN:macros/latex/contrib/oberdiek/epstopdf.dtx
%    CTAN:macros/latex/contrib/oberdiek/epstopdf.pdf
%
% Unpacking:
%    (a) If epstopdf.ins is present:
%           tex epstopdf.ins
%    (b) Without epstopdf.ins:
%           tex epstopdf.dtx
%    (c) If you insist on using LaTeX
%           latex \let\install=y\input{epstopdf.dtx}
%        (quote the arguments according to the demands of your shell)
%
% Documentation:
%    (a) If epstopdf.drv is present:
%           latex epstopdf.drv
%    (b) Without epstopdf.drv:
%           latex epstopdf.dtx; ...
%    The class ltxdoc loads the configuration file ltxdoc.cfg
%    if available. Here you can specify further options, e.g.
%    use A4 as paper format:
%       \PassOptionsToClass{a4paper}{article}
%
%    Programm calls to get the documentation (example):
%       pdflatex epstopdf.dtx
%       makeindex -s gind.ist epstopdf.idx
%       pdflatex epstopdf.dtx
%       makeindex -s gind.ist epstopdf.idx
%       pdflatex epstopdf.dtx
%
% Installation:
%    TDS:tex/latex/oberdiek/epstopdf.sty
%    TDS:tex/latex/oberdiek/epstopdf-base.sty
%    TDS:doc/latex/oberdiek/epstopdf.pdf
%    TDS:doc/latex/oberdiek/test/epstopdf-test1.tex
%    TDS:source/latex/oberdiek/epstopdf.dtx
%
%<*ignore>
\begingroup
  \catcode123=1 %
  \catcode125=2 %
  \def\x{LaTeX2e}%
\expandafter\endgroup
\ifcase 0\ifx\install y1\fi\expandafter
         \ifx\csname processbatchFile\endcsname\relax\else1\fi
         \ifx\fmtname\x\else 1\fi\relax
\else\csname fi\endcsname
%</ignore>
%<*install>
\input docstrip.tex
\Msg{************************************************************************}
\Msg{* Installation}
\Msg{* Package: epstopdf 2016/05/15 v2.6 Conversion with epstopdf on the fly (HO)}
\Msg{************************************************************************}

\keepsilent
\askforoverwritefalse

\let\MetaPrefix\relax
\preamble

This is a generated file.

Project: epstopdf
Version: 2016/05/15 v2.6

Copyright (C) 2001, 2006-2010 by
   Heiko Oberdiek <heiko.oberdiek at googlemail.com>

This work may be distributed and/or modified under the
conditions of the LaTeX Project Public License, either
version 1.3c of this license or (at your option) any later
version. This version of this license is in
   http://www.latex-project.org/lppl/lppl-1-3c.txt
and the latest version of this license is in
   http://www.latex-project.org/lppl.txt
and version 1.3 or later is part of all distributions of
LaTeX version 2005/12/01 or later.

This work has the LPPL maintenance status "maintained".

This Current Maintainer of this work is Heiko Oberdiek.

This work consists of the main source file epstopdf.dtx
and the derived files
   epstopdf.sty, epstopdf.pdf, epstopdf.ins, epstopdf.drv,
   epstopdf-base.sty, epstopdf-test1.tex.

\endpreamble
\let\MetaPrefix\DoubleperCent

\generate{%
  \file{epstopdf.ins}{\from{epstopdf.dtx}{install}}%
  \file{epstopdf.drv}{\from{epstopdf.dtx}{driver}}%
  \usedir{tex/latex/oberdiek}%
  \file{epstopdf.sty}{\from{epstopdf.dtx}{package}}%
  \file{epstopdf-base.sty}{\from{epstopdf.dtx}{base}}%
  \usedir{doc/latex/oberdiek/test}%
  \file{epstopdf-test1.tex}{\from{epstopdf.dtx}{test1}}%
  \nopreamble
  \nopostamble
  \usedir{source/latex/oberdiek/catalogue}%
  \file{epstopdf-pkg.xml}{\from{epstopdf.dtx}{catalogue}}%
}

\catcode32=13\relax% active space
\let =\space%
\Msg{************************************************************************}
\Msg{*}
\Msg{* To finish the installation you have to move the following}
\Msg{* files into a directory searched by TeX:}
\Msg{*}
\Msg{*     epstopdf.sty, epstopdf-base.sty}
\Msg{*}
\Msg{* To produce the documentation run the file `epstopdf.drv'}
\Msg{* through LaTeX.}
\Msg{*}
\Msg{* Happy TeXing!}
\Msg{*}
\Msg{************************************************************************}

\endbatchfile
%</install>
%<*ignore>
\fi
%</ignore>
%<*driver>
\NeedsTeXFormat{LaTeX2e}
\ProvidesFile{epstopdf.drv}%
  [2016/05/15 v2.6 Conversion with epstopdf on the fly (HO)]%
\documentclass{ltxdoc}
\usepackage{holtxdoc}[2011/11/22]
\begin{document}
  \DocInput{epstopdf.dtx}%
\end{document}
%</driver>
% \fi
%
%
% \CharacterTable
%  {Upper-case    \A\B\C\D\E\F\G\H\I\J\K\L\M\N\O\P\Q\R\S\T\U\V\W\X\Y\Z
%   Lower-case    \a\b\c\d\e\f\g\h\i\j\k\l\m\n\o\p\q\r\s\t\u\v\w\x\y\z
%   Digits        \0\1\2\3\4\5\6\7\8\9
%   Exclamation   \!     Double quote  \"     Hash (number) \#
%   Dollar        \$     Percent       \%     Ampersand     \&
%   Acute accent  \'     Left paren    \(     Right paren   \)
%   Asterisk      \*     Plus          \+     Comma         \,
%   Minus         \-     Point         \.     Solidus       \/
%   Colon         \:     Semicolon     \;     Less than     \<
%   Equals        \=     Greater than  \>     Question mark \?
%   Commercial at \@     Left bracket  \[     Backslash     \\
%   Right bracket \]     Circumflex    \^     Underscore    \_
%   Grave accent  \`     Left brace    \{     Vertical bar  \|
%   Right brace   \}     Tilde         \~}
%
% \GetFileInfo{epstopdf.drv}
%
% \title{The \xpackage{epstopdf} package}
% \date{2016/05/15 v2.6}
% \author{Heiko Oberdiek\thanks
% {Please report any issues at https://github.com/ho-tex/oberdiek/issues}\\
% \xemail{heiko.oberdiek at googlemail.com}}
%
% \maketitle
%
% \begin{abstract}
% This packages adds support of handling eps images
% to package \xpackage{graphics} or \xpackage{graphicx}
% with option \xoption{pdftex}. If an eps image is detected,
% epstopdf is automatically called to convert it to pdf format.
% \end{abstract}
%
% \tableofcontents
%
% \providecommand*{\xprogram}[1]{\textsf{#1}}
%
% \section{Documentation}
%
% \subsection{Introduction}
%
% \LaTeX\ provides its graphics bundle to include graphics files.
% Both packages \xpackage{graphics} or \xpackage{graphicx} may
% be used. the latter one loads the first and adds options
% in key value style for \cs{includegraphics}.
%
% Usually the drivers do not support all kind of graphics files.
% Other image types must be converted, before they become
% usuable. In case of driver \xprogram{dvips}, the graphics
% rule may contain a conversion rule. Then all that package
% \xpackage{graphics} must know is the bounding box, the command
% is passed to \xprogram{dvips} that calls it and embeds the
% converted image.
%
% However, \pdfTeX\ has its driver for PDF output already
% build in. It's graphics inclusion commands (\cs{pdfximage})
% does not allow the execution of external commands.
% Therefore commands in the last argument of
% \cs{DeclareGraphicsRule} were of no use.
% But external programs can be called within \pdfTeX.
% This feature is called ``shell escape'' or ``write 18''
% and must usually enabled explicitely because of security reasons.
% Now, this package \xpackage{epstopdf} hooks
% into package \xpackage{graphics}' code to catch
% that argument with the external command and executes it
% to convert the graphics file to a supported format and
% passes the control of graphics inclusion back to package
% \xpackage{graphics}.
%
% \subsection{Requirements}
%
% \begin{itemize}
% \item
%   The feature \cs{write18} must be enabled. This allows the
%   running of external programs during \TeX's compile run.
%   Keep in mind that this is a security risk. The feature is an
%   addition to \cs{TeX}. Mik\TeX, te\TeX, \TeX\ Live support it.
%   In Web2C based \TeX\ distributions (te\TeX, \TeX\ Live) it can be
%   enabled in the configuration file \xfile{texmf.cnf}:
%   \begin{quote}
%     |shell_escape = 1|
%   \end{quote}
%   Because of the security risk, it is better to do it on the
%   command line only:
%   \begin{quote}
%   \texttt{--shell-escape} (te\TeX, \TeX\ Live)\\
%   \texttt{--enable-write18} (MiK\TeX)
%   \end{quote}
%   Example:
%   \begin{quote}
%     |pdflatex -shell-escape test.tex|
%   \end{quote}
% \item
%   The program \xprogram{epstopdf} for the conversion from EPS
%   to PDF. However, other programs can
%   be used and configured by \cs{DeclareGraphicsRule}.
%   Example:
%   \begin{quote}
%   |\epstopdfDeclareGraphicsRule{.eps}{pdf}{.pdf}{%|\\
%   |  ps2pdf -dEPSCrop #1 \OutputFile|\\
%   |}|
%   \end{quote}
% \end{itemize}
%
% \subsection{Usage}
%
% The package is loaded after graphic\{s,x\}, e.g.:
% \begin{quote}
%   |\usepackage[pdftex]{graphicx}|\\
%   |\usepackage{epstopdf}|
% \end{quote}
% Now images with file name extension \xfile{.eps} are detected
% and supported using \cs{includegraphics}.
%
% If the graphics file name is explicitly specified with extension
% \xfile{.eps} the new rule for EPS files is called and
% the conversion performed. If option \xoption{update} is in force
% then the conversion step is dropped if the target file already
% exists and is not older then the EPS file.
%
% The situation is more complicate if the graphics file is
% given without file name extension. Then the \xpackage{graphics}
% package must search for a supported image file.
% The possible extensions are stored in the graphics extension
% list, that can be set by \cs{DeclareGraphicsExtensions}.
% The algorithm:
% \begin{quote}
%    \def\>{\mbox{}\quad}
%    function search( \meta{filebase} )\\
%    \>foreach \meta{ext} in \meta{graphics extensions}\\
%    \>\>foreach \meta{dir} in \meta{current directory}, \meta{\cs{graphicspath}}\\
%    \>\>\>\meta{file} := \meta{dir} + \meta{filebase} + \meta{ext}\\
%    \>\>\>if exist \meta{file}\\
%    \>\>\>\>return found\\
%    \>return not found
% \end{quote}
% Package \xpackage{epstopdf} puts \xfile{.eps} at the end of the
% graphics extension search list. This is the behaviour of option
% \xoption{append} that is enabled by default.
% That means, the conversion is called last
% unless a supported file type cannot be found earlier. This avoids
% unnecessary conversion steps that slow down the \LaTeX\ run.
% If you want to use option \xoption{update} and your \pdfTeX\ supports
% it, then an outdated PDF file also would be found earlier unless
% \xoption{suffix} is used that is the default since version 2.0.
%
% With an empty option \xoption{suffix} and option
% \xoption{prepend} there is a risk that an original PDF file is overwritten:
%
% If the original image file is the PDF file and there is also a generated
% EPS file, then the original PDF file can be regenerated (depending
% on the option settings) and the original PDF file gets lost.
% Therefore option \xoption{suffix} is introduced in version 1.9
% to create a separate name space for generated output files.
%
% \textbf{Note:} Usually the conversion program needs the exact location
% of the image file. Usually the current directory works.
% Also if the image file is found using \cs{graphicspath}, the
% location is known. However, if the image is somewhere
% in a directory of environment variable |TEXINPUTS|, then
% the package does not know the exact location and the
% conversion program will not find the image file unless it
% implements a search using |TEXINPUTS| (program \xprogram{kpsewhich}
% may be of help in this task).
%
% \subsection{Options}
%
% Options can be given as package options or later using:
% \begin{declcs}{epstopdfsetup} \M{key value list}
% \end{declcs}
% \LaTeX\ expands the option list before passing the option list
% to the package's option handling code. This can fail
% for option \xoption{suffix} if it contains some of the macros
% described below. Use \cs{epstopdfsetup} after the package is loaded.
% Or load package \xpackage{kvoptions-patch} before. This package
% is also loaded by option \xoption{patch} of package \xpackage{kvoptions}.
% \LaTeX's option code is redefined to respect key value options and
% let the values untouched.
%
% \begin{description}
% \def\It#1{\item[\xoption{#1}:]}^^A
% \It{update}
%   The conversion program is only called, if the target file
%   does not exist or is older than the source image file.
% \It{append}
%   Puts the extension \xfile{.eps} at the end of the graphics extension list
%   (default).
% \It{prepend}
%   Puts the extension \xfile{.eps} at the begin of the graphics extension list.
% \It{outdir}
%   The converted file may put in an other output directory.
%   The value of \xoption{outdir} must include the directory separator.
%   Example for the current directory:
%   \begin{quote}
%   |\epstopdfsetup{outdir=./}|
%   \end{quote}
%   For other directories ensure, that they can be found.
%   See \cs{graphicspath} or |TEXINPUTS|.
% \It{suffix}
%   This option takes a string that is put between the file name base
%   and the extension of the output file. Rationale: It can happen,
%   that a PDF file is the original file and the EPS file the
%   generated file. If now the package thinks, that the PDF file
%   is the generated file, it will `regnerate' it. But in reality
%   the original file is lost. Therefore I recommend to use this
%   option always to generate a separate name space for generated
%   files. Proposed value is \texttt{-generated} or \texttt{.generated}.
%   The suffix \texttt{.generated} will also work here without the need for
%   package \xpackage{grffile}).
%
%   Example:
%   \begin{quote}
%   |\epstopdfsetup{suffix=-generated}|\\
%   Then \xfile{foo.eps} is converted to \xfile{foo-generated.pdf}.
%   \end{quote}
%   \cs{SourceExt} can be used inside the suffix string.
%   It's will be replaced by the extension of the image source
%   file without the leading dot, for instance:
%   \begin{quote}
%   |\epstopdfsetup{suffix=-\SourceExt-converted-to}|\\
%   \xfile{foo.eps} $\Rightarrow$ \xfile{foo-eps-converted-to.pdf}
%   \end{quote}
%   See also the next option \xoption{prefersuffix} that
%   modifies the behaviour of option \xoption{suffix} in some
%   cases.
%
%   Default for \xoption{suffix} is `|-\SourceExt-converted-to|'.
%
% \It{prefersuffix}
%   If a suffix is set by option \xoption{suffix}, then there can be
%   two image file names that could be taken into account for inclusion:
%   A image file name with the suffix string inside
%   its name and a image file name without; e.g. for \xfile{foo.eps}
%   the names could be:
%   \begin{quote}
%   \xfile{foo-suffix.pdf}, \xfile{foo.pdf}
%   \end{quote}
%   If option \xoption{perfersuffix} is turned on, the
%   file \xfile{foo-suffix.pdf} and its generation is preferred
%   over using \xfile{foo.pdf}. Otherwise \xfile{foo.pdf} is included
%   without generating \xfile{foo-suffix.pdf}.
%   The default of option \xoption{prefersuffix} is \texttt{true}.
% \It{program@epstopdf}
%   The name for the conversion program from EPS to PDF,
%   default is `|epstopdf|'.
% \It{verbose}
%   It prints some information about the image in the \xfile{.log} file
%   (default).
% \end{description}
%
% \subsection{Configuration}
%
% \subsubsection{System configuration file \xfile{epstopdf-sys.cfg}}
%
% If \xfile{epstopdf-sys} exists it is loaded at the end of the
% package \xpackage{epstopdf-base} and before the user configuration
% file. It's intended for TeX distributors. Thus they could add
% additional conversion rules (e.g., .gif -> .png) or set options.
%
% \subsubsection{User configuration file \xfile{epstopdf.cfg}}
%
% A configuration file \xfile{epstopdf.cfg} is loaded at the end
% of the package if it exists. It can be used for changing
% the default option setting. Example:
% \begin{quote}
% |\epstopdfsetup{verbose=false}|
% \end{quote}
%
% \subsubsection{Conversion program}
%
% You can use \cs{DeclareGraphicsRule} in a similar way as the
% route via \xprogram{dvips} to specify the conversion
% command line.
% The conversion argument starts with a back tick, followed
% by the conversion command including parameters.
%
% The whole conversion argument should also be wrapped inside
% \cs{epstopdfcall}. This reduces the problem with packages
% (e.g. \xpackage{pst-pdf}) that use the conversion argument
% and expands it. Macros \cs{SourceFile}, \cs{OutputFile}, and
% \cs{SourceExt} are not defined outside \xpackage{epstopdf-base}'s
% \cs{Gin@setfile} and error messages because of undefined
% command names are the result. If \cs{epstopdfcall} detects that
% it is called outside \xpackage{epstopdf-base}'s \cs{Gin@setfile}
% then it replaces the conversion argument by
% package \xpackage{graphics}'s
% default, usually the image file.
%
% The following macros are available inside:
% \begin{description}
% \def\It#1{\item[\cs{#1}:]}
% \It{OutputFile}: output file name (with known path and extension)
% \It{SourceFile}: source file name (with known path and extension),
%                  usually the same as |#1|,
% \It{SourceExt}: source extension without leading dot.
% \end{description}
%
% \paragraph{Conversion from EPS to PDF.}
% Other programs than \xprogram{epstopdf} can be used to convert from
% EPS to PDF. Example that uses \xprogram{Ghostscript}:
% \begin{quote}
%   |\DeclareGraphicsRule{.eps}{pdf}{.pdf}{%|\\
%   |  \epstopdfcall{`ps2pdf -dEPSCrop #1 \noexpand\OutputFile}%|\\
%   |}|
% \end{quote}
% \cs{DeclareGraphicsRule} expands the argument, therefore \cs{noexpand}
% is necessary. As convenience package \xpackage{epstopdf-base} defines
% \cs{epstopdfDeclareGraphicsRule}. Then the conversion argument
% is not expanded, \cs{epstopdfcall} and the back tick are added:
% \begin{quote}
%   |\epstopdfDeclareGraphicsRule{.eps}{pdf}{.pdf}{%|\\
%   |  ps2pdf -dEPSCrop #1 \OutputFile|\\
%   |}|
% \end{quote}
%
% Also \cs{OutputFile} respects the setting of option \xoption{outdir}.
%
% \subsection{Other image formats}
%
% The support that package \xpackage{epstopdf} implements is not
% limited to EPS files. Other image conversions can be declared.
% The following example shows it for GIF images under Unix with
% ImageMagick's \xprogram{convert}:
%\begin{quote}
%\begin{verbatim}
%\epstopdfDeclareGraphicsRule{.gif}{png}{.png}{%
%  convert #1 \OutputFile
%}
%\end{verbatim}
%\end{quote}
% The file extension \xfile{.gif} can be added to the extension list that
% package \xpackage{graphics} searches if the file extension
% is not given in \cs{includegraphics}.
% The list can be set by \cs{GraphicsExtensions}.
% \begin{quote}
% |\AppendGraphicsExtensions{.gif}|\\
% or\\
% |\PrependGraphicsExtensions{.gif}|
% \end{quote}
%
% \StopEventually{
% }
%
% \section{Implementation}
%
%    \begin{macrocode}
%<*package>
%    \end{macrocode}
%
% \subsection{Wrapper package}
%
%    Reload check, especially if the package is not used with \LaTeX.
%    \begin{macrocode}
\begingroup\catcode61\catcode48\catcode32=10\relax%
  \catcode13=5 % ^^M
  \endlinechar=13 %
  \catcode35=6 % #
  \catcode39=12 % '
  \catcode44=12 % ,
  \catcode45=12 % -
  \catcode46=12 % .
  \catcode58=12 % :
  \catcode64=11 % @
  \catcode123=1 % {
  \catcode125=2 % }
  \expandafter\let\expandafter\x\csname ver@epstopdf.sty\endcsname
  \ifx\x\relax % plain-TeX, first loading
  \else
    \def\empty{}%
    \ifx\x\empty % LaTeX, first loading,
      % variable is initialized, but \ProvidesPackage not yet seen
    \else
      \expandafter\ifx\csname PackageInfo\endcsname\relax
        \def\x#1#2{%
          \immediate\write-1{Package #1 Info: #2.}%
        }%
      \else
        \def\x#1#2{\PackageInfo{#1}{#2, stopped}}%
      \fi
      \x{epstopdf}{The package is already loaded}%
      \aftergroup\endinput
    \fi
  \fi
\endgroup%
%    \end{macrocode}
%    Package identification:
%    \begin{macrocode}
\begingroup\catcode61\catcode48\catcode32=10\relax%
  \catcode13=5 % ^^M
  \endlinechar=13 %
  \catcode35=6 % #
  \catcode39=12 % '
  \catcode40=12 % (
  \catcode41=12 % )
  \catcode44=12 % ,
  \catcode45=12 % -
  \catcode46=12 % .
  \catcode47=12 % /
  \catcode58=12 % :
  \catcode64=11 % @
  \catcode91=12 % [
  \catcode93=12 % ]
  \catcode123=1 % {
  \catcode125=2 % }
  \expandafter\ifx\csname ProvidesPackage\endcsname\relax
    \def\x#1#2#3[#4]{\endgroup
      \immediate\write-1{Package: #3 #4}%
      \xdef#1{#4}%
    }%
  \else
    \def\x#1#2[#3]{\endgroup
      #2[{#3}]%
      \ifx#1\@undefined
        \xdef#1{#3}%
      \fi
      \ifx#1\relax
        \xdef#1{#3}%
      \fi
    }%
  \fi
\expandafter\x\csname ver@epstopdf.sty\endcsname
\ProvidesPackage{epstopdf}%
  [2016/05/15 v2.6 Conversion with epstopdf on the fly (HO)]%
%    \end{macrocode}
%    Larger catcode set because of configuration files needed.
%    \begin{macrocode}
\begingroup\catcode61\catcode48\catcode32=10\relax%
  \catcode13=5 % ^^M
  \endlinechar=13 %
  \catcode123=1 % {
  \catcode125=2 % }
  \catcode64=11 % @
  \def\x{\endgroup
    \expandafter\edef\csname ETE@AtEnd\endcsname{%
      \endlinechar=\the\endlinechar\relax
      \catcode13=\the\catcode13\relax
      \catcode32=\the\catcode32\relax
      \catcode35=\the\catcode35\relax
      \catcode61=\the\catcode61\relax
      \catcode64=\the\catcode64\relax
      \catcode123=\the\catcode123\relax
      \catcode125=\the\catcode125\relax
    }%
  }%
\x\catcode61\catcode48\catcode32=10\relax%
\catcode13=5 % ^^M
\endlinechar=13 %
\catcode35=6 % #
\catcode64=11 % @
\catcode123=1 % {
\catcode125=2 % }
\def\TMP@EnsureCode#1#2{%
  \edef\ETE@AtEnd{%
    \ETE@AtEnd
    \catcode#1=\the\catcode#1\relax
  }%
  \catcode#1=#2\relax
}
\TMP@EnsureCode{39}{12}% '
\TMP@EnsureCode{40}{12}% (
\TMP@EnsureCode{41}{12}% )
\TMP@EnsureCode{42}{12}% *
\TMP@EnsureCode{45}{12}% -
\TMP@EnsureCode{47}{12}% /
\TMP@EnsureCode{91}{12}% [
\TMP@EnsureCode{93}{12}% ]
\edef\ETE@AtEnd{\ETE@AtEnd\noexpand\endinput}
%    \end{macrocode}
%
%    \begin{macrocode}
\let\ETE@SavedAtEnd\ETE@AtEnd
\RequirePackage{epstopdf-base}[2016/05/15]
\let\ETE@AtEnd\ETE@SavedAtEnd
%    \end{macrocode}
%
% \subsubsection{Option handling}
%
%    \begin{macrocode}
\DeclareOption*{%
  \expandafter\epstopdfsetup\expandafter{\CurrentOption}%
}
\ProcessOptions*\relax
%    \end{macrocode}
%
%    \begin{macrocode}
\ETE@AtEnd%
%</package>
%    \end{macrocode}
%
% \subsection{Base package}
%
%    \begin{macrocode}
%<*base>
%    \end{macrocode}
%
% \subsection{Preparations}
%
% \subsubsection{Relead check and identification}
%
%    Reload check, especially if the package is not used with \LaTeX.
%    \begin{macrocode}
\begingroup\catcode61\catcode48\catcode32=10\relax%
  \catcode13=5 % ^^M
  \endlinechar=13 %
  \catcode35=6 % #
  \catcode39=12 % '
  \catcode44=12 % ,
  \catcode45=12 % -
  \catcode46=12 % .
  \catcode58=12 % :
  \catcode64=11 % @
  \catcode123=1 % {
  \catcode125=2 % }
  \expandafter\let\expandafter\x\csname ver@epstopdf-base.sty\endcsname
  \ifx\x\relax % plain-TeX, first loading
  \else
    \def\empty{}%
    \ifx\x\empty % LaTeX, first loading,
      % variable is initialized, but \ProvidesPackage not yet seen
    \else
      \expandafter\ifx\csname PackageInfo\endcsname\relax
        \def\x#1#2{%
          \immediate\write-1{Package #1 Info: #2.}%
        }%
      \else
        \def\x#1#2{\PackageInfo{#1}{#2, stopped}}%
      \fi
      \x{epstopdf-base}{The package is already loaded}%
      \aftergroup\endinput
    \fi
  \fi
\endgroup%
%    \end{macrocode}
%    Package identification:
%    \begin{macrocode}
\begingroup\catcode61\catcode48\catcode32=10\relax%
  \catcode13=5 % ^^M
  \endlinechar=13 %
  \catcode35=6 % #
  \catcode39=12 % '
  \catcode40=12 % (
  \catcode41=12 % )
  \catcode44=12 % ,
  \catcode45=12 % -
  \catcode46=12 % .
  \catcode47=12 % /
  \catcode58=12 % :
  \catcode64=11 % @
  \catcode91=12 % [
  \catcode93=12 % ]
  \catcode123=1 % {
  \catcode125=2 % }
  \expandafter\ifx\csname ProvidesPackage\endcsname\relax
    \def\x#1#2#3[#4]{\endgroup
      \immediate\write-1{Package: #3 #4}%
      \xdef#1{#4}%
    }%
  \else
    \def\x#1#2[#3]{\endgroup
      #2[{#3}]%
      \ifx#1\@undefined
        \xdef#1{#3}%
      \fi
      \ifx#1\relax
        \xdef#1{#3}%
      \fi
    }%
  \fi
\expandafter\x\csname ver@epstopdf-base.sty\endcsname
\ProvidesPackage{epstopdf-base}%
  [2016/05/15 v2.6 Base part for package epstopdf]%
%    \end{macrocode}
%
% \subsubsection{Catcodes}
%
%    \begin{macrocode}
\begingroup\catcode61\catcode48\catcode32=10\relax%
  \catcode13=5 % ^^M
  \endlinechar=13 %
  \catcode123=1 % {
  \catcode125=2 % }
  \catcode64=11 % @
  \def\x{\endgroup
    \expandafter\edef\csname ETE@AtEnd\endcsname{%
      \endlinechar=\the\endlinechar\relax
      \catcode13=\the\catcode13\relax
      \catcode32=\the\catcode32\relax
      \catcode35=\the\catcode35\relax
      \catcode61=\the\catcode61\relax
      \catcode64=\the\catcode64\relax
      \catcode123=\the\catcode123\relax
      \catcode125=\the\catcode125\relax
    }%
  }%
\x\catcode61\catcode48\catcode32=10\relax%
\catcode13=5 % ^^M
\endlinechar=13 %
\catcode35=6 % #
\catcode64=11 % @
\catcode123=1 % {
\catcode125=2 % }
\def\TMP@EnsureCode#1#2{%
  \edef\ETE@AtEnd{%
    \ETE@AtEnd
    \catcode#1=\the\catcode#1\relax
  }%
  \catcode#1=#2\relax
}
\TMP@EnsureCode{33}{12}% !
\TMP@EnsureCode{39}{12}% '
\TMP@EnsureCode{42}{12}% *
\TMP@EnsureCode{44}{12}% ,
\TMP@EnsureCode{45}{12}% -
\TMP@EnsureCode{46}{12}% .
\TMP@EnsureCode{47}{12}% /
\TMP@EnsureCode{58}{12}% :
\TMP@EnsureCode{60}{12}% <
\TMP@EnsureCode{62}{12}% >
\TMP@EnsureCode{96}{12}% `
\edef\ETE@AtEnd{\ETE@AtEnd\noexpand\endinput}
%    \end{macrocode}
%
% \subsubsection{Load packages}
%
%    \begin{macrocode}
\RequirePackage{infwarerr}[2007/09/09]
\RequirePackage{grfext}\relax
\RequirePackage{kvoptions}[2007/10/02]
\RequirePackage{pdftexcmds}[2007/11/11]
%    \end{macrocode}
%
% \subsection{Checks}
%
% Check, whether package graphics is loaded
% (also graphicx loads graphics).
% Because miniltx.tex does not know \cs{@ifpackageloaded}
% we test for \cs{Gin@setfile} instead.
%    \begin{macrocode}
\begingroup\expandafter\expandafter\expandafter\endgroup
\expandafter\ifx\csname Gin@setfile\endcsname\relax
  \@PackageWarningNoLine{epstopdf}{%
    No graphics package \string`graphic{s,x}\string' loaded%
  }%
  \newcommand*{\epstopdfsetup}[1]{}%
  \expandafter\ETE@AtEnd
\fi%
%    \end{macrocode}
%
% Check, whether pdftex.def is loaded. \cs{ver@pdftex.def} is not
% available with miniltx.tex, thus we test for \cs{Gin@driver}.
%    \begin{macrocode}
\begingroup
  \def\x{luatex.def}%
  \ifx\Gin@driver\x
  \else
  \def\x{pdftex.def}%
  \ifx\Gin@driver\x
  \else
    \@PackageWarningNoLine{epstopdf}{%
      Drivers other than `pdftex' and `luatex' are not supported%
    }%
    \endgroup
    \newcommand*{\epstopdfsetup}[1]{}%
    \expandafter\ETE@AtEnd
  \fi%
  \fi%
\endgroup
%    \end{macrocode}
%
% Check, whether the shell escape feature is enabled.
%
%    \begin{macrocode}
\begingroup
  \expandafter\ifx\csname pdf@shellescape\endcsname\relax
  \else
    \ifnum\pdf@shellescape>0 %
    \else
      \@PackageWarningNoLine{epstopdf}{%
        Shell escape feature is not enabled%
      }%
    \fi
  \fi
\endgroup
%    \end{macrocode}
%
% \subsection{Package loading}
%
%    \begin{macrocode}
%    \end{macrocode}
%
% \subsection{Options}
%
%    \begin{macrocode}
\SetupKeyvalOptions{family=ETE,prefix=ETE@}
\DeclareBoolOption{update}
\DeclareBoolOption{verbose}
\newif\ifETE@prepend
\DeclareVoidOption{prepend}{\ETE@prependtrue}
\DeclareVoidOption{append}{\ETE@prependfalse}
\DeclareStringOption{outdir}
\DeclareStringOption{suffix}
\DeclareBoolOption{prefersuffix}
\DeclareStringOption{program@epstopdf}
%    \end{macrocode}
%    Options \xoption{disable} and \xoption{enable} are for testing
%    only. Therefore they are not documented on purpose.
%    \begin{macrocode}
\DeclareBoolOption{disable}
\DeclareComplementaryOption{enable}{disable}
%    \end{macrocode}
%    \begin{macrocode}
\newcommand*{\epstopdfsetup}{\setkeys{ETE}}
%    \end{macrocode}
%
% \subsubsection{Default setting}
%
%    \begin{macrocode}
\epstopdfsetup{%
  verbose,%
  enable,%
  append,%
  update,%
  prefersuffix,%
  suffix=-\SourceExt-converted-to,%
  program@epstopdf=epstopdf%
}
%    \end{macrocode}
%
% \subsection{Make and verbose}
%
%    \begin{macrocode}
\begingroup\expandafter\expandafter\expandafter\endgroup
\expandafter\ifx\csname pdf@filemoddate\endcsname\relax
  \def\ETE@Make#1#2{%
    \ifETE@update
      \ETE@WarnModDate
    \fi
    \@firstofone
  }%
  \def\ETE@WarnModDate{%
    \@PackageWarningNoLine{epstopdf}{%
      \string\pdffilemoddate\space is not available,\MessageBreak
      option `update' will be ignored%
    }%
    \global\let\ETE@WarnModDate\relax
  }%
  \def\ETE@FileInfo#1#2{#1 file: <#2>}%
\else
  \def\ETE@Make#1#2{%
    \ifETE@update
      \ifnum\pdf@strcmp{\pdf@filemoddate{#1}}{\pdf@filemoddate{#2}}>0 %
        \expandafter\expandafter\expandafter\@firstofone
      \else
        \@PackageInfoNoLine{epstopdf}{%
          Output file is already uptodate%
        }%
        \expandafter\expandafter\expandafter\@gobble
      \fi
    \else
      \expandafter\@firstofone
    \fi
  }%
  \def\ETE@FileInfo#1#2{%
    #1 file: <#2>%
    \expandafter\expandafter\expandafter
    \ETE@Date\pdf@filemoddate{#2}\@nil
    \expandafter\expandafter\expandafter
    \ETE@Size\pdf@filesize{#2}\@nil
  }%
  \def\ETE@Date#1\@nil{%
    \ifx\\#1\\%
    \else
      \ETE@@Date#1\@nil
    \fi
  }%
  \def\ETE@@Date#1:#2#3#4#5#6#7#8#9{%
    \MessageBreak
    \@spaces\space\space\space date: #2#3#4#5-#6#7-#8#9 %
    \ETE@@Time
  }%
  \def\ETE@@Time#1#2#3#4#5#6#7\@nil{%
    #1#2:#3#4:#5#6%
  }%
  \def\ETE@Size#1\@nil{%
    \ifx\\#1\\%
    \else
      \MessageBreak
      \@spaces\space\space\space size: #1 bytes%
    \fi
  }%
\fi
%    \end{macrocode}
%
% \subsection{Adding conversion support}
%
%    Patch \cs{Gin@setfile} to execute |#3|, if it contains
%    a command.
%    \begin{macrocode}
\expandafter\ifx\csname ETE@OrgGin@setfile\endcsname\relax
  \let\ETE@OrgGin@setfile\Gin@setfile
\else
  \@PackageError{epstopdf}{%
    Command \string\ETE@OrgGin@setfile\space
    already defined.\MessageBreak
  }{%
    Probably some package has included the code of this package%
    \MessageBreak
    instead of using \string\RequirePackage{epstopdf}.%
    \MessageBreak
    \@ehc
  }%
\fi
\def\ETE@IfFileExists{%
  \begingroup\expandafter\expandafter\expandafter\endgroup
  \expandafter\ifx\csname grffile@IfFileExists\endcsname\relax
    \expandafter\IfFileExists
  \else
    \global\let\ETE@IfFileExists\grffile@IfFileExists
    \expandafter\grffile@IfFileExists
  \fi
}
\def\ETE@Skip#1#2\x\fi\fi{%
  \fi
  \fi
  \endgroup
  \fi
  \fi
  #1%
}
\newif\ifETE@InsideSetfile
\newcommand*{\epstopdfcall}[1]{%
  \ifETE@InsideSetfile
    \expandafter\@firstoftwo
  \else
    \expandafter\@secondoftwo
  \fi
  {`#1}%
  {\Gin@base\Gin@ext}%
}
\def\ETE@DefCommandLine#1{%
  \edef\CommandLine{\expandafter\fi\if`#1}%
}
\def\ETE@DefX#1{%
  \expandafter\expandafter\expandafter\def
  \expandafter\expandafter\expandafter\x
  \expandafter\expandafter\expandafter{%
    \expandafter\fi\if`#1\relax\else
  }%
}
\def\ETE@Gin@setfile#1#2#3{%
  \ifETE@disable
    \ETE@OrgGin@setfile{#1}{#2}{#3}%
  \else
    \begingroup
      \ETE@InsideSetfiletrue
      \ETE@DefX{#3}%
    \expandafter\endgroup
    \ifx\x\@empty
      \ETE@OrgGin@setfile{#1}{#2}{#3}%
    \else
      \begingroup
        \ETE@InsideSetfiletrue
        \def\GraphicsType{#1}%
        \def\GraphicsRead{#2}%
        \ifETE@prefersuffix
        \else
          \ifx\ETE@suffix\@empty
          \else
            \ETE@IfFileExists{\Gin@base\GraphicsRead}{%
              \ETE@Skip{%
                \ETE@OrgGin@setfile{#1}{#2}{\Gin@base#2}%
              }%
            }{%
              \let\next\relax
            }%
            \next
          \fi
        \fi
        \ifx\Gin@ext\relax
          \let\SourceExt\Gin@eext
          \def\SourceFile{\Gin@base\Gin@eext}%
        \else
          \let\SourceExt\Gin@ext
          \def\SourceFile{\Gin@base\Gin@ext}%
        \fi
        \edef\SourceExt{% remove dot
          \expandafter\@cdr\SourceExt\@empty\@nil
        }%
        \let\OutputDirectory\ETE@outdir
        \ifx\OutputDirectory\@empty
          \edef\OutputFile{\ETE@GenerateName{\Gin@base}{#2}}%
        \else
          \begingroup
            \filename@parse{\Gin@base#2}%
            \edef\x{\endgroup
              \def\noexpand\OutputFile{%
                \ETE@GenerateName{%
                  \OutputDirectory\filename@base
                }{#2}%
              }%
            }%
          \x
        \fi
        \ETE@DefCommandLine{#3}%
        \ifETE@verbose
          \@PackageInfo{epstopdf}{%
            \ETE@FileInfo{Source}\SourceFile\MessageBreak
            \ETE@FileInfo{Output}\OutputFile\MessageBreak
            Command: <\CommandLine>\MessageBreak
            \string\includegraphics
          }%
        \fi
        \ETE@Make\SourceFile\OutputFile{%
          \pdf@system{\CommandLine}%
          \ifETE@verbose
            \@PackageInfoNoLine{epstopdf}{%
              \ETE@FileInfo{Result}\OutputFile
            }%
          \fi
        }%
        \edef\x{\endgroup
          \ifx\OutputDirectory\@empty
          \else
            \def\noexpand\Gin@base{%
              \OutputDirectory\noexpand\filename@base
            }%
          \fi
          \ifx\ETE@suffix\@empty
          \else
            \edef\noexpand\Gin@base{%
              \noexpand\Gin@base\ETE@suffix
            }%
          \fi
          \noexpand\ETE@OrgGin@setfile{%
            \GraphicsType
          }{%
            \GraphicsRead
          }{%
            \OutputFile
          }%
        }%
      \x
    \fi
  \fi
}
\let\Gin@setfile\ETE@Gin@setfile
\def\ETE@GenerateName#1#2{%
  #1\ETE@suffix#2%
}
%    \end{macrocode}
%
% \subsection{Declare graphics rule}
%
%    \begin{macrocode}
\newcommand*{\epstopdfDeclareGraphicsRule}[4]{%
  \ifx\\#4\\%
    \@PackageError{epstopdf-base}{%
      Conversion command is missing%
    }\@ehc
  \else
    \begingroup
      \@ifundefined{Gin@rule@#1}{%
      }{%
        \@PackageInfo{epstopdf-base}{%
          Redefining graphics rule for `#1'%
        }%
      }%
    \endgroup
    \@namedef{Gin@rule@#1}##1{{#2}{#3}{\epstopdfcall{#4}}}%
  \fi
}
%    \end{macrocode}
%
% \cs{DeclareGraphicsRule} for .eps
%    \begin{macrocode}
\epstopdfDeclareGraphicsRule{.eps}{pdf}{.pdf}{%
  \ETE@epstopdf{#1}%
}
\def\ETE@epstopdf#1{%
  \ETE@program@epstopdf\space
  \ifcase\ifx\OutputDirectory\@empty
           \ifx\ETE@suffix\@empty
             1%
           \fi
         \fi
         0 %
    --outfile=\OutputFile\space
  \fi
  #1%
}
%    \end{macrocode}
%
%    \begin{macrocode}
\ifETE@prepend
  \expandafter\PrependGraphicsExtensions
\else
  \expandafter\AppendGraphicsExtensions
\fi
{.eps}
\let\ETE@prepend\@undefined
\DeclareVoidOption{prepend}{%
  \PrependGraphicsExtensions{.eps}%
}
\let\ETE@append\@undefined
\DeclareVoidOption{append}{%
  \AppendGraphicsExtensions{.eps}%
}
%    \end{macrocode}
%
%    \begin{macrocode}
\InputIfFileExists{epstopdf-sys.cfg}{}{}
\InputIfFileExists{epstopdf.cfg}{}{}
%    \end{macrocode}
%
%    \begin{macrocode}
\ETE@AtEnd%
%</base>
%    \end{macrocode}
%
% \section{Test}
%
% \subsection{Preface for standard catcode check}
%
%    \begin{macrocode}
%<*test1>
\input miniltx.tex\relax
\def\Gin@driver{pdftex.def}
\input graphicx.sty\relax
\resetatcatcode
%</test1>
%    \end{macrocode}
%
% \subsection{Catcode checks for loading}
%
%    \begin{macrocode}
%<*test1>
%    \end{macrocode}
%    \begin{macrocode}
\catcode`\{=1 %
\catcode`\}=2 %
\catcode`\#=6 %
\catcode`\@=11 %
\expandafter\ifx\csname count@\endcsname\relax
  \countdef\count@=255 %
\fi
\expandafter\ifx\csname @gobble\endcsname\relax
  \long\def\@gobble#1{}%
\fi
\expandafter\ifx\csname @firstofone\endcsname\relax
  \long\def\@firstofone#1{#1}%
\fi
\expandafter\ifx\csname loop\endcsname\relax
  \expandafter\@firstofone
\else
  \expandafter\@gobble
\fi
{%
  \def\loop#1\repeat{%
    \def\body{#1}%
    \iterate
  }%
  \def\iterate{%
    \body
      \let\next\iterate
    \else
      \let\next\relax
    \fi
    \next
  }%
  \let\repeat=\fi
}%
\def\RestoreCatcodes{}
\count@=0 %
\loop
  \edef\RestoreCatcodes{%
    \RestoreCatcodes
    \catcode\the\count@=\the\catcode\count@\relax
  }%
\ifnum\count@<255 %
  \advance\count@ 1 %
\repeat

\def\RangeCatcodeInvalid#1#2{%
  \count@=#1\relax
  \loop
    \catcode\count@=15 %
  \ifnum\count@<#2\relax
    \advance\count@ 1 %
  \repeat
}
\def\RangeCatcodeCheck#1#2#3{%
  \count@=#1\relax
  \loop
    \ifnum#3=\catcode\count@
    \else
      \errmessage{%
        Character \the\count@\space
        with wrong catcode \the\catcode\count@\space
        instead of \number#3%
      }%
    \fi
  \ifnum\count@<#2\relax
    \advance\count@ 1 %
  \repeat
}
\def\space{ }
\expandafter\ifx\csname LoadCommand\endcsname\relax
  \def\LoadCommand{\input epstopdf.sty\relax}%
\fi
\def\Test{%
  \RangeCatcodeInvalid{0}{47}%
  \RangeCatcodeInvalid{58}{64}%
  \RangeCatcodeInvalid{91}{96}%
  \RangeCatcodeInvalid{123}{255}%
  \catcode`\@=12 %
  \catcode`\\=0 %
  \catcode`\%=14 %
  \LoadCommand
  \RangeCatcodeCheck{0}{36}{15}%
  \RangeCatcodeCheck{37}{37}{14}%
  \RangeCatcodeCheck{38}{47}{15}%
  \RangeCatcodeCheck{48}{57}{12}%
  \RangeCatcodeCheck{58}{63}{15}%
  \RangeCatcodeCheck{64}{64}{12}%
  \RangeCatcodeCheck{65}{90}{11}%
  \RangeCatcodeCheck{91}{91}{15}%
  \RangeCatcodeCheck{92}{92}{0}%
  \RangeCatcodeCheck{93}{96}{15}%
  \RangeCatcodeCheck{97}{122}{11}%
  \RangeCatcodeCheck{123}{255}{15}%
  \RestoreCatcodes
}
\Test
\csname @@end\endcsname
\end
%    \end{macrocode}
%    \begin{macrocode}
%</test1>
%    \end{macrocode}
%
% \section{Installation}
%
% \subsection{Download}
%
% \paragraph{Package.} This package is available on
% CTAN\footnote{\url{http://ctan.org/pkg/epstopdf}}:
% \begin{description}
% \item[\CTAN{macros/latex/contrib/oberdiek/epstopdf.dtx}] The source file.
% \item[\CTAN{macros/latex/contrib/oberdiek/epstopdf.pdf}] Documentation.
% \end{description}
%
%
% \paragraph{Bundle.} All the packages of the bundle `oberdiek'
% are also available in a TDS compliant ZIP archive. There
% the packages are already unpacked and the documentation files
% are generated. The files and directories obey the TDS standard.
% \begin{description}
% \item[\CTAN{install/macros/latex/contrib/oberdiek.tds.zip}]
% \end{description}
% \emph{TDS} refers to the standard ``A Directory Structure
% for \TeX\ Files'' (\CTAN{tds/tds.pdf}). Directories
% with \xfile{texmf} in their name are usually organized this way.
%
% \subsection{Bundle installation}
%
% \paragraph{Unpacking.} Unpack the \xfile{oberdiek.tds.zip} in the
% TDS tree (also known as \xfile{texmf} tree) of your choice.
% Example (linux):
% \begin{quote}
%   |unzip oberdiek.tds.zip -d ~/texmf|
% \end{quote}
%
% \paragraph{Script installation.}
% Check the directory \xfile{TDS:scripts/oberdiek/} for
% scripts that need further installation steps.
% Package \xpackage{attachfile2} comes with the Perl script
% \xfile{pdfatfi.pl} that should be installed in such a way
% that it can be called as \texttt{pdfatfi}.
% Example (linux):
% \begin{quote}
%   |chmod +x scripts/oberdiek/pdfatfi.pl|\\
%   |cp scripts/oberdiek/pdfatfi.pl /usr/local/bin/|
% \end{quote}
%
% \subsection{Package installation}
%
% \paragraph{Unpacking.} The \xfile{.dtx} file is a self-extracting
% \docstrip\ archive. The files are extracted by running the
% \xfile{.dtx} through \plainTeX:
% \begin{quote}
%   \verb|tex epstopdf.dtx|
% \end{quote}
%
% \paragraph{TDS.} Now the different files must be moved into
% the different directories in your installation TDS tree
% (also known as \xfile{texmf} tree):
% \begin{quote}
% \def\t{^^A
% \begin{tabular}{@{}>{\ttfamily}l@{ $\rightarrow$ }>{\ttfamily}l@{}}
%   epstopdf.sty & tex/latex/oberdiek/epstopdf.sty\\
%   epstopdf-base.sty & tex/latex/oberdiek/epstopdf-base.sty\\
%   epstopdf.pdf & doc/latex/oberdiek/epstopdf.pdf\\
%   test/epstopdf-test1.tex & doc/latex/oberdiek/test/epstopdf-test1.tex\\
%   epstopdf.dtx & source/latex/oberdiek/epstopdf.dtx\\
% \end{tabular}^^A
% }^^A
% \sbox0{\t}^^A
% \ifdim\wd0>\linewidth
%   \begingroup
%     \advance\linewidth by\leftmargin
%     \advance\linewidth by\rightmargin
%   \edef\x{\endgroup
%     \def\noexpand\lw{\the\linewidth}^^A
%   }\x
%   \def\lwbox{^^A
%     \leavevmode
%     \hbox to \linewidth{^^A
%       \kern-\leftmargin\relax
%       \hss
%       \usebox0
%       \hss
%       \kern-\rightmargin\relax
%     }^^A
%   }^^A
%   \ifdim\wd0>\lw
%     \sbox0{\small\t}^^A
%     \ifdim\wd0>\linewidth
%       \ifdim\wd0>\lw
%         \sbox0{\footnotesize\t}^^A
%         \ifdim\wd0>\linewidth
%           \ifdim\wd0>\lw
%             \sbox0{\scriptsize\t}^^A
%             \ifdim\wd0>\linewidth
%               \ifdim\wd0>\lw
%                 \sbox0{\tiny\t}^^A
%                 \ifdim\wd0>\linewidth
%                   \lwbox
%                 \else
%                   \usebox0
%                 \fi
%               \else
%                 \lwbox
%               \fi
%             \else
%               \usebox0
%             \fi
%           \else
%             \lwbox
%           \fi
%         \else
%           \usebox0
%         \fi
%       \else
%         \lwbox
%       \fi
%     \else
%       \usebox0
%     \fi
%   \else
%     \lwbox
%   \fi
% \else
%   \usebox0
% \fi
% \end{quote}
% If you have a \xfile{docstrip.cfg} that configures and enables \docstrip's
% TDS installing feature, then some files can already be in the right
% place, see the documentation of \docstrip.
%
% \subsection{Refresh file name databases}
%
% If your \TeX~distribution
% (\teTeX, \mikTeX, \dots) relies on file name databases, you must refresh
% these. For example, \teTeX\ users run \verb|texhash| or
% \verb|mktexlsr|.
%
% \subsection{Some details for the interested}
%
% \paragraph{Attached source.}
%
% The PDF documentation on CTAN also includes the
% \xfile{.dtx} source file. It can be extracted by
% AcrobatReader 6 or higher. Another option is \textsf{pdftk},
% e.g. unpack the file into the current directory:
% \begin{quote}
%   \verb|pdftk epstopdf.pdf unpack_files output .|
% \end{quote}
%
% \paragraph{Unpacking with \LaTeX.}
% The \xfile{.dtx} chooses its action depending on the format:
% \begin{description}
% \item[\plainTeX:] Run \docstrip\ and extract the files.
% \item[\LaTeX:] Generate the documentation.
% \end{description}
% If you insist on using \LaTeX\ for \docstrip\ (really,
% \docstrip\ does not need \LaTeX), then inform the autodetect routine
% about your intention:
% \begin{quote}
%   \verb|latex \let\install=y\input{epstopdf.dtx}|
% \end{quote}
% Do not forget to quote the argument according to the demands
% of your shell.
%
% \paragraph{Generating the documentation.}
% You can use both the \xfile{.dtx} or the \xfile{.drv} to generate
% the documentation. The process can be configured by the
% configuration file \xfile{ltxdoc.cfg}. For instance, put this
% line into this file, if you want to have A4 as paper format:
% \begin{quote}
%   \verb|\PassOptionsToClass{a4paper}{article}|
% \end{quote}
% An example follows how to generate the
% documentation with pdf\LaTeX:
% \begin{quote}
%\begin{verbatim}
%pdflatex epstopdf.dtx
%makeindex -s gind.ist epstopdf.idx
%pdflatex epstopdf.dtx
%makeindex -s gind.ist epstopdf.idx
%pdflatex epstopdf.dtx
%\end{verbatim}
% \end{quote}
%
% \section{Catalogue}
%
% The following XML file can be used as source for the
% \href{http://mirror.ctan.org/help/Catalogue/catalogue.html}{\TeX\ Catalogue}.
% The elements \texttt{caption} and \texttt{description} are imported
% from the original XML file from the Catalogue.
% The name of the XML file in the Catalogue is \xfile{epstopdf-pkg.xml}.
%    \begin{macrocode}
%<*catalogue>
<?xml version='1.0' encoding='us-ascii'?>
<!DOCTYPE entry SYSTEM 'catalogue.dtd'>
<entry datestamp='$Date$' modifier='$Author$' id='epstopdf-pkg'>
  <name>epstopdf-pkg</name>
  <caption>Call epstopdf "on the fly"</caption>
  <authorref id='auth:oberdiek'/>
  <copyright owner='Heiko Oberdiek' year='2001,2006-2010'/>
  <license type='lppl1.3'/>
  <version number='2.6'/>
  <description>
    The package adds support for EPS files in the
    <xref refid='graphicx'>graphicx</xref> package when running under
    PDFTeX.  If an EPS graphic is detected, the package spawns a
    process to convert the EPS to PDF, using the script
    <xref refid='epstopdf'>epstopdf</xref>.  This of course requires
    that shell escape is enabled for the PDFTeX run.
    <p/>
    The package is part of the <xref refid='oberdiek'>oberdiek</xref>
    bundle.
  </description>
  <documentation details='Package documentation'
      href='ctan:/macros/latex/contrib/oberdiek/epstopdf.pdf'/>
  <ctan file='true' path='/macros/latex/contrib/oberdiek/epstopdf.dtx'/>
  <miktex location='oberdiek'/>
  <texlive location='oberdiek'/>
  <install path='/macros/latex/contrib/oberdiek/oberdiek.tds.zip'/>
</entry>
%</catalogue>
%    \end{macrocode}
%
% \begin{History}
%   \begin{Version}{2001/01/06 v1.0}
%   \item
%     First public version,
%     published in the \pdfTeX\ mailing list.
%   \end{Version}
%   \begin{Version}{2001/02/04 v1.1}
%   \item
%     Minor documentation update.
%   \item
%     CTAN.
%   \end{Version}
%   \begin{Version}{2006/02/20 v1.2}
%   \item
%     DTX framework.
%   \item
%     Compatibility for \xfile{miniltx.tex}.
%   \end{Version}
%   \begin{Version}{2006/08/26 v1.3}
%   \item
%     Check for \cs{write18} if available and print a warning
%     if the feature is not enabled.
%   \end{Version}
%   \begin{Version}{2007/04/26 v1.4}
%   \item
%     Documentation rewritten and extended.
%   \end{Version}
%   \begin{Version}{2007/10/02 v1.5}
%   \item
%     New option \xoption{update}: If the converted file exists, it will be only
%     converted if it is out of date.
%   \item
%     Updating the extension list is delegated to package \xpackage{grfext}.
%     Fine tuning is done by the new options \xoption{append}, \xoption{prepend}.
%   \item
%     New option \xoption{outdir} for changing the output directory.
%   \item
%     New option \xoption{verbose}.
%   \item
%     \cs{SourceFile} and \cs{OutputFile} introduced.
%   \item
%     Configuration file support added.
%   \end{Version}
%   \begin{Version}{2007/11/11 v1.6}
%   \item
%     Use of package \xpackage{pdftexcmds} for \LuaTeX\ support.
%   \end{Version}
%   \begin{Version}{2008/05/06 v1.7}
%   \item
%     Warning messages uses ``loaded'' instead of ``found''.
%   \end{Version}
%   \begin{Version}{2009/03/01 v1.8}
%   \item
%     Warning message for missing \xfile{pdftex.def} changed.
%   \end{Version}
%   \begin{Version}{2009/07/06 v1.9}
%   \item
%     Option \xoption{suffix} added.
%   \end{Version}
%   \begin{Version}{2009/07/07 v1.10}
%   \item
%     \cs{SourceExt} added.
%   \item
%     If option \xoption{suffix} is set, the inclusion of
%     an image without the suffix namespace is preferred over generating the
%     the image within the suffix namespace.
%   \end{Version}
%   \begin{Version}{2009/07/12 v2.0}
%   \item
%     New default settings.
%   \item
%     Package is split into \xpackage{epstopdf} that only takes package
%     options and loads \xpackage{epstopdf-base} that does the work.
%   \item
%     \cs{epstopdfDeclareGraphicsRule} and \cs{epstopdfcall} added.
%   \item
%     \xfile{epstopdf-sys.cfg} is loaded before \xfile{epstopdf.cfg}
%     if \xfile{epstopdf-sys.cfg} exists.
%   \end{Version}
%   \begin{Version}{2009/07/15 v2.1}
%   \item
%     Default setting: \xoption{verbose} is now turned on as the
%     documentation for v2.0 said.
%   \item
%     Documentation fixes.
%   \end{Version}
%   \begin{Version}{2009/07/16 v2.2}
%   \item
%     Fixed redefined \cs{Gin@setfile}.
%   \item
%     Documentation extended for package options.
%   \end{Version}
%   \begin{Version}{2009/09/24 v2.3}
%   \item
%     Bug fix for the case that both option \xoption{suffix} and
%     \xoption{outdir} are used.
%   \end{Version}
%   \begin{Version}{2009/10/17 v2.4}
%   \item
%     The name of the program `epstopdf' can be configured via
%     the new option \xoption{program@epstopdf}.
%   \end{Version}
%   \begin{Version}{2016/05/15 v2.5}
%   \item
%     Wording of warning message fixed (Karl Berry).
%   \item
%     \cs{ETE@Gin@setfile} added (Karl Berry).
%   \end{Version}
%   \begin{Version}{2016/05/15 v2.6}
%   \item
%     luaTeX compatibility
%   \end{Version}
% \end{History}
%
% \PrintIndex
%
% \Finale
\endinput

%        (quote the arguments according to the demands of your shell)
%
% Documentation:
%    (a) If epstopdf.drv is present:
%           latex epstopdf.drv
%    (b) Without epstopdf.drv:
%           latex epstopdf.dtx; ...
%    The class ltxdoc loads the configuration file ltxdoc.cfg
%    if available. Here you can specify further options, e.g.
%    use A4 as paper format:
%       \PassOptionsToClass{a4paper}{article}
%
%    Programm calls to get the documentation (example):
%       pdflatex epstopdf.dtx
%       makeindex -s gind.ist epstopdf.idx
%       pdflatex epstopdf.dtx
%       makeindex -s gind.ist epstopdf.idx
%       pdflatex epstopdf.dtx
%
% Installation:
%    TDS:tex/latex/oberdiek/epstopdf.sty
%    TDS:tex/latex/oberdiek/epstopdf-base.sty
%    TDS:doc/latex/oberdiek/epstopdf.pdf
%    TDS:doc/latex/oberdiek/test/epstopdf-test1.tex
%    TDS:source/latex/oberdiek/epstopdf.dtx
%
%<*ignore>
\begingroup
  \catcode123=1 %
  \catcode125=2 %
  \def\x{LaTeX2e}%
\expandafter\endgroup
\ifcase 0\ifx\install y1\fi\expandafter
         \ifx\csname processbatchFile\endcsname\relax\else1\fi
         \ifx\fmtname\x\else 1\fi\relax
\else\csname fi\endcsname
%</ignore>
%<*install>
\input docstrip.tex
\Msg{************************************************************************}
\Msg{* Installation}
\Msg{* Package: epstopdf 2016/05/15 v2.6 Conversion with epstopdf on the fly (HO)}
\Msg{************************************************************************}

\keepsilent
\askforoverwritefalse

\let\MetaPrefix\relax
\preamble

This is a generated file.

Project: epstopdf
Version: 2016/05/15 v2.6

Copyright (C) 2001, 2006-2010 by
   Heiko Oberdiek <heiko.oberdiek at googlemail.com>

This work may be distributed and/or modified under the
conditions of the LaTeX Project Public License, either
version 1.3c of this license or (at your option) any later
version. This version of this license is in
   http://www.latex-project.org/lppl/lppl-1-3c.txt
and the latest version of this license is in
   http://www.latex-project.org/lppl.txt
and version 1.3 or later is part of all distributions of
LaTeX version 2005/12/01 or later.

This work has the LPPL maintenance status "maintained".

This Current Maintainer of this work is Heiko Oberdiek.

This work consists of the main source file epstopdf.dtx
and the derived files
   epstopdf.sty, epstopdf.pdf, epstopdf.ins, epstopdf.drv,
   epstopdf-base.sty, epstopdf-test1.tex.

\endpreamble
\let\MetaPrefix\DoubleperCent

\generate{%
  \file{epstopdf.ins}{\from{epstopdf.dtx}{install}}%
  \file{epstopdf.drv}{\from{epstopdf.dtx}{driver}}%
  \usedir{tex/latex/oberdiek}%
  \file{epstopdf.sty}{\from{epstopdf.dtx}{package}}%
  \file{epstopdf-base.sty}{\from{epstopdf.dtx}{base}}%
  \usedir{doc/latex/oberdiek/test}%
  \file{epstopdf-test1.tex}{\from{epstopdf.dtx}{test1}}%
  \nopreamble
  \nopostamble
  \usedir{source/latex/oberdiek/catalogue}%
  \file{epstopdf-pkg.xml}{\from{epstopdf.dtx}{catalogue}}%
}

\catcode32=13\relax% active space
\let =\space%
\Msg{************************************************************************}
\Msg{*}
\Msg{* To finish the installation you have to move the following}
\Msg{* files into a directory searched by TeX:}
\Msg{*}
\Msg{*     epstopdf.sty, epstopdf-base.sty}
\Msg{*}
\Msg{* To produce the documentation run the file `epstopdf.drv'}
\Msg{* through LaTeX.}
\Msg{*}
\Msg{* Happy TeXing!}
\Msg{*}
\Msg{************************************************************************}

\endbatchfile
%</install>
%<*ignore>
\fi
%</ignore>
%<*driver>
\NeedsTeXFormat{LaTeX2e}
\ProvidesFile{epstopdf.drv}%
  [2016/05/15 v2.6 Conversion with epstopdf on the fly (HO)]%
\documentclass{ltxdoc}
\usepackage{holtxdoc}[2011/11/22]
\begin{document}
  \DocInput{epstopdf.dtx}%
\end{document}
%</driver>
% \fi
%
%
% \CharacterTable
%  {Upper-case    \A\B\C\D\E\F\G\H\I\J\K\L\M\N\O\P\Q\R\S\T\U\V\W\X\Y\Z
%   Lower-case    \a\b\c\d\e\f\g\h\i\j\k\l\m\n\o\p\q\r\s\t\u\v\w\x\y\z
%   Digits        \0\1\2\3\4\5\6\7\8\9
%   Exclamation   \!     Double quote  \"     Hash (number) \#
%   Dollar        \$     Percent       \%     Ampersand     \&
%   Acute accent  \'     Left paren    \(     Right paren   \)
%   Asterisk      \*     Plus          \+     Comma         \,
%   Minus         \-     Point         \.     Solidus       \/
%   Colon         \:     Semicolon     \;     Less than     \<
%   Equals        \=     Greater than  \>     Question mark \?
%   Commercial at \@     Left bracket  \[     Backslash     \\
%   Right bracket \]     Circumflex    \^     Underscore    \_
%   Grave accent  \`     Left brace    \{     Vertical bar  \|
%   Right brace   \}     Tilde         \~}
%
% \GetFileInfo{epstopdf.drv}
%
% \title{The \xpackage{epstopdf} package}
% \date{2016/05/15 v2.6}
% \author{Heiko Oberdiek\thanks
% {Please report any issues at https://github.com/ho-tex/oberdiek/issues}\\
% \xemail{heiko.oberdiek at googlemail.com}}
%
% \maketitle
%
% \begin{abstract}
% This packages adds support of handling eps images
% to package \xpackage{graphics} or \xpackage{graphicx}
% with option \xoption{pdftex}. If an eps image is detected,
% epstopdf is automatically called to convert it to pdf format.
% \end{abstract}
%
% \tableofcontents
%
% \providecommand*{\xprogram}[1]{\textsf{#1}}
%
% \section{Documentation}
%
% \subsection{Introduction}
%
% \LaTeX\ provides its graphics bundle to include graphics files.
% Both packages \xpackage{graphics} or \xpackage{graphicx} may
% be used. the latter one loads the first and adds options
% in key value style for \cs{includegraphics}.
%
% Usually the drivers do not support all kind of graphics files.
% Other image types must be converted, before they become
% usuable. In case of driver \xprogram{dvips}, the graphics
% rule may contain a conversion rule. Then all that package
% \xpackage{graphics} must know is the bounding box, the command
% is passed to \xprogram{dvips} that calls it and embeds the
% converted image.
%
% However, \pdfTeX\ has its driver for PDF output already
% build in. It's graphics inclusion commands (\cs{pdfximage})
% does not allow the execution of external commands.
% Therefore commands in the last argument of
% \cs{DeclareGraphicsRule} were of no use.
% But external programs can be called within \pdfTeX.
% This feature is called ``shell escape'' or ``write 18''
% and must usually enabled explicitely because of security reasons.
% Now, this package \xpackage{epstopdf} hooks
% into package \xpackage{graphics}' code to catch
% that argument with the external command and executes it
% to convert the graphics file to a supported format and
% passes the control of graphics inclusion back to package
% \xpackage{graphics}.
%
% \subsection{Requirements}
%
% \begin{itemize}
% \item
%   The feature \cs{write18} must be enabled. This allows the
%   running of external programs during \TeX's compile run.
%   Keep in mind that this is a security risk. The feature is an
%   addition to \cs{TeX}. Mik\TeX, te\TeX, \TeX\ Live support it.
%   In Web2C based \TeX\ distributions (te\TeX, \TeX\ Live) it can be
%   enabled in the configuration file \xfile{texmf.cnf}:
%   \begin{quote}
%     |shell_escape = 1|
%   \end{quote}
%   Because of the security risk, it is better to do it on the
%   command line only:
%   \begin{quote}
%   \texttt{--shell-escape} (te\TeX, \TeX\ Live)\\
%   \texttt{--enable-write18} (MiK\TeX)
%   \end{quote}
%   Example:
%   \begin{quote}
%     |pdflatex -shell-escape test.tex|
%   \end{quote}
% \item
%   The program \xprogram{epstopdf} for the conversion from EPS
%   to PDF. However, other programs can
%   be used and configured by \cs{DeclareGraphicsRule}.
%   Example:
%   \begin{quote}
%   |\epstopdfDeclareGraphicsRule{.eps}{pdf}{.pdf}{%|\\
%   |  ps2pdf -dEPSCrop #1 \OutputFile|\\
%   |}|
%   \end{quote}
% \end{itemize}
%
% \subsection{Usage}
%
% The package is loaded after graphic\{s,x\}, e.g.:
% \begin{quote}
%   |\usepackage[pdftex]{graphicx}|\\
%   |\usepackage{epstopdf}|
% \end{quote}
% Now images with file name extension \xfile{.eps} are detected
% and supported using \cs{includegraphics}.
%
% If the graphics file name is explicitly specified with extension
% \xfile{.eps} the new rule for EPS files is called and
% the conversion performed. If option \xoption{update} is in force
% then the conversion step is dropped if the target file already
% exists and is not older then the EPS file.
%
% The situation is more complicate if the graphics file is
% given without file name extension. Then the \xpackage{graphics}
% package must search for a supported image file.
% The possible extensions are stored in the graphics extension
% list, that can be set by \cs{DeclareGraphicsExtensions}.
% The algorithm:
% \begin{quote}
%    \def\>{\mbox{}\quad}
%    function search( \meta{filebase} )\\
%    \>foreach \meta{ext} in \meta{graphics extensions}\\
%    \>\>foreach \meta{dir} in \meta{current directory}, \meta{\cs{graphicspath}}\\
%    \>\>\>\meta{file} := \meta{dir} + \meta{filebase} + \meta{ext}\\
%    \>\>\>if exist \meta{file}\\
%    \>\>\>\>return found\\
%    \>return not found
% \end{quote}
% Package \xpackage{epstopdf} puts \xfile{.eps} at the end of the
% graphics extension search list. This is the behaviour of option
% \xoption{append} that is enabled by default.
% That means, the conversion is called last
% unless a supported file type cannot be found earlier. This avoids
% unnecessary conversion steps that slow down the \LaTeX\ run.
% If you want to use option \xoption{update} and your \pdfTeX\ supports
% it, then an outdated PDF file also would be found earlier unless
% \xoption{suffix} is used that is the default since version 2.0.
%
% With an empty option \xoption{suffix} and option
% \xoption{prepend} there is a risk that an original PDF file is overwritten:
%
% If the original image file is the PDF file and there is also a generated
% EPS file, then the original PDF file can be regenerated (depending
% on the option settings) and the original PDF file gets lost.
% Therefore option \xoption{suffix} is introduced in version 1.9
% to create a separate name space for generated output files.
%
% \textbf{Note:} Usually the conversion program needs the exact location
% of the image file. Usually the current directory works.
% Also if the image file is found using \cs{graphicspath}, the
% location is known. However, if the image is somewhere
% in a directory of environment variable |TEXINPUTS|, then
% the package does not know the exact location and the
% conversion program will not find the image file unless it
% implements a search using |TEXINPUTS| (program \xprogram{kpsewhich}
% may be of help in this task).
%
% \subsection{Options}
%
% Options can be given as package options or later using:
% \begin{declcs}{epstopdfsetup} \M{key value list}
% \end{declcs}
% \LaTeX\ expands the option list before passing the option list
% to the package's option handling code. This can fail
% for option \xoption{suffix} if it contains some of the macros
% described below. Use \cs{epstopdfsetup} after the package is loaded.
% Or load package \xpackage{kvoptions-patch} before. This package
% is also loaded by option \xoption{patch} of package \xpackage{kvoptions}.
% \LaTeX's option code is redefined to respect key value options and
% let the values untouched.
%
% \begin{description}
% \def\It#1{\item[\xoption{#1}:]}^^A
% \It{update}
%   The conversion program is only called, if the target file
%   does not exist or is older than the source image file.
% \It{append}
%   Puts the extension \xfile{.eps} at the end of the graphics extension list
%   (default).
% \It{prepend}
%   Puts the extension \xfile{.eps} at the begin of the graphics extension list.
% \It{outdir}
%   The converted file may put in an other output directory.
%   The value of \xoption{outdir} must include the directory separator.
%   Example for the current directory:
%   \begin{quote}
%   |\epstopdfsetup{outdir=./}|
%   \end{quote}
%   For other directories ensure, that they can be found.
%   See \cs{graphicspath} or |TEXINPUTS|.
% \It{suffix}
%   This option takes a string that is put between the file name base
%   and the extension of the output file. Rationale: It can happen,
%   that a PDF file is the original file and the EPS file the
%   generated file. If now the package thinks, that the PDF file
%   is the generated file, it will `regnerate' it. But in reality
%   the original file is lost. Therefore I recommend to use this
%   option always to generate a separate name space for generated
%   files. Proposed value is \texttt{-generated} or \texttt{.generated}.
%   The suffix \texttt{.generated} will also work here without the need for
%   package \xpackage{grffile}).
%
%   Example:
%   \begin{quote}
%   |\epstopdfsetup{suffix=-generated}|\\
%   Then \xfile{foo.eps} is converted to \xfile{foo-generated.pdf}.
%   \end{quote}
%   \cs{SourceExt} can be used inside the suffix string.
%   It's will be replaced by the extension of the image source
%   file without the leading dot, for instance:
%   \begin{quote}
%   |\epstopdfsetup{suffix=-\SourceExt-converted-to}|\\
%   \xfile{foo.eps} $\Rightarrow$ \xfile{foo-eps-converted-to.pdf}
%   \end{quote}
%   See also the next option \xoption{prefersuffix} that
%   modifies the behaviour of option \xoption{suffix} in some
%   cases.
%
%   Default for \xoption{suffix} is `|-\SourceExt-converted-to|'.
%
% \It{prefersuffix}
%   If a suffix is set by option \xoption{suffix}, then there can be
%   two image file names that could be taken into account for inclusion:
%   A image file name with the suffix string inside
%   its name and a image file name without; e.g. for \xfile{foo.eps}
%   the names could be:
%   \begin{quote}
%   \xfile{foo-suffix.pdf}, \xfile{foo.pdf}
%   \end{quote}
%   If option \xoption{perfersuffix} is turned on, the
%   file \xfile{foo-suffix.pdf} and its generation is preferred
%   over using \xfile{foo.pdf}. Otherwise \xfile{foo.pdf} is included
%   without generating \xfile{foo-suffix.pdf}.
%   The default of option \xoption{prefersuffix} is \texttt{true}.
% \It{program@epstopdf}
%   The name for the conversion program from EPS to PDF,
%   default is `|epstopdf|'.
% \It{verbose}
%   It prints some information about the image in the \xfile{.log} file
%   (default).
% \end{description}
%
% \subsection{Configuration}
%
% \subsubsection{System configuration file \xfile{epstopdf-sys.cfg}}
%
% If \xfile{epstopdf-sys} exists it is loaded at the end of the
% package \xpackage{epstopdf-base} and before the user configuration
% file. It's intended for TeX distributors. Thus they could add
% additional conversion rules (e.g., .gif -> .png) or set options.
%
% \subsubsection{User configuration file \xfile{epstopdf.cfg}}
%
% A configuration file \xfile{epstopdf.cfg} is loaded at the end
% of the package if it exists. It can be used for changing
% the default option setting. Example:
% \begin{quote}
% |\epstopdfsetup{verbose=false}|
% \end{quote}
%
% \subsubsection{Conversion program}
%
% You can use \cs{DeclareGraphicsRule} in a similar way as the
% route via \xprogram{dvips} to specify the conversion
% command line.
% The conversion argument starts with a back tick, followed
% by the conversion command including parameters.
%
% The whole conversion argument should also be wrapped inside
% \cs{epstopdfcall}. This reduces the problem with packages
% (e.g. \xpackage{pst-pdf}) that use the conversion argument
% and expands it. Macros \cs{SourceFile}, \cs{OutputFile}, and
% \cs{SourceExt} are not defined outside \xpackage{epstopdf-base}'s
% \cs{Gin@setfile} and error messages because of undefined
% command names are the result. If \cs{epstopdfcall} detects that
% it is called outside \xpackage{epstopdf-base}'s \cs{Gin@setfile}
% then it replaces the conversion argument by
% package \xpackage{graphics}'s
% default, usually the image file.
%
% The following macros are available inside:
% \begin{description}
% \def\It#1{\item[\cs{#1}:]}
% \It{OutputFile}: output file name (with known path and extension)
% \It{SourceFile}: source file name (with known path and extension),
%                  usually the same as |#1|,
% \It{SourceExt}: source extension without leading dot.
% \end{description}
%
% \paragraph{Conversion from EPS to PDF.}
% Other programs than \xprogram{epstopdf} can be used to convert from
% EPS to PDF. Example that uses \xprogram{Ghostscript}:
% \begin{quote}
%   |\DeclareGraphicsRule{.eps}{pdf}{.pdf}{%|\\
%   |  \epstopdfcall{`ps2pdf -dEPSCrop #1 \noexpand\OutputFile}%|\\
%   |}|
% \end{quote}
% \cs{DeclareGraphicsRule} expands the argument, therefore \cs{noexpand}
% is necessary. As convenience package \xpackage{epstopdf-base} defines
% \cs{epstopdfDeclareGraphicsRule}. Then the conversion argument
% is not expanded, \cs{epstopdfcall} and the back tick are added:
% \begin{quote}
%   |\epstopdfDeclareGraphicsRule{.eps}{pdf}{.pdf}{%|\\
%   |  ps2pdf -dEPSCrop #1 \OutputFile|\\
%   |}|
% \end{quote}
%
% Also \cs{OutputFile} respects the setting of option \xoption{outdir}.
%
% \subsection{Other image formats}
%
% The support that package \xpackage{epstopdf} implements is not
% limited to EPS files. Other image conversions can be declared.
% The following example shows it for GIF images under Unix with
% ImageMagick's \xprogram{convert}:
%\begin{quote}
%\begin{verbatim}
%\epstopdfDeclareGraphicsRule{.gif}{png}{.png}{%
%  convert #1 \OutputFile
%}
%\end{verbatim}
%\end{quote}
% The file extension \xfile{.gif} can be added to the extension list that
% package \xpackage{graphics} searches if the file extension
% is not given in \cs{includegraphics}.
% The list can be set by \cs{GraphicsExtensions}.
% \begin{quote}
% |\AppendGraphicsExtensions{.gif}|\\
% or\\
% |\PrependGraphicsExtensions{.gif}|
% \end{quote}
%
% \StopEventually{
% }
%
% \section{Implementation}
%
%    \begin{macrocode}
%<*package>
%    \end{macrocode}
%
% \subsection{Wrapper package}
%
%    Reload check, especially if the package is not used with \LaTeX.
%    \begin{macrocode}
\begingroup\catcode61\catcode48\catcode32=10\relax%
  \catcode13=5 % ^^M
  \endlinechar=13 %
  \catcode35=6 % #
  \catcode39=12 % '
  \catcode44=12 % ,
  \catcode45=12 % -
  \catcode46=12 % .
  \catcode58=12 % :
  \catcode64=11 % @
  \catcode123=1 % {
  \catcode125=2 % }
  \expandafter\let\expandafter\x\csname ver@epstopdf.sty\endcsname
  \ifx\x\relax % plain-TeX, first loading
  \else
    \def\empty{}%
    \ifx\x\empty % LaTeX, first loading,
      % variable is initialized, but \ProvidesPackage not yet seen
    \else
      \expandafter\ifx\csname PackageInfo\endcsname\relax
        \def\x#1#2{%
          \immediate\write-1{Package #1 Info: #2.}%
        }%
      \else
        \def\x#1#2{\PackageInfo{#1}{#2, stopped}}%
      \fi
      \x{epstopdf}{The package is already loaded}%
      \aftergroup\endinput
    \fi
  \fi
\endgroup%
%    \end{macrocode}
%    Package identification:
%    \begin{macrocode}
\begingroup\catcode61\catcode48\catcode32=10\relax%
  \catcode13=5 % ^^M
  \endlinechar=13 %
  \catcode35=6 % #
  \catcode39=12 % '
  \catcode40=12 % (
  \catcode41=12 % )
  \catcode44=12 % ,
  \catcode45=12 % -
  \catcode46=12 % .
  \catcode47=12 % /
  \catcode58=12 % :
  \catcode64=11 % @
  \catcode91=12 % [
  \catcode93=12 % ]
  \catcode123=1 % {
  \catcode125=2 % }
  \expandafter\ifx\csname ProvidesPackage\endcsname\relax
    \def\x#1#2#3[#4]{\endgroup
      \immediate\write-1{Package: #3 #4}%
      \xdef#1{#4}%
    }%
  \else
    \def\x#1#2[#3]{\endgroup
      #2[{#3}]%
      \ifx#1\@undefined
        \xdef#1{#3}%
      \fi
      \ifx#1\relax
        \xdef#1{#3}%
      \fi
    }%
  \fi
\expandafter\x\csname ver@epstopdf.sty\endcsname
\ProvidesPackage{epstopdf}%
  [2016/05/15 v2.6 Conversion with epstopdf on the fly (HO)]%
%    \end{macrocode}
%    Larger catcode set because of configuration files needed.
%    \begin{macrocode}
\begingroup\catcode61\catcode48\catcode32=10\relax%
  \catcode13=5 % ^^M
  \endlinechar=13 %
  \catcode123=1 % {
  \catcode125=2 % }
  \catcode64=11 % @
  \def\x{\endgroup
    \expandafter\edef\csname ETE@AtEnd\endcsname{%
      \endlinechar=\the\endlinechar\relax
      \catcode13=\the\catcode13\relax
      \catcode32=\the\catcode32\relax
      \catcode35=\the\catcode35\relax
      \catcode61=\the\catcode61\relax
      \catcode64=\the\catcode64\relax
      \catcode123=\the\catcode123\relax
      \catcode125=\the\catcode125\relax
    }%
  }%
\x\catcode61\catcode48\catcode32=10\relax%
\catcode13=5 % ^^M
\endlinechar=13 %
\catcode35=6 % #
\catcode64=11 % @
\catcode123=1 % {
\catcode125=2 % }
\def\TMP@EnsureCode#1#2{%
  \edef\ETE@AtEnd{%
    \ETE@AtEnd
    \catcode#1=\the\catcode#1\relax
  }%
  \catcode#1=#2\relax
}
\TMP@EnsureCode{39}{12}% '
\TMP@EnsureCode{40}{12}% (
\TMP@EnsureCode{41}{12}% )
\TMP@EnsureCode{42}{12}% *
\TMP@EnsureCode{45}{12}% -
\TMP@EnsureCode{47}{12}% /
\TMP@EnsureCode{91}{12}% [
\TMP@EnsureCode{93}{12}% ]
\edef\ETE@AtEnd{\ETE@AtEnd\noexpand\endinput}
%    \end{macrocode}
%
%    \begin{macrocode}
\let\ETE@SavedAtEnd\ETE@AtEnd
\RequirePackage{epstopdf-base}[2016/05/15]
\let\ETE@AtEnd\ETE@SavedAtEnd
%    \end{macrocode}
%
% \subsubsection{Option handling}
%
%    \begin{macrocode}
\DeclareOption*{%
  \expandafter\epstopdfsetup\expandafter{\CurrentOption}%
}
\ProcessOptions*\relax
%    \end{macrocode}
%
%    \begin{macrocode}
\ETE@AtEnd%
%</package>
%    \end{macrocode}
%
% \subsection{Base package}
%
%    \begin{macrocode}
%<*base>
%    \end{macrocode}
%
% \subsection{Preparations}
%
% \subsubsection{Relead check and identification}
%
%    Reload check, especially if the package is not used with \LaTeX.
%    \begin{macrocode}
\begingroup\catcode61\catcode48\catcode32=10\relax%
  \catcode13=5 % ^^M
  \endlinechar=13 %
  \catcode35=6 % #
  \catcode39=12 % '
  \catcode44=12 % ,
  \catcode45=12 % -
  \catcode46=12 % .
  \catcode58=12 % :
  \catcode64=11 % @
  \catcode123=1 % {
  \catcode125=2 % }
  \expandafter\let\expandafter\x\csname ver@epstopdf-base.sty\endcsname
  \ifx\x\relax % plain-TeX, first loading
  \else
    \def\empty{}%
    \ifx\x\empty % LaTeX, first loading,
      % variable is initialized, but \ProvidesPackage not yet seen
    \else
      \expandafter\ifx\csname PackageInfo\endcsname\relax
        \def\x#1#2{%
          \immediate\write-1{Package #1 Info: #2.}%
        }%
      \else
        \def\x#1#2{\PackageInfo{#1}{#2, stopped}}%
      \fi
      \x{epstopdf-base}{The package is already loaded}%
      \aftergroup\endinput
    \fi
  \fi
\endgroup%
%    \end{macrocode}
%    Package identification:
%    \begin{macrocode}
\begingroup\catcode61\catcode48\catcode32=10\relax%
  \catcode13=5 % ^^M
  \endlinechar=13 %
  \catcode35=6 % #
  \catcode39=12 % '
  \catcode40=12 % (
  \catcode41=12 % )
  \catcode44=12 % ,
  \catcode45=12 % -
  \catcode46=12 % .
  \catcode47=12 % /
  \catcode58=12 % :
  \catcode64=11 % @
  \catcode91=12 % [
  \catcode93=12 % ]
  \catcode123=1 % {
  \catcode125=2 % }
  \expandafter\ifx\csname ProvidesPackage\endcsname\relax
    \def\x#1#2#3[#4]{\endgroup
      \immediate\write-1{Package: #3 #4}%
      \xdef#1{#4}%
    }%
  \else
    \def\x#1#2[#3]{\endgroup
      #2[{#3}]%
      \ifx#1\@undefined
        \xdef#1{#3}%
      \fi
      \ifx#1\relax
        \xdef#1{#3}%
      \fi
    }%
  \fi
\expandafter\x\csname ver@epstopdf-base.sty\endcsname
\ProvidesPackage{epstopdf-base}%
  [2016/05/15 v2.6 Base part for package epstopdf]%
%    \end{macrocode}
%
% \subsubsection{Catcodes}
%
%    \begin{macrocode}
\begingroup\catcode61\catcode48\catcode32=10\relax%
  \catcode13=5 % ^^M
  \endlinechar=13 %
  \catcode123=1 % {
  \catcode125=2 % }
  \catcode64=11 % @
  \def\x{\endgroup
    \expandafter\edef\csname ETE@AtEnd\endcsname{%
      \endlinechar=\the\endlinechar\relax
      \catcode13=\the\catcode13\relax
      \catcode32=\the\catcode32\relax
      \catcode35=\the\catcode35\relax
      \catcode61=\the\catcode61\relax
      \catcode64=\the\catcode64\relax
      \catcode123=\the\catcode123\relax
      \catcode125=\the\catcode125\relax
    }%
  }%
\x\catcode61\catcode48\catcode32=10\relax%
\catcode13=5 % ^^M
\endlinechar=13 %
\catcode35=6 % #
\catcode64=11 % @
\catcode123=1 % {
\catcode125=2 % }
\def\TMP@EnsureCode#1#2{%
  \edef\ETE@AtEnd{%
    \ETE@AtEnd
    \catcode#1=\the\catcode#1\relax
  }%
  \catcode#1=#2\relax
}
\TMP@EnsureCode{33}{12}% !
\TMP@EnsureCode{39}{12}% '
\TMP@EnsureCode{42}{12}% *
\TMP@EnsureCode{44}{12}% ,
\TMP@EnsureCode{45}{12}% -
\TMP@EnsureCode{46}{12}% .
\TMP@EnsureCode{47}{12}% /
\TMP@EnsureCode{58}{12}% :
\TMP@EnsureCode{60}{12}% <
\TMP@EnsureCode{62}{12}% >
\TMP@EnsureCode{96}{12}% `
\edef\ETE@AtEnd{\ETE@AtEnd\noexpand\endinput}
%    \end{macrocode}
%
% \subsubsection{Load packages}
%
%    \begin{macrocode}
\RequirePackage{infwarerr}[2007/09/09]
\RequirePackage{grfext}\relax
\RequirePackage{kvoptions}[2007/10/02]
\RequirePackage{pdftexcmds}[2007/11/11]
%    \end{macrocode}
%
% \subsection{Checks}
%
% Check, whether package graphics is loaded
% (also graphicx loads graphics).
% Because miniltx.tex does not know \cs{@ifpackageloaded}
% we test for \cs{Gin@setfile} instead.
%    \begin{macrocode}
\begingroup\expandafter\expandafter\expandafter\endgroup
\expandafter\ifx\csname Gin@setfile\endcsname\relax
  \@PackageWarningNoLine{epstopdf}{%
    No graphics package \string`graphic{s,x}\string' loaded%
  }%
  \newcommand*{\epstopdfsetup}[1]{}%
  \expandafter\ETE@AtEnd
\fi%
%    \end{macrocode}
%
% Check, whether pdftex.def is loaded. \cs{ver@pdftex.def} is not
% available with miniltx.tex, thus we test for \cs{Gin@driver}.
%    \begin{macrocode}
\begingroup
  \def\x{luatex.def}%
  \ifx\Gin@driver\x
  \else
  \def\x{pdftex.def}%
  \ifx\Gin@driver\x
  \else
    \@PackageWarningNoLine{epstopdf}{%
      Drivers other than `pdftex' and `luatex' are not supported%
    }%
    \endgroup
    \newcommand*{\epstopdfsetup}[1]{}%
    \expandafter\ETE@AtEnd
  \fi%
  \fi%
\endgroup
%    \end{macrocode}
%
% Check, whether the shell escape feature is enabled.
%
%    \begin{macrocode}
\begingroup
  \expandafter\ifx\csname pdf@shellescape\endcsname\relax
  \else
    \ifnum\pdf@shellescape>0 %
    \else
      \@PackageWarningNoLine{epstopdf}{%
        Shell escape feature is not enabled%
      }%
    \fi
  \fi
\endgroup
%    \end{macrocode}
%
% \subsection{Package loading}
%
%    \begin{macrocode}
%    \end{macrocode}
%
% \subsection{Options}
%
%    \begin{macrocode}
\SetupKeyvalOptions{family=ETE,prefix=ETE@}
\DeclareBoolOption{update}
\DeclareBoolOption{verbose}
\newif\ifETE@prepend
\DeclareVoidOption{prepend}{\ETE@prependtrue}
\DeclareVoidOption{append}{\ETE@prependfalse}
\DeclareStringOption{outdir}
\DeclareStringOption{suffix}
\DeclareBoolOption{prefersuffix}
\DeclareStringOption{program@epstopdf}
%    \end{macrocode}
%    Options \xoption{disable} and \xoption{enable} are for testing
%    only. Therefore they are not documented on purpose.
%    \begin{macrocode}
\DeclareBoolOption{disable}
\DeclareComplementaryOption{enable}{disable}
%    \end{macrocode}
%    \begin{macrocode}
\newcommand*{\epstopdfsetup}{\setkeys{ETE}}
%    \end{macrocode}
%
% \subsubsection{Default setting}
%
%    \begin{macrocode}
\epstopdfsetup{%
  verbose,%
  enable,%
  append,%
  update,%
  prefersuffix,%
  suffix=-\SourceExt-converted-to,%
  program@epstopdf=epstopdf%
}
%    \end{macrocode}
%
% \subsection{Make and verbose}
%
%    \begin{macrocode}
\begingroup\expandafter\expandafter\expandafter\endgroup
\expandafter\ifx\csname pdf@filemoddate\endcsname\relax
  \def\ETE@Make#1#2{%
    \ifETE@update
      \ETE@WarnModDate
    \fi
    \@firstofone
  }%
  \def\ETE@WarnModDate{%
    \@PackageWarningNoLine{epstopdf}{%
      \string\pdffilemoddate\space is not available,\MessageBreak
      option `update' will be ignored%
    }%
    \global\let\ETE@WarnModDate\relax
  }%
  \def\ETE@FileInfo#1#2{#1 file: <#2>}%
\else
  \def\ETE@Make#1#2{%
    \ifETE@update
      \ifnum\pdf@strcmp{\pdf@filemoddate{#1}}{\pdf@filemoddate{#2}}>0 %
        \expandafter\expandafter\expandafter\@firstofone
      \else
        \@PackageInfoNoLine{epstopdf}{%
          Output file is already uptodate%
        }%
        \expandafter\expandafter\expandafter\@gobble
      \fi
    \else
      \expandafter\@firstofone
    \fi
  }%
  \def\ETE@FileInfo#1#2{%
    #1 file: <#2>%
    \expandafter\expandafter\expandafter
    \ETE@Date\pdf@filemoddate{#2}\@nil
    \expandafter\expandafter\expandafter
    \ETE@Size\pdf@filesize{#2}\@nil
  }%
  \def\ETE@Date#1\@nil{%
    \ifx\\#1\\%
    \else
      \ETE@@Date#1\@nil
    \fi
  }%
  \def\ETE@@Date#1:#2#3#4#5#6#7#8#9{%
    \MessageBreak
    \@spaces\space\space\space date: #2#3#4#5-#6#7-#8#9 %
    \ETE@@Time
  }%
  \def\ETE@@Time#1#2#3#4#5#6#7\@nil{%
    #1#2:#3#4:#5#6%
  }%
  \def\ETE@Size#1\@nil{%
    \ifx\\#1\\%
    \else
      \MessageBreak
      \@spaces\space\space\space size: #1 bytes%
    \fi
  }%
\fi
%    \end{macrocode}
%
% \subsection{Adding conversion support}
%
%    Patch \cs{Gin@setfile} to execute |#3|, if it contains
%    a command.
%    \begin{macrocode}
\expandafter\ifx\csname ETE@OrgGin@setfile\endcsname\relax
  \let\ETE@OrgGin@setfile\Gin@setfile
\else
  \@PackageError{epstopdf}{%
    Command \string\ETE@OrgGin@setfile\space
    already defined.\MessageBreak
  }{%
    Probably some package has included the code of this package%
    \MessageBreak
    instead of using \string\RequirePackage{epstopdf}.%
    \MessageBreak
    \@ehc
  }%
\fi
\def\ETE@IfFileExists{%
  \begingroup\expandafter\expandafter\expandafter\endgroup
  \expandafter\ifx\csname grffile@IfFileExists\endcsname\relax
    \expandafter\IfFileExists
  \else
    \global\let\ETE@IfFileExists\grffile@IfFileExists
    \expandafter\grffile@IfFileExists
  \fi
}
\def\ETE@Skip#1#2\x\fi\fi{%
  \fi
  \fi
  \endgroup
  \fi
  \fi
  #1%
}
\newif\ifETE@InsideSetfile
\newcommand*{\epstopdfcall}[1]{%
  \ifETE@InsideSetfile
    \expandafter\@firstoftwo
  \else
    \expandafter\@secondoftwo
  \fi
  {`#1}%
  {\Gin@base\Gin@ext}%
}
\def\ETE@DefCommandLine#1{%
  \edef\CommandLine{\expandafter\fi\if`#1}%
}
\def\ETE@DefX#1{%
  \expandafter\expandafter\expandafter\def
  \expandafter\expandafter\expandafter\x
  \expandafter\expandafter\expandafter{%
    \expandafter\fi\if`#1\relax\else
  }%
}
\def\ETE@Gin@setfile#1#2#3{%
  \ifETE@disable
    \ETE@OrgGin@setfile{#1}{#2}{#3}%
  \else
    \begingroup
      \ETE@InsideSetfiletrue
      \ETE@DefX{#3}%
    \expandafter\endgroup
    \ifx\x\@empty
      \ETE@OrgGin@setfile{#1}{#2}{#3}%
    \else
      \begingroup
        \ETE@InsideSetfiletrue
        \def\GraphicsType{#1}%
        \def\GraphicsRead{#2}%
        \ifETE@prefersuffix
        \else
          \ifx\ETE@suffix\@empty
          \else
            \ETE@IfFileExists{\Gin@base\GraphicsRead}{%
              \ETE@Skip{%
                \ETE@OrgGin@setfile{#1}{#2}{\Gin@base#2}%
              }%
            }{%
              \let\next\relax
            }%
            \next
          \fi
        \fi
        \ifx\Gin@ext\relax
          \let\SourceExt\Gin@eext
          \def\SourceFile{\Gin@base\Gin@eext}%
        \else
          \let\SourceExt\Gin@ext
          \def\SourceFile{\Gin@base\Gin@ext}%
        \fi
        \edef\SourceExt{% remove dot
          \expandafter\@cdr\SourceExt\@empty\@nil
        }%
        \let\OutputDirectory\ETE@outdir
        \ifx\OutputDirectory\@empty
          \edef\OutputFile{\ETE@GenerateName{\Gin@base}{#2}}%
        \else
          \begingroup
            \filename@parse{\Gin@base#2}%
            \edef\x{\endgroup
              \def\noexpand\OutputFile{%
                \ETE@GenerateName{%
                  \OutputDirectory\filename@base
                }{#2}%
              }%
            }%
          \x
        \fi
        \ETE@DefCommandLine{#3}%
        \ifETE@verbose
          \@PackageInfo{epstopdf}{%
            \ETE@FileInfo{Source}\SourceFile\MessageBreak
            \ETE@FileInfo{Output}\OutputFile\MessageBreak
            Command: <\CommandLine>\MessageBreak
            \string\includegraphics
          }%
        \fi
        \ETE@Make\SourceFile\OutputFile{%
          \pdf@system{\CommandLine}%
          \ifETE@verbose
            \@PackageInfoNoLine{epstopdf}{%
              \ETE@FileInfo{Result}\OutputFile
            }%
          \fi
        }%
        \edef\x{\endgroup
          \ifx\OutputDirectory\@empty
          \else
            \def\noexpand\Gin@base{%
              \OutputDirectory\noexpand\filename@base
            }%
          \fi
          \ifx\ETE@suffix\@empty
          \else
            \edef\noexpand\Gin@base{%
              \noexpand\Gin@base\ETE@suffix
            }%
          \fi
          \noexpand\ETE@OrgGin@setfile{%
            \GraphicsType
          }{%
            \GraphicsRead
          }{%
            \OutputFile
          }%
        }%
      \x
    \fi
  \fi
}
\let\Gin@setfile\ETE@Gin@setfile
\def\ETE@GenerateName#1#2{%
  #1\ETE@suffix#2%
}
%    \end{macrocode}
%
% \subsection{Declare graphics rule}
%
%    \begin{macrocode}
\newcommand*{\epstopdfDeclareGraphicsRule}[4]{%
  \ifx\\#4\\%
    \@PackageError{epstopdf-base}{%
      Conversion command is missing%
    }\@ehc
  \else
    \begingroup
      \@ifundefined{Gin@rule@#1}{%
      }{%
        \@PackageInfo{epstopdf-base}{%
          Redefining graphics rule for `#1'%
        }%
      }%
    \endgroup
    \@namedef{Gin@rule@#1}##1{{#2}{#3}{\epstopdfcall{#4}}}%
  \fi
}
%    \end{macrocode}
%
% \cs{DeclareGraphicsRule} for .eps
%    \begin{macrocode}
\epstopdfDeclareGraphicsRule{.eps}{pdf}{.pdf}{%
  \ETE@epstopdf{#1}%
}
\def\ETE@epstopdf#1{%
  \ETE@program@epstopdf\space
  \ifcase\ifx\OutputDirectory\@empty
           \ifx\ETE@suffix\@empty
             1%
           \fi
         \fi
         0 %
    --outfile=\OutputFile\space
  \fi
  #1%
}
%    \end{macrocode}
%
%    \begin{macrocode}
\ifETE@prepend
  \expandafter\PrependGraphicsExtensions
\else
  \expandafter\AppendGraphicsExtensions
\fi
{.eps}
\let\ETE@prepend\@undefined
\DeclareVoidOption{prepend}{%
  \PrependGraphicsExtensions{.eps}%
}
\let\ETE@append\@undefined
\DeclareVoidOption{append}{%
  \AppendGraphicsExtensions{.eps}%
}
%    \end{macrocode}
%
%    \begin{macrocode}
\InputIfFileExists{epstopdf-sys.cfg}{}{}
\InputIfFileExists{epstopdf.cfg}{}{}
%    \end{macrocode}
%
%    \begin{macrocode}
\ETE@AtEnd%
%</base>
%    \end{macrocode}
%
% \section{Test}
%
% \subsection{Preface for standard catcode check}
%
%    \begin{macrocode}
%<*test1>
\input miniltx.tex\relax
\def\Gin@driver{pdftex.def}
\input graphicx.sty\relax
\resetatcatcode
%</test1>
%    \end{macrocode}
%
% \subsection{Catcode checks for loading}
%
%    \begin{macrocode}
%<*test1>
%    \end{macrocode}
%    \begin{macrocode}
\catcode`\{=1 %
\catcode`\}=2 %
\catcode`\#=6 %
\catcode`\@=11 %
\expandafter\ifx\csname count@\endcsname\relax
  \countdef\count@=255 %
\fi
\expandafter\ifx\csname @gobble\endcsname\relax
  \long\def\@gobble#1{}%
\fi
\expandafter\ifx\csname @firstofone\endcsname\relax
  \long\def\@firstofone#1{#1}%
\fi
\expandafter\ifx\csname loop\endcsname\relax
  \expandafter\@firstofone
\else
  \expandafter\@gobble
\fi
{%
  \def\loop#1\repeat{%
    \def\body{#1}%
    \iterate
  }%
  \def\iterate{%
    \body
      \let\next\iterate
    \else
      \let\next\relax
    \fi
    \next
  }%
  \let\repeat=\fi
}%
\def\RestoreCatcodes{}
\count@=0 %
\loop
  \edef\RestoreCatcodes{%
    \RestoreCatcodes
    \catcode\the\count@=\the\catcode\count@\relax
  }%
\ifnum\count@<255 %
  \advance\count@ 1 %
\repeat

\def\RangeCatcodeInvalid#1#2{%
  \count@=#1\relax
  \loop
    \catcode\count@=15 %
  \ifnum\count@<#2\relax
    \advance\count@ 1 %
  \repeat
}
\def\RangeCatcodeCheck#1#2#3{%
  \count@=#1\relax
  \loop
    \ifnum#3=\catcode\count@
    \else
      \errmessage{%
        Character \the\count@\space
        with wrong catcode \the\catcode\count@\space
        instead of \number#3%
      }%
    \fi
  \ifnum\count@<#2\relax
    \advance\count@ 1 %
  \repeat
}
\def\space{ }
\expandafter\ifx\csname LoadCommand\endcsname\relax
  \def\LoadCommand{\input epstopdf.sty\relax}%
\fi
\def\Test{%
  \RangeCatcodeInvalid{0}{47}%
  \RangeCatcodeInvalid{58}{64}%
  \RangeCatcodeInvalid{91}{96}%
  \RangeCatcodeInvalid{123}{255}%
  \catcode`\@=12 %
  \catcode`\\=0 %
  \catcode`\%=14 %
  \LoadCommand
  \RangeCatcodeCheck{0}{36}{15}%
  \RangeCatcodeCheck{37}{37}{14}%
  \RangeCatcodeCheck{38}{47}{15}%
  \RangeCatcodeCheck{48}{57}{12}%
  \RangeCatcodeCheck{58}{63}{15}%
  \RangeCatcodeCheck{64}{64}{12}%
  \RangeCatcodeCheck{65}{90}{11}%
  \RangeCatcodeCheck{91}{91}{15}%
  \RangeCatcodeCheck{92}{92}{0}%
  \RangeCatcodeCheck{93}{96}{15}%
  \RangeCatcodeCheck{97}{122}{11}%
  \RangeCatcodeCheck{123}{255}{15}%
  \RestoreCatcodes
}
\Test
\csname @@end\endcsname
\end
%    \end{macrocode}
%    \begin{macrocode}
%</test1>
%    \end{macrocode}
%
% \section{Installation}
%
% \subsection{Download}
%
% \paragraph{Package.} This package is available on
% CTAN\footnote{\url{http://ctan.org/pkg/epstopdf}}:
% \begin{description}
% \item[\CTAN{macros/latex/contrib/oberdiek/epstopdf.dtx}] The source file.
% \item[\CTAN{macros/latex/contrib/oberdiek/epstopdf.pdf}] Documentation.
% \end{description}
%
%
% \paragraph{Bundle.} All the packages of the bundle `oberdiek'
% are also available in a TDS compliant ZIP archive. There
% the packages are already unpacked and the documentation files
% are generated. The files and directories obey the TDS standard.
% \begin{description}
% \item[\CTAN{install/macros/latex/contrib/oberdiek.tds.zip}]
% \end{description}
% \emph{TDS} refers to the standard ``A Directory Structure
% for \TeX\ Files'' (\CTAN{tds/tds.pdf}). Directories
% with \xfile{texmf} in their name are usually organized this way.
%
% \subsection{Bundle installation}
%
% \paragraph{Unpacking.} Unpack the \xfile{oberdiek.tds.zip} in the
% TDS tree (also known as \xfile{texmf} tree) of your choice.
% Example (linux):
% \begin{quote}
%   |unzip oberdiek.tds.zip -d ~/texmf|
% \end{quote}
%
% \paragraph{Script installation.}
% Check the directory \xfile{TDS:scripts/oberdiek/} for
% scripts that need further installation steps.
% Package \xpackage{attachfile2} comes with the Perl script
% \xfile{pdfatfi.pl} that should be installed in such a way
% that it can be called as \texttt{pdfatfi}.
% Example (linux):
% \begin{quote}
%   |chmod +x scripts/oberdiek/pdfatfi.pl|\\
%   |cp scripts/oberdiek/pdfatfi.pl /usr/local/bin/|
% \end{quote}
%
% \subsection{Package installation}
%
% \paragraph{Unpacking.} The \xfile{.dtx} file is a self-extracting
% \docstrip\ archive. The files are extracted by running the
% \xfile{.dtx} through \plainTeX:
% \begin{quote}
%   \verb|tex epstopdf.dtx|
% \end{quote}
%
% \paragraph{TDS.} Now the different files must be moved into
% the different directories in your installation TDS tree
% (also known as \xfile{texmf} tree):
% \begin{quote}
% \def\t{^^A
% \begin{tabular}{@{}>{\ttfamily}l@{ $\rightarrow$ }>{\ttfamily}l@{}}
%   epstopdf.sty & tex/latex/oberdiek/epstopdf.sty\\
%   epstopdf-base.sty & tex/latex/oberdiek/epstopdf-base.sty\\
%   epstopdf.pdf & doc/latex/oberdiek/epstopdf.pdf\\
%   test/epstopdf-test1.tex & doc/latex/oberdiek/test/epstopdf-test1.tex\\
%   epstopdf.dtx & source/latex/oberdiek/epstopdf.dtx\\
% \end{tabular}^^A
% }^^A
% \sbox0{\t}^^A
% \ifdim\wd0>\linewidth
%   \begingroup
%     \advance\linewidth by\leftmargin
%     \advance\linewidth by\rightmargin
%   \edef\x{\endgroup
%     \def\noexpand\lw{\the\linewidth}^^A
%   }\x
%   \def\lwbox{^^A
%     \leavevmode
%     \hbox to \linewidth{^^A
%       \kern-\leftmargin\relax
%       \hss
%       \usebox0
%       \hss
%       \kern-\rightmargin\relax
%     }^^A
%   }^^A
%   \ifdim\wd0>\lw
%     \sbox0{\small\t}^^A
%     \ifdim\wd0>\linewidth
%       \ifdim\wd0>\lw
%         \sbox0{\footnotesize\t}^^A
%         \ifdim\wd0>\linewidth
%           \ifdim\wd0>\lw
%             \sbox0{\scriptsize\t}^^A
%             \ifdim\wd0>\linewidth
%               \ifdim\wd0>\lw
%                 \sbox0{\tiny\t}^^A
%                 \ifdim\wd0>\linewidth
%                   \lwbox
%                 \else
%                   \usebox0
%                 \fi
%               \else
%                 \lwbox
%               \fi
%             \else
%               \usebox0
%             \fi
%           \else
%             \lwbox
%           \fi
%         \else
%           \usebox0
%         \fi
%       \else
%         \lwbox
%       \fi
%     \else
%       \usebox0
%     \fi
%   \else
%     \lwbox
%   \fi
% \else
%   \usebox0
% \fi
% \end{quote}
% If you have a \xfile{docstrip.cfg} that configures and enables \docstrip's
% TDS installing feature, then some files can already be in the right
% place, see the documentation of \docstrip.
%
% \subsection{Refresh file name databases}
%
% If your \TeX~distribution
% (\teTeX, \mikTeX, \dots) relies on file name databases, you must refresh
% these. For example, \teTeX\ users run \verb|texhash| or
% \verb|mktexlsr|.
%
% \subsection{Some details for the interested}
%
% \paragraph{Attached source.}
%
% The PDF documentation on CTAN also includes the
% \xfile{.dtx} source file. It can be extracted by
% AcrobatReader 6 or higher. Another option is \textsf{pdftk},
% e.g. unpack the file into the current directory:
% \begin{quote}
%   \verb|pdftk epstopdf.pdf unpack_files output .|
% \end{quote}
%
% \paragraph{Unpacking with \LaTeX.}
% The \xfile{.dtx} chooses its action depending on the format:
% \begin{description}
% \item[\plainTeX:] Run \docstrip\ and extract the files.
% \item[\LaTeX:] Generate the documentation.
% \end{description}
% If you insist on using \LaTeX\ for \docstrip\ (really,
% \docstrip\ does not need \LaTeX), then inform the autodetect routine
% about your intention:
% \begin{quote}
%   \verb|latex \let\install=y% \iffalse meta-comment
%
% File: epstopdf.dtx
% Version: 2016/05/15 v2.6
% Info: Conversion with epstopdf on the fly
%
% Copyright (C) 2001, 2006-2010 by
%    Heiko Oberdiek <heiko.oberdiek at googlemail.com>
%
% This work may be distributed and/or modified under the
% conditions of the LaTeX Project Public License, either
% version 1.3c of this license or (at your option) any later
% version. This version of this license is in
%    http://www.latex-project.org/lppl/lppl-1-3c.txt
% and the latest version of this license is in
%    http://www.latex-project.org/lppl.txt
% and version 1.3 or later is part of all distributions of
% LaTeX version 2005/12/01 or later.
%
% This work has the LPPL maintenance status "maintained".
%
% This Current Maintainer of this work is Heiko Oberdiek.
%
% This work consists of the main source file epstopdf.dtx
% and the derived files
%    epstopdf.sty, epstopdf.pdf, epstopdf.ins, epstopdf.drv,
%    epstopdf-base.sty, epstopdf-test1.tex.
%
% Distribution:
%    CTAN:macros/latex/contrib/oberdiek/epstopdf.dtx
%    CTAN:macros/latex/contrib/oberdiek/epstopdf.pdf
%
% Unpacking:
%    (a) If epstopdf.ins is present:
%           tex epstopdf.ins
%    (b) Without epstopdf.ins:
%           tex epstopdf.dtx
%    (c) If you insist on using LaTeX
%           latex \let\install=y\input{epstopdf.dtx}
%        (quote the arguments according to the demands of your shell)
%
% Documentation:
%    (a) If epstopdf.drv is present:
%           latex epstopdf.drv
%    (b) Without epstopdf.drv:
%           latex epstopdf.dtx; ...
%    The class ltxdoc loads the configuration file ltxdoc.cfg
%    if available. Here you can specify further options, e.g.
%    use A4 as paper format:
%       \PassOptionsToClass{a4paper}{article}
%
%    Programm calls to get the documentation (example):
%       pdflatex epstopdf.dtx
%       makeindex -s gind.ist epstopdf.idx
%       pdflatex epstopdf.dtx
%       makeindex -s gind.ist epstopdf.idx
%       pdflatex epstopdf.dtx
%
% Installation:
%    TDS:tex/latex/oberdiek/epstopdf.sty
%    TDS:tex/latex/oberdiek/epstopdf-base.sty
%    TDS:doc/latex/oberdiek/epstopdf.pdf
%    TDS:doc/latex/oberdiek/test/epstopdf-test1.tex
%    TDS:source/latex/oberdiek/epstopdf.dtx
%
%<*ignore>
\begingroup
  \catcode123=1 %
  \catcode125=2 %
  \def\x{LaTeX2e}%
\expandafter\endgroup
\ifcase 0\ifx\install y1\fi\expandafter
         \ifx\csname processbatchFile\endcsname\relax\else1\fi
         \ifx\fmtname\x\else 1\fi\relax
\else\csname fi\endcsname
%</ignore>
%<*install>
\input docstrip.tex
\Msg{************************************************************************}
\Msg{* Installation}
\Msg{* Package: epstopdf 2016/05/15 v2.6 Conversion with epstopdf on the fly (HO)}
\Msg{************************************************************************}

\keepsilent
\askforoverwritefalse

\let\MetaPrefix\relax
\preamble

This is a generated file.

Project: epstopdf
Version: 2016/05/15 v2.6

Copyright (C) 2001, 2006-2010 by
   Heiko Oberdiek <heiko.oberdiek at googlemail.com>

This work may be distributed and/or modified under the
conditions of the LaTeX Project Public License, either
version 1.3c of this license or (at your option) any later
version. This version of this license is in
   http://www.latex-project.org/lppl/lppl-1-3c.txt
and the latest version of this license is in
   http://www.latex-project.org/lppl.txt
and version 1.3 or later is part of all distributions of
LaTeX version 2005/12/01 or later.

This work has the LPPL maintenance status "maintained".

This Current Maintainer of this work is Heiko Oberdiek.

This work consists of the main source file epstopdf.dtx
and the derived files
   epstopdf.sty, epstopdf.pdf, epstopdf.ins, epstopdf.drv,
   epstopdf-base.sty, epstopdf-test1.tex.

\endpreamble
\let\MetaPrefix\DoubleperCent

\generate{%
  \file{epstopdf.ins}{\from{epstopdf.dtx}{install}}%
  \file{epstopdf.drv}{\from{epstopdf.dtx}{driver}}%
  \usedir{tex/latex/oberdiek}%
  \file{epstopdf.sty}{\from{epstopdf.dtx}{package}}%
  \file{epstopdf-base.sty}{\from{epstopdf.dtx}{base}}%
  \usedir{doc/latex/oberdiek/test}%
  \file{epstopdf-test1.tex}{\from{epstopdf.dtx}{test1}}%
  \nopreamble
  \nopostamble
  \usedir{source/latex/oberdiek/catalogue}%
  \file{epstopdf-pkg.xml}{\from{epstopdf.dtx}{catalogue}}%
}

\catcode32=13\relax% active space
\let =\space%
\Msg{************************************************************************}
\Msg{*}
\Msg{* To finish the installation you have to move the following}
\Msg{* files into a directory searched by TeX:}
\Msg{*}
\Msg{*     epstopdf.sty, epstopdf-base.sty}
\Msg{*}
\Msg{* To produce the documentation run the file `epstopdf.drv'}
\Msg{* through LaTeX.}
\Msg{*}
\Msg{* Happy TeXing!}
\Msg{*}
\Msg{************************************************************************}

\endbatchfile
%</install>
%<*ignore>
\fi
%</ignore>
%<*driver>
\NeedsTeXFormat{LaTeX2e}
\ProvidesFile{epstopdf.drv}%
  [2016/05/15 v2.6 Conversion with epstopdf on the fly (HO)]%
\documentclass{ltxdoc}
\usepackage{holtxdoc}[2011/11/22]
\begin{document}
  \DocInput{epstopdf.dtx}%
\end{document}
%</driver>
% \fi
%
%
% \CharacterTable
%  {Upper-case    \A\B\C\D\E\F\G\H\I\J\K\L\M\N\O\P\Q\R\S\T\U\V\W\X\Y\Z
%   Lower-case    \a\b\c\d\e\f\g\h\i\j\k\l\m\n\o\p\q\r\s\t\u\v\w\x\y\z
%   Digits        \0\1\2\3\4\5\6\7\8\9
%   Exclamation   \!     Double quote  \"     Hash (number) \#
%   Dollar        \$     Percent       \%     Ampersand     \&
%   Acute accent  \'     Left paren    \(     Right paren   \)
%   Asterisk      \*     Plus          \+     Comma         \,
%   Minus         \-     Point         \.     Solidus       \/
%   Colon         \:     Semicolon     \;     Less than     \<
%   Equals        \=     Greater than  \>     Question mark \?
%   Commercial at \@     Left bracket  \[     Backslash     \\
%   Right bracket \]     Circumflex    \^     Underscore    \_
%   Grave accent  \`     Left brace    \{     Vertical bar  \|
%   Right brace   \}     Tilde         \~}
%
% \GetFileInfo{epstopdf.drv}
%
% \title{The \xpackage{epstopdf} package}
% \date{2016/05/15 v2.6}
% \author{Heiko Oberdiek\thanks
% {Please report any issues at https://github.com/ho-tex/oberdiek/issues}\\
% \xemail{heiko.oberdiek at googlemail.com}}
%
% \maketitle
%
% \begin{abstract}
% This packages adds support of handling eps images
% to package \xpackage{graphics} or \xpackage{graphicx}
% with option \xoption{pdftex}. If an eps image is detected,
% epstopdf is automatically called to convert it to pdf format.
% \end{abstract}
%
% \tableofcontents
%
% \providecommand*{\xprogram}[1]{\textsf{#1}}
%
% \section{Documentation}
%
% \subsection{Introduction}
%
% \LaTeX\ provides its graphics bundle to include graphics files.
% Both packages \xpackage{graphics} or \xpackage{graphicx} may
% be used. the latter one loads the first and adds options
% in key value style for \cs{includegraphics}.
%
% Usually the drivers do not support all kind of graphics files.
% Other image types must be converted, before they become
% usuable. In case of driver \xprogram{dvips}, the graphics
% rule may contain a conversion rule. Then all that package
% \xpackage{graphics} must know is the bounding box, the command
% is passed to \xprogram{dvips} that calls it and embeds the
% converted image.
%
% However, \pdfTeX\ has its driver for PDF output already
% build in. It's graphics inclusion commands (\cs{pdfximage})
% does not allow the execution of external commands.
% Therefore commands in the last argument of
% \cs{DeclareGraphicsRule} were of no use.
% But external programs can be called within \pdfTeX.
% This feature is called ``shell escape'' or ``write 18''
% and must usually enabled explicitely because of security reasons.
% Now, this package \xpackage{epstopdf} hooks
% into package \xpackage{graphics}' code to catch
% that argument with the external command and executes it
% to convert the graphics file to a supported format and
% passes the control of graphics inclusion back to package
% \xpackage{graphics}.
%
% \subsection{Requirements}
%
% \begin{itemize}
% \item
%   The feature \cs{write18} must be enabled. This allows the
%   running of external programs during \TeX's compile run.
%   Keep in mind that this is a security risk. The feature is an
%   addition to \cs{TeX}. Mik\TeX, te\TeX, \TeX\ Live support it.
%   In Web2C based \TeX\ distributions (te\TeX, \TeX\ Live) it can be
%   enabled in the configuration file \xfile{texmf.cnf}:
%   \begin{quote}
%     |shell_escape = 1|
%   \end{quote}
%   Because of the security risk, it is better to do it on the
%   command line only:
%   \begin{quote}
%   \texttt{--shell-escape} (te\TeX, \TeX\ Live)\\
%   \texttt{--enable-write18} (MiK\TeX)
%   \end{quote}
%   Example:
%   \begin{quote}
%     |pdflatex -shell-escape test.tex|
%   \end{quote}
% \item
%   The program \xprogram{epstopdf} for the conversion from EPS
%   to PDF. However, other programs can
%   be used and configured by \cs{DeclareGraphicsRule}.
%   Example:
%   \begin{quote}
%   |\epstopdfDeclareGraphicsRule{.eps}{pdf}{.pdf}{%|\\
%   |  ps2pdf -dEPSCrop #1 \OutputFile|\\
%   |}|
%   \end{quote}
% \end{itemize}
%
% \subsection{Usage}
%
% The package is loaded after graphic\{s,x\}, e.g.:
% \begin{quote}
%   |\usepackage[pdftex]{graphicx}|\\
%   |\usepackage{epstopdf}|
% \end{quote}
% Now images with file name extension \xfile{.eps} are detected
% and supported using \cs{includegraphics}.
%
% If the graphics file name is explicitly specified with extension
% \xfile{.eps} the new rule for EPS files is called and
% the conversion performed. If option \xoption{update} is in force
% then the conversion step is dropped if the target file already
% exists and is not older then the EPS file.
%
% The situation is more complicate if the graphics file is
% given without file name extension. Then the \xpackage{graphics}
% package must search for a supported image file.
% The possible extensions are stored in the graphics extension
% list, that can be set by \cs{DeclareGraphicsExtensions}.
% The algorithm:
% \begin{quote}
%    \def\>{\mbox{}\quad}
%    function search( \meta{filebase} )\\
%    \>foreach \meta{ext} in \meta{graphics extensions}\\
%    \>\>foreach \meta{dir} in \meta{current directory}, \meta{\cs{graphicspath}}\\
%    \>\>\>\meta{file} := \meta{dir} + \meta{filebase} + \meta{ext}\\
%    \>\>\>if exist \meta{file}\\
%    \>\>\>\>return found\\
%    \>return not found
% \end{quote}
% Package \xpackage{epstopdf} puts \xfile{.eps} at the end of the
% graphics extension search list. This is the behaviour of option
% \xoption{append} that is enabled by default.
% That means, the conversion is called last
% unless a supported file type cannot be found earlier. This avoids
% unnecessary conversion steps that slow down the \LaTeX\ run.
% If you want to use option \xoption{update} and your \pdfTeX\ supports
% it, then an outdated PDF file also would be found earlier unless
% \xoption{suffix} is used that is the default since version 2.0.
%
% With an empty option \xoption{suffix} and option
% \xoption{prepend} there is a risk that an original PDF file is overwritten:
%
% If the original image file is the PDF file and there is also a generated
% EPS file, then the original PDF file can be regenerated (depending
% on the option settings) and the original PDF file gets lost.
% Therefore option \xoption{suffix} is introduced in version 1.9
% to create a separate name space for generated output files.
%
% \textbf{Note:} Usually the conversion program needs the exact location
% of the image file. Usually the current directory works.
% Also if the image file is found using \cs{graphicspath}, the
% location is known. However, if the image is somewhere
% in a directory of environment variable |TEXINPUTS|, then
% the package does not know the exact location and the
% conversion program will not find the image file unless it
% implements a search using |TEXINPUTS| (program \xprogram{kpsewhich}
% may be of help in this task).
%
% \subsection{Options}
%
% Options can be given as package options or later using:
% \begin{declcs}{epstopdfsetup} \M{key value list}
% \end{declcs}
% \LaTeX\ expands the option list before passing the option list
% to the package's option handling code. This can fail
% for option \xoption{suffix} if it contains some of the macros
% described below. Use \cs{epstopdfsetup} after the package is loaded.
% Or load package \xpackage{kvoptions-patch} before. This package
% is also loaded by option \xoption{patch} of package \xpackage{kvoptions}.
% \LaTeX's option code is redefined to respect key value options and
% let the values untouched.
%
% \begin{description}
% \def\It#1{\item[\xoption{#1}:]}^^A
% \It{update}
%   The conversion program is only called, if the target file
%   does not exist or is older than the source image file.
% \It{append}
%   Puts the extension \xfile{.eps} at the end of the graphics extension list
%   (default).
% \It{prepend}
%   Puts the extension \xfile{.eps} at the begin of the graphics extension list.
% \It{outdir}
%   The converted file may put in an other output directory.
%   The value of \xoption{outdir} must include the directory separator.
%   Example for the current directory:
%   \begin{quote}
%   |\epstopdfsetup{outdir=./}|
%   \end{quote}
%   For other directories ensure, that they can be found.
%   See \cs{graphicspath} or |TEXINPUTS|.
% \It{suffix}
%   This option takes a string that is put between the file name base
%   and the extension of the output file. Rationale: It can happen,
%   that a PDF file is the original file and the EPS file the
%   generated file. If now the package thinks, that the PDF file
%   is the generated file, it will `regnerate' it. But in reality
%   the original file is lost. Therefore I recommend to use this
%   option always to generate a separate name space for generated
%   files. Proposed value is \texttt{-generated} or \texttt{.generated}.
%   The suffix \texttt{.generated} will also work here without the need for
%   package \xpackage{grffile}).
%
%   Example:
%   \begin{quote}
%   |\epstopdfsetup{suffix=-generated}|\\
%   Then \xfile{foo.eps} is converted to \xfile{foo-generated.pdf}.
%   \end{quote}
%   \cs{SourceExt} can be used inside the suffix string.
%   It's will be replaced by the extension of the image source
%   file without the leading dot, for instance:
%   \begin{quote}
%   |\epstopdfsetup{suffix=-\SourceExt-converted-to}|\\
%   \xfile{foo.eps} $\Rightarrow$ \xfile{foo-eps-converted-to.pdf}
%   \end{quote}
%   See also the next option \xoption{prefersuffix} that
%   modifies the behaviour of option \xoption{suffix} in some
%   cases.
%
%   Default for \xoption{suffix} is `|-\SourceExt-converted-to|'.
%
% \It{prefersuffix}
%   If a suffix is set by option \xoption{suffix}, then there can be
%   two image file names that could be taken into account for inclusion:
%   A image file name with the suffix string inside
%   its name and a image file name without; e.g. for \xfile{foo.eps}
%   the names could be:
%   \begin{quote}
%   \xfile{foo-suffix.pdf}, \xfile{foo.pdf}
%   \end{quote}
%   If option \xoption{perfersuffix} is turned on, the
%   file \xfile{foo-suffix.pdf} and its generation is preferred
%   over using \xfile{foo.pdf}. Otherwise \xfile{foo.pdf} is included
%   without generating \xfile{foo-suffix.pdf}.
%   The default of option \xoption{prefersuffix} is \texttt{true}.
% \It{program@epstopdf}
%   The name for the conversion program from EPS to PDF,
%   default is `|epstopdf|'.
% \It{verbose}
%   It prints some information about the image in the \xfile{.log} file
%   (default).
% \end{description}
%
% \subsection{Configuration}
%
% \subsubsection{System configuration file \xfile{epstopdf-sys.cfg}}
%
% If \xfile{epstopdf-sys} exists it is loaded at the end of the
% package \xpackage{epstopdf-base} and before the user configuration
% file. It's intended for TeX distributors. Thus they could add
% additional conversion rules (e.g., .gif -> .png) or set options.
%
% \subsubsection{User configuration file \xfile{epstopdf.cfg}}
%
% A configuration file \xfile{epstopdf.cfg} is loaded at the end
% of the package if it exists. It can be used for changing
% the default option setting. Example:
% \begin{quote}
% |\epstopdfsetup{verbose=false}|
% \end{quote}
%
% \subsubsection{Conversion program}
%
% You can use \cs{DeclareGraphicsRule} in a similar way as the
% route via \xprogram{dvips} to specify the conversion
% command line.
% The conversion argument starts with a back tick, followed
% by the conversion command including parameters.
%
% The whole conversion argument should also be wrapped inside
% \cs{epstopdfcall}. This reduces the problem with packages
% (e.g. \xpackage{pst-pdf}) that use the conversion argument
% and expands it. Macros \cs{SourceFile}, \cs{OutputFile}, and
% \cs{SourceExt} are not defined outside \xpackage{epstopdf-base}'s
% \cs{Gin@setfile} and error messages because of undefined
% command names are the result. If \cs{epstopdfcall} detects that
% it is called outside \xpackage{epstopdf-base}'s \cs{Gin@setfile}
% then it replaces the conversion argument by
% package \xpackage{graphics}'s
% default, usually the image file.
%
% The following macros are available inside:
% \begin{description}
% \def\It#1{\item[\cs{#1}:]}
% \It{OutputFile}: output file name (with known path and extension)
% \It{SourceFile}: source file name (with known path and extension),
%                  usually the same as |#1|,
% \It{SourceExt}: source extension without leading dot.
% \end{description}
%
% \paragraph{Conversion from EPS to PDF.}
% Other programs than \xprogram{epstopdf} can be used to convert from
% EPS to PDF. Example that uses \xprogram{Ghostscript}:
% \begin{quote}
%   |\DeclareGraphicsRule{.eps}{pdf}{.pdf}{%|\\
%   |  \epstopdfcall{`ps2pdf -dEPSCrop #1 \noexpand\OutputFile}%|\\
%   |}|
% \end{quote}
% \cs{DeclareGraphicsRule} expands the argument, therefore \cs{noexpand}
% is necessary. As convenience package \xpackage{epstopdf-base} defines
% \cs{epstopdfDeclareGraphicsRule}. Then the conversion argument
% is not expanded, \cs{epstopdfcall} and the back tick are added:
% \begin{quote}
%   |\epstopdfDeclareGraphicsRule{.eps}{pdf}{.pdf}{%|\\
%   |  ps2pdf -dEPSCrop #1 \OutputFile|\\
%   |}|
% \end{quote}
%
% Also \cs{OutputFile} respects the setting of option \xoption{outdir}.
%
% \subsection{Other image formats}
%
% The support that package \xpackage{epstopdf} implements is not
% limited to EPS files. Other image conversions can be declared.
% The following example shows it for GIF images under Unix with
% ImageMagick's \xprogram{convert}:
%\begin{quote}
%\begin{verbatim}
%\epstopdfDeclareGraphicsRule{.gif}{png}{.png}{%
%  convert #1 \OutputFile
%}
%\end{verbatim}
%\end{quote}
% The file extension \xfile{.gif} can be added to the extension list that
% package \xpackage{graphics} searches if the file extension
% is not given in \cs{includegraphics}.
% The list can be set by \cs{GraphicsExtensions}.
% \begin{quote}
% |\AppendGraphicsExtensions{.gif}|\\
% or\\
% |\PrependGraphicsExtensions{.gif}|
% \end{quote}
%
% \StopEventually{
% }
%
% \section{Implementation}
%
%    \begin{macrocode}
%<*package>
%    \end{macrocode}
%
% \subsection{Wrapper package}
%
%    Reload check, especially if the package is not used with \LaTeX.
%    \begin{macrocode}
\begingroup\catcode61\catcode48\catcode32=10\relax%
  \catcode13=5 % ^^M
  \endlinechar=13 %
  \catcode35=6 % #
  \catcode39=12 % '
  \catcode44=12 % ,
  \catcode45=12 % -
  \catcode46=12 % .
  \catcode58=12 % :
  \catcode64=11 % @
  \catcode123=1 % {
  \catcode125=2 % }
  \expandafter\let\expandafter\x\csname ver@epstopdf.sty\endcsname
  \ifx\x\relax % plain-TeX, first loading
  \else
    \def\empty{}%
    \ifx\x\empty % LaTeX, first loading,
      % variable is initialized, but \ProvidesPackage not yet seen
    \else
      \expandafter\ifx\csname PackageInfo\endcsname\relax
        \def\x#1#2{%
          \immediate\write-1{Package #1 Info: #2.}%
        }%
      \else
        \def\x#1#2{\PackageInfo{#1}{#2, stopped}}%
      \fi
      \x{epstopdf}{The package is already loaded}%
      \aftergroup\endinput
    \fi
  \fi
\endgroup%
%    \end{macrocode}
%    Package identification:
%    \begin{macrocode}
\begingroup\catcode61\catcode48\catcode32=10\relax%
  \catcode13=5 % ^^M
  \endlinechar=13 %
  \catcode35=6 % #
  \catcode39=12 % '
  \catcode40=12 % (
  \catcode41=12 % )
  \catcode44=12 % ,
  \catcode45=12 % -
  \catcode46=12 % .
  \catcode47=12 % /
  \catcode58=12 % :
  \catcode64=11 % @
  \catcode91=12 % [
  \catcode93=12 % ]
  \catcode123=1 % {
  \catcode125=2 % }
  \expandafter\ifx\csname ProvidesPackage\endcsname\relax
    \def\x#1#2#3[#4]{\endgroup
      \immediate\write-1{Package: #3 #4}%
      \xdef#1{#4}%
    }%
  \else
    \def\x#1#2[#3]{\endgroup
      #2[{#3}]%
      \ifx#1\@undefined
        \xdef#1{#3}%
      \fi
      \ifx#1\relax
        \xdef#1{#3}%
      \fi
    }%
  \fi
\expandafter\x\csname ver@epstopdf.sty\endcsname
\ProvidesPackage{epstopdf}%
  [2016/05/15 v2.6 Conversion with epstopdf on the fly (HO)]%
%    \end{macrocode}
%    Larger catcode set because of configuration files needed.
%    \begin{macrocode}
\begingroup\catcode61\catcode48\catcode32=10\relax%
  \catcode13=5 % ^^M
  \endlinechar=13 %
  \catcode123=1 % {
  \catcode125=2 % }
  \catcode64=11 % @
  \def\x{\endgroup
    \expandafter\edef\csname ETE@AtEnd\endcsname{%
      \endlinechar=\the\endlinechar\relax
      \catcode13=\the\catcode13\relax
      \catcode32=\the\catcode32\relax
      \catcode35=\the\catcode35\relax
      \catcode61=\the\catcode61\relax
      \catcode64=\the\catcode64\relax
      \catcode123=\the\catcode123\relax
      \catcode125=\the\catcode125\relax
    }%
  }%
\x\catcode61\catcode48\catcode32=10\relax%
\catcode13=5 % ^^M
\endlinechar=13 %
\catcode35=6 % #
\catcode64=11 % @
\catcode123=1 % {
\catcode125=2 % }
\def\TMP@EnsureCode#1#2{%
  \edef\ETE@AtEnd{%
    \ETE@AtEnd
    \catcode#1=\the\catcode#1\relax
  }%
  \catcode#1=#2\relax
}
\TMP@EnsureCode{39}{12}% '
\TMP@EnsureCode{40}{12}% (
\TMP@EnsureCode{41}{12}% )
\TMP@EnsureCode{42}{12}% *
\TMP@EnsureCode{45}{12}% -
\TMP@EnsureCode{47}{12}% /
\TMP@EnsureCode{91}{12}% [
\TMP@EnsureCode{93}{12}% ]
\edef\ETE@AtEnd{\ETE@AtEnd\noexpand\endinput}
%    \end{macrocode}
%
%    \begin{macrocode}
\let\ETE@SavedAtEnd\ETE@AtEnd
\RequirePackage{epstopdf-base}[2016/05/15]
\let\ETE@AtEnd\ETE@SavedAtEnd
%    \end{macrocode}
%
% \subsubsection{Option handling}
%
%    \begin{macrocode}
\DeclareOption*{%
  \expandafter\epstopdfsetup\expandafter{\CurrentOption}%
}
\ProcessOptions*\relax
%    \end{macrocode}
%
%    \begin{macrocode}
\ETE@AtEnd%
%</package>
%    \end{macrocode}
%
% \subsection{Base package}
%
%    \begin{macrocode}
%<*base>
%    \end{macrocode}
%
% \subsection{Preparations}
%
% \subsubsection{Relead check and identification}
%
%    Reload check, especially if the package is not used with \LaTeX.
%    \begin{macrocode}
\begingroup\catcode61\catcode48\catcode32=10\relax%
  \catcode13=5 % ^^M
  \endlinechar=13 %
  \catcode35=6 % #
  \catcode39=12 % '
  \catcode44=12 % ,
  \catcode45=12 % -
  \catcode46=12 % .
  \catcode58=12 % :
  \catcode64=11 % @
  \catcode123=1 % {
  \catcode125=2 % }
  \expandafter\let\expandafter\x\csname ver@epstopdf-base.sty\endcsname
  \ifx\x\relax % plain-TeX, first loading
  \else
    \def\empty{}%
    \ifx\x\empty % LaTeX, first loading,
      % variable is initialized, but \ProvidesPackage not yet seen
    \else
      \expandafter\ifx\csname PackageInfo\endcsname\relax
        \def\x#1#2{%
          \immediate\write-1{Package #1 Info: #2.}%
        }%
      \else
        \def\x#1#2{\PackageInfo{#1}{#2, stopped}}%
      \fi
      \x{epstopdf-base}{The package is already loaded}%
      \aftergroup\endinput
    \fi
  \fi
\endgroup%
%    \end{macrocode}
%    Package identification:
%    \begin{macrocode}
\begingroup\catcode61\catcode48\catcode32=10\relax%
  \catcode13=5 % ^^M
  \endlinechar=13 %
  \catcode35=6 % #
  \catcode39=12 % '
  \catcode40=12 % (
  \catcode41=12 % )
  \catcode44=12 % ,
  \catcode45=12 % -
  \catcode46=12 % .
  \catcode47=12 % /
  \catcode58=12 % :
  \catcode64=11 % @
  \catcode91=12 % [
  \catcode93=12 % ]
  \catcode123=1 % {
  \catcode125=2 % }
  \expandafter\ifx\csname ProvidesPackage\endcsname\relax
    \def\x#1#2#3[#4]{\endgroup
      \immediate\write-1{Package: #3 #4}%
      \xdef#1{#4}%
    }%
  \else
    \def\x#1#2[#3]{\endgroup
      #2[{#3}]%
      \ifx#1\@undefined
        \xdef#1{#3}%
      \fi
      \ifx#1\relax
        \xdef#1{#3}%
      \fi
    }%
  \fi
\expandafter\x\csname ver@epstopdf-base.sty\endcsname
\ProvidesPackage{epstopdf-base}%
  [2016/05/15 v2.6 Base part for package epstopdf]%
%    \end{macrocode}
%
% \subsubsection{Catcodes}
%
%    \begin{macrocode}
\begingroup\catcode61\catcode48\catcode32=10\relax%
  \catcode13=5 % ^^M
  \endlinechar=13 %
  \catcode123=1 % {
  \catcode125=2 % }
  \catcode64=11 % @
  \def\x{\endgroup
    \expandafter\edef\csname ETE@AtEnd\endcsname{%
      \endlinechar=\the\endlinechar\relax
      \catcode13=\the\catcode13\relax
      \catcode32=\the\catcode32\relax
      \catcode35=\the\catcode35\relax
      \catcode61=\the\catcode61\relax
      \catcode64=\the\catcode64\relax
      \catcode123=\the\catcode123\relax
      \catcode125=\the\catcode125\relax
    }%
  }%
\x\catcode61\catcode48\catcode32=10\relax%
\catcode13=5 % ^^M
\endlinechar=13 %
\catcode35=6 % #
\catcode64=11 % @
\catcode123=1 % {
\catcode125=2 % }
\def\TMP@EnsureCode#1#2{%
  \edef\ETE@AtEnd{%
    \ETE@AtEnd
    \catcode#1=\the\catcode#1\relax
  }%
  \catcode#1=#2\relax
}
\TMP@EnsureCode{33}{12}% !
\TMP@EnsureCode{39}{12}% '
\TMP@EnsureCode{42}{12}% *
\TMP@EnsureCode{44}{12}% ,
\TMP@EnsureCode{45}{12}% -
\TMP@EnsureCode{46}{12}% .
\TMP@EnsureCode{47}{12}% /
\TMP@EnsureCode{58}{12}% :
\TMP@EnsureCode{60}{12}% <
\TMP@EnsureCode{62}{12}% >
\TMP@EnsureCode{96}{12}% `
\edef\ETE@AtEnd{\ETE@AtEnd\noexpand\endinput}
%    \end{macrocode}
%
% \subsubsection{Load packages}
%
%    \begin{macrocode}
\RequirePackage{infwarerr}[2007/09/09]
\RequirePackage{grfext}\relax
\RequirePackage{kvoptions}[2007/10/02]
\RequirePackage{pdftexcmds}[2007/11/11]
%    \end{macrocode}
%
% \subsection{Checks}
%
% Check, whether package graphics is loaded
% (also graphicx loads graphics).
% Because miniltx.tex does not know \cs{@ifpackageloaded}
% we test for \cs{Gin@setfile} instead.
%    \begin{macrocode}
\begingroup\expandafter\expandafter\expandafter\endgroup
\expandafter\ifx\csname Gin@setfile\endcsname\relax
  \@PackageWarningNoLine{epstopdf}{%
    No graphics package \string`graphic{s,x}\string' loaded%
  }%
  \newcommand*{\epstopdfsetup}[1]{}%
  \expandafter\ETE@AtEnd
\fi%
%    \end{macrocode}
%
% Check, whether pdftex.def is loaded. \cs{ver@pdftex.def} is not
% available with miniltx.tex, thus we test for \cs{Gin@driver}.
%    \begin{macrocode}
\begingroup
  \def\x{luatex.def}%
  \ifx\Gin@driver\x
  \else
  \def\x{pdftex.def}%
  \ifx\Gin@driver\x
  \else
    \@PackageWarningNoLine{epstopdf}{%
      Drivers other than `pdftex' and `luatex' are not supported%
    }%
    \endgroup
    \newcommand*{\epstopdfsetup}[1]{}%
    \expandafter\ETE@AtEnd
  \fi%
  \fi%
\endgroup
%    \end{macrocode}
%
% Check, whether the shell escape feature is enabled.
%
%    \begin{macrocode}
\begingroup
  \expandafter\ifx\csname pdf@shellescape\endcsname\relax
  \else
    \ifnum\pdf@shellescape>0 %
    \else
      \@PackageWarningNoLine{epstopdf}{%
        Shell escape feature is not enabled%
      }%
    \fi
  \fi
\endgroup
%    \end{macrocode}
%
% \subsection{Package loading}
%
%    \begin{macrocode}
%    \end{macrocode}
%
% \subsection{Options}
%
%    \begin{macrocode}
\SetupKeyvalOptions{family=ETE,prefix=ETE@}
\DeclareBoolOption{update}
\DeclareBoolOption{verbose}
\newif\ifETE@prepend
\DeclareVoidOption{prepend}{\ETE@prependtrue}
\DeclareVoidOption{append}{\ETE@prependfalse}
\DeclareStringOption{outdir}
\DeclareStringOption{suffix}
\DeclareBoolOption{prefersuffix}
\DeclareStringOption{program@epstopdf}
%    \end{macrocode}
%    Options \xoption{disable} and \xoption{enable} are for testing
%    only. Therefore they are not documented on purpose.
%    \begin{macrocode}
\DeclareBoolOption{disable}
\DeclareComplementaryOption{enable}{disable}
%    \end{macrocode}
%    \begin{macrocode}
\newcommand*{\epstopdfsetup}{\setkeys{ETE}}
%    \end{macrocode}
%
% \subsubsection{Default setting}
%
%    \begin{macrocode}
\epstopdfsetup{%
  verbose,%
  enable,%
  append,%
  update,%
  prefersuffix,%
  suffix=-\SourceExt-converted-to,%
  program@epstopdf=epstopdf%
}
%    \end{macrocode}
%
% \subsection{Make and verbose}
%
%    \begin{macrocode}
\begingroup\expandafter\expandafter\expandafter\endgroup
\expandafter\ifx\csname pdf@filemoddate\endcsname\relax
  \def\ETE@Make#1#2{%
    \ifETE@update
      \ETE@WarnModDate
    \fi
    \@firstofone
  }%
  \def\ETE@WarnModDate{%
    \@PackageWarningNoLine{epstopdf}{%
      \string\pdffilemoddate\space is not available,\MessageBreak
      option `update' will be ignored%
    }%
    \global\let\ETE@WarnModDate\relax
  }%
  \def\ETE@FileInfo#1#2{#1 file: <#2>}%
\else
  \def\ETE@Make#1#2{%
    \ifETE@update
      \ifnum\pdf@strcmp{\pdf@filemoddate{#1}}{\pdf@filemoddate{#2}}>0 %
        \expandafter\expandafter\expandafter\@firstofone
      \else
        \@PackageInfoNoLine{epstopdf}{%
          Output file is already uptodate%
        }%
        \expandafter\expandafter\expandafter\@gobble
      \fi
    \else
      \expandafter\@firstofone
    \fi
  }%
  \def\ETE@FileInfo#1#2{%
    #1 file: <#2>%
    \expandafter\expandafter\expandafter
    \ETE@Date\pdf@filemoddate{#2}\@nil
    \expandafter\expandafter\expandafter
    \ETE@Size\pdf@filesize{#2}\@nil
  }%
  \def\ETE@Date#1\@nil{%
    \ifx\\#1\\%
    \else
      \ETE@@Date#1\@nil
    \fi
  }%
  \def\ETE@@Date#1:#2#3#4#5#6#7#8#9{%
    \MessageBreak
    \@spaces\space\space\space date: #2#3#4#5-#6#7-#8#9 %
    \ETE@@Time
  }%
  \def\ETE@@Time#1#2#3#4#5#6#7\@nil{%
    #1#2:#3#4:#5#6%
  }%
  \def\ETE@Size#1\@nil{%
    \ifx\\#1\\%
    \else
      \MessageBreak
      \@spaces\space\space\space size: #1 bytes%
    \fi
  }%
\fi
%    \end{macrocode}
%
% \subsection{Adding conversion support}
%
%    Patch \cs{Gin@setfile} to execute |#3|, if it contains
%    a command.
%    \begin{macrocode}
\expandafter\ifx\csname ETE@OrgGin@setfile\endcsname\relax
  \let\ETE@OrgGin@setfile\Gin@setfile
\else
  \@PackageError{epstopdf}{%
    Command \string\ETE@OrgGin@setfile\space
    already defined.\MessageBreak
  }{%
    Probably some package has included the code of this package%
    \MessageBreak
    instead of using \string\RequirePackage{epstopdf}.%
    \MessageBreak
    \@ehc
  }%
\fi
\def\ETE@IfFileExists{%
  \begingroup\expandafter\expandafter\expandafter\endgroup
  \expandafter\ifx\csname grffile@IfFileExists\endcsname\relax
    \expandafter\IfFileExists
  \else
    \global\let\ETE@IfFileExists\grffile@IfFileExists
    \expandafter\grffile@IfFileExists
  \fi
}
\def\ETE@Skip#1#2\x\fi\fi{%
  \fi
  \fi
  \endgroup
  \fi
  \fi
  #1%
}
\newif\ifETE@InsideSetfile
\newcommand*{\epstopdfcall}[1]{%
  \ifETE@InsideSetfile
    \expandafter\@firstoftwo
  \else
    \expandafter\@secondoftwo
  \fi
  {`#1}%
  {\Gin@base\Gin@ext}%
}
\def\ETE@DefCommandLine#1{%
  \edef\CommandLine{\expandafter\fi\if`#1}%
}
\def\ETE@DefX#1{%
  \expandafter\expandafter\expandafter\def
  \expandafter\expandafter\expandafter\x
  \expandafter\expandafter\expandafter{%
    \expandafter\fi\if`#1\relax\else
  }%
}
\def\ETE@Gin@setfile#1#2#3{%
  \ifETE@disable
    \ETE@OrgGin@setfile{#1}{#2}{#3}%
  \else
    \begingroup
      \ETE@InsideSetfiletrue
      \ETE@DefX{#3}%
    \expandafter\endgroup
    \ifx\x\@empty
      \ETE@OrgGin@setfile{#1}{#2}{#3}%
    \else
      \begingroup
        \ETE@InsideSetfiletrue
        \def\GraphicsType{#1}%
        \def\GraphicsRead{#2}%
        \ifETE@prefersuffix
        \else
          \ifx\ETE@suffix\@empty
          \else
            \ETE@IfFileExists{\Gin@base\GraphicsRead}{%
              \ETE@Skip{%
                \ETE@OrgGin@setfile{#1}{#2}{\Gin@base#2}%
              }%
            }{%
              \let\next\relax
            }%
            \next
          \fi
        \fi
        \ifx\Gin@ext\relax
          \let\SourceExt\Gin@eext
          \def\SourceFile{\Gin@base\Gin@eext}%
        \else
          \let\SourceExt\Gin@ext
          \def\SourceFile{\Gin@base\Gin@ext}%
        \fi
        \edef\SourceExt{% remove dot
          \expandafter\@cdr\SourceExt\@empty\@nil
        }%
        \let\OutputDirectory\ETE@outdir
        \ifx\OutputDirectory\@empty
          \edef\OutputFile{\ETE@GenerateName{\Gin@base}{#2}}%
        \else
          \begingroup
            \filename@parse{\Gin@base#2}%
            \edef\x{\endgroup
              \def\noexpand\OutputFile{%
                \ETE@GenerateName{%
                  \OutputDirectory\filename@base
                }{#2}%
              }%
            }%
          \x
        \fi
        \ETE@DefCommandLine{#3}%
        \ifETE@verbose
          \@PackageInfo{epstopdf}{%
            \ETE@FileInfo{Source}\SourceFile\MessageBreak
            \ETE@FileInfo{Output}\OutputFile\MessageBreak
            Command: <\CommandLine>\MessageBreak
            \string\includegraphics
          }%
        \fi
        \ETE@Make\SourceFile\OutputFile{%
          \pdf@system{\CommandLine}%
          \ifETE@verbose
            \@PackageInfoNoLine{epstopdf}{%
              \ETE@FileInfo{Result}\OutputFile
            }%
          \fi
        }%
        \edef\x{\endgroup
          \ifx\OutputDirectory\@empty
          \else
            \def\noexpand\Gin@base{%
              \OutputDirectory\noexpand\filename@base
            }%
          \fi
          \ifx\ETE@suffix\@empty
          \else
            \edef\noexpand\Gin@base{%
              \noexpand\Gin@base\ETE@suffix
            }%
          \fi
          \noexpand\ETE@OrgGin@setfile{%
            \GraphicsType
          }{%
            \GraphicsRead
          }{%
            \OutputFile
          }%
        }%
      \x
    \fi
  \fi
}
\let\Gin@setfile\ETE@Gin@setfile
\def\ETE@GenerateName#1#2{%
  #1\ETE@suffix#2%
}
%    \end{macrocode}
%
% \subsection{Declare graphics rule}
%
%    \begin{macrocode}
\newcommand*{\epstopdfDeclareGraphicsRule}[4]{%
  \ifx\\#4\\%
    \@PackageError{epstopdf-base}{%
      Conversion command is missing%
    }\@ehc
  \else
    \begingroup
      \@ifundefined{Gin@rule@#1}{%
      }{%
        \@PackageInfo{epstopdf-base}{%
          Redefining graphics rule for `#1'%
        }%
      }%
    \endgroup
    \@namedef{Gin@rule@#1}##1{{#2}{#3}{\epstopdfcall{#4}}}%
  \fi
}
%    \end{macrocode}
%
% \cs{DeclareGraphicsRule} for .eps
%    \begin{macrocode}
\epstopdfDeclareGraphicsRule{.eps}{pdf}{.pdf}{%
  \ETE@epstopdf{#1}%
}
\def\ETE@epstopdf#1{%
  \ETE@program@epstopdf\space
  \ifcase\ifx\OutputDirectory\@empty
           \ifx\ETE@suffix\@empty
             1%
           \fi
         \fi
         0 %
    --outfile=\OutputFile\space
  \fi
  #1%
}
%    \end{macrocode}
%
%    \begin{macrocode}
\ifETE@prepend
  \expandafter\PrependGraphicsExtensions
\else
  \expandafter\AppendGraphicsExtensions
\fi
{.eps}
\let\ETE@prepend\@undefined
\DeclareVoidOption{prepend}{%
  \PrependGraphicsExtensions{.eps}%
}
\let\ETE@append\@undefined
\DeclareVoidOption{append}{%
  \AppendGraphicsExtensions{.eps}%
}
%    \end{macrocode}
%
%    \begin{macrocode}
\InputIfFileExists{epstopdf-sys.cfg}{}{}
\InputIfFileExists{epstopdf.cfg}{}{}
%    \end{macrocode}
%
%    \begin{macrocode}
\ETE@AtEnd%
%</base>
%    \end{macrocode}
%
% \section{Test}
%
% \subsection{Preface for standard catcode check}
%
%    \begin{macrocode}
%<*test1>
\input miniltx.tex\relax
\def\Gin@driver{pdftex.def}
\input graphicx.sty\relax
\resetatcatcode
%</test1>
%    \end{macrocode}
%
% \subsection{Catcode checks for loading}
%
%    \begin{macrocode}
%<*test1>
%    \end{macrocode}
%    \begin{macrocode}
\catcode`\{=1 %
\catcode`\}=2 %
\catcode`\#=6 %
\catcode`\@=11 %
\expandafter\ifx\csname count@\endcsname\relax
  \countdef\count@=255 %
\fi
\expandafter\ifx\csname @gobble\endcsname\relax
  \long\def\@gobble#1{}%
\fi
\expandafter\ifx\csname @firstofone\endcsname\relax
  \long\def\@firstofone#1{#1}%
\fi
\expandafter\ifx\csname loop\endcsname\relax
  \expandafter\@firstofone
\else
  \expandafter\@gobble
\fi
{%
  \def\loop#1\repeat{%
    \def\body{#1}%
    \iterate
  }%
  \def\iterate{%
    \body
      \let\next\iterate
    \else
      \let\next\relax
    \fi
    \next
  }%
  \let\repeat=\fi
}%
\def\RestoreCatcodes{}
\count@=0 %
\loop
  \edef\RestoreCatcodes{%
    \RestoreCatcodes
    \catcode\the\count@=\the\catcode\count@\relax
  }%
\ifnum\count@<255 %
  \advance\count@ 1 %
\repeat

\def\RangeCatcodeInvalid#1#2{%
  \count@=#1\relax
  \loop
    \catcode\count@=15 %
  \ifnum\count@<#2\relax
    \advance\count@ 1 %
  \repeat
}
\def\RangeCatcodeCheck#1#2#3{%
  \count@=#1\relax
  \loop
    \ifnum#3=\catcode\count@
    \else
      \errmessage{%
        Character \the\count@\space
        with wrong catcode \the\catcode\count@\space
        instead of \number#3%
      }%
    \fi
  \ifnum\count@<#2\relax
    \advance\count@ 1 %
  \repeat
}
\def\space{ }
\expandafter\ifx\csname LoadCommand\endcsname\relax
  \def\LoadCommand{\input epstopdf.sty\relax}%
\fi
\def\Test{%
  \RangeCatcodeInvalid{0}{47}%
  \RangeCatcodeInvalid{58}{64}%
  \RangeCatcodeInvalid{91}{96}%
  \RangeCatcodeInvalid{123}{255}%
  \catcode`\@=12 %
  \catcode`\\=0 %
  \catcode`\%=14 %
  \LoadCommand
  \RangeCatcodeCheck{0}{36}{15}%
  \RangeCatcodeCheck{37}{37}{14}%
  \RangeCatcodeCheck{38}{47}{15}%
  \RangeCatcodeCheck{48}{57}{12}%
  \RangeCatcodeCheck{58}{63}{15}%
  \RangeCatcodeCheck{64}{64}{12}%
  \RangeCatcodeCheck{65}{90}{11}%
  \RangeCatcodeCheck{91}{91}{15}%
  \RangeCatcodeCheck{92}{92}{0}%
  \RangeCatcodeCheck{93}{96}{15}%
  \RangeCatcodeCheck{97}{122}{11}%
  \RangeCatcodeCheck{123}{255}{15}%
  \RestoreCatcodes
}
\Test
\csname @@end\endcsname
\end
%    \end{macrocode}
%    \begin{macrocode}
%</test1>
%    \end{macrocode}
%
% \section{Installation}
%
% \subsection{Download}
%
% \paragraph{Package.} This package is available on
% CTAN\footnote{\url{http://ctan.org/pkg/epstopdf}}:
% \begin{description}
% \item[\CTAN{macros/latex/contrib/oberdiek/epstopdf.dtx}] The source file.
% \item[\CTAN{macros/latex/contrib/oberdiek/epstopdf.pdf}] Documentation.
% \end{description}
%
%
% \paragraph{Bundle.} All the packages of the bundle `oberdiek'
% are also available in a TDS compliant ZIP archive. There
% the packages are already unpacked and the documentation files
% are generated. The files and directories obey the TDS standard.
% \begin{description}
% \item[\CTAN{install/macros/latex/contrib/oberdiek.tds.zip}]
% \end{description}
% \emph{TDS} refers to the standard ``A Directory Structure
% for \TeX\ Files'' (\CTAN{tds/tds.pdf}). Directories
% with \xfile{texmf} in their name are usually organized this way.
%
% \subsection{Bundle installation}
%
% \paragraph{Unpacking.} Unpack the \xfile{oberdiek.tds.zip} in the
% TDS tree (also known as \xfile{texmf} tree) of your choice.
% Example (linux):
% \begin{quote}
%   |unzip oberdiek.tds.zip -d ~/texmf|
% \end{quote}
%
% \paragraph{Script installation.}
% Check the directory \xfile{TDS:scripts/oberdiek/} for
% scripts that need further installation steps.
% Package \xpackage{attachfile2} comes with the Perl script
% \xfile{pdfatfi.pl} that should be installed in such a way
% that it can be called as \texttt{pdfatfi}.
% Example (linux):
% \begin{quote}
%   |chmod +x scripts/oberdiek/pdfatfi.pl|\\
%   |cp scripts/oberdiek/pdfatfi.pl /usr/local/bin/|
% \end{quote}
%
% \subsection{Package installation}
%
% \paragraph{Unpacking.} The \xfile{.dtx} file is a self-extracting
% \docstrip\ archive. The files are extracted by running the
% \xfile{.dtx} through \plainTeX:
% \begin{quote}
%   \verb|tex epstopdf.dtx|
% \end{quote}
%
% \paragraph{TDS.} Now the different files must be moved into
% the different directories in your installation TDS tree
% (also known as \xfile{texmf} tree):
% \begin{quote}
% \def\t{^^A
% \begin{tabular}{@{}>{\ttfamily}l@{ $\rightarrow$ }>{\ttfamily}l@{}}
%   epstopdf.sty & tex/latex/oberdiek/epstopdf.sty\\
%   epstopdf-base.sty & tex/latex/oberdiek/epstopdf-base.sty\\
%   epstopdf.pdf & doc/latex/oberdiek/epstopdf.pdf\\
%   test/epstopdf-test1.tex & doc/latex/oberdiek/test/epstopdf-test1.tex\\
%   epstopdf.dtx & source/latex/oberdiek/epstopdf.dtx\\
% \end{tabular}^^A
% }^^A
% \sbox0{\t}^^A
% \ifdim\wd0>\linewidth
%   \begingroup
%     \advance\linewidth by\leftmargin
%     \advance\linewidth by\rightmargin
%   \edef\x{\endgroup
%     \def\noexpand\lw{\the\linewidth}^^A
%   }\x
%   \def\lwbox{^^A
%     \leavevmode
%     \hbox to \linewidth{^^A
%       \kern-\leftmargin\relax
%       \hss
%       \usebox0
%       \hss
%       \kern-\rightmargin\relax
%     }^^A
%   }^^A
%   \ifdim\wd0>\lw
%     \sbox0{\small\t}^^A
%     \ifdim\wd0>\linewidth
%       \ifdim\wd0>\lw
%         \sbox0{\footnotesize\t}^^A
%         \ifdim\wd0>\linewidth
%           \ifdim\wd0>\lw
%             \sbox0{\scriptsize\t}^^A
%             \ifdim\wd0>\linewidth
%               \ifdim\wd0>\lw
%                 \sbox0{\tiny\t}^^A
%                 \ifdim\wd0>\linewidth
%                   \lwbox
%                 \else
%                   \usebox0
%                 \fi
%               \else
%                 \lwbox
%               \fi
%             \else
%               \usebox0
%             \fi
%           \else
%             \lwbox
%           \fi
%         \else
%           \usebox0
%         \fi
%       \else
%         \lwbox
%       \fi
%     \else
%       \usebox0
%     \fi
%   \else
%     \lwbox
%   \fi
% \else
%   \usebox0
% \fi
% \end{quote}
% If you have a \xfile{docstrip.cfg} that configures and enables \docstrip's
% TDS installing feature, then some files can already be in the right
% place, see the documentation of \docstrip.
%
% \subsection{Refresh file name databases}
%
% If your \TeX~distribution
% (\teTeX, \mikTeX, \dots) relies on file name databases, you must refresh
% these. For example, \teTeX\ users run \verb|texhash| or
% \verb|mktexlsr|.
%
% \subsection{Some details for the interested}
%
% \paragraph{Attached source.}
%
% The PDF documentation on CTAN also includes the
% \xfile{.dtx} source file. It can be extracted by
% AcrobatReader 6 or higher. Another option is \textsf{pdftk},
% e.g. unpack the file into the current directory:
% \begin{quote}
%   \verb|pdftk epstopdf.pdf unpack_files output .|
% \end{quote}
%
% \paragraph{Unpacking with \LaTeX.}
% The \xfile{.dtx} chooses its action depending on the format:
% \begin{description}
% \item[\plainTeX:] Run \docstrip\ and extract the files.
% \item[\LaTeX:] Generate the documentation.
% \end{description}
% If you insist on using \LaTeX\ for \docstrip\ (really,
% \docstrip\ does not need \LaTeX), then inform the autodetect routine
% about your intention:
% \begin{quote}
%   \verb|latex \let\install=y\input{epstopdf.dtx}|
% \end{quote}
% Do not forget to quote the argument according to the demands
% of your shell.
%
% \paragraph{Generating the documentation.}
% You can use both the \xfile{.dtx} or the \xfile{.drv} to generate
% the documentation. The process can be configured by the
% configuration file \xfile{ltxdoc.cfg}. For instance, put this
% line into this file, if you want to have A4 as paper format:
% \begin{quote}
%   \verb|\PassOptionsToClass{a4paper}{article}|
% \end{quote}
% An example follows how to generate the
% documentation with pdf\LaTeX:
% \begin{quote}
%\begin{verbatim}
%pdflatex epstopdf.dtx
%makeindex -s gind.ist epstopdf.idx
%pdflatex epstopdf.dtx
%makeindex -s gind.ist epstopdf.idx
%pdflatex epstopdf.dtx
%\end{verbatim}
% \end{quote}
%
% \section{Catalogue}
%
% The following XML file can be used as source for the
% \href{http://mirror.ctan.org/help/Catalogue/catalogue.html}{\TeX\ Catalogue}.
% The elements \texttt{caption} and \texttt{description} are imported
% from the original XML file from the Catalogue.
% The name of the XML file in the Catalogue is \xfile{epstopdf-pkg.xml}.
%    \begin{macrocode}
%<*catalogue>
<?xml version='1.0' encoding='us-ascii'?>
<!DOCTYPE entry SYSTEM 'catalogue.dtd'>
<entry datestamp='$Date$' modifier='$Author$' id='epstopdf-pkg'>
  <name>epstopdf-pkg</name>
  <caption>Call epstopdf "on the fly"</caption>
  <authorref id='auth:oberdiek'/>
  <copyright owner='Heiko Oberdiek' year='2001,2006-2010'/>
  <license type='lppl1.3'/>
  <version number='2.6'/>
  <description>
    The package adds support for EPS files in the
    <xref refid='graphicx'>graphicx</xref> package when running under
    PDFTeX.  If an EPS graphic is detected, the package spawns a
    process to convert the EPS to PDF, using the script
    <xref refid='epstopdf'>epstopdf</xref>.  This of course requires
    that shell escape is enabled for the PDFTeX run.
    <p/>
    The package is part of the <xref refid='oberdiek'>oberdiek</xref>
    bundle.
  </description>
  <documentation details='Package documentation'
      href='ctan:/macros/latex/contrib/oberdiek/epstopdf.pdf'/>
  <ctan file='true' path='/macros/latex/contrib/oberdiek/epstopdf.dtx'/>
  <miktex location='oberdiek'/>
  <texlive location='oberdiek'/>
  <install path='/macros/latex/contrib/oberdiek/oberdiek.tds.zip'/>
</entry>
%</catalogue>
%    \end{macrocode}
%
% \begin{History}
%   \begin{Version}{2001/01/06 v1.0}
%   \item
%     First public version,
%     published in the \pdfTeX\ mailing list.
%   \end{Version}
%   \begin{Version}{2001/02/04 v1.1}
%   \item
%     Minor documentation update.
%   \item
%     CTAN.
%   \end{Version}
%   \begin{Version}{2006/02/20 v1.2}
%   \item
%     DTX framework.
%   \item
%     Compatibility for \xfile{miniltx.tex}.
%   \end{Version}
%   \begin{Version}{2006/08/26 v1.3}
%   \item
%     Check for \cs{write18} if available and print a warning
%     if the feature is not enabled.
%   \end{Version}
%   \begin{Version}{2007/04/26 v1.4}
%   \item
%     Documentation rewritten and extended.
%   \end{Version}
%   \begin{Version}{2007/10/02 v1.5}
%   \item
%     New option \xoption{update}: If the converted file exists, it will be only
%     converted if it is out of date.
%   \item
%     Updating the extension list is delegated to package \xpackage{grfext}.
%     Fine tuning is done by the new options \xoption{append}, \xoption{prepend}.
%   \item
%     New option \xoption{outdir} for changing the output directory.
%   \item
%     New option \xoption{verbose}.
%   \item
%     \cs{SourceFile} and \cs{OutputFile} introduced.
%   \item
%     Configuration file support added.
%   \end{Version}
%   \begin{Version}{2007/11/11 v1.6}
%   \item
%     Use of package \xpackage{pdftexcmds} for \LuaTeX\ support.
%   \end{Version}
%   \begin{Version}{2008/05/06 v1.7}
%   \item
%     Warning messages uses ``loaded'' instead of ``found''.
%   \end{Version}
%   \begin{Version}{2009/03/01 v1.8}
%   \item
%     Warning message for missing \xfile{pdftex.def} changed.
%   \end{Version}
%   \begin{Version}{2009/07/06 v1.9}
%   \item
%     Option \xoption{suffix} added.
%   \end{Version}
%   \begin{Version}{2009/07/07 v1.10}
%   \item
%     \cs{SourceExt} added.
%   \item
%     If option \xoption{suffix} is set, the inclusion of
%     an image without the suffix namespace is preferred over generating the
%     the image within the suffix namespace.
%   \end{Version}
%   \begin{Version}{2009/07/12 v2.0}
%   \item
%     New default settings.
%   \item
%     Package is split into \xpackage{epstopdf} that only takes package
%     options and loads \xpackage{epstopdf-base} that does the work.
%   \item
%     \cs{epstopdfDeclareGraphicsRule} and \cs{epstopdfcall} added.
%   \item
%     \xfile{epstopdf-sys.cfg} is loaded before \xfile{epstopdf.cfg}
%     if \xfile{epstopdf-sys.cfg} exists.
%   \end{Version}
%   \begin{Version}{2009/07/15 v2.1}
%   \item
%     Default setting: \xoption{verbose} is now turned on as the
%     documentation for v2.0 said.
%   \item
%     Documentation fixes.
%   \end{Version}
%   \begin{Version}{2009/07/16 v2.2}
%   \item
%     Fixed redefined \cs{Gin@setfile}.
%   \item
%     Documentation extended for package options.
%   \end{Version}
%   \begin{Version}{2009/09/24 v2.3}
%   \item
%     Bug fix for the case that both option \xoption{suffix} and
%     \xoption{outdir} are used.
%   \end{Version}
%   \begin{Version}{2009/10/17 v2.4}
%   \item
%     The name of the program `epstopdf' can be configured via
%     the new option \xoption{program@epstopdf}.
%   \end{Version}
%   \begin{Version}{2016/05/15 v2.5}
%   \item
%     Wording of warning message fixed (Karl Berry).
%   \item
%     \cs{ETE@Gin@setfile} added (Karl Berry).
%   \end{Version}
%   \begin{Version}{2016/05/15 v2.6}
%   \item
%     luaTeX compatibility
%   \end{Version}
% \end{History}
%
% \PrintIndex
%
% \Finale
\endinput
|
% \end{quote}
% Do not forget to quote the argument according to the demands
% of your shell.
%
% \paragraph{Generating the documentation.}
% You can use both the \xfile{.dtx} or the \xfile{.drv} to generate
% the documentation. The process can be configured by the
% configuration file \xfile{ltxdoc.cfg}. For instance, put this
% line into this file, if you want to have A4 as paper format:
% \begin{quote}
%   \verb|\PassOptionsToClass{a4paper}{article}|
% \end{quote}
% An example follows how to generate the
% documentation with pdf\LaTeX:
% \begin{quote}
%\begin{verbatim}
%pdflatex epstopdf.dtx
%makeindex -s gind.ist epstopdf.idx
%pdflatex epstopdf.dtx
%makeindex -s gind.ist epstopdf.idx
%pdflatex epstopdf.dtx
%\end{verbatim}
% \end{quote}
%
% \section{Catalogue}
%
% The following XML file can be used as source for the
% \href{http://mirror.ctan.org/help/Catalogue/catalogue.html}{\TeX\ Catalogue}.
% The elements \texttt{caption} and \texttt{description} are imported
% from the original XML file from the Catalogue.
% The name of the XML file in the Catalogue is \xfile{epstopdf-pkg.xml}.
%    \begin{macrocode}
%<*catalogue>
<?xml version='1.0' encoding='us-ascii'?>
<!DOCTYPE entry SYSTEM 'catalogue.dtd'>
<entry datestamp='$Date$' modifier='$Author$' id='epstopdf-pkg'>
  <name>epstopdf-pkg</name>
  <caption>Call epstopdf "on the fly"</caption>
  <authorref id='auth:oberdiek'/>
  <copyright owner='Heiko Oberdiek' year='2001,2006-2010'/>
  <license type='lppl1.3'/>
  <version number='2.6'/>
  <description>
    The package adds support for EPS files in the
    <xref refid='graphicx'>graphicx</xref> package when running under
    PDFTeX.  If an EPS graphic is detected, the package spawns a
    process to convert the EPS to PDF, using the script
    <xref refid='epstopdf'>epstopdf</xref>.  This of course requires
    that shell escape is enabled for the PDFTeX run.
    <p/>
    The package is part of the <xref refid='oberdiek'>oberdiek</xref>
    bundle.
  </description>
  <documentation details='Package documentation'
      href='ctan:/macros/latex/contrib/oberdiek/epstopdf.pdf'/>
  <ctan file='true' path='/macros/latex/contrib/oberdiek/epstopdf.dtx'/>
  <miktex location='oberdiek'/>
  <texlive location='oberdiek'/>
  <install path='/macros/latex/contrib/oberdiek/oberdiek.tds.zip'/>
</entry>
%</catalogue>
%    \end{macrocode}
%
% \begin{History}
%   \begin{Version}{2001/01/06 v1.0}
%   \item
%     First public version,
%     published in the \pdfTeX\ mailing list.
%   \end{Version}
%   \begin{Version}{2001/02/04 v1.1}
%   \item
%     Minor documentation update.
%   \item
%     CTAN.
%   \end{Version}
%   \begin{Version}{2006/02/20 v1.2}
%   \item
%     DTX framework.
%   \item
%     Compatibility for \xfile{miniltx.tex}.
%   \end{Version}
%   \begin{Version}{2006/08/26 v1.3}
%   \item
%     Check for \cs{write18} if available and print a warning
%     if the feature is not enabled.
%   \end{Version}
%   \begin{Version}{2007/04/26 v1.4}
%   \item
%     Documentation rewritten and extended.
%   \end{Version}
%   \begin{Version}{2007/10/02 v1.5}
%   \item
%     New option \xoption{update}: If the converted file exists, it will be only
%     converted if it is out of date.
%   \item
%     Updating the extension list is delegated to package \xpackage{grfext}.
%     Fine tuning is done by the new options \xoption{append}, \xoption{prepend}.
%   \item
%     New option \xoption{outdir} for changing the output directory.
%   \item
%     New option \xoption{verbose}.
%   \item
%     \cs{SourceFile} and \cs{OutputFile} introduced.
%   \item
%     Configuration file support added.
%   \end{Version}
%   \begin{Version}{2007/11/11 v1.6}
%   \item
%     Use of package \xpackage{pdftexcmds} for \LuaTeX\ support.
%   \end{Version}
%   \begin{Version}{2008/05/06 v1.7}
%   \item
%     Warning messages uses ``loaded'' instead of ``found''.
%   \end{Version}
%   \begin{Version}{2009/03/01 v1.8}
%   \item
%     Warning message for missing \xfile{pdftex.def} changed.
%   \end{Version}
%   \begin{Version}{2009/07/06 v1.9}
%   \item
%     Option \xoption{suffix} added.
%   \end{Version}
%   \begin{Version}{2009/07/07 v1.10}
%   \item
%     \cs{SourceExt} added.
%   \item
%     If option \xoption{suffix} is set, the inclusion of
%     an image without the suffix namespace is preferred over generating the
%     the image within the suffix namespace.
%   \end{Version}
%   \begin{Version}{2009/07/12 v2.0}
%   \item
%     New default settings.
%   \item
%     Package is split into \xpackage{epstopdf} that only takes package
%     options and loads \xpackage{epstopdf-base} that does the work.
%   \item
%     \cs{epstopdfDeclareGraphicsRule} and \cs{epstopdfcall} added.
%   \item
%     \xfile{epstopdf-sys.cfg} is loaded before \xfile{epstopdf.cfg}
%     if \xfile{epstopdf-sys.cfg} exists.
%   \end{Version}
%   \begin{Version}{2009/07/15 v2.1}
%   \item
%     Default setting: \xoption{verbose} is now turned on as the
%     documentation for v2.0 said.
%   \item
%     Documentation fixes.
%   \end{Version}
%   \begin{Version}{2009/07/16 v2.2}
%   \item
%     Fixed redefined \cs{Gin@setfile}.
%   \item
%     Documentation extended for package options.
%   \end{Version}
%   \begin{Version}{2009/09/24 v2.3}
%   \item
%     Bug fix for the case that both option \xoption{suffix} and
%     \xoption{outdir} are used.
%   \end{Version}
%   \begin{Version}{2009/10/17 v2.4}
%   \item
%     The name of the program `epstopdf' can be configured via
%     the new option \xoption{program@epstopdf}.
%   \end{Version}
%   \begin{Version}{2016/05/15 v2.5}
%   \item
%     Wording of warning message fixed (Karl Berry).
%   \item
%     \cs{ETE@Gin@setfile} added (Karl Berry).
%   \end{Version}
%   \begin{Version}{2016/05/15 v2.6}
%   \item
%     luaTeX compatibility
%   \end{Version}
% \end{History}
%
% \PrintIndex
%
% \Finale
\endinput

%        (quote the arguments according to the demands of your shell)
%
% Documentation:
%    (a) If epstopdf.drv is present:
%           latex epstopdf.drv
%    (b) Without epstopdf.drv:
%           latex epstopdf.dtx; ...
%    The class ltxdoc loads the configuration file ltxdoc.cfg
%    if available. Here you can specify further options, e.g.
%    use A4 as paper format:
%       \PassOptionsToClass{a4paper}{article}
%
%    Programm calls to get the documentation (example):
%       pdflatex epstopdf.dtx
%       makeindex -s gind.ist epstopdf.idx
%       pdflatex epstopdf.dtx
%       makeindex -s gind.ist epstopdf.idx
%       pdflatex epstopdf.dtx
%
% Installation:
%    TDS:tex/latex/oberdiek/epstopdf.sty
%    TDS:tex/latex/oberdiek/epstopdf-base.sty
%    TDS:doc/latex/oberdiek/epstopdf.pdf
%    TDS:doc/latex/oberdiek/test/epstopdf-test1.tex
%    TDS:source/latex/oberdiek/epstopdf.dtx
%
%<*ignore>
\begingroup
  \catcode123=1 %
  \catcode125=2 %
  \def\x{LaTeX2e}%
\expandafter\endgroup
\ifcase 0\ifx\install y1\fi\expandafter
         \ifx\csname processbatchFile\endcsname\relax\else1\fi
         \ifx\fmtname\x\else 1\fi\relax
\else\csname fi\endcsname
%</ignore>
%<*install>
\input docstrip.tex
\Msg{************************************************************************}
\Msg{* Installation}
\Msg{* Package: epstopdf 2016/05/15 v2.6 Conversion with epstopdf on the fly (HO)}
\Msg{************************************************************************}

\keepsilent
\askforoverwritefalse

\let\MetaPrefix\relax
\preamble

This is a generated file.

Project: epstopdf
Version: 2016/05/15 v2.6

Copyright (C) 2001, 2006-2010 by
   Heiko Oberdiek <heiko.oberdiek at googlemail.com>

This work may be distributed and/or modified under the
conditions of the LaTeX Project Public License, either
version 1.3c of this license or (at your option) any later
version. This version of this license is in
   http://www.latex-project.org/lppl/lppl-1-3c.txt
and the latest version of this license is in
   http://www.latex-project.org/lppl.txt
and version 1.3 or later is part of all distributions of
LaTeX version 2005/12/01 or later.

This work has the LPPL maintenance status "maintained".

This Current Maintainer of this work is Heiko Oberdiek.

This work consists of the main source file epstopdf.dtx
and the derived files
   epstopdf.sty, epstopdf.pdf, epstopdf.ins, epstopdf.drv,
   epstopdf-base.sty, epstopdf-test1.tex.

\endpreamble
\let\MetaPrefix\DoubleperCent

\generate{%
  \file{epstopdf.ins}{\from{epstopdf.dtx}{install}}%
  \file{epstopdf.drv}{\from{epstopdf.dtx}{driver}}%
  \usedir{tex/latex/oberdiek}%
  \file{epstopdf.sty}{\from{epstopdf.dtx}{package}}%
  \file{epstopdf-base.sty}{\from{epstopdf.dtx}{base}}%
  \usedir{doc/latex/oberdiek/test}%
  \file{epstopdf-test1.tex}{\from{epstopdf.dtx}{test1}}%
  \nopreamble
  \nopostamble
  \usedir{source/latex/oberdiek/catalogue}%
  \file{epstopdf-pkg.xml}{\from{epstopdf.dtx}{catalogue}}%
}

\catcode32=13\relax% active space
\let =\space%
\Msg{************************************************************************}
\Msg{*}
\Msg{* To finish the installation you have to move the following}
\Msg{* files into a directory searched by TeX:}
\Msg{*}
\Msg{*     epstopdf.sty, epstopdf-base.sty}
\Msg{*}
\Msg{* To produce the documentation run the file `epstopdf.drv'}
\Msg{* through LaTeX.}
\Msg{*}
\Msg{* Happy TeXing!}
\Msg{*}
\Msg{************************************************************************}

\endbatchfile
%</install>
%<*ignore>
\fi
%</ignore>
%<*driver>
\NeedsTeXFormat{LaTeX2e}
\ProvidesFile{epstopdf.drv}%
  [2016/05/15 v2.6 Conversion with epstopdf on the fly (HO)]%
\documentclass{ltxdoc}
\usepackage{holtxdoc}[2011/11/22]
\begin{document}
  \DocInput{epstopdf.dtx}%
\end{document}
%</driver>
% \fi
%
%
% \CharacterTable
%  {Upper-case    \A\B\C\D\E\F\G\H\I\J\K\L\M\N\O\P\Q\R\S\T\U\V\W\X\Y\Z
%   Lower-case    \a\b\c\d\e\f\g\h\i\j\k\l\m\n\o\p\q\r\s\t\u\v\w\x\y\z
%   Digits        \0\1\2\3\4\5\6\7\8\9
%   Exclamation   \!     Double quote  \"     Hash (number) \#
%   Dollar        \$     Percent       \%     Ampersand     \&
%   Acute accent  \'     Left paren    \(     Right paren   \)
%   Asterisk      \*     Plus          \+     Comma         \,
%   Minus         \-     Point         \.     Solidus       \/
%   Colon         \:     Semicolon     \;     Less than     \<
%   Equals        \=     Greater than  \>     Question mark \?
%   Commercial at \@     Left bracket  \[     Backslash     \\
%   Right bracket \]     Circumflex    \^     Underscore    \_
%   Grave accent  \`     Left brace    \{     Vertical bar  \|
%   Right brace   \}     Tilde         \~}
%
% \GetFileInfo{epstopdf.drv}
%
% \title{The \xpackage{epstopdf} package}
% \date{2016/05/15 v2.6}
% \author{Heiko Oberdiek\thanks
% {Please report any issues at https://github.com/ho-tex/oberdiek/issues}\\
% \xemail{heiko.oberdiek at googlemail.com}}
%
% \maketitle
%
% \begin{abstract}
% This packages adds support of handling eps images
% to package \xpackage{graphics} or \xpackage{graphicx}
% with option \xoption{pdftex}. If an eps image is detected,
% epstopdf is automatically called to convert it to pdf format.
% \end{abstract}
%
% \tableofcontents
%
% \providecommand*{\xprogram}[1]{\textsf{#1}}
%
% \section{Documentation}
%
% \subsection{Introduction}
%
% \LaTeX\ provides its graphics bundle to include graphics files.
% Both packages \xpackage{graphics} or \xpackage{graphicx} may
% be used. the latter one loads the first and adds options
% in key value style for \cs{includegraphics}.
%
% Usually the drivers do not support all kind of graphics files.
% Other image types must be converted, before they become
% usuable. In case of driver \xprogram{dvips}, the graphics
% rule may contain a conversion rule. Then all that package
% \xpackage{graphics} must know is the bounding box, the command
% is passed to \xprogram{dvips} that calls it and embeds the
% converted image.
%
% However, \pdfTeX\ has its driver for PDF output already
% build in. It's graphics inclusion commands (\cs{pdfximage})
% does not allow the execution of external commands.
% Therefore commands in the last argument of
% \cs{DeclareGraphicsRule} were of no use.
% But external programs can be called within \pdfTeX.
% This feature is called ``shell escape'' or ``write 18''
% and must usually enabled explicitely because of security reasons.
% Now, this package \xpackage{epstopdf} hooks
% into package \xpackage{graphics}' code to catch
% that argument with the external command and executes it
% to convert the graphics file to a supported format and
% passes the control of graphics inclusion back to package
% \xpackage{graphics}.
%
% \subsection{Requirements}
%
% \begin{itemize}
% \item
%   The feature \cs{write18} must be enabled. This allows the
%   running of external programs during \TeX's compile run.
%   Keep in mind that this is a security risk. The feature is an
%   addition to \cs{TeX}. Mik\TeX, te\TeX, \TeX\ Live support it.
%   In Web2C based \TeX\ distributions (te\TeX, \TeX\ Live) it can be
%   enabled in the configuration file \xfile{texmf.cnf}:
%   \begin{quote}
%     |shell_escape = 1|
%   \end{quote}
%   Because of the security risk, it is better to do it on the
%   command line only:
%   \begin{quote}
%   \texttt{--shell-escape} (te\TeX, \TeX\ Live)\\
%   \texttt{--enable-write18} (MiK\TeX)
%   \end{quote}
%   Example:
%   \begin{quote}
%     |pdflatex -shell-escape test.tex|
%   \end{quote}
% \item
%   The program \xprogram{epstopdf} for the conversion from EPS
%   to PDF. However, other programs can
%   be used and configured by \cs{DeclareGraphicsRule}.
%   Example:
%   \begin{quote}
%   |\epstopdfDeclareGraphicsRule{.eps}{pdf}{.pdf}{%|\\
%   |  ps2pdf -dEPSCrop #1 \OutputFile|\\
%   |}|
%   \end{quote}
% \end{itemize}
%
% \subsection{Usage}
%
% The package is loaded after graphic\{s,x\}, e.g.:
% \begin{quote}
%   |\usepackage[pdftex]{graphicx}|\\
%   |\usepackage{epstopdf}|
% \end{quote}
% Now images with file name extension \xfile{.eps} are detected
% and supported using \cs{includegraphics}.
%
% If the graphics file name is explicitly specified with extension
% \xfile{.eps} the new rule for EPS files is called and
% the conversion performed. If option \xoption{update} is in force
% then the conversion step is dropped if the target file already
% exists and is not older then the EPS file.
%
% The situation is more complicate if the graphics file is
% given without file name extension. Then the \xpackage{graphics}
% package must search for a supported image file.
% The possible extensions are stored in the graphics extension
% list, that can be set by \cs{DeclareGraphicsExtensions}.
% The algorithm:
% \begin{quote}
%    \def\>{\mbox{}\quad}
%    function search( \meta{filebase} )\\
%    \>foreach \meta{ext} in \meta{graphics extensions}\\
%    \>\>foreach \meta{dir} in \meta{current directory}, \meta{\cs{graphicspath}}\\
%    \>\>\>\meta{file} := \meta{dir} + \meta{filebase} + \meta{ext}\\
%    \>\>\>if exist \meta{file}\\
%    \>\>\>\>return found\\
%    \>return not found
% \end{quote}
% Package \xpackage{epstopdf} puts \xfile{.eps} at the end of the
% graphics extension search list. This is the behaviour of option
% \xoption{append} that is enabled by default.
% That means, the conversion is called last
% unless a supported file type cannot be found earlier. This avoids
% unnecessary conversion steps that slow down the \LaTeX\ run.
% If you want to use option \xoption{update} and your \pdfTeX\ supports
% it, then an outdated PDF file also would be found earlier unless
% \xoption{suffix} is used that is the default since version 2.0.
%
% With an empty option \xoption{suffix} and option
% \xoption{prepend} there is a risk that an original PDF file is overwritten:
%
% If the original image file is the PDF file and there is also a generated
% EPS file, then the original PDF file can be regenerated (depending
% on the option settings) and the original PDF file gets lost.
% Therefore option \xoption{suffix} is introduced in version 1.9
% to create a separate name space for generated output files.
%
% \textbf{Note:} Usually the conversion program needs the exact location
% of the image file. Usually the current directory works.
% Also if the image file is found using \cs{graphicspath}, the
% location is known. However, if the image is somewhere
% in a directory of environment variable |TEXINPUTS|, then
% the package does not know the exact location and the
% conversion program will not find the image file unless it
% implements a search using |TEXINPUTS| (program \xprogram{kpsewhich}
% may be of help in this task).
%
% \subsection{Options}
%
% Options can be given as package options or later using:
% \begin{declcs}{epstopdfsetup} \M{key value list}
% \end{declcs}
% \LaTeX\ expands the option list before passing the option list
% to the package's option handling code. This can fail
% for option \xoption{suffix} if it contains some of the macros
% described below. Use \cs{epstopdfsetup} after the package is loaded.
% Or load package \xpackage{kvoptions-patch} before. This package
% is also loaded by option \xoption{patch} of package \xpackage{kvoptions}.
% \LaTeX's option code is redefined to respect key value options and
% let the values untouched.
%
% \begin{description}
% \def\It#1{\item[\xoption{#1}:]}^^A
% \It{update}
%   The conversion program is only called, if the target file
%   does not exist or is older than the source image file.
% \It{append}
%   Puts the extension \xfile{.eps} at the end of the graphics extension list
%   (default).
% \It{prepend}
%   Puts the extension \xfile{.eps} at the begin of the graphics extension list.
% \It{outdir}
%   The converted file may put in an other output directory.
%   The value of \xoption{outdir} must include the directory separator.
%   Example for the current directory:
%   \begin{quote}
%   |\epstopdfsetup{outdir=./}|
%   \end{quote}
%   For other directories ensure, that they can be found.
%   See \cs{graphicspath} or |TEXINPUTS|.
% \It{suffix}
%   This option takes a string that is put between the file name base
%   and the extension of the output file. Rationale: It can happen,
%   that a PDF file is the original file and the EPS file the
%   generated file. If now the package thinks, that the PDF file
%   is the generated file, it will `regnerate' it. But in reality
%   the original file is lost. Therefore I recommend to use this
%   option always to generate a separate name space for generated
%   files. Proposed value is \texttt{-generated} or \texttt{.generated}.
%   The suffix \texttt{.generated} will also work here without the need for
%   package \xpackage{grffile}).
%
%   Example:
%   \begin{quote}
%   |\epstopdfsetup{suffix=-generated}|\\
%   Then \xfile{foo.eps} is converted to \xfile{foo-generated.pdf}.
%   \end{quote}
%   \cs{SourceExt} can be used inside the suffix string.
%   It's will be replaced by the extension of the image source
%   file without the leading dot, for instance:
%   \begin{quote}
%   |\epstopdfsetup{suffix=-\SourceExt-converted-to}|\\
%   \xfile{foo.eps} $\Rightarrow$ \xfile{foo-eps-converted-to.pdf}
%   \end{quote}
%   See also the next option \xoption{prefersuffix} that
%   modifies the behaviour of option \xoption{suffix} in some
%   cases.
%
%   Default for \xoption{suffix} is `|-\SourceExt-converted-to|'.
%
% \It{prefersuffix}
%   If a suffix is set by option \xoption{suffix}, then there can be
%   two image file names that could be taken into account for inclusion:
%   A image file name with the suffix string inside
%   its name and a image file name without; e.g. for \xfile{foo.eps}
%   the names could be:
%   \begin{quote}
%   \xfile{foo-suffix.pdf}, \xfile{foo.pdf}
%   \end{quote}
%   If option \xoption{perfersuffix} is turned on, the
%   file \xfile{foo-suffix.pdf} and its generation is preferred
%   over using \xfile{foo.pdf}. Otherwise \xfile{foo.pdf} is included
%   without generating \xfile{foo-suffix.pdf}.
%   The default of option \xoption{prefersuffix} is \texttt{true}.
% \It{program@epstopdf}
%   The name for the conversion program from EPS to PDF,
%   default is `|epstopdf|'.
% \It{verbose}
%   It prints some information about the image in the \xfile{.log} file
%   (default).
% \end{description}
%
% \subsection{Configuration}
%
% \subsubsection{System configuration file \xfile{epstopdf-sys.cfg}}
%
% If \xfile{epstopdf-sys} exists it is loaded at the end of the
% package \xpackage{epstopdf-base} and before the user configuration
% file. It's intended for TeX distributors. Thus they could add
% additional conversion rules (e.g., .gif -> .png) or set options.
%
% \subsubsection{User configuration file \xfile{epstopdf.cfg}}
%
% A configuration file \xfile{epstopdf.cfg} is loaded at the end
% of the package if it exists. It can be used for changing
% the default option setting. Example:
% \begin{quote}
% |\epstopdfsetup{verbose=false}|
% \end{quote}
%
% \subsubsection{Conversion program}
%
% You can use \cs{DeclareGraphicsRule} in a similar way as the
% route via \xprogram{dvips} to specify the conversion
% command line.
% The conversion argument starts with a back tick, followed
% by the conversion command including parameters.
%
% The whole conversion argument should also be wrapped inside
% \cs{epstopdfcall}. This reduces the problem with packages
% (e.g. \xpackage{pst-pdf}) that use the conversion argument
% and expands it. Macros \cs{SourceFile}, \cs{OutputFile}, and
% \cs{SourceExt} are not defined outside \xpackage{epstopdf-base}'s
% \cs{Gin@setfile} and error messages because of undefined
% command names are the result. If \cs{epstopdfcall} detects that
% it is called outside \xpackage{epstopdf-base}'s \cs{Gin@setfile}
% then it replaces the conversion argument by
% package \xpackage{graphics}'s
% default, usually the image file.
%
% The following macros are available inside:
% \begin{description}
% \def\It#1{\item[\cs{#1}:]}
% \It{OutputFile}: output file name (with known path and extension)
% \It{SourceFile}: source file name (with known path and extension),
%                  usually the same as |#1|,
% \It{SourceExt}: source extension without leading dot.
% \end{description}
%
% \paragraph{Conversion from EPS to PDF.}
% Other programs than \xprogram{epstopdf} can be used to convert from
% EPS to PDF. Example that uses \xprogram{Ghostscript}:
% \begin{quote}
%   |\DeclareGraphicsRule{.eps}{pdf}{.pdf}{%|\\
%   |  \epstopdfcall{`ps2pdf -dEPSCrop #1 \noexpand\OutputFile}%|\\
%   |}|
% \end{quote}
% \cs{DeclareGraphicsRule} expands the argument, therefore \cs{noexpand}
% is necessary. As convenience package \xpackage{epstopdf-base} defines
% \cs{epstopdfDeclareGraphicsRule}. Then the conversion argument
% is not expanded, \cs{epstopdfcall} and the back tick are added:
% \begin{quote}
%   |\epstopdfDeclareGraphicsRule{.eps}{pdf}{.pdf}{%|\\
%   |  ps2pdf -dEPSCrop #1 \OutputFile|\\
%   |}|
% \end{quote}
%
% Also \cs{OutputFile} respects the setting of option \xoption{outdir}.
%
% \subsection{Other image formats}
%
% The support that package \xpackage{epstopdf} implements is not
% limited to EPS files. Other image conversions can be declared.
% The following example shows it for GIF images under Unix with
% ImageMagick's \xprogram{convert}:
%\begin{quote}
%\begin{verbatim}
%\epstopdfDeclareGraphicsRule{.gif}{png}{.png}{%
%  convert #1 \OutputFile
%}
%\end{verbatim}
%\end{quote}
% The file extension \xfile{.gif} can be added to the extension list that
% package \xpackage{graphics} searches if the file extension
% is not given in \cs{includegraphics}.
% The list can be set by \cs{GraphicsExtensions}.
% \begin{quote}
% |\AppendGraphicsExtensions{.gif}|\\
% or\\
% |\PrependGraphicsExtensions{.gif}|
% \end{quote}
%
% \StopEventually{
% }
%
% \section{Implementation}
%
%    \begin{macrocode}
%<*package>
%    \end{macrocode}
%
% \subsection{Wrapper package}
%
%    Reload check, especially if the package is not used with \LaTeX.
%    \begin{macrocode}
\begingroup\catcode61\catcode48\catcode32=10\relax%
  \catcode13=5 % ^^M
  \endlinechar=13 %
  \catcode35=6 % #
  \catcode39=12 % '
  \catcode44=12 % ,
  \catcode45=12 % -
  \catcode46=12 % .
  \catcode58=12 % :
  \catcode64=11 % @
  \catcode123=1 % {
  \catcode125=2 % }
  \expandafter\let\expandafter\x\csname ver@epstopdf.sty\endcsname
  \ifx\x\relax % plain-TeX, first loading
  \else
    \def\empty{}%
    \ifx\x\empty % LaTeX, first loading,
      % variable is initialized, but \ProvidesPackage not yet seen
    \else
      \expandafter\ifx\csname PackageInfo\endcsname\relax
        \def\x#1#2{%
          \immediate\write-1{Package #1 Info: #2.}%
        }%
      \else
        \def\x#1#2{\PackageInfo{#1}{#2, stopped}}%
      \fi
      \x{epstopdf}{The package is already loaded}%
      \aftergroup\endinput
    \fi
  \fi
\endgroup%
%    \end{macrocode}
%    Package identification:
%    \begin{macrocode}
\begingroup\catcode61\catcode48\catcode32=10\relax%
  \catcode13=5 % ^^M
  \endlinechar=13 %
  \catcode35=6 % #
  \catcode39=12 % '
  \catcode40=12 % (
  \catcode41=12 % )
  \catcode44=12 % ,
  \catcode45=12 % -
  \catcode46=12 % .
  \catcode47=12 % /
  \catcode58=12 % :
  \catcode64=11 % @
  \catcode91=12 % [
  \catcode93=12 % ]
  \catcode123=1 % {
  \catcode125=2 % }
  \expandafter\ifx\csname ProvidesPackage\endcsname\relax
    \def\x#1#2#3[#4]{\endgroup
      \immediate\write-1{Package: #3 #4}%
      \xdef#1{#4}%
    }%
  \else
    \def\x#1#2[#3]{\endgroup
      #2[{#3}]%
      \ifx#1\@undefined
        \xdef#1{#3}%
      \fi
      \ifx#1\relax
        \xdef#1{#3}%
      \fi
    }%
  \fi
\expandafter\x\csname ver@epstopdf.sty\endcsname
\ProvidesPackage{epstopdf}%
  [2016/05/15 v2.6 Conversion with epstopdf on the fly (HO)]%
%    \end{macrocode}
%    Larger catcode set because of configuration files needed.
%    \begin{macrocode}
\begingroup\catcode61\catcode48\catcode32=10\relax%
  \catcode13=5 % ^^M
  \endlinechar=13 %
  \catcode123=1 % {
  \catcode125=2 % }
  \catcode64=11 % @
  \def\x{\endgroup
    \expandafter\edef\csname ETE@AtEnd\endcsname{%
      \endlinechar=\the\endlinechar\relax
      \catcode13=\the\catcode13\relax
      \catcode32=\the\catcode32\relax
      \catcode35=\the\catcode35\relax
      \catcode61=\the\catcode61\relax
      \catcode64=\the\catcode64\relax
      \catcode123=\the\catcode123\relax
      \catcode125=\the\catcode125\relax
    }%
  }%
\x\catcode61\catcode48\catcode32=10\relax%
\catcode13=5 % ^^M
\endlinechar=13 %
\catcode35=6 % #
\catcode64=11 % @
\catcode123=1 % {
\catcode125=2 % }
\def\TMP@EnsureCode#1#2{%
  \edef\ETE@AtEnd{%
    \ETE@AtEnd
    \catcode#1=\the\catcode#1\relax
  }%
  \catcode#1=#2\relax
}
\TMP@EnsureCode{39}{12}% '
\TMP@EnsureCode{40}{12}% (
\TMP@EnsureCode{41}{12}% )
\TMP@EnsureCode{42}{12}% *
\TMP@EnsureCode{45}{12}% -
\TMP@EnsureCode{47}{12}% /
\TMP@EnsureCode{91}{12}% [
\TMP@EnsureCode{93}{12}% ]
\edef\ETE@AtEnd{\ETE@AtEnd\noexpand\endinput}
%    \end{macrocode}
%
%    \begin{macrocode}
\let\ETE@SavedAtEnd\ETE@AtEnd
\RequirePackage{epstopdf-base}[2016/05/15]
\let\ETE@AtEnd\ETE@SavedAtEnd
%    \end{macrocode}
%
% \subsubsection{Option handling}
%
%    \begin{macrocode}
\DeclareOption*{%
  \expandafter\epstopdfsetup\expandafter{\CurrentOption}%
}
\ProcessOptions*\relax
%    \end{macrocode}
%
%    \begin{macrocode}
\ETE@AtEnd%
%</package>
%    \end{macrocode}
%
% \subsection{Base package}
%
%    \begin{macrocode}
%<*base>
%    \end{macrocode}
%
% \subsection{Preparations}
%
% \subsubsection{Relead check and identification}
%
%    Reload check, especially if the package is not used with \LaTeX.
%    \begin{macrocode}
\begingroup\catcode61\catcode48\catcode32=10\relax%
  \catcode13=5 % ^^M
  \endlinechar=13 %
  \catcode35=6 % #
  \catcode39=12 % '
  \catcode44=12 % ,
  \catcode45=12 % -
  \catcode46=12 % .
  \catcode58=12 % :
  \catcode64=11 % @
  \catcode123=1 % {
  \catcode125=2 % }
  \expandafter\let\expandafter\x\csname ver@epstopdf-base.sty\endcsname
  \ifx\x\relax % plain-TeX, first loading
  \else
    \def\empty{}%
    \ifx\x\empty % LaTeX, first loading,
      % variable is initialized, but \ProvidesPackage not yet seen
    \else
      \expandafter\ifx\csname PackageInfo\endcsname\relax
        \def\x#1#2{%
          \immediate\write-1{Package #1 Info: #2.}%
        }%
      \else
        \def\x#1#2{\PackageInfo{#1}{#2, stopped}}%
      \fi
      \x{epstopdf-base}{The package is already loaded}%
      \aftergroup\endinput
    \fi
  \fi
\endgroup%
%    \end{macrocode}
%    Package identification:
%    \begin{macrocode}
\begingroup\catcode61\catcode48\catcode32=10\relax%
  \catcode13=5 % ^^M
  \endlinechar=13 %
  \catcode35=6 % #
  \catcode39=12 % '
  \catcode40=12 % (
  \catcode41=12 % )
  \catcode44=12 % ,
  \catcode45=12 % -
  \catcode46=12 % .
  \catcode47=12 % /
  \catcode58=12 % :
  \catcode64=11 % @
  \catcode91=12 % [
  \catcode93=12 % ]
  \catcode123=1 % {
  \catcode125=2 % }
  \expandafter\ifx\csname ProvidesPackage\endcsname\relax
    \def\x#1#2#3[#4]{\endgroup
      \immediate\write-1{Package: #3 #4}%
      \xdef#1{#4}%
    }%
  \else
    \def\x#1#2[#3]{\endgroup
      #2[{#3}]%
      \ifx#1\@undefined
        \xdef#1{#3}%
      \fi
      \ifx#1\relax
        \xdef#1{#3}%
      \fi
    }%
  \fi
\expandafter\x\csname ver@epstopdf-base.sty\endcsname
\ProvidesPackage{epstopdf-base}%
  [2016/05/15 v2.6 Base part for package epstopdf]%
%    \end{macrocode}
%
% \subsubsection{Catcodes}
%
%    \begin{macrocode}
\begingroup\catcode61\catcode48\catcode32=10\relax%
  \catcode13=5 % ^^M
  \endlinechar=13 %
  \catcode123=1 % {
  \catcode125=2 % }
  \catcode64=11 % @
  \def\x{\endgroup
    \expandafter\edef\csname ETE@AtEnd\endcsname{%
      \endlinechar=\the\endlinechar\relax
      \catcode13=\the\catcode13\relax
      \catcode32=\the\catcode32\relax
      \catcode35=\the\catcode35\relax
      \catcode61=\the\catcode61\relax
      \catcode64=\the\catcode64\relax
      \catcode123=\the\catcode123\relax
      \catcode125=\the\catcode125\relax
    }%
  }%
\x\catcode61\catcode48\catcode32=10\relax%
\catcode13=5 % ^^M
\endlinechar=13 %
\catcode35=6 % #
\catcode64=11 % @
\catcode123=1 % {
\catcode125=2 % }
\def\TMP@EnsureCode#1#2{%
  \edef\ETE@AtEnd{%
    \ETE@AtEnd
    \catcode#1=\the\catcode#1\relax
  }%
  \catcode#1=#2\relax
}
\TMP@EnsureCode{33}{12}% !
\TMP@EnsureCode{39}{12}% '
\TMP@EnsureCode{42}{12}% *
\TMP@EnsureCode{44}{12}% ,
\TMP@EnsureCode{45}{12}% -
\TMP@EnsureCode{46}{12}% .
\TMP@EnsureCode{47}{12}% /
\TMP@EnsureCode{58}{12}% :
\TMP@EnsureCode{60}{12}% <
\TMP@EnsureCode{62}{12}% >
\TMP@EnsureCode{96}{12}% `
\edef\ETE@AtEnd{\ETE@AtEnd\noexpand\endinput}
%    \end{macrocode}
%
% \subsubsection{Load packages}
%
%    \begin{macrocode}
\RequirePackage{infwarerr}[2007/09/09]
\RequirePackage{grfext}\relax
\RequirePackage{kvoptions}[2007/10/02]
\RequirePackage{pdftexcmds}[2007/11/11]
%    \end{macrocode}
%
% \subsection{Checks}
%
% Check, whether package graphics is loaded
% (also graphicx loads graphics).
% Because miniltx.tex does not know \cs{@ifpackageloaded}
% we test for \cs{Gin@setfile} instead.
%    \begin{macrocode}
\begingroup\expandafter\expandafter\expandafter\endgroup
\expandafter\ifx\csname Gin@setfile\endcsname\relax
  \@PackageWarningNoLine{epstopdf}{%
    No graphics package \string`graphic{s,x}\string' loaded%
  }%
  \newcommand*{\epstopdfsetup}[1]{}%
  \expandafter\ETE@AtEnd
\fi%
%    \end{macrocode}
%
% Check, whether pdftex.def is loaded. \cs{ver@pdftex.def} is not
% available with miniltx.tex, thus we test for \cs{Gin@driver}.
%    \begin{macrocode}
\begingroup
  \def\x{luatex.def}%
  \ifx\Gin@driver\x
  \else
  \def\x{pdftex.def}%
  \ifx\Gin@driver\x
  \else
    \@PackageWarningNoLine{epstopdf}{%
      Drivers other than `pdftex' and `luatex' are not supported%
    }%
    \endgroup
    \newcommand*{\epstopdfsetup}[1]{}%
    \expandafter\ETE@AtEnd
  \fi%
  \fi%
\endgroup
%    \end{macrocode}
%
% Check, whether the shell escape feature is enabled.
%
%    \begin{macrocode}
\begingroup
  \expandafter\ifx\csname pdf@shellescape\endcsname\relax
  \else
    \ifnum\pdf@shellescape>0 %
    \else
      \@PackageWarningNoLine{epstopdf}{%
        Shell escape feature is not enabled%
      }%
    \fi
  \fi
\endgroup
%    \end{macrocode}
%
% \subsection{Package loading}
%
%    \begin{macrocode}
%    \end{macrocode}
%
% \subsection{Options}
%
%    \begin{macrocode}
\SetupKeyvalOptions{family=ETE,prefix=ETE@}
\DeclareBoolOption{update}
\DeclareBoolOption{verbose}
\newif\ifETE@prepend
\DeclareVoidOption{prepend}{\ETE@prependtrue}
\DeclareVoidOption{append}{\ETE@prependfalse}
\DeclareStringOption{outdir}
\DeclareStringOption{suffix}
\DeclareBoolOption{prefersuffix}
\DeclareStringOption{program@epstopdf}
%    \end{macrocode}
%    Options \xoption{disable} and \xoption{enable} are for testing
%    only. Therefore they are not documented on purpose.
%    \begin{macrocode}
\DeclareBoolOption{disable}
\DeclareComplementaryOption{enable}{disable}
%    \end{macrocode}
%    \begin{macrocode}
\newcommand*{\epstopdfsetup}{\setkeys{ETE}}
%    \end{macrocode}
%
% \subsubsection{Default setting}
%
%    \begin{macrocode}
\epstopdfsetup{%
  verbose,%
  enable,%
  append,%
  update,%
  prefersuffix,%
  suffix=-\SourceExt-converted-to,%
  program@epstopdf=epstopdf%
}
%    \end{macrocode}
%
% \subsection{Make and verbose}
%
%    \begin{macrocode}
\begingroup\expandafter\expandafter\expandafter\endgroup
\expandafter\ifx\csname pdf@filemoddate\endcsname\relax
  \def\ETE@Make#1#2{%
    \ifETE@update
      \ETE@WarnModDate
    \fi
    \@firstofone
  }%
  \def\ETE@WarnModDate{%
    \@PackageWarningNoLine{epstopdf}{%
      \string\pdffilemoddate\space is not available,\MessageBreak
      option `update' will be ignored%
    }%
    \global\let\ETE@WarnModDate\relax
  }%
  \def\ETE@FileInfo#1#2{#1 file: <#2>}%
\else
  \def\ETE@Make#1#2{%
    \ifETE@update
      \ifnum\pdf@strcmp{\pdf@filemoddate{#1}}{\pdf@filemoddate{#2}}>0 %
        \expandafter\expandafter\expandafter\@firstofone
      \else
        \@PackageInfoNoLine{epstopdf}{%
          Output file is already uptodate%
        }%
        \expandafter\expandafter\expandafter\@gobble
      \fi
    \else
      \expandafter\@firstofone
    \fi
  }%
  \def\ETE@FileInfo#1#2{%
    #1 file: <#2>%
    \expandafter\expandafter\expandafter
    \ETE@Date\pdf@filemoddate{#2}\@nil
    \expandafter\expandafter\expandafter
    \ETE@Size\pdf@filesize{#2}\@nil
  }%
  \def\ETE@Date#1\@nil{%
    \ifx\\#1\\%
    \else
      \ETE@@Date#1\@nil
    \fi
  }%
  \def\ETE@@Date#1:#2#3#4#5#6#7#8#9{%
    \MessageBreak
    \@spaces\space\space\space date: #2#3#4#5-#6#7-#8#9 %
    \ETE@@Time
  }%
  \def\ETE@@Time#1#2#3#4#5#6#7\@nil{%
    #1#2:#3#4:#5#6%
  }%
  \def\ETE@Size#1\@nil{%
    \ifx\\#1\\%
    \else
      \MessageBreak
      \@spaces\space\space\space size: #1 bytes%
    \fi
  }%
\fi
%    \end{macrocode}
%
% \subsection{Adding conversion support}
%
%    Patch \cs{Gin@setfile} to execute |#3|, if it contains
%    a command.
%    \begin{macrocode}
\expandafter\ifx\csname ETE@OrgGin@setfile\endcsname\relax
  \let\ETE@OrgGin@setfile\Gin@setfile
\else
  \@PackageError{epstopdf}{%
    Command \string\ETE@OrgGin@setfile\space
    already defined.\MessageBreak
  }{%
    Probably some package has included the code of this package%
    \MessageBreak
    instead of using \string\RequirePackage{epstopdf}.%
    \MessageBreak
    \@ehc
  }%
\fi
\def\ETE@IfFileExists{%
  \begingroup\expandafter\expandafter\expandafter\endgroup
  \expandafter\ifx\csname grffile@IfFileExists\endcsname\relax
    \expandafter\IfFileExists
  \else
    \global\let\ETE@IfFileExists\grffile@IfFileExists
    \expandafter\grffile@IfFileExists
  \fi
}
\def\ETE@Skip#1#2\x\fi\fi{%
  \fi
  \fi
  \endgroup
  \fi
  \fi
  #1%
}
\newif\ifETE@InsideSetfile
\newcommand*{\epstopdfcall}[1]{%
  \ifETE@InsideSetfile
    \expandafter\@firstoftwo
  \else
    \expandafter\@secondoftwo
  \fi
  {`#1}%
  {\Gin@base\Gin@ext}%
}
\def\ETE@DefCommandLine#1{%
  \edef\CommandLine{\expandafter\fi\if`#1}%
}
\def\ETE@DefX#1{%
  \expandafter\expandafter\expandafter\def
  \expandafter\expandafter\expandafter\x
  \expandafter\expandafter\expandafter{%
    \expandafter\fi\if`#1\relax\else
  }%
}
\def\ETE@Gin@setfile#1#2#3{%
  \ifETE@disable
    \ETE@OrgGin@setfile{#1}{#2}{#3}%
  \else
    \begingroup
      \ETE@InsideSetfiletrue
      \ETE@DefX{#3}%
    \expandafter\endgroup
    \ifx\x\@empty
      \ETE@OrgGin@setfile{#1}{#2}{#3}%
    \else
      \begingroup
        \ETE@InsideSetfiletrue
        \def\GraphicsType{#1}%
        \def\GraphicsRead{#2}%
        \ifETE@prefersuffix
        \else
          \ifx\ETE@suffix\@empty
          \else
            \ETE@IfFileExists{\Gin@base\GraphicsRead}{%
              \ETE@Skip{%
                \ETE@OrgGin@setfile{#1}{#2}{\Gin@base#2}%
              }%
            }{%
              \let\next\relax
            }%
            \next
          \fi
        \fi
        \ifx\Gin@ext\relax
          \let\SourceExt\Gin@eext
          \def\SourceFile{\Gin@base\Gin@eext}%
        \else
          \let\SourceExt\Gin@ext
          \def\SourceFile{\Gin@base\Gin@ext}%
        \fi
        \edef\SourceExt{% remove dot
          \expandafter\@cdr\SourceExt\@empty\@nil
        }%
        \let\OutputDirectory\ETE@outdir
        \ifx\OutputDirectory\@empty
          \edef\OutputFile{\ETE@GenerateName{\Gin@base}{#2}}%
        \else
          \begingroup
            \filename@parse{\Gin@base#2}%
            \edef\x{\endgroup
              \def\noexpand\OutputFile{%
                \ETE@GenerateName{%
                  \OutputDirectory\filename@base
                }{#2}%
              }%
            }%
          \x
        \fi
        \ETE@DefCommandLine{#3}%
        \ifETE@verbose
          \@PackageInfo{epstopdf}{%
            \ETE@FileInfo{Source}\SourceFile\MessageBreak
            \ETE@FileInfo{Output}\OutputFile\MessageBreak
            Command: <\CommandLine>\MessageBreak
            \string\includegraphics
          }%
        \fi
        \ETE@Make\SourceFile\OutputFile{%
          \pdf@system{\CommandLine}%
          \ifETE@verbose
            \@PackageInfoNoLine{epstopdf}{%
              \ETE@FileInfo{Result}\OutputFile
            }%
          \fi
        }%
        \edef\x{\endgroup
          \ifx\OutputDirectory\@empty
          \else
            \def\noexpand\Gin@base{%
              \OutputDirectory\noexpand\filename@base
            }%
          \fi
          \ifx\ETE@suffix\@empty
          \else
            \edef\noexpand\Gin@base{%
              \noexpand\Gin@base\ETE@suffix
            }%
          \fi
          \noexpand\ETE@OrgGin@setfile{%
            \GraphicsType
          }{%
            \GraphicsRead
          }{%
            \OutputFile
          }%
        }%
      \x
    \fi
  \fi
}
\let\Gin@setfile\ETE@Gin@setfile
\def\ETE@GenerateName#1#2{%
  #1\ETE@suffix#2%
}
%    \end{macrocode}
%
% \subsection{Declare graphics rule}
%
%    \begin{macrocode}
\newcommand*{\epstopdfDeclareGraphicsRule}[4]{%
  \ifx\\#4\\%
    \@PackageError{epstopdf-base}{%
      Conversion command is missing%
    }\@ehc
  \else
    \begingroup
      \@ifundefined{Gin@rule@#1}{%
      }{%
        \@PackageInfo{epstopdf-base}{%
          Redefining graphics rule for `#1'%
        }%
      }%
    \endgroup
    \@namedef{Gin@rule@#1}##1{{#2}{#3}{\epstopdfcall{#4}}}%
  \fi
}
%    \end{macrocode}
%
% \cs{DeclareGraphicsRule} for .eps
%    \begin{macrocode}
\epstopdfDeclareGraphicsRule{.eps}{pdf}{.pdf}{%
  \ETE@epstopdf{#1}%
}
\def\ETE@epstopdf#1{%
  \ETE@program@epstopdf\space
  \ifcase\ifx\OutputDirectory\@empty
           \ifx\ETE@suffix\@empty
             1%
           \fi
         \fi
         0 %
    --outfile=\OutputFile\space
  \fi
  #1%
}
%    \end{macrocode}
%
%    \begin{macrocode}
\ifETE@prepend
  \expandafter\PrependGraphicsExtensions
\else
  \expandafter\AppendGraphicsExtensions
\fi
{.eps}
\let\ETE@prepend\@undefined
\DeclareVoidOption{prepend}{%
  \PrependGraphicsExtensions{.eps}%
}
\let\ETE@append\@undefined
\DeclareVoidOption{append}{%
  \AppendGraphicsExtensions{.eps}%
}
%    \end{macrocode}
%
%    \begin{macrocode}
\InputIfFileExists{epstopdf-sys.cfg}{}{}
\InputIfFileExists{epstopdf.cfg}{}{}
%    \end{macrocode}
%
%    \begin{macrocode}
\ETE@AtEnd%
%</base>
%    \end{macrocode}
%
% \section{Test}
%
% \subsection{Preface for standard catcode check}
%
%    \begin{macrocode}
%<*test1>
\input miniltx.tex\relax
\def\Gin@driver{pdftex.def}
\input graphicx.sty\relax
\resetatcatcode
%</test1>
%    \end{macrocode}
%
% \subsection{Catcode checks for loading}
%
%    \begin{macrocode}
%<*test1>
%    \end{macrocode}
%    \begin{macrocode}
\catcode`\{=1 %
\catcode`\}=2 %
\catcode`\#=6 %
\catcode`\@=11 %
\expandafter\ifx\csname count@\endcsname\relax
  \countdef\count@=255 %
\fi
\expandafter\ifx\csname @gobble\endcsname\relax
  \long\def\@gobble#1{}%
\fi
\expandafter\ifx\csname @firstofone\endcsname\relax
  \long\def\@firstofone#1{#1}%
\fi
\expandafter\ifx\csname loop\endcsname\relax
  \expandafter\@firstofone
\else
  \expandafter\@gobble
\fi
{%
  \def\loop#1\repeat{%
    \def\body{#1}%
    \iterate
  }%
  \def\iterate{%
    \body
      \let\next\iterate
    \else
      \let\next\relax
    \fi
    \next
  }%
  \let\repeat=\fi
}%
\def\RestoreCatcodes{}
\count@=0 %
\loop
  \edef\RestoreCatcodes{%
    \RestoreCatcodes
    \catcode\the\count@=\the\catcode\count@\relax
  }%
\ifnum\count@<255 %
  \advance\count@ 1 %
\repeat

\def\RangeCatcodeInvalid#1#2{%
  \count@=#1\relax
  \loop
    \catcode\count@=15 %
  \ifnum\count@<#2\relax
    \advance\count@ 1 %
  \repeat
}
\def\RangeCatcodeCheck#1#2#3{%
  \count@=#1\relax
  \loop
    \ifnum#3=\catcode\count@
    \else
      \errmessage{%
        Character \the\count@\space
        with wrong catcode \the\catcode\count@\space
        instead of \number#3%
      }%
    \fi
  \ifnum\count@<#2\relax
    \advance\count@ 1 %
  \repeat
}
\def\space{ }
\expandafter\ifx\csname LoadCommand\endcsname\relax
  \def\LoadCommand{\input epstopdf.sty\relax}%
\fi
\def\Test{%
  \RangeCatcodeInvalid{0}{47}%
  \RangeCatcodeInvalid{58}{64}%
  \RangeCatcodeInvalid{91}{96}%
  \RangeCatcodeInvalid{123}{255}%
  \catcode`\@=12 %
  \catcode`\\=0 %
  \catcode`\%=14 %
  \LoadCommand
  \RangeCatcodeCheck{0}{36}{15}%
  \RangeCatcodeCheck{37}{37}{14}%
  \RangeCatcodeCheck{38}{47}{15}%
  \RangeCatcodeCheck{48}{57}{12}%
  \RangeCatcodeCheck{58}{63}{15}%
  \RangeCatcodeCheck{64}{64}{12}%
  \RangeCatcodeCheck{65}{90}{11}%
  \RangeCatcodeCheck{91}{91}{15}%
  \RangeCatcodeCheck{92}{92}{0}%
  \RangeCatcodeCheck{93}{96}{15}%
  \RangeCatcodeCheck{97}{122}{11}%
  \RangeCatcodeCheck{123}{255}{15}%
  \RestoreCatcodes
}
\Test
\csname @@end\endcsname
\end
%    \end{macrocode}
%    \begin{macrocode}
%</test1>
%    \end{macrocode}
%
% \section{Installation}
%
% \subsection{Download}
%
% \paragraph{Package.} This package is available on
% CTAN\footnote{\url{http://ctan.org/pkg/epstopdf}}:
% \begin{description}
% \item[\CTAN{macros/latex/contrib/oberdiek/epstopdf.dtx}] The source file.
% \item[\CTAN{macros/latex/contrib/oberdiek/epstopdf.pdf}] Documentation.
% \end{description}
%
%
% \paragraph{Bundle.} All the packages of the bundle `oberdiek'
% are also available in a TDS compliant ZIP archive. There
% the packages are already unpacked and the documentation files
% are generated. The files and directories obey the TDS standard.
% \begin{description}
% \item[\CTAN{install/macros/latex/contrib/oberdiek.tds.zip}]
% \end{description}
% \emph{TDS} refers to the standard ``A Directory Structure
% for \TeX\ Files'' (\CTAN{tds/tds.pdf}). Directories
% with \xfile{texmf} in their name are usually organized this way.
%
% \subsection{Bundle installation}
%
% \paragraph{Unpacking.} Unpack the \xfile{oberdiek.tds.zip} in the
% TDS tree (also known as \xfile{texmf} tree) of your choice.
% Example (linux):
% \begin{quote}
%   |unzip oberdiek.tds.zip -d ~/texmf|
% \end{quote}
%
% \paragraph{Script installation.}
% Check the directory \xfile{TDS:scripts/oberdiek/} for
% scripts that need further installation steps.
% Package \xpackage{attachfile2} comes with the Perl script
% \xfile{pdfatfi.pl} that should be installed in such a way
% that it can be called as \texttt{pdfatfi}.
% Example (linux):
% \begin{quote}
%   |chmod +x scripts/oberdiek/pdfatfi.pl|\\
%   |cp scripts/oberdiek/pdfatfi.pl /usr/local/bin/|
% \end{quote}
%
% \subsection{Package installation}
%
% \paragraph{Unpacking.} The \xfile{.dtx} file is a self-extracting
% \docstrip\ archive. The files are extracted by running the
% \xfile{.dtx} through \plainTeX:
% \begin{quote}
%   \verb|tex epstopdf.dtx|
% \end{quote}
%
% \paragraph{TDS.} Now the different files must be moved into
% the different directories in your installation TDS tree
% (also known as \xfile{texmf} tree):
% \begin{quote}
% \def\t{^^A
% \begin{tabular}{@{}>{\ttfamily}l@{ $\rightarrow$ }>{\ttfamily}l@{}}
%   epstopdf.sty & tex/latex/oberdiek/epstopdf.sty\\
%   epstopdf-base.sty & tex/latex/oberdiek/epstopdf-base.sty\\
%   epstopdf.pdf & doc/latex/oberdiek/epstopdf.pdf\\
%   test/epstopdf-test1.tex & doc/latex/oberdiek/test/epstopdf-test1.tex\\
%   epstopdf.dtx & source/latex/oberdiek/epstopdf.dtx\\
% \end{tabular}^^A
% }^^A
% \sbox0{\t}^^A
% \ifdim\wd0>\linewidth
%   \begingroup
%     \advance\linewidth by\leftmargin
%     \advance\linewidth by\rightmargin
%   \edef\x{\endgroup
%     \def\noexpand\lw{\the\linewidth}^^A
%   }\x
%   \def\lwbox{^^A
%     \leavevmode
%     \hbox to \linewidth{^^A
%       \kern-\leftmargin\relax
%       \hss
%       \usebox0
%       \hss
%       \kern-\rightmargin\relax
%     }^^A
%   }^^A
%   \ifdim\wd0>\lw
%     \sbox0{\small\t}^^A
%     \ifdim\wd0>\linewidth
%       \ifdim\wd0>\lw
%         \sbox0{\footnotesize\t}^^A
%         \ifdim\wd0>\linewidth
%           \ifdim\wd0>\lw
%             \sbox0{\scriptsize\t}^^A
%             \ifdim\wd0>\linewidth
%               \ifdim\wd0>\lw
%                 \sbox0{\tiny\t}^^A
%                 \ifdim\wd0>\linewidth
%                   \lwbox
%                 \else
%                   \usebox0
%                 \fi
%               \else
%                 \lwbox
%               \fi
%             \else
%               \usebox0
%             \fi
%           \else
%             \lwbox
%           \fi
%         \else
%           \usebox0
%         \fi
%       \else
%         \lwbox
%       \fi
%     \else
%       \usebox0
%     \fi
%   \else
%     \lwbox
%   \fi
% \else
%   \usebox0
% \fi
% \end{quote}
% If you have a \xfile{docstrip.cfg} that configures and enables \docstrip's
% TDS installing feature, then some files can already be in the right
% place, see the documentation of \docstrip.
%
% \subsection{Refresh file name databases}
%
% If your \TeX~distribution
% (\teTeX, \mikTeX, \dots) relies on file name databases, you must refresh
% these. For example, \teTeX\ users run \verb|texhash| or
% \verb|mktexlsr|.
%
% \subsection{Some details for the interested}
%
% \paragraph{Attached source.}
%
% The PDF documentation on CTAN also includes the
% \xfile{.dtx} source file. It can be extracted by
% AcrobatReader 6 or higher. Another option is \textsf{pdftk},
% e.g. unpack the file into the current directory:
% \begin{quote}
%   \verb|pdftk epstopdf.pdf unpack_files output .|
% \end{quote}
%
% \paragraph{Unpacking with \LaTeX.}
% The \xfile{.dtx} chooses its action depending on the format:
% \begin{description}
% \item[\plainTeX:] Run \docstrip\ and extract the files.
% \item[\LaTeX:] Generate the documentation.
% \end{description}
% If you insist on using \LaTeX\ for \docstrip\ (really,
% \docstrip\ does not need \LaTeX), then inform the autodetect routine
% about your intention:
% \begin{quote}
%   \verb|latex \let\install=y% \iffalse meta-comment
%
% File: epstopdf.dtx
% Version: 2016/05/15 v2.6
% Info: Conversion with epstopdf on the fly
%
% Copyright (C) 2001, 2006-2010 by
%    Heiko Oberdiek <heiko.oberdiek at googlemail.com>
%
% This work may be distributed and/or modified under the
% conditions of the LaTeX Project Public License, either
% version 1.3c of this license or (at your option) any later
% version. This version of this license is in
%    http://www.latex-project.org/lppl/lppl-1-3c.txt
% and the latest version of this license is in
%    http://www.latex-project.org/lppl.txt
% and version 1.3 or later is part of all distributions of
% LaTeX version 2005/12/01 or later.
%
% This work has the LPPL maintenance status "maintained".
%
% This Current Maintainer of this work is Heiko Oberdiek.
%
% This work consists of the main source file epstopdf.dtx
% and the derived files
%    epstopdf.sty, epstopdf.pdf, epstopdf.ins, epstopdf.drv,
%    epstopdf-base.sty, epstopdf-test1.tex.
%
% Distribution:
%    CTAN:macros/latex/contrib/oberdiek/epstopdf.dtx
%    CTAN:macros/latex/contrib/oberdiek/epstopdf.pdf
%
% Unpacking:
%    (a) If epstopdf.ins is present:
%           tex epstopdf.ins
%    (b) Without epstopdf.ins:
%           tex epstopdf.dtx
%    (c) If you insist on using LaTeX
%           latex \let\install=y% \iffalse meta-comment
%
% File: epstopdf.dtx
% Version: 2016/05/15 v2.6
% Info: Conversion with epstopdf on the fly
%
% Copyright (C) 2001, 2006-2010 by
%    Heiko Oberdiek <heiko.oberdiek at googlemail.com>
%
% This work may be distributed and/or modified under the
% conditions of the LaTeX Project Public License, either
% version 1.3c of this license or (at your option) any later
% version. This version of this license is in
%    http://www.latex-project.org/lppl/lppl-1-3c.txt
% and the latest version of this license is in
%    http://www.latex-project.org/lppl.txt
% and version 1.3 or later is part of all distributions of
% LaTeX version 2005/12/01 or later.
%
% This work has the LPPL maintenance status "maintained".
%
% This Current Maintainer of this work is Heiko Oberdiek.
%
% This work consists of the main source file epstopdf.dtx
% and the derived files
%    epstopdf.sty, epstopdf.pdf, epstopdf.ins, epstopdf.drv,
%    epstopdf-base.sty, epstopdf-test1.tex.
%
% Distribution:
%    CTAN:macros/latex/contrib/oberdiek/epstopdf.dtx
%    CTAN:macros/latex/contrib/oberdiek/epstopdf.pdf
%
% Unpacking:
%    (a) If epstopdf.ins is present:
%           tex epstopdf.ins
%    (b) Without epstopdf.ins:
%           tex epstopdf.dtx
%    (c) If you insist on using LaTeX
%           latex \let\install=y\input{epstopdf.dtx}
%        (quote the arguments according to the demands of your shell)
%
% Documentation:
%    (a) If epstopdf.drv is present:
%           latex epstopdf.drv
%    (b) Without epstopdf.drv:
%           latex epstopdf.dtx; ...
%    The class ltxdoc loads the configuration file ltxdoc.cfg
%    if available. Here you can specify further options, e.g.
%    use A4 as paper format:
%       \PassOptionsToClass{a4paper}{article}
%
%    Programm calls to get the documentation (example):
%       pdflatex epstopdf.dtx
%       makeindex -s gind.ist epstopdf.idx
%       pdflatex epstopdf.dtx
%       makeindex -s gind.ist epstopdf.idx
%       pdflatex epstopdf.dtx
%
% Installation:
%    TDS:tex/latex/oberdiek/epstopdf.sty
%    TDS:tex/latex/oberdiek/epstopdf-base.sty
%    TDS:doc/latex/oberdiek/epstopdf.pdf
%    TDS:doc/latex/oberdiek/test/epstopdf-test1.tex
%    TDS:source/latex/oberdiek/epstopdf.dtx
%
%<*ignore>
\begingroup
  \catcode123=1 %
  \catcode125=2 %
  \def\x{LaTeX2e}%
\expandafter\endgroup
\ifcase 0\ifx\install y1\fi\expandafter
         \ifx\csname processbatchFile\endcsname\relax\else1\fi
         \ifx\fmtname\x\else 1\fi\relax
\else\csname fi\endcsname
%</ignore>
%<*install>
\input docstrip.tex
\Msg{************************************************************************}
\Msg{* Installation}
\Msg{* Package: epstopdf 2016/05/15 v2.6 Conversion with epstopdf on the fly (HO)}
\Msg{************************************************************************}

\keepsilent
\askforoverwritefalse

\let\MetaPrefix\relax
\preamble

This is a generated file.

Project: epstopdf
Version: 2016/05/15 v2.6

Copyright (C) 2001, 2006-2010 by
   Heiko Oberdiek <heiko.oberdiek at googlemail.com>

This work may be distributed and/or modified under the
conditions of the LaTeX Project Public License, either
version 1.3c of this license or (at your option) any later
version. This version of this license is in
   http://www.latex-project.org/lppl/lppl-1-3c.txt
and the latest version of this license is in
   http://www.latex-project.org/lppl.txt
and version 1.3 or later is part of all distributions of
LaTeX version 2005/12/01 or later.

This work has the LPPL maintenance status "maintained".

This Current Maintainer of this work is Heiko Oberdiek.

This work consists of the main source file epstopdf.dtx
and the derived files
   epstopdf.sty, epstopdf.pdf, epstopdf.ins, epstopdf.drv,
   epstopdf-base.sty, epstopdf-test1.tex.

\endpreamble
\let\MetaPrefix\DoubleperCent

\generate{%
  \file{epstopdf.ins}{\from{epstopdf.dtx}{install}}%
  \file{epstopdf.drv}{\from{epstopdf.dtx}{driver}}%
  \usedir{tex/latex/oberdiek}%
  \file{epstopdf.sty}{\from{epstopdf.dtx}{package}}%
  \file{epstopdf-base.sty}{\from{epstopdf.dtx}{base}}%
  \usedir{doc/latex/oberdiek/test}%
  \file{epstopdf-test1.tex}{\from{epstopdf.dtx}{test1}}%
  \nopreamble
  \nopostamble
  \usedir{source/latex/oberdiek/catalogue}%
  \file{epstopdf-pkg.xml}{\from{epstopdf.dtx}{catalogue}}%
}

\catcode32=13\relax% active space
\let =\space%
\Msg{************************************************************************}
\Msg{*}
\Msg{* To finish the installation you have to move the following}
\Msg{* files into a directory searched by TeX:}
\Msg{*}
\Msg{*     epstopdf.sty, epstopdf-base.sty}
\Msg{*}
\Msg{* To produce the documentation run the file `epstopdf.drv'}
\Msg{* through LaTeX.}
\Msg{*}
\Msg{* Happy TeXing!}
\Msg{*}
\Msg{************************************************************************}

\endbatchfile
%</install>
%<*ignore>
\fi
%</ignore>
%<*driver>
\NeedsTeXFormat{LaTeX2e}
\ProvidesFile{epstopdf.drv}%
  [2016/05/15 v2.6 Conversion with epstopdf on the fly (HO)]%
\documentclass{ltxdoc}
\usepackage{holtxdoc}[2011/11/22]
\begin{document}
  \DocInput{epstopdf.dtx}%
\end{document}
%</driver>
% \fi
%
%
% \CharacterTable
%  {Upper-case    \A\B\C\D\E\F\G\H\I\J\K\L\M\N\O\P\Q\R\S\T\U\V\W\X\Y\Z
%   Lower-case    \a\b\c\d\e\f\g\h\i\j\k\l\m\n\o\p\q\r\s\t\u\v\w\x\y\z
%   Digits        \0\1\2\3\4\5\6\7\8\9
%   Exclamation   \!     Double quote  \"     Hash (number) \#
%   Dollar        \$     Percent       \%     Ampersand     \&
%   Acute accent  \'     Left paren    \(     Right paren   \)
%   Asterisk      \*     Plus          \+     Comma         \,
%   Minus         \-     Point         \.     Solidus       \/
%   Colon         \:     Semicolon     \;     Less than     \<
%   Equals        \=     Greater than  \>     Question mark \?
%   Commercial at \@     Left bracket  \[     Backslash     \\
%   Right bracket \]     Circumflex    \^     Underscore    \_
%   Grave accent  \`     Left brace    \{     Vertical bar  \|
%   Right brace   \}     Tilde         \~}
%
% \GetFileInfo{epstopdf.drv}
%
% \title{The \xpackage{epstopdf} package}
% \date{2016/05/15 v2.6}
% \author{Heiko Oberdiek\thanks
% {Please report any issues at https://github.com/ho-tex/oberdiek/issues}\\
% \xemail{heiko.oberdiek at googlemail.com}}
%
% \maketitle
%
% \begin{abstract}
% This packages adds support of handling eps images
% to package \xpackage{graphics} or \xpackage{graphicx}
% with option \xoption{pdftex}. If an eps image is detected,
% epstopdf is automatically called to convert it to pdf format.
% \end{abstract}
%
% \tableofcontents
%
% \providecommand*{\xprogram}[1]{\textsf{#1}}
%
% \section{Documentation}
%
% \subsection{Introduction}
%
% \LaTeX\ provides its graphics bundle to include graphics files.
% Both packages \xpackage{graphics} or \xpackage{graphicx} may
% be used. the latter one loads the first and adds options
% in key value style for \cs{includegraphics}.
%
% Usually the drivers do not support all kind of graphics files.
% Other image types must be converted, before they become
% usuable. In case of driver \xprogram{dvips}, the graphics
% rule may contain a conversion rule. Then all that package
% \xpackage{graphics} must know is the bounding box, the command
% is passed to \xprogram{dvips} that calls it and embeds the
% converted image.
%
% However, \pdfTeX\ has its driver for PDF output already
% build in. It's graphics inclusion commands (\cs{pdfximage})
% does not allow the execution of external commands.
% Therefore commands in the last argument of
% \cs{DeclareGraphicsRule} were of no use.
% But external programs can be called within \pdfTeX.
% This feature is called ``shell escape'' or ``write 18''
% and must usually enabled explicitely because of security reasons.
% Now, this package \xpackage{epstopdf} hooks
% into package \xpackage{graphics}' code to catch
% that argument with the external command and executes it
% to convert the graphics file to a supported format and
% passes the control of graphics inclusion back to package
% \xpackage{graphics}.
%
% \subsection{Requirements}
%
% \begin{itemize}
% \item
%   The feature \cs{write18} must be enabled. This allows the
%   running of external programs during \TeX's compile run.
%   Keep in mind that this is a security risk. The feature is an
%   addition to \cs{TeX}. Mik\TeX, te\TeX, \TeX\ Live support it.
%   In Web2C based \TeX\ distributions (te\TeX, \TeX\ Live) it can be
%   enabled in the configuration file \xfile{texmf.cnf}:
%   \begin{quote}
%     |shell_escape = 1|
%   \end{quote}
%   Because of the security risk, it is better to do it on the
%   command line only:
%   \begin{quote}
%   \texttt{--shell-escape} (te\TeX, \TeX\ Live)\\
%   \texttt{--enable-write18} (MiK\TeX)
%   \end{quote}
%   Example:
%   \begin{quote}
%     |pdflatex -shell-escape test.tex|
%   \end{quote}
% \item
%   The program \xprogram{epstopdf} for the conversion from EPS
%   to PDF. However, other programs can
%   be used and configured by \cs{DeclareGraphicsRule}.
%   Example:
%   \begin{quote}
%   |\epstopdfDeclareGraphicsRule{.eps}{pdf}{.pdf}{%|\\
%   |  ps2pdf -dEPSCrop #1 \OutputFile|\\
%   |}|
%   \end{quote}
% \end{itemize}
%
% \subsection{Usage}
%
% The package is loaded after graphic\{s,x\}, e.g.:
% \begin{quote}
%   |\usepackage[pdftex]{graphicx}|\\
%   |\usepackage{epstopdf}|
% \end{quote}
% Now images with file name extension \xfile{.eps} are detected
% and supported using \cs{includegraphics}.
%
% If the graphics file name is explicitly specified with extension
% \xfile{.eps} the new rule for EPS files is called and
% the conversion performed. If option \xoption{update} is in force
% then the conversion step is dropped if the target file already
% exists and is not older then the EPS file.
%
% The situation is more complicate if the graphics file is
% given without file name extension. Then the \xpackage{graphics}
% package must search for a supported image file.
% The possible extensions are stored in the graphics extension
% list, that can be set by \cs{DeclareGraphicsExtensions}.
% The algorithm:
% \begin{quote}
%    \def\>{\mbox{}\quad}
%    function search( \meta{filebase} )\\
%    \>foreach \meta{ext} in \meta{graphics extensions}\\
%    \>\>foreach \meta{dir} in \meta{current directory}, \meta{\cs{graphicspath}}\\
%    \>\>\>\meta{file} := \meta{dir} + \meta{filebase} + \meta{ext}\\
%    \>\>\>if exist \meta{file}\\
%    \>\>\>\>return found\\
%    \>return not found
% \end{quote}
% Package \xpackage{epstopdf} puts \xfile{.eps} at the end of the
% graphics extension search list. This is the behaviour of option
% \xoption{append} that is enabled by default.
% That means, the conversion is called last
% unless a supported file type cannot be found earlier. This avoids
% unnecessary conversion steps that slow down the \LaTeX\ run.
% If you want to use option \xoption{update} and your \pdfTeX\ supports
% it, then an outdated PDF file also would be found earlier unless
% \xoption{suffix} is used that is the default since version 2.0.
%
% With an empty option \xoption{suffix} and option
% \xoption{prepend} there is a risk that an original PDF file is overwritten:
%
% If the original image file is the PDF file and there is also a generated
% EPS file, then the original PDF file can be regenerated (depending
% on the option settings) and the original PDF file gets lost.
% Therefore option \xoption{suffix} is introduced in version 1.9
% to create a separate name space for generated output files.
%
% \textbf{Note:} Usually the conversion program needs the exact location
% of the image file. Usually the current directory works.
% Also if the image file is found using \cs{graphicspath}, the
% location is known. However, if the image is somewhere
% in a directory of environment variable |TEXINPUTS|, then
% the package does not know the exact location and the
% conversion program will not find the image file unless it
% implements a search using |TEXINPUTS| (program \xprogram{kpsewhich}
% may be of help in this task).
%
% \subsection{Options}
%
% Options can be given as package options or later using:
% \begin{declcs}{epstopdfsetup} \M{key value list}
% \end{declcs}
% \LaTeX\ expands the option list before passing the option list
% to the package's option handling code. This can fail
% for option \xoption{suffix} if it contains some of the macros
% described below. Use \cs{epstopdfsetup} after the package is loaded.
% Or load package \xpackage{kvoptions-patch} before. This package
% is also loaded by option \xoption{patch} of package \xpackage{kvoptions}.
% \LaTeX's option code is redefined to respect key value options and
% let the values untouched.
%
% \begin{description}
% \def\It#1{\item[\xoption{#1}:]}^^A
% \It{update}
%   The conversion program is only called, if the target file
%   does not exist or is older than the source image file.
% \It{append}
%   Puts the extension \xfile{.eps} at the end of the graphics extension list
%   (default).
% \It{prepend}
%   Puts the extension \xfile{.eps} at the begin of the graphics extension list.
% \It{outdir}
%   The converted file may put in an other output directory.
%   The value of \xoption{outdir} must include the directory separator.
%   Example for the current directory:
%   \begin{quote}
%   |\epstopdfsetup{outdir=./}|
%   \end{quote}
%   For other directories ensure, that they can be found.
%   See \cs{graphicspath} or |TEXINPUTS|.
% \It{suffix}
%   This option takes a string that is put between the file name base
%   and the extension of the output file. Rationale: It can happen,
%   that a PDF file is the original file and the EPS file the
%   generated file. If now the package thinks, that the PDF file
%   is the generated file, it will `regnerate' it. But in reality
%   the original file is lost. Therefore I recommend to use this
%   option always to generate a separate name space for generated
%   files. Proposed value is \texttt{-generated} or \texttt{.generated}.
%   The suffix \texttt{.generated} will also work here without the need for
%   package \xpackage{grffile}).
%
%   Example:
%   \begin{quote}
%   |\epstopdfsetup{suffix=-generated}|\\
%   Then \xfile{foo.eps} is converted to \xfile{foo-generated.pdf}.
%   \end{quote}
%   \cs{SourceExt} can be used inside the suffix string.
%   It's will be replaced by the extension of the image source
%   file without the leading dot, for instance:
%   \begin{quote}
%   |\epstopdfsetup{suffix=-\SourceExt-converted-to}|\\
%   \xfile{foo.eps} $\Rightarrow$ \xfile{foo-eps-converted-to.pdf}
%   \end{quote}
%   See also the next option \xoption{prefersuffix} that
%   modifies the behaviour of option \xoption{suffix} in some
%   cases.
%
%   Default for \xoption{suffix} is `|-\SourceExt-converted-to|'.
%
% \It{prefersuffix}
%   If a suffix is set by option \xoption{suffix}, then there can be
%   two image file names that could be taken into account for inclusion:
%   A image file name with the suffix string inside
%   its name and a image file name without; e.g. for \xfile{foo.eps}
%   the names could be:
%   \begin{quote}
%   \xfile{foo-suffix.pdf}, \xfile{foo.pdf}
%   \end{quote}
%   If option \xoption{perfersuffix} is turned on, the
%   file \xfile{foo-suffix.pdf} and its generation is preferred
%   over using \xfile{foo.pdf}. Otherwise \xfile{foo.pdf} is included
%   without generating \xfile{foo-suffix.pdf}.
%   The default of option \xoption{prefersuffix} is \texttt{true}.
% \It{program@epstopdf}
%   The name for the conversion program from EPS to PDF,
%   default is `|epstopdf|'.
% \It{verbose}
%   It prints some information about the image in the \xfile{.log} file
%   (default).
% \end{description}
%
% \subsection{Configuration}
%
% \subsubsection{System configuration file \xfile{epstopdf-sys.cfg}}
%
% If \xfile{epstopdf-sys} exists it is loaded at the end of the
% package \xpackage{epstopdf-base} and before the user configuration
% file. It's intended for TeX distributors. Thus they could add
% additional conversion rules (e.g., .gif -> .png) or set options.
%
% \subsubsection{User configuration file \xfile{epstopdf.cfg}}
%
% A configuration file \xfile{epstopdf.cfg} is loaded at the end
% of the package if it exists. It can be used for changing
% the default option setting. Example:
% \begin{quote}
% |\epstopdfsetup{verbose=false}|
% \end{quote}
%
% \subsubsection{Conversion program}
%
% You can use \cs{DeclareGraphicsRule} in a similar way as the
% route via \xprogram{dvips} to specify the conversion
% command line.
% The conversion argument starts with a back tick, followed
% by the conversion command including parameters.
%
% The whole conversion argument should also be wrapped inside
% \cs{epstopdfcall}. This reduces the problem with packages
% (e.g. \xpackage{pst-pdf}) that use the conversion argument
% and expands it. Macros \cs{SourceFile}, \cs{OutputFile}, and
% \cs{SourceExt} are not defined outside \xpackage{epstopdf-base}'s
% \cs{Gin@setfile} and error messages because of undefined
% command names are the result. If \cs{epstopdfcall} detects that
% it is called outside \xpackage{epstopdf-base}'s \cs{Gin@setfile}
% then it replaces the conversion argument by
% package \xpackage{graphics}'s
% default, usually the image file.
%
% The following macros are available inside:
% \begin{description}
% \def\It#1{\item[\cs{#1}:]}
% \It{OutputFile}: output file name (with known path and extension)
% \It{SourceFile}: source file name (with known path and extension),
%                  usually the same as |#1|,
% \It{SourceExt}: source extension without leading dot.
% \end{description}
%
% \paragraph{Conversion from EPS to PDF.}
% Other programs than \xprogram{epstopdf} can be used to convert from
% EPS to PDF. Example that uses \xprogram{Ghostscript}:
% \begin{quote}
%   |\DeclareGraphicsRule{.eps}{pdf}{.pdf}{%|\\
%   |  \epstopdfcall{`ps2pdf -dEPSCrop #1 \noexpand\OutputFile}%|\\
%   |}|
% \end{quote}
% \cs{DeclareGraphicsRule} expands the argument, therefore \cs{noexpand}
% is necessary. As convenience package \xpackage{epstopdf-base} defines
% \cs{epstopdfDeclareGraphicsRule}. Then the conversion argument
% is not expanded, \cs{epstopdfcall} and the back tick are added:
% \begin{quote}
%   |\epstopdfDeclareGraphicsRule{.eps}{pdf}{.pdf}{%|\\
%   |  ps2pdf -dEPSCrop #1 \OutputFile|\\
%   |}|
% \end{quote}
%
% Also \cs{OutputFile} respects the setting of option \xoption{outdir}.
%
% \subsection{Other image formats}
%
% The support that package \xpackage{epstopdf} implements is not
% limited to EPS files. Other image conversions can be declared.
% The following example shows it for GIF images under Unix with
% ImageMagick's \xprogram{convert}:
%\begin{quote}
%\begin{verbatim}
%\epstopdfDeclareGraphicsRule{.gif}{png}{.png}{%
%  convert #1 \OutputFile
%}
%\end{verbatim}
%\end{quote}
% The file extension \xfile{.gif} can be added to the extension list that
% package \xpackage{graphics} searches if the file extension
% is not given in \cs{includegraphics}.
% The list can be set by \cs{GraphicsExtensions}.
% \begin{quote}
% |\AppendGraphicsExtensions{.gif}|\\
% or\\
% |\PrependGraphicsExtensions{.gif}|
% \end{quote}
%
% \StopEventually{
% }
%
% \section{Implementation}
%
%    \begin{macrocode}
%<*package>
%    \end{macrocode}
%
% \subsection{Wrapper package}
%
%    Reload check, especially if the package is not used with \LaTeX.
%    \begin{macrocode}
\begingroup\catcode61\catcode48\catcode32=10\relax%
  \catcode13=5 % ^^M
  \endlinechar=13 %
  \catcode35=6 % #
  \catcode39=12 % '
  \catcode44=12 % ,
  \catcode45=12 % -
  \catcode46=12 % .
  \catcode58=12 % :
  \catcode64=11 % @
  \catcode123=1 % {
  \catcode125=2 % }
  \expandafter\let\expandafter\x\csname ver@epstopdf.sty\endcsname
  \ifx\x\relax % plain-TeX, first loading
  \else
    \def\empty{}%
    \ifx\x\empty % LaTeX, first loading,
      % variable is initialized, but \ProvidesPackage not yet seen
    \else
      \expandafter\ifx\csname PackageInfo\endcsname\relax
        \def\x#1#2{%
          \immediate\write-1{Package #1 Info: #2.}%
        }%
      \else
        \def\x#1#2{\PackageInfo{#1}{#2, stopped}}%
      \fi
      \x{epstopdf}{The package is already loaded}%
      \aftergroup\endinput
    \fi
  \fi
\endgroup%
%    \end{macrocode}
%    Package identification:
%    \begin{macrocode}
\begingroup\catcode61\catcode48\catcode32=10\relax%
  \catcode13=5 % ^^M
  \endlinechar=13 %
  \catcode35=6 % #
  \catcode39=12 % '
  \catcode40=12 % (
  \catcode41=12 % )
  \catcode44=12 % ,
  \catcode45=12 % -
  \catcode46=12 % .
  \catcode47=12 % /
  \catcode58=12 % :
  \catcode64=11 % @
  \catcode91=12 % [
  \catcode93=12 % ]
  \catcode123=1 % {
  \catcode125=2 % }
  \expandafter\ifx\csname ProvidesPackage\endcsname\relax
    \def\x#1#2#3[#4]{\endgroup
      \immediate\write-1{Package: #3 #4}%
      \xdef#1{#4}%
    }%
  \else
    \def\x#1#2[#3]{\endgroup
      #2[{#3}]%
      \ifx#1\@undefined
        \xdef#1{#3}%
      \fi
      \ifx#1\relax
        \xdef#1{#3}%
      \fi
    }%
  \fi
\expandafter\x\csname ver@epstopdf.sty\endcsname
\ProvidesPackage{epstopdf}%
  [2016/05/15 v2.6 Conversion with epstopdf on the fly (HO)]%
%    \end{macrocode}
%    Larger catcode set because of configuration files needed.
%    \begin{macrocode}
\begingroup\catcode61\catcode48\catcode32=10\relax%
  \catcode13=5 % ^^M
  \endlinechar=13 %
  \catcode123=1 % {
  \catcode125=2 % }
  \catcode64=11 % @
  \def\x{\endgroup
    \expandafter\edef\csname ETE@AtEnd\endcsname{%
      \endlinechar=\the\endlinechar\relax
      \catcode13=\the\catcode13\relax
      \catcode32=\the\catcode32\relax
      \catcode35=\the\catcode35\relax
      \catcode61=\the\catcode61\relax
      \catcode64=\the\catcode64\relax
      \catcode123=\the\catcode123\relax
      \catcode125=\the\catcode125\relax
    }%
  }%
\x\catcode61\catcode48\catcode32=10\relax%
\catcode13=5 % ^^M
\endlinechar=13 %
\catcode35=6 % #
\catcode64=11 % @
\catcode123=1 % {
\catcode125=2 % }
\def\TMP@EnsureCode#1#2{%
  \edef\ETE@AtEnd{%
    \ETE@AtEnd
    \catcode#1=\the\catcode#1\relax
  }%
  \catcode#1=#2\relax
}
\TMP@EnsureCode{39}{12}% '
\TMP@EnsureCode{40}{12}% (
\TMP@EnsureCode{41}{12}% )
\TMP@EnsureCode{42}{12}% *
\TMP@EnsureCode{45}{12}% -
\TMP@EnsureCode{47}{12}% /
\TMP@EnsureCode{91}{12}% [
\TMP@EnsureCode{93}{12}% ]
\edef\ETE@AtEnd{\ETE@AtEnd\noexpand\endinput}
%    \end{macrocode}
%
%    \begin{macrocode}
\let\ETE@SavedAtEnd\ETE@AtEnd
\RequirePackage{epstopdf-base}[2016/05/15]
\let\ETE@AtEnd\ETE@SavedAtEnd
%    \end{macrocode}
%
% \subsubsection{Option handling}
%
%    \begin{macrocode}
\DeclareOption*{%
  \expandafter\epstopdfsetup\expandafter{\CurrentOption}%
}
\ProcessOptions*\relax
%    \end{macrocode}
%
%    \begin{macrocode}
\ETE@AtEnd%
%</package>
%    \end{macrocode}
%
% \subsection{Base package}
%
%    \begin{macrocode}
%<*base>
%    \end{macrocode}
%
% \subsection{Preparations}
%
% \subsubsection{Relead check and identification}
%
%    Reload check, especially if the package is not used with \LaTeX.
%    \begin{macrocode}
\begingroup\catcode61\catcode48\catcode32=10\relax%
  \catcode13=5 % ^^M
  \endlinechar=13 %
  \catcode35=6 % #
  \catcode39=12 % '
  \catcode44=12 % ,
  \catcode45=12 % -
  \catcode46=12 % .
  \catcode58=12 % :
  \catcode64=11 % @
  \catcode123=1 % {
  \catcode125=2 % }
  \expandafter\let\expandafter\x\csname ver@epstopdf-base.sty\endcsname
  \ifx\x\relax % plain-TeX, first loading
  \else
    \def\empty{}%
    \ifx\x\empty % LaTeX, first loading,
      % variable is initialized, but \ProvidesPackage not yet seen
    \else
      \expandafter\ifx\csname PackageInfo\endcsname\relax
        \def\x#1#2{%
          \immediate\write-1{Package #1 Info: #2.}%
        }%
      \else
        \def\x#1#2{\PackageInfo{#1}{#2, stopped}}%
      \fi
      \x{epstopdf-base}{The package is already loaded}%
      \aftergroup\endinput
    \fi
  \fi
\endgroup%
%    \end{macrocode}
%    Package identification:
%    \begin{macrocode}
\begingroup\catcode61\catcode48\catcode32=10\relax%
  \catcode13=5 % ^^M
  \endlinechar=13 %
  \catcode35=6 % #
  \catcode39=12 % '
  \catcode40=12 % (
  \catcode41=12 % )
  \catcode44=12 % ,
  \catcode45=12 % -
  \catcode46=12 % .
  \catcode47=12 % /
  \catcode58=12 % :
  \catcode64=11 % @
  \catcode91=12 % [
  \catcode93=12 % ]
  \catcode123=1 % {
  \catcode125=2 % }
  \expandafter\ifx\csname ProvidesPackage\endcsname\relax
    \def\x#1#2#3[#4]{\endgroup
      \immediate\write-1{Package: #3 #4}%
      \xdef#1{#4}%
    }%
  \else
    \def\x#1#2[#3]{\endgroup
      #2[{#3}]%
      \ifx#1\@undefined
        \xdef#1{#3}%
      \fi
      \ifx#1\relax
        \xdef#1{#3}%
      \fi
    }%
  \fi
\expandafter\x\csname ver@epstopdf-base.sty\endcsname
\ProvidesPackage{epstopdf-base}%
  [2016/05/15 v2.6 Base part for package epstopdf]%
%    \end{macrocode}
%
% \subsubsection{Catcodes}
%
%    \begin{macrocode}
\begingroup\catcode61\catcode48\catcode32=10\relax%
  \catcode13=5 % ^^M
  \endlinechar=13 %
  \catcode123=1 % {
  \catcode125=2 % }
  \catcode64=11 % @
  \def\x{\endgroup
    \expandafter\edef\csname ETE@AtEnd\endcsname{%
      \endlinechar=\the\endlinechar\relax
      \catcode13=\the\catcode13\relax
      \catcode32=\the\catcode32\relax
      \catcode35=\the\catcode35\relax
      \catcode61=\the\catcode61\relax
      \catcode64=\the\catcode64\relax
      \catcode123=\the\catcode123\relax
      \catcode125=\the\catcode125\relax
    }%
  }%
\x\catcode61\catcode48\catcode32=10\relax%
\catcode13=5 % ^^M
\endlinechar=13 %
\catcode35=6 % #
\catcode64=11 % @
\catcode123=1 % {
\catcode125=2 % }
\def\TMP@EnsureCode#1#2{%
  \edef\ETE@AtEnd{%
    \ETE@AtEnd
    \catcode#1=\the\catcode#1\relax
  }%
  \catcode#1=#2\relax
}
\TMP@EnsureCode{33}{12}% !
\TMP@EnsureCode{39}{12}% '
\TMP@EnsureCode{42}{12}% *
\TMP@EnsureCode{44}{12}% ,
\TMP@EnsureCode{45}{12}% -
\TMP@EnsureCode{46}{12}% .
\TMP@EnsureCode{47}{12}% /
\TMP@EnsureCode{58}{12}% :
\TMP@EnsureCode{60}{12}% <
\TMP@EnsureCode{62}{12}% >
\TMP@EnsureCode{96}{12}% `
\edef\ETE@AtEnd{\ETE@AtEnd\noexpand\endinput}
%    \end{macrocode}
%
% \subsubsection{Load packages}
%
%    \begin{macrocode}
\RequirePackage{infwarerr}[2007/09/09]
\RequirePackage{grfext}\relax
\RequirePackage{kvoptions}[2007/10/02]
\RequirePackage{pdftexcmds}[2007/11/11]
%    \end{macrocode}
%
% \subsection{Checks}
%
% Check, whether package graphics is loaded
% (also graphicx loads graphics).
% Because miniltx.tex does not know \cs{@ifpackageloaded}
% we test for \cs{Gin@setfile} instead.
%    \begin{macrocode}
\begingroup\expandafter\expandafter\expandafter\endgroup
\expandafter\ifx\csname Gin@setfile\endcsname\relax
  \@PackageWarningNoLine{epstopdf}{%
    No graphics package \string`graphic{s,x}\string' loaded%
  }%
  \newcommand*{\epstopdfsetup}[1]{}%
  \expandafter\ETE@AtEnd
\fi%
%    \end{macrocode}
%
% Check, whether pdftex.def is loaded. \cs{ver@pdftex.def} is not
% available with miniltx.tex, thus we test for \cs{Gin@driver}.
%    \begin{macrocode}
\begingroup
  \def\x{luatex.def}%
  \ifx\Gin@driver\x
  \else
  \def\x{pdftex.def}%
  \ifx\Gin@driver\x
  \else
    \@PackageWarningNoLine{epstopdf}{%
      Drivers other than `pdftex' and `luatex' are not supported%
    }%
    \endgroup
    \newcommand*{\epstopdfsetup}[1]{}%
    \expandafter\ETE@AtEnd
  \fi%
  \fi%
\endgroup
%    \end{macrocode}
%
% Check, whether the shell escape feature is enabled.
%
%    \begin{macrocode}
\begingroup
  \expandafter\ifx\csname pdf@shellescape\endcsname\relax
  \else
    \ifnum\pdf@shellescape>0 %
    \else
      \@PackageWarningNoLine{epstopdf}{%
        Shell escape feature is not enabled%
      }%
    \fi
  \fi
\endgroup
%    \end{macrocode}
%
% \subsection{Package loading}
%
%    \begin{macrocode}
%    \end{macrocode}
%
% \subsection{Options}
%
%    \begin{macrocode}
\SetupKeyvalOptions{family=ETE,prefix=ETE@}
\DeclareBoolOption{update}
\DeclareBoolOption{verbose}
\newif\ifETE@prepend
\DeclareVoidOption{prepend}{\ETE@prependtrue}
\DeclareVoidOption{append}{\ETE@prependfalse}
\DeclareStringOption{outdir}
\DeclareStringOption{suffix}
\DeclareBoolOption{prefersuffix}
\DeclareStringOption{program@epstopdf}
%    \end{macrocode}
%    Options \xoption{disable} and \xoption{enable} are for testing
%    only. Therefore they are not documented on purpose.
%    \begin{macrocode}
\DeclareBoolOption{disable}
\DeclareComplementaryOption{enable}{disable}
%    \end{macrocode}
%    \begin{macrocode}
\newcommand*{\epstopdfsetup}{\setkeys{ETE}}
%    \end{macrocode}
%
% \subsubsection{Default setting}
%
%    \begin{macrocode}
\epstopdfsetup{%
  verbose,%
  enable,%
  append,%
  update,%
  prefersuffix,%
  suffix=-\SourceExt-converted-to,%
  program@epstopdf=epstopdf%
}
%    \end{macrocode}
%
% \subsection{Make and verbose}
%
%    \begin{macrocode}
\begingroup\expandafter\expandafter\expandafter\endgroup
\expandafter\ifx\csname pdf@filemoddate\endcsname\relax
  \def\ETE@Make#1#2{%
    \ifETE@update
      \ETE@WarnModDate
    \fi
    \@firstofone
  }%
  \def\ETE@WarnModDate{%
    \@PackageWarningNoLine{epstopdf}{%
      \string\pdffilemoddate\space is not available,\MessageBreak
      option `update' will be ignored%
    }%
    \global\let\ETE@WarnModDate\relax
  }%
  \def\ETE@FileInfo#1#2{#1 file: <#2>}%
\else
  \def\ETE@Make#1#2{%
    \ifETE@update
      \ifnum\pdf@strcmp{\pdf@filemoddate{#1}}{\pdf@filemoddate{#2}}>0 %
        \expandafter\expandafter\expandafter\@firstofone
      \else
        \@PackageInfoNoLine{epstopdf}{%
          Output file is already uptodate%
        }%
        \expandafter\expandafter\expandafter\@gobble
      \fi
    \else
      \expandafter\@firstofone
    \fi
  }%
  \def\ETE@FileInfo#1#2{%
    #1 file: <#2>%
    \expandafter\expandafter\expandafter
    \ETE@Date\pdf@filemoddate{#2}\@nil
    \expandafter\expandafter\expandafter
    \ETE@Size\pdf@filesize{#2}\@nil
  }%
  \def\ETE@Date#1\@nil{%
    \ifx\\#1\\%
    \else
      \ETE@@Date#1\@nil
    \fi
  }%
  \def\ETE@@Date#1:#2#3#4#5#6#7#8#9{%
    \MessageBreak
    \@spaces\space\space\space date: #2#3#4#5-#6#7-#8#9 %
    \ETE@@Time
  }%
  \def\ETE@@Time#1#2#3#4#5#6#7\@nil{%
    #1#2:#3#4:#5#6%
  }%
  \def\ETE@Size#1\@nil{%
    \ifx\\#1\\%
    \else
      \MessageBreak
      \@spaces\space\space\space size: #1 bytes%
    \fi
  }%
\fi
%    \end{macrocode}
%
% \subsection{Adding conversion support}
%
%    Patch \cs{Gin@setfile} to execute |#3|, if it contains
%    a command.
%    \begin{macrocode}
\expandafter\ifx\csname ETE@OrgGin@setfile\endcsname\relax
  \let\ETE@OrgGin@setfile\Gin@setfile
\else
  \@PackageError{epstopdf}{%
    Command \string\ETE@OrgGin@setfile\space
    already defined.\MessageBreak
  }{%
    Probably some package has included the code of this package%
    \MessageBreak
    instead of using \string\RequirePackage{epstopdf}.%
    \MessageBreak
    \@ehc
  }%
\fi
\def\ETE@IfFileExists{%
  \begingroup\expandafter\expandafter\expandafter\endgroup
  \expandafter\ifx\csname grffile@IfFileExists\endcsname\relax
    \expandafter\IfFileExists
  \else
    \global\let\ETE@IfFileExists\grffile@IfFileExists
    \expandafter\grffile@IfFileExists
  \fi
}
\def\ETE@Skip#1#2\x\fi\fi{%
  \fi
  \fi
  \endgroup
  \fi
  \fi
  #1%
}
\newif\ifETE@InsideSetfile
\newcommand*{\epstopdfcall}[1]{%
  \ifETE@InsideSetfile
    \expandafter\@firstoftwo
  \else
    \expandafter\@secondoftwo
  \fi
  {`#1}%
  {\Gin@base\Gin@ext}%
}
\def\ETE@DefCommandLine#1{%
  \edef\CommandLine{\expandafter\fi\if`#1}%
}
\def\ETE@DefX#1{%
  \expandafter\expandafter\expandafter\def
  \expandafter\expandafter\expandafter\x
  \expandafter\expandafter\expandafter{%
    \expandafter\fi\if`#1\relax\else
  }%
}
\def\ETE@Gin@setfile#1#2#3{%
  \ifETE@disable
    \ETE@OrgGin@setfile{#1}{#2}{#3}%
  \else
    \begingroup
      \ETE@InsideSetfiletrue
      \ETE@DefX{#3}%
    \expandafter\endgroup
    \ifx\x\@empty
      \ETE@OrgGin@setfile{#1}{#2}{#3}%
    \else
      \begingroup
        \ETE@InsideSetfiletrue
        \def\GraphicsType{#1}%
        \def\GraphicsRead{#2}%
        \ifETE@prefersuffix
        \else
          \ifx\ETE@suffix\@empty
          \else
            \ETE@IfFileExists{\Gin@base\GraphicsRead}{%
              \ETE@Skip{%
                \ETE@OrgGin@setfile{#1}{#2}{\Gin@base#2}%
              }%
            }{%
              \let\next\relax
            }%
            \next
          \fi
        \fi
        \ifx\Gin@ext\relax
          \let\SourceExt\Gin@eext
          \def\SourceFile{\Gin@base\Gin@eext}%
        \else
          \let\SourceExt\Gin@ext
          \def\SourceFile{\Gin@base\Gin@ext}%
        \fi
        \edef\SourceExt{% remove dot
          \expandafter\@cdr\SourceExt\@empty\@nil
        }%
        \let\OutputDirectory\ETE@outdir
        \ifx\OutputDirectory\@empty
          \edef\OutputFile{\ETE@GenerateName{\Gin@base}{#2}}%
        \else
          \begingroup
            \filename@parse{\Gin@base#2}%
            \edef\x{\endgroup
              \def\noexpand\OutputFile{%
                \ETE@GenerateName{%
                  \OutputDirectory\filename@base
                }{#2}%
              }%
            }%
          \x
        \fi
        \ETE@DefCommandLine{#3}%
        \ifETE@verbose
          \@PackageInfo{epstopdf}{%
            \ETE@FileInfo{Source}\SourceFile\MessageBreak
            \ETE@FileInfo{Output}\OutputFile\MessageBreak
            Command: <\CommandLine>\MessageBreak
            \string\includegraphics
          }%
        \fi
        \ETE@Make\SourceFile\OutputFile{%
          \pdf@system{\CommandLine}%
          \ifETE@verbose
            \@PackageInfoNoLine{epstopdf}{%
              \ETE@FileInfo{Result}\OutputFile
            }%
          \fi
        }%
        \edef\x{\endgroup
          \ifx\OutputDirectory\@empty
          \else
            \def\noexpand\Gin@base{%
              \OutputDirectory\noexpand\filename@base
            }%
          \fi
          \ifx\ETE@suffix\@empty
          \else
            \edef\noexpand\Gin@base{%
              \noexpand\Gin@base\ETE@suffix
            }%
          \fi
          \noexpand\ETE@OrgGin@setfile{%
            \GraphicsType
          }{%
            \GraphicsRead
          }{%
            \OutputFile
          }%
        }%
      \x
    \fi
  \fi
}
\let\Gin@setfile\ETE@Gin@setfile
\def\ETE@GenerateName#1#2{%
  #1\ETE@suffix#2%
}
%    \end{macrocode}
%
% \subsection{Declare graphics rule}
%
%    \begin{macrocode}
\newcommand*{\epstopdfDeclareGraphicsRule}[4]{%
  \ifx\\#4\\%
    \@PackageError{epstopdf-base}{%
      Conversion command is missing%
    }\@ehc
  \else
    \begingroup
      \@ifundefined{Gin@rule@#1}{%
      }{%
        \@PackageInfo{epstopdf-base}{%
          Redefining graphics rule for `#1'%
        }%
      }%
    \endgroup
    \@namedef{Gin@rule@#1}##1{{#2}{#3}{\epstopdfcall{#4}}}%
  \fi
}
%    \end{macrocode}
%
% \cs{DeclareGraphicsRule} for .eps
%    \begin{macrocode}
\epstopdfDeclareGraphicsRule{.eps}{pdf}{.pdf}{%
  \ETE@epstopdf{#1}%
}
\def\ETE@epstopdf#1{%
  \ETE@program@epstopdf\space
  \ifcase\ifx\OutputDirectory\@empty
           \ifx\ETE@suffix\@empty
             1%
           \fi
         \fi
         0 %
    --outfile=\OutputFile\space
  \fi
  #1%
}
%    \end{macrocode}
%
%    \begin{macrocode}
\ifETE@prepend
  \expandafter\PrependGraphicsExtensions
\else
  \expandafter\AppendGraphicsExtensions
\fi
{.eps}
\let\ETE@prepend\@undefined
\DeclareVoidOption{prepend}{%
  \PrependGraphicsExtensions{.eps}%
}
\let\ETE@append\@undefined
\DeclareVoidOption{append}{%
  \AppendGraphicsExtensions{.eps}%
}
%    \end{macrocode}
%
%    \begin{macrocode}
\InputIfFileExists{epstopdf-sys.cfg}{}{}
\InputIfFileExists{epstopdf.cfg}{}{}
%    \end{macrocode}
%
%    \begin{macrocode}
\ETE@AtEnd%
%</base>
%    \end{macrocode}
%
% \section{Test}
%
% \subsection{Preface for standard catcode check}
%
%    \begin{macrocode}
%<*test1>
\input miniltx.tex\relax
\def\Gin@driver{pdftex.def}
\input graphicx.sty\relax
\resetatcatcode
%</test1>
%    \end{macrocode}
%
% \subsection{Catcode checks for loading}
%
%    \begin{macrocode}
%<*test1>
%    \end{macrocode}
%    \begin{macrocode}
\catcode`\{=1 %
\catcode`\}=2 %
\catcode`\#=6 %
\catcode`\@=11 %
\expandafter\ifx\csname count@\endcsname\relax
  \countdef\count@=255 %
\fi
\expandafter\ifx\csname @gobble\endcsname\relax
  \long\def\@gobble#1{}%
\fi
\expandafter\ifx\csname @firstofone\endcsname\relax
  \long\def\@firstofone#1{#1}%
\fi
\expandafter\ifx\csname loop\endcsname\relax
  \expandafter\@firstofone
\else
  \expandafter\@gobble
\fi
{%
  \def\loop#1\repeat{%
    \def\body{#1}%
    \iterate
  }%
  \def\iterate{%
    \body
      \let\next\iterate
    \else
      \let\next\relax
    \fi
    \next
  }%
  \let\repeat=\fi
}%
\def\RestoreCatcodes{}
\count@=0 %
\loop
  \edef\RestoreCatcodes{%
    \RestoreCatcodes
    \catcode\the\count@=\the\catcode\count@\relax
  }%
\ifnum\count@<255 %
  \advance\count@ 1 %
\repeat

\def\RangeCatcodeInvalid#1#2{%
  \count@=#1\relax
  \loop
    \catcode\count@=15 %
  \ifnum\count@<#2\relax
    \advance\count@ 1 %
  \repeat
}
\def\RangeCatcodeCheck#1#2#3{%
  \count@=#1\relax
  \loop
    \ifnum#3=\catcode\count@
    \else
      \errmessage{%
        Character \the\count@\space
        with wrong catcode \the\catcode\count@\space
        instead of \number#3%
      }%
    \fi
  \ifnum\count@<#2\relax
    \advance\count@ 1 %
  \repeat
}
\def\space{ }
\expandafter\ifx\csname LoadCommand\endcsname\relax
  \def\LoadCommand{\input epstopdf.sty\relax}%
\fi
\def\Test{%
  \RangeCatcodeInvalid{0}{47}%
  \RangeCatcodeInvalid{58}{64}%
  \RangeCatcodeInvalid{91}{96}%
  \RangeCatcodeInvalid{123}{255}%
  \catcode`\@=12 %
  \catcode`\\=0 %
  \catcode`\%=14 %
  \LoadCommand
  \RangeCatcodeCheck{0}{36}{15}%
  \RangeCatcodeCheck{37}{37}{14}%
  \RangeCatcodeCheck{38}{47}{15}%
  \RangeCatcodeCheck{48}{57}{12}%
  \RangeCatcodeCheck{58}{63}{15}%
  \RangeCatcodeCheck{64}{64}{12}%
  \RangeCatcodeCheck{65}{90}{11}%
  \RangeCatcodeCheck{91}{91}{15}%
  \RangeCatcodeCheck{92}{92}{0}%
  \RangeCatcodeCheck{93}{96}{15}%
  \RangeCatcodeCheck{97}{122}{11}%
  \RangeCatcodeCheck{123}{255}{15}%
  \RestoreCatcodes
}
\Test
\csname @@end\endcsname
\end
%    \end{macrocode}
%    \begin{macrocode}
%</test1>
%    \end{macrocode}
%
% \section{Installation}
%
% \subsection{Download}
%
% \paragraph{Package.} This package is available on
% CTAN\footnote{\url{http://ctan.org/pkg/epstopdf}}:
% \begin{description}
% \item[\CTAN{macros/latex/contrib/oberdiek/epstopdf.dtx}] The source file.
% \item[\CTAN{macros/latex/contrib/oberdiek/epstopdf.pdf}] Documentation.
% \end{description}
%
%
% \paragraph{Bundle.} All the packages of the bundle `oberdiek'
% are also available in a TDS compliant ZIP archive. There
% the packages are already unpacked and the documentation files
% are generated. The files and directories obey the TDS standard.
% \begin{description}
% \item[\CTAN{install/macros/latex/contrib/oberdiek.tds.zip}]
% \end{description}
% \emph{TDS} refers to the standard ``A Directory Structure
% for \TeX\ Files'' (\CTAN{tds/tds.pdf}). Directories
% with \xfile{texmf} in their name are usually organized this way.
%
% \subsection{Bundle installation}
%
% \paragraph{Unpacking.} Unpack the \xfile{oberdiek.tds.zip} in the
% TDS tree (also known as \xfile{texmf} tree) of your choice.
% Example (linux):
% \begin{quote}
%   |unzip oberdiek.tds.zip -d ~/texmf|
% \end{quote}
%
% \paragraph{Script installation.}
% Check the directory \xfile{TDS:scripts/oberdiek/} for
% scripts that need further installation steps.
% Package \xpackage{attachfile2} comes with the Perl script
% \xfile{pdfatfi.pl} that should be installed in such a way
% that it can be called as \texttt{pdfatfi}.
% Example (linux):
% \begin{quote}
%   |chmod +x scripts/oberdiek/pdfatfi.pl|\\
%   |cp scripts/oberdiek/pdfatfi.pl /usr/local/bin/|
% \end{quote}
%
% \subsection{Package installation}
%
% \paragraph{Unpacking.} The \xfile{.dtx} file is a self-extracting
% \docstrip\ archive. The files are extracted by running the
% \xfile{.dtx} through \plainTeX:
% \begin{quote}
%   \verb|tex epstopdf.dtx|
% \end{quote}
%
% \paragraph{TDS.} Now the different files must be moved into
% the different directories in your installation TDS tree
% (also known as \xfile{texmf} tree):
% \begin{quote}
% \def\t{^^A
% \begin{tabular}{@{}>{\ttfamily}l@{ $\rightarrow$ }>{\ttfamily}l@{}}
%   epstopdf.sty & tex/latex/oberdiek/epstopdf.sty\\
%   epstopdf-base.sty & tex/latex/oberdiek/epstopdf-base.sty\\
%   epstopdf.pdf & doc/latex/oberdiek/epstopdf.pdf\\
%   test/epstopdf-test1.tex & doc/latex/oberdiek/test/epstopdf-test1.tex\\
%   epstopdf.dtx & source/latex/oberdiek/epstopdf.dtx\\
% \end{tabular}^^A
% }^^A
% \sbox0{\t}^^A
% \ifdim\wd0>\linewidth
%   \begingroup
%     \advance\linewidth by\leftmargin
%     \advance\linewidth by\rightmargin
%   \edef\x{\endgroup
%     \def\noexpand\lw{\the\linewidth}^^A
%   }\x
%   \def\lwbox{^^A
%     \leavevmode
%     \hbox to \linewidth{^^A
%       \kern-\leftmargin\relax
%       \hss
%       \usebox0
%       \hss
%       \kern-\rightmargin\relax
%     }^^A
%   }^^A
%   \ifdim\wd0>\lw
%     \sbox0{\small\t}^^A
%     \ifdim\wd0>\linewidth
%       \ifdim\wd0>\lw
%         \sbox0{\footnotesize\t}^^A
%         \ifdim\wd0>\linewidth
%           \ifdim\wd0>\lw
%             \sbox0{\scriptsize\t}^^A
%             \ifdim\wd0>\linewidth
%               \ifdim\wd0>\lw
%                 \sbox0{\tiny\t}^^A
%                 \ifdim\wd0>\linewidth
%                   \lwbox
%                 \else
%                   \usebox0
%                 \fi
%               \else
%                 \lwbox
%               \fi
%             \else
%               \usebox0
%             \fi
%           \else
%             \lwbox
%           \fi
%         \else
%           \usebox0
%         \fi
%       \else
%         \lwbox
%       \fi
%     \else
%       \usebox0
%     \fi
%   \else
%     \lwbox
%   \fi
% \else
%   \usebox0
% \fi
% \end{quote}
% If you have a \xfile{docstrip.cfg} that configures and enables \docstrip's
% TDS installing feature, then some files can already be in the right
% place, see the documentation of \docstrip.
%
% \subsection{Refresh file name databases}
%
% If your \TeX~distribution
% (\teTeX, \mikTeX, \dots) relies on file name databases, you must refresh
% these. For example, \teTeX\ users run \verb|texhash| or
% \verb|mktexlsr|.
%
% \subsection{Some details for the interested}
%
% \paragraph{Attached source.}
%
% The PDF documentation on CTAN also includes the
% \xfile{.dtx} source file. It can be extracted by
% AcrobatReader 6 or higher. Another option is \textsf{pdftk},
% e.g. unpack the file into the current directory:
% \begin{quote}
%   \verb|pdftk epstopdf.pdf unpack_files output .|
% \end{quote}
%
% \paragraph{Unpacking with \LaTeX.}
% The \xfile{.dtx} chooses its action depending on the format:
% \begin{description}
% \item[\plainTeX:] Run \docstrip\ and extract the files.
% \item[\LaTeX:] Generate the documentation.
% \end{description}
% If you insist on using \LaTeX\ for \docstrip\ (really,
% \docstrip\ does not need \LaTeX), then inform the autodetect routine
% about your intention:
% \begin{quote}
%   \verb|latex \let\install=y\input{epstopdf.dtx}|
% \end{quote}
% Do not forget to quote the argument according to the demands
% of your shell.
%
% \paragraph{Generating the documentation.}
% You can use both the \xfile{.dtx} or the \xfile{.drv} to generate
% the documentation. The process can be configured by the
% configuration file \xfile{ltxdoc.cfg}. For instance, put this
% line into this file, if you want to have A4 as paper format:
% \begin{quote}
%   \verb|\PassOptionsToClass{a4paper}{article}|
% \end{quote}
% An example follows how to generate the
% documentation with pdf\LaTeX:
% \begin{quote}
%\begin{verbatim}
%pdflatex epstopdf.dtx
%makeindex -s gind.ist epstopdf.idx
%pdflatex epstopdf.dtx
%makeindex -s gind.ist epstopdf.idx
%pdflatex epstopdf.dtx
%\end{verbatim}
% \end{quote}
%
% \section{Catalogue}
%
% The following XML file can be used as source for the
% \href{http://mirror.ctan.org/help/Catalogue/catalogue.html}{\TeX\ Catalogue}.
% The elements \texttt{caption} and \texttt{description} are imported
% from the original XML file from the Catalogue.
% The name of the XML file in the Catalogue is \xfile{epstopdf-pkg.xml}.
%    \begin{macrocode}
%<*catalogue>
<?xml version='1.0' encoding='us-ascii'?>
<!DOCTYPE entry SYSTEM 'catalogue.dtd'>
<entry datestamp='$Date$' modifier='$Author$' id='epstopdf-pkg'>
  <name>epstopdf-pkg</name>
  <caption>Call epstopdf "on the fly"</caption>
  <authorref id='auth:oberdiek'/>
  <copyright owner='Heiko Oberdiek' year='2001,2006-2010'/>
  <license type='lppl1.3'/>
  <version number='2.6'/>
  <description>
    The package adds support for EPS files in the
    <xref refid='graphicx'>graphicx</xref> package when running under
    PDFTeX.  If an EPS graphic is detected, the package spawns a
    process to convert the EPS to PDF, using the script
    <xref refid='epstopdf'>epstopdf</xref>.  This of course requires
    that shell escape is enabled for the PDFTeX run.
    <p/>
    The package is part of the <xref refid='oberdiek'>oberdiek</xref>
    bundle.
  </description>
  <documentation details='Package documentation'
      href='ctan:/macros/latex/contrib/oberdiek/epstopdf.pdf'/>
  <ctan file='true' path='/macros/latex/contrib/oberdiek/epstopdf.dtx'/>
  <miktex location='oberdiek'/>
  <texlive location='oberdiek'/>
  <install path='/macros/latex/contrib/oberdiek/oberdiek.tds.zip'/>
</entry>
%</catalogue>
%    \end{macrocode}
%
% \begin{History}
%   \begin{Version}{2001/01/06 v1.0}
%   \item
%     First public version,
%     published in the \pdfTeX\ mailing list.
%   \end{Version}
%   \begin{Version}{2001/02/04 v1.1}
%   \item
%     Minor documentation update.
%   \item
%     CTAN.
%   \end{Version}
%   \begin{Version}{2006/02/20 v1.2}
%   \item
%     DTX framework.
%   \item
%     Compatibility for \xfile{miniltx.tex}.
%   \end{Version}
%   \begin{Version}{2006/08/26 v1.3}
%   \item
%     Check for \cs{write18} if available and print a warning
%     if the feature is not enabled.
%   \end{Version}
%   \begin{Version}{2007/04/26 v1.4}
%   \item
%     Documentation rewritten and extended.
%   \end{Version}
%   \begin{Version}{2007/10/02 v1.5}
%   \item
%     New option \xoption{update}: If the converted file exists, it will be only
%     converted if it is out of date.
%   \item
%     Updating the extension list is delegated to package \xpackage{grfext}.
%     Fine tuning is done by the new options \xoption{append}, \xoption{prepend}.
%   \item
%     New option \xoption{outdir} for changing the output directory.
%   \item
%     New option \xoption{verbose}.
%   \item
%     \cs{SourceFile} and \cs{OutputFile} introduced.
%   \item
%     Configuration file support added.
%   \end{Version}
%   \begin{Version}{2007/11/11 v1.6}
%   \item
%     Use of package \xpackage{pdftexcmds} for \LuaTeX\ support.
%   \end{Version}
%   \begin{Version}{2008/05/06 v1.7}
%   \item
%     Warning messages uses ``loaded'' instead of ``found''.
%   \end{Version}
%   \begin{Version}{2009/03/01 v1.8}
%   \item
%     Warning message for missing \xfile{pdftex.def} changed.
%   \end{Version}
%   \begin{Version}{2009/07/06 v1.9}
%   \item
%     Option \xoption{suffix} added.
%   \end{Version}
%   \begin{Version}{2009/07/07 v1.10}
%   \item
%     \cs{SourceExt} added.
%   \item
%     If option \xoption{suffix} is set, the inclusion of
%     an image without the suffix namespace is preferred over generating the
%     the image within the suffix namespace.
%   \end{Version}
%   \begin{Version}{2009/07/12 v2.0}
%   \item
%     New default settings.
%   \item
%     Package is split into \xpackage{epstopdf} that only takes package
%     options and loads \xpackage{epstopdf-base} that does the work.
%   \item
%     \cs{epstopdfDeclareGraphicsRule} and \cs{epstopdfcall} added.
%   \item
%     \xfile{epstopdf-sys.cfg} is loaded before \xfile{epstopdf.cfg}
%     if \xfile{epstopdf-sys.cfg} exists.
%   \end{Version}
%   \begin{Version}{2009/07/15 v2.1}
%   \item
%     Default setting: \xoption{verbose} is now turned on as the
%     documentation for v2.0 said.
%   \item
%     Documentation fixes.
%   \end{Version}
%   \begin{Version}{2009/07/16 v2.2}
%   \item
%     Fixed redefined \cs{Gin@setfile}.
%   \item
%     Documentation extended for package options.
%   \end{Version}
%   \begin{Version}{2009/09/24 v2.3}
%   \item
%     Bug fix for the case that both option \xoption{suffix} and
%     \xoption{outdir} are used.
%   \end{Version}
%   \begin{Version}{2009/10/17 v2.4}
%   \item
%     The name of the program `epstopdf' can be configured via
%     the new option \xoption{program@epstopdf}.
%   \end{Version}
%   \begin{Version}{2016/05/15 v2.5}
%   \item
%     Wording of warning message fixed (Karl Berry).
%   \item
%     \cs{ETE@Gin@setfile} added (Karl Berry).
%   \end{Version}
%   \begin{Version}{2016/05/15 v2.6}
%   \item
%     luaTeX compatibility
%   \end{Version}
% \end{History}
%
% \PrintIndex
%
% \Finale
\endinput

%        (quote the arguments according to the demands of your shell)
%
% Documentation:
%    (a) If epstopdf.drv is present:
%           latex epstopdf.drv
%    (b) Without epstopdf.drv:
%           latex epstopdf.dtx; ...
%    The class ltxdoc loads the configuration file ltxdoc.cfg
%    if available. Here you can specify further options, e.g.
%    use A4 as paper format:
%       \PassOptionsToClass{a4paper}{article}
%
%    Programm calls to get the documentation (example):
%       pdflatex epstopdf.dtx
%       makeindex -s gind.ist epstopdf.idx
%       pdflatex epstopdf.dtx
%       makeindex -s gind.ist epstopdf.idx
%       pdflatex epstopdf.dtx
%
% Installation:
%    TDS:tex/latex/oberdiek/epstopdf.sty
%    TDS:tex/latex/oberdiek/epstopdf-base.sty
%    TDS:doc/latex/oberdiek/epstopdf.pdf
%    TDS:doc/latex/oberdiek/test/epstopdf-test1.tex
%    TDS:source/latex/oberdiek/epstopdf.dtx
%
%<*ignore>
\begingroup
  \catcode123=1 %
  \catcode125=2 %
  \def\x{LaTeX2e}%
\expandafter\endgroup
\ifcase 0\ifx\install y1\fi\expandafter
         \ifx\csname processbatchFile\endcsname\relax\else1\fi
         \ifx\fmtname\x\else 1\fi\relax
\else\csname fi\endcsname
%</ignore>
%<*install>
\input docstrip.tex
\Msg{************************************************************************}
\Msg{* Installation}
\Msg{* Package: epstopdf 2016/05/15 v2.6 Conversion with epstopdf on the fly (HO)}
\Msg{************************************************************************}

\keepsilent
\askforoverwritefalse

\let\MetaPrefix\relax
\preamble

This is a generated file.

Project: epstopdf
Version: 2016/05/15 v2.6

Copyright (C) 2001, 2006-2010 by
   Heiko Oberdiek <heiko.oberdiek at googlemail.com>

This work may be distributed and/or modified under the
conditions of the LaTeX Project Public License, either
version 1.3c of this license or (at your option) any later
version. This version of this license is in
   http://www.latex-project.org/lppl/lppl-1-3c.txt
and the latest version of this license is in
   http://www.latex-project.org/lppl.txt
and version 1.3 or later is part of all distributions of
LaTeX version 2005/12/01 or later.

This work has the LPPL maintenance status "maintained".

This Current Maintainer of this work is Heiko Oberdiek.

This work consists of the main source file epstopdf.dtx
and the derived files
   epstopdf.sty, epstopdf.pdf, epstopdf.ins, epstopdf.drv,
   epstopdf-base.sty, epstopdf-test1.tex.

\endpreamble
\let\MetaPrefix\DoubleperCent

\generate{%
  \file{epstopdf.ins}{\from{epstopdf.dtx}{install}}%
  \file{epstopdf.drv}{\from{epstopdf.dtx}{driver}}%
  \usedir{tex/latex/oberdiek}%
  \file{epstopdf.sty}{\from{epstopdf.dtx}{package}}%
  \file{epstopdf-base.sty}{\from{epstopdf.dtx}{base}}%
  \usedir{doc/latex/oberdiek/test}%
  \file{epstopdf-test1.tex}{\from{epstopdf.dtx}{test1}}%
  \nopreamble
  \nopostamble
  \usedir{source/latex/oberdiek/catalogue}%
  \file{epstopdf-pkg.xml}{\from{epstopdf.dtx}{catalogue}}%
}

\catcode32=13\relax% active space
\let =\space%
\Msg{************************************************************************}
\Msg{*}
\Msg{* To finish the installation you have to move the following}
\Msg{* files into a directory searched by TeX:}
\Msg{*}
\Msg{*     epstopdf.sty, epstopdf-base.sty}
\Msg{*}
\Msg{* To produce the documentation run the file `epstopdf.drv'}
\Msg{* through LaTeX.}
\Msg{*}
\Msg{* Happy TeXing!}
\Msg{*}
\Msg{************************************************************************}

\endbatchfile
%</install>
%<*ignore>
\fi
%</ignore>
%<*driver>
\NeedsTeXFormat{LaTeX2e}
\ProvidesFile{epstopdf.drv}%
  [2016/05/15 v2.6 Conversion with epstopdf on the fly (HO)]%
\documentclass{ltxdoc}
\usepackage{holtxdoc}[2011/11/22]
\begin{document}
  \DocInput{epstopdf.dtx}%
\end{document}
%</driver>
% \fi
%
%
% \CharacterTable
%  {Upper-case    \A\B\C\D\E\F\G\H\I\J\K\L\M\N\O\P\Q\R\S\T\U\V\W\X\Y\Z
%   Lower-case    \a\b\c\d\e\f\g\h\i\j\k\l\m\n\o\p\q\r\s\t\u\v\w\x\y\z
%   Digits        \0\1\2\3\4\5\6\7\8\9
%   Exclamation   \!     Double quote  \"     Hash (number) \#
%   Dollar        \$     Percent       \%     Ampersand     \&
%   Acute accent  \'     Left paren    \(     Right paren   \)
%   Asterisk      \*     Plus          \+     Comma         \,
%   Minus         \-     Point         \.     Solidus       \/
%   Colon         \:     Semicolon     \;     Less than     \<
%   Equals        \=     Greater than  \>     Question mark \?
%   Commercial at \@     Left bracket  \[     Backslash     \\
%   Right bracket \]     Circumflex    \^     Underscore    \_
%   Grave accent  \`     Left brace    \{     Vertical bar  \|
%   Right brace   \}     Tilde         \~}
%
% \GetFileInfo{epstopdf.drv}
%
% \title{The \xpackage{epstopdf} package}
% \date{2016/05/15 v2.6}
% \author{Heiko Oberdiek\thanks
% {Please report any issues at https://github.com/ho-tex/oberdiek/issues}\\
% \xemail{heiko.oberdiek at googlemail.com}}
%
% \maketitle
%
% \begin{abstract}
% This packages adds support of handling eps images
% to package \xpackage{graphics} or \xpackage{graphicx}
% with option \xoption{pdftex}. If an eps image is detected,
% epstopdf is automatically called to convert it to pdf format.
% \end{abstract}
%
% \tableofcontents
%
% \providecommand*{\xprogram}[1]{\textsf{#1}}
%
% \section{Documentation}
%
% \subsection{Introduction}
%
% \LaTeX\ provides its graphics bundle to include graphics files.
% Both packages \xpackage{graphics} or \xpackage{graphicx} may
% be used. the latter one loads the first and adds options
% in key value style for \cs{includegraphics}.
%
% Usually the drivers do not support all kind of graphics files.
% Other image types must be converted, before they become
% usuable. In case of driver \xprogram{dvips}, the graphics
% rule may contain a conversion rule. Then all that package
% \xpackage{graphics} must know is the bounding box, the command
% is passed to \xprogram{dvips} that calls it and embeds the
% converted image.
%
% However, \pdfTeX\ has its driver for PDF output already
% build in. It's graphics inclusion commands (\cs{pdfximage})
% does not allow the execution of external commands.
% Therefore commands in the last argument of
% \cs{DeclareGraphicsRule} were of no use.
% But external programs can be called within \pdfTeX.
% This feature is called ``shell escape'' or ``write 18''
% and must usually enabled explicitely because of security reasons.
% Now, this package \xpackage{epstopdf} hooks
% into package \xpackage{graphics}' code to catch
% that argument with the external command and executes it
% to convert the graphics file to a supported format and
% passes the control of graphics inclusion back to package
% \xpackage{graphics}.
%
% \subsection{Requirements}
%
% \begin{itemize}
% \item
%   The feature \cs{write18} must be enabled. This allows the
%   running of external programs during \TeX's compile run.
%   Keep in mind that this is a security risk. The feature is an
%   addition to \cs{TeX}. Mik\TeX, te\TeX, \TeX\ Live support it.
%   In Web2C based \TeX\ distributions (te\TeX, \TeX\ Live) it can be
%   enabled in the configuration file \xfile{texmf.cnf}:
%   \begin{quote}
%     |shell_escape = 1|
%   \end{quote}
%   Because of the security risk, it is better to do it on the
%   command line only:
%   \begin{quote}
%   \texttt{--shell-escape} (te\TeX, \TeX\ Live)\\
%   \texttt{--enable-write18} (MiK\TeX)
%   \end{quote}
%   Example:
%   \begin{quote}
%     |pdflatex -shell-escape test.tex|
%   \end{quote}
% \item
%   The program \xprogram{epstopdf} for the conversion from EPS
%   to PDF. However, other programs can
%   be used and configured by \cs{DeclareGraphicsRule}.
%   Example:
%   \begin{quote}
%   |\epstopdfDeclareGraphicsRule{.eps}{pdf}{.pdf}{%|\\
%   |  ps2pdf -dEPSCrop #1 \OutputFile|\\
%   |}|
%   \end{quote}
% \end{itemize}
%
% \subsection{Usage}
%
% The package is loaded after graphic\{s,x\}, e.g.:
% \begin{quote}
%   |\usepackage[pdftex]{graphicx}|\\
%   |\usepackage{epstopdf}|
% \end{quote}
% Now images with file name extension \xfile{.eps} are detected
% and supported using \cs{includegraphics}.
%
% If the graphics file name is explicitly specified with extension
% \xfile{.eps} the new rule for EPS files is called and
% the conversion performed. If option \xoption{update} is in force
% then the conversion step is dropped if the target file already
% exists and is not older then the EPS file.
%
% The situation is more complicate if the graphics file is
% given without file name extension. Then the \xpackage{graphics}
% package must search for a supported image file.
% The possible extensions are stored in the graphics extension
% list, that can be set by \cs{DeclareGraphicsExtensions}.
% The algorithm:
% \begin{quote}
%    \def\>{\mbox{}\quad}
%    function search( \meta{filebase} )\\
%    \>foreach \meta{ext} in \meta{graphics extensions}\\
%    \>\>foreach \meta{dir} in \meta{current directory}, \meta{\cs{graphicspath}}\\
%    \>\>\>\meta{file} := \meta{dir} + \meta{filebase} + \meta{ext}\\
%    \>\>\>if exist \meta{file}\\
%    \>\>\>\>return found\\
%    \>return not found
% \end{quote}
% Package \xpackage{epstopdf} puts \xfile{.eps} at the end of the
% graphics extension search list. This is the behaviour of option
% \xoption{append} that is enabled by default.
% That means, the conversion is called last
% unless a supported file type cannot be found earlier. This avoids
% unnecessary conversion steps that slow down the \LaTeX\ run.
% If you want to use option \xoption{update} and your \pdfTeX\ supports
% it, then an outdated PDF file also would be found earlier unless
% \xoption{suffix} is used that is the default since version 2.0.
%
% With an empty option \xoption{suffix} and option
% \xoption{prepend} there is a risk that an original PDF file is overwritten:
%
% If the original image file is the PDF file and there is also a generated
% EPS file, then the original PDF file can be regenerated (depending
% on the option settings) and the original PDF file gets lost.
% Therefore option \xoption{suffix} is introduced in version 1.9
% to create a separate name space for generated output files.
%
% \textbf{Note:} Usually the conversion program needs the exact location
% of the image file. Usually the current directory works.
% Also if the image file is found using \cs{graphicspath}, the
% location is known. However, if the image is somewhere
% in a directory of environment variable |TEXINPUTS|, then
% the package does not know the exact location and the
% conversion program will not find the image file unless it
% implements a search using |TEXINPUTS| (program \xprogram{kpsewhich}
% may be of help in this task).
%
% \subsection{Options}
%
% Options can be given as package options or later using:
% \begin{declcs}{epstopdfsetup} \M{key value list}
% \end{declcs}
% \LaTeX\ expands the option list before passing the option list
% to the package's option handling code. This can fail
% for option \xoption{suffix} if it contains some of the macros
% described below. Use \cs{epstopdfsetup} after the package is loaded.
% Or load package \xpackage{kvoptions-patch} before. This package
% is also loaded by option \xoption{patch} of package \xpackage{kvoptions}.
% \LaTeX's option code is redefined to respect key value options and
% let the values untouched.
%
% \begin{description}
% \def\It#1{\item[\xoption{#1}:]}^^A
% \It{update}
%   The conversion program is only called, if the target file
%   does not exist or is older than the source image file.
% \It{append}
%   Puts the extension \xfile{.eps} at the end of the graphics extension list
%   (default).
% \It{prepend}
%   Puts the extension \xfile{.eps} at the begin of the graphics extension list.
% \It{outdir}
%   The converted file may put in an other output directory.
%   The value of \xoption{outdir} must include the directory separator.
%   Example for the current directory:
%   \begin{quote}
%   |\epstopdfsetup{outdir=./}|
%   \end{quote}
%   For other directories ensure, that they can be found.
%   See \cs{graphicspath} or |TEXINPUTS|.
% \It{suffix}
%   This option takes a string that is put between the file name base
%   and the extension of the output file. Rationale: It can happen,
%   that a PDF file is the original file and the EPS file the
%   generated file. If now the package thinks, that the PDF file
%   is the generated file, it will `regnerate' it. But in reality
%   the original file is lost. Therefore I recommend to use this
%   option always to generate a separate name space for generated
%   files. Proposed value is \texttt{-generated} or \texttt{.generated}.
%   The suffix \texttt{.generated} will also work here without the need for
%   package \xpackage{grffile}).
%
%   Example:
%   \begin{quote}
%   |\epstopdfsetup{suffix=-generated}|\\
%   Then \xfile{foo.eps} is converted to \xfile{foo-generated.pdf}.
%   \end{quote}
%   \cs{SourceExt} can be used inside the suffix string.
%   It's will be replaced by the extension of the image source
%   file without the leading dot, for instance:
%   \begin{quote}
%   |\epstopdfsetup{suffix=-\SourceExt-converted-to}|\\
%   \xfile{foo.eps} $\Rightarrow$ \xfile{foo-eps-converted-to.pdf}
%   \end{quote}
%   See also the next option \xoption{prefersuffix} that
%   modifies the behaviour of option \xoption{suffix} in some
%   cases.
%
%   Default for \xoption{suffix} is `|-\SourceExt-converted-to|'.
%
% \It{prefersuffix}
%   If a suffix is set by option \xoption{suffix}, then there can be
%   two image file names that could be taken into account for inclusion:
%   A image file name with the suffix string inside
%   its name and a image file name without; e.g. for \xfile{foo.eps}
%   the names could be:
%   \begin{quote}
%   \xfile{foo-suffix.pdf}, \xfile{foo.pdf}
%   \end{quote}
%   If option \xoption{perfersuffix} is turned on, the
%   file \xfile{foo-suffix.pdf} and its generation is preferred
%   over using \xfile{foo.pdf}. Otherwise \xfile{foo.pdf} is included
%   without generating \xfile{foo-suffix.pdf}.
%   The default of option \xoption{prefersuffix} is \texttt{true}.
% \It{program@epstopdf}
%   The name for the conversion program from EPS to PDF,
%   default is `|epstopdf|'.
% \It{verbose}
%   It prints some information about the image in the \xfile{.log} file
%   (default).
% \end{description}
%
% \subsection{Configuration}
%
% \subsubsection{System configuration file \xfile{epstopdf-sys.cfg}}
%
% If \xfile{epstopdf-sys} exists it is loaded at the end of the
% package \xpackage{epstopdf-base} and before the user configuration
% file. It's intended for TeX distributors. Thus they could add
% additional conversion rules (e.g., .gif -> .png) or set options.
%
% \subsubsection{User configuration file \xfile{epstopdf.cfg}}
%
% A configuration file \xfile{epstopdf.cfg} is loaded at the end
% of the package if it exists. It can be used for changing
% the default option setting. Example:
% \begin{quote}
% |\epstopdfsetup{verbose=false}|
% \end{quote}
%
% \subsubsection{Conversion program}
%
% You can use \cs{DeclareGraphicsRule} in a similar way as the
% route via \xprogram{dvips} to specify the conversion
% command line.
% The conversion argument starts with a back tick, followed
% by the conversion command including parameters.
%
% The whole conversion argument should also be wrapped inside
% \cs{epstopdfcall}. This reduces the problem with packages
% (e.g. \xpackage{pst-pdf}) that use the conversion argument
% and expands it. Macros \cs{SourceFile}, \cs{OutputFile}, and
% \cs{SourceExt} are not defined outside \xpackage{epstopdf-base}'s
% \cs{Gin@setfile} and error messages because of undefined
% command names are the result. If \cs{epstopdfcall} detects that
% it is called outside \xpackage{epstopdf-base}'s \cs{Gin@setfile}
% then it replaces the conversion argument by
% package \xpackage{graphics}'s
% default, usually the image file.
%
% The following macros are available inside:
% \begin{description}
% \def\It#1{\item[\cs{#1}:]}
% \It{OutputFile}: output file name (with known path and extension)
% \It{SourceFile}: source file name (with known path and extension),
%                  usually the same as |#1|,
% \It{SourceExt}: source extension without leading dot.
% \end{description}
%
% \paragraph{Conversion from EPS to PDF.}
% Other programs than \xprogram{epstopdf} can be used to convert from
% EPS to PDF. Example that uses \xprogram{Ghostscript}:
% \begin{quote}
%   |\DeclareGraphicsRule{.eps}{pdf}{.pdf}{%|\\
%   |  \epstopdfcall{`ps2pdf -dEPSCrop #1 \noexpand\OutputFile}%|\\
%   |}|
% \end{quote}
% \cs{DeclareGraphicsRule} expands the argument, therefore \cs{noexpand}
% is necessary. As convenience package \xpackage{epstopdf-base} defines
% \cs{epstopdfDeclareGraphicsRule}. Then the conversion argument
% is not expanded, \cs{epstopdfcall} and the back tick are added:
% \begin{quote}
%   |\epstopdfDeclareGraphicsRule{.eps}{pdf}{.pdf}{%|\\
%   |  ps2pdf -dEPSCrop #1 \OutputFile|\\
%   |}|
% \end{quote}
%
% Also \cs{OutputFile} respects the setting of option \xoption{outdir}.
%
% \subsection{Other image formats}
%
% The support that package \xpackage{epstopdf} implements is not
% limited to EPS files. Other image conversions can be declared.
% The following example shows it for GIF images under Unix with
% ImageMagick's \xprogram{convert}:
%\begin{quote}
%\begin{verbatim}
%\epstopdfDeclareGraphicsRule{.gif}{png}{.png}{%
%  convert #1 \OutputFile
%}
%\end{verbatim}
%\end{quote}
% The file extension \xfile{.gif} can be added to the extension list that
% package \xpackage{graphics} searches if the file extension
% is not given in \cs{includegraphics}.
% The list can be set by \cs{GraphicsExtensions}.
% \begin{quote}
% |\AppendGraphicsExtensions{.gif}|\\
% or\\
% |\PrependGraphicsExtensions{.gif}|
% \end{quote}
%
% \StopEventually{
% }
%
% \section{Implementation}
%
%    \begin{macrocode}
%<*package>
%    \end{macrocode}
%
% \subsection{Wrapper package}
%
%    Reload check, especially if the package is not used with \LaTeX.
%    \begin{macrocode}
\begingroup\catcode61\catcode48\catcode32=10\relax%
  \catcode13=5 % ^^M
  \endlinechar=13 %
  \catcode35=6 % #
  \catcode39=12 % '
  \catcode44=12 % ,
  \catcode45=12 % -
  \catcode46=12 % .
  \catcode58=12 % :
  \catcode64=11 % @
  \catcode123=1 % {
  \catcode125=2 % }
  \expandafter\let\expandafter\x\csname ver@epstopdf.sty\endcsname
  \ifx\x\relax % plain-TeX, first loading
  \else
    \def\empty{}%
    \ifx\x\empty % LaTeX, first loading,
      % variable is initialized, but \ProvidesPackage not yet seen
    \else
      \expandafter\ifx\csname PackageInfo\endcsname\relax
        \def\x#1#2{%
          \immediate\write-1{Package #1 Info: #2.}%
        }%
      \else
        \def\x#1#2{\PackageInfo{#1}{#2, stopped}}%
      \fi
      \x{epstopdf}{The package is already loaded}%
      \aftergroup\endinput
    \fi
  \fi
\endgroup%
%    \end{macrocode}
%    Package identification:
%    \begin{macrocode}
\begingroup\catcode61\catcode48\catcode32=10\relax%
  \catcode13=5 % ^^M
  \endlinechar=13 %
  \catcode35=6 % #
  \catcode39=12 % '
  \catcode40=12 % (
  \catcode41=12 % )
  \catcode44=12 % ,
  \catcode45=12 % -
  \catcode46=12 % .
  \catcode47=12 % /
  \catcode58=12 % :
  \catcode64=11 % @
  \catcode91=12 % [
  \catcode93=12 % ]
  \catcode123=1 % {
  \catcode125=2 % }
  \expandafter\ifx\csname ProvidesPackage\endcsname\relax
    \def\x#1#2#3[#4]{\endgroup
      \immediate\write-1{Package: #3 #4}%
      \xdef#1{#4}%
    }%
  \else
    \def\x#1#2[#3]{\endgroup
      #2[{#3}]%
      \ifx#1\@undefined
        \xdef#1{#3}%
      \fi
      \ifx#1\relax
        \xdef#1{#3}%
      \fi
    }%
  \fi
\expandafter\x\csname ver@epstopdf.sty\endcsname
\ProvidesPackage{epstopdf}%
  [2016/05/15 v2.6 Conversion with epstopdf on the fly (HO)]%
%    \end{macrocode}
%    Larger catcode set because of configuration files needed.
%    \begin{macrocode}
\begingroup\catcode61\catcode48\catcode32=10\relax%
  \catcode13=5 % ^^M
  \endlinechar=13 %
  \catcode123=1 % {
  \catcode125=2 % }
  \catcode64=11 % @
  \def\x{\endgroup
    \expandafter\edef\csname ETE@AtEnd\endcsname{%
      \endlinechar=\the\endlinechar\relax
      \catcode13=\the\catcode13\relax
      \catcode32=\the\catcode32\relax
      \catcode35=\the\catcode35\relax
      \catcode61=\the\catcode61\relax
      \catcode64=\the\catcode64\relax
      \catcode123=\the\catcode123\relax
      \catcode125=\the\catcode125\relax
    }%
  }%
\x\catcode61\catcode48\catcode32=10\relax%
\catcode13=5 % ^^M
\endlinechar=13 %
\catcode35=6 % #
\catcode64=11 % @
\catcode123=1 % {
\catcode125=2 % }
\def\TMP@EnsureCode#1#2{%
  \edef\ETE@AtEnd{%
    \ETE@AtEnd
    \catcode#1=\the\catcode#1\relax
  }%
  \catcode#1=#2\relax
}
\TMP@EnsureCode{39}{12}% '
\TMP@EnsureCode{40}{12}% (
\TMP@EnsureCode{41}{12}% )
\TMP@EnsureCode{42}{12}% *
\TMP@EnsureCode{45}{12}% -
\TMP@EnsureCode{47}{12}% /
\TMP@EnsureCode{91}{12}% [
\TMP@EnsureCode{93}{12}% ]
\edef\ETE@AtEnd{\ETE@AtEnd\noexpand\endinput}
%    \end{macrocode}
%
%    \begin{macrocode}
\let\ETE@SavedAtEnd\ETE@AtEnd
\RequirePackage{epstopdf-base}[2016/05/15]
\let\ETE@AtEnd\ETE@SavedAtEnd
%    \end{macrocode}
%
% \subsubsection{Option handling}
%
%    \begin{macrocode}
\DeclareOption*{%
  \expandafter\epstopdfsetup\expandafter{\CurrentOption}%
}
\ProcessOptions*\relax
%    \end{macrocode}
%
%    \begin{macrocode}
\ETE@AtEnd%
%</package>
%    \end{macrocode}
%
% \subsection{Base package}
%
%    \begin{macrocode}
%<*base>
%    \end{macrocode}
%
% \subsection{Preparations}
%
% \subsubsection{Relead check and identification}
%
%    Reload check, especially if the package is not used with \LaTeX.
%    \begin{macrocode}
\begingroup\catcode61\catcode48\catcode32=10\relax%
  \catcode13=5 % ^^M
  \endlinechar=13 %
  \catcode35=6 % #
  \catcode39=12 % '
  \catcode44=12 % ,
  \catcode45=12 % -
  \catcode46=12 % .
  \catcode58=12 % :
  \catcode64=11 % @
  \catcode123=1 % {
  \catcode125=2 % }
  \expandafter\let\expandafter\x\csname ver@epstopdf-base.sty\endcsname
  \ifx\x\relax % plain-TeX, first loading
  \else
    \def\empty{}%
    \ifx\x\empty % LaTeX, first loading,
      % variable is initialized, but \ProvidesPackage not yet seen
    \else
      \expandafter\ifx\csname PackageInfo\endcsname\relax
        \def\x#1#2{%
          \immediate\write-1{Package #1 Info: #2.}%
        }%
      \else
        \def\x#1#2{\PackageInfo{#1}{#2, stopped}}%
      \fi
      \x{epstopdf-base}{The package is already loaded}%
      \aftergroup\endinput
    \fi
  \fi
\endgroup%
%    \end{macrocode}
%    Package identification:
%    \begin{macrocode}
\begingroup\catcode61\catcode48\catcode32=10\relax%
  \catcode13=5 % ^^M
  \endlinechar=13 %
  \catcode35=6 % #
  \catcode39=12 % '
  \catcode40=12 % (
  \catcode41=12 % )
  \catcode44=12 % ,
  \catcode45=12 % -
  \catcode46=12 % .
  \catcode47=12 % /
  \catcode58=12 % :
  \catcode64=11 % @
  \catcode91=12 % [
  \catcode93=12 % ]
  \catcode123=1 % {
  \catcode125=2 % }
  \expandafter\ifx\csname ProvidesPackage\endcsname\relax
    \def\x#1#2#3[#4]{\endgroup
      \immediate\write-1{Package: #3 #4}%
      \xdef#1{#4}%
    }%
  \else
    \def\x#1#2[#3]{\endgroup
      #2[{#3}]%
      \ifx#1\@undefined
        \xdef#1{#3}%
      \fi
      \ifx#1\relax
        \xdef#1{#3}%
      \fi
    }%
  \fi
\expandafter\x\csname ver@epstopdf-base.sty\endcsname
\ProvidesPackage{epstopdf-base}%
  [2016/05/15 v2.6 Base part for package epstopdf]%
%    \end{macrocode}
%
% \subsubsection{Catcodes}
%
%    \begin{macrocode}
\begingroup\catcode61\catcode48\catcode32=10\relax%
  \catcode13=5 % ^^M
  \endlinechar=13 %
  \catcode123=1 % {
  \catcode125=2 % }
  \catcode64=11 % @
  \def\x{\endgroup
    \expandafter\edef\csname ETE@AtEnd\endcsname{%
      \endlinechar=\the\endlinechar\relax
      \catcode13=\the\catcode13\relax
      \catcode32=\the\catcode32\relax
      \catcode35=\the\catcode35\relax
      \catcode61=\the\catcode61\relax
      \catcode64=\the\catcode64\relax
      \catcode123=\the\catcode123\relax
      \catcode125=\the\catcode125\relax
    }%
  }%
\x\catcode61\catcode48\catcode32=10\relax%
\catcode13=5 % ^^M
\endlinechar=13 %
\catcode35=6 % #
\catcode64=11 % @
\catcode123=1 % {
\catcode125=2 % }
\def\TMP@EnsureCode#1#2{%
  \edef\ETE@AtEnd{%
    \ETE@AtEnd
    \catcode#1=\the\catcode#1\relax
  }%
  \catcode#1=#2\relax
}
\TMP@EnsureCode{33}{12}% !
\TMP@EnsureCode{39}{12}% '
\TMP@EnsureCode{42}{12}% *
\TMP@EnsureCode{44}{12}% ,
\TMP@EnsureCode{45}{12}% -
\TMP@EnsureCode{46}{12}% .
\TMP@EnsureCode{47}{12}% /
\TMP@EnsureCode{58}{12}% :
\TMP@EnsureCode{60}{12}% <
\TMP@EnsureCode{62}{12}% >
\TMP@EnsureCode{96}{12}% `
\edef\ETE@AtEnd{\ETE@AtEnd\noexpand\endinput}
%    \end{macrocode}
%
% \subsubsection{Load packages}
%
%    \begin{macrocode}
\RequirePackage{infwarerr}[2007/09/09]
\RequirePackage{grfext}\relax
\RequirePackage{kvoptions}[2007/10/02]
\RequirePackage{pdftexcmds}[2007/11/11]
%    \end{macrocode}
%
% \subsection{Checks}
%
% Check, whether package graphics is loaded
% (also graphicx loads graphics).
% Because miniltx.tex does not know \cs{@ifpackageloaded}
% we test for \cs{Gin@setfile} instead.
%    \begin{macrocode}
\begingroup\expandafter\expandafter\expandafter\endgroup
\expandafter\ifx\csname Gin@setfile\endcsname\relax
  \@PackageWarningNoLine{epstopdf}{%
    No graphics package \string`graphic{s,x}\string' loaded%
  }%
  \newcommand*{\epstopdfsetup}[1]{}%
  \expandafter\ETE@AtEnd
\fi%
%    \end{macrocode}
%
% Check, whether pdftex.def is loaded. \cs{ver@pdftex.def} is not
% available with miniltx.tex, thus we test for \cs{Gin@driver}.
%    \begin{macrocode}
\begingroup
  \def\x{luatex.def}%
  \ifx\Gin@driver\x
  \else
  \def\x{pdftex.def}%
  \ifx\Gin@driver\x
  \else
    \@PackageWarningNoLine{epstopdf}{%
      Drivers other than `pdftex' and `luatex' are not supported%
    }%
    \endgroup
    \newcommand*{\epstopdfsetup}[1]{}%
    \expandafter\ETE@AtEnd
  \fi%
  \fi%
\endgroup
%    \end{macrocode}
%
% Check, whether the shell escape feature is enabled.
%
%    \begin{macrocode}
\begingroup
  \expandafter\ifx\csname pdf@shellescape\endcsname\relax
  \else
    \ifnum\pdf@shellescape>0 %
    \else
      \@PackageWarningNoLine{epstopdf}{%
        Shell escape feature is not enabled%
      }%
    \fi
  \fi
\endgroup
%    \end{macrocode}
%
% \subsection{Package loading}
%
%    \begin{macrocode}
%    \end{macrocode}
%
% \subsection{Options}
%
%    \begin{macrocode}
\SetupKeyvalOptions{family=ETE,prefix=ETE@}
\DeclareBoolOption{update}
\DeclareBoolOption{verbose}
\newif\ifETE@prepend
\DeclareVoidOption{prepend}{\ETE@prependtrue}
\DeclareVoidOption{append}{\ETE@prependfalse}
\DeclareStringOption{outdir}
\DeclareStringOption{suffix}
\DeclareBoolOption{prefersuffix}
\DeclareStringOption{program@epstopdf}
%    \end{macrocode}
%    Options \xoption{disable} and \xoption{enable} are for testing
%    only. Therefore they are not documented on purpose.
%    \begin{macrocode}
\DeclareBoolOption{disable}
\DeclareComplementaryOption{enable}{disable}
%    \end{macrocode}
%    \begin{macrocode}
\newcommand*{\epstopdfsetup}{\setkeys{ETE}}
%    \end{macrocode}
%
% \subsubsection{Default setting}
%
%    \begin{macrocode}
\epstopdfsetup{%
  verbose,%
  enable,%
  append,%
  update,%
  prefersuffix,%
  suffix=-\SourceExt-converted-to,%
  program@epstopdf=epstopdf%
}
%    \end{macrocode}
%
% \subsection{Make and verbose}
%
%    \begin{macrocode}
\begingroup\expandafter\expandafter\expandafter\endgroup
\expandafter\ifx\csname pdf@filemoddate\endcsname\relax
  \def\ETE@Make#1#2{%
    \ifETE@update
      \ETE@WarnModDate
    \fi
    \@firstofone
  }%
  \def\ETE@WarnModDate{%
    \@PackageWarningNoLine{epstopdf}{%
      \string\pdffilemoddate\space is not available,\MessageBreak
      option `update' will be ignored%
    }%
    \global\let\ETE@WarnModDate\relax
  }%
  \def\ETE@FileInfo#1#2{#1 file: <#2>}%
\else
  \def\ETE@Make#1#2{%
    \ifETE@update
      \ifnum\pdf@strcmp{\pdf@filemoddate{#1}}{\pdf@filemoddate{#2}}>0 %
        \expandafter\expandafter\expandafter\@firstofone
      \else
        \@PackageInfoNoLine{epstopdf}{%
          Output file is already uptodate%
        }%
        \expandafter\expandafter\expandafter\@gobble
      \fi
    \else
      \expandafter\@firstofone
    \fi
  }%
  \def\ETE@FileInfo#1#2{%
    #1 file: <#2>%
    \expandafter\expandafter\expandafter
    \ETE@Date\pdf@filemoddate{#2}\@nil
    \expandafter\expandafter\expandafter
    \ETE@Size\pdf@filesize{#2}\@nil
  }%
  \def\ETE@Date#1\@nil{%
    \ifx\\#1\\%
    \else
      \ETE@@Date#1\@nil
    \fi
  }%
  \def\ETE@@Date#1:#2#3#4#5#6#7#8#9{%
    \MessageBreak
    \@spaces\space\space\space date: #2#3#4#5-#6#7-#8#9 %
    \ETE@@Time
  }%
  \def\ETE@@Time#1#2#3#4#5#6#7\@nil{%
    #1#2:#3#4:#5#6%
  }%
  \def\ETE@Size#1\@nil{%
    \ifx\\#1\\%
    \else
      \MessageBreak
      \@spaces\space\space\space size: #1 bytes%
    \fi
  }%
\fi
%    \end{macrocode}
%
% \subsection{Adding conversion support}
%
%    Patch \cs{Gin@setfile} to execute |#3|, if it contains
%    a command.
%    \begin{macrocode}
\expandafter\ifx\csname ETE@OrgGin@setfile\endcsname\relax
  \let\ETE@OrgGin@setfile\Gin@setfile
\else
  \@PackageError{epstopdf}{%
    Command \string\ETE@OrgGin@setfile\space
    already defined.\MessageBreak
  }{%
    Probably some package has included the code of this package%
    \MessageBreak
    instead of using \string\RequirePackage{epstopdf}.%
    \MessageBreak
    \@ehc
  }%
\fi
\def\ETE@IfFileExists{%
  \begingroup\expandafter\expandafter\expandafter\endgroup
  \expandafter\ifx\csname grffile@IfFileExists\endcsname\relax
    \expandafter\IfFileExists
  \else
    \global\let\ETE@IfFileExists\grffile@IfFileExists
    \expandafter\grffile@IfFileExists
  \fi
}
\def\ETE@Skip#1#2\x\fi\fi{%
  \fi
  \fi
  \endgroup
  \fi
  \fi
  #1%
}
\newif\ifETE@InsideSetfile
\newcommand*{\epstopdfcall}[1]{%
  \ifETE@InsideSetfile
    \expandafter\@firstoftwo
  \else
    \expandafter\@secondoftwo
  \fi
  {`#1}%
  {\Gin@base\Gin@ext}%
}
\def\ETE@DefCommandLine#1{%
  \edef\CommandLine{\expandafter\fi\if`#1}%
}
\def\ETE@DefX#1{%
  \expandafter\expandafter\expandafter\def
  \expandafter\expandafter\expandafter\x
  \expandafter\expandafter\expandafter{%
    \expandafter\fi\if`#1\relax\else
  }%
}
\def\ETE@Gin@setfile#1#2#3{%
  \ifETE@disable
    \ETE@OrgGin@setfile{#1}{#2}{#3}%
  \else
    \begingroup
      \ETE@InsideSetfiletrue
      \ETE@DefX{#3}%
    \expandafter\endgroup
    \ifx\x\@empty
      \ETE@OrgGin@setfile{#1}{#2}{#3}%
    \else
      \begingroup
        \ETE@InsideSetfiletrue
        \def\GraphicsType{#1}%
        \def\GraphicsRead{#2}%
        \ifETE@prefersuffix
        \else
          \ifx\ETE@suffix\@empty
          \else
            \ETE@IfFileExists{\Gin@base\GraphicsRead}{%
              \ETE@Skip{%
                \ETE@OrgGin@setfile{#1}{#2}{\Gin@base#2}%
              }%
            }{%
              \let\next\relax
            }%
            \next
          \fi
        \fi
        \ifx\Gin@ext\relax
          \let\SourceExt\Gin@eext
          \def\SourceFile{\Gin@base\Gin@eext}%
        \else
          \let\SourceExt\Gin@ext
          \def\SourceFile{\Gin@base\Gin@ext}%
        \fi
        \edef\SourceExt{% remove dot
          \expandafter\@cdr\SourceExt\@empty\@nil
        }%
        \let\OutputDirectory\ETE@outdir
        \ifx\OutputDirectory\@empty
          \edef\OutputFile{\ETE@GenerateName{\Gin@base}{#2}}%
        \else
          \begingroup
            \filename@parse{\Gin@base#2}%
            \edef\x{\endgroup
              \def\noexpand\OutputFile{%
                \ETE@GenerateName{%
                  \OutputDirectory\filename@base
                }{#2}%
              }%
            }%
          \x
        \fi
        \ETE@DefCommandLine{#3}%
        \ifETE@verbose
          \@PackageInfo{epstopdf}{%
            \ETE@FileInfo{Source}\SourceFile\MessageBreak
            \ETE@FileInfo{Output}\OutputFile\MessageBreak
            Command: <\CommandLine>\MessageBreak
            \string\includegraphics
          }%
        \fi
        \ETE@Make\SourceFile\OutputFile{%
          \pdf@system{\CommandLine}%
          \ifETE@verbose
            \@PackageInfoNoLine{epstopdf}{%
              \ETE@FileInfo{Result}\OutputFile
            }%
          \fi
        }%
        \edef\x{\endgroup
          \ifx\OutputDirectory\@empty
          \else
            \def\noexpand\Gin@base{%
              \OutputDirectory\noexpand\filename@base
            }%
          \fi
          \ifx\ETE@suffix\@empty
          \else
            \edef\noexpand\Gin@base{%
              \noexpand\Gin@base\ETE@suffix
            }%
          \fi
          \noexpand\ETE@OrgGin@setfile{%
            \GraphicsType
          }{%
            \GraphicsRead
          }{%
            \OutputFile
          }%
        }%
      \x
    \fi
  \fi
}
\let\Gin@setfile\ETE@Gin@setfile
\def\ETE@GenerateName#1#2{%
  #1\ETE@suffix#2%
}
%    \end{macrocode}
%
% \subsection{Declare graphics rule}
%
%    \begin{macrocode}
\newcommand*{\epstopdfDeclareGraphicsRule}[4]{%
  \ifx\\#4\\%
    \@PackageError{epstopdf-base}{%
      Conversion command is missing%
    }\@ehc
  \else
    \begingroup
      \@ifundefined{Gin@rule@#1}{%
      }{%
        \@PackageInfo{epstopdf-base}{%
          Redefining graphics rule for `#1'%
        }%
      }%
    \endgroup
    \@namedef{Gin@rule@#1}##1{{#2}{#3}{\epstopdfcall{#4}}}%
  \fi
}
%    \end{macrocode}
%
% \cs{DeclareGraphicsRule} for .eps
%    \begin{macrocode}
\epstopdfDeclareGraphicsRule{.eps}{pdf}{.pdf}{%
  \ETE@epstopdf{#1}%
}
\def\ETE@epstopdf#1{%
  \ETE@program@epstopdf\space
  \ifcase\ifx\OutputDirectory\@empty
           \ifx\ETE@suffix\@empty
             1%
           \fi
         \fi
         0 %
    --outfile=\OutputFile\space
  \fi
  #1%
}
%    \end{macrocode}
%
%    \begin{macrocode}
\ifETE@prepend
  \expandafter\PrependGraphicsExtensions
\else
  \expandafter\AppendGraphicsExtensions
\fi
{.eps}
\let\ETE@prepend\@undefined
\DeclareVoidOption{prepend}{%
  \PrependGraphicsExtensions{.eps}%
}
\let\ETE@append\@undefined
\DeclareVoidOption{append}{%
  \AppendGraphicsExtensions{.eps}%
}
%    \end{macrocode}
%
%    \begin{macrocode}
\InputIfFileExists{epstopdf-sys.cfg}{}{}
\InputIfFileExists{epstopdf.cfg}{}{}
%    \end{macrocode}
%
%    \begin{macrocode}
\ETE@AtEnd%
%</base>
%    \end{macrocode}
%
% \section{Test}
%
% \subsection{Preface for standard catcode check}
%
%    \begin{macrocode}
%<*test1>
\input miniltx.tex\relax
\def\Gin@driver{pdftex.def}
\input graphicx.sty\relax
\resetatcatcode
%</test1>
%    \end{macrocode}
%
% \subsection{Catcode checks for loading}
%
%    \begin{macrocode}
%<*test1>
%    \end{macrocode}
%    \begin{macrocode}
\catcode`\{=1 %
\catcode`\}=2 %
\catcode`\#=6 %
\catcode`\@=11 %
\expandafter\ifx\csname count@\endcsname\relax
  \countdef\count@=255 %
\fi
\expandafter\ifx\csname @gobble\endcsname\relax
  \long\def\@gobble#1{}%
\fi
\expandafter\ifx\csname @firstofone\endcsname\relax
  \long\def\@firstofone#1{#1}%
\fi
\expandafter\ifx\csname loop\endcsname\relax
  \expandafter\@firstofone
\else
  \expandafter\@gobble
\fi
{%
  \def\loop#1\repeat{%
    \def\body{#1}%
    \iterate
  }%
  \def\iterate{%
    \body
      \let\next\iterate
    \else
      \let\next\relax
    \fi
    \next
  }%
  \let\repeat=\fi
}%
\def\RestoreCatcodes{}
\count@=0 %
\loop
  \edef\RestoreCatcodes{%
    \RestoreCatcodes
    \catcode\the\count@=\the\catcode\count@\relax
  }%
\ifnum\count@<255 %
  \advance\count@ 1 %
\repeat

\def\RangeCatcodeInvalid#1#2{%
  \count@=#1\relax
  \loop
    \catcode\count@=15 %
  \ifnum\count@<#2\relax
    \advance\count@ 1 %
  \repeat
}
\def\RangeCatcodeCheck#1#2#3{%
  \count@=#1\relax
  \loop
    \ifnum#3=\catcode\count@
    \else
      \errmessage{%
        Character \the\count@\space
        with wrong catcode \the\catcode\count@\space
        instead of \number#3%
      }%
    \fi
  \ifnum\count@<#2\relax
    \advance\count@ 1 %
  \repeat
}
\def\space{ }
\expandafter\ifx\csname LoadCommand\endcsname\relax
  \def\LoadCommand{\input epstopdf.sty\relax}%
\fi
\def\Test{%
  \RangeCatcodeInvalid{0}{47}%
  \RangeCatcodeInvalid{58}{64}%
  \RangeCatcodeInvalid{91}{96}%
  \RangeCatcodeInvalid{123}{255}%
  \catcode`\@=12 %
  \catcode`\\=0 %
  \catcode`\%=14 %
  \LoadCommand
  \RangeCatcodeCheck{0}{36}{15}%
  \RangeCatcodeCheck{37}{37}{14}%
  \RangeCatcodeCheck{38}{47}{15}%
  \RangeCatcodeCheck{48}{57}{12}%
  \RangeCatcodeCheck{58}{63}{15}%
  \RangeCatcodeCheck{64}{64}{12}%
  \RangeCatcodeCheck{65}{90}{11}%
  \RangeCatcodeCheck{91}{91}{15}%
  \RangeCatcodeCheck{92}{92}{0}%
  \RangeCatcodeCheck{93}{96}{15}%
  \RangeCatcodeCheck{97}{122}{11}%
  \RangeCatcodeCheck{123}{255}{15}%
  \RestoreCatcodes
}
\Test
\csname @@end\endcsname
\end
%    \end{macrocode}
%    \begin{macrocode}
%</test1>
%    \end{macrocode}
%
% \section{Installation}
%
% \subsection{Download}
%
% \paragraph{Package.} This package is available on
% CTAN\footnote{\url{http://ctan.org/pkg/epstopdf}}:
% \begin{description}
% \item[\CTAN{macros/latex/contrib/oberdiek/epstopdf.dtx}] The source file.
% \item[\CTAN{macros/latex/contrib/oberdiek/epstopdf.pdf}] Documentation.
% \end{description}
%
%
% \paragraph{Bundle.} All the packages of the bundle `oberdiek'
% are also available in a TDS compliant ZIP archive. There
% the packages are already unpacked and the documentation files
% are generated. The files and directories obey the TDS standard.
% \begin{description}
% \item[\CTAN{install/macros/latex/contrib/oberdiek.tds.zip}]
% \end{description}
% \emph{TDS} refers to the standard ``A Directory Structure
% for \TeX\ Files'' (\CTAN{tds/tds.pdf}). Directories
% with \xfile{texmf} in their name are usually organized this way.
%
% \subsection{Bundle installation}
%
% \paragraph{Unpacking.} Unpack the \xfile{oberdiek.tds.zip} in the
% TDS tree (also known as \xfile{texmf} tree) of your choice.
% Example (linux):
% \begin{quote}
%   |unzip oberdiek.tds.zip -d ~/texmf|
% \end{quote}
%
% \paragraph{Script installation.}
% Check the directory \xfile{TDS:scripts/oberdiek/} for
% scripts that need further installation steps.
% Package \xpackage{attachfile2} comes with the Perl script
% \xfile{pdfatfi.pl} that should be installed in such a way
% that it can be called as \texttt{pdfatfi}.
% Example (linux):
% \begin{quote}
%   |chmod +x scripts/oberdiek/pdfatfi.pl|\\
%   |cp scripts/oberdiek/pdfatfi.pl /usr/local/bin/|
% \end{quote}
%
% \subsection{Package installation}
%
% \paragraph{Unpacking.} The \xfile{.dtx} file is a self-extracting
% \docstrip\ archive. The files are extracted by running the
% \xfile{.dtx} through \plainTeX:
% \begin{quote}
%   \verb|tex epstopdf.dtx|
% \end{quote}
%
% \paragraph{TDS.} Now the different files must be moved into
% the different directories in your installation TDS tree
% (also known as \xfile{texmf} tree):
% \begin{quote}
% \def\t{^^A
% \begin{tabular}{@{}>{\ttfamily}l@{ $\rightarrow$ }>{\ttfamily}l@{}}
%   epstopdf.sty & tex/latex/oberdiek/epstopdf.sty\\
%   epstopdf-base.sty & tex/latex/oberdiek/epstopdf-base.sty\\
%   epstopdf.pdf & doc/latex/oberdiek/epstopdf.pdf\\
%   test/epstopdf-test1.tex & doc/latex/oberdiek/test/epstopdf-test1.tex\\
%   epstopdf.dtx & source/latex/oberdiek/epstopdf.dtx\\
% \end{tabular}^^A
% }^^A
% \sbox0{\t}^^A
% \ifdim\wd0>\linewidth
%   \begingroup
%     \advance\linewidth by\leftmargin
%     \advance\linewidth by\rightmargin
%   \edef\x{\endgroup
%     \def\noexpand\lw{\the\linewidth}^^A
%   }\x
%   \def\lwbox{^^A
%     \leavevmode
%     \hbox to \linewidth{^^A
%       \kern-\leftmargin\relax
%       \hss
%       \usebox0
%       \hss
%       \kern-\rightmargin\relax
%     }^^A
%   }^^A
%   \ifdim\wd0>\lw
%     \sbox0{\small\t}^^A
%     \ifdim\wd0>\linewidth
%       \ifdim\wd0>\lw
%         \sbox0{\footnotesize\t}^^A
%         \ifdim\wd0>\linewidth
%           \ifdim\wd0>\lw
%             \sbox0{\scriptsize\t}^^A
%             \ifdim\wd0>\linewidth
%               \ifdim\wd0>\lw
%                 \sbox0{\tiny\t}^^A
%                 \ifdim\wd0>\linewidth
%                   \lwbox
%                 \else
%                   \usebox0
%                 \fi
%               \else
%                 \lwbox
%               \fi
%             \else
%               \usebox0
%             \fi
%           \else
%             \lwbox
%           \fi
%         \else
%           \usebox0
%         \fi
%       \else
%         \lwbox
%       \fi
%     \else
%       \usebox0
%     \fi
%   \else
%     \lwbox
%   \fi
% \else
%   \usebox0
% \fi
% \end{quote}
% If you have a \xfile{docstrip.cfg} that configures and enables \docstrip's
% TDS installing feature, then some files can already be in the right
% place, see the documentation of \docstrip.
%
% \subsection{Refresh file name databases}
%
% If your \TeX~distribution
% (\teTeX, \mikTeX, \dots) relies on file name databases, you must refresh
% these. For example, \teTeX\ users run \verb|texhash| or
% \verb|mktexlsr|.
%
% \subsection{Some details for the interested}
%
% \paragraph{Attached source.}
%
% The PDF documentation on CTAN also includes the
% \xfile{.dtx} source file. It can be extracted by
% AcrobatReader 6 or higher. Another option is \textsf{pdftk},
% e.g. unpack the file into the current directory:
% \begin{quote}
%   \verb|pdftk epstopdf.pdf unpack_files output .|
% \end{quote}
%
% \paragraph{Unpacking with \LaTeX.}
% The \xfile{.dtx} chooses its action depending on the format:
% \begin{description}
% \item[\plainTeX:] Run \docstrip\ and extract the files.
% \item[\LaTeX:] Generate the documentation.
% \end{description}
% If you insist on using \LaTeX\ for \docstrip\ (really,
% \docstrip\ does not need \LaTeX), then inform the autodetect routine
% about your intention:
% \begin{quote}
%   \verb|latex \let\install=y% \iffalse meta-comment
%
% File: epstopdf.dtx
% Version: 2016/05/15 v2.6
% Info: Conversion with epstopdf on the fly
%
% Copyright (C) 2001, 2006-2010 by
%    Heiko Oberdiek <heiko.oberdiek at googlemail.com>
%
% This work may be distributed and/or modified under the
% conditions of the LaTeX Project Public License, either
% version 1.3c of this license or (at your option) any later
% version. This version of this license is in
%    http://www.latex-project.org/lppl/lppl-1-3c.txt
% and the latest version of this license is in
%    http://www.latex-project.org/lppl.txt
% and version 1.3 or later is part of all distributions of
% LaTeX version 2005/12/01 or later.
%
% This work has the LPPL maintenance status "maintained".
%
% This Current Maintainer of this work is Heiko Oberdiek.
%
% This work consists of the main source file epstopdf.dtx
% and the derived files
%    epstopdf.sty, epstopdf.pdf, epstopdf.ins, epstopdf.drv,
%    epstopdf-base.sty, epstopdf-test1.tex.
%
% Distribution:
%    CTAN:macros/latex/contrib/oberdiek/epstopdf.dtx
%    CTAN:macros/latex/contrib/oberdiek/epstopdf.pdf
%
% Unpacking:
%    (a) If epstopdf.ins is present:
%           tex epstopdf.ins
%    (b) Without epstopdf.ins:
%           tex epstopdf.dtx
%    (c) If you insist on using LaTeX
%           latex \let\install=y\input{epstopdf.dtx}
%        (quote the arguments according to the demands of your shell)
%
% Documentation:
%    (a) If epstopdf.drv is present:
%           latex epstopdf.drv
%    (b) Without epstopdf.drv:
%           latex epstopdf.dtx; ...
%    The class ltxdoc loads the configuration file ltxdoc.cfg
%    if available. Here you can specify further options, e.g.
%    use A4 as paper format:
%       \PassOptionsToClass{a4paper}{article}
%
%    Programm calls to get the documentation (example):
%       pdflatex epstopdf.dtx
%       makeindex -s gind.ist epstopdf.idx
%       pdflatex epstopdf.dtx
%       makeindex -s gind.ist epstopdf.idx
%       pdflatex epstopdf.dtx
%
% Installation:
%    TDS:tex/latex/oberdiek/epstopdf.sty
%    TDS:tex/latex/oberdiek/epstopdf-base.sty
%    TDS:doc/latex/oberdiek/epstopdf.pdf
%    TDS:doc/latex/oberdiek/test/epstopdf-test1.tex
%    TDS:source/latex/oberdiek/epstopdf.dtx
%
%<*ignore>
\begingroup
  \catcode123=1 %
  \catcode125=2 %
  \def\x{LaTeX2e}%
\expandafter\endgroup
\ifcase 0\ifx\install y1\fi\expandafter
         \ifx\csname processbatchFile\endcsname\relax\else1\fi
         \ifx\fmtname\x\else 1\fi\relax
\else\csname fi\endcsname
%</ignore>
%<*install>
\input docstrip.tex
\Msg{************************************************************************}
\Msg{* Installation}
\Msg{* Package: epstopdf 2016/05/15 v2.6 Conversion with epstopdf on the fly (HO)}
\Msg{************************************************************************}

\keepsilent
\askforoverwritefalse

\let\MetaPrefix\relax
\preamble

This is a generated file.

Project: epstopdf
Version: 2016/05/15 v2.6

Copyright (C) 2001, 2006-2010 by
   Heiko Oberdiek <heiko.oberdiek at googlemail.com>

This work may be distributed and/or modified under the
conditions of the LaTeX Project Public License, either
version 1.3c of this license or (at your option) any later
version. This version of this license is in
   http://www.latex-project.org/lppl/lppl-1-3c.txt
and the latest version of this license is in
   http://www.latex-project.org/lppl.txt
and version 1.3 or later is part of all distributions of
LaTeX version 2005/12/01 or later.

This work has the LPPL maintenance status "maintained".

This Current Maintainer of this work is Heiko Oberdiek.

This work consists of the main source file epstopdf.dtx
and the derived files
   epstopdf.sty, epstopdf.pdf, epstopdf.ins, epstopdf.drv,
   epstopdf-base.sty, epstopdf-test1.tex.

\endpreamble
\let\MetaPrefix\DoubleperCent

\generate{%
  \file{epstopdf.ins}{\from{epstopdf.dtx}{install}}%
  \file{epstopdf.drv}{\from{epstopdf.dtx}{driver}}%
  \usedir{tex/latex/oberdiek}%
  \file{epstopdf.sty}{\from{epstopdf.dtx}{package}}%
  \file{epstopdf-base.sty}{\from{epstopdf.dtx}{base}}%
  \usedir{doc/latex/oberdiek/test}%
  \file{epstopdf-test1.tex}{\from{epstopdf.dtx}{test1}}%
  \nopreamble
  \nopostamble
  \usedir{source/latex/oberdiek/catalogue}%
  \file{epstopdf-pkg.xml}{\from{epstopdf.dtx}{catalogue}}%
}

\catcode32=13\relax% active space
\let =\space%
\Msg{************************************************************************}
\Msg{*}
\Msg{* To finish the installation you have to move the following}
\Msg{* files into a directory searched by TeX:}
\Msg{*}
\Msg{*     epstopdf.sty, epstopdf-base.sty}
\Msg{*}
\Msg{* To produce the documentation run the file `epstopdf.drv'}
\Msg{* through LaTeX.}
\Msg{*}
\Msg{* Happy TeXing!}
\Msg{*}
\Msg{************************************************************************}

\endbatchfile
%</install>
%<*ignore>
\fi
%</ignore>
%<*driver>
\NeedsTeXFormat{LaTeX2e}
\ProvidesFile{epstopdf.drv}%
  [2016/05/15 v2.6 Conversion with epstopdf on the fly (HO)]%
\documentclass{ltxdoc}
\usepackage{holtxdoc}[2011/11/22]
\begin{document}
  \DocInput{epstopdf.dtx}%
\end{document}
%</driver>
% \fi
%
%
% \CharacterTable
%  {Upper-case    \A\B\C\D\E\F\G\H\I\J\K\L\M\N\O\P\Q\R\S\T\U\V\W\X\Y\Z
%   Lower-case    \a\b\c\d\e\f\g\h\i\j\k\l\m\n\o\p\q\r\s\t\u\v\w\x\y\z
%   Digits        \0\1\2\3\4\5\6\7\8\9
%   Exclamation   \!     Double quote  \"     Hash (number) \#
%   Dollar        \$     Percent       \%     Ampersand     \&
%   Acute accent  \'     Left paren    \(     Right paren   \)
%   Asterisk      \*     Plus          \+     Comma         \,
%   Minus         \-     Point         \.     Solidus       \/
%   Colon         \:     Semicolon     \;     Less than     \<
%   Equals        \=     Greater than  \>     Question mark \?
%   Commercial at \@     Left bracket  \[     Backslash     \\
%   Right bracket \]     Circumflex    \^     Underscore    \_
%   Grave accent  \`     Left brace    \{     Vertical bar  \|
%   Right brace   \}     Tilde         \~}
%
% \GetFileInfo{epstopdf.drv}
%
% \title{The \xpackage{epstopdf} package}
% \date{2016/05/15 v2.6}
% \author{Heiko Oberdiek\thanks
% {Please report any issues at https://github.com/ho-tex/oberdiek/issues}\\
% \xemail{heiko.oberdiek at googlemail.com}}
%
% \maketitle
%
% \begin{abstract}
% This packages adds support of handling eps images
% to package \xpackage{graphics} or \xpackage{graphicx}
% with option \xoption{pdftex}. If an eps image is detected,
% epstopdf is automatically called to convert it to pdf format.
% \end{abstract}
%
% \tableofcontents
%
% \providecommand*{\xprogram}[1]{\textsf{#1}}
%
% \section{Documentation}
%
% \subsection{Introduction}
%
% \LaTeX\ provides its graphics bundle to include graphics files.
% Both packages \xpackage{graphics} or \xpackage{graphicx} may
% be used. the latter one loads the first and adds options
% in key value style for \cs{includegraphics}.
%
% Usually the drivers do not support all kind of graphics files.
% Other image types must be converted, before they become
% usuable. In case of driver \xprogram{dvips}, the graphics
% rule may contain a conversion rule. Then all that package
% \xpackage{graphics} must know is the bounding box, the command
% is passed to \xprogram{dvips} that calls it and embeds the
% converted image.
%
% However, \pdfTeX\ has its driver for PDF output already
% build in. It's graphics inclusion commands (\cs{pdfximage})
% does not allow the execution of external commands.
% Therefore commands in the last argument of
% \cs{DeclareGraphicsRule} were of no use.
% But external programs can be called within \pdfTeX.
% This feature is called ``shell escape'' or ``write 18''
% and must usually enabled explicitely because of security reasons.
% Now, this package \xpackage{epstopdf} hooks
% into package \xpackage{graphics}' code to catch
% that argument with the external command and executes it
% to convert the graphics file to a supported format and
% passes the control of graphics inclusion back to package
% \xpackage{graphics}.
%
% \subsection{Requirements}
%
% \begin{itemize}
% \item
%   The feature \cs{write18} must be enabled. This allows the
%   running of external programs during \TeX's compile run.
%   Keep in mind that this is a security risk. The feature is an
%   addition to \cs{TeX}. Mik\TeX, te\TeX, \TeX\ Live support it.
%   In Web2C based \TeX\ distributions (te\TeX, \TeX\ Live) it can be
%   enabled in the configuration file \xfile{texmf.cnf}:
%   \begin{quote}
%     |shell_escape = 1|
%   \end{quote}
%   Because of the security risk, it is better to do it on the
%   command line only:
%   \begin{quote}
%   \texttt{--shell-escape} (te\TeX, \TeX\ Live)\\
%   \texttt{--enable-write18} (MiK\TeX)
%   \end{quote}
%   Example:
%   \begin{quote}
%     |pdflatex -shell-escape test.tex|
%   \end{quote}
% \item
%   The program \xprogram{epstopdf} for the conversion from EPS
%   to PDF. However, other programs can
%   be used and configured by \cs{DeclareGraphicsRule}.
%   Example:
%   \begin{quote}
%   |\epstopdfDeclareGraphicsRule{.eps}{pdf}{.pdf}{%|\\
%   |  ps2pdf -dEPSCrop #1 \OutputFile|\\
%   |}|
%   \end{quote}
% \end{itemize}
%
% \subsection{Usage}
%
% The package is loaded after graphic\{s,x\}, e.g.:
% \begin{quote}
%   |\usepackage[pdftex]{graphicx}|\\
%   |\usepackage{epstopdf}|
% \end{quote}
% Now images with file name extension \xfile{.eps} are detected
% and supported using \cs{includegraphics}.
%
% If the graphics file name is explicitly specified with extension
% \xfile{.eps} the new rule for EPS files is called and
% the conversion performed. If option \xoption{update} is in force
% then the conversion step is dropped if the target file already
% exists and is not older then the EPS file.
%
% The situation is more complicate if the graphics file is
% given without file name extension. Then the \xpackage{graphics}
% package must search for a supported image file.
% The possible extensions are stored in the graphics extension
% list, that can be set by \cs{DeclareGraphicsExtensions}.
% The algorithm:
% \begin{quote}
%    \def\>{\mbox{}\quad}
%    function search( \meta{filebase} )\\
%    \>foreach \meta{ext} in \meta{graphics extensions}\\
%    \>\>foreach \meta{dir} in \meta{current directory}, \meta{\cs{graphicspath}}\\
%    \>\>\>\meta{file} := \meta{dir} + \meta{filebase} + \meta{ext}\\
%    \>\>\>if exist \meta{file}\\
%    \>\>\>\>return found\\
%    \>return not found
% \end{quote}
% Package \xpackage{epstopdf} puts \xfile{.eps} at the end of the
% graphics extension search list. This is the behaviour of option
% \xoption{append} that is enabled by default.
% That means, the conversion is called last
% unless a supported file type cannot be found earlier. This avoids
% unnecessary conversion steps that slow down the \LaTeX\ run.
% If you want to use option \xoption{update} and your \pdfTeX\ supports
% it, then an outdated PDF file also would be found earlier unless
% \xoption{suffix} is used that is the default since version 2.0.
%
% With an empty option \xoption{suffix} and option
% \xoption{prepend} there is a risk that an original PDF file is overwritten:
%
% If the original image file is the PDF file and there is also a generated
% EPS file, then the original PDF file can be regenerated (depending
% on the option settings) and the original PDF file gets lost.
% Therefore option \xoption{suffix} is introduced in version 1.9
% to create a separate name space for generated output files.
%
% \textbf{Note:} Usually the conversion program needs the exact location
% of the image file. Usually the current directory works.
% Also if the image file is found using \cs{graphicspath}, the
% location is known. However, if the image is somewhere
% in a directory of environment variable |TEXINPUTS|, then
% the package does not know the exact location and the
% conversion program will not find the image file unless it
% implements a search using |TEXINPUTS| (program \xprogram{kpsewhich}
% may be of help in this task).
%
% \subsection{Options}
%
% Options can be given as package options or later using:
% \begin{declcs}{epstopdfsetup} \M{key value list}
% \end{declcs}
% \LaTeX\ expands the option list before passing the option list
% to the package's option handling code. This can fail
% for option \xoption{suffix} if it contains some of the macros
% described below. Use \cs{epstopdfsetup} after the package is loaded.
% Or load package \xpackage{kvoptions-patch} before. This package
% is also loaded by option \xoption{patch} of package \xpackage{kvoptions}.
% \LaTeX's option code is redefined to respect key value options and
% let the values untouched.
%
% \begin{description}
% \def\It#1{\item[\xoption{#1}:]}^^A
% \It{update}
%   The conversion program is only called, if the target file
%   does not exist or is older than the source image file.
% \It{append}
%   Puts the extension \xfile{.eps} at the end of the graphics extension list
%   (default).
% \It{prepend}
%   Puts the extension \xfile{.eps} at the begin of the graphics extension list.
% \It{outdir}
%   The converted file may put in an other output directory.
%   The value of \xoption{outdir} must include the directory separator.
%   Example for the current directory:
%   \begin{quote}
%   |\epstopdfsetup{outdir=./}|
%   \end{quote}
%   For other directories ensure, that they can be found.
%   See \cs{graphicspath} or |TEXINPUTS|.
% \It{suffix}
%   This option takes a string that is put between the file name base
%   and the extension of the output file. Rationale: It can happen,
%   that a PDF file is the original file and the EPS file the
%   generated file. If now the package thinks, that the PDF file
%   is the generated file, it will `regnerate' it. But in reality
%   the original file is lost. Therefore I recommend to use this
%   option always to generate a separate name space for generated
%   files. Proposed value is \texttt{-generated} or \texttt{.generated}.
%   The suffix \texttt{.generated} will also work here without the need for
%   package \xpackage{grffile}).
%
%   Example:
%   \begin{quote}
%   |\epstopdfsetup{suffix=-generated}|\\
%   Then \xfile{foo.eps} is converted to \xfile{foo-generated.pdf}.
%   \end{quote}
%   \cs{SourceExt} can be used inside the suffix string.
%   It's will be replaced by the extension of the image source
%   file without the leading dot, for instance:
%   \begin{quote}
%   |\epstopdfsetup{suffix=-\SourceExt-converted-to}|\\
%   \xfile{foo.eps} $\Rightarrow$ \xfile{foo-eps-converted-to.pdf}
%   \end{quote}
%   See also the next option \xoption{prefersuffix} that
%   modifies the behaviour of option \xoption{suffix} in some
%   cases.
%
%   Default for \xoption{suffix} is `|-\SourceExt-converted-to|'.
%
% \It{prefersuffix}
%   If a suffix is set by option \xoption{suffix}, then there can be
%   two image file names that could be taken into account for inclusion:
%   A image file name with the suffix string inside
%   its name and a image file name without; e.g. for \xfile{foo.eps}
%   the names could be:
%   \begin{quote}
%   \xfile{foo-suffix.pdf}, \xfile{foo.pdf}
%   \end{quote}
%   If option \xoption{perfersuffix} is turned on, the
%   file \xfile{foo-suffix.pdf} and its generation is preferred
%   over using \xfile{foo.pdf}. Otherwise \xfile{foo.pdf} is included
%   without generating \xfile{foo-suffix.pdf}.
%   The default of option \xoption{prefersuffix} is \texttt{true}.
% \It{program@epstopdf}
%   The name for the conversion program from EPS to PDF,
%   default is `|epstopdf|'.
% \It{verbose}
%   It prints some information about the image in the \xfile{.log} file
%   (default).
% \end{description}
%
% \subsection{Configuration}
%
% \subsubsection{System configuration file \xfile{epstopdf-sys.cfg}}
%
% If \xfile{epstopdf-sys} exists it is loaded at the end of the
% package \xpackage{epstopdf-base} and before the user configuration
% file. It's intended for TeX distributors. Thus they could add
% additional conversion rules (e.g., .gif -> .png) or set options.
%
% \subsubsection{User configuration file \xfile{epstopdf.cfg}}
%
% A configuration file \xfile{epstopdf.cfg} is loaded at the end
% of the package if it exists. It can be used for changing
% the default option setting. Example:
% \begin{quote}
% |\epstopdfsetup{verbose=false}|
% \end{quote}
%
% \subsubsection{Conversion program}
%
% You can use \cs{DeclareGraphicsRule} in a similar way as the
% route via \xprogram{dvips} to specify the conversion
% command line.
% The conversion argument starts with a back tick, followed
% by the conversion command including parameters.
%
% The whole conversion argument should also be wrapped inside
% \cs{epstopdfcall}. This reduces the problem with packages
% (e.g. \xpackage{pst-pdf}) that use the conversion argument
% and expands it. Macros \cs{SourceFile}, \cs{OutputFile}, and
% \cs{SourceExt} are not defined outside \xpackage{epstopdf-base}'s
% \cs{Gin@setfile} and error messages because of undefined
% command names are the result. If \cs{epstopdfcall} detects that
% it is called outside \xpackage{epstopdf-base}'s \cs{Gin@setfile}
% then it replaces the conversion argument by
% package \xpackage{graphics}'s
% default, usually the image file.
%
% The following macros are available inside:
% \begin{description}
% \def\It#1{\item[\cs{#1}:]}
% \It{OutputFile}: output file name (with known path and extension)
% \It{SourceFile}: source file name (with known path and extension),
%                  usually the same as |#1|,
% \It{SourceExt}: source extension without leading dot.
% \end{description}
%
% \paragraph{Conversion from EPS to PDF.}
% Other programs than \xprogram{epstopdf} can be used to convert from
% EPS to PDF. Example that uses \xprogram{Ghostscript}:
% \begin{quote}
%   |\DeclareGraphicsRule{.eps}{pdf}{.pdf}{%|\\
%   |  \epstopdfcall{`ps2pdf -dEPSCrop #1 \noexpand\OutputFile}%|\\
%   |}|
% \end{quote}
% \cs{DeclareGraphicsRule} expands the argument, therefore \cs{noexpand}
% is necessary. As convenience package \xpackage{epstopdf-base} defines
% \cs{epstopdfDeclareGraphicsRule}. Then the conversion argument
% is not expanded, \cs{epstopdfcall} and the back tick are added:
% \begin{quote}
%   |\epstopdfDeclareGraphicsRule{.eps}{pdf}{.pdf}{%|\\
%   |  ps2pdf -dEPSCrop #1 \OutputFile|\\
%   |}|
% \end{quote}
%
% Also \cs{OutputFile} respects the setting of option \xoption{outdir}.
%
% \subsection{Other image formats}
%
% The support that package \xpackage{epstopdf} implements is not
% limited to EPS files. Other image conversions can be declared.
% The following example shows it for GIF images under Unix with
% ImageMagick's \xprogram{convert}:
%\begin{quote}
%\begin{verbatim}
%\epstopdfDeclareGraphicsRule{.gif}{png}{.png}{%
%  convert #1 \OutputFile
%}
%\end{verbatim}
%\end{quote}
% The file extension \xfile{.gif} can be added to the extension list that
% package \xpackage{graphics} searches if the file extension
% is not given in \cs{includegraphics}.
% The list can be set by \cs{GraphicsExtensions}.
% \begin{quote}
% |\AppendGraphicsExtensions{.gif}|\\
% or\\
% |\PrependGraphicsExtensions{.gif}|
% \end{quote}
%
% \StopEventually{
% }
%
% \section{Implementation}
%
%    \begin{macrocode}
%<*package>
%    \end{macrocode}
%
% \subsection{Wrapper package}
%
%    Reload check, especially if the package is not used with \LaTeX.
%    \begin{macrocode}
\begingroup\catcode61\catcode48\catcode32=10\relax%
  \catcode13=5 % ^^M
  \endlinechar=13 %
  \catcode35=6 % #
  \catcode39=12 % '
  \catcode44=12 % ,
  \catcode45=12 % -
  \catcode46=12 % .
  \catcode58=12 % :
  \catcode64=11 % @
  \catcode123=1 % {
  \catcode125=2 % }
  \expandafter\let\expandafter\x\csname ver@epstopdf.sty\endcsname
  \ifx\x\relax % plain-TeX, first loading
  \else
    \def\empty{}%
    \ifx\x\empty % LaTeX, first loading,
      % variable is initialized, but \ProvidesPackage not yet seen
    \else
      \expandafter\ifx\csname PackageInfo\endcsname\relax
        \def\x#1#2{%
          \immediate\write-1{Package #1 Info: #2.}%
        }%
      \else
        \def\x#1#2{\PackageInfo{#1}{#2, stopped}}%
      \fi
      \x{epstopdf}{The package is already loaded}%
      \aftergroup\endinput
    \fi
  \fi
\endgroup%
%    \end{macrocode}
%    Package identification:
%    \begin{macrocode}
\begingroup\catcode61\catcode48\catcode32=10\relax%
  \catcode13=5 % ^^M
  \endlinechar=13 %
  \catcode35=6 % #
  \catcode39=12 % '
  \catcode40=12 % (
  \catcode41=12 % )
  \catcode44=12 % ,
  \catcode45=12 % -
  \catcode46=12 % .
  \catcode47=12 % /
  \catcode58=12 % :
  \catcode64=11 % @
  \catcode91=12 % [
  \catcode93=12 % ]
  \catcode123=1 % {
  \catcode125=2 % }
  \expandafter\ifx\csname ProvidesPackage\endcsname\relax
    \def\x#1#2#3[#4]{\endgroup
      \immediate\write-1{Package: #3 #4}%
      \xdef#1{#4}%
    }%
  \else
    \def\x#1#2[#3]{\endgroup
      #2[{#3}]%
      \ifx#1\@undefined
        \xdef#1{#3}%
      \fi
      \ifx#1\relax
        \xdef#1{#3}%
      \fi
    }%
  \fi
\expandafter\x\csname ver@epstopdf.sty\endcsname
\ProvidesPackage{epstopdf}%
  [2016/05/15 v2.6 Conversion with epstopdf on the fly (HO)]%
%    \end{macrocode}
%    Larger catcode set because of configuration files needed.
%    \begin{macrocode}
\begingroup\catcode61\catcode48\catcode32=10\relax%
  \catcode13=5 % ^^M
  \endlinechar=13 %
  \catcode123=1 % {
  \catcode125=2 % }
  \catcode64=11 % @
  \def\x{\endgroup
    \expandafter\edef\csname ETE@AtEnd\endcsname{%
      \endlinechar=\the\endlinechar\relax
      \catcode13=\the\catcode13\relax
      \catcode32=\the\catcode32\relax
      \catcode35=\the\catcode35\relax
      \catcode61=\the\catcode61\relax
      \catcode64=\the\catcode64\relax
      \catcode123=\the\catcode123\relax
      \catcode125=\the\catcode125\relax
    }%
  }%
\x\catcode61\catcode48\catcode32=10\relax%
\catcode13=5 % ^^M
\endlinechar=13 %
\catcode35=6 % #
\catcode64=11 % @
\catcode123=1 % {
\catcode125=2 % }
\def\TMP@EnsureCode#1#2{%
  \edef\ETE@AtEnd{%
    \ETE@AtEnd
    \catcode#1=\the\catcode#1\relax
  }%
  \catcode#1=#2\relax
}
\TMP@EnsureCode{39}{12}% '
\TMP@EnsureCode{40}{12}% (
\TMP@EnsureCode{41}{12}% )
\TMP@EnsureCode{42}{12}% *
\TMP@EnsureCode{45}{12}% -
\TMP@EnsureCode{47}{12}% /
\TMP@EnsureCode{91}{12}% [
\TMP@EnsureCode{93}{12}% ]
\edef\ETE@AtEnd{\ETE@AtEnd\noexpand\endinput}
%    \end{macrocode}
%
%    \begin{macrocode}
\let\ETE@SavedAtEnd\ETE@AtEnd
\RequirePackage{epstopdf-base}[2016/05/15]
\let\ETE@AtEnd\ETE@SavedAtEnd
%    \end{macrocode}
%
% \subsubsection{Option handling}
%
%    \begin{macrocode}
\DeclareOption*{%
  \expandafter\epstopdfsetup\expandafter{\CurrentOption}%
}
\ProcessOptions*\relax
%    \end{macrocode}
%
%    \begin{macrocode}
\ETE@AtEnd%
%</package>
%    \end{macrocode}
%
% \subsection{Base package}
%
%    \begin{macrocode}
%<*base>
%    \end{macrocode}
%
% \subsection{Preparations}
%
% \subsubsection{Relead check and identification}
%
%    Reload check, especially if the package is not used with \LaTeX.
%    \begin{macrocode}
\begingroup\catcode61\catcode48\catcode32=10\relax%
  \catcode13=5 % ^^M
  \endlinechar=13 %
  \catcode35=6 % #
  \catcode39=12 % '
  \catcode44=12 % ,
  \catcode45=12 % -
  \catcode46=12 % .
  \catcode58=12 % :
  \catcode64=11 % @
  \catcode123=1 % {
  \catcode125=2 % }
  \expandafter\let\expandafter\x\csname ver@epstopdf-base.sty\endcsname
  \ifx\x\relax % plain-TeX, first loading
  \else
    \def\empty{}%
    \ifx\x\empty % LaTeX, first loading,
      % variable is initialized, but \ProvidesPackage not yet seen
    \else
      \expandafter\ifx\csname PackageInfo\endcsname\relax
        \def\x#1#2{%
          \immediate\write-1{Package #1 Info: #2.}%
        }%
      \else
        \def\x#1#2{\PackageInfo{#1}{#2, stopped}}%
      \fi
      \x{epstopdf-base}{The package is already loaded}%
      \aftergroup\endinput
    \fi
  \fi
\endgroup%
%    \end{macrocode}
%    Package identification:
%    \begin{macrocode}
\begingroup\catcode61\catcode48\catcode32=10\relax%
  \catcode13=5 % ^^M
  \endlinechar=13 %
  \catcode35=6 % #
  \catcode39=12 % '
  \catcode40=12 % (
  \catcode41=12 % )
  \catcode44=12 % ,
  \catcode45=12 % -
  \catcode46=12 % .
  \catcode47=12 % /
  \catcode58=12 % :
  \catcode64=11 % @
  \catcode91=12 % [
  \catcode93=12 % ]
  \catcode123=1 % {
  \catcode125=2 % }
  \expandafter\ifx\csname ProvidesPackage\endcsname\relax
    \def\x#1#2#3[#4]{\endgroup
      \immediate\write-1{Package: #3 #4}%
      \xdef#1{#4}%
    }%
  \else
    \def\x#1#2[#3]{\endgroup
      #2[{#3}]%
      \ifx#1\@undefined
        \xdef#1{#3}%
      \fi
      \ifx#1\relax
        \xdef#1{#3}%
      \fi
    }%
  \fi
\expandafter\x\csname ver@epstopdf-base.sty\endcsname
\ProvidesPackage{epstopdf-base}%
  [2016/05/15 v2.6 Base part for package epstopdf]%
%    \end{macrocode}
%
% \subsubsection{Catcodes}
%
%    \begin{macrocode}
\begingroup\catcode61\catcode48\catcode32=10\relax%
  \catcode13=5 % ^^M
  \endlinechar=13 %
  \catcode123=1 % {
  \catcode125=2 % }
  \catcode64=11 % @
  \def\x{\endgroup
    \expandafter\edef\csname ETE@AtEnd\endcsname{%
      \endlinechar=\the\endlinechar\relax
      \catcode13=\the\catcode13\relax
      \catcode32=\the\catcode32\relax
      \catcode35=\the\catcode35\relax
      \catcode61=\the\catcode61\relax
      \catcode64=\the\catcode64\relax
      \catcode123=\the\catcode123\relax
      \catcode125=\the\catcode125\relax
    }%
  }%
\x\catcode61\catcode48\catcode32=10\relax%
\catcode13=5 % ^^M
\endlinechar=13 %
\catcode35=6 % #
\catcode64=11 % @
\catcode123=1 % {
\catcode125=2 % }
\def\TMP@EnsureCode#1#2{%
  \edef\ETE@AtEnd{%
    \ETE@AtEnd
    \catcode#1=\the\catcode#1\relax
  }%
  \catcode#1=#2\relax
}
\TMP@EnsureCode{33}{12}% !
\TMP@EnsureCode{39}{12}% '
\TMP@EnsureCode{42}{12}% *
\TMP@EnsureCode{44}{12}% ,
\TMP@EnsureCode{45}{12}% -
\TMP@EnsureCode{46}{12}% .
\TMP@EnsureCode{47}{12}% /
\TMP@EnsureCode{58}{12}% :
\TMP@EnsureCode{60}{12}% <
\TMP@EnsureCode{62}{12}% >
\TMP@EnsureCode{96}{12}% `
\edef\ETE@AtEnd{\ETE@AtEnd\noexpand\endinput}
%    \end{macrocode}
%
% \subsubsection{Load packages}
%
%    \begin{macrocode}
\RequirePackage{infwarerr}[2007/09/09]
\RequirePackage{grfext}\relax
\RequirePackage{kvoptions}[2007/10/02]
\RequirePackage{pdftexcmds}[2007/11/11]
%    \end{macrocode}
%
% \subsection{Checks}
%
% Check, whether package graphics is loaded
% (also graphicx loads graphics).
% Because miniltx.tex does not know \cs{@ifpackageloaded}
% we test for \cs{Gin@setfile} instead.
%    \begin{macrocode}
\begingroup\expandafter\expandafter\expandafter\endgroup
\expandafter\ifx\csname Gin@setfile\endcsname\relax
  \@PackageWarningNoLine{epstopdf}{%
    No graphics package \string`graphic{s,x}\string' loaded%
  }%
  \newcommand*{\epstopdfsetup}[1]{}%
  \expandafter\ETE@AtEnd
\fi%
%    \end{macrocode}
%
% Check, whether pdftex.def is loaded. \cs{ver@pdftex.def} is not
% available with miniltx.tex, thus we test for \cs{Gin@driver}.
%    \begin{macrocode}
\begingroup
  \def\x{luatex.def}%
  \ifx\Gin@driver\x
  \else
  \def\x{pdftex.def}%
  \ifx\Gin@driver\x
  \else
    \@PackageWarningNoLine{epstopdf}{%
      Drivers other than `pdftex' and `luatex' are not supported%
    }%
    \endgroup
    \newcommand*{\epstopdfsetup}[1]{}%
    \expandafter\ETE@AtEnd
  \fi%
  \fi%
\endgroup
%    \end{macrocode}
%
% Check, whether the shell escape feature is enabled.
%
%    \begin{macrocode}
\begingroup
  \expandafter\ifx\csname pdf@shellescape\endcsname\relax
  \else
    \ifnum\pdf@shellescape>0 %
    \else
      \@PackageWarningNoLine{epstopdf}{%
        Shell escape feature is not enabled%
      }%
    \fi
  \fi
\endgroup
%    \end{macrocode}
%
% \subsection{Package loading}
%
%    \begin{macrocode}
%    \end{macrocode}
%
% \subsection{Options}
%
%    \begin{macrocode}
\SetupKeyvalOptions{family=ETE,prefix=ETE@}
\DeclareBoolOption{update}
\DeclareBoolOption{verbose}
\newif\ifETE@prepend
\DeclareVoidOption{prepend}{\ETE@prependtrue}
\DeclareVoidOption{append}{\ETE@prependfalse}
\DeclareStringOption{outdir}
\DeclareStringOption{suffix}
\DeclareBoolOption{prefersuffix}
\DeclareStringOption{program@epstopdf}
%    \end{macrocode}
%    Options \xoption{disable} and \xoption{enable} are for testing
%    only. Therefore they are not documented on purpose.
%    \begin{macrocode}
\DeclareBoolOption{disable}
\DeclareComplementaryOption{enable}{disable}
%    \end{macrocode}
%    \begin{macrocode}
\newcommand*{\epstopdfsetup}{\setkeys{ETE}}
%    \end{macrocode}
%
% \subsubsection{Default setting}
%
%    \begin{macrocode}
\epstopdfsetup{%
  verbose,%
  enable,%
  append,%
  update,%
  prefersuffix,%
  suffix=-\SourceExt-converted-to,%
  program@epstopdf=epstopdf%
}
%    \end{macrocode}
%
% \subsection{Make and verbose}
%
%    \begin{macrocode}
\begingroup\expandafter\expandafter\expandafter\endgroup
\expandafter\ifx\csname pdf@filemoddate\endcsname\relax
  \def\ETE@Make#1#2{%
    \ifETE@update
      \ETE@WarnModDate
    \fi
    \@firstofone
  }%
  \def\ETE@WarnModDate{%
    \@PackageWarningNoLine{epstopdf}{%
      \string\pdffilemoddate\space is not available,\MessageBreak
      option `update' will be ignored%
    }%
    \global\let\ETE@WarnModDate\relax
  }%
  \def\ETE@FileInfo#1#2{#1 file: <#2>}%
\else
  \def\ETE@Make#1#2{%
    \ifETE@update
      \ifnum\pdf@strcmp{\pdf@filemoddate{#1}}{\pdf@filemoddate{#2}}>0 %
        \expandafter\expandafter\expandafter\@firstofone
      \else
        \@PackageInfoNoLine{epstopdf}{%
          Output file is already uptodate%
        }%
        \expandafter\expandafter\expandafter\@gobble
      \fi
    \else
      \expandafter\@firstofone
    \fi
  }%
  \def\ETE@FileInfo#1#2{%
    #1 file: <#2>%
    \expandafter\expandafter\expandafter
    \ETE@Date\pdf@filemoddate{#2}\@nil
    \expandafter\expandafter\expandafter
    \ETE@Size\pdf@filesize{#2}\@nil
  }%
  \def\ETE@Date#1\@nil{%
    \ifx\\#1\\%
    \else
      \ETE@@Date#1\@nil
    \fi
  }%
  \def\ETE@@Date#1:#2#3#4#5#6#7#8#9{%
    \MessageBreak
    \@spaces\space\space\space date: #2#3#4#5-#6#7-#8#9 %
    \ETE@@Time
  }%
  \def\ETE@@Time#1#2#3#4#5#6#7\@nil{%
    #1#2:#3#4:#5#6%
  }%
  \def\ETE@Size#1\@nil{%
    \ifx\\#1\\%
    \else
      \MessageBreak
      \@spaces\space\space\space size: #1 bytes%
    \fi
  }%
\fi
%    \end{macrocode}
%
% \subsection{Adding conversion support}
%
%    Patch \cs{Gin@setfile} to execute |#3|, if it contains
%    a command.
%    \begin{macrocode}
\expandafter\ifx\csname ETE@OrgGin@setfile\endcsname\relax
  \let\ETE@OrgGin@setfile\Gin@setfile
\else
  \@PackageError{epstopdf}{%
    Command \string\ETE@OrgGin@setfile\space
    already defined.\MessageBreak
  }{%
    Probably some package has included the code of this package%
    \MessageBreak
    instead of using \string\RequirePackage{epstopdf}.%
    \MessageBreak
    \@ehc
  }%
\fi
\def\ETE@IfFileExists{%
  \begingroup\expandafter\expandafter\expandafter\endgroup
  \expandafter\ifx\csname grffile@IfFileExists\endcsname\relax
    \expandafter\IfFileExists
  \else
    \global\let\ETE@IfFileExists\grffile@IfFileExists
    \expandafter\grffile@IfFileExists
  \fi
}
\def\ETE@Skip#1#2\x\fi\fi{%
  \fi
  \fi
  \endgroup
  \fi
  \fi
  #1%
}
\newif\ifETE@InsideSetfile
\newcommand*{\epstopdfcall}[1]{%
  \ifETE@InsideSetfile
    \expandafter\@firstoftwo
  \else
    \expandafter\@secondoftwo
  \fi
  {`#1}%
  {\Gin@base\Gin@ext}%
}
\def\ETE@DefCommandLine#1{%
  \edef\CommandLine{\expandafter\fi\if`#1}%
}
\def\ETE@DefX#1{%
  \expandafter\expandafter\expandafter\def
  \expandafter\expandafter\expandafter\x
  \expandafter\expandafter\expandafter{%
    \expandafter\fi\if`#1\relax\else
  }%
}
\def\ETE@Gin@setfile#1#2#3{%
  \ifETE@disable
    \ETE@OrgGin@setfile{#1}{#2}{#3}%
  \else
    \begingroup
      \ETE@InsideSetfiletrue
      \ETE@DefX{#3}%
    \expandafter\endgroup
    \ifx\x\@empty
      \ETE@OrgGin@setfile{#1}{#2}{#3}%
    \else
      \begingroup
        \ETE@InsideSetfiletrue
        \def\GraphicsType{#1}%
        \def\GraphicsRead{#2}%
        \ifETE@prefersuffix
        \else
          \ifx\ETE@suffix\@empty
          \else
            \ETE@IfFileExists{\Gin@base\GraphicsRead}{%
              \ETE@Skip{%
                \ETE@OrgGin@setfile{#1}{#2}{\Gin@base#2}%
              }%
            }{%
              \let\next\relax
            }%
            \next
          \fi
        \fi
        \ifx\Gin@ext\relax
          \let\SourceExt\Gin@eext
          \def\SourceFile{\Gin@base\Gin@eext}%
        \else
          \let\SourceExt\Gin@ext
          \def\SourceFile{\Gin@base\Gin@ext}%
        \fi
        \edef\SourceExt{% remove dot
          \expandafter\@cdr\SourceExt\@empty\@nil
        }%
        \let\OutputDirectory\ETE@outdir
        \ifx\OutputDirectory\@empty
          \edef\OutputFile{\ETE@GenerateName{\Gin@base}{#2}}%
        \else
          \begingroup
            \filename@parse{\Gin@base#2}%
            \edef\x{\endgroup
              \def\noexpand\OutputFile{%
                \ETE@GenerateName{%
                  \OutputDirectory\filename@base
                }{#2}%
              }%
            }%
          \x
        \fi
        \ETE@DefCommandLine{#3}%
        \ifETE@verbose
          \@PackageInfo{epstopdf}{%
            \ETE@FileInfo{Source}\SourceFile\MessageBreak
            \ETE@FileInfo{Output}\OutputFile\MessageBreak
            Command: <\CommandLine>\MessageBreak
            \string\includegraphics
          }%
        \fi
        \ETE@Make\SourceFile\OutputFile{%
          \pdf@system{\CommandLine}%
          \ifETE@verbose
            \@PackageInfoNoLine{epstopdf}{%
              \ETE@FileInfo{Result}\OutputFile
            }%
          \fi
        }%
        \edef\x{\endgroup
          \ifx\OutputDirectory\@empty
          \else
            \def\noexpand\Gin@base{%
              \OutputDirectory\noexpand\filename@base
            }%
          \fi
          \ifx\ETE@suffix\@empty
          \else
            \edef\noexpand\Gin@base{%
              \noexpand\Gin@base\ETE@suffix
            }%
          \fi
          \noexpand\ETE@OrgGin@setfile{%
            \GraphicsType
          }{%
            \GraphicsRead
          }{%
            \OutputFile
          }%
        }%
      \x
    \fi
  \fi
}
\let\Gin@setfile\ETE@Gin@setfile
\def\ETE@GenerateName#1#2{%
  #1\ETE@suffix#2%
}
%    \end{macrocode}
%
% \subsection{Declare graphics rule}
%
%    \begin{macrocode}
\newcommand*{\epstopdfDeclareGraphicsRule}[4]{%
  \ifx\\#4\\%
    \@PackageError{epstopdf-base}{%
      Conversion command is missing%
    }\@ehc
  \else
    \begingroup
      \@ifundefined{Gin@rule@#1}{%
      }{%
        \@PackageInfo{epstopdf-base}{%
          Redefining graphics rule for `#1'%
        }%
      }%
    \endgroup
    \@namedef{Gin@rule@#1}##1{{#2}{#3}{\epstopdfcall{#4}}}%
  \fi
}
%    \end{macrocode}
%
% \cs{DeclareGraphicsRule} for .eps
%    \begin{macrocode}
\epstopdfDeclareGraphicsRule{.eps}{pdf}{.pdf}{%
  \ETE@epstopdf{#1}%
}
\def\ETE@epstopdf#1{%
  \ETE@program@epstopdf\space
  \ifcase\ifx\OutputDirectory\@empty
           \ifx\ETE@suffix\@empty
             1%
           \fi
         \fi
         0 %
    --outfile=\OutputFile\space
  \fi
  #1%
}
%    \end{macrocode}
%
%    \begin{macrocode}
\ifETE@prepend
  \expandafter\PrependGraphicsExtensions
\else
  \expandafter\AppendGraphicsExtensions
\fi
{.eps}
\let\ETE@prepend\@undefined
\DeclareVoidOption{prepend}{%
  \PrependGraphicsExtensions{.eps}%
}
\let\ETE@append\@undefined
\DeclareVoidOption{append}{%
  \AppendGraphicsExtensions{.eps}%
}
%    \end{macrocode}
%
%    \begin{macrocode}
\InputIfFileExists{epstopdf-sys.cfg}{}{}
\InputIfFileExists{epstopdf.cfg}{}{}
%    \end{macrocode}
%
%    \begin{macrocode}
\ETE@AtEnd%
%</base>
%    \end{macrocode}
%
% \section{Test}
%
% \subsection{Preface for standard catcode check}
%
%    \begin{macrocode}
%<*test1>
\input miniltx.tex\relax
\def\Gin@driver{pdftex.def}
\input graphicx.sty\relax
\resetatcatcode
%</test1>
%    \end{macrocode}
%
% \subsection{Catcode checks for loading}
%
%    \begin{macrocode}
%<*test1>
%    \end{macrocode}
%    \begin{macrocode}
\catcode`\{=1 %
\catcode`\}=2 %
\catcode`\#=6 %
\catcode`\@=11 %
\expandafter\ifx\csname count@\endcsname\relax
  \countdef\count@=255 %
\fi
\expandafter\ifx\csname @gobble\endcsname\relax
  \long\def\@gobble#1{}%
\fi
\expandafter\ifx\csname @firstofone\endcsname\relax
  \long\def\@firstofone#1{#1}%
\fi
\expandafter\ifx\csname loop\endcsname\relax
  \expandafter\@firstofone
\else
  \expandafter\@gobble
\fi
{%
  \def\loop#1\repeat{%
    \def\body{#1}%
    \iterate
  }%
  \def\iterate{%
    \body
      \let\next\iterate
    \else
      \let\next\relax
    \fi
    \next
  }%
  \let\repeat=\fi
}%
\def\RestoreCatcodes{}
\count@=0 %
\loop
  \edef\RestoreCatcodes{%
    \RestoreCatcodes
    \catcode\the\count@=\the\catcode\count@\relax
  }%
\ifnum\count@<255 %
  \advance\count@ 1 %
\repeat

\def\RangeCatcodeInvalid#1#2{%
  \count@=#1\relax
  \loop
    \catcode\count@=15 %
  \ifnum\count@<#2\relax
    \advance\count@ 1 %
  \repeat
}
\def\RangeCatcodeCheck#1#2#3{%
  \count@=#1\relax
  \loop
    \ifnum#3=\catcode\count@
    \else
      \errmessage{%
        Character \the\count@\space
        with wrong catcode \the\catcode\count@\space
        instead of \number#3%
      }%
    \fi
  \ifnum\count@<#2\relax
    \advance\count@ 1 %
  \repeat
}
\def\space{ }
\expandafter\ifx\csname LoadCommand\endcsname\relax
  \def\LoadCommand{\input epstopdf.sty\relax}%
\fi
\def\Test{%
  \RangeCatcodeInvalid{0}{47}%
  \RangeCatcodeInvalid{58}{64}%
  \RangeCatcodeInvalid{91}{96}%
  \RangeCatcodeInvalid{123}{255}%
  \catcode`\@=12 %
  \catcode`\\=0 %
  \catcode`\%=14 %
  \LoadCommand
  \RangeCatcodeCheck{0}{36}{15}%
  \RangeCatcodeCheck{37}{37}{14}%
  \RangeCatcodeCheck{38}{47}{15}%
  \RangeCatcodeCheck{48}{57}{12}%
  \RangeCatcodeCheck{58}{63}{15}%
  \RangeCatcodeCheck{64}{64}{12}%
  \RangeCatcodeCheck{65}{90}{11}%
  \RangeCatcodeCheck{91}{91}{15}%
  \RangeCatcodeCheck{92}{92}{0}%
  \RangeCatcodeCheck{93}{96}{15}%
  \RangeCatcodeCheck{97}{122}{11}%
  \RangeCatcodeCheck{123}{255}{15}%
  \RestoreCatcodes
}
\Test
\csname @@end\endcsname
\end
%    \end{macrocode}
%    \begin{macrocode}
%</test1>
%    \end{macrocode}
%
% \section{Installation}
%
% \subsection{Download}
%
% \paragraph{Package.} This package is available on
% CTAN\footnote{\url{http://ctan.org/pkg/epstopdf}}:
% \begin{description}
% \item[\CTAN{macros/latex/contrib/oberdiek/epstopdf.dtx}] The source file.
% \item[\CTAN{macros/latex/contrib/oberdiek/epstopdf.pdf}] Documentation.
% \end{description}
%
%
% \paragraph{Bundle.} All the packages of the bundle `oberdiek'
% are also available in a TDS compliant ZIP archive. There
% the packages are already unpacked and the documentation files
% are generated. The files and directories obey the TDS standard.
% \begin{description}
% \item[\CTAN{install/macros/latex/contrib/oberdiek.tds.zip}]
% \end{description}
% \emph{TDS} refers to the standard ``A Directory Structure
% for \TeX\ Files'' (\CTAN{tds/tds.pdf}). Directories
% with \xfile{texmf} in their name are usually organized this way.
%
% \subsection{Bundle installation}
%
% \paragraph{Unpacking.} Unpack the \xfile{oberdiek.tds.zip} in the
% TDS tree (also known as \xfile{texmf} tree) of your choice.
% Example (linux):
% \begin{quote}
%   |unzip oberdiek.tds.zip -d ~/texmf|
% \end{quote}
%
% \paragraph{Script installation.}
% Check the directory \xfile{TDS:scripts/oberdiek/} for
% scripts that need further installation steps.
% Package \xpackage{attachfile2} comes with the Perl script
% \xfile{pdfatfi.pl} that should be installed in such a way
% that it can be called as \texttt{pdfatfi}.
% Example (linux):
% \begin{quote}
%   |chmod +x scripts/oberdiek/pdfatfi.pl|\\
%   |cp scripts/oberdiek/pdfatfi.pl /usr/local/bin/|
% \end{quote}
%
% \subsection{Package installation}
%
% \paragraph{Unpacking.} The \xfile{.dtx} file is a self-extracting
% \docstrip\ archive. The files are extracted by running the
% \xfile{.dtx} through \plainTeX:
% \begin{quote}
%   \verb|tex epstopdf.dtx|
% \end{quote}
%
% \paragraph{TDS.} Now the different files must be moved into
% the different directories in your installation TDS tree
% (also known as \xfile{texmf} tree):
% \begin{quote}
% \def\t{^^A
% \begin{tabular}{@{}>{\ttfamily}l@{ $\rightarrow$ }>{\ttfamily}l@{}}
%   epstopdf.sty & tex/latex/oberdiek/epstopdf.sty\\
%   epstopdf-base.sty & tex/latex/oberdiek/epstopdf-base.sty\\
%   epstopdf.pdf & doc/latex/oberdiek/epstopdf.pdf\\
%   test/epstopdf-test1.tex & doc/latex/oberdiek/test/epstopdf-test1.tex\\
%   epstopdf.dtx & source/latex/oberdiek/epstopdf.dtx\\
% \end{tabular}^^A
% }^^A
% \sbox0{\t}^^A
% \ifdim\wd0>\linewidth
%   \begingroup
%     \advance\linewidth by\leftmargin
%     \advance\linewidth by\rightmargin
%   \edef\x{\endgroup
%     \def\noexpand\lw{\the\linewidth}^^A
%   }\x
%   \def\lwbox{^^A
%     \leavevmode
%     \hbox to \linewidth{^^A
%       \kern-\leftmargin\relax
%       \hss
%       \usebox0
%       \hss
%       \kern-\rightmargin\relax
%     }^^A
%   }^^A
%   \ifdim\wd0>\lw
%     \sbox0{\small\t}^^A
%     \ifdim\wd0>\linewidth
%       \ifdim\wd0>\lw
%         \sbox0{\footnotesize\t}^^A
%         \ifdim\wd0>\linewidth
%           \ifdim\wd0>\lw
%             \sbox0{\scriptsize\t}^^A
%             \ifdim\wd0>\linewidth
%               \ifdim\wd0>\lw
%                 \sbox0{\tiny\t}^^A
%                 \ifdim\wd0>\linewidth
%                   \lwbox
%                 \else
%                   \usebox0
%                 \fi
%               \else
%                 \lwbox
%               \fi
%             \else
%               \usebox0
%             \fi
%           \else
%             \lwbox
%           \fi
%         \else
%           \usebox0
%         \fi
%       \else
%         \lwbox
%       \fi
%     \else
%       \usebox0
%     \fi
%   \else
%     \lwbox
%   \fi
% \else
%   \usebox0
% \fi
% \end{quote}
% If you have a \xfile{docstrip.cfg} that configures and enables \docstrip's
% TDS installing feature, then some files can already be in the right
% place, see the documentation of \docstrip.
%
% \subsection{Refresh file name databases}
%
% If your \TeX~distribution
% (\teTeX, \mikTeX, \dots) relies on file name databases, you must refresh
% these. For example, \teTeX\ users run \verb|texhash| or
% \verb|mktexlsr|.
%
% \subsection{Some details for the interested}
%
% \paragraph{Attached source.}
%
% The PDF documentation on CTAN also includes the
% \xfile{.dtx} source file. It can be extracted by
% AcrobatReader 6 or higher. Another option is \textsf{pdftk},
% e.g. unpack the file into the current directory:
% \begin{quote}
%   \verb|pdftk epstopdf.pdf unpack_files output .|
% \end{quote}
%
% \paragraph{Unpacking with \LaTeX.}
% The \xfile{.dtx} chooses its action depending on the format:
% \begin{description}
% \item[\plainTeX:] Run \docstrip\ and extract the files.
% \item[\LaTeX:] Generate the documentation.
% \end{description}
% If you insist on using \LaTeX\ for \docstrip\ (really,
% \docstrip\ does not need \LaTeX), then inform the autodetect routine
% about your intention:
% \begin{quote}
%   \verb|latex \let\install=y\input{epstopdf.dtx}|
% \end{quote}
% Do not forget to quote the argument according to the demands
% of your shell.
%
% \paragraph{Generating the documentation.}
% You can use both the \xfile{.dtx} or the \xfile{.drv} to generate
% the documentation. The process can be configured by the
% configuration file \xfile{ltxdoc.cfg}. For instance, put this
% line into this file, if you want to have A4 as paper format:
% \begin{quote}
%   \verb|\PassOptionsToClass{a4paper}{article}|
% \end{quote}
% An example follows how to generate the
% documentation with pdf\LaTeX:
% \begin{quote}
%\begin{verbatim}
%pdflatex epstopdf.dtx
%makeindex -s gind.ist epstopdf.idx
%pdflatex epstopdf.dtx
%makeindex -s gind.ist epstopdf.idx
%pdflatex epstopdf.dtx
%\end{verbatim}
% \end{quote}
%
% \section{Catalogue}
%
% The following XML file can be used as source for the
% \href{http://mirror.ctan.org/help/Catalogue/catalogue.html}{\TeX\ Catalogue}.
% The elements \texttt{caption} and \texttt{description} are imported
% from the original XML file from the Catalogue.
% The name of the XML file in the Catalogue is \xfile{epstopdf-pkg.xml}.
%    \begin{macrocode}
%<*catalogue>
<?xml version='1.0' encoding='us-ascii'?>
<!DOCTYPE entry SYSTEM 'catalogue.dtd'>
<entry datestamp='$Date$' modifier='$Author$' id='epstopdf-pkg'>
  <name>epstopdf-pkg</name>
  <caption>Call epstopdf "on the fly"</caption>
  <authorref id='auth:oberdiek'/>
  <copyright owner='Heiko Oberdiek' year='2001,2006-2010'/>
  <license type='lppl1.3'/>
  <version number='2.6'/>
  <description>
    The package adds support for EPS files in the
    <xref refid='graphicx'>graphicx</xref> package when running under
    PDFTeX.  If an EPS graphic is detected, the package spawns a
    process to convert the EPS to PDF, using the script
    <xref refid='epstopdf'>epstopdf</xref>.  This of course requires
    that shell escape is enabled for the PDFTeX run.
    <p/>
    The package is part of the <xref refid='oberdiek'>oberdiek</xref>
    bundle.
  </description>
  <documentation details='Package documentation'
      href='ctan:/macros/latex/contrib/oberdiek/epstopdf.pdf'/>
  <ctan file='true' path='/macros/latex/contrib/oberdiek/epstopdf.dtx'/>
  <miktex location='oberdiek'/>
  <texlive location='oberdiek'/>
  <install path='/macros/latex/contrib/oberdiek/oberdiek.tds.zip'/>
</entry>
%</catalogue>
%    \end{macrocode}
%
% \begin{History}
%   \begin{Version}{2001/01/06 v1.0}
%   \item
%     First public version,
%     published in the \pdfTeX\ mailing list.
%   \end{Version}
%   \begin{Version}{2001/02/04 v1.1}
%   \item
%     Minor documentation update.
%   \item
%     CTAN.
%   \end{Version}
%   \begin{Version}{2006/02/20 v1.2}
%   \item
%     DTX framework.
%   \item
%     Compatibility for \xfile{miniltx.tex}.
%   \end{Version}
%   \begin{Version}{2006/08/26 v1.3}
%   \item
%     Check for \cs{write18} if available and print a warning
%     if the feature is not enabled.
%   \end{Version}
%   \begin{Version}{2007/04/26 v1.4}
%   \item
%     Documentation rewritten and extended.
%   \end{Version}
%   \begin{Version}{2007/10/02 v1.5}
%   \item
%     New option \xoption{update}: If the converted file exists, it will be only
%     converted if it is out of date.
%   \item
%     Updating the extension list is delegated to package \xpackage{grfext}.
%     Fine tuning is done by the new options \xoption{append}, \xoption{prepend}.
%   \item
%     New option \xoption{outdir} for changing the output directory.
%   \item
%     New option \xoption{verbose}.
%   \item
%     \cs{SourceFile} and \cs{OutputFile} introduced.
%   \item
%     Configuration file support added.
%   \end{Version}
%   \begin{Version}{2007/11/11 v1.6}
%   \item
%     Use of package \xpackage{pdftexcmds} for \LuaTeX\ support.
%   \end{Version}
%   \begin{Version}{2008/05/06 v1.7}
%   \item
%     Warning messages uses ``loaded'' instead of ``found''.
%   \end{Version}
%   \begin{Version}{2009/03/01 v1.8}
%   \item
%     Warning message for missing \xfile{pdftex.def} changed.
%   \end{Version}
%   \begin{Version}{2009/07/06 v1.9}
%   \item
%     Option \xoption{suffix} added.
%   \end{Version}
%   \begin{Version}{2009/07/07 v1.10}
%   \item
%     \cs{SourceExt} added.
%   \item
%     If option \xoption{suffix} is set, the inclusion of
%     an image without the suffix namespace is preferred over generating the
%     the image within the suffix namespace.
%   \end{Version}
%   \begin{Version}{2009/07/12 v2.0}
%   \item
%     New default settings.
%   \item
%     Package is split into \xpackage{epstopdf} that only takes package
%     options and loads \xpackage{epstopdf-base} that does the work.
%   \item
%     \cs{epstopdfDeclareGraphicsRule} and \cs{epstopdfcall} added.
%   \item
%     \xfile{epstopdf-sys.cfg} is loaded before \xfile{epstopdf.cfg}
%     if \xfile{epstopdf-sys.cfg} exists.
%   \end{Version}
%   \begin{Version}{2009/07/15 v2.1}
%   \item
%     Default setting: \xoption{verbose} is now turned on as the
%     documentation for v2.0 said.
%   \item
%     Documentation fixes.
%   \end{Version}
%   \begin{Version}{2009/07/16 v2.2}
%   \item
%     Fixed redefined \cs{Gin@setfile}.
%   \item
%     Documentation extended for package options.
%   \end{Version}
%   \begin{Version}{2009/09/24 v2.3}
%   \item
%     Bug fix for the case that both option \xoption{suffix} and
%     \xoption{outdir} are used.
%   \end{Version}
%   \begin{Version}{2009/10/17 v2.4}
%   \item
%     The name of the program `epstopdf' can be configured via
%     the new option \xoption{program@epstopdf}.
%   \end{Version}
%   \begin{Version}{2016/05/15 v2.5}
%   \item
%     Wording of warning message fixed (Karl Berry).
%   \item
%     \cs{ETE@Gin@setfile} added (Karl Berry).
%   \end{Version}
%   \begin{Version}{2016/05/15 v2.6}
%   \item
%     luaTeX compatibility
%   \end{Version}
% \end{History}
%
% \PrintIndex
%
% \Finale
\endinput
|
% \end{quote}
% Do not forget to quote the argument according to the demands
% of your shell.
%
% \paragraph{Generating the documentation.}
% You can use both the \xfile{.dtx} or the \xfile{.drv} to generate
% the documentation. The process can be configured by the
% configuration file \xfile{ltxdoc.cfg}. For instance, put this
% line into this file, if you want to have A4 as paper format:
% \begin{quote}
%   \verb|\PassOptionsToClass{a4paper}{article}|
% \end{quote}
% An example follows how to generate the
% documentation with pdf\LaTeX:
% \begin{quote}
%\begin{verbatim}
%pdflatex epstopdf.dtx
%makeindex -s gind.ist epstopdf.idx
%pdflatex epstopdf.dtx
%makeindex -s gind.ist epstopdf.idx
%pdflatex epstopdf.dtx
%\end{verbatim}
% \end{quote}
%
% \section{Catalogue}
%
% The following XML file can be used as source for the
% \href{http://mirror.ctan.org/help/Catalogue/catalogue.html}{\TeX\ Catalogue}.
% The elements \texttt{caption} and \texttt{description} are imported
% from the original XML file from the Catalogue.
% The name of the XML file in the Catalogue is \xfile{epstopdf-pkg.xml}.
%    \begin{macrocode}
%<*catalogue>
<?xml version='1.0' encoding='us-ascii'?>
<!DOCTYPE entry SYSTEM 'catalogue.dtd'>
<entry datestamp='$Date$' modifier='$Author$' id='epstopdf-pkg'>
  <name>epstopdf-pkg</name>
  <caption>Call epstopdf "on the fly"</caption>
  <authorref id='auth:oberdiek'/>
  <copyright owner='Heiko Oberdiek' year='2001,2006-2010'/>
  <license type='lppl1.3'/>
  <version number='2.6'/>
  <description>
    The package adds support for EPS files in the
    <xref refid='graphicx'>graphicx</xref> package when running under
    PDFTeX.  If an EPS graphic is detected, the package spawns a
    process to convert the EPS to PDF, using the script
    <xref refid='epstopdf'>epstopdf</xref>.  This of course requires
    that shell escape is enabled for the PDFTeX run.
    <p/>
    The package is part of the <xref refid='oberdiek'>oberdiek</xref>
    bundle.
  </description>
  <documentation details='Package documentation'
      href='ctan:/macros/latex/contrib/oberdiek/epstopdf.pdf'/>
  <ctan file='true' path='/macros/latex/contrib/oberdiek/epstopdf.dtx'/>
  <miktex location='oberdiek'/>
  <texlive location='oberdiek'/>
  <install path='/macros/latex/contrib/oberdiek/oberdiek.tds.zip'/>
</entry>
%</catalogue>
%    \end{macrocode}
%
% \begin{History}
%   \begin{Version}{2001/01/06 v1.0}
%   \item
%     First public version,
%     published in the \pdfTeX\ mailing list.
%   \end{Version}
%   \begin{Version}{2001/02/04 v1.1}
%   \item
%     Minor documentation update.
%   \item
%     CTAN.
%   \end{Version}
%   \begin{Version}{2006/02/20 v1.2}
%   \item
%     DTX framework.
%   \item
%     Compatibility for \xfile{miniltx.tex}.
%   \end{Version}
%   \begin{Version}{2006/08/26 v1.3}
%   \item
%     Check for \cs{write18} if available and print a warning
%     if the feature is not enabled.
%   \end{Version}
%   \begin{Version}{2007/04/26 v1.4}
%   \item
%     Documentation rewritten and extended.
%   \end{Version}
%   \begin{Version}{2007/10/02 v1.5}
%   \item
%     New option \xoption{update}: If the converted file exists, it will be only
%     converted if it is out of date.
%   \item
%     Updating the extension list is delegated to package \xpackage{grfext}.
%     Fine tuning is done by the new options \xoption{append}, \xoption{prepend}.
%   \item
%     New option \xoption{outdir} for changing the output directory.
%   \item
%     New option \xoption{verbose}.
%   \item
%     \cs{SourceFile} and \cs{OutputFile} introduced.
%   \item
%     Configuration file support added.
%   \end{Version}
%   \begin{Version}{2007/11/11 v1.6}
%   \item
%     Use of package \xpackage{pdftexcmds} for \LuaTeX\ support.
%   \end{Version}
%   \begin{Version}{2008/05/06 v1.7}
%   \item
%     Warning messages uses ``loaded'' instead of ``found''.
%   \end{Version}
%   \begin{Version}{2009/03/01 v1.8}
%   \item
%     Warning message for missing \xfile{pdftex.def} changed.
%   \end{Version}
%   \begin{Version}{2009/07/06 v1.9}
%   \item
%     Option \xoption{suffix} added.
%   \end{Version}
%   \begin{Version}{2009/07/07 v1.10}
%   \item
%     \cs{SourceExt} added.
%   \item
%     If option \xoption{suffix} is set, the inclusion of
%     an image without the suffix namespace is preferred over generating the
%     the image within the suffix namespace.
%   \end{Version}
%   \begin{Version}{2009/07/12 v2.0}
%   \item
%     New default settings.
%   \item
%     Package is split into \xpackage{epstopdf} that only takes package
%     options and loads \xpackage{epstopdf-base} that does the work.
%   \item
%     \cs{epstopdfDeclareGraphicsRule} and \cs{epstopdfcall} added.
%   \item
%     \xfile{epstopdf-sys.cfg} is loaded before \xfile{epstopdf.cfg}
%     if \xfile{epstopdf-sys.cfg} exists.
%   \end{Version}
%   \begin{Version}{2009/07/15 v2.1}
%   \item
%     Default setting: \xoption{verbose} is now turned on as the
%     documentation for v2.0 said.
%   \item
%     Documentation fixes.
%   \end{Version}
%   \begin{Version}{2009/07/16 v2.2}
%   \item
%     Fixed redefined \cs{Gin@setfile}.
%   \item
%     Documentation extended for package options.
%   \end{Version}
%   \begin{Version}{2009/09/24 v2.3}
%   \item
%     Bug fix for the case that both option \xoption{suffix} and
%     \xoption{outdir} are used.
%   \end{Version}
%   \begin{Version}{2009/10/17 v2.4}
%   \item
%     The name of the program `epstopdf' can be configured via
%     the new option \xoption{program@epstopdf}.
%   \end{Version}
%   \begin{Version}{2016/05/15 v2.5}
%   \item
%     Wording of warning message fixed (Karl Berry).
%   \item
%     \cs{ETE@Gin@setfile} added (Karl Berry).
%   \end{Version}
%   \begin{Version}{2016/05/15 v2.6}
%   \item
%     luaTeX compatibility
%   \end{Version}
% \end{History}
%
% \PrintIndex
%
% \Finale
\endinput
|
% \end{quote}
% Do not forget to quote the argument according to the demands
% of your shell.
%
% \paragraph{Generating the documentation.}
% You can use both the \xfile{.dtx} or the \xfile{.drv} to generate
% the documentation. The process can be configured by the
% configuration file \xfile{ltxdoc.cfg}. For instance, put this
% line into this file, if you want to have A4 as paper format:
% \begin{quote}
%   \verb|\PassOptionsToClass{a4paper}{article}|
% \end{quote}
% An example follows how to generate the
% documentation with pdf\LaTeX:
% \begin{quote}
%\begin{verbatim}
%pdflatex epstopdf.dtx
%makeindex -s gind.ist epstopdf.idx
%pdflatex epstopdf.dtx
%makeindex -s gind.ist epstopdf.idx
%pdflatex epstopdf.dtx
%\end{verbatim}
% \end{quote}
%
% \section{Catalogue}
%
% The following XML file can be used as source for the
% \href{http://mirror.ctan.org/help/Catalogue/catalogue.html}{\TeX\ Catalogue}.
% The elements \texttt{caption} and \texttt{description} are imported
% from the original XML file from the Catalogue.
% The name of the XML file in the Catalogue is \xfile{epstopdf-pkg.xml}.
%    \begin{macrocode}
%<*catalogue>
<?xml version='1.0' encoding='us-ascii'?>
<!DOCTYPE entry SYSTEM 'catalogue.dtd'>
<entry datestamp='$Date$' modifier='$Author$' id='epstopdf-pkg'>
  <name>epstopdf-pkg</name>
  <caption>Call epstopdf "on the fly"</caption>
  <authorref id='auth:oberdiek'/>
  <copyright owner='Heiko Oberdiek' year='2001,2006-2010'/>
  <license type='lppl1.3'/>
  <version number='2.6'/>
  <description>
    The package adds support for EPS files in the
    <xref refid='graphicx'>graphicx</xref> package when running under
    PDFTeX.  If an EPS graphic is detected, the package spawns a
    process to convert the EPS to PDF, using the script
    <xref refid='epstopdf'>epstopdf</xref>.  This of course requires
    that shell escape is enabled for the PDFTeX run.
    <p/>
    The package is part of the <xref refid='oberdiek'>oberdiek</xref>
    bundle.
  </description>
  <documentation details='Package documentation'
      href='ctan:/macros/latex/contrib/oberdiek/epstopdf.pdf'/>
  <ctan file='true' path='/macros/latex/contrib/oberdiek/epstopdf.dtx'/>
  <miktex location='oberdiek'/>
  <texlive location='oberdiek'/>
  <install path='/macros/latex/contrib/oberdiek/oberdiek.tds.zip'/>
</entry>
%</catalogue>
%    \end{macrocode}
%
% \begin{History}
%   \begin{Version}{2001/01/06 v1.0}
%   \item
%     First public version,
%     published in the \pdfTeX\ mailing list.
%   \end{Version}
%   \begin{Version}{2001/02/04 v1.1}
%   \item
%     Minor documentation update.
%   \item
%     CTAN.
%   \end{Version}
%   \begin{Version}{2006/02/20 v1.2}
%   \item
%     DTX framework.
%   \item
%     Compatibility for \xfile{miniltx.tex}.
%   \end{Version}
%   \begin{Version}{2006/08/26 v1.3}
%   \item
%     Check for \cs{write18} if available and print a warning
%     if the feature is not enabled.
%   \end{Version}
%   \begin{Version}{2007/04/26 v1.4}
%   \item
%     Documentation rewritten and extended.
%   \end{Version}
%   \begin{Version}{2007/10/02 v1.5}
%   \item
%     New option \xoption{update}: If the converted file exists, it will be only
%     converted if it is out of date.
%   \item
%     Updating the extension list is delegated to package \xpackage{grfext}.
%     Fine tuning is done by the new options \xoption{append}, \xoption{prepend}.
%   \item
%     New option \xoption{outdir} for changing the output directory.
%   \item
%     New option \xoption{verbose}.
%   \item
%     \cs{SourceFile} and \cs{OutputFile} introduced.
%   \item
%     Configuration file support added.
%   \end{Version}
%   \begin{Version}{2007/11/11 v1.6}
%   \item
%     Use of package \xpackage{pdftexcmds} for \LuaTeX\ support.
%   \end{Version}
%   \begin{Version}{2008/05/06 v1.7}
%   \item
%     Warning messages uses ``loaded'' instead of ``found''.
%   \end{Version}
%   \begin{Version}{2009/03/01 v1.8}
%   \item
%     Warning message for missing \xfile{pdftex.def} changed.
%   \end{Version}
%   \begin{Version}{2009/07/06 v1.9}
%   \item
%     Option \xoption{suffix} added.
%   \end{Version}
%   \begin{Version}{2009/07/07 v1.10}
%   \item
%     \cs{SourceExt} added.
%   \item
%     If option \xoption{suffix} is set, the inclusion of
%     an image without the suffix namespace is preferred over generating the
%     the image within the suffix namespace.
%   \end{Version}
%   \begin{Version}{2009/07/12 v2.0}
%   \item
%     New default settings.
%   \item
%     Package is split into \xpackage{epstopdf} that only takes package
%     options and loads \xpackage{epstopdf-base} that does the work.
%   \item
%     \cs{epstopdfDeclareGraphicsRule} and \cs{epstopdfcall} added.
%   \item
%     \xfile{epstopdf-sys.cfg} is loaded before \xfile{epstopdf.cfg}
%     if \xfile{epstopdf-sys.cfg} exists.
%   \end{Version}
%   \begin{Version}{2009/07/15 v2.1}
%   \item
%     Default setting: \xoption{verbose} is now turned on as the
%     documentation for v2.0 said.
%   \item
%     Documentation fixes.
%   \end{Version}
%   \begin{Version}{2009/07/16 v2.2}
%   \item
%     Fixed redefined \cs{Gin@setfile}.
%   \item
%     Documentation extended for package options.
%   \end{Version}
%   \begin{Version}{2009/09/24 v2.3}
%   \item
%     Bug fix for the case that both option \xoption{suffix} and
%     \xoption{outdir} are used.
%   \end{Version}
%   \begin{Version}{2009/10/17 v2.4}
%   \item
%     The name of the program `epstopdf' can be configured via
%     the new option \xoption{program@epstopdf}.
%   \end{Version}
%   \begin{Version}{2016/05/15 v2.5}
%   \item
%     Wording of warning message fixed (Karl Berry).
%   \item
%     \cs{ETE@Gin@setfile} added (Karl Berry).
%   \end{Version}
%   \begin{Version}{2016/05/15 v2.6}
%   \item
%     luaTeX compatibility
%   \end{Version}
% \end{History}
%
% \PrintIndex
%
% \Finale
\endinput

%        (quote the arguments according to the demands of your shell)
%
% Documentation:
%    (a) If epstopdf.drv is present:
%           latex epstopdf.drv
%    (b) Without epstopdf.drv:
%           latex epstopdf.dtx; ...
%    The class ltxdoc loads the configuration file ltxdoc.cfg
%    if available. Here you can specify further options, e.g.
%    use A4 as paper format:
%       \PassOptionsToClass{a4paper}{article}
%
%    Programm calls to get the documentation (example):
%       pdflatex epstopdf.dtx
%       makeindex -s gind.ist epstopdf.idx
%       pdflatex epstopdf.dtx
%       makeindex -s gind.ist epstopdf.idx
%       pdflatex epstopdf.dtx
%
% Installation:
%    TDS:tex/latex/oberdiek/epstopdf.sty
%    TDS:tex/latex/oberdiek/epstopdf-base.sty
%    TDS:doc/latex/oberdiek/epstopdf.pdf
%    TDS:doc/latex/oberdiek/test/epstopdf-test1.tex
%    TDS:source/latex/oberdiek/epstopdf.dtx
%
%<*ignore>
\begingroup
  \catcode123=1 %
  \catcode125=2 %
  \def\x{LaTeX2e}%
\expandafter\endgroup
\ifcase 0\ifx\install y1\fi\expandafter
         \ifx\csname processbatchFile\endcsname\relax\else1\fi
         \ifx\fmtname\x\else 1\fi\relax
\else\csname fi\endcsname
%</ignore>
%<*install>
\input docstrip.tex
\Msg{************************************************************************}
\Msg{* Installation}
\Msg{* Package: epstopdf 2016/05/15 v2.6 Conversion with epstopdf on the fly (HO)}
\Msg{************************************************************************}

\keepsilent
\askforoverwritefalse

\let\MetaPrefix\relax
\preamble

This is a generated file.

Project: epstopdf
Version: 2016/05/15 v2.6

Copyright (C) 2001, 2006-2010 by
   Heiko Oberdiek <heiko.oberdiek at googlemail.com>

This work may be distributed and/or modified under the
conditions of the LaTeX Project Public License, either
version 1.3c of this license or (at your option) any later
version. This version of this license is in
   http://www.latex-project.org/lppl/lppl-1-3c.txt
and the latest version of this license is in
   http://www.latex-project.org/lppl.txt
and version 1.3 or later is part of all distributions of
LaTeX version 2005/12/01 or later.

This work has the LPPL maintenance status "maintained".

This Current Maintainer of this work is Heiko Oberdiek.

This work consists of the main source file epstopdf.dtx
and the derived files
   epstopdf.sty, epstopdf.pdf, epstopdf.ins, epstopdf.drv,
   epstopdf-base.sty, epstopdf-test1.tex.

\endpreamble
\let\MetaPrefix\DoubleperCent

\generate{%
  \file{epstopdf.ins}{\from{epstopdf.dtx}{install}}%
  \file{epstopdf.drv}{\from{epstopdf.dtx}{driver}}%
  \usedir{tex/latex/oberdiek}%
  \file{epstopdf.sty}{\from{epstopdf.dtx}{package}}%
  \file{epstopdf-base.sty}{\from{epstopdf.dtx}{base}}%
%  \usedir{doc/latex/oberdiek/test}%
%  \file{epstopdf-test1.tex}{\from{epstopdf.dtx}{test1}}%
  \nopreamble
  \nopostamble
%  \usedir{source/latex/oberdiek/catalogue}%
%  \file{epstopdf-pkg.xml}{\from{epstopdf.dtx}{catalogue}}%
}

\catcode32=13\relax% active space
\let =\space%
\Msg{************************************************************************}
\Msg{*}
\Msg{* To finish the installation you have to move the following}
\Msg{* files into a directory searched by TeX:}
\Msg{*}
\Msg{*     epstopdf.sty, epstopdf-base.sty}
\Msg{*}
\Msg{* To produce the documentation run the file `epstopdf.drv'}
\Msg{* through LaTeX.}
\Msg{*}
\Msg{* Happy TeXing!}
\Msg{*}
\Msg{************************************************************************}

\endbatchfile
%</install>
%<*ignore>
\fi
%</ignore>
%<*driver>
\NeedsTeXFormat{LaTeX2e}
\ProvidesFile{epstopdf.drv}%
  [2016/05/15 v2.6 Conversion with epstopdf on the fly (HO)]%
\documentclass{ltxdoc}
\usepackage{holtxdoc}[2011/11/22]
\begin{document}
  \DocInput{epstopdf.dtx}%
\end{document}
%</driver>
% \fi
%
%
% \CharacterTable
%  {Upper-case    \A\B\C\D\E\F\G\H\I\J\K\L\M\N\O\P\Q\R\S\T\U\V\W\X\Y\Z
%   Lower-case    \a\b\c\d\e\f\g\h\i\j\k\l\m\n\o\p\q\r\s\t\u\v\w\x\y\z
%   Digits        \0\1\2\3\4\5\6\7\8\9
%   Exclamation   \!     Double quote  \"     Hash (number) \#
%   Dollar        \$     Percent       \%     Ampersand     \&
%   Acute accent  \'     Left paren    \(     Right paren   \)
%   Asterisk      \*     Plus          \+     Comma         \,
%   Minus         \-     Point         \.     Solidus       \/
%   Colon         \:     Semicolon     \;     Less than     \<
%   Equals        \=     Greater than  \>     Question mark \?
%   Commercial at \@     Left bracket  \[     Backslash     \\
%   Right bracket \]     Circumflex    \^     Underscore    \_
%   Grave accent  \`     Left brace    \{     Vertical bar  \|
%   Right brace   \}     Tilde         \~}
%
% \GetFileInfo{epstopdf.drv}
%
% \title{The \xpackage{epstopdf} package}
% \date{2016/05/15 v2.6}
% \author{Heiko Oberdiek\thanks
% {Please report any issues at https://github.com/ho-tex/oberdiek/issues}\\
% \xemail{heiko.oberdiek at googlemail.com}}
%
% \maketitle
%
% \begin{abstract}
% This packages adds support of handling eps images
% to package \xpackage{graphics} or \xpackage{graphicx}
% with option \xoption{pdftex}. If an eps image is detected,
% epstopdf is automatically called to convert it to pdf format.
% \end{abstract}
%
% \tableofcontents
%
% \providecommand*{\xprogram}[1]{\textsf{#1}}
%
% \section{Documentation}
%
% \subsection{Introduction}
%
% \LaTeX\ provides its graphics bundle to include graphics files.
% Both packages \xpackage{graphics} or \xpackage{graphicx} may
% be used. the latter one loads the first and adds options
% in key value style for \cs{includegraphics}.
%
% Usually the drivers do not support all kind of graphics files.
% Other image types must be converted, before they become
% usuable. In case of driver \xprogram{dvips}, the graphics
% rule may contain a conversion rule. Then all that package
% \xpackage{graphics} must know is the bounding box, the command
% is passed to \xprogram{dvips} that calls it and embeds the
% converted image.
%
% However, \pdfTeX\ has its driver for PDF output already
% build in. It's graphics inclusion commands (\cs{pdfximage})
% does not allow the execution of external commands.
% Therefore commands in the last argument of
% \cs{DeclareGraphicsRule} were of no use.
% But external programs can be called within \pdfTeX.
% This feature is called ``shell escape'' or ``write 18''
% and must usually enabled explicitely because of security reasons.
% Now, this package \xpackage{epstopdf} hooks
% into package \xpackage{graphics}' code to catch
% that argument with the external command and executes it
% to convert the graphics file to a supported format and
% passes the control of graphics inclusion back to package
% \xpackage{graphics}.
%
% \subsection{Requirements}
%
% \begin{itemize}
% \item
%   The feature \cs{write18} must be enabled. This allows the
%   running of external programs during \TeX's compile run.
%   Keep in mind that this is a security risk. The feature is an
%   addition to \cs{TeX}. Mik\TeX, te\TeX, \TeX\ Live support it.
%   In Web2C based \TeX\ distributions (te\TeX, \TeX\ Live) it can be
%   enabled in the configuration file \xfile{texmf.cnf}:
%   \begin{quote}
%     |shell_escape = 1|
%   \end{quote}
%   Because of the security risk, it is better to do it on the
%   command line only:
%   \begin{quote}
%   \texttt{--shell-escape} (te\TeX, \TeX\ Live)\\
%   \texttt{--enable-write18} (MiK\TeX)
%   \end{quote}
%   Example:
%   \begin{quote}
%     |pdflatex -shell-escape test.tex|
%   \end{quote}
% \item
%   The program \xprogram{epstopdf} for the conversion from EPS
%   to PDF. However, other programs can
%   be used and configured by \cs{DeclareGraphicsRule}.
%   Example:
%   \begin{quote}
%   |\epstopdfDeclareGraphicsRule{.eps}{pdf}{.pdf}{%|\\
%   |  ps2pdf -dEPSCrop #1 \OutputFile|\\
%   |}|
%   \end{quote}
% \end{itemize}
%
% \subsection{Usage}
%
% The package is loaded after graphic\{s,x\}, e.g.:
% \begin{quote}
%   |\usepackage[pdftex]{graphicx}|\\
%   |\usepackage{epstopdf}|
% \end{quote}
% Now images with file name extension \xfile{.eps} are detected
% and supported using \cs{includegraphics}.
%
% If the graphics file name is explicitly specified with extension
% \xfile{.eps} the new rule for EPS files is called and
% the conversion performed. If option \xoption{update} is in force
% then the conversion step is dropped if the target file already
% exists and is not older then the EPS file.
%
% The situation is more complicate if the graphics file is
% given without file name extension. Then the \xpackage{graphics}
% package must search for a supported image file.
% The possible extensions are stored in the graphics extension
% list, that can be set by \cs{DeclareGraphicsExtensions}.
% The algorithm:
% \begin{quote}
%    \def\>{\mbox{}\quad}
%    function search( \meta{filebase} )\\
%    \>foreach \meta{ext} in \meta{graphics extensions}\\
%    \>\>foreach \meta{dir} in \meta{current directory}, \meta{\cs{graphicspath}}\\
%    \>\>\>\meta{file} := \meta{dir} + \meta{filebase} + \meta{ext}\\
%    \>\>\>if exist \meta{file}\\
%    \>\>\>\>return found\\
%    \>return not found
% \end{quote}
% Package \xpackage{epstopdf} puts \xfile{.eps} at the end of the
% graphics extension search list. This is the behaviour of option
% \xoption{append} that is enabled by default.
% That means, the conversion is called last
% unless a supported file type cannot be found earlier. This avoids
% unnecessary conversion steps that slow down the \LaTeX\ run.
% If you want to use option \xoption{update} and your \pdfTeX\ supports
% it, then an outdated PDF file also would be found earlier unless
% \xoption{suffix} is used that is the default since version 2.0.
%
% With an empty option \xoption{suffix} and option
% \xoption{prepend} there is a risk that an original PDF file is overwritten:
%
% If the original image file is the PDF file and there is also a generated
% EPS file, then the original PDF file can be regenerated (depending
% on the option settings) and the original PDF file gets lost.
% Therefore option \xoption{suffix} is introduced in version 1.9
% to create a separate name space for generated output files.
%
% \textbf{Note:} Usually the conversion program needs the exact location
% of the image file. Usually the current directory works.
% Also if the image file is found using \cs{graphicspath}, the
% location is known. However, if the image is somewhere
% in a directory of environment variable |TEXINPUTS|, then
% the package does not know the exact location and the
% conversion program will not find the image file unless it
% implements a search using |TEXINPUTS| (program \xprogram{kpsewhich}
% may be of help in this task).
%
% \subsection{Options}
%
% Options can be given as package options or later using:
% \begin{declcs}{epstopdfsetup} \M{key value list}
% \end{declcs}
% \LaTeX\ expands the option list before passing the option list
% to the package's option handling code. This can fail
% for option \xoption{suffix} if it contains some of the macros
% described below. Use \cs{epstopdfsetup} after the package is loaded.
% Or load package \xpackage{kvoptions-patch} before. This package
% is also loaded by option \xoption{patch} of package \xpackage{kvoptions}.
% \LaTeX's option code is redefined to respect key value options and
% let the values untouched.
%
% \begin{description}
% \def\It#1{\item[\xoption{#1}:]}^^A
% \It{update}
%   The conversion program is only called, if the target file
%   does not exist or is older than the source image file.
% \It{append}
%   Puts the extension \xfile{.eps} at the end of the graphics extension list
%   (default).
% \It{prepend}
%   Puts the extension \xfile{.eps} at the begin of the graphics extension list.
% \It{outdir}
%   The converted file may put in an other output directory.
%   The value of \xoption{outdir} must include the directory separator.
%   Example for the current directory:
%   \begin{quote}
%   |\epstopdfsetup{outdir=./}|
%   \end{quote}
%   For other directories ensure, that they can be found.
%   See \cs{graphicspath} or |TEXINPUTS|.
% \It{suffix}
%   This option takes a string that is put between the file name base
%   and the extension of the output file. Rationale: It can happen,
%   that a PDF file is the original file and the EPS file the
%   generated file. If now the package thinks, that the PDF file
%   is the generated file, it will `regnerate' it. But in reality
%   the original file is lost. Therefore I recommend to use this
%   option always to generate a separate name space for generated
%   files. Proposed value is \texttt{-generated} or \texttt{.generated}.
%   The suffix \texttt{.generated} will also work here without the need for
%   package \xpackage{grffile}).
%
%   Example:
%   \begin{quote}
%   |\epstopdfsetup{suffix=-generated}|\\
%   Then \xfile{foo.eps} is converted to \xfile{foo-generated.pdf}.
%   \end{quote}
%   \cs{SourceExt} can be used inside the suffix string.
%   It's will be replaced by the extension of the image source
%   file without the leading dot, for instance:
%   \begin{quote}
%   |\epstopdfsetup{suffix=-\SourceExt-converted-to}|\\
%   \xfile{foo.eps} $\Rightarrow$ \xfile{foo-eps-converted-to.pdf}
%   \end{quote}
%   See also the next option \xoption{prefersuffix} that
%   modifies the behaviour of option \xoption{suffix} in some
%   cases.
%
%   Default for \xoption{suffix} is `|-\SourceExt-converted-to|'.
%
% \It{prefersuffix}
%   If a suffix is set by option \xoption{suffix}, then there can be
%   two image file names that could be taken into account for inclusion:
%   A image file name with the suffix string inside
%   its name and a image file name without; e.g. for \xfile{foo.eps}
%   the names could be:
%   \begin{quote}
%   \xfile{foo-suffix.pdf}, \xfile{foo.pdf}
%   \end{quote}
%   If option \xoption{perfersuffix} is turned on, the
%   file \xfile{foo-suffix.pdf} and its generation is preferred
%   over using \xfile{foo.pdf}. Otherwise \xfile{foo.pdf} is included
%   without generating \xfile{foo-suffix.pdf}.
%   The default of option \xoption{prefersuffix} is \texttt{true}.
% \It{program@epstopdf}
%   The name for the conversion program from EPS to PDF,
%   default is `|epstopdf|'.
% \It{verbose}
%   It prints some information about the image in the \xfile{.log} file
%   (default).
% \end{description}
%
% \subsection{Configuration}
%
% \subsubsection{System configuration file \xfile{epstopdf-sys.cfg}}
%
% If \xfile{epstopdf-sys} exists it is loaded at the end of the
% package \xpackage{epstopdf-base} and before the user configuration
% file. It's intended for TeX distributors. Thus they could add
% additional conversion rules (e.g., .gif -> .png) or set options.
%
% \subsubsection{User configuration file \xfile{epstopdf.cfg}}
%
% A configuration file \xfile{epstopdf.cfg} is loaded at the end
% of the package if it exists. It can be used for changing
% the default option setting. Example:
% \begin{quote}
% |\epstopdfsetup{verbose=false}|
% \end{quote}
%
% \subsubsection{Conversion program}
%
% You can use \cs{DeclareGraphicsRule} in a similar way as the
% route via \xprogram{dvips} to specify the conversion
% command line.
% The conversion argument starts with a back tick, followed
% by the conversion command including parameters.
%
% The whole conversion argument should also be wrapped inside
% \cs{epstopdfcall}. This reduces the problem with packages
% (e.g. \xpackage{pst-pdf}) that use the conversion argument
% and expands it. Macros \cs{SourceFile}, \cs{OutputFile}, and
% \cs{SourceExt} are not defined outside \xpackage{epstopdf-base}'s
% \cs{Gin@setfile} and error messages because of undefined
% command names are the result. If \cs{epstopdfcall} detects that
% it is called outside \xpackage{epstopdf-base}'s \cs{Gin@setfile}
% then it replaces the conversion argument by
% package \xpackage{graphics}'s
% default, usually the image file.
%
% The following macros are available inside:
% \begin{description}
% \def\It#1{\item[\cs{#1}:]}
% \It{OutputFile}: output file name (with known path and extension)
% \It{SourceFile}: source file name (with known path and extension),
%                  usually the same as |#1|,
% \It{SourceExt}: source extension without leading dot.
% \end{description}
%
% \paragraph{Conversion from EPS to PDF.}
% Other programs than \xprogram{epstopdf} can be used to convert from
% EPS to PDF. Example that uses \xprogram{Ghostscript}:
% \begin{quote}
%   |\DeclareGraphicsRule{.eps}{pdf}{.pdf}{%|\\
%   |  \epstopdfcall{`ps2pdf -dEPSCrop #1 \noexpand\OutputFile}%|\\
%   |}|
% \end{quote}
% \cs{DeclareGraphicsRule} expands the argument, therefore \cs{noexpand}
% is necessary. As convenience package \xpackage{epstopdf-base} defines
% \cs{epstopdfDeclareGraphicsRule}. Then the conversion argument
% is not expanded, \cs{epstopdfcall} and the back tick are added:
% \begin{quote}
%   |\epstopdfDeclareGraphicsRule{.eps}{pdf}{.pdf}{%|\\
%   |  ps2pdf -dEPSCrop #1 \OutputFile|\\
%   |}|
% \end{quote}
%
% Also \cs{OutputFile} respects the setting of option \xoption{outdir}.
%
% \subsection{Other image formats}
%
% The support that package \xpackage{epstopdf} implements is not
% limited to EPS files. Other image conversions can be declared.
% The following example shows it for GIF images under Unix with
% ImageMagick's \xprogram{convert}:
%\begin{quote}
%\begin{verbatim}
%\epstopdfDeclareGraphicsRule{.gif}{png}{.png}{%
%  convert #1 \OutputFile
%}
%\end{verbatim}
%\end{quote}
% The file extension \xfile{.gif} can be added to the extension list that
% package \xpackage{graphics} searches if the file extension
% is not given in \cs{includegraphics}.
% The list can be set by \cs{GraphicsExtensions}.
% \begin{quote}
% |\AppendGraphicsExtensions{.gif}|\\
% or\\
% |\PrependGraphicsExtensions{.gif}|
% \end{quote}
%
% \StopEventually{
% }
%
% \section{Implementation}
%
%    \begin{macrocode}
%<*package>
%    \end{macrocode}
%
% \subsection{Wrapper package}
%
%    Reload check, especially if the package is not used with \LaTeX.
%    \begin{macrocode}
\begingroup\catcode61\catcode48\catcode32=10\relax%
  \catcode13=5 % ^^M
  \endlinechar=13 %
  \catcode35=6 % #
  \catcode39=12 % '
  \catcode44=12 % ,
  \catcode45=12 % -
  \catcode46=12 % .
  \catcode58=12 % :
  \catcode64=11 % @
  \catcode123=1 % {
  \catcode125=2 % }
  \expandafter\let\expandafter\x\csname ver@epstopdf.sty\endcsname
  \ifx\x\relax % plain-TeX, first loading
  \else
    \def\empty{}%
    \ifx\x\empty % LaTeX, first loading,
      % variable is initialized, but \ProvidesPackage not yet seen
    \else
      \expandafter\ifx\csname PackageInfo\endcsname\relax
        \def\x#1#2{%
          \immediate\write-1{Package #1 Info: #2.}%
        }%
      \else
        \def\x#1#2{\PackageInfo{#1}{#2, stopped}}%
      \fi
      \x{epstopdf}{The package is already loaded}%
      \aftergroup\endinput
    \fi
  \fi
\endgroup%
%    \end{macrocode}
%    Package identification:
%    \begin{macrocode}
\begingroup\catcode61\catcode48\catcode32=10\relax%
  \catcode13=5 % ^^M
  \endlinechar=13 %
  \catcode35=6 % #
  \catcode39=12 % '
  \catcode40=12 % (
  \catcode41=12 % )
  \catcode44=12 % ,
  \catcode45=12 % -
  \catcode46=12 % .
  \catcode47=12 % /
  \catcode58=12 % :
  \catcode64=11 % @
  \catcode91=12 % [
  \catcode93=12 % ]
  \catcode123=1 % {
  \catcode125=2 % }
  \expandafter\ifx\csname ProvidesPackage\endcsname\relax
    \def\x#1#2#3[#4]{\endgroup
      \immediate\write-1{Package: #3 #4}%
      \xdef#1{#4}%
    }%
  \else
    \def\x#1#2[#3]{\endgroup
      #2[{#3}]%
      \ifx#1\@undefined
        \xdef#1{#3}%
      \fi
      \ifx#1\relax
        \xdef#1{#3}%
      \fi
    }%
  \fi
\expandafter\x\csname ver@epstopdf.sty\endcsname
\ProvidesPackage{epstopdf}%
  [2016/05/15 v2.6 Conversion with epstopdf on the fly (HO)]%
%    \end{macrocode}
%    Larger catcode set because of configuration files needed.
%    \begin{macrocode}
\begingroup\catcode61\catcode48\catcode32=10\relax%
  \catcode13=5 % ^^M
  \endlinechar=13 %
  \catcode123=1 % {
  \catcode125=2 % }
  \catcode64=11 % @
  \def\x{\endgroup
    \expandafter\edef\csname ETE@AtEnd\endcsname{%
      \endlinechar=\the\endlinechar\relax
      \catcode13=\the\catcode13\relax
      \catcode32=\the\catcode32\relax
      \catcode35=\the\catcode35\relax
      \catcode61=\the\catcode61\relax
      \catcode64=\the\catcode64\relax
      \catcode123=\the\catcode123\relax
      \catcode125=\the\catcode125\relax
    }%
  }%
\x\catcode61\catcode48\catcode32=10\relax%
\catcode13=5 % ^^M
\endlinechar=13 %
\catcode35=6 % #
\catcode64=11 % @
\catcode123=1 % {
\catcode125=2 % }
\def\TMP@EnsureCode#1#2{%
  \edef\ETE@AtEnd{%
    \ETE@AtEnd
    \catcode#1=\the\catcode#1\relax
  }%
  \catcode#1=#2\relax
}
\TMP@EnsureCode{39}{12}% '
\TMP@EnsureCode{40}{12}% (
\TMP@EnsureCode{41}{12}% )
\TMP@EnsureCode{42}{12}% *
\TMP@EnsureCode{45}{12}% -
\TMP@EnsureCode{47}{12}% /
\TMP@EnsureCode{91}{12}% [
\TMP@EnsureCode{93}{12}% ]
\edef\ETE@AtEnd{\ETE@AtEnd\noexpand\endinput}
%    \end{macrocode}
%
%    \begin{macrocode}
\let\ETE@SavedAtEnd\ETE@AtEnd
\RequirePackage{epstopdf-base}[2016/05/15]
\let\ETE@AtEnd\ETE@SavedAtEnd
%    \end{macrocode}
%
% \subsubsection{Option handling}
%
%    \begin{macrocode}
\DeclareOption*{%
  \expandafter\epstopdfsetup\expandafter{\CurrentOption}%
}
\ProcessOptions*\relax
%    \end{macrocode}
%
%    \begin{macrocode}
\ETE@AtEnd%
%</package>
%    \end{macrocode}
%
% \subsection{Base package}
%
%    \begin{macrocode}
%<*base>
%    \end{macrocode}
%
% \subsection{Preparations}
%
% \subsubsection{Relead check and identification}
%
%    Reload check, especially if the package is not used with \LaTeX.
%    \begin{macrocode}
\begingroup\catcode61\catcode48\catcode32=10\relax%
  \catcode13=5 % ^^M
  \endlinechar=13 %
  \catcode35=6 % #
  \catcode39=12 % '
  \catcode44=12 % ,
  \catcode45=12 % -
  \catcode46=12 % .
  \catcode58=12 % :
  \catcode64=11 % @
  \catcode123=1 % {
  \catcode125=2 % }
  \expandafter\let\expandafter\x\csname ver@epstopdf-base.sty\endcsname
  \ifx\x\relax % plain-TeX, first loading
  \else
    \def\empty{}%
    \ifx\x\empty % LaTeX, first loading,
      % variable is initialized, but \ProvidesPackage not yet seen
    \else
      \expandafter\ifx\csname PackageInfo\endcsname\relax
        \def\x#1#2{%
          \immediate\write-1{Package #1 Info: #2.}%
        }%
      \else
        \def\x#1#2{\PackageInfo{#1}{#2, stopped}}%
      \fi
      \x{epstopdf-base}{The package is already loaded}%
      \aftergroup\endinput
    \fi
  \fi
\endgroup%
%    \end{macrocode}
%    Package identification:
%    \begin{macrocode}
\begingroup\catcode61\catcode48\catcode32=10\relax%
  \catcode13=5 % ^^M
  \endlinechar=13 %
  \catcode35=6 % #
  \catcode39=12 % '
  \catcode40=12 % (
  \catcode41=12 % )
  \catcode44=12 % ,
  \catcode45=12 % -
  \catcode46=12 % .
  \catcode47=12 % /
  \catcode58=12 % :
  \catcode64=11 % @
  \catcode91=12 % [
  \catcode93=12 % ]
  \catcode123=1 % {
  \catcode125=2 % }
  \expandafter\ifx\csname ProvidesPackage\endcsname\relax
    \def\x#1#2#3[#4]{\endgroup
      \immediate\write-1{Package: #3 #4}%
      \xdef#1{#4}%
    }%
  \else
    \def\x#1#2[#3]{\endgroup
      #2[{#3}]%
      \ifx#1\@undefined
        \xdef#1{#3}%
      \fi
      \ifx#1\relax
        \xdef#1{#3}%
      \fi
    }%
  \fi
\expandafter\x\csname ver@epstopdf-base.sty\endcsname
\ProvidesPackage{epstopdf-base}%
  [2016/05/15 v2.6 Base part for package epstopdf]%
%    \end{macrocode}
%
% \subsubsection{Catcodes}
%
%    \begin{macrocode}
\begingroup\catcode61\catcode48\catcode32=10\relax%
  \catcode13=5 % ^^M
  \endlinechar=13 %
  \catcode123=1 % {
  \catcode125=2 % }
  \catcode64=11 % @
  \def\x{\endgroup
    \expandafter\edef\csname ETE@AtEnd\endcsname{%
      \endlinechar=\the\endlinechar\relax
      \catcode13=\the\catcode13\relax
      \catcode32=\the\catcode32\relax
      \catcode35=\the\catcode35\relax
      \catcode61=\the\catcode61\relax
      \catcode64=\the\catcode64\relax
      \catcode123=\the\catcode123\relax
      \catcode125=\the\catcode125\relax
    }%
  }%
\x\catcode61\catcode48\catcode32=10\relax%
\catcode13=5 % ^^M
\endlinechar=13 %
\catcode35=6 % #
\catcode64=11 % @
\catcode123=1 % {
\catcode125=2 % }
\def\TMP@EnsureCode#1#2{%
  \edef\ETE@AtEnd{%
    \ETE@AtEnd
    \catcode#1=\the\catcode#1\relax
  }%
  \catcode#1=#2\relax
}
\TMP@EnsureCode{33}{12}% !
\TMP@EnsureCode{39}{12}% '
\TMP@EnsureCode{42}{12}% *
\TMP@EnsureCode{44}{12}% ,
\TMP@EnsureCode{45}{12}% -
\TMP@EnsureCode{46}{12}% .
\TMP@EnsureCode{47}{12}% /
\TMP@EnsureCode{58}{12}% :
\TMP@EnsureCode{60}{12}% <
\TMP@EnsureCode{62}{12}% >
\TMP@EnsureCode{96}{12}% `
\edef\ETE@AtEnd{\ETE@AtEnd\noexpand\endinput}
%    \end{macrocode}
%
% \subsubsection{Load packages}
%
%    \begin{macrocode}
\RequirePackage{infwarerr}[2007/09/09]
\RequirePackage{grfext}\relax
\RequirePackage{kvoptions}[2007/10/02]
\RequirePackage{pdftexcmds}[2007/11/11]
%    \end{macrocode}
%
% \subsection{Checks}
%
% Check, whether package graphics is loaded
% (also graphicx loads graphics).
% Because miniltx.tex does not know \cs{@ifpackageloaded}
% we test for \cs{Gin@setfile} instead.
%    \begin{macrocode}
\begingroup\expandafter\expandafter\expandafter\endgroup
\expandafter\ifx\csname Gin@setfile\endcsname\relax
  \@PackageWarningNoLine{epstopdf}{%
    No graphics package \string`graphic{s,x}\string' loaded%
  }%
  \newcommand*{\epstopdfsetup}[1]{}%
  \expandafter\ETE@AtEnd
\fi%
%    \end{macrocode}
%
% Check, whether pdftex.def is loaded. \cs{ver@pdftex.def} is not
% available with miniltx.tex, thus we test for \cs{Gin@driver}.
%    \begin{macrocode}
\begingroup
  \def\x{luatex.def}%
  \ifx\Gin@driver\x
  \else
  \def\x{pdftex.def}%
  \ifx\Gin@driver\x
  \else
    \@PackageWarningNoLine{epstopdf}{%
      Drivers other than `pdftex' and `luatex' are not supported%
    }%
    \endgroup
    \newcommand*{\epstopdfsetup}[1]{}%
    \expandafter\ETE@AtEnd
  \fi%
  \fi%
\endgroup
%    \end{macrocode}
%
% Check, whether the shell escape feature is enabled.
%
%    \begin{macrocode}
\begingroup
  \expandafter\ifx\csname pdf@shellescape\endcsname\relax
  \else
    \ifnum\pdf@shellescape>0 %
    \else
      \@PackageWarningNoLine{epstopdf}{%
        Shell escape feature is not enabled%
      }%
    \fi
  \fi
\endgroup
%    \end{macrocode}
%
% \subsection{Package loading}
%
%    \begin{macrocode}
%    \end{macrocode}
%
% \subsection{Options}
%
%    \begin{macrocode}
\SetupKeyvalOptions{family=ETE,prefix=ETE@}
\DeclareBoolOption{update}
\DeclareBoolOption{verbose}
\newif\ifETE@prepend
\DeclareVoidOption{prepend}{\ETE@prependtrue}
\DeclareVoidOption{append}{\ETE@prependfalse}
\DeclareStringOption{outdir}
\DeclareStringOption{suffix}
\DeclareBoolOption{prefersuffix}
\DeclareStringOption{program@epstopdf}
%    \end{macrocode}
%    Options \xoption{disable} and \xoption{enable} are for testing
%    only. Therefore they are not documented on purpose.
%    \begin{macrocode}
\DeclareBoolOption{disable}
\DeclareComplementaryOption{enable}{disable}
%    \end{macrocode}
%    \begin{macrocode}
\newcommand*{\epstopdfsetup}{\setkeys{ETE}}
%    \end{macrocode}
%
% \subsubsection{Default setting}
%
%    \begin{macrocode}
\epstopdfsetup{%
  verbose,%
  enable,%
  append,%
  update,%
  prefersuffix,%
  suffix=-\SourceExt-converted-to,%
  program@epstopdf=epstopdf%
}
%    \end{macrocode}
%
% \subsection{Make and verbose}
%
%    \begin{macrocode}
\begingroup\expandafter\expandafter\expandafter\endgroup
\expandafter\ifx\csname pdf@filemoddate\endcsname\relax
  \def\ETE@Make#1#2{%
    \ifETE@update
      \ETE@WarnModDate
    \fi
    \@firstofone
  }%
  \def\ETE@WarnModDate{%
    \@PackageWarningNoLine{epstopdf}{%
      \string\pdffilemoddate\space is not available,\MessageBreak
      option `update' will be ignored%
    }%
    \global\let\ETE@WarnModDate\relax
  }%
  \def\ETE@FileInfo#1#2{#1 file: <#2>}%
\else
  \def\ETE@Make#1#2{%
    \ifETE@update
      \ifnum\pdf@strcmp{\pdf@filemoddate{#1}}{\pdf@filemoddate{#2}}>0 %
        \expandafter\expandafter\expandafter\@firstofone
      \else
        \@PackageInfoNoLine{epstopdf}{%
          Output file is already uptodate%
        }%
        \expandafter\expandafter\expandafter\@gobble
      \fi
    \else
      \expandafter\@firstofone
    \fi
  }%
  \def\ETE@FileInfo#1#2{%
    #1 file: <#2>%
    \expandafter\expandafter\expandafter
    \ETE@Date\pdf@filemoddate{#2}\@nil
    \expandafter\expandafter\expandafter
    \ETE@Size\pdf@filesize{#2}\@nil
  }%
  \def\ETE@Date#1\@nil{%
    \ifx\\#1\\%
    \else
      \ETE@@Date#1\@nil
    \fi
  }%
  \def\ETE@@Date#1:#2#3#4#5#6#7#8#9{%
    \MessageBreak
    \@spaces\space\space\space date: #2#3#4#5-#6#7-#8#9 %
    \ETE@@Time
  }%
  \def\ETE@@Time#1#2#3#4#5#6#7\@nil{%
    #1#2:#3#4:#5#6%
  }%
  \def\ETE@Size#1\@nil{%
    \ifx\\#1\\%
    \else
      \MessageBreak
      \@spaces\space\space\space size: #1 bytes%
    \fi
  }%
\fi
%    \end{macrocode}
%
% \subsection{Adding conversion support}
%
%    Patch \cs{Gin@setfile} to execute |#3|, if it contains
%    a command.
%    \begin{macrocode}
\expandafter\ifx\csname ETE@OrgGin@setfile\endcsname\relax
  \let\ETE@OrgGin@setfile\Gin@setfile
\else
  \@PackageError{epstopdf}{%
    Command \string\ETE@OrgGin@setfile\space
    already defined.\MessageBreak
  }{%
    Probably some package has included the code of this package%
    \MessageBreak
    instead of using \string\RequirePackage{epstopdf}.%
    \MessageBreak
    \@ehc
  }%
\fi
\def\ETE@IfFileExists{%
  \begingroup\expandafter\expandafter\expandafter\endgroup
  \expandafter\ifx\csname grffile@IfFileExists\endcsname\relax
    \expandafter\IfFileExists
  \else
    \global\let\ETE@IfFileExists\grffile@IfFileExists
    \expandafter\grffile@IfFileExists
  \fi
}
\def\ETE@Skip#1#2\x\fi\fi{%
  \fi
  \fi
  \endgroup
  \fi
  \fi
  #1%
}
\newif\ifETE@InsideSetfile
\newcommand*{\epstopdfcall}[1]{%
  \ifETE@InsideSetfile
    \expandafter\@firstoftwo
  \else
    \expandafter\@secondoftwo
  \fi
  {`#1}%
  {\Gin@base\Gin@ext}%
}
\def\ETE@DefCommandLine#1{%
  \edef\CommandLine{\expandafter\fi\if`#1}%
}
\def\ETE@DefX#1{%
  \expandafter\expandafter\expandafter\def
  \expandafter\expandafter\expandafter\x
  \expandafter\expandafter\expandafter{%
    \expandafter\fi\if`#1\relax\else
  }%
}
\def\ETE@Gin@setfile#1#2#3{%
  \ifETE@disable
    \ETE@OrgGin@setfile{#1}{#2}{#3}%
  \else
    \begingroup
      \ETE@InsideSetfiletrue
      \ETE@DefX{#3}%
    \expandafter\endgroup
    \ifx\x\@empty
      \ETE@OrgGin@setfile{#1}{#2}{#3}%
    \else
      \begingroup
        \ETE@InsideSetfiletrue
        \def\GraphicsType{#1}%
        \def\GraphicsRead{#2}%
        \ifETE@prefersuffix
        \else
          \ifx\ETE@suffix\@empty
          \else
            \ETE@IfFileExists{\Gin@base\GraphicsRead}{%
              \ETE@Skip{%
                \ETE@OrgGin@setfile{#1}{#2}{\Gin@base#2}%
              }%
            }{%
              \let\next\relax
            }%
            \next
          \fi
        \fi
        \ifx\Gin@ext\relax
          \let\SourceExt\Gin@eext
          \def\SourceFile{\Gin@base\Gin@eext}%
        \else
          \let\SourceExt\Gin@ext
          \def\SourceFile{\Gin@base\Gin@ext}%
        \fi
        \edef\SourceExt{% remove dot
          \expandafter\@cdr\SourceExt\@empty\@nil
        }%
        \let\OutputDirectory\ETE@outdir
        \ifx\OutputDirectory\@empty
          \edef\OutputFile{\ETE@GenerateName{\Gin@base}{#2}}%
        \else
          \begingroup
            \filename@parse{\Gin@base#2}%
            \edef\x{\endgroup
              \def\noexpand\OutputFile{%
                \ETE@GenerateName{%
                  \OutputDirectory\filename@base
                }{#2}%
              }%
            }%
          \x
        \fi
        \ETE@DefCommandLine{#3}%
        \ifETE@verbose
          \@PackageInfo{epstopdf}{%
            \ETE@FileInfo{Source}\SourceFile\MessageBreak
            \ETE@FileInfo{Output}\OutputFile\MessageBreak
            Command: <\CommandLine>\MessageBreak
            \string\includegraphics
          }%
        \fi
        \ETE@Make\SourceFile\OutputFile{%
          \pdf@system{\CommandLine}%
          \ifETE@verbose
            \@PackageInfoNoLine{epstopdf}{%
              \ETE@FileInfo{Result}\OutputFile
            }%
          \fi
        }%
        \edef\x{\endgroup
          \ifx\OutputDirectory\@empty
          \else
            \def\noexpand\Gin@base{%
              \OutputDirectory\noexpand\filename@base
            }%
          \fi
          \ifx\ETE@suffix\@empty
          \else
            \edef\noexpand\Gin@base{%
              \noexpand\Gin@base\ETE@suffix
            }%
          \fi
          \noexpand\ETE@OrgGin@setfile{%
            \GraphicsType
          }{%
            \GraphicsRead
          }{%
            \OutputFile
          }%
        }%
      \x
    \fi
  \fi
}
\let\Gin@setfile\ETE@Gin@setfile
\def\ETE@GenerateName#1#2{%
  #1\ETE@suffix#2%
}
%    \end{macrocode}
%
% \subsection{Declare graphics rule}
%
%    \begin{macrocode}
\newcommand*{\epstopdfDeclareGraphicsRule}[4]{%
  \ifx\\#4\\%
    \@PackageError{epstopdf-base}{%
      Conversion command is missing%
    }\@ehc
  \else
    \begingroup
      \@ifundefined{Gin@rule@#1}{%
      }{%
        \@PackageInfo{epstopdf-base}{%
          Redefining graphics rule for `#1'%
        }%
      }%
    \endgroup
    \@namedef{Gin@rule@#1}##1{{#2}{#3}{\epstopdfcall{#4}}}%
  \fi
}
%    \end{macrocode}
%
% \cs{DeclareGraphicsRule} for .eps
%    \begin{macrocode}
\epstopdfDeclareGraphicsRule{.eps}{pdf}{.pdf}{%
  \ETE@epstopdf{#1}%
}
\def\ETE@epstopdf#1{%
  \ETE@program@epstopdf\space
  \ifcase\ifx\OutputDirectory\@empty
           \ifx\ETE@suffix\@empty
             1%
           \fi
         \fi
         0 %
    --outfile=\OutputFile\space
  \fi
  #1%
}
%    \end{macrocode}
%
%    \begin{macrocode}
\ifETE@prepend
  \expandafter\PrependGraphicsExtensions
\else
  \expandafter\AppendGraphicsExtensions
\fi
{.eps}
\let\ETE@prepend\@undefined
\DeclareVoidOption{prepend}{%
  \PrependGraphicsExtensions{.eps}%
}
\let\ETE@append\@undefined
\DeclareVoidOption{append}{%
  \AppendGraphicsExtensions{.eps}%
}
%    \end{macrocode}
%
%    \begin{macrocode}
\InputIfFileExists{epstopdf-sys.cfg}{}{}
\InputIfFileExists{epstopdf.cfg}{}{}
%    \end{macrocode}
%
%    \begin{macrocode}
\ETE@AtEnd%
%</base>
%    \end{macrocode}
%
% \section{Test}
%
% \subsection{Preface for standard catcode check}
%
%    \begin{macrocode}
%<*test1>
\input miniltx.tex\relax
\def\Gin@driver{pdftex.def}
\input graphicx.sty\relax
\resetatcatcode
%</test1>
%    \end{macrocode}
%
% \subsection{Catcode checks for loading}
%
%    \begin{macrocode}
%<*test1>
%    \end{macrocode}
%    \begin{macrocode}
\catcode`\{=1 %
\catcode`\}=2 %
\catcode`\#=6 %
\catcode`\@=11 %
\expandafter\ifx\csname count@\endcsname\relax
  \countdef\count@=255 %
\fi
\expandafter\ifx\csname @gobble\endcsname\relax
  \long\def\@gobble#1{}%
\fi
\expandafter\ifx\csname @firstofone\endcsname\relax
  \long\def\@firstofone#1{#1}%
\fi
\expandafter\ifx\csname loop\endcsname\relax
  \expandafter\@firstofone
\else
  \expandafter\@gobble
\fi
{%
  \def\loop#1\repeat{%
    \def\body{#1}%
    \iterate
  }%
  \def\iterate{%
    \body
      \let\next\iterate
    \else
      \let\next\relax
    \fi
    \next
  }%
  \let\repeat=\fi
}%
\def\RestoreCatcodes{}
\count@=0 %
\loop
  \edef\RestoreCatcodes{%
    \RestoreCatcodes
    \catcode\the\count@=\the\catcode\count@\relax
  }%
\ifnum\count@<255 %
  \advance\count@ 1 %
\repeat

\def\RangeCatcodeInvalid#1#2{%
  \count@=#1\relax
  \loop
    \catcode\count@=15 %
  \ifnum\count@<#2\relax
    \advance\count@ 1 %
  \repeat
}
\def\RangeCatcodeCheck#1#2#3{%
  \count@=#1\relax
  \loop
    \ifnum#3=\catcode\count@
    \else
      \errmessage{%
        Character \the\count@\space
        with wrong catcode \the\catcode\count@\space
        instead of \number#3%
      }%
    \fi
  \ifnum\count@<#2\relax
    \advance\count@ 1 %
  \repeat
}
\def\space{ }
\expandafter\ifx\csname LoadCommand\endcsname\relax
  \def\LoadCommand{\input epstopdf.sty\relax}%
\fi
\def\Test{%
  \RangeCatcodeInvalid{0}{47}%
  \RangeCatcodeInvalid{58}{64}%
  \RangeCatcodeInvalid{91}{96}%
  \RangeCatcodeInvalid{123}{255}%
  \catcode`\@=12 %
  \catcode`\\=0 %
  \catcode`\%=14 %
  \LoadCommand
  \RangeCatcodeCheck{0}{36}{15}%
  \RangeCatcodeCheck{37}{37}{14}%
  \RangeCatcodeCheck{38}{47}{15}%
  \RangeCatcodeCheck{48}{57}{12}%
  \RangeCatcodeCheck{58}{63}{15}%
  \RangeCatcodeCheck{64}{64}{12}%
  \RangeCatcodeCheck{65}{90}{11}%
  \RangeCatcodeCheck{91}{91}{15}%
  \RangeCatcodeCheck{92}{92}{0}%
  \RangeCatcodeCheck{93}{96}{15}%
  \RangeCatcodeCheck{97}{122}{11}%
  \RangeCatcodeCheck{123}{255}{15}%
  \RestoreCatcodes
}
\Test
\csname @@end\endcsname
\end
%    \end{macrocode}
%    \begin{macrocode}
%</test1>
%    \end{macrocode}
%
% \section{Installation}
%
% \subsection{Download}
%
% \paragraph{Package.} This package is available on
% CTAN\footnote{\url{http://ctan.org/pkg/epstopdf}}:
% \begin{description}
% \item[\CTAN{macros/latex/contrib/oberdiek/epstopdf.dtx}] The source file.
% \item[\CTAN{macros/latex/contrib/oberdiek/epstopdf.pdf}] Documentation.
% \end{description}
%
%
% \paragraph{Bundle.} All the packages of the bundle `oberdiek'
% are also available in a TDS compliant ZIP archive. There
% the packages are already unpacked and the documentation files
% are generated. The files and directories obey the TDS standard.
% \begin{description}
% \item[\CTAN{install/macros/latex/contrib/oberdiek.tds.zip}]
% \end{description}
% \emph{TDS} refers to the standard ``A Directory Structure
% for \TeX\ Files'' (\CTAN{tds/tds.pdf}). Directories
% with \xfile{texmf} in their name are usually organized this way.
%
% \subsection{Bundle installation}
%
% \paragraph{Unpacking.} Unpack the \xfile{oberdiek.tds.zip} in the
% TDS tree (also known as \xfile{texmf} tree) of your choice.
% Example (linux):
% \begin{quote}
%   |unzip oberdiek.tds.zip -d ~/texmf|
% \end{quote}
%
% \paragraph{Script installation.}
% Check the directory \xfile{TDS:scripts/oberdiek/} for
% scripts that need further installation steps.
% Package \xpackage{attachfile2} comes with the Perl script
% \xfile{pdfatfi.pl} that should be installed in such a way
% that it can be called as \texttt{pdfatfi}.
% Example (linux):
% \begin{quote}
%   |chmod +x scripts/oberdiek/pdfatfi.pl|\\
%   |cp scripts/oberdiek/pdfatfi.pl /usr/local/bin/|
% \end{quote}
%
% \subsection{Package installation}
%
% \paragraph{Unpacking.} The \xfile{.dtx} file is a self-extracting
% \docstrip\ archive. The files are extracted by running the
% \xfile{.dtx} through \plainTeX:
% \begin{quote}
%   \verb|tex epstopdf.dtx|
% \end{quote}
%
% \paragraph{TDS.} Now the different files must be moved into
% the different directories in your installation TDS tree
% (also known as \xfile{texmf} tree):
% \begin{quote}
% \def\t{^^A
% \begin{tabular}{@{}>{\ttfamily}l@{ $\rightarrow$ }>{\ttfamily}l@{}}
%   epstopdf.sty & tex/latex/oberdiek/epstopdf.sty\\
%   epstopdf-base.sty & tex/latex/oberdiek/epstopdf-base.sty\\
%   epstopdf.pdf & doc/latex/oberdiek/epstopdf.pdf\\
%   test/epstopdf-test1.tex & doc/latex/oberdiek/test/epstopdf-test1.tex\\
%   epstopdf.dtx & source/latex/oberdiek/epstopdf.dtx\\
% \end{tabular}^^A
% }^^A
% \sbox0{\t}^^A
% \ifdim\wd0>\linewidth
%   \begingroup
%     \advance\linewidth by\leftmargin
%     \advance\linewidth by\rightmargin
%   \edef\x{\endgroup
%     \def\noexpand\lw{\the\linewidth}^^A
%   }\x
%   \def\lwbox{^^A
%     \leavevmode
%     \hbox to \linewidth{^^A
%       \kern-\leftmargin\relax
%       \hss
%       \usebox0
%       \hss
%       \kern-\rightmargin\relax
%     }^^A
%   }^^A
%   \ifdim\wd0>\lw
%     \sbox0{\small\t}^^A
%     \ifdim\wd0>\linewidth
%       \ifdim\wd0>\lw
%         \sbox0{\footnotesize\t}^^A
%         \ifdim\wd0>\linewidth
%           \ifdim\wd0>\lw
%             \sbox0{\scriptsize\t}^^A
%             \ifdim\wd0>\linewidth
%               \ifdim\wd0>\lw
%                 \sbox0{\tiny\t}^^A
%                 \ifdim\wd0>\linewidth
%                   \lwbox
%                 \else
%                   \usebox0
%                 \fi
%               \else
%                 \lwbox
%               \fi
%             \else
%               \usebox0
%             \fi
%           \else
%             \lwbox
%           \fi
%         \else
%           \usebox0
%         \fi
%       \else
%         \lwbox
%       \fi
%     \else
%       \usebox0
%     \fi
%   \else
%     \lwbox
%   \fi
% \else
%   \usebox0
% \fi
% \end{quote}
% If you have a \xfile{docstrip.cfg} that configures and enables \docstrip's
% TDS installing feature, then some files can already be in the right
% place, see the documentation of \docstrip.
%
% \subsection{Refresh file name databases}
%
% If your \TeX~distribution
% (\teTeX, \mikTeX, \dots) relies on file name databases, you must refresh
% these. For example, \teTeX\ users run \verb|texhash| or
% \verb|mktexlsr|.
%
% \subsection{Some details for the interested}
%
% \paragraph{Attached source.}
%
% The PDF documentation on CTAN also includes the
% \xfile{.dtx} source file. It can be extracted by
% AcrobatReader 6 or higher. Another option is \textsf{pdftk},
% e.g. unpack the file into the current directory:
% \begin{quote}
%   \verb|pdftk epstopdf.pdf unpack_files output .|
% \end{quote}
%
% \paragraph{Unpacking with \LaTeX.}
% The \xfile{.dtx} chooses its action depending on the format:
% \begin{description}
% \item[\plainTeX:] Run \docstrip\ and extract the files.
% \item[\LaTeX:] Generate the documentation.
% \end{description}
% If you insist on using \LaTeX\ for \docstrip\ (really,
% \docstrip\ does not need \LaTeX), then inform the autodetect routine
% about your intention:
% \begin{quote}
%   \verb|latex \let\install=y% \iffalse meta-comment
%
% File: epstopdf.dtx
% Version: 2016/05/15 v2.6
% Info: Conversion with epstopdf on the fly
%
% Copyright (C) 2001, 2006-2010 by
%    Heiko Oberdiek <heiko.oberdiek at googlemail.com>
%
% This work may be distributed and/or modified under the
% conditions of the LaTeX Project Public License, either
% version 1.3c of this license or (at your option) any later
% version. This version of this license is in
%    http://www.latex-project.org/lppl/lppl-1-3c.txt
% and the latest version of this license is in
%    http://www.latex-project.org/lppl.txt
% and version 1.3 or later is part of all distributions of
% LaTeX version 2005/12/01 or later.
%
% This work has the LPPL maintenance status "maintained".
%
% This Current Maintainer of this work is Heiko Oberdiek.
%
% This work consists of the main source file epstopdf.dtx
% and the derived files
%    epstopdf.sty, epstopdf.pdf, epstopdf.ins, epstopdf.drv,
%    epstopdf-base.sty, epstopdf-test1.tex.
%
% Distribution:
%    CTAN:macros/latex/contrib/oberdiek/epstopdf.dtx
%    CTAN:macros/latex/contrib/oberdiek/epstopdf.pdf
%
% Unpacking:
%    (a) If epstopdf.ins is present:
%           tex epstopdf.ins
%    (b) Without epstopdf.ins:
%           tex epstopdf.dtx
%    (c) If you insist on using LaTeX
%           latex \let\install=y% \iffalse meta-comment
%
% File: epstopdf.dtx
% Version: 2016/05/15 v2.6
% Info: Conversion with epstopdf on the fly
%
% Copyright (C) 2001, 2006-2010 by
%    Heiko Oberdiek <heiko.oberdiek at googlemail.com>
%
% This work may be distributed and/or modified under the
% conditions of the LaTeX Project Public License, either
% version 1.3c of this license or (at your option) any later
% version. This version of this license is in
%    http://www.latex-project.org/lppl/lppl-1-3c.txt
% and the latest version of this license is in
%    http://www.latex-project.org/lppl.txt
% and version 1.3 or later is part of all distributions of
% LaTeX version 2005/12/01 or later.
%
% This work has the LPPL maintenance status "maintained".
%
% This Current Maintainer of this work is Heiko Oberdiek.
%
% This work consists of the main source file epstopdf.dtx
% and the derived files
%    epstopdf.sty, epstopdf.pdf, epstopdf.ins, epstopdf.drv,
%    epstopdf-base.sty, epstopdf-test1.tex.
%
% Distribution:
%    CTAN:macros/latex/contrib/oberdiek/epstopdf.dtx
%    CTAN:macros/latex/contrib/oberdiek/epstopdf.pdf
%
% Unpacking:
%    (a) If epstopdf.ins is present:
%           tex epstopdf.ins
%    (b) Without epstopdf.ins:
%           tex epstopdf.dtx
%    (c) If you insist on using LaTeX
%           latex \let\install=y% \iffalse meta-comment
%
% File: epstopdf.dtx
% Version: 2016/05/15 v2.6
% Info: Conversion with epstopdf on the fly
%
% Copyright (C) 2001, 2006-2010 by
%    Heiko Oberdiek <heiko.oberdiek at googlemail.com>
%
% This work may be distributed and/or modified under the
% conditions of the LaTeX Project Public License, either
% version 1.3c of this license or (at your option) any later
% version. This version of this license is in
%    http://www.latex-project.org/lppl/lppl-1-3c.txt
% and the latest version of this license is in
%    http://www.latex-project.org/lppl.txt
% and version 1.3 or later is part of all distributions of
% LaTeX version 2005/12/01 or later.
%
% This work has the LPPL maintenance status "maintained".
%
% This Current Maintainer of this work is Heiko Oberdiek.
%
% This work consists of the main source file epstopdf.dtx
% and the derived files
%    epstopdf.sty, epstopdf.pdf, epstopdf.ins, epstopdf.drv,
%    epstopdf-base.sty, epstopdf-test1.tex.
%
% Distribution:
%    CTAN:macros/latex/contrib/oberdiek/epstopdf.dtx
%    CTAN:macros/latex/contrib/oberdiek/epstopdf.pdf
%
% Unpacking:
%    (a) If epstopdf.ins is present:
%           tex epstopdf.ins
%    (b) Without epstopdf.ins:
%           tex epstopdf.dtx
%    (c) If you insist on using LaTeX
%           latex \let\install=y\input{epstopdf.dtx}
%        (quote the arguments according to the demands of your shell)
%
% Documentation:
%    (a) If epstopdf.drv is present:
%           latex epstopdf.drv
%    (b) Without epstopdf.drv:
%           latex epstopdf.dtx; ...
%    The class ltxdoc loads the configuration file ltxdoc.cfg
%    if available. Here you can specify further options, e.g.
%    use A4 as paper format:
%       \PassOptionsToClass{a4paper}{article}
%
%    Programm calls to get the documentation (example):
%       pdflatex epstopdf.dtx
%       makeindex -s gind.ist epstopdf.idx
%       pdflatex epstopdf.dtx
%       makeindex -s gind.ist epstopdf.idx
%       pdflatex epstopdf.dtx
%
% Installation:
%    TDS:tex/latex/oberdiek/epstopdf.sty
%    TDS:tex/latex/oberdiek/epstopdf-base.sty
%    TDS:doc/latex/oberdiek/epstopdf.pdf
%    TDS:doc/latex/oberdiek/test/epstopdf-test1.tex
%    TDS:source/latex/oberdiek/epstopdf.dtx
%
%<*ignore>
\begingroup
  \catcode123=1 %
  \catcode125=2 %
  \def\x{LaTeX2e}%
\expandafter\endgroup
\ifcase 0\ifx\install y1\fi\expandafter
         \ifx\csname processbatchFile\endcsname\relax\else1\fi
         \ifx\fmtname\x\else 1\fi\relax
\else\csname fi\endcsname
%</ignore>
%<*install>
\input docstrip.tex
\Msg{************************************************************************}
\Msg{* Installation}
\Msg{* Package: epstopdf 2016/05/15 v2.6 Conversion with epstopdf on the fly (HO)}
\Msg{************************************************************************}

\keepsilent
\askforoverwritefalse

\let\MetaPrefix\relax
\preamble

This is a generated file.

Project: epstopdf
Version: 2016/05/15 v2.6

Copyright (C) 2001, 2006-2010 by
   Heiko Oberdiek <heiko.oberdiek at googlemail.com>

This work may be distributed and/or modified under the
conditions of the LaTeX Project Public License, either
version 1.3c of this license or (at your option) any later
version. This version of this license is in
   http://www.latex-project.org/lppl/lppl-1-3c.txt
and the latest version of this license is in
   http://www.latex-project.org/lppl.txt
and version 1.3 or later is part of all distributions of
LaTeX version 2005/12/01 or later.

This work has the LPPL maintenance status "maintained".

This Current Maintainer of this work is Heiko Oberdiek.

This work consists of the main source file epstopdf.dtx
and the derived files
   epstopdf.sty, epstopdf.pdf, epstopdf.ins, epstopdf.drv,
   epstopdf-base.sty, epstopdf-test1.tex.

\endpreamble
\let\MetaPrefix\DoubleperCent

\generate{%
  \file{epstopdf.ins}{\from{epstopdf.dtx}{install}}%
  \file{epstopdf.drv}{\from{epstopdf.dtx}{driver}}%
  \usedir{tex/latex/oberdiek}%
  \file{epstopdf.sty}{\from{epstopdf.dtx}{package}}%
  \file{epstopdf-base.sty}{\from{epstopdf.dtx}{base}}%
  \usedir{doc/latex/oberdiek/test}%
  \file{epstopdf-test1.tex}{\from{epstopdf.dtx}{test1}}%
  \nopreamble
  \nopostamble
  \usedir{source/latex/oberdiek/catalogue}%
  \file{epstopdf-pkg.xml}{\from{epstopdf.dtx}{catalogue}}%
}

\catcode32=13\relax% active space
\let =\space%
\Msg{************************************************************************}
\Msg{*}
\Msg{* To finish the installation you have to move the following}
\Msg{* files into a directory searched by TeX:}
\Msg{*}
\Msg{*     epstopdf.sty, epstopdf-base.sty}
\Msg{*}
\Msg{* To produce the documentation run the file `epstopdf.drv'}
\Msg{* through LaTeX.}
\Msg{*}
\Msg{* Happy TeXing!}
\Msg{*}
\Msg{************************************************************************}

\endbatchfile
%</install>
%<*ignore>
\fi
%</ignore>
%<*driver>
\NeedsTeXFormat{LaTeX2e}
\ProvidesFile{epstopdf.drv}%
  [2016/05/15 v2.6 Conversion with epstopdf on the fly (HO)]%
\documentclass{ltxdoc}
\usepackage{holtxdoc}[2011/11/22]
\begin{document}
  \DocInput{epstopdf.dtx}%
\end{document}
%</driver>
% \fi
%
%
% \CharacterTable
%  {Upper-case    \A\B\C\D\E\F\G\H\I\J\K\L\M\N\O\P\Q\R\S\T\U\V\W\X\Y\Z
%   Lower-case    \a\b\c\d\e\f\g\h\i\j\k\l\m\n\o\p\q\r\s\t\u\v\w\x\y\z
%   Digits        \0\1\2\3\4\5\6\7\8\9
%   Exclamation   \!     Double quote  \"     Hash (number) \#
%   Dollar        \$     Percent       \%     Ampersand     \&
%   Acute accent  \'     Left paren    \(     Right paren   \)
%   Asterisk      \*     Plus          \+     Comma         \,
%   Minus         \-     Point         \.     Solidus       \/
%   Colon         \:     Semicolon     \;     Less than     \<
%   Equals        \=     Greater than  \>     Question mark \?
%   Commercial at \@     Left bracket  \[     Backslash     \\
%   Right bracket \]     Circumflex    \^     Underscore    \_
%   Grave accent  \`     Left brace    \{     Vertical bar  \|
%   Right brace   \}     Tilde         \~}
%
% \GetFileInfo{epstopdf.drv}
%
% \title{The \xpackage{epstopdf} package}
% \date{2016/05/15 v2.6}
% \author{Heiko Oberdiek\thanks
% {Please report any issues at https://github.com/ho-tex/oberdiek/issues}\\
% \xemail{heiko.oberdiek at googlemail.com}}
%
% \maketitle
%
% \begin{abstract}
% This packages adds support of handling eps images
% to package \xpackage{graphics} or \xpackage{graphicx}
% with option \xoption{pdftex}. If an eps image is detected,
% epstopdf is automatically called to convert it to pdf format.
% \end{abstract}
%
% \tableofcontents
%
% \providecommand*{\xprogram}[1]{\textsf{#1}}
%
% \section{Documentation}
%
% \subsection{Introduction}
%
% \LaTeX\ provides its graphics bundle to include graphics files.
% Both packages \xpackage{graphics} or \xpackage{graphicx} may
% be used. the latter one loads the first and adds options
% in key value style for \cs{includegraphics}.
%
% Usually the drivers do not support all kind of graphics files.
% Other image types must be converted, before they become
% usuable. In case of driver \xprogram{dvips}, the graphics
% rule may contain a conversion rule. Then all that package
% \xpackage{graphics} must know is the bounding box, the command
% is passed to \xprogram{dvips} that calls it and embeds the
% converted image.
%
% However, \pdfTeX\ has its driver for PDF output already
% build in. It's graphics inclusion commands (\cs{pdfximage})
% does not allow the execution of external commands.
% Therefore commands in the last argument of
% \cs{DeclareGraphicsRule} were of no use.
% But external programs can be called within \pdfTeX.
% This feature is called ``shell escape'' or ``write 18''
% and must usually enabled explicitely because of security reasons.
% Now, this package \xpackage{epstopdf} hooks
% into package \xpackage{graphics}' code to catch
% that argument with the external command and executes it
% to convert the graphics file to a supported format and
% passes the control of graphics inclusion back to package
% \xpackage{graphics}.
%
% \subsection{Requirements}
%
% \begin{itemize}
% \item
%   The feature \cs{write18} must be enabled. This allows the
%   running of external programs during \TeX's compile run.
%   Keep in mind that this is a security risk. The feature is an
%   addition to \cs{TeX}. Mik\TeX, te\TeX, \TeX\ Live support it.
%   In Web2C based \TeX\ distributions (te\TeX, \TeX\ Live) it can be
%   enabled in the configuration file \xfile{texmf.cnf}:
%   \begin{quote}
%     |shell_escape = 1|
%   \end{quote}
%   Because of the security risk, it is better to do it on the
%   command line only:
%   \begin{quote}
%   \texttt{--shell-escape} (te\TeX, \TeX\ Live)\\
%   \texttt{--enable-write18} (MiK\TeX)
%   \end{quote}
%   Example:
%   \begin{quote}
%     |pdflatex -shell-escape test.tex|
%   \end{quote}
% \item
%   The program \xprogram{epstopdf} for the conversion from EPS
%   to PDF. However, other programs can
%   be used and configured by \cs{DeclareGraphicsRule}.
%   Example:
%   \begin{quote}
%   |\epstopdfDeclareGraphicsRule{.eps}{pdf}{.pdf}{%|\\
%   |  ps2pdf -dEPSCrop #1 \OutputFile|\\
%   |}|
%   \end{quote}
% \end{itemize}
%
% \subsection{Usage}
%
% The package is loaded after graphic\{s,x\}, e.g.:
% \begin{quote}
%   |\usepackage[pdftex]{graphicx}|\\
%   |\usepackage{epstopdf}|
% \end{quote}
% Now images with file name extension \xfile{.eps} are detected
% and supported using \cs{includegraphics}.
%
% If the graphics file name is explicitly specified with extension
% \xfile{.eps} the new rule for EPS files is called and
% the conversion performed. If option \xoption{update} is in force
% then the conversion step is dropped if the target file already
% exists and is not older then the EPS file.
%
% The situation is more complicate if the graphics file is
% given without file name extension. Then the \xpackage{graphics}
% package must search for a supported image file.
% The possible extensions are stored in the graphics extension
% list, that can be set by \cs{DeclareGraphicsExtensions}.
% The algorithm:
% \begin{quote}
%    \def\>{\mbox{}\quad}
%    function search( \meta{filebase} )\\
%    \>foreach \meta{ext} in \meta{graphics extensions}\\
%    \>\>foreach \meta{dir} in \meta{current directory}, \meta{\cs{graphicspath}}\\
%    \>\>\>\meta{file} := \meta{dir} + \meta{filebase} + \meta{ext}\\
%    \>\>\>if exist \meta{file}\\
%    \>\>\>\>return found\\
%    \>return not found
% \end{quote}
% Package \xpackage{epstopdf} puts \xfile{.eps} at the end of the
% graphics extension search list. This is the behaviour of option
% \xoption{append} that is enabled by default.
% That means, the conversion is called last
% unless a supported file type cannot be found earlier. This avoids
% unnecessary conversion steps that slow down the \LaTeX\ run.
% If you want to use option \xoption{update} and your \pdfTeX\ supports
% it, then an outdated PDF file also would be found earlier unless
% \xoption{suffix} is used that is the default since version 2.0.
%
% With an empty option \xoption{suffix} and option
% \xoption{prepend} there is a risk that an original PDF file is overwritten:
%
% If the original image file is the PDF file and there is also a generated
% EPS file, then the original PDF file can be regenerated (depending
% on the option settings) and the original PDF file gets lost.
% Therefore option \xoption{suffix} is introduced in version 1.9
% to create a separate name space for generated output files.
%
% \textbf{Note:} Usually the conversion program needs the exact location
% of the image file. Usually the current directory works.
% Also if the image file is found using \cs{graphicspath}, the
% location is known. However, if the image is somewhere
% in a directory of environment variable |TEXINPUTS|, then
% the package does not know the exact location and the
% conversion program will not find the image file unless it
% implements a search using |TEXINPUTS| (program \xprogram{kpsewhich}
% may be of help in this task).
%
% \subsection{Options}
%
% Options can be given as package options or later using:
% \begin{declcs}{epstopdfsetup} \M{key value list}
% \end{declcs}
% \LaTeX\ expands the option list before passing the option list
% to the package's option handling code. This can fail
% for option \xoption{suffix} if it contains some of the macros
% described below. Use \cs{epstopdfsetup} after the package is loaded.
% Or load package \xpackage{kvoptions-patch} before. This package
% is also loaded by option \xoption{patch} of package \xpackage{kvoptions}.
% \LaTeX's option code is redefined to respect key value options and
% let the values untouched.
%
% \begin{description}
% \def\It#1{\item[\xoption{#1}:]}^^A
% \It{update}
%   The conversion program is only called, if the target file
%   does not exist or is older than the source image file.
% \It{append}
%   Puts the extension \xfile{.eps} at the end of the graphics extension list
%   (default).
% \It{prepend}
%   Puts the extension \xfile{.eps} at the begin of the graphics extension list.
% \It{outdir}
%   The converted file may put in an other output directory.
%   The value of \xoption{outdir} must include the directory separator.
%   Example for the current directory:
%   \begin{quote}
%   |\epstopdfsetup{outdir=./}|
%   \end{quote}
%   For other directories ensure, that they can be found.
%   See \cs{graphicspath} or |TEXINPUTS|.
% \It{suffix}
%   This option takes a string that is put between the file name base
%   and the extension of the output file. Rationale: It can happen,
%   that a PDF file is the original file and the EPS file the
%   generated file. If now the package thinks, that the PDF file
%   is the generated file, it will `regnerate' it. But in reality
%   the original file is lost. Therefore I recommend to use this
%   option always to generate a separate name space for generated
%   files. Proposed value is \texttt{-generated} or \texttt{.generated}.
%   The suffix \texttt{.generated} will also work here without the need for
%   package \xpackage{grffile}).
%
%   Example:
%   \begin{quote}
%   |\epstopdfsetup{suffix=-generated}|\\
%   Then \xfile{foo.eps} is converted to \xfile{foo-generated.pdf}.
%   \end{quote}
%   \cs{SourceExt} can be used inside the suffix string.
%   It's will be replaced by the extension of the image source
%   file without the leading dot, for instance:
%   \begin{quote}
%   |\epstopdfsetup{suffix=-\SourceExt-converted-to}|\\
%   \xfile{foo.eps} $\Rightarrow$ \xfile{foo-eps-converted-to.pdf}
%   \end{quote}
%   See also the next option \xoption{prefersuffix} that
%   modifies the behaviour of option \xoption{suffix} in some
%   cases.
%
%   Default for \xoption{suffix} is `|-\SourceExt-converted-to|'.
%
% \It{prefersuffix}
%   If a suffix is set by option \xoption{suffix}, then there can be
%   two image file names that could be taken into account for inclusion:
%   A image file name with the suffix string inside
%   its name and a image file name without; e.g. for \xfile{foo.eps}
%   the names could be:
%   \begin{quote}
%   \xfile{foo-suffix.pdf}, \xfile{foo.pdf}
%   \end{quote}
%   If option \xoption{perfersuffix} is turned on, the
%   file \xfile{foo-suffix.pdf} and its generation is preferred
%   over using \xfile{foo.pdf}. Otherwise \xfile{foo.pdf} is included
%   without generating \xfile{foo-suffix.pdf}.
%   The default of option \xoption{prefersuffix} is \texttt{true}.
% \It{program@epstopdf}
%   The name for the conversion program from EPS to PDF,
%   default is `|epstopdf|'.
% \It{verbose}
%   It prints some information about the image in the \xfile{.log} file
%   (default).
% \end{description}
%
% \subsection{Configuration}
%
% \subsubsection{System configuration file \xfile{epstopdf-sys.cfg}}
%
% If \xfile{epstopdf-sys} exists it is loaded at the end of the
% package \xpackage{epstopdf-base} and before the user configuration
% file. It's intended for TeX distributors. Thus they could add
% additional conversion rules (e.g., .gif -> .png) or set options.
%
% \subsubsection{User configuration file \xfile{epstopdf.cfg}}
%
% A configuration file \xfile{epstopdf.cfg} is loaded at the end
% of the package if it exists. It can be used for changing
% the default option setting. Example:
% \begin{quote}
% |\epstopdfsetup{verbose=false}|
% \end{quote}
%
% \subsubsection{Conversion program}
%
% You can use \cs{DeclareGraphicsRule} in a similar way as the
% route via \xprogram{dvips} to specify the conversion
% command line.
% The conversion argument starts with a back tick, followed
% by the conversion command including parameters.
%
% The whole conversion argument should also be wrapped inside
% \cs{epstopdfcall}. This reduces the problem with packages
% (e.g. \xpackage{pst-pdf}) that use the conversion argument
% and expands it. Macros \cs{SourceFile}, \cs{OutputFile}, and
% \cs{SourceExt} are not defined outside \xpackage{epstopdf-base}'s
% \cs{Gin@setfile} and error messages because of undefined
% command names are the result. If \cs{epstopdfcall} detects that
% it is called outside \xpackage{epstopdf-base}'s \cs{Gin@setfile}
% then it replaces the conversion argument by
% package \xpackage{graphics}'s
% default, usually the image file.
%
% The following macros are available inside:
% \begin{description}
% \def\It#1{\item[\cs{#1}:]}
% \It{OutputFile}: output file name (with known path and extension)
% \It{SourceFile}: source file name (with known path and extension),
%                  usually the same as |#1|,
% \It{SourceExt}: source extension without leading dot.
% \end{description}
%
% \paragraph{Conversion from EPS to PDF.}
% Other programs than \xprogram{epstopdf} can be used to convert from
% EPS to PDF. Example that uses \xprogram{Ghostscript}:
% \begin{quote}
%   |\DeclareGraphicsRule{.eps}{pdf}{.pdf}{%|\\
%   |  \epstopdfcall{`ps2pdf -dEPSCrop #1 \noexpand\OutputFile}%|\\
%   |}|
% \end{quote}
% \cs{DeclareGraphicsRule} expands the argument, therefore \cs{noexpand}
% is necessary. As convenience package \xpackage{epstopdf-base} defines
% \cs{epstopdfDeclareGraphicsRule}. Then the conversion argument
% is not expanded, \cs{epstopdfcall} and the back tick are added:
% \begin{quote}
%   |\epstopdfDeclareGraphicsRule{.eps}{pdf}{.pdf}{%|\\
%   |  ps2pdf -dEPSCrop #1 \OutputFile|\\
%   |}|
% \end{quote}
%
% Also \cs{OutputFile} respects the setting of option \xoption{outdir}.
%
% \subsection{Other image formats}
%
% The support that package \xpackage{epstopdf} implements is not
% limited to EPS files. Other image conversions can be declared.
% The following example shows it for GIF images under Unix with
% ImageMagick's \xprogram{convert}:
%\begin{quote}
%\begin{verbatim}
%\epstopdfDeclareGraphicsRule{.gif}{png}{.png}{%
%  convert #1 \OutputFile
%}
%\end{verbatim}
%\end{quote}
% The file extension \xfile{.gif} can be added to the extension list that
% package \xpackage{graphics} searches if the file extension
% is not given in \cs{includegraphics}.
% The list can be set by \cs{GraphicsExtensions}.
% \begin{quote}
% |\AppendGraphicsExtensions{.gif}|\\
% or\\
% |\PrependGraphicsExtensions{.gif}|
% \end{quote}
%
% \StopEventually{
% }
%
% \section{Implementation}
%
%    \begin{macrocode}
%<*package>
%    \end{macrocode}
%
% \subsection{Wrapper package}
%
%    Reload check, especially if the package is not used with \LaTeX.
%    \begin{macrocode}
\begingroup\catcode61\catcode48\catcode32=10\relax%
  \catcode13=5 % ^^M
  \endlinechar=13 %
  \catcode35=6 % #
  \catcode39=12 % '
  \catcode44=12 % ,
  \catcode45=12 % -
  \catcode46=12 % .
  \catcode58=12 % :
  \catcode64=11 % @
  \catcode123=1 % {
  \catcode125=2 % }
  \expandafter\let\expandafter\x\csname ver@epstopdf.sty\endcsname
  \ifx\x\relax % plain-TeX, first loading
  \else
    \def\empty{}%
    \ifx\x\empty % LaTeX, first loading,
      % variable is initialized, but \ProvidesPackage not yet seen
    \else
      \expandafter\ifx\csname PackageInfo\endcsname\relax
        \def\x#1#2{%
          \immediate\write-1{Package #1 Info: #2.}%
        }%
      \else
        \def\x#1#2{\PackageInfo{#1}{#2, stopped}}%
      \fi
      \x{epstopdf}{The package is already loaded}%
      \aftergroup\endinput
    \fi
  \fi
\endgroup%
%    \end{macrocode}
%    Package identification:
%    \begin{macrocode}
\begingroup\catcode61\catcode48\catcode32=10\relax%
  \catcode13=5 % ^^M
  \endlinechar=13 %
  \catcode35=6 % #
  \catcode39=12 % '
  \catcode40=12 % (
  \catcode41=12 % )
  \catcode44=12 % ,
  \catcode45=12 % -
  \catcode46=12 % .
  \catcode47=12 % /
  \catcode58=12 % :
  \catcode64=11 % @
  \catcode91=12 % [
  \catcode93=12 % ]
  \catcode123=1 % {
  \catcode125=2 % }
  \expandafter\ifx\csname ProvidesPackage\endcsname\relax
    \def\x#1#2#3[#4]{\endgroup
      \immediate\write-1{Package: #3 #4}%
      \xdef#1{#4}%
    }%
  \else
    \def\x#1#2[#3]{\endgroup
      #2[{#3}]%
      \ifx#1\@undefined
        \xdef#1{#3}%
      \fi
      \ifx#1\relax
        \xdef#1{#3}%
      \fi
    }%
  \fi
\expandafter\x\csname ver@epstopdf.sty\endcsname
\ProvidesPackage{epstopdf}%
  [2016/05/15 v2.6 Conversion with epstopdf on the fly (HO)]%
%    \end{macrocode}
%    Larger catcode set because of configuration files needed.
%    \begin{macrocode}
\begingroup\catcode61\catcode48\catcode32=10\relax%
  \catcode13=5 % ^^M
  \endlinechar=13 %
  \catcode123=1 % {
  \catcode125=2 % }
  \catcode64=11 % @
  \def\x{\endgroup
    \expandafter\edef\csname ETE@AtEnd\endcsname{%
      \endlinechar=\the\endlinechar\relax
      \catcode13=\the\catcode13\relax
      \catcode32=\the\catcode32\relax
      \catcode35=\the\catcode35\relax
      \catcode61=\the\catcode61\relax
      \catcode64=\the\catcode64\relax
      \catcode123=\the\catcode123\relax
      \catcode125=\the\catcode125\relax
    }%
  }%
\x\catcode61\catcode48\catcode32=10\relax%
\catcode13=5 % ^^M
\endlinechar=13 %
\catcode35=6 % #
\catcode64=11 % @
\catcode123=1 % {
\catcode125=2 % }
\def\TMP@EnsureCode#1#2{%
  \edef\ETE@AtEnd{%
    \ETE@AtEnd
    \catcode#1=\the\catcode#1\relax
  }%
  \catcode#1=#2\relax
}
\TMP@EnsureCode{39}{12}% '
\TMP@EnsureCode{40}{12}% (
\TMP@EnsureCode{41}{12}% )
\TMP@EnsureCode{42}{12}% *
\TMP@EnsureCode{45}{12}% -
\TMP@EnsureCode{47}{12}% /
\TMP@EnsureCode{91}{12}% [
\TMP@EnsureCode{93}{12}% ]
\edef\ETE@AtEnd{\ETE@AtEnd\noexpand\endinput}
%    \end{macrocode}
%
%    \begin{macrocode}
\let\ETE@SavedAtEnd\ETE@AtEnd
\RequirePackage{epstopdf-base}[2016/05/15]
\let\ETE@AtEnd\ETE@SavedAtEnd
%    \end{macrocode}
%
% \subsubsection{Option handling}
%
%    \begin{macrocode}
\DeclareOption*{%
  \expandafter\epstopdfsetup\expandafter{\CurrentOption}%
}
\ProcessOptions*\relax
%    \end{macrocode}
%
%    \begin{macrocode}
\ETE@AtEnd%
%</package>
%    \end{macrocode}
%
% \subsection{Base package}
%
%    \begin{macrocode}
%<*base>
%    \end{macrocode}
%
% \subsection{Preparations}
%
% \subsubsection{Relead check and identification}
%
%    Reload check, especially if the package is not used with \LaTeX.
%    \begin{macrocode}
\begingroup\catcode61\catcode48\catcode32=10\relax%
  \catcode13=5 % ^^M
  \endlinechar=13 %
  \catcode35=6 % #
  \catcode39=12 % '
  \catcode44=12 % ,
  \catcode45=12 % -
  \catcode46=12 % .
  \catcode58=12 % :
  \catcode64=11 % @
  \catcode123=1 % {
  \catcode125=2 % }
  \expandafter\let\expandafter\x\csname ver@epstopdf-base.sty\endcsname
  \ifx\x\relax % plain-TeX, first loading
  \else
    \def\empty{}%
    \ifx\x\empty % LaTeX, first loading,
      % variable is initialized, but \ProvidesPackage not yet seen
    \else
      \expandafter\ifx\csname PackageInfo\endcsname\relax
        \def\x#1#2{%
          \immediate\write-1{Package #1 Info: #2.}%
        }%
      \else
        \def\x#1#2{\PackageInfo{#1}{#2, stopped}}%
      \fi
      \x{epstopdf-base}{The package is already loaded}%
      \aftergroup\endinput
    \fi
  \fi
\endgroup%
%    \end{macrocode}
%    Package identification:
%    \begin{macrocode}
\begingroup\catcode61\catcode48\catcode32=10\relax%
  \catcode13=5 % ^^M
  \endlinechar=13 %
  \catcode35=6 % #
  \catcode39=12 % '
  \catcode40=12 % (
  \catcode41=12 % )
  \catcode44=12 % ,
  \catcode45=12 % -
  \catcode46=12 % .
  \catcode47=12 % /
  \catcode58=12 % :
  \catcode64=11 % @
  \catcode91=12 % [
  \catcode93=12 % ]
  \catcode123=1 % {
  \catcode125=2 % }
  \expandafter\ifx\csname ProvidesPackage\endcsname\relax
    \def\x#1#2#3[#4]{\endgroup
      \immediate\write-1{Package: #3 #4}%
      \xdef#1{#4}%
    }%
  \else
    \def\x#1#2[#3]{\endgroup
      #2[{#3}]%
      \ifx#1\@undefined
        \xdef#1{#3}%
      \fi
      \ifx#1\relax
        \xdef#1{#3}%
      \fi
    }%
  \fi
\expandafter\x\csname ver@epstopdf-base.sty\endcsname
\ProvidesPackage{epstopdf-base}%
  [2016/05/15 v2.6 Base part for package epstopdf]%
%    \end{macrocode}
%
% \subsubsection{Catcodes}
%
%    \begin{macrocode}
\begingroup\catcode61\catcode48\catcode32=10\relax%
  \catcode13=5 % ^^M
  \endlinechar=13 %
  \catcode123=1 % {
  \catcode125=2 % }
  \catcode64=11 % @
  \def\x{\endgroup
    \expandafter\edef\csname ETE@AtEnd\endcsname{%
      \endlinechar=\the\endlinechar\relax
      \catcode13=\the\catcode13\relax
      \catcode32=\the\catcode32\relax
      \catcode35=\the\catcode35\relax
      \catcode61=\the\catcode61\relax
      \catcode64=\the\catcode64\relax
      \catcode123=\the\catcode123\relax
      \catcode125=\the\catcode125\relax
    }%
  }%
\x\catcode61\catcode48\catcode32=10\relax%
\catcode13=5 % ^^M
\endlinechar=13 %
\catcode35=6 % #
\catcode64=11 % @
\catcode123=1 % {
\catcode125=2 % }
\def\TMP@EnsureCode#1#2{%
  \edef\ETE@AtEnd{%
    \ETE@AtEnd
    \catcode#1=\the\catcode#1\relax
  }%
  \catcode#1=#2\relax
}
\TMP@EnsureCode{33}{12}% !
\TMP@EnsureCode{39}{12}% '
\TMP@EnsureCode{42}{12}% *
\TMP@EnsureCode{44}{12}% ,
\TMP@EnsureCode{45}{12}% -
\TMP@EnsureCode{46}{12}% .
\TMP@EnsureCode{47}{12}% /
\TMP@EnsureCode{58}{12}% :
\TMP@EnsureCode{60}{12}% <
\TMP@EnsureCode{62}{12}% >
\TMP@EnsureCode{96}{12}% `
\edef\ETE@AtEnd{\ETE@AtEnd\noexpand\endinput}
%    \end{macrocode}
%
% \subsubsection{Load packages}
%
%    \begin{macrocode}
\RequirePackage{infwarerr}[2007/09/09]
\RequirePackage{grfext}\relax
\RequirePackage{kvoptions}[2007/10/02]
\RequirePackage{pdftexcmds}[2007/11/11]
%    \end{macrocode}
%
% \subsection{Checks}
%
% Check, whether package graphics is loaded
% (also graphicx loads graphics).
% Because miniltx.tex does not know \cs{@ifpackageloaded}
% we test for \cs{Gin@setfile} instead.
%    \begin{macrocode}
\begingroup\expandafter\expandafter\expandafter\endgroup
\expandafter\ifx\csname Gin@setfile\endcsname\relax
  \@PackageWarningNoLine{epstopdf}{%
    No graphics package \string`graphic{s,x}\string' loaded%
  }%
  \newcommand*{\epstopdfsetup}[1]{}%
  \expandafter\ETE@AtEnd
\fi%
%    \end{macrocode}
%
% Check, whether pdftex.def is loaded. \cs{ver@pdftex.def} is not
% available with miniltx.tex, thus we test for \cs{Gin@driver}.
%    \begin{macrocode}
\begingroup
  \def\x{luatex.def}%
  \ifx\Gin@driver\x
  \else
  \def\x{pdftex.def}%
  \ifx\Gin@driver\x
  \else
    \@PackageWarningNoLine{epstopdf}{%
      Drivers other than `pdftex' and `luatex' are not supported%
    }%
    \endgroup
    \newcommand*{\epstopdfsetup}[1]{}%
    \expandafter\ETE@AtEnd
  \fi%
  \fi%
\endgroup
%    \end{macrocode}
%
% Check, whether the shell escape feature is enabled.
%
%    \begin{macrocode}
\begingroup
  \expandafter\ifx\csname pdf@shellescape\endcsname\relax
  \else
    \ifnum\pdf@shellescape>0 %
    \else
      \@PackageWarningNoLine{epstopdf}{%
        Shell escape feature is not enabled%
      }%
    \fi
  \fi
\endgroup
%    \end{macrocode}
%
% \subsection{Package loading}
%
%    \begin{macrocode}
%    \end{macrocode}
%
% \subsection{Options}
%
%    \begin{macrocode}
\SetupKeyvalOptions{family=ETE,prefix=ETE@}
\DeclareBoolOption{update}
\DeclareBoolOption{verbose}
\newif\ifETE@prepend
\DeclareVoidOption{prepend}{\ETE@prependtrue}
\DeclareVoidOption{append}{\ETE@prependfalse}
\DeclareStringOption{outdir}
\DeclareStringOption{suffix}
\DeclareBoolOption{prefersuffix}
\DeclareStringOption{program@epstopdf}
%    \end{macrocode}
%    Options \xoption{disable} and \xoption{enable} are for testing
%    only. Therefore they are not documented on purpose.
%    \begin{macrocode}
\DeclareBoolOption{disable}
\DeclareComplementaryOption{enable}{disable}
%    \end{macrocode}
%    \begin{macrocode}
\newcommand*{\epstopdfsetup}{\setkeys{ETE}}
%    \end{macrocode}
%
% \subsubsection{Default setting}
%
%    \begin{macrocode}
\epstopdfsetup{%
  verbose,%
  enable,%
  append,%
  update,%
  prefersuffix,%
  suffix=-\SourceExt-converted-to,%
  program@epstopdf=epstopdf%
}
%    \end{macrocode}
%
% \subsection{Make and verbose}
%
%    \begin{macrocode}
\begingroup\expandafter\expandafter\expandafter\endgroup
\expandafter\ifx\csname pdf@filemoddate\endcsname\relax
  \def\ETE@Make#1#2{%
    \ifETE@update
      \ETE@WarnModDate
    \fi
    \@firstofone
  }%
  \def\ETE@WarnModDate{%
    \@PackageWarningNoLine{epstopdf}{%
      \string\pdffilemoddate\space is not available,\MessageBreak
      option `update' will be ignored%
    }%
    \global\let\ETE@WarnModDate\relax
  }%
  \def\ETE@FileInfo#1#2{#1 file: <#2>}%
\else
  \def\ETE@Make#1#2{%
    \ifETE@update
      \ifnum\pdf@strcmp{\pdf@filemoddate{#1}}{\pdf@filemoddate{#2}}>0 %
        \expandafter\expandafter\expandafter\@firstofone
      \else
        \@PackageInfoNoLine{epstopdf}{%
          Output file is already uptodate%
        }%
        \expandafter\expandafter\expandafter\@gobble
      \fi
    \else
      \expandafter\@firstofone
    \fi
  }%
  \def\ETE@FileInfo#1#2{%
    #1 file: <#2>%
    \expandafter\expandafter\expandafter
    \ETE@Date\pdf@filemoddate{#2}\@nil
    \expandafter\expandafter\expandafter
    \ETE@Size\pdf@filesize{#2}\@nil
  }%
  \def\ETE@Date#1\@nil{%
    \ifx\\#1\\%
    \else
      \ETE@@Date#1\@nil
    \fi
  }%
  \def\ETE@@Date#1:#2#3#4#5#6#7#8#9{%
    \MessageBreak
    \@spaces\space\space\space date: #2#3#4#5-#6#7-#8#9 %
    \ETE@@Time
  }%
  \def\ETE@@Time#1#2#3#4#5#6#7\@nil{%
    #1#2:#3#4:#5#6%
  }%
  \def\ETE@Size#1\@nil{%
    \ifx\\#1\\%
    \else
      \MessageBreak
      \@spaces\space\space\space size: #1 bytes%
    \fi
  }%
\fi
%    \end{macrocode}
%
% \subsection{Adding conversion support}
%
%    Patch \cs{Gin@setfile} to execute |#3|, if it contains
%    a command.
%    \begin{macrocode}
\expandafter\ifx\csname ETE@OrgGin@setfile\endcsname\relax
  \let\ETE@OrgGin@setfile\Gin@setfile
\else
  \@PackageError{epstopdf}{%
    Command \string\ETE@OrgGin@setfile\space
    already defined.\MessageBreak
  }{%
    Probably some package has included the code of this package%
    \MessageBreak
    instead of using \string\RequirePackage{epstopdf}.%
    \MessageBreak
    \@ehc
  }%
\fi
\def\ETE@IfFileExists{%
  \begingroup\expandafter\expandafter\expandafter\endgroup
  \expandafter\ifx\csname grffile@IfFileExists\endcsname\relax
    \expandafter\IfFileExists
  \else
    \global\let\ETE@IfFileExists\grffile@IfFileExists
    \expandafter\grffile@IfFileExists
  \fi
}
\def\ETE@Skip#1#2\x\fi\fi{%
  \fi
  \fi
  \endgroup
  \fi
  \fi
  #1%
}
\newif\ifETE@InsideSetfile
\newcommand*{\epstopdfcall}[1]{%
  \ifETE@InsideSetfile
    \expandafter\@firstoftwo
  \else
    \expandafter\@secondoftwo
  \fi
  {`#1}%
  {\Gin@base\Gin@ext}%
}
\def\ETE@DefCommandLine#1{%
  \edef\CommandLine{\expandafter\fi\if`#1}%
}
\def\ETE@DefX#1{%
  \expandafter\expandafter\expandafter\def
  \expandafter\expandafter\expandafter\x
  \expandafter\expandafter\expandafter{%
    \expandafter\fi\if`#1\relax\else
  }%
}
\def\ETE@Gin@setfile#1#2#3{%
  \ifETE@disable
    \ETE@OrgGin@setfile{#1}{#2}{#3}%
  \else
    \begingroup
      \ETE@InsideSetfiletrue
      \ETE@DefX{#3}%
    \expandafter\endgroup
    \ifx\x\@empty
      \ETE@OrgGin@setfile{#1}{#2}{#3}%
    \else
      \begingroup
        \ETE@InsideSetfiletrue
        \def\GraphicsType{#1}%
        \def\GraphicsRead{#2}%
        \ifETE@prefersuffix
        \else
          \ifx\ETE@suffix\@empty
          \else
            \ETE@IfFileExists{\Gin@base\GraphicsRead}{%
              \ETE@Skip{%
                \ETE@OrgGin@setfile{#1}{#2}{\Gin@base#2}%
              }%
            }{%
              \let\next\relax
            }%
            \next
          \fi
        \fi
        \ifx\Gin@ext\relax
          \let\SourceExt\Gin@eext
          \def\SourceFile{\Gin@base\Gin@eext}%
        \else
          \let\SourceExt\Gin@ext
          \def\SourceFile{\Gin@base\Gin@ext}%
        \fi
        \edef\SourceExt{% remove dot
          \expandafter\@cdr\SourceExt\@empty\@nil
        }%
        \let\OutputDirectory\ETE@outdir
        \ifx\OutputDirectory\@empty
          \edef\OutputFile{\ETE@GenerateName{\Gin@base}{#2}}%
        \else
          \begingroup
            \filename@parse{\Gin@base#2}%
            \edef\x{\endgroup
              \def\noexpand\OutputFile{%
                \ETE@GenerateName{%
                  \OutputDirectory\filename@base
                }{#2}%
              }%
            }%
          \x
        \fi
        \ETE@DefCommandLine{#3}%
        \ifETE@verbose
          \@PackageInfo{epstopdf}{%
            \ETE@FileInfo{Source}\SourceFile\MessageBreak
            \ETE@FileInfo{Output}\OutputFile\MessageBreak
            Command: <\CommandLine>\MessageBreak
            \string\includegraphics
          }%
        \fi
        \ETE@Make\SourceFile\OutputFile{%
          \pdf@system{\CommandLine}%
          \ifETE@verbose
            \@PackageInfoNoLine{epstopdf}{%
              \ETE@FileInfo{Result}\OutputFile
            }%
          \fi
        }%
        \edef\x{\endgroup
          \ifx\OutputDirectory\@empty
          \else
            \def\noexpand\Gin@base{%
              \OutputDirectory\noexpand\filename@base
            }%
          \fi
          \ifx\ETE@suffix\@empty
          \else
            \edef\noexpand\Gin@base{%
              \noexpand\Gin@base\ETE@suffix
            }%
          \fi
          \noexpand\ETE@OrgGin@setfile{%
            \GraphicsType
          }{%
            \GraphicsRead
          }{%
            \OutputFile
          }%
        }%
      \x
    \fi
  \fi
}
\let\Gin@setfile\ETE@Gin@setfile
\def\ETE@GenerateName#1#2{%
  #1\ETE@suffix#2%
}
%    \end{macrocode}
%
% \subsection{Declare graphics rule}
%
%    \begin{macrocode}
\newcommand*{\epstopdfDeclareGraphicsRule}[4]{%
  \ifx\\#4\\%
    \@PackageError{epstopdf-base}{%
      Conversion command is missing%
    }\@ehc
  \else
    \begingroup
      \@ifundefined{Gin@rule@#1}{%
      }{%
        \@PackageInfo{epstopdf-base}{%
          Redefining graphics rule for `#1'%
        }%
      }%
    \endgroup
    \@namedef{Gin@rule@#1}##1{{#2}{#3}{\epstopdfcall{#4}}}%
  \fi
}
%    \end{macrocode}
%
% \cs{DeclareGraphicsRule} for .eps
%    \begin{macrocode}
\epstopdfDeclareGraphicsRule{.eps}{pdf}{.pdf}{%
  \ETE@epstopdf{#1}%
}
\def\ETE@epstopdf#1{%
  \ETE@program@epstopdf\space
  \ifcase\ifx\OutputDirectory\@empty
           \ifx\ETE@suffix\@empty
             1%
           \fi
         \fi
         0 %
    --outfile=\OutputFile\space
  \fi
  #1%
}
%    \end{macrocode}
%
%    \begin{macrocode}
\ifETE@prepend
  \expandafter\PrependGraphicsExtensions
\else
  \expandafter\AppendGraphicsExtensions
\fi
{.eps}
\let\ETE@prepend\@undefined
\DeclareVoidOption{prepend}{%
  \PrependGraphicsExtensions{.eps}%
}
\let\ETE@append\@undefined
\DeclareVoidOption{append}{%
  \AppendGraphicsExtensions{.eps}%
}
%    \end{macrocode}
%
%    \begin{macrocode}
\InputIfFileExists{epstopdf-sys.cfg}{}{}
\InputIfFileExists{epstopdf.cfg}{}{}
%    \end{macrocode}
%
%    \begin{macrocode}
\ETE@AtEnd%
%</base>
%    \end{macrocode}
%
% \section{Test}
%
% \subsection{Preface for standard catcode check}
%
%    \begin{macrocode}
%<*test1>
\input miniltx.tex\relax
\def\Gin@driver{pdftex.def}
\input graphicx.sty\relax
\resetatcatcode
%</test1>
%    \end{macrocode}
%
% \subsection{Catcode checks for loading}
%
%    \begin{macrocode}
%<*test1>
%    \end{macrocode}
%    \begin{macrocode}
\catcode`\{=1 %
\catcode`\}=2 %
\catcode`\#=6 %
\catcode`\@=11 %
\expandafter\ifx\csname count@\endcsname\relax
  \countdef\count@=255 %
\fi
\expandafter\ifx\csname @gobble\endcsname\relax
  \long\def\@gobble#1{}%
\fi
\expandafter\ifx\csname @firstofone\endcsname\relax
  \long\def\@firstofone#1{#1}%
\fi
\expandafter\ifx\csname loop\endcsname\relax
  \expandafter\@firstofone
\else
  \expandafter\@gobble
\fi
{%
  \def\loop#1\repeat{%
    \def\body{#1}%
    \iterate
  }%
  \def\iterate{%
    \body
      \let\next\iterate
    \else
      \let\next\relax
    \fi
    \next
  }%
  \let\repeat=\fi
}%
\def\RestoreCatcodes{}
\count@=0 %
\loop
  \edef\RestoreCatcodes{%
    \RestoreCatcodes
    \catcode\the\count@=\the\catcode\count@\relax
  }%
\ifnum\count@<255 %
  \advance\count@ 1 %
\repeat

\def\RangeCatcodeInvalid#1#2{%
  \count@=#1\relax
  \loop
    \catcode\count@=15 %
  \ifnum\count@<#2\relax
    \advance\count@ 1 %
  \repeat
}
\def\RangeCatcodeCheck#1#2#3{%
  \count@=#1\relax
  \loop
    \ifnum#3=\catcode\count@
    \else
      \errmessage{%
        Character \the\count@\space
        with wrong catcode \the\catcode\count@\space
        instead of \number#3%
      }%
    \fi
  \ifnum\count@<#2\relax
    \advance\count@ 1 %
  \repeat
}
\def\space{ }
\expandafter\ifx\csname LoadCommand\endcsname\relax
  \def\LoadCommand{\input epstopdf.sty\relax}%
\fi
\def\Test{%
  \RangeCatcodeInvalid{0}{47}%
  \RangeCatcodeInvalid{58}{64}%
  \RangeCatcodeInvalid{91}{96}%
  \RangeCatcodeInvalid{123}{255}%
  \catcode`\@=12 %
  \catcode`\\=0 %
  \catcode`\%=14 %
  \LoadCommand
  \RangeCatcodeCheck{0}{36}{15}%
  \RangeCatcodeCheck{37}{37}{14}%
  \RangeCatcodeCheck{38}{47}{15}%
  \RangeCatcodeCheck{48}{57}{12}%
  \RangeCatcodeCheck{58}{63}{15}%
  \RangeCatcodeCheck{64}{64}{12}%
  \RangeCatcodeCheck{65}{90}{11}%
  \RangeCatcodeCheck{91}{91}{15}%
  \RangeCatcodeCheck{92}{92}{0}%
  \RangeCatcodeCheck{93}{96}{15}%
  \RangeCatcodeCheck{97}{122}{11}%
  \RangeCatcodeCheck{123}{255}{15}%
  \RestoreCatcodes
}
\Test
\csname @@end\endcsname
\end
%    \end{macrocode}
%    \begin{macrocode}
%</test1>
%    \end{macrocode}
%
% \section{Installation}
%
% \subsection{Download}
%
% \paragraph{Package.} This package is available on
% CTAN\footnote{\url{http://ctan.org/pkg/epstopdf}}:
% \begin{description}
% \item[\CTAN{macros/latex/contrib/oberdiek/epstopdf.dtx}] The source file.
% \item[\CTAN{macros/latex/contrib/oberdiek/epstopdf.pdf}] Documentation.
% \end{description}
%
%
% \paragraph{Bundle.} All the packages of the bundle `oberdiek'
% are also available in a TDS compliant ZIP archive. There
% the packages are already unpacked and the documentation files
% are generated. The files and directories obey the TDS standard.
% \begin{description}
% \item[\CTAN{install/macros/latex/contrib/oberdiek.tds.zip}]
% \end{description}
% \emph{TDS} refers to the standard ``A Directory Structure
% for \TeX\ Files'' (\CTAN{tds/tds.pdf}). Directories
% with \xfile{texmf} in their name are usually organized this way.
%
% \subsection{Bundle installation}
%
% \paragraph{Unpacking.} Unpack the \xfile{oberdiek.tds.zip} in the
% TDS tree (also known as \xfile{texmf} tree) of your choice.
% Example (linux):
% \begin{quote}
%   |unzip oberdiek.tds.zip -d ~/texmf|
% \end{quote}
%
% \paragraph{Script installation.}
% Check the directory \xfile{TDS:scripts/oberdiek/} for
% scripts that need further installation steps.
% Package \xpackage{attachfile2} comes with the Perl script
% \xfile{pdfatfi.pl} that should be installed in such a way
% that it can be called as \texttt{pdfatfi}.
% Example (linux):
% \begin{quote}
%   |chmod +x scripts/oberdiek/pdfatfi.pl|\\
%   |cp scripts/oberdiek/pdfatfi.pl /usr/local/bin/|
% \end{quote}
%
% \subsection{Package installation}
%
% \paragraph{Unpacking.} The \xfile{.dtx} file is a self-extracting
% \docstrip\ archive. The files are extracted by running the
% \xfile{.dtx} through \plainTeX:
% \begin{quote}
%   \verb|tex epstopdf.dtx|
% \end{quote}
%
% \paragraph{TDS.} Now the different files must be moved into
% the different directories in your installation TDS tree
% (also known as \xfile{texmf} tree):
% \begin{quote}
% \def\t{^^A
% \begin{tabular}{@{}>{\ttfamily}l@{ $\rightarrow$ }>{\ttfamily}l@{}}
%   epstopdf.sty & tex/latex/oberdiek/epstopdf.sty\\
%   epstopdf-base.sty & tex/latex/oberdiek/epstopdf-base.sty\\
%   epstopdf.pdf & doc/latex/oberdiek/epstopdf.pdf\\
%   test/epstopdf-test1.tex & doc/latex/oberdiek/test/epstopdf-test1.tex\\
%   epstopdf.dtx & source/latex/oberdiek/epstopdf.dtx\\
% \end{tabular}^^A
% }^^A
% \sbox0{\t}^^A
% \ifdim\wd0>\linewidth
%   \begingroup
%     \advance\linewidth by\leftmargin
%     \advance\linewidth by\rightmargin
%   \edef\x{\endgroup
%     \def\noexpand\lw{\the\linewidth}^^A
%   }\x
%   \def\lwbox{^^A
%     \leavevmode
%     \hbox to \linewidth{^^A
%       \kern-\leftmargin\relax
%       \hss
%       \usebox0
%       \hss
%       \kern-\rightmargin\relax
%     }^^A
%   }^^A
%   \ifdim\wd0>\lw
%     \sbox0{\small\t}^^A
%     \ifdim\wd0>\linewidth
%       \ifdim\wd0>\lw
%         \sbox0{\footnotesize\t}^^A
%         \ifdim\wd0>\linewidth
%           \ifdim\wd0>\lw
%             \sbox0{\scriptsize\t}^^A
%             \ifdim\wd0>\linewidth
%               \ifdim\wd0>\lw
%                 \sbox0{\tiny\t}^^A
%                 \ifdim\wd0>\linewidth
%                   \lwbox
%                 \else
%                   \usebox0
%                 \fi
%               \else
%                 \lwbox
%               \fi
%             \else
%               \usebox0
%             \fi
%           \else
%             \lwbox
%           \fi
%         \else
%           \usebox0
%         \fi
%       \else
%         \lwbox
%       \fi
%     \else
%       \usebox0
%     \fi
%   \else
%     \lwbox
%   \fi
% \else
%   \usebox0
% \fi
% \end{quote}
% If you have a \xfile{docstrip.cfg} that configures and enables \docstrip's
% TDS installing feature, then some files can already be in the right
% place, see the documentation of \docstrip.
%
% \subsection{Refresh file name databases}
%
% If your \TeX~distribution
% (\teTeX, \mikTeX, \dots) relies on file name databases, you must refresh
% these. For example, \teTeX\ users run \verb|texhash| or
% \verb|mktexlsr|.
%
% \subsection{Some details for the interested}
%
% \paragraph{Attached source.}
%
% The PDF documentation on CTAN also includes the
% \xfile{.dtx} source file. It can be extracted by
% AcrobatReader 6 or higher. Another option is \textsf{pdftk},
% e.g. unpack the file into the current directory:
% \begin{quote}
%   \verb|pdftk epstopdf.pdf unpack_files output .|
% \end{quote}
%
% \paragraph{Unpacking with \LaTeX.}
% The \xfile{.dtx} chooses its action depending on the format:
% \begin{description}
% \item[\plainTeX:] Run \docstrip\ and extract the files.
% \item[\LaTeX:] Generate the documentation.
% \end{description}
% If you insist on using \LaTeX\ for \docstrip\ (really,
% \docstrip\ does not need \LaTeX), then inform the autodetect routine
% about your intention:
% \begin{quote}
%   \verb|latex \let\install=y\input{epstopdf.dtx}|
% \end{quote}
% Do not forget to quote the argument according to the demands
% of your shell.
%
% \paragraph{Generating the documentation.}
% You can use both the \xfile{.dtx} or the \xfile{.drv} to generate
% the documentation. The process can be configured by the
% configuration file \xfile{ltxdoc.cfg}. For instance, put this
% line into this file, if you want to have A4 as paper format:
% \begin{quote}
%   \verb|\PassOptionsToClass{a4paper}{article}|
% \end{quote}
% An example follows how to generate the
% documentation with pdf\LaTeX:
% \begin{quote}
%\begin{verbatim}
%pdflatex epstopdf.dtx
%makeindex -s gind.ist epstopdf.idx
%pdflatex epstopdf.dtx
%makeindex -s gind.ist epstopdf.idx
%pdflatex epstopdf.dtx
%\end{verbatim}
% \end{quote}
%
% \section{Catalogue}
%
% The following XML file can be used as source for the
% \href{http://mirror.ctan.org/help/Catalogue/catalogue.html}{\TeX\ Catalogue}.
% The elements \texttt{caption} and \texttt{description} are imported
% from the original XML file from the Catalogue.
% The name of the XML file in the Catalogue is \xfile{epstopdf-pkg.xml}.
%    \begin{macrocode}
%<*catalogue>
<?xml version='1.0' encoding='us-ascii'?>
<!DOCTYPE entry SYSTEM 'catalogue.dtd'>
<entry datestamp='$Date$' modifier='$Author$' id='epstopdf-pkg'>
  <name>epstopdf-pkg</name>
  <caption>Call epstopdf "on the fly"</caption>
  <authorref id='auth:oberdiek'/>
  <copyright owner='Heiko Oberdiek' year='2001,2006-2010'/>
  <license type='lppl1.3'/>
  <version number='2.6'/>
  <description>
    The package adds support for EPS files in the
    <xref refid='graphicx'>graphicx</xref> package when running under
    PDFTeX.  If an EPS graphic is detected, the package spawns a
    process to convert the EPS to PDF, using the script
    <xref refid='epstopdf'>epstopdf</xref>.  This of course requires
    that shell escape is enabled for the PDFTeX run.
    <p/>
    The package is part of the <xref refid='oberdiek'>oberdiek</xref>
    bundle.
  </description>
  <documentation details='Package documentation'
      href='ctan:/macros/latex/contrib/oberdiek/epstopdf.pdf'/>
  <ctan file='true' path='/macros/latex/contrib/oberdiek/epstopdf.dtx'/>
  <miktex location='oberdiek'/>
  <texlive location='oberdiek'/>
  <install path='/macros/latex/contrib/oberdiek/oberdiek.tds.zip'/>
</entry>
%</catalogue>
%    \end{macrocode}
%
% \begin{History}
%   \begin{Version}{2001/01/06 v1.0}
%   \item
%     First public version,
%     published in the \pdfTeX\ mailing list.
%   \end{Version}
%   \begin{Version}{2001/02/04 v1.1}
%   \item
%     Minor documentation update.
%   \item
%     CTAN.
%   \end{Version}
%   \begin{Version}{2006/02/20 v1.2}
%   \item
%     DTX framework.
%   \item
%     Compatibility for \xfile{miniltx.tex}.
%   \end{Version}
%   \begin{Version}{2006/08/26 v1.3}
%   \item
%     Check for \cs{write18} if available and print a warning
%     if the feature is not enabled.
%   \end{Version}
%   \begin{Version}{2007/04/26 v1.4}
%   \item
%     Documentation rewritten and extended.
%   \end{Version}
%   \begin{Version}{2007/10/02 v1.5}
%   \item
%     New option \xoption{update}: If the converted file exists, it will be only
%     converted if it is out of date.
%   \item
%     Updating the extension list is delegated to package \xpackage{grfext}.
%     Fine tuning is done by the new options \xoption{append}, \xoption{prepend}.
%   \item
%     New option \xoption{outdir} for changing the output directory.
%   \item
%     New option \xoption{verbose}.
%   \item
%     \cs{SourceFile} and \cs{OutputFile} introduced.
%   \item
%     Configuration file support added.
%   \end{Version}
%   \begin{Version}{2007/11/11 v1.6}
%   \item
%     Use of package \xpackage{pdftexcmds} for \LuaTeX\ support.
%   \end{Version}
%   \begin{Version}{2008/05/06 v1.7}
%   \item
%     Warning messages uses ``loaded'' instead of ``found''.
%   \end{Version}
%   \begin{Version}{2009/03/01 v1.8}
%   \item
%     Warning message for missing \xfile{pdftex.def} changed.
%   \end{Version}
%   \begin{Version}{2009/07/06 v1.9}
%   \item
%     Option \xoption{suffix} added.
%   \end{Version}
%   \begin{Version}{2009/07/07 v1.10}
%   \item
%     \cs{SourceExt} added.
%   \item
%     If option \xoption{suffix} is set, the inclusion of
%     an image without the suffix namespace is preferred over generating the
%     the image within the suffix namespace.
%   \end{Version}
%   \begin{Version}{2009/07/12 v2.0}
%   \item
%     New default settings.
%   \item
%     Package is split into \xpackage{epstopdf} that only takes package
%     options and loads \xpackage{epstopdf-base} that does the work.
%   \item
%     \cs{epstopdfDeclareGraphicsRule} and \cs{epstopdfcall} added.
%   \item
%     \xfile{epstopdf-sys.cfg} is loaded before \xfile{epstopdf.cfg}
%     if \xfile{epstopdf-sys.cfg} exists.
%   \end{Version}
%   \begin{Version}{2009/07/15 v2.1}
%   \item
%     Default setting: \xoption{verbose} is now turned on as the
%     documentation for v2.0 said.
%   \item
%     Documentation fixes.
%   \end{Version}
%   \begin{Version}{2009/07/16 v2.2}
%   \item
%     Fixed redefined \cs{Gin@setfile}.
%   \item
%     Documentation extended for package options.
%   \end{Version}
%   \begin{Version}{2009/09/24 v2.3}
%   \item
%     Bug fix for the case that both option \xoption{suffix} and
%     \xoption{outdir} are used.
%   \end{Version}
%   \begin{Version}{2009/10/17 v2.4}
%   \item
%     The name of the program `epstopdf' can be configured via
%     the new option \xoption{program@epstopdf}.
%   \end{Version}
%   \begin{Version}{2016/05/15 v2.5}
%   \item
%     Wording of warning message fixed (Karl Berry).
%   \item
%     \cs{ETE@Gin@setfile} added (Karl Berry).
%   \end{Version}
%   \begin{Version}{2016/05/15 v2.6}
%   \item
%     luaTeX compatibility
%   \end{Version}
% \end{History}
%
% \PrintIndex
%
% \Finale
\endinput

%        (quote the arguments according to the demands of your shell)
%
% Documentation:
%    (a) If epstopdf.drv is present:
%           latex epstopdf.drv
%    (b) Without epstopdf.drv:
%           latex epstopdf.dtx; ...
%    The class ltxdoc loads the configuration file ltxdoc.cfg
%    if available. Here you can specify further options, e.g.
%    use A4 as paper format:
%       \PassOptionsToClass{a4paper}{article}
%
%    Programm calls to get the documentation (example):
%       pdflatex epstopdf.dtx
%       makeindex -s gind.ist epstopdf.idx
%       pdflatex epstopdf.dtx
%       makeindex -s gind.ist epstopdf.idx
%       pdflatex epstopdf.dtx
%
% Installation:
%    TDS:tex/latex/oberdiek/epstopdf.sty
%    TDS:tex/latex/oberdiek/epstopdf-base.sty
%    TDS:doc/latex/oberdiek/epstopdf.pdf
%    TDS:doc/latex/oberdiek/test/epstopdf-test1.tex
%    TDS:source/latex/oberdiek/epstopdf.dtx
%
%<*ignore>
\begingroup
  \catcode123=1 %
  \catcode125=2 %
  \def\x{LaTeX2e}%
\expandafter\endgroup
\ifcase 0\ifx\install y1\fi\expandafter
         \ifx\csname processbatchFile\endcsname\relax\else1\fi
         \ifx\fmtname\x\else 1\fi\relax
\else\csname fi\endcsname
%</ignore>
%<*install>
\input docstrip.tex
\Msg{************************************************************************}
\Msg{* Installation}
\Msg{* Package: epstopdf 2016/05/15 v2.6 Conversion with epstopdf on the fly (HO)}
\Msg{************************************************************************}

\keepsilent
\askforoverwritefalse

\let\MetaPrefix\relax
\preamble

This is a generated file.

Project: epstopdf
Version: 2016/05/15 v2.6

Copyright (C) 2001, 2006-2010 by
   Heiko Oberdiek <heiko.oberdiek at googlemail.com>

This work may be distributed and/or modified under the
conditions of the LaTeX Project Public License, either
version 1.3c of this license or (at your option) any later
version. This version of this license is in
   http://www.latex-project.org/lppl/lppl-1-3c.txt
and the latest version of this license is in
   http://www.latex-project.org/lppl.txt
and version 1.3 or later is part of all distributions of
LaTeX version 2005/12/01 or later.

This work has the LPPL maintenance status "maintained".

This Current Maintainer of this work is Heiko Oberdiek.

This work consists of the main source file epstopdf.dtx
and the derived files
   epstopdf.sty, epstopdf.pdf, epstopdf.ins, epstopdf.drv,
   epstopdf-base.sty, epstopdf-test1.tex.

\endpreamble
\let\MetaPrefix\DoubleperCent

\generate{%
  \file{epstopdf.ins}{\from{epstopdf.dtx}{install}}%
  \file{epstopdf.drv}{\from{epstopdf.dtx}{driver}}%
  \usedir{tex/latex/oberdiek}%
  \file{epstopdf.sty}{\from{epstopdf.dtx}{package}}%
  \file{epstopdf-base.sty}{\from{epstopdf.dtx}{base}}%
  \usedir{doc/latex/oberdiek/test}%
  \file{epstopdf-test1.tex}{\from{epstopdf.dtx}{test1}}%
  \nopreamble
  \nopostamble
  \usedir{source/latex/oberdiek/catalogue}%
  \file{epstopdf-pkg.xml}{\from{epstopdf.dtx}{catalogue}}%
}

\catcode32=13\relax% active space
\let =\space%
\Msg{************************************************************************}
\Msg{*}
\Msg{* To finish the installation you have to move the following}
\Msg{* files into a directory searched by TeX:}
\Msg{*}
\Msg{*     epstopdf.sty, epstopdf-base.sty}
\Msg{*}
\Msg{* To produce the documentation run the file `epstopdf.drv'}
\Msg{* through LaTeX.}
\Msg{*}
\Msg{* Happy TeXing!}
\Msg{*}
\Msg{************************************************************************}

\endbatchfile
%</install>
%<*ignore>
\fi
%</ignore>
%<*driver>
\NeedsTeXFormat{LaTeX2e}
\ProvidesFile{epstopdf.drv}%
  [2016/05/15 v2.6 Conversion with epstopdf on the fly (HO)]%
\documentclass{ltxdoc}
\usepackage{holtxdoc}[2011/11/22]
\begin{document}
  \DocInput{epstopdf.dtx}%
\end{document}
%</driver>
% \fi
%
%
% \CharacterTable
%  {Upper-case    \A\B\C\D\E\F\G\H\I\J\K\L\M\N\O\P\Q\R\S\T\U\V\W\X\Y\Z
%   Lower-case    \a\b\c\d\e\f\g\h\i\j\k\l\m\n\o\p\q\r\s\t\u\v\w\x\y\z
%   Digits        \0\1\2\3\4\5\6\7\8\9
%   Exclamation   \!     Double quote  \"     Hash (number) \#
%   Dollar        \$     Percent       \%     Ampersand     \&
%   Acute accent  \'     Left paren    \(     Right paren   \)
%   Asterisk      \*     Plus          \+     Comma         \,
%   Minus         \-     Point         \.     Solidus       \/
%   Colon         \:     Semicolon     \;     Less than     \<
%   Equals        \=     Greater than  \>     Question mark \?
%   Commercial at \@     Left bracket  \[     Backslash     \\
%   Right bracket \]     Circumflex    \^     Underscore    \_
%   Grave accent  \`     Left brace    \{     Vertical bar  \|
%   Right brace   \}     Tilde         \~}
%
% \GetFileInfo{epstopdf.drv}
%
% \title{The \xpackage{epstopdf} package}
% \date{2016/05/15 v2.6}
% \author{Heiko Oberdiek\thanks
% {Please report any issues at https://github.com/ho-tex/oberdiek/issues}\\
% \xemail{heiko.oberdiek at googlemail.com}}
%
% \maketitle
%
% \begin{abstract}
% This packages adds support of handling eps images
% to package \xpackage{graphics} or \xpackage{graphicx}
% with option \xoption{pdftex}. If an eps image is detected,
% epstopdf is automatically called to convert it to pdf format.
% \end{abstract}
%
% \tableofcontents
%
% \providecommand*{\xprogram}[1]{\textsf{#1}}
%
% \section{Documentation}
%
% \subsection{Introduction}
%
% \LaTeX\ provides its graphics bundle to include graphics files.
% Both packages \xpackage{graphics} or \xpackage{graphicx} may
% be used. the latter one loads the first and adds options
% in key value style for \cs{includegraphics}.
%
% Usually the drivers do not support all kind of graphics files.
% Other image types must be converted, before they become
% usuable. In case of driver \xprogram{dvips}, the graphics
% rule may contain a conversion rule. Then all that package
% \xpackage{graphics} must know is the bounding box, the command
% is passed to \xprogram{dvips} that calls it and embeds the
% converted image.
%
% However, \pdfTeX\ has its driver for PDF output already
% build in. It's graphics inclusion commands (\cs{pdfximage})
% does not allow the execution of external commands.
% Therefore commands in the last argument of
% \cs{DeclareGraphicsRule} were of no use.
% But external programs can be called within \pdfTeX.
% This feature is called ``shell escape'' or ``write 18''
% and must usually enabled explicitely because of security reasons.
% Now, this package \xpackage{epstopdf} hooks
% into package \xpackage{graphics}' code to catch
% that argument with the external command and executes it
% to convert the graphics file to a supported format and
% passes the control of graphics inclusion back to package
% \xpackage{graphics}.
%
% \subsection{Requirements}
%
% \begin{itemize}
% \item
%   The feature \cs{write18} must be enabled. This allows the
%   running of external programs during \TeX's compile run.
%   Keep in mind that this is a security risk. The feature is an
%   addition to \cs{TeX}. Mik\TeX, te\TeX, \TeX\ Live support it.
%   In Web2C based \TeX\ distributions (te\TeX, \TeX\ Live) it can be
%   enabled in the configuration file \xfile{texmf.cnf}:
%   \begin{quote}
%     |shell_escape = 1|
%   \end{quote}
%   Because of the security risk, it is better to do it on the
%   command line only:
%   \begin{quote}
%   \texttt{--shell-escape} (te\TeX, \TeX\ Live)\\
%   \texttt{--enable-write18} (MiK\TeX)
%   \end{quote}
%   Example:
%   \begin{quote}
%     |pdflatex -shell-escape test.tex|
%   \end{quote}
% \item
%   The program \xprogram{epstopdf} for the conversion from EPS
%   to PDF. However, other programs can
%   be used and configured by \cs{DeclareGraphicsRule}.
%   Example:
%   \begin{quote}
%   |\epstopdfDeclareGraphicsRule{.eps}{pdf}{.pdf}{%|\\
%   |  ps2pdf -dEPSCrop #1 \OutputFile|\\
%   |}|
%   \end{quote}
% \end{itemize}
%
% \subsection{Usage}
%
% The package is loaded after graphic\{s,x\}, e.g.:
% \begin{quote}
%   |\usepackage[pdftex]{graphicx}|\\
%   |\usepackage{epstopdf}|
% \end{quote}
% Now images with file name extension \xfile{.eps} are detected
% and supported using \cs{includegraphics}.
%
% If the graphics file name is explicitly specified with extension
% \xfile{.eps} the new rule for EPS files is called and
% the conversion performed. If option \xoption{update} is in force
% then the conversion step is dropped if the target file already
% exists and is not older then the EPS file.
%
% The situation is more complicate if the graphics file is
% given without file name extension. Then the \xpackage{graphics}
% package must search for a supported image file.
% The possible extensions are stored in the graphics extension
% list, that can be set by \cs{DeclareGraphicsExtensions}.
% The algorithm:
% \begin{quote}
%    \def\>{\mbox{}\quad}
%    function search( \meta{filebase} )\\
%    \>foreach \meta{ext} in \meta{graphics extensions}\\
%    \>\>foreach \meta{dir} in \meta{current directory}, \meta{\cs{graphicspath}}\\
%    \>\>\>\meta{file} := \meta{dir} + \meta{filebase} + \meta{ext}\\
%    \>\>\>if exist \meta{file}\\
%    \>\>\>\>return found\\
%    \>return not found
% \end{quote}
% Package \xpackage{epstopdf} puts \xfile{.eps} at the end of the
% graphics extension search list. This is the behaviour of option
% \xoption{append} that is enabled by default.
% That means, the conversion is called last
% unless a supported file type cannot be found earlier. This avoids
% unnecessary conversion steps that slow down the \LaTeX\ run.
% If you want to use option \xoption{update} and your \pdfTeX\ supports
% it, then an outdated PDF file also would be found earlier unless
% \xoption{suffix} is used that is the default since version 2.0.
%
% With an empty option \xoption{suffix} and option
% \xoption{prepend} there is a risk that an original PDF file is overwritten:
%
% If the original image file is the PDF file and there is also a generated
% EPS file, then the original PDF file can be regenerated (depending
% on the option settings) and the original PDF file gets lost.
% Therefore option \xoption{suffix} is introduced in version 1.9
% to create a separate name space for generated output files.
%
% \textbf{Note:} Usually the conversion program needs the exact location
% of the image file. Usually the current directory works.
% Also if the image file is found using \cs{graphicspath}, the
% location is known. However, if the image is somewhere
% in a directory of environment variable |TEXINPUTS|, then
% the package does not know the exact location and the
% conversion program will not find the image file unless it
% implements a search using |TEXINPUTS| (program \xprogram{kpsewhich}
% may be of help in this task).
%
% \subsection{Options}
%
% Options can be given as package options or later using:
% \begin{declcs}{epstopdfsetup} \M{key value list}
% \end{declcs}
% \LaTeX\ expands the option list before passing the option list
% to the package's option handling code. This can fail
% for option \xoption{suffix} if it contains some of the macros
% described below. Use \cs{epstopdfsetup} after the package is loaded.
% Or load package \xpackage{kvoptions-patch} before. This package
% is also loaded by option \xoption{patch} of package \xpackage{kvoptions}.
% \LaTeX's option code is redefined to respect key value options and
% let the values untouched.
%
% \begin{description}
% \def\It#1{\item[\xoption{#1}:]}^^A
% \It{update}
%   The conversion program is only called, if the target file
%   does not exist or is older than the source image file.
% \It{append}
%   Puts the extension \xfile{.eps} at the end of the graphics extension list
%   (default).
% \It{prepend}
%   Puts the extension \xfile{.eps} at the begin of the graphics extension list.
% \It{outdir}
%   The converted file may put in an other output directory.
%   The value of \xoption{outdir} must include the directory separator.
%   Example for the current directory:
%   \begin{quote}
%   |\epstopdfsetup{outdir=./}|
%   \end{quote}
%   For other directories ensure, that they can be found.
%   See \cs{graphicspath} or |TEXINPUTS|.
% \It{suffix}
%   This option takes a string that is put between the file name base
%   and the extension of the output file. Rationale: It can happen,
%   that a PDF file is the original file and the EPS file the
%   generated file. If now the package thinks, that the PDF file
%   is the generated file, it will `regnerate' it. But in reality
%   the original file is lost. Therefore I recommend to use this
%   option always to generate a separate name space for generated
%   files. Proposed value is \texttt{-generated} or \texttt{.generated}.
%   The suffix \texttt{.generated} will also work here without the need for
%   package \xpackage{grffile}).
%
%   Example:
%   \begin{quote}
%   |\epstopdfsetup{suffix=-generated}|\\
%   Then \xfile{foo.eps} is converted to \xfile{foo-generated.pdf}.
%   \end{quote}
%   \cs{SourceExt} can be used inside the suffix string.
%   It's will be replaced by the extension of the image source
%   file without the leading dot, for instance:
%   \begin{quote}
%   |\epstopdfsetup{suffix=-\SourceExt-converted-to}|\\
%   \xfile{foo.eps} $\Rightarrow$ \xfile{foo-eps-converted-to.pdf}
%   \end{quote}
%   See also the next option \xoption{prefersuffix} that
%   modifies the behaviour of option \xoption{suffix} in some
%   cases.
%
%   Default for \xoption{suffix} is `|-\SourceExt-converted-to|'.
%
% \It{prefersuffix}
%   If a suffix is set by option \xoption{suffix}, then there can be
%   two image file names that could be taken into account for inclusion:
%   A image file name with the suffix string inside
%   its name and a image file name without; e.g. for \xfile{foo.eps}
%   the names could be:
%   \begin{quote}
%   \xfile{foo-suffix.pdf}, \xfile{foo.pdf}
%   \end{quote}
%   If option \xoption{perfersuffix} is turned on, the
%   file \xfile{foo-suffix.pdf} and its generation is preferred
%   over using \xfile{foo.pdf}. Otherwise \xfile{foo.pdf} is included
%   without generating \xfile{foo-suffix.pdf}.
%   The default of option \xoption{prefersuffix} is \texttt{true}.
% \It{program@epstopdf}
%   The name for the conversion program from EPS to PDF,
%   default is `|epstopdf|'.
% \It{verbose}
%   It prints some information about the image in the \xfile{.log} file
%   (default).
% \end{description}
%
% \subsection{Configuration}
%
% \subsubsection{System configuration file \xfile{epstopdf-sys.cfg}}
%
% If \xfile{epstopdf-sys} exists it is loaded at the end of the
% package \xpackage{epstopdf-base} and before the user configuration
% file. It's intended for TeX distributors. Thus they could add
% additional conversion rules (e.g., .gif -> .png) or set options.
%
% \subsubsection{User configuration file \xfile{epstopdf.cfg}}
%
% A configuration file \xfile{epstopdf.cfg} is loaded at the end
% of the package if it exists. It can be used for changing
% the default option setting. Example:
% \begin{quote}
% |\epstopdfsetup{verbose=false}|
% \end{quote}
%
% \subsubsection{Conversion program}
%
% You can use \cs{DeclareGraphicsRule} in a similar way as the
% route via \xprogram{dvips} to specify the conversion
% command line.
% The conversion argument starts with a back tick, followed
% by the conversion command including parameters.
%
% The whole conversion argument should also be wrapped inside
% \cs{epstopdfcall}. This reduces the problem with packages
% (e.g. \xpackage{pst-pdf}) that use the conversion argument
% and expands it. Macros \cs{SourceFile}, \cs{OutputFile}, and
% \cs{SourceExt} are not defined outside \xpackage{epstopdf-base}'s
% \cs{Gin@setfile} and error messages because of undefined
% command names are the result. If \cs{epstopdfcall} detects that
% it is called outside \xpackage{epstopdf-base}'s \cs{Gin@setfile}
% then it replaces the conversion argument by
% package \xpackage{graphics}'s
% default, usually the image file.
%
% The following macros are available inside:
% \begin{description}
% \def\It#1{\item[\cs{#1}:]}
% \It{OutputFile}: output file name (with known path and extension)
% \It{SourceFile}: source file name (with known path and extension),
%                  usually the same as |#1|,
% \It{SourceExt}: source extension without leading dot.
% \end{description}
%
% \paragraph{Conversion from EPS to PDF.}
% Other programs than \xprogram{epstopdf} can be used to convert from
% EPS to PDF. Example that uses \xprogram{Ghostscript}:
% \begin{quote}
%   |\DeclareGraphicsRule{.eps}{pdf}{.pdf}{%|\\
%   |  \epstopdfcall{`ps2pdf -dEPSCrop #1 \noexpand\OutputFile}%|\\
%   |}|
% \end{quote}
% \cs{DeclareGraphicsRule} expands the argument, therefore \cs{noexpand}
% is necessary. As convenience package \xpackage{epstopdf-base} defines
% \cs{epstopdfDeclareGraphicsRule}. Then the conversion argument
% is not expanded, \cs{epstopdfcall} and the back tick are added:
% \begin{quote}
%   |\epstopdfDeclareGraphicsRule{.eps}{pdf}{.pdf}{%|\\
%   |  ps2pdf -dEPSCrop #1 \OutputFile|\\
%   |}|
% \end{quote}
%
% Also \cs{OutputFile} respects the setting of option \xoption{outdir}.
%
% \subsection{Other image formats}
%
% The support that package \xpackage{epstopdf} implements is not
% limited to EPS files. Other image conversions can be declared.
% The following example shows it for GIF images under Unix with
% ImageMagick's \xprogram{convert}:
%\begin{quote}
%\begin{verbatim}
%\epstopdfDeclareGraphicsRule{.gif}{png}{.png}{%
%  convert #1 \OutputFile
%}
%\end{verbatim}
%\end{quote}
% The file extension \xfile{.gif} can be added to the extension list that
% package \xpackage{graphics} searches if the file extension
% is not given in \cs{includegraphics}.
% The list can be set by \cs{GraphicsExtensions}.
% \begin{quote}
% |\AppendGraphicsExtensions{.gif}|\\
% or\\
% |\PrependGraphicsExtensions{.gif}|
% \end{quote}
%
% \StopEventually{
% }
%
% \section{Implementation}
%
%    \begin{macrocode}
%<*package>
%    \end{macrocode}
%
% \subsection{Wrapper package}
%
%    Reload check, especially if the package is not used with \LaTeX.
%    \begin{macrocode}
\begingroup\catcode61\catcode48\catcode32=10\relax%
  \catcode13=5 % ^^M
  \endlinechar=13 %
  \catcode35=6 % #
  \catcode39=12 % '
  \catcode44=12 % ,
  \catcode45=12 % -
  \catcode46=12 % .
  \catcode58=12 % :
  \catcode64=11 % @
  \catcode123=1 % {
  \catcode125=2 % }
  \expandafter\let\expandafter\x\csname ver@epstopdf.sty\endcsname
  \ifx\x\relax % plain-TeX, first loading
  \else
    \def\empty{}%
    \ifx\x\empty % LaTeX, first loading,
      % variable is initialized, but \ProvidesPackage not yet seen
    \else
      \expandafter\ifx\csname PackageInfo\endcsname\relax
        \def\x#1#2{%
          \immediate\write-1{Package #1 Info: #2.}%
        }%
      \else
        \def\x#1#2{\PackageInfo{#1}{#2, stopped}}%
      \fi
      \x{epstopdf}{The package is already loaded}%
      \aftergroup\endinput
    \fi
  \fi
\endgroup%
%    \end{macrocode}
%    Package identification:
%    \begin{macrocode}
\begingroup\catcode61\catcode48\catcode32=10\relax%
  \catcode13=5 % ^^M
  \endlinechar=13 %
  \catcode35=6 % #
  \catcode39=12 % '
  \catcode40=12 % (
  \catcode41=12 % )
  \catcode44=12 % ,
  \catcode45=12 % -
  \catcode46=12 % .
  \catcode47=12 % /
  \catcode58=12 % :
  \catcode64=11 % @
  \catcode91=12 % [
  \catcode93=12 % ]
  \catcode123=1 % {
  \catcode125=2 % }
  \expandafter\ifx\csname ProvidesPackage\endcsname\relax
    \def\x#1#2#3[#4]{\endgroup
      \immediate\write-1{Package: #3 #4}%
      \xdef#1{#4}%
    }%
  \else
    \def\x#1#2[#3]{\endgroup
      #2[{#3}]%
      \ifx#1\@undefined
        \xdef#1{#3}%
      \fi
      \ifx#1\relax
        \xdef#1{#3}%
      \fi
    }%
  \fi
\expandafter\x\csname ver@epstopdf.sty\endcsname
\ProvidesPackage{epstopdf}%
  [2016/05/15 v2.6 Conversion with epstopdf on the fly (HO)]%
%    \end{macrocode}
%    Larger catcode set because of configuration files needed.
%    \begin{macrocode}
\begingroup\catcode61\catcode48\catcode32=10\relax%
  \catcode13=5 % ^^M
  \endlinechar=13 %
  \catcode123=1 % {
  \catcode125=2 % }
  \catcode64=11 % @
  \def\x{\endgroup
    \expandafter\edef\csname ETE@AtEnd\endcsname{%
      \endlinechar=\the\endlinechar\relax
      \catcode13=\the\catcode13\relax
      \catcode32=\the\catcode32\relax
      \catcode35=\the\catcode35\relax
      \catcode61=\the\catcode61\relax
      \catcode64=\the\catcode64\relax
      \catcode123=\the\catcode123\relax
      \catcode125=\the\catcode125\relax
    }%
  }%
\x\catcode61\catcode48\catcode32=10\relax%
\catcode13=5 % ^^M
\endlinechar=13 %
\catcode35=6 % #
\catcode64=11 % @
\catcode123=1 % {
\catcode125=2 % }
\def\TMP@EnsureCode#1#2{%
  \edef\ETE@AtEnd{%
    \ETE@AtEnd
    \catcode#1=\the\catcode#1\relax
  }%
  \catcode#1=#2\relax
}
\TMP@EnsureCode{39}{12}% '
\TMP@EnsureCode{40}{12}% (
\TMP@EnsureCode{41}{12}% )
\TMP@EnsureCode{42}{12}% *
\TMP@EnsureCode{45}{12}% -
\TMP@EnsureCode{47}{12}% /
\TMP@EnsureCode{91}{12}% [
\TMP@EnsureCode{93}{12}% ]
\edef\ETE@AtEnd{\ETE@AtEnd\noexpand\endinput}
%    \end{macrocode}
%
%    \begin{macrocode}
\let\ETE@SavedAtEnd\ETE@AtEnd
\RequirePackage{epstopdf-base}[2016/05/15]
\let\ETE@AtEnd\ETE@SavedAtEnd
%    \end{macrocode}
%
% \subsubsection{Option handling}
%
%    \begin{macrocode}
\DeclareOption*{%
  \expandafter\epstopdfsetup\expandafter{\CurrentOption}%
}
\ProcessOptions*\relax
%    \end{macrocode}
%
%    \begin{macrocode}
\ETE@AtEnd%
%</package>
%    \end{macrocode}
%
% \subsection{Base package}
%
%    \begin{macrocode}
%<*base>
%    \end{macrocode}
%
% \subsection{Preparations}
%
% \subsubsection{Relead check and identification}
%
%    Reload check, especially if the package is not used with \LaTeX.
%    \begin{macrocode}
\begingroup\catcode61\catcode48\catcode32=10\relax%
  \catcode13=5 % ^^M
  \endlinechar=13 %
  \catcode35=6 % #
  \catcode39=12 % '
  \catcode44=12 % ,
  \catcode45=12 % -
  \catcode46=12 % .
  \catcode58=12 % :
  \catcode64=11 % @
  \catcode123=1 % {
  \catcode125=2 % }
  \expandafter\let\expandafter\x\csname ver@epstopdf-base.sty\endcsname
  \ifx\x\relax % plain-TeX, first loading
  \else
    \def\empty{}%
    \ifx\x\empty % LaTeX, first loading,
      % variable is initialized, but \ProvidesPackage not yet seen
    \else
      \expandafter\ifx\csname PackageInfo\endcsname\relax
        \def\x#1#2{%
          \immediate\write-1{Package #1 Info: #2.}%
        }%
      \else
        \def\x#1#2{\PackageInfo{#1}{#2, stopped}}%
      \fi
      \x{epstopdf-base}{The package is already loaded}%
      \aftergroup\endinput
    \fi
  \fi
\endgroup%
%    \end{macrocode}
%    Package identification:
%    \begin{macrocode}
\begingroup\catcode61\catcode48\catcode32=10\relax%
  \catcode13=5 % ^^M
  \endlinechar=13 %
  \catcode35=6 % #
  \catcode39=12 % '
  \catcode40=12 % (
  \catcode41=12 % )
  \catcode44=12 % ,
  \catcode45=12 % -
  \catcode46=12 % .
  \catcode47=12 % /
  \catcode58=12 % :
  \catcode64=11 % @
  \catcode91=12 % [
  \catcode93=12 % ]
  \catcode123=1 % {
  \catcode125=2 % }
  \expandafter\ifx\csname ProvidesPackage\endcsname\relax
    \def\x#1#2#3[#4]{\endgroup
      \immediate\write-1{Package: #3 #4}%
      \xdef#1{#4}%
    }%
  \else
    \def\x#1#2[#3]{\endgroup
      #2[{#3}]%
      \ifx#1\@undefined
        \xdef#1{#3}%
      \fi
      \ifx#1\relax
        \xdef#1{#3}%
      \fi
    }%
  \fi
\expandafter\x\csname ver@epstopdf-base.sty\endcsname
\ProvidesPackage{epstopdf-base}%
  [2016/05/15 v2.6 Base part for package epstopdf]%
%    \end{macrocode}
%
% \subsubsection{Catcodes}
%
%    \begin{macrocode}
\begingroup\catcode61\catcode48\catcode32=10\relax%
  \catcode13=5 % ^^M
  \endlinechar=13 %
  \catcode123=1 % {
  \catcode125=2 % }
  \catcode64=11 % @
  \def\x{\endgroup
    \expandafter\edef\csname ETE@AtEnd\endcsname{%
      \endlinechar=\the\endlinechar\relax
      \catcode13=\the\catcode13\relax
      \catcode32=\the\catcode32\relax
      \catcode35=\the\catcode35\relax
      \catcode61=\the\catcode61\relax
      \catcode64=\the\catcode64\relax
      \catcode123=\the\catcode123\relax
      \catcode125=\the\catcode125\relax
    }%
  }%
\x\catcode61\catcode48\catcode32=10\relax%
\catcode13=5 % ^^M
\endlinechar=13 %
\catcode35=6 % #
\catcode64=11 % @
\catcode123=1 % {
\catcode125=2 % }
\def\TMP@EnsureCode#1#2{%
  \edef\ETE@AtEnd{%
    \ETE@AtEnd
    \catcode#1=\the\catcode#1\relax
  }%
  \catcode#1=#2\relax
}
\TMP@EnsureCode{33}{12}% !
\TMP@EnsureCode{39}{12}% '
\TMP@EnsureCode{42}{12}% *
\TMP@EnsureCode{44}{12}% ,
\TMP@EnsureCode{45}{12}% -
\TMP@EnsureCode{46}{12}% .
\TMP@EnsureCode{47}{12}% /
\TMP@EnsureCode{58}{12}% :
\TMP@EnsureCode{60}{12}% <
\TMP@EnsureCode{62}{12}% >
\TMP@EnsureCode{96}{12}% `
\edef\ETE@AtEnd{\ETE@AtEnd\noexpand\endinput}
%    \end{macrocode}
%
% \subsubsection{Load packages}
%
%    \begin{macrocode}
\RequirePackage{infwarerr}[2007/09/09]
\RequirePackage{grfext}\relax
\RequirePackage{kvoptions}[2007/10/02]
\RequirePackage{pdftexcmds}[2007/11/11]
%    \end{macrocode}
%
% \subsection{Checks}
%
% Check, whether package graphics is loaded
% (also graphicx loads graphics).
% Because miniltx.tex does not know \cs{@ifpackageloaded}
% we test for \cs{Gin@setfile} instead.
%    \begin{macrocode}
\begingroup\expandafter\expandafter\expandafter\endgroup
\expandafter\ifx\csname Gin@setfile\endcsname\relax
  \@PackageWarningNoLine{epstopdf}{%
    No graphics package \string`graphic{s,x}\string' loaded%
  }%
  \newcommand*{\epstopdfsetup}[1]{}%
  \expandafter\ETE@AtEnd
\fi%
%    \end{macrocode}
%
% Check, whether pdftex.def is loaded. \cs{ver@pdftex.def} is not
% available with miniltx.tex, thus we test for \cs{Gin@driver}.
%    \begin{macrocode}
\begingroup
  \def\x{luatex.def}%
  \ifx\Gin@driver\x
  \else
  \def\x{pdftex.def}%
  \ifx\Gin@driver\x
  \else
    \@PackageWarningNoLine{epstopdf}{%
      Drivers other than `pdftex' and `luatex' are not supported%
    }%
    \endgroup
    \newcommand*{\epstopdfsetup}[1]{}%
    \expandafter\ETE@AtEnd
  \fi%
  \fi%
\endgroup
%    \end{macrocode}
%
% Check, whether the shell escape feature is enabled.
%
%    \begin{macrocode}
\begingroup
  \expandafter\ifx\csname pdf@shellescape\endcsname\relax
  \else
    \ifnum\pdf@shellescape>0 %
    \else
      \@PackageWarningNoLine{epstopdf}{%
        Shell escape feature is not enabled%
      }%
    \fi
  \fi
\endgroup
%    \end{macrocode}
%
% \subsection{Package loading}
%
%    \begin{macrocode}
%    \end{macrocode}
%
% \subsection{Options}
%
%    \begin{macrocode}
\SetupKeyvalOptions{family=ETE,prefix=ETE@}
\DeclareBoolOption{update}
\DeclareBoolOption{verbose}
\newif\ifETE@prepend
\DeclareVoidOption{prepend}{\ETE@prependtrue}
\DeclareVoidOption{append}{\ETE@prependfalse}
\DeclareStringOption{outdir}
\DeclareStringOption{suffix}
\DeclareBoolOption{prefersuffix}
\DeclareStringOption{program@epstopdf}
%    \end{macrocode}
%    Options \xoption{disable} and \xoption{enable} are for testing
%    only. Therefore they are not documented on purpose.
%    \begin{macrocode}
\DeclareBoolOption{disable}
\DeclareComplementaryOption{enable}{disable}
%    \end{macrocode}
%    \begin{macrocode}
\newcommand*{\epstopdfsetup}{\setkeys{ETE}}
%    \end{macrocode}
%
% \subsubsection{Default setting}
%
%    \begin{macrocode}
\epstopdfsetup{%
  verbose,%
  enable,%
  append,%
  update,%
  prefersuffix,%
  suffix=-\SourceExt-converted-to,%
  program@epstopdf=epstopdf%
}
%    \end{macrocode}
%
% \subsection{Make and verbose}
%
%    \begin{macrocode}
\begingroup\expandafter\expandafter\expandafter\endgroup
\expandafter\ifx\csname pdf@filemoddate\endcsname\relax
  \def\ETE@Make#1#2{%
    \ifETE@update
      \ETE@WarnModDate
    \fi
    \@firstofone
  }%
  \def\ETE@WarnModDate{%
    \@PackageWarningNoLine{epstopdf}{%
      \string\pdffilemoddate\space is not available,\MessageBreak
      option `update' will be ignored%
    }%
    \global\let\ETE@WarnModDate\relax
  }%
  \def\ETE@FileInfo#1#2{#1 file: <#2>}%
\else
  \def\ETE@Make#1#2{%
    \ifETE@update
      \ifnum\pdf@strcmp{\pdf@filemoddate{#1}}{\pdf@filemoddate{#2}}>0 %
        \expandafter\expandafter\expandafter\@firstofone
      \else
        \@PackageInfoNoLine{epstopdf}{%
          Output file is already uptodate%
        }%
        \expandafter\expandafter\expandafter\@gobble
      \fi
    \else
      \expandafter\@firstofone
    \fi
  }%
  \def\ETE@FileInfo#1#2{%
    #1 file: <#2>%
    \expandafter\expandafter\expandafter
    \ETE@Date\pdf@filemoddate{#2}\@nil
    \expandafter\expandafter\expandafter
    \ETE@Size\pdf@filesize{#2}\@nil
  }%
  \def\ETE@Date#1\@nil{%
    \ifx\\#1\\%
    \else
      \ETE@@Date#1\@nil
    \fi
  }%
  \def\ETE@@Date#1:#2#3#4#5#6#7#8#9{%
    \MessageBreak
    \@spaces\space\space\space date: #2#3#4#5-#6#7-#8#9 %
    \ETE@@Time
  }%
  \def\ETE@@Time#1#2#3#4#5#6#7\@nil{%
    #1#2:#3#4:#5#6%
  }%
  \def\ETE@Size#1\@nil{%
    \ifx\\#1\\%
    \else
      \MessageBreak
      \@spaces\space\space\space size: #1 bytes%
    \fi
  }%
\fi
%    \end{macrocode}
%
% \subsection{Adding conversion support}
%
%    Patch \cs{Gin@setfile} to execute |#3|, if it contains
%    a command.
%    \begin{macrocode}
\expandafter\ifx\csname ETE@OrgGin@setfile\endcsname\relax
  \let\ETE@OrgGin@setfile\Gin@setfile
\else
  \@PackageError{epstopdf}{%
    Command \string\ETE@OrgGin@setfile\space
    already defined.\MessageBreak
  }{%
    Probably some package has included the code of this package%
    \MessageBreak
    instead of using \string\RequirePackage{epstopdf}.%
    \MessageBreak
    \@ehc
  }%
\fi
\def\ETE@IfFileExists{%
  \begingroup\expandafter\expandafter\expandafter\endgroup
  \expandafter\ifx\csname grffile@IfFileExists\endcsname\relax
    \expandafter\IfFileExists
  \else
    \global\let\ETE@IfFileExists\grffile@IfFileExists
    \expandafter\grffile@IfFileExists
  \fi
}
\def\ETE@Skip#1#2\x\fi\fi{%
  \fi
  \fi
  \endgroup
  \fi
  \fi
  #1%
}
\newif\ifETE@InsideSetfile
\newcommand*{\epstopdfcall}[1]{%
  \ifETE@InsideSetfile
    \expandafter\@firstoftwo
  \else
    \expandafter\@secondoftwo
  \fi
  {`#1}%
  {\Gin@base\Gin@ext}%
}
\def\ETE@DefCommandLine#1{%
  \edef\CommandLine{\expandafter\fi\if`#1}%
}
\def\ETE@DefX#1{%
  \expandafter\expandafter\expandafter\def
  \expandafter\expandafter\expandafter\x
  \expandafter\expandafter\expandafter{%
    \expandafter\fi\if`#1\relax\else
  }%
}
\def\ETE@Gin@setfile#1#2#3{%
  \ifETE@disable
    \ETE@OrgGin@setfile{#1}{#2}{#3}%
  \else
    \begingroup
      \ETE@InsideSetfiletrue
      \ETE@DefX{#3}%
    \expandafter\endgroup
    \ifx\x\@empty
      \ETE@OrgGin@setfile{#1}{#2}{#3}%
    \else
      \begingroup
        \ETE@InsideSetfiletrue
        \def\GraphicsType{#1}%
        \def\GraphicsRead{#2}%
        \ifETE@prefersuffix
        \else
          \ifx\ETE@suffix\@empty
          \else
            \ETE@IfFileExists{\Gin@base\GraphicsRead}{%
              \ETE@Skip{%
                \ETE@OrgGin@setfile{#1}{#2}{\Gin@base#2}%
              }%
            }{%
              \let\next\relax
            }%
            \next
          \fi
        \fi
        \ifx\Gin@ext\relax
          \let\SourceExt\Gin@eext
          \def\SourceFile{\Gin@base\Gin@eext}%
        \else
          \let\SourceExt\Gin@ext
          \def\SourceFile{\Gin@base\Gin@ext}%
        \fi
        \edef\SourceExt{% remove dot
          \expandafter\@cdr\SourceExt\@empty\@nil
        }%
        \let\OutputDirectory\ETE@outdir
        \ifx\OutputDirectory\@empty
          \edef\OutputFile{\ETE@GenerateName{\Gin@base}{#2}}%
        \else
          \begingroup
            \filename@parse{\Gin@base#2}%
            \edef\x{\endgroup
              \def\noexpand\OutputFile{%
                \ETE@GenerateName{%
                  \OutputDirectory\filename@base
                }{#2}%
              }%
            }%
          \x
        \fi
        \ETE@DefCommandLine{#3}%
        \ifETE@verbose
          \@PackageInfo{epstopdf}{%
            \ETE@FileInfo{Source}\SourceFile\MessageBreak
            \ETE@FileInfo{Output}\OutputFile\MessageBreak
            Command: <\CommandLine>\MessageBreak
            \string\includegraphics
          }%
        \fi
        \ETE@Make\SourceFile\OutputFile{%
          \pdf@system{\CommandLine}%
          \ifETE@verbose
            \@PackageInfoNoLine{epstopdf}{%
              \ETE@FileInfo{Result}\OutputFile
            }%
          \fi
        }%
        \edef\x{\endgroup
          \ifx\OutputDirectory\@empty
          \else
            \def\noexpand\Gin@base{%
              \OutputDirectory\noexpand\filename@base
            }%
          \fi
          \ifx\ETE@suffix\@empty
          \else
            \edef\noexpand\Gin@base{%
              \noexpand\Gin@base\ETE@suffix
            }%
          \fi
          \noexpand\ETE@OrgGin@setfile{%
            \GraphicsType
          }{%
            \GraphicsRead
          }{%
            \OutputFile
          }%
        }%
      \x
    \fi
  \fi
}
\let\Gin@setfile\ETE@Gin@setfile
\def\ETE@GenerateName#1#2{%
  #1\ETE@suffix#2%
}
%    \end{macrocode}
%
% \subsection{Declare graphics rule}
%
%    \begin{macrocode}
\newcommand*{\epstopdfDeclareGraphicsRule}[4]{%
  \ifx\\#4\\%
    \@PackageError{epstopdf-base}{%
      Conversion command is missing%
    }\@ehc
  \else
    \begingroup
      \@ifundefined{Gin@rule@#1}{%
      }{%
        \@PackageInfo{epstopdf-base}{%
          Redefining graphics rule for `#1'%
        }%
      }%
    \endgroup
    \@namedef{Gin@rule@#1}##1{{#2}{#3}{\epstopdfcall{#4}}}%
  \fi
}
%    \end{macrocode}
%
% \cs{DeclareGraphicsRule} for .eps
%    \begin{macrocode}
\epstopdfDeclareGraphicsRule{.eps}{pdf}{.pdf}{%
  \ETE@epstopdf{#1}%
}
\def\ETE@epstopdf#1{%
  \ETE@program@epstopdf\space
  \ifcase\ifx\OutputDirectory\@empty
           \ifx\ETE@suffix\@empty
             1%
           \fi
         \fi
         0 %
    --outfile=\OutputFile\space
  \fi
  #1%
}
%    \end{macrocode}
%
%    \begin{macrocode}
\ifETE@prepend
  \expandafter\PrependGraphicsExtensions
\else
  \expandafter\AppendGraphicsExtensions
\fi
{.eps}
\let\ETE@prepend\@undefined
\DeclareVoidOption{prepend}{%
  \PrependGraphicsExtensions{.eps}%
}
\let\ETE@append\@undefined
\DeclareVoidOption{append}{%
  \AppendGraphicsExtensions{.eps}%
}
%    \end{macrocode}
%
%    \begin{macrocode}
\InputIfFileExists{epstopdf-sys.cfg}{}{}
\InputIfFileExists{epstopdf.cfg}{}{}
%    \end{macrocode}
%
%    \begin{macrocode}
\ETE@AtEnd%
%</base>
%    \end{macrocode}
%
% \section{Test}
%
% \subsection{Preface for standard catcode check}
%
%    \begin{macrocode}
%<*test1>
\input miniltx.tex\relax
\def\Gin@driver{pdftex.def}
\input graphicx.sty\relax
\resetatcatcode
%</test1>
%    \end{macrocode}
%
% \subsection{Catcode checks for loading}
%
%    \begin{macrocode}
%<*test1>
%    \end{macrocode}
%    \begin{macrocode}
\catcode`\{=1 %
\catcode`\}=2 %
\catcode`\#=6 %
\catcode`\@=11 %
\expandafter\ifx\csname count@\endcsname\relax
  \countdef\count@=255 %
\fi
\expandafter\ifx\csname @gobble\endcsname\relax
  \long\def\@gobble#1{}%
\fi
\expandafter\ifx\csname @firstofone\endcsname\relax
  \long\def\@firstofone#1{#1}%
\fi
\expandafter\ifx\csname loop\endcsname\relax
  \expandafter\@firstofone
\else
  \expandafter\@gobble
\fi
{%
  \def\loop#1\repeat{%
    \def\body{#1}%
    \iterate
  }%
  \def\iterate{%
    \body
      \let\next\iterate
    \else
      \let\next\relax
    \fi
    \next
  }%
  \let\repeat=\fi
}%
\def\RestoreCatcodes{}
\count@=0 %
\loop
  \edef\RestoreCatcodes{%
    \RestoreCatcodes
    \catcode\the\count@=\the\catcode\count@\relax
  }%
\ifnum\count@<255 %
  \advance\count@ 1 %
\repeat

\def\RangeCatcodeInvalid#1#2{%
  \count@=#1\relax
  \loop
    \catcode\count@=15 %
  \ifnum\count@<#2\relax
    \advance\count@ 1 %
  \repeat
}
\def\RangeCatcodeCheck#1#2#3{%
  \count@=#1\relax
  \loop
    \ifnum#3=\catcode\count@
    \else
      \errmessage{%
        Character \the\count@\space
        with wrong catcode \the\catcode\count@\space
        instead of \number#3%
      }%
    \fi
  \ifnum\count@<#2\relax
    \advance\count@ 1 %
  \repeat
}
\def\space{ }
\expandafter\ifx\csname LoadCommand\endcsname\relax
  \def\LoadCommand{\input epstopdf.sty\relax}%
\fi
\def\Test{%
  \RangeCatcodeInvalid{0}{47}%
  \RangeCatcodeInvalid{58}{64}%
  \RangeCatcodeInvalid{91}{96}%
  \RangeCatcodeInvalid{123}{255}%
  \catcode`\@=12 %
  \catcode`\\=0 %
  \catcode`\%=14 %
  \LoadCommand
  \RangeCatcodeCheck{0}{36}{15}%
  \RangeCatcodeCheck{37}{37}{14}%
  \RangeCatcodeCheck{38}{47}{15}%
  \RangeCatcodeCheck{48}{57}{12}%
  \RangeCatcodeCheck{58}{63}{15}%
  \RangeCatcodeCheck{64}{64}{12}%
  \RangeCatcodeCheck{65}{90}{11}%
  \RangeCatcodeCheck{91}{91}{15}%
  \RangeCatcodeCheck{92}{92}{0}%
  \RangeCatcodeCheck{93}{96}{15}%
  \RangeCatcodeCheck{97}{122}{11}%
  \RangeCatcodeCheck{123}{255}{15}%
  \RestoreCatcodes
}
\Test
\csname @@end\endcsname
\end
%    \end{macrocode}
%    \begin{macrocode}
%</test1>
%    \end{macrocode}
%
% \section{Installation}
%
% \subsection{Download}
%
% \paragraph{Package.} This package is available on
% CTAN\footnote{\url{http://ctan.org/pkg/epstopdf}}:
% \begin{description}
% \item[\CTAN{macros/latex/contrib/oberdiek/epstopdf.dtx}] The source file.
% \item[\CTAN{macros/latex/contrib/oberdiek/epstopdf.pdf}] Documentation.
% \end{description}
%
%
% \paragraph{Bundle.} All the packages of the bundle `oberdiek'
% are also available in a TDS compliant ZIP archive. There
% the packages are already unpacked and the documentation files
% are generated. The files and directories obey the TDS standard.
% \begin{description}
% \item[\CTAN{install/macros/latex/contrib/oberdiek.tds.zip}]
% \end{description}
% \emph{TDS} refers to the standard ``A Directory Structure
% for \TeX\ Files'' (\CTAN{tds/tds.pdf}). Directories
% with \xfile{texmf} in their name are usually organized this way.
%
% \subsection{Bundle installation}
%
% \paragraph{Unpacking.} Unpack the \xfile{oberdiek.tds.zip} in the
% TDS tree (also known as \xfile{texmf} tree) of your choice.
% Example (linux):
% \begin{quote}
%   |unzip oberdiek.tds.zip -d ~/texmf|
% \end{quote}
%
% \paragraph{Script installation.}
% Check the directory \xfile{TDS:scripts/oberdiek/} for
% scripts that need further installation steps.
% Package \xpackage{attachfile2} comes with the Perl script
% \xfile{pdfatfi.pl} that should be installed in such a way
% that it can be called as \texttt{pdfatfi}.
% Example (linux):
% \begin{quote}
%   |chmod +x scripts/oberdiek/pdfatfi.pl|\\
%   |cp scripts/oberdiek/pdfatfi.pl /usr/local/bin/|
% \end{quote}
%
% \subsection{Package installation}
%
% \paragraph{Unpacking.} The \xfile{.dtx} file is a self-extracting
% \docstrip\ archive. The files are extracted by running the
% \xfile{.dtx} through \plainTeX:
% \begin{quote}
%   \verb|tex epstopdf.dtx|
% \end{quote}
%
% \paragraph{TDS.} Now the different files must be moved into
% the different directories in your installation TDS tree
% (also known as \xfile{texmf} tree):
% \begin{quote}
% \def\t{^^A
% \begin{tabular}{@{}>{\ttfamily}l@{ $\rightarrow$ }>{\ttfamily}l@{}}
%   epstopdf.sty & tex/latex/oberdiek/epstopdf.sty\\
%   epstopdf-base.sty & tex/latex/oberdiek/epstopdf-base.sty\\
%   epstopdf.pdf & doc/latex/oberdiek/epstopdf.pdf\\
%   test/epstopdf-test1.tex & doc/latex/oberdiek/test/epstopdf-test1.tex\\
%   epstopdf.dtx & source/latex/oberdiek/epstopdf.dtx\\
% \end{tabular}^^A
% }^^A
% \sbox0{\t}^^A
% \ifdim\wd0>\linewidth
%   \begingroup
%     \advance\linewidth by\leftmargin
%     \advance\linewidth by\rightmargin
%   \edef\x{\endgroup
%     \def\noexpand\lw{\the\linewidth}^^A
%   }\x
%   \def\lwbox{^^A
%     \leavevmode
%     \hbox to \linewidth{^^A
%       \kern-\leftmargin\relax
%       \hss
%       \usebox0
%       \hss
%       \kern-\rightmargin\relax
%     }^^A
%   }^^A
%   \ifdim\wd0>\lw
%     \sbox0{\small\t}^^A
%     \ifdim\wd0>\linewidth
%       \ifdim\wd0>\lw
%         \sbox0{\footnotesize\t}^^A
%         \ifdim\wd0>\linewidth
%           \ifdim\wd0>\lw
%             \sbox0{\scriptsize\t}^^A
%             \ifdim\wd0>\linewidth
%               \ifdim\wd0>\lw
%                 \sbox0{\tiny\t}^^A
%                 \ifdim\wd0>\linewidth
%                   \lwbox
%                 \else
%                   \usebox0
%                 \fi
%               \else
%                 \lwbox
%               \fi
%             \else
%               \usebox0
%             \fi
%           \else
%             \lwbox
%           \fi
%         \else
%           \usebox0
%         \fi
%       \else
%         \lwbox
%       \fi
%     \else
%       \usebox0
%     \fi
%   \else
%     \lwbox
%   \fi
% \else
%   \usebox0
% \fi
% \end{quote}
% If you have a \xfile{docstrip.cfg} that configures and enables \docstrip's
% TDS installing feature, then some files can already be in the right
% place, see the documentation of \docstrip.
%
% \subsection{Refresh file name databases}
%
% If your \TeX~distribution
% (\teTeX, \mikTeX, \dots) relies on file name databases, you must refresh
% these. For example, \teTeX\ users run \verb|texhash| or
% \verb|mktexlsr|.
%
% \subsection{Some details for the interested}
%
% \paragraph{Attached source.}
%
% The PDF documentation on CTAN also includes the
% \xfile{.dtx} source file. It can be extracted by
% AcrobatReader 6 or higher. Another option is \textsf{pdftk},
% e.g. unpack the file into the current directory:
% \begin{quote}
%   \verb|pdftk epstopdf.pdf unpack_files output .|
% \end{quote}
%
% \paragraph{Unpacking with \LaTeX.}
% The \xfile{.dtx} chooses its action depending on the format:
% \begin{description}
% \item[\plainTeX:] Run \docstrip\ and extract the files.
% \item[\LaTeX:] Generate the documentation.
% \end{description}
% If you insist on using \LaTeX\ for \docstrip\ (really,
% \docstrip\ does not need \LaTeX), then inform the autodetect routine
% about your intention:
% \begin{quote}
%   \verb|latex \let\install=y% \iffalse meta-comment
%
% File: epstopdf.dtx
% Version: 2016/05/15 v2.6
% Info: Conversion with epstopdf on the fly
%
% Copyright (C) 2001, 2006-2010 by
%    Heiko Oberdiek <heiko.oberdiek at googlemail.com>
%
% This work may be distributed and/or modified under the
% conditions of the LaTeX Project Public License, either
% version 1.3c of this license or (at your option) any later
% version. This version of this license is in
%    http://www.latex-project.org/lppl/lppl-1-3c.txt
% and the latest version of this license is in
%    http://www.latex-project.org/lppl.txt
% and version 1.3 or later is part of all distributions of
% LaTeX version 2005/12/01 or later.
%
% This work has the LPPL maintenance status "maintained".
%
% This Current Maintainer of this work is Heiko Oberdiek.
%
% This work consists of the main source file epstopdf.dtx
% and the derived files
%    epstopdf.sty, epstopdf.pdf, epstopdf.ins, epstopdf.drv,
%    epstopdf-base.sty, epstopdf-test1.tex.
%
% Distribution:
%    CTAN:macros/latex/contrib/oberdiek/epstopdf.dtx
%    CTAN:macros/latex/contrib/oberdiek/epstopdf.pdf
%
% Unpacking:
%    (a) If epstopdf.ins is present:
%           tex epstopdf.ins
%    (b) Without epstopdf.ins:
%           tex epstopdf.dtx
%    (c) If you insist on using LaTeX
%           latex \let\install=y\input{epstopdf.dtx}
%        (quote the arguments according to the demands of your shell)
%
% Documentation:
%    (a) If epstopdf.drv is present:
%           latex epstopdf.drv
%    (b) Without epstopdf.drv:
%           latex epstopdf.dtx; ...
%    The class ltxdoc loads the configuration file ltxdoc.cfg
%    if available. Here you can specify further options, e.g.
%    use A4 as paper format:
%       \PassOptionsToClass{a4paper}{article}
%
%    Programm calls to get the documentation (example):
%       pdflatex epstopdf.dtx
%       makeindex -s gind.ist epstopdf.idx
%       pdflatex epstopdf.dtx
%       makeindex -s gind.ist epstopdf.idx
%       pdflatex epstopdf.dtx
%
% Installation:
%    TDS:tex/latex/oberdiek/epstopdf.sty
%    TDS:tex/latex/oberdiek/epstopdf-base.sty
%    TDS:doc/latex/oberdiek/epstopdf.pdf
%    TDS:doc/latex/oberdiek/test/epstopdf-test1.tex
%    TDS:source/latex/oberdiek/epstopdf.dtx
%
%<*ignore>
\begingroup
  \catcode123=1 %
  \catcode125=2 %
  \def\x{LaTeX2e}%
\expandafter\endgroup
\ifcase 0\ifx\install y1\fi\expandafter
         \ifx\csname processbatchFile\endcsname\relax\else1\fi
         \ifx\fmtname\x\else 1\fi\relax
\else\csname fi\endcsname
%</ignore>
%<*install>
\input docstrip.tex
\Msg{************************************************************************}
\Msg{* Installation}
\Msg{* Package: epstopdf 2016/05/15 v2.6 Conversion with epstopdf on the fly (HO)}
\Msg{************************************************************************}

\keepsilent
\askforoverwritefalse

\let\MetaPrefix\relax
\preamble

This is a generated file.

Project: epstopdf
Version: 2016/05/15 v2.6

Copyright (C) 2001, 2006-2010 by
   Heiko Oberdiek <heiko.oberdiek at googlemail.com>

This work may be distributed and/or modified under the
conditions of the LaTeX Project Public License, either
version 1.3c of this license or (at your option) any later
version. This version of this license is in
   http://www.latex-project.org/lppl/lppl-1-3c.txt
and the latest version of this license is in
   http://www.latex-project.org/lppl.txt
and version 1.3 or later is part of all distributions of
LaTeX version 2005/12/01 or later.

This work has the LPPL maintenance status "maintained".

This Current Maintainer of this work is Heiko Oberdiek.

This work consists of the main source file epstopdf.dtx
and the derived files
   epstopdf.sty, epstopdf.pdf, epstopdf.ins, epstopdf.drv,
   epstopdf-base.sty, epstopdf-test1.tex.

\endpreamble
\let\MetaPrefix\DoubleperCent

\generate{%
  \file{epstopdf.ins}{\from{epstopdf.dtx}{install}}%
  \file{epstopdf.drv}{\from{epstopdf.dtx}{driver}}%
  \usedir{tex/latex/oberdiek}%
  \file{epstopdf.sty}{\from{epstopdf.dtx}{package}}%
  \file{epstopdf-base.sty}{\from{epstopdf.dtx}{base}}%
  \usedir{doc/latex/oberdiek/test}%
  \file{epstopdf-test1.tex}{\from{epstopdf.dtx}{test1}}%
  \nopreamble
  \nopostamble
  \usedir{source/latex/oberdiek/catalogue}%
  \file{epstopdf-pkg.xml}{\from{epstopdf.dtx}{catalogue}}%
}

\catcode32=13\relax% active space
\let =\space%
\Msg{************************************************************************}
\Msg{*}
\Msg{* To finish the installation you have to move the following}
\Msg{* files into a directory searched by TeX:}
\Msg{*}
\Msg{*     epstopdf.sty, epstopdf-base.sty}
\Msg{*}
\Msg{* To produce the documentation run the file `epstopdf.drv'}
\Msg{* through LaTeX.}
\Msg{*}
\Msg{* Happy TeXing!}
\Msg{*}
\Msg{************************************************************************}

\endbatchfile
%</install>
%<*ignore>
\fi
%</ignore>
%<*driver>
\NeedsTeXFormat{LaTeX2e}
\ProvidesFile{epstopdf.drv}%
  [2016/05/15 v2.6 Conversion with epstopdf on the fly (HO)]%
\documentclass{ltxdoc}
\usepackage{holtxdoc}[2011/11/22]
\begin{document}
  \DocInput{epstopdf.dtx}%
\end{document}
%</driver>
% \fi
%
%
% \CharacterTable
%  {Upper-case    \A\B\C\D\E\F\G\H\I\J\K\L\M\N\O\P\Q\R\S\T\U\V\W\X\Y\Z
%   Lower-case    \a\b\c\d\e\f\g\h\i\j\k\l\m\n\o\p\q\r\s\t\u\v\w\x\y\z
%   Digits        \0\1\2\3\4\5\6\7\8\9
%   Exclamation   \!     Double quote  \"     Hash (number) \#
%   Dollar        \$     Percent       \%     Ampersand     \&
%   Acute accent  \'     Left paren    \(     Right paren   \)
%   Asterisk      \*     Plus          \+     Comma         \,
%   Minus         \-     Point         \.     Solidus       \/
%   Colon         \:     Semicolon     \;     Less than     \<
%   Equals        \=     Greater than  \>     Question mark \?
%   Commercial at \@     Left bracket  \[     Backslash     \\
%   Right bracket \]     Circumflex    \^     Underscore    \_
%   Grave accent  \`     Left brace    \{     Vertical bar  \|
%   Right brace   \}     Tilde         \~}
%
% \GetFileInfo{epstopdf.drv}
%
% \title{The \xpackage{epstopdf} package}
% \date{2016/05/15 v2.6}
% \author{Heiko Oberdiek\thanks
% {Please report any issues at https://github.com/ho-tex/oberdiek/issues}\\
% \xemail{heiko.oberdiek at googlemail.com}}
%
% \maketitle
%
% \begin{abstract}
% This packages adds support of handling eps images
% to package \xpackage{graphics} or \xpackage{graphicx}
% with option \xoption{pdftex}. If an eps image is detected,
% epstopdf is automatically called to convert it to pdf format.
% \end{abstract}
%
% \tableofcontents
%
% \providecommand*{\xprogram}[1]{\textsf{#1}}
%
% \section{Documentation}
%
% \subsection{Introduction}
%
% \LaTeX\ provides its graphics bundle to include graphics files.
% Both packages \xpackage{graphics} or \xpackage{graphicx} may
% be used. the latter one loads the first and adds options
% in key value style for \cs{includegraphics}.
%
% Usually the drivers do not support all kind of graphics files.
% Other image types must be converted, before they become
% usuable. In case of driver \xprogram{dvips}, the graphics
% rule may contain a conversion rule. Then all that package
% \xpackage{graphics} must know is the bounding box, the command
% is passed to \xprogram{dvips} that calls it and embeds the
% converted image.
%
% However, \pdfTeX\ has its driver for PDF output already
% build in. It's graphics inclusion commands (\cs{pdfximage})
% does not allow the execution of external commands.
% Therefore commands in the last argument of
% \cs{DeclareGraphicsRule} were of no use.
% But external programs can be called within \pdfTeX.
% This feature is called ``shell escape'' or ``write 18''
% and must usually enabled explicitely because of security reasons.
% Now, this package \xpackage{epstopdf} hooks
% into package \xpackage{graphics}' code to catch
% that argument with the external command and executes it
% to convert the graphics file to a supported format and
% passes the control of graphics inclusion back to package
% \xpackage{graphics}.
%
% \subsection{Requirements}
%
% \begin{itemize}
% \item
%   The feature \cs{write18} must be enabled. This allows the
%   running of external programs during \TeX's compile run.
%   Keep in mind that this is a security risk. The feature is an
%   addition to \cs{TeX}. Mik\TeX, te\TeX, \TeX\ Live support it.
%   In Web2C based \TeX\ distributions (te\TeX, \TeX\ Live) it can be
%   enabled in the configuration file \xfile{texmf.cnf}:
%   \begin{quote}
%     |shell_escape = 1|
%   \end{quote}
%   Because of the security risk, it is better to do it on the
%   command line only:
%   \begin{quote}
%   \texttt{--shell-escape} (te\TeX, \TeX\ Live)\\
%   \texttt{--enable-write18} (MiK\TeX)
%   \end{quote}
%   Example:
%   \begin{quote}
%     |pdflatex -shell-escape test.tex|
%   \end{quote}
% \item
%   The program \xprogram{epstopdf} for the conversion from EPS
%   to PDF. However, other programs can
%   be used and configured by \cs{DeclareGraphicsRule}.
%   Example:
%   \begin{quote}
%   |\epstopdfDeclareGraphicsRule{.eps}{pdf}{.pdf}{%|\\
%   |  ps2pdf -dEPSCrop #1 \OutputFile|\\
%   |}|
%   \end{quote}
% \end{itemize}
%
% \subsection{Usage}
%
% The package is loaded after graphic\{s,x\}, e.g.:
% \begin{quote}
%   |\usepackage[pdftex]{graphicx}|\\
%   |\usepackage{epstopdf}|
% \end{quote}
% Now images with file name extension \xfile{.eps} are detected
% and supported using \cs{includegraphics}.
%
% If the graphics file name is explicitly specified with extension
% \xfile{.eps} the new rule for EPS files is called and
% the conversion performed. If option \xoption{update} is in force
% then the conversion step is dropped if the target file already
% exists and is not older then the EPS file.
%
% The situation is more complicate if the graphics file is
% given without file name extension. Then the \xpackage{graphics}
% package must search for a supported image file.
% The possible extensions are stored in the graphics extension
% list, that can be set by \cs{DeclareGraphicsExtensions}.
% The algorithm:
% \begin{quote}
%    \def\>{\mbox{}\quad}
%    function search( \meta{filebase} )\\
%    \>foreach \meta{ext} in \meta{graphics extensions}\\
%    \>\>foreach \meta{dir} in \meta{current directory}, \meta{\cs{graphicspath}}\\
%    \>\>\>\meta{file} := \meta{dir} + \meta{filebase} + \meta{ext}\\
%    \>\>\>if exist \meta{file}\\
%    \>\>\>\>return found\\
%    \>return not found
% \end{quote}
% Package \xpackage{epstopdf} puts \xfile{.eps} at the end of the
% graphics extension search list. This is the behaviour of option
% \xoption{append} that is enabled by default.
% That means, the conversion is called last
% unless a supported file type cannot be found earlier. This avoids
% unnecessary conversion steps that slow down the \LaTeX\ run.
% If you want to use option \xoption{update} and your \pdfTeX\ supports
% it, then an outdated PDF file also would be found earlier unless
% \xoption{suffix} is used that is the default since version 2.0.
%
% With an empty option \xoption{suffix} and option
% \xoption{prepend} there is a risk that an original PDF file is overwritten:
%
% If the original image file is the PDF file and there is also a generated
% EPS file, then the original PDF file can be regenerated (depending
% on the option settings) and the original PDF file gets lost.
% Therefore option \xoption{suffix} is introduced in version 1.9
% to create a separate name space for generated output files.
%
% \textbf{Note:} Usually the conversion program needs the exact location
% of the image file. Usually the current directory works.
% Also if the image file is found using \cs{graphicspath}, the
% location is known. However, if the image is somewhere
% in a directory of environment variable |TEXINPUTS|, then
% the package does not know the exact location and the
% conversion program will not find the image file unless it
% implements a search using |TEXINPUTS| (program \xprogram{kpsewhich}
% may be of help in this task).
%
% \subsection{Options}
%
% Options can be given as package options or later using:
% \begin{declcs}{epstopdfsetup} \M{key value list}
% \end{declcs}
% \LaTeX\ expands the option list before passing the option list
% to the package's option handling code. This can fail
% for option \xoption{suffix} if it contains some of the macros
% described below. Use \cs{epstopdfsetup} after the package is loaded.
% Or load package \xpackage{kvoptions-patch} before. This package
% is also loaded by option \xoption{patch} of package \xpackage{kvoptions}.
% \LaTeX's option code is redefined to respect key value options and
% let the values untouched.
%
% \begin{description}
% \def\It#1{\item[\xoption{#1}:]}^^A
% \It{update}
%   The conversion program is only called, if the target file
%   does not exist or is older than the source image file.
% \It{append}
%   Puts the extension \xfile{.eps} at the end of the graphics extension list
%   (default).
% \It{prepend}
%   Puts the extension \xfile{.eps} at the begin of the graphics extension list.
% \It{outdir}
%   The converted file may put in an other output directory.
%   The value of \xoption{outdir} must include the directory separator.
%   Example for the current directory:
%   \begin{quote}
%   |\epstopdfsetup{outdir=./}|
%   \end{quote}
%   For other directories ensure, that they can be found.
%   See \cs{graphicspath} or |TEXINPUTS|.
% \It{suffix}
%   This option takes a string that is put between the file name base
%   and the extension of the output file. Rationale: It can happen,
%   that a PDF file is the original file and the EPS file the
%   generated file. If now the package thinks, that the PDF file
%   is the generated file, it will `regnerate' it. But in reality
%   the original file is lost. Therefore I recommend to use this
%   option always to generate a separate name space for generated
%   files. Proposed value is \texttt{-generated} or \texttt{.generated}.
%   The suffix \texttt{.generated} will also work here without the need for
%   package \xpackage{grffile}).
%
%   Example:
%   \begin{quote}
%   |\epstopdfsetup{suffix=-generated}|\\
%   Then \xfile{foo.eps} is converted to \xfile{foo-generated.pdf}.
%   \end{quote}
%   \cs{SourceExt} can be used inside the suffix string.
%   It's will be replaced by the extension of the image source
%   file without the leading dot, for instance:
%   \begin{quote}
%   |\epstopdfsetup{suffix=-\SourceExt-converted-to}|\\
%   \xfile{foo.eps} $\Rightarrow$ \xfile{foo-eps-converted-to.pdf}
%   \end{quote}
%   See also the next option \xoption{prefersuffix} that
%   modifies the behaviour of option \xoption{suffix} in some
%   cases.
%
%   Default for \xoption{suffix} is `|-\SourceExt-converted-to|'.
%
% \It{prefersuffix}
%   If a suffix is set by option \xoption{suffix}, then there can be
%   two image file names that could be taken into account for inclusion:
%   A image file name with the suffix string inside
%   its name and a image file name without; e.g. for \xfile{foo.eps}
%   the names could be:
%   \begin{quote}
%   \xfile{foo-suffix.pdf}, \xfile{foo.pdf}
%   \end{quote}
%   If option \xoption{perfersuffix} is turned on, the
%   file \xfile{foo-suffix.pdf} and its generation is preferred
%   over using \xfile{foo.pdf}. Otherwise \xfile{foo.pdf} is included
%   without generating \xfile{foo-suffix.pdf}.
%   The default of option \xoption{prefersuffix} is \texttt{true}.
% \It{program@epstopdf}
%   The name for the conversion program from EPS to PDF,
%   default is `|epstopdf|'.
% \It{verbose}
%   It prints some information about the image in the \xfile{.log} file
%   (default).
% \end{description}
%
% \subsection{Configuration}
%
% \subsubsection{System configuration file \xfile{epstopdf-sys.cfg}}
%
% If \xfile{epstopdf-sys} exists it is loaded at the end of the
% package \xpackage{epstopdf-base} and before the user configuration
% file. It's intended for TeX distributors. Thus they could add
% additional conversion rules (e.g., .gif -> .png) or set options.
%
% \subsubsection{User configuration file \xfile{epstopdf.cfg}}
%
% A configuration file \xfile{epstopdf.cfg} is loaded at the end
% of the package if it exists. It can be used for changing
% the default option setting. Example:
% \begin{quote}
% |\epstopdfsetup{verbose=false}|
% \end{quote}
%
% \subsubsection{Conversion program}
%
% You can use \cs{DeclareGraphicsRule} in a similar way as the
% route via \xprogram{dvips} to specify the conversion
% command line.
% The conversion argument starts with a back tick, followed
% by the conversion command including parameters.
%
% The whole conversion argument should also be wrapped inside
% \cs{epstopdfcall}. This reduces the problem with packages
% (e.g. \xpackage{pst-pdf}) that use the conversion argument
% and expands it. Macros \cs{SourceFile}, \cs{OutputFile}, and
% \cs{SourceExt} are not defined outside \xpackage{epstopdf-base}'s
% \cs{Gin@setfile} and error messages because of undefined
% command names are the result. If \cs{epstopdfcall} detects that
% it is called outside \xpackage{epstopdf-base}'s \cs{Gin@setfile}
% then it replaces the conversion argument by
% package \xpackage{graphics}'s
% default, usually the image file.
%
% The following macros are available inside:
% \begin{description}
% \def\It#1{\item[\cs{#1}:]}
% \It{OutputFile}: output file name (with known path and extension)
% \It{SourceFile}: source file name (with known path and extension),
%                  usually the same as |#1|,
% \It{SourceExt}: source extension without leading dot.
% \end{description}
%
% \paragraph{Conversion from EPS to PDF.}
% Other programs than \xprogram{epstopdf} can be used to convert from
% EPS to PDF. Example that uses \xprogram{Ghostscript}:
% \begin{quote}
%   |\DeclareGraphicsRule{.eps}{pdf}{.pdf}{%|\\
%   |  \epstopdfcall{`ps2pdf -dEPSCrop #1 \noexpand\OutputFile}%|\\
%   |}|
% \end{quote}
% \cs{DeclareGraphicsRule} expands the argument, therefore \cs{noexpand}
% is necessary. As convenience package \xpackage{epstopdf-base} defines
% \cs{epstopdfDeclareGraphicsRule}. Then the conversion argument
% is not expanded, \cs{epstopdfcall} and the back tick are added:
% \begin{quote}
%   |\epstopdfDeclareGraphicsRule{.eps}{pdf}{.pdf}{%|\\
%   |  ps2pdf -dEPSCrop #1 \OutputFile|\\
%   |}|
% \end{quote}
%
% Also \cs{OutputFile} respects the setting of option \xoption{outdir}.
%
% \subsection{Other image formats}
%
% The support that package \xpackage{epstopdf} implements is not
% limited to EPS files. Other image conversions can be declared.
% The following example shows it for GIF images under Unix with
% ImageMagick's \xprogram{convert}:
%\begin{quote}
%\begin{verbatim}
%\epstopdfDeclareGraphicsRule{.gif}{png}{.png}{%
%  convert #1 \OutputFile
%}
%\end{verbatim}
%\end{quote}
% The file extension \xfile{.gif} can be added to the extension list that
% package \xpackage{graphics} searches if the file extension
% is not given in \cs{includegraphics}.
% The list can be set by \cs{GraphicsExtensions}.
% \begin{quote}
% |\AppendGraphicsExtensions{.gif}|\\
% or\\
% |\PrependGraphicsExtensions{.gif}|
% \end{quote}
%
% \StopEventually{
% }
%
% \section{Implementation}
%
%    \begin{macrocode}
%<*package>
%    \end{macrocode}
%
% \subsection{Wrapper package}
%
%    Reload check, especially if the package is not used with \LaTeX.
%    \begin{macrocode}
\begingroup\catcode61\catcode48\catcode32=10\relax%
  \catcode13=5 % ^^M
  \endlinechar=13 %
  \catcode35=6 % #
  \catcode39=12 % '
  \catcode44=12 % ,
  \catcode45=12 % -
  \catcode46=12 % .
  \catcode58=12 % :
  \catcode64=11 % @
  \catcode123=1 % {
  \catcode125=2 % }
  \expandafter\let\expandafter\x\csname ver@epstopdf.sty\endcsname
  \ifx\x\relax % plain-TeX, first loading
  \else
    \def\empty{}%
    \ifx\x\empty % LaTeX, first loading,
      % variable is initialized, but \ProvidesPackage not yet seen
    \else
      \expandafter\ifx\csname PackageInfo\endcsname\relax
        \def\x#1#2{%
          \immediate\write-1{Package #1 Info: #2.}%
        }%
      \else
        \def\x#1#2{\PackageInfo{#1}{#2, stopped}}%
      \fi
      \x{epstopdf}{The package is already loaded}%
      \aftergroup\endinput
    \fi
  \fi
\endgroup%
%    \end{macrocode}
%    Package identification:
%    \begin{macrocode}
\begingroup\catcode61\catcode48\catcode32=10\relax%
  \catcode13=5 % ^^M
  \endlinechar=13 %
  \catcode35=6 % #
  \catcode39=12 % '
  \catcode40=12 % (
  \catcode41=12 % )
  \catcode44=12 % ,
  \catcode45=12 % -
  \catcode46=12 % .
  \catcode47=12 % /
  \catcode58=12 % :
  \catcode64=11 % @
  \catcode91=12 % [
  \catcode93=12 % ]
  \catcode123=1 % {
  \catcode125=2 % }
  \expandafter\ifx\csname ProvidesPackage\endcsname\relax
    \def\x#1#2#3[#4]{\endgroup
      \immediate\write-1{Package: #3 #4}%
      \xdef#1{#4}%
    }%
  \else
    \def\x#1#2[#3]{\endgroup
      #2[{#3}]%
      \ifx#1\@undefined
        \xdef#1{#3}%
      \fi
      \ifx#1\relax
        \xdef#1{#3}%
      \fi
    }%
  \fi
\expandafter\x\csname ver@epstopdf.sty\endcsname
\ProvidesPackage{epstopdf}%
  [2016/05/15 v2.6 Conversion with epstopdf on the fly (HO)]%
%    \end{macrocode}
%    Larger catcode set because of configuration files needed.
%    \begin{macrocode}
\begingroup\catcode61\catcode48\catcode32=10\relax%
  \catcode13=5 % ^^M
  \endlinechar=13 %
  \catcode123=1 % {
  \catcode125=2 % }
  \catcode64=11 % @
  \def\x{\endgroup
    \expandafter\edef\csname ETE@AtEnd\endcsname{%
      \endlinechar=\the\endlinechar\relax
      \catcode13=\the\catcode13\relax
      \catcode32=\the\catcode32\relax
      \catcode35=\the\catcode35\relax
      \catcode61=\the\catcode61\relax
      \catcode64=\the\catcode64\relax
      \catcode123=\the\catcode123\relax
      \catcode125=\the\catcode125\relax
    }%
  }%
\x\catcode61\catcode48\catcode32=10\relax%
\catcode13=5 % ^^M
\endlinechar=13 %
\catcode35=6 % #
\catcode64=11 % @
\catcode123=1 % {
\catcode125=2 % }
\def\TMP@EnsureCode#1#2{%
  \edef\ETE@AtEnd{%
    \ETE@AtEnd
    \catcode#1=\the\catcode#1\relax
  }%
  \catcode#1=#2\relax
}
\TMP@EnsureCode{39}{12}% '
\TMP@EnsureCode{40}{12}% (
\TMP@EnsureCode{41}{12}% )
\TMP@EnsureCode{42}{12}% *
\TMP@EnsureCode{45}{12}% -
\TMP@EnsureCode{47}{12}% /
\TMP@EnsureCode{91}{12}% [
\TMP@EnsureCode{93}{12}% ]
\edef\ETE@AtEnd{\ETE@AtEnd\noexpand\endinput}
%    \end{macrocode}
%
%    \begin{macrocode}
\let\ETE@SavedAtEnd\ETE@AtEnd
\RequirePackage{epstopdf-base}[2016/05/15]
\let\ETE@AtEnd\ETE@SavedAtEnd
%    \end{macrocode}
%
% \subsubsection{Option handling}
%
%    \begin{macrocode}
\DeclareOption*{%
  \expandafter\epstopdfsetup\expandafter{\CurrentOption}%
}
\ProcessOptions*\relax
%    \end{macrocode}
%
%    \begin{macrocode}
\ETE@AtEnd%
%</package>
%    \end{macrocode}
%
% \subsection{Base package}
%
%    \begin{macrocode}
%<*base>
%    \end{macrocode}
%
% \subsection{Preparations}
%
% \subsubsection{Relead check and identification}
%
%    Reload check, especially if the package is not used with \LaTeX.
%    \begin{macrocode}
\begingroup\catcode61\catcode48\catcode32=10\relax%
  \catcode13=5 % ^^M
  \endlinechar=13 %
  \catcode35=6 % #
  \catcode39=12 % '
  \catcode44=12 % ,
  \catcode45=12 % -
  \catcode46=12 % .
  \catcode58=12 % :
  \catcode64=11 % @
  \catcode123=1 % {
  \catcode125=2 % }
  \expandafter\let\expandafter\x\csname ver@epstopdf-base.sty\endcsname
  \ifx\x\relax % plain-TeX, first loading
  \else
    \def\empty{}%
    \ifx\x\empty % LaTeX, first loading,
      % variable is initialized, but \ProvidesPackage not yet seen
    \else
      \expandafter\ifx\csname PackageInfo\endcsname\relax
        \def\x#1#2{%
          \immediate\write-1{Package #1 Info: #2.}%
        }%
      \else
        \def\x#1#2{\PackageInfo{#1}{#2, stopped}}%
      \fi
      \x{epstopdf-base}{The package is already loaded}%
      \aftergroup\endinput
    \fi
  \fi
\endgroup%
%    \end{macrocode}
%    Package identification:
%    \begin{macrocode}
\begingroup\catcode61\catcode48\catcode32=10\relax%
  \catcode13=5 % ^^M
  \endlinechar=13 %
  \catcode35=6 % #
  \catcode39=12 % '
  \catcode40=12 % (
  \catcode41=12 % )
  \catcode44=12 % ,
  \catcode45=12 % -
  \catcode46=12 % .
  \catcode47=12 % /
  \catcode58=12 % :
  \catcode64=11 % @
  \catcode91=12 % [
  \catcode93=12 % ]
  \catcode123=1 % {
  \catcode125=2 % }
  \expandafter\ifx\csname ProvidesPackage\endcsname\relax
    \def\x#1#2#3[#4]{\endgroup
      \immediate\write-1{Package: #3 #4}%
      \xdef#1{#4}%
    }%
  \else
    \def\x#1#2[#3]{\endgroup
      #2[{#3}]%
      \ifx#1\@undefined
        \xdef#1{#3}%
      \fi
      \ifx#1\relax
        \xdef#1{#3}%
      \fi
    }%
  \fi
\expandafter\x\csname ver@epstopdf-base.sty\endcsname
\ProvidesPackage{epstopdf-base}%
  [2016/05/15 v2.6 Base part for package epstopdf]%
%    \end{macrocode}
%
% \subsubsection{Catcodes}
%
%    \begin{macrocode}
\begingroup\catcode61\catcode48\catcode32=10\relax%
  \catcode13=5 % ^^M
  \endlinechar=13 %
  \catcode123=1 % {
  \catcode125=2 % }
  \catcode64=11 % @
  \def\x{\endgroup
    \expandafter\edef\csname ETE@AtEnd\endcsname{%
      \endlinechar=\the\endlinechar\relax
      \catcode13=\the\catcode13\relax
      \catcode32=\the\catcode32\relax
      \catcode35=\the\catcode35\relax
      \catcode61=\the\catcode61\relax
      \catcode64=\the\catcode64\relax
      \catcode123=\the\catcode123\relax
      \catcode125=\the\catcode125\relax
    }%
  }%
\x\catcode61\catcode48\catcode32=10\relax%
\catcode13=5 % ^^M
\endlinechar=13 %
\catcode35=6 % #
\catcode64=11 % @
\catcode123=1 % {
\catcode125=2 % }
\def\TMP@EnsureCode#1#2{%
  \edef\ETE@AtEnd{%
    \ETE@AtEnd
    \catcode#1=\the\catcode#1\relax
  }%
  \catcode#1=#2\relax
}
\TMP@EnsureCode{33}{12}% !
\TMP@EnsureCode{39}{12}% '
\TMP@EnsureCode{42}{12}% *
\TMP@EnsureCode{44}{12}% ,
\TMP@EnsureCode{45}{12}% -
\TMP@EnsureCode{46}{12}% .
\TMP@EnsureCode{47}{12}% /
\TMP@EnsureCode{58}{12}% :
\TMP@EnsureCode{60}{12}% <
\TMP@EnsureCode{62}{12}% >
\TMP@EnsureCode{96}{12}% `
\edef\ETE@AtEnd{\ETE@AtEnd\noexpand\endinput}
%    \end{macrocode}
%
% \subsubsection{Load packages}
%
%    \begin{macrocode}
\RequirePackage{infwarerr}[2007/09/09]
\RequirePackage{grfext}\relax
\RequirePackage{kvoptions}[2007/10/02]
\RequirePackage{pdftexcmds}[2007/11/11]
%    \end{macrocode}
%
% \subsection{Checks}
%
% Check, whether package graphics is loaded
% (also graphicx loads graphics).
% Because miniltx.tex does not know \cs{@ifpackageloaded}
% we test for \cs{Gin@setfile} instead.
%    \begin{macrocode}
\begingroup\expandafter\expandafter\expandafter\endgroup
\expandafter\ifx\csname Gin@setfile\endcsname\relax
  \@PackageWarningNoLine{epstopdf}{%
    No graphics package \string`graphic{s,x}\string' loaded%
  }%
  \newcommand*{\epstopdfsetup}[1]{}%
  \expandafter\ETE@AtEnd
\fi%
%    \end{macrocode}
%
% Check, whether pdftex.def is loaded. \cs{ver@pdftex.def} is not
% available with miniltx.tex, thus we test for \cs{Gin@driver}.
%    \begin{macrocode}
\begingroup
  \def\x{luatex.def}%
  \ifx\Gin@driver\x
  \else
  \def\x{pdftex.def}%
  \ifx\Gin@driver\x
  \else
    \@PackageWarningNoLine{epstopdf}{%
      Drivers other than `pdftex' and `luatex' are not supported%
    }%
    \endgroup
    \newcommand*{\epstopdfsetup}[1]{}%
    \expandafter\ETE@AtEnd
  \fi%
  \fi%
\endgroup
%    \end{macrocode}
%
% Check, whether the shell escape feature is enabled.
%
%    \begin{macrocode}
\begingroup
  \expandafter\ifx\csname pdf@shellescape\endcsname\relax
  \else
    \ifnum\pdf@shellescape>0 %
    \else
      \@PackageWarningNoLine{epstopdf}{%
        Shell escape feature is not enabled%
      }%
    \fi
  \fi
\endgroup
%    \end{macrocode}
%
% \subsection{Package loading}
%
%    \begin{macrocode}
%    \end{macrocode}
%
% \subsection{Options}
%
%    \begin{macrocode}
\SetupKeyvalOptions{family=ETE,prefix=ETE@}
\DeclareBoolOption{update}
\DeclareBoolOption{verbose}
\newif\ifETE@prepend
\DeclareVoidOption{prepend}{\ETE@prependtrue}
\DeclareVoidOption{append}{\ETE@prependfalse}
\DeclareStringOption{outdir}
\DeclareStringOption{suffix}
\DeclareBoolOption{prefersuffix}
\DeclareStringOption{program@epstopdf}
%    \end{macrocode}
%    Options \xoption{disable} and \xoption{enable} are for testing
%    only. Therefore they are not documented on purpose.
%    \begin{macrocode}
\DeclareBoolOption{disable}
\DeclareComplementaryOption{enable}{disable}
%    \end{macrocode}
%    \begin{macrocode}
\newcommand*{\epstopdfsetup}{\setkeys{ETE}}
%    \end{macrocode}
%
% \subsubsection{Default setting}
%
%    \begin{macrocode}
\epstopdfsetup{%
  verbose,%
  enable,%
  append,%
  update,%
  prefersuffix,%
  suffix=-\SourceExt-converted-to,%
  program@epstopdf=epstopdf%
}
%    \end{macrocode}
%
% \subsection{Make and verbose}
%
%    \begin{macrocode}
\begingroup\expandafter\expandafter\expandafter\endgroup
\expandafter\ifx\csname pdf@filemoddate\endcsname\relax
  \def\ETE@Make#1#2{%
    \ifETE@update
      \ETE@WarnModDate
    \fi
    \@firstofone
  }%
  \def\ETE@WarnModDate{%
    \@PackageWarningNoLine{epstopdf}{%
      \string\pdffilemoddate\space is not available,\MessageBreak
      option `update' will be ignored%
    }%
    \global\let\ETE@WarnModDate\relax
  }%
  \def\ETE@FileInfo#1#2{#1 file: <#2>}%
\else
  \def\ETE@Make#1#2{%
    \ifETE@update
      \ifnum\pdf@strcmp{\pdf@filemoddate{#1}}{\pdf@filemoddate{#2}}>0 %
        \expandafter\expandafter\expandafter\@firstofone
      \else
        \@PackageInfoNoLine{epstopdf}{%
          Output file is already uptodate%
        }%
        \expandafter\expandafter\expandafter\@gobble
      \fi
    \else
      \expandafter\@firstofone
    \fi
  }%
  \def\ETE@FileInfo#1#2{%
    #1 file: <#2>%
    \expandafter\expandafter\expandafter
    \ETE@Date\pdf@filemoddate{#2}\@nil
    \expandafter\expandafter\expandafter
    \ETE@Size\pdf@filesize{#2}\@nil
  }%
  \def\ETE@Date#1\@nil{%
    \ifx\\#1\\%
    \else
      \ETE@@Date#1\@nil
    \fi
  }%
  \def\ETE@@Date#1:#2#3#4#5#6#7#8#9{%
    \MessageBreak
    \@spaces\space\space\space date: #2#3#4#5-#6#7-#8#9 %
    \ETE@@Time
  }%
  \def\ETE@@Time#1#2#3#4#5#6#7\@nil{%
    #1#2:#3#4:#5#6%
  }%
  \def\ETE@Size#1\@nil{%
    \ifx\\#1\\%
    \else
      \MessageBreak
      \@spaces\space\space\space size: #1 bytes%
    \fi
  }%
\fi
%    \end{macrocode}
%
% \subsection{Adding conversion support}
%
%    Patch \cs{Gin@setfile} to execute |#3|, if it contains
%    a command.
%    \begin{macrocode}
\expandafter\ifx\csname ETE@OrgGin@setfile\endcsname\relax
  \let\ETE@OrgGin@setfile\Gin@setfile
\else
  \@PackageError{epstopdf}{%
    Command \string\ETE@OrgGin@setfile\space
    already defined.\MessageBreak
  }{%
    Probably some package has included the code of this package%
    \MessageBreak
    instead of using \string\RequirePackage{epstopdf}.%
    \MessageBreak
    \@ehc
  }%
\fi
\def\ETE@IfFileExists{%
  \begingroup\expandafter\expandafter\expandafter\endgroup
  \expandafter\ifx\csname grffile@IfFileExists\endcsname\relax
    \expandafter\IfFileExists
  \else
    \global\let\ETE@IfFileExists\grffile@IfFileExists
    \expandafter\grffile@IfFileExists
  \fi
}
\def\ETE@Skip#1#2\x\fi\fi{%
  \fi
  \fi
  \endgroup
  \fi
  \fi
  #1%
}
\newif\ifETE@InsideSetfile
\newcommand*{\epstopdfcall}[1]{%
  \ifETE@InsideSetfile
    \expandafter\@firstoftwo
  \else
    \expandafter\@secondoftwo
  \fi
  {`#1}%
  {\Gin@base\Gin@ext}%
}
\def\ETE@DefCommandLine#1{%
  \edef\CommandLine{\expandafter\fi\if`#1}%
}
\def\ETE@DefX#1{%
  \expandafter\expandafter\expandafter\def
  \expandafter\expandafter\expandafter\x
  \expandafter\expandafter\expandafter{%
    \expandafter\fi\if`#1\relax\else
  }%
}
\def\ETE@Gin@setfile#1#2#3{%
  \ifETE@disable
    \ETE@OrgGin@setfile{#1}{#2}{#3}%
  \else
    \begingroup
      \ETE@InsideSetfiletrue
      \ETE@DefX{#3}%
    \expandafter\endgroup
    \ifx\x\@empty
      \ETE@OrgGin@setfile{#1}{#2}{#3}%
    \else
      \begingroup
        \ETE@InsideSetfiletrue
        \def\GraphicsType{#1}%
        \def\GraphicsRead{#2}%
        \ifETE@prefersuffix
        \else
          \ifx\ETE@suffix\@empty
          \else
            \ETE@IfFileExists{\Gin@base\GraphicsRead}{%
              \ETE@Skip{%
                \ETE@OrgGin@setfile{#1}{#2}{\Gin@base#2}%
              }%
            }{%
              \let\next\relax
            }%
            \next
          \fi
        \fi
        \ifx\Gin@ext\relax
          \let\SourceExt\Gin@eext
          \def\SourceFile{\Gin@base\Gin@eext}%
        \else
          \let\SourceExt\Gin@ext
          \def\SourceFile{\Gin@base\Gin@ext}%
        \fi
        \edef\SourceExt{% remove dot
          \expandafter\@cdr\SourceExt\@empty\@nil
        }%
        \let\OutputDirectory\ETE@outdir
        \ifx\OutputDirectory\@empty
          \edef\OutputFile{\ETE@GenerateName{\Gin@base}{#2}}%
        \else
          \begingroup
            \filename@parse{\Gin@base#2}%
            \edef\x{\endgroup
              \def\noexpand\OutputFile{%
                \ETE@GenerateName{%
                  \OutputDirectory\filename@base
                }{#2}%
              }%
            }%
          \x
        \fi
        \ETE@DefCommandLine{#3}%
        \ifETE@verbose
          \@PackageInfo{epstopdf}{%
            \ETE@FileInfo{Source}\SourceFile\MessageBreak
            \ETE@FileInfo{Output}\OutputFile\MessageBreak
            Command: <\CommandLine>\MessageBreak
            \string\includegraphics
          }%
        \fi
        \ETE@Make\SourceFile\OutputFile{%
          \pdf@system{\CommandLine}%
          \ifETE@verbose
            \@PackageInfoNoLine{epstopdf}{%
              \ETE@FileInfo{Result}\OutputFile
            }%
          \fi
        }%
        \edef\x{\endgroup
          \ifx\OutputDirectory\@empty
          \else
            \def\noexpand\Gin@base{%
              \OutputDirectory\noexpand\filename@base
            }%
          \fi
          \ifx\ETE@suffix\@empty
          \else
            \edef\noexpand\Gin@base{%
              \noexpand\Gin@base\ETE@suffix
            }%
          \fi
          \noexpand\ETE@OrgGin@setfile{%
            \GraphicsType
          }{%
            \GraphicsRead
          }{%
            \OutputFile
          }%
        }%
      \x
    \fi
  \fi
}
\let\Gin@setfile\ETE@Gin@setfile
\def\ETE@GenerateName#1#2{%
  #1\ETE@suffix#2%
}
%    \end{macrocode}
%
% \subsection{Declare graphics rule}
%
%    \begin{macrocode}
\newcommand*{\epstopdfDeclareGraphicsRule}[4]{%
  \ifx\\#4\\%
    \@PackageError{epstopdf-base}{%
      Conversion command is missing%
    }\@ehc
  \else
    \begingroup
      \@ifundefined{Gin@rule@#1}{%
      }{%
        \@PackageInfo{epstopdf-base}{%
          Redefining graphics rule for `#1'%
        }%
      }%
    \endgroup
    \@namedef{Gin@rule@#1}##1{{#2}{#3}{\epstopdfcall{#4}}}%
  \fi
}
%    \end{macrocode}
%
% \cs{DeclareGraphicsRule} for .eps
%    \begin{macrocode}
\epstopdfDeclareGraphicsRule{.eps}{pdf}{.pdf}{%
  \ETE@epstopdf{#1}%
}
\def\ETE@epstopdf#1{%
  \ETE@program@epstopdf\space
  \ifcase\ifx\OutputDirectory\@empty
           \ifx\ETE@suffix\@empty
             1%
           \fi
         \fi
         0 %
    --outfile=\OutputFile\space
  \fi
  #1%
}
%    \end{macrocode}
%
%    \begin{macrocode}
\ifETE@prepend
  \expandafter\PrependGraphicsExtensions
\else
  \expandafter\AppendGraphicsExtensions
\fi
{.eps}
\let\ETE@prepend\@undefined
\DeclareVoidOption{prepend}{%
  \PrependGraphicsExtensions{.eps}%
}
\let\ETE@append\@undefined
\DeclareVoidOption{append}{%
  \AppendGraphicsExtensions{.eps}%
}
%    \end{macrocode}
%
%    \begin{macrocode}
\InputIfFileExists{epstopdf-sys.cfg}{}{}
\InputIfFileExists{epstopdf.cfg}{}{}
%    \end{macrocode}
%
%    \begin{macrocode}
\ETE@AtEnd%
%</base>
%    \end{macrocode}
%
% \section{Test}
%
% \subsection{Preface for standard catcode check}
%
%    \begin{macrocode}
%<*test1>
\input miniltx.tex\relax
\def\Gin@driver{pdftex.def}
\input graphicx.sty\relax
\resetatcatcode
%</test1>
%    \end{macrocode}
%
% \subsection{Catcode checks for loading}
%
%    \begin{macrocode}
%<*test1>
%    \end{macrocode}
%    \begin{macrocode}
\catcode`\{=1 %
\catcode`\}=2 %
\catcode`\#=6 %
\catcode`\@=11 %
\expandafter\ifx\csname count@\endcsname\relax
  \countdef\count@=255 %
\fi
\expandafter\ifx\csname @gobble\endcsname\relax
  \long\def\@gobble#1{}%
\fi
\expandafter\ifx\csname @firstofone\endcsname\relax
  \long\def\@firstofone#1{#1}%
\fi
\expandafter\ifx\csname loop\endcsname\relax
  \expandafter\@firstofone
\else
  \expandafter\@gobble
\fi
{%
  \def\loop#1\repeat{%
    \def\body{#1}%
    \iterate
  }%
  \def\iterate{%
    \body
      \let\next\iterate
    \else
      \let\next\relax
    \fi
    \next
  }%
  \let\repeat=\fi
}%
\def\RestoreCatcodes{}
\count@=0 %
\loop
  \edef\RestoreCatcodes{%
    \RestoreCatcodes
    \catcode\the\count@=\the\catcode\count@\relax
  }%
\ifnum\count@<255 %
  \advance\count@ 1 %
\repeat

\def\RangeCatcodeInvalid#1#2{%
  \count@=#1\relax
  \loop
    \catcode\count@=15 %
  \ifnum\count@<#2\relax
    \advance\count@ 1 %
  \repeat
}
\def\RangeCatcodeCheck#1#2#3{%
  \count@=#1\relax
  \loop
    \ifnum#3=\catcode\count@
    \else
      \errmessage{%
        Character \the\count@\space
        with wrong catcode \the\catcode\count@\space
        instead of \number#3%
      }%
    \fi
  \ifnum\count@<#2\relax
    \advance\count@ 1 %
  \repeat
}
\def\space{ }
\expandafter\ifx\csname LoadCommand\endcsname\relax
  \def\LoadCommand{\input epstopdf.sty\relax}%
\fi
\def\Test{%
  \RangeCatcodeInvalid{0}{47}%
  \RangeCatcodeInvalid{58}{64}%
  \RangeCatcodeInvalid{91}{96}%
  \RangeCatcodeInvalid{123}{255}%
  \catcode`\@=12 %
  \catcode`\\=0 %
  \catcode`\%=14 %
  \LoadCommand
  \RangeCatcodeCheck{0}{36}{15}%
  \RangeCatcodeCheck{37}{37}{14}%
  \RangeCatcodeCheck{38}{47}{15}%
  \RangeCatcodeCheck{48}{57}{12}%
  \RangeCatcodeCheck{58}{63}{15}%
  \RangeCatcodeCheck{64}{64}{12}%
  \RangeCatcodeCheck{65}{90}{11}%
  \RangeCatcodeCheck{91}{91}{15}%
  \RangeCatcodeCheck{92}{92}{0}%
  \RangeCatcodeCheck{93}{96}{15}%
  \RangeCatcodeCheck{97}{122}{11}%
  \RangeCatcodeCheck{123}{255}{15}%
  \RestoreCatcodes
}
\Test
\csname @@end\endcsname
\end
%    \end{macrocode}
%    \begin{macrocode}
%</test1>
%    \end{macrocode}
%
% \section{Installation}
%
% \subsection{Download}
%
% \paragraph{Package.} This package is available on
% CTAN\footnote{\url{http://ctan.org/pkg/epstopdf}}:
% \begin{description}
% \item[\CTAN{macros/latex/contrib/oberdiek/epstopdf.dtx}] The source file.
% \item[\CTAN{macros/latex/contrib/oberdiek/epstopdf.pdf}] Documentation.
% \end{description}
%
%
% \paragraph{Bundle.} All the packages of the bundle `oberdiek'
% are also available in a TDS compliant ZIP archive. There
% the packages are already unpacked and the documentation files
% are generated. The files and directories obey the TDS standard.
% \begin{description}
% \item[\CTAN{install/macros/latex/contrib/oberdiek.tds.zip}]
% \end{description}
% \emph{TDS} refers to the standard ``A Directory Structure
% for \TeX\ Files'' (\CTAN{tds/tds.pdf}). Directories
% with \xfile{texmf} in their name are usually organized this way.
%
% \subsection{Bundle installation}
%
% \paragraph{Unpacking.} Unpack the \xfile{oberdiek.tds.zip} in the
% TDS tree (also known as \xfile{texmf} tree) of your choice.
% Example (linux):
% \begin{quote}
%   |unzip oberdiek.tds.zip -d ~/texmf|
% \end{quote}
%
% \paragraph{Script installation.}
% Check the directory \xfile{TDS:scripts/oberdiek/} for
% scripts that need further installation steps.
% Package \xpackage{attachfile2} comes with the Perl script
% \xfile{pdfatfi.pl} that should be installed in such a way
% that it can be called as \texttt{pdfatfi}.
% Example (linux):
% \begin{quote}
%   |chmod +x scripts/oberdiek/pdfatfi.pl|\\
%   |cp scripts/oberdiek/pdfatfi.pl /usr/local/bin/|
% \end{quote}
%
% \subsection{Package installation}
%
% \paragraph{Unpacking.} The \xfile{.dtx} file is a self-extracting
% \docstrip\ archive. The files are extracted by running the
% \xfile{.dtx} through \plainTeX:
% \begin{quote}
%   \verb|tex epstopdf.dtx|
% \end{quote}
%
% \paragraph{TDS.} Now the different files must be moved into
% the different directories in your installation TDS tree
% (also known as \xfile{texmf} tree):
% \begin{quote}
% \def\t{^^A
% \begin{tabular}{@{}>{\ttfamily}l@{ $\rightarrow$ }>{\ttfamily}l@{}}
%   epstopdf.sty & tex/latex/oberdiek/epstopdf.sty\\
%   epstopdf-base.sty & tex/latex/oberdiek/epstopdf-base.sty\\
%   epstopdf.pdf & doc/latex/oberdiek/epstopdf.pdf\\
%   test/epstopdf-test1.tex & doc/latex/oberdiek/test/epstopdf-test1.tex\\
%   epstopdf.dtx & source/latex/oberdiek/epstopdf.dtx\\
% \end{tabular}^^A
% }^^A
% \sbox0{\t}^^A
% \ifdim\wd0>\linewidth
%   \begingroup
%     \advance\linewidth by\leftmargin
%     \advance\linewidth by\rightmargin
%   \edef\x{\endgroup
%     \def\noexpand\lw{\the\linewidth}^^A
%   }\x
%   \def\lwbox{^^A
%     \leavevmode
%     \hbox to \linewidth{^^A
%       \kern-\leftmargin\relax
%       \hss
%       \usebox0
%       \hss
%       \kern-\rightmargin\relax
%     }^^A
%   }^^A
%   \ifdim\wd0>\lw
%     \sbox0{\small\t}^^A
%     \ifdim\wd0>\linewidth
%       \ifdim\wd0>\lw
%         \sbox0{\footnotesize\t}^^A
%         \ifdim\wd0>\linewidth
%           \ifdim\wd0>\lw
%             \sbox0{\scriptsize\t}^^A
%             \ifdim\wd0>\linewidth
%               \ifdim\wd0>\lw
%                 \sbox0{\tiny\t}^^A
%                 \ifdim\wd0>\linewidth
%                   \lwbox
%                 \else
%                   \usebox0
%                 \fi
%               \else
%                 \lwbox
%               \fi
%             \else
%               \usebox0
%             \fi
%           \else
%             \lwbox
%           \fi
%         \else
%           \usebox0
%         \fi
%       \else
%         \lwbox
%       \fi
%     \else
%       \usebox0
%     \fi
%   \else
%     \lwbox
%   \fi
% \else
%   \usebox0
% \fi
% \end{quote}
% If you have a \xfile{docstrip.cfg} that configures and enables \docstrip's
% TDS installing feature, then some files can already be in the right
% place, see the documentation of \docstrip.
%
% \subsection{Refresh file name databases}
%
% If your \TeX~distribution
% (\teTeX, \mikTeX, \dots) relies on file name databases, you must refresh
% these. For example, \teTeX\ users run \verb|texhash| or
% \verb|mktexlsr|.
%
% \subsection{Some details for the interested}
%
% \paragraph{Attached source.}
%
% The PDF documentation on CTAN also includes the
% \xfile{.dtx} source file. It can be extracted by
% AcrobatReader 6 or higher. Another option is \textsf{pdftk},
% e.g. unpack the file into the current directory:
% \begin{quote}
%   \verb|pdftk epstopdf.pdf unpack_files output .|
% \end{quote}
%
% \paragraph{Unpacking with \LaTeX.}
% The \xfile{.dtx} chooses its action depending on the format:
% \begin{description}
% \item[\plainTeX:] Run \docstrip\ and extract the files.
% \item[\LaTeX:] Generate the documentation.
% \end{description}
% If you insist on using \LaTeX\ for \docstrip\ (really,
% \docstrip\ does not need \LaTeX), then inform the autodetect routine
% about your intention:
% \begin{quote}
%   \verb|latex \let\install=y\input{epstopdf.dtx}|
% \end{quote}
% Do not forget to quote the argument according to the demands
% of your shell.
%
% \paragraph{Generating the documentation.}
% You can use both the \xfile{.dtx} or the \xfile{.drv} to generate
% the documentation. The process can be configured by the
% configuration file \xfile{ltxdoc.cfg}. For instance, put this
% line into this file, if you want to have A4 as paper format:
% \begin{quote}
%   \verb|\PassOptionsToClass{a4paper}{article}|
% \end{quote}
% An example follows how to generate the
% documentation with pdf\LaTeX:
% \begin{quote}
%\begin{verbatim}
%pdflatex epstopdf.dtx
%makeindex -s gind.ist epstopdf.idx
%pdflatex epstopdf.dtx
%makeindex -s gind.ist epstopdf.idx
%pdflatex epstopdf.dtx
%\end{verbatim}
% \end{quote}
%
% \section{Catalogue}
%
% The following XML file can be used as source for the
% \href{http://mirror.ctan.org/help/Catalogue/catalogue.html}{\TeX\ Catalogue}.
% The elements \texttt{caption} and \texttt{description} are imported
% from the original XML file from the Catalogue.
% The name of the XML file in the Catalogue is \xfile{epstopdf-pkg.xml}.
%    \begin{macrocode}
%<*catalogue>
<?xml version='1.0' encoding='us-ascii'?>
<!DOCTYPE entry SYSTEM 'catalogue.dtd'>
<entry datestamp='$Date$' modifier='$Author$' id='epstopdf-pkg'>
  <name>epstopdf-pkg</name>
  <caption>Call epstopdf "on the fly"</caption>
  <authorref id='auth:oberdiek'/>
  <copyright owner='Heiko Oberdiek' year='2001,2006-2010'/>
  <license type='lppl1.3'/>
  <version number='2.6'/>
  <description>
    The package adds support for EPS files in the
    <xref refid='graphicx'>graphicx</xref> package when running under
    PDFTeX.  If an EPS graphic is detected, the package spawns a
    process to convert the EPS to PDF, using the script
    <xref refid='epstopdf'>epstopdf</xref>.  This of course requires
    that shell escape is enabled for the PDFTeX run.
    <p/>
    The package is part of the <xref refid='oberdiek'>oberdiek</xref>
    bundle.
  </description>
  <documentation details='Package documentation'
      href='ctan:/macros/latex/contrib/oberdiek/epstopdf.pdf'/>
  <ctan file='true' path='/macros/latex/contrib/oberdiek/epstopdf.dtx'/>
  <miktex location='oberdiek'/>
  <texlive location='oberdiek'/>
  <install path='/macros/latex/contrib/oberdiek/oberdiek.tds.zip'/>
</entry>
%</catalogue>
%    \end{macrocode}
%
% \begin{History}
%   \begin{Version}{2001/01/06 v1.0}
%   \item
%     First public version,
%     published in the \pdfTeX\ mailing list.
%   \end{Version}
%   \begin{Version}{2001/02/04 v1.1}
%   \item
%     Minor documentation update.
%   \item
%     CTAN.
%   \end{Version}
%   \begin{Version}{2006/02/20 v1.2}
%   \item
%     DTX framework.
%   \item
%     Compatibility for \xfile{miniltx.tex}.
%   \end{Version}
%   \begin{Version}{2006/08/26 v1.3}
%   \item
%     Check for \cs{write18} if available and print a warning
%     if the feature is not enabled.
%   \end{Version}
%   \begin{Version}{2007/04/26 v1.4}
%   \item
%     Documentation rewritten and extended.
%   \end{Version}
%   \begin{Version}{2007/10/02 v1.5}
%   \item
%     New option \xoption{update}: If the converted file exists, it will be only
%     converted if it is out of date.
%   \item
%     Updating the extension list is delegated to package \xpackage{grfext}.
%     Fine tuning is done by the new options \xoption{append}, \xoption{prepend}.
%   \item
%     New option \xoption{outdir} for changing the output directory.
%   \item
%     New option \xoption{verbose}.
%   \item
%     \cs{SourceFile} and \cs{OutputFile} introduced.
%   \item
%     Configuration file support added.
%   \end{Version}
%   \begin{Version}{2007/11/11 v1.6}
%   \item
%     Use of package \xpackage{pdftexcmds} for \LuaTeX\ support.
%   \end{Version}
%   \begin{Version}{2008/05/06 v1.7}
%   \item
%     Warning messages uses ``loaded'' instead of ``found''.
%   \end{Version}
%   \begin{Version}{2009/03/01 v1.8}
%   \item
%     Warning message for missing \xfile{pdftex.def} changed.
%   \end{Version}
%   \begin{Version}{2009/07/06 v1.9}
%   \item
%     Option \xoption{suffix} added.
%   \end{Version}
%   \begin{Version}{2009/07/07 v1.10}
%   \item
%     \cs{SourceExt} added.
%   \item
%     If option \xoption{suffix} is set, the inclusion of
%     an image without the suffix namespace is preferred over generating the
%     the image within the suffix namespace.
%   \end{Version}
%   \begin{Version}{2009/07/12 v2.0}
%   \item
%     New default settings.
%   \item
%     Package is split into \xpackage{epstopdf} that only takes package
%     options and loads \xpackage{epstopdf-base} that does the work.
%   \item
%     \cs{epstopdfDeclareGraphicsRule} and \cs{epstopdfcall} added.
%   \item
%     \xfile{epstopdf-sys.cfg} is loaded before \xfile{epstopdf.cfg}
%     if \xfile{epstopdf-sys.cfg} exists.
%   \end{Version}
%   \begin{Version}{2009/07/15 v2.1}
%   \item
%     Default setting: \xoption{verbose} is now turned on as the
%     documentation for v2.0 said.
%   \item
%     Documentation fixes.
%   \end{Version}
%   \begin{Version}{2009/07/16 v2.2}
%   \item
%     Fixed redefined \cs{Gin@setfile}.
%   \item
%     Documentation extended for package options.
%   \end{Version}
%   \begin{Version}{2009/09/24 v2.3}
%   \item
%     Bug fix for the case that both option \xoption{suffix} and
%     \xoption{outdir} are used.
%   \end{Version}
%   \begin{Version}{2009/10/17 v2.4}
%   \item
%     The name of the program `epstopdf' can be configured via
%     the new option \xoption{program@epstopdf}.
%   \end{Version}
%   \begin{Version}{2016/05/15 v2.5}
%   \item
%     Wording of warning message fixed (Karl Berry).
%   \item
%     \cs{ETE@Gin@setfile} added (Karl Berry).
%   \end{Version}
%   \begin{Version}{2016/05/15 v2.6}
%   \item
%     luaTeX compatibility
%   \end{Version}
% \end{History}
%
% \PrintIndex
%
% \Finale
\endinput
|
% \end{quote}
% Do not forget to quote the argument according to the demands
% of your shell.
%
% \paragraph{Generating the documentation.}
% You can use both the \xfile{.dtx} or the \xfile{.drv} to generate
% the documentation. The process can be configured by the
% configuration file \xfile{ltxdoc.cfg}. For instance, put this
% line into this file, if you want to have A4 as paper format:
% \begin{quote}
%   \verb|\PassOptionsToClass{a4paper}{article}|
% \end{quote}
% An example follows how to generate the
% documentation with pdf\LaTeX:
% \begin{quote}
%\begin{verbatim}
%pdflatex epstopdf.dtx
%makeindex -s gind.ist epstopdf.idx
%pdflatex epstopdf.dtx
%makeindex -s gind.ist epstopdf.idx
%pdflatex epstopdf.dtx
%\end{verbatim}
% \end{quote}
%
% \section{Catalogue}
%
% The following XML file can be used as source for the
% \href{http://mirror.ctan.org/help/Catalogue/catalogue.html}{\TeX\ Catalogue}.
% The elements \texttt{caption} and \texttt{description} are imported
% from the original XML file from the Catalogue.
% The name of the XML file in the Catalogue is \xfile{epstopdf-pkg.xml}.
%    \begin{macrocode}
%<*catalogue>
<?xml version='1.0' encoding='us-ascii'?>
<!DOCTYPE entry SYSTEM 'catalogue.dtd'>
<entry datestamp='$Date$' modifier='$Author$' id='epstopdf-pkg'>
  <name>epstopdf-pkg</name>
  <caption>Call epstopdf "on the fly"</caption>
  <authorref id='auth:oberdiek'/>
  <copyright owner='Heiko Oberdiek' year='2001,2006-2010'/>
  <license type='lppl1.3'/>
  <version number='2.6'/>
  <description>
    The package adds support for EPS files in the
    <xref refid='graphicx'>graphicx</xref> package when running under
    PDFTeX.  If an EPS graphic is detected, the package spawns a
    process to convert the EPS to PDF, using the script
    <xref refid='epstopdf'>epstopdf</xref>.  This of course requires
    that shell escape is enabled for the PDFTeX run.
    <p/>
    The package is part of the <xref refid='oberdiek'>oberdiek</xref>
    bundle.
  </description>
  <documentation details='Package documentation'
      href='ctan:/macros/latex/contrib/oberdiek/epstopdf.pdf'/>
  <ctan file='true' path='/macros/latex/contrib/oberdiek/epstopdf.dtx'/>
  <miktex location='oberdiek'/>
  <texlive location='oberdiek'/>
  <install path='/macros/latex/contrib/oberdiek/oberdiek.tds.zip'/>
</entry>
%</catalogue>
%    \end{macrocode}
%
% \begin{History}
%   \begin{Version}{2001/01/06 v1.0}
%   \item
%     First public version,
%     published in the \pdfTeX\ mailing list.
%   \end{Version}
%   \begin{Version}{2001/02/04 v1.1}
%   \item
%     Minor documentation update.
%   \item
%     CTAN.
%   \end{Version}
%   \begin{Version}{2006/02/20 v1.2}
%   \item
%     DTX framework.
%   \item
%     Compatibility for \xfile{miniltx.tex}.
%   \end{Version}
%   \begin{Version}{2006/08/26 v1.3}
%   \item
%     Check for \cs{write18} if available and print a warning
%     if the feature is not enabled.
%   \end{Version}
%   \begin{Version}{2007/04/26 v1.4}
%   \item
%     Documentation rewritten and extended.
%   \end{Version}
%   \begin{Version}{2007/10/02 v1.5}
%   \item
%     New option \xoption{update}: If the converted file exists, it will be only
%     converted if it is out of date.
%   \item
%     Updating the extension list is delegated to package \xpackage{grfext}.
%     Fine tuning is done by the new options \xoption{append}, \xoption{prepend}.
%   \item
%     New option \xoption{outdir} for changing the output directory.
%   \item
%     New option \xoption{verbose}.
%   \item
%     \cs{SourceFile} and \cs{OutputFile} introduced.
%   \item
%     Configuration file support added.
%   \end{Version}
%   \begin{Version}{2007/11/11 v1.6}
%   \item
%     Use of package \xpackage{pdftexcmds} for \LuaTeX\ support.
%   \end{Version}
%   \begin{Version}{2008/05/06 v1.7}
%   \item
%     Warning messages uses ``loaded'' instead of ``found''.
%   \end{Version}
%   \begin{Version}{2009/03/01 v1.8}
%   \item
%     Warning message for missing \xfile{pdftex.def} changed.
%   \end{Version}
%   \begin{Version}{2009/07/06 v1.9}
%   \item
%     Option \xoption{suffix} added.
%   \end{Version}
%   \begin{Version}{2009/07/07 v1.10}
%   \item
%     \cs{SourceExt} added.
%   \item
%     If option \xoption{suffix} is set, the inclusion of
%     an image without the suffix namespace is preferred over generating the
%     the image within the suffix namespace.
%   \end{Version}
%   \begin{Version}{2009/07/12 v2.0}
%   \item
%     New default settings.
%   \item
%     Package is split into \xpackage{epstopdf} that only takes package
%     options and loads \xpackage{epstopdf-base} that does the work.
%   \item
%     \cs{epstopdfDeclareGraphicsRule} and \cs{epstopdfcall} added.
%   \item
%     \xfile{epstopdf-sys.cfg} is loaded before \xfile{epstopdf.cfg}
%     if \xfile{epstopdf-sys.cfg} exists.
%   \end{Version}
%   \begin{Version}{2009/07/15 v2.1}
%   \item
%     Default setting: \xoption{verbose} is now turned on as the
%     documentation for v2.0 said.
%   \item
%     Documentation fixes.
%   \end{Version}
%   \begin{Version}{2009/07/16 v2.2}
%   \item
%     Fixed redefined \cs{Gin@setfile}.
%   \item
%     Documentation extended for package options.
%   \end{Version}
%   \begin{Version}{2009/09/24 v2.3}
%   \item
%     Bug fix for the case that both option \xoption{suffix} and
%     \xoption{outdir} are used.
%   \end{Version}
%   \begin{Version}{2009/10/17 v2.4}
%   \item
%     The name of the program `epstopdf' can be configured via
%     the new option \xoption{program@epstopdf}.
%   \end{Version}
%   \begin{Version}{2016/05/15 v2.5}
%   \item
%     Wording of warning message fixed (Karl Berry).
%   \item
%     \cs{ETE@Gin@setfile} added (Karl Berry).
%   \end{Version}
%   \begin{Version}{2016/05/15 v2.6}
%   \item
%     luaTeX compatibility
%   \end{Version}
% \end{History}
%
% \PrintIndex
%
% \Finale
\endinput

%        (quote the arguments according to the demands of your shell)
%
% Documentation:
%    (a) If epstopdf.drv is present:
%           latex epstopdf.drv
%    (b) Without epstopdf.drv:
%           latex epstopdf.dtx; ...
%    The class ltxdoc loads the configuration file ltxdoc.cfg
%    if available. Here you can specify further options, e.g.
%    use A4 as paper format:
%       \PassOptionsToClass{a4paper}{article}
%
%    Programm calls to get the documentation (example):
%       pdflatex epstopdf.dtx
%       makeindex -s gind.ist epstopdf.idx
%       pdflatex epstopdf.dtx
%       makeindex -s gind.ist epstopdf.idx
%       pdflatex epstopdf.dtx
%
% Installation:
%    TDS:tex/latex/oberdiek/epstopdf.sty
%    TDS:tex/latex/oberdiek/epstopdf-base.sty
%    TDS:doc/latex/oberdiek/epstopdf.pdf
%    TDS:doc/latex/oberdiek/test/epstopdf-test1.tex
%    TDS:source/latex/oberdiek/epstopdf.dtx
%
%<*ignore>
\begingroup
  \catcode123=1 %
  \catcode125=2 %
  \def\x{LaTeX2e}%
\expandafter\endgroup
\ifcase 0\ifx\install y1\fi\expandafter
         \ifx\csname processbatchFile\endcsname\relax\else1\fi
         \ifx\fmtname\x\else 1\fi\relax
\else\csname fi\endcsname
%</ignore>
%<*install>
\input docstrip.tex
\Msg{************************************************************************}
\Msg{* Installation}
\Msg{* Package: epstopdf 2016/05/15 v2.6 Conversion with epstopdf on the fly (HO)}
\Msg{************************************************************************}

\keepsilent
\askforoverwritefalse

\let\MetaPrefix\relax
\preamble

This is a generated file.

Project: epstopdf
Version: 2016/05/15 v2.6

Copyright (C) 2001, 2006-2010 by
   Heiko Oberdiek <heiko.oberdiek at googlemail.com>

This work may be distributed and/or modified under the
conditions of the LaTeX Project Public License, either
version 1.3c of this license or (at your option) any later
version. This version of this license is in
   http://www.latex-project.org/lppl/lppl-1-3c.txt
and the latest version of this license is in
   http://www.latex-project.org/lppl.txt
and version 1.3 or later is part of all distributions of
LaTeX version 2005/12/01 or later.

This work has the LPPL maintenance status "maintained".

This Current Maintainer of this work is Heiko Oberdiek.

This work consists of the main source file epstopdf.dtx
and the derived files
   epstopdf.sty, epstopdf.pdf, epstopdf.ins, epstopdf.drv,
   epstopdf-base.sty, epstopdf-test1.tex.

\endpreamble
\let\MetaPrefix\DoubleperCent

\generate{%
  \file{epstopdf.ins}{\from{epstopdf.dtx}{install}}%
  \file{epstopdf.drv}{\from{epstopdf.dtx}{driver}}%
  \usedir{tex/latex/oberdiek}%
  \file{epstopdf.sty}{\from{epstopdf.dtx}{package}}%
  \file{epstopdf-base.sty}{\from{epstopdf.dtx}{base}}%
  \usedir{doc/latex/oberdiek/test}%
  \file{epstopdf-test1.tex}{\from{epstopdf.dtx}{test1}}%
  \nopreamble
  \nopostamble
  \usedir{source/latex/oberdiek/catalogue}%
  \file{epstopdf-pkg.xml}{\from{epstopdf.dtx}{catalogue}}%
}

\catcode32=13\relax% active space
\let =\space%
\Msg{************************************************************************}
\Msg{*}
\Msg{* To finish the installation you have to move the following}
\Msg{* files into a directory searched by TeX:}
\Msg{*}
\Msg{*     epstopdf.sty, epstopdf-base.sty}
\Msg{*}
\Msg{* To produce the documentation run the file `epstopdf.drv'}
\Msg{* through LaTeX.}
\Msg{*}
\Msg{* Happy TeXing!}
\Msg{*}
\Msg{************************************************************************}

\endbatchfile
%</install>
%<*ignore>
\fi
%</ignore>
%<*driver>
\NeedsTeXFormat{LaTeX2e}
\ProvidesFile{epstopdf.drv}%
  [2016/05/15 v2.6 Conversion with epstopdf on the fly (HO)]%
\documentclass{ltxdoc}
\usepackage{holtxdoc}[2011/11/22]
\begin{document}
  \DocInput{epstopdf.dtx}%
\end{document}
%</driver>
% \fi
%
%
% \CharacterTable
%  {Upper-case    \A\B\C\D\E\F\G\H\I\J\K\L\M\N\O\P\Q\R\S\T\U\V\W\X\Y\Z
%   Lower-case    \a\b\c\d\e\f\g\h\i\j\k\l\m\n\o\p\q\r\s\t\u\v\w\x\y\z
%   Digits        \0\1\2\3\4\5\6\7\8\9
%   Exclamation   \!     Double quote  \"     Hash (number) \#
%   Dollar        \$     Percent       \%     Ampersand     \&
%   Acute accent  \'     Left paren    \(     Right paren   \)
%   Asterisk      \*     Plus          \+     Comma         \,
%   Minus         \-     Point         \.     Solidus       \/
%   Colon         \:     Semicolon     \;     Less than     \<
%   Equals        \=     Greater than  \>     Question mark \?
%   Commercial at \@     Left bracket  \[     Backslash     \\
%   Right bracket \]     Circumflex    \^     Underscore    \_
%   Grave accent  \`     Left brace    \{     Vertical bar  \|
%   Right brace   \}     Tilde         \~}
%
% \GetFileInfo{epstopdf.drv}
%
% \title{The \xpackage{epstopdf} package}
% \date{2016/05/15 v2.6}
% \author{Heiko Oberdiek\thanks
% {Please report any issues at https://github.com/ho-tex/oberdiek/issues}\\
% \xemail{heiko.oberdiek at googlemail.com}}
%
% \maketitle
%
% \begin{abstract}
% This packages adds support of handling eps images
% to package \xpackage{graphics} or \xpackage{graphicx}
% with option \xoption{pdftex}. If an eps image is detected,
% epstopdf is automatically called to convert it to pdf format.
% \end{abstract}
%
% \tableofcontents
%
% \providecommand*{\xprogram}[1]{\textsf{#1}}
%
% \section{Documentation}
%
% \subsection{Introduction}
%
% \LaTeX\ provides its graphics bundle to include graphics files.
% Both packages \xpackage{graphics} or \xpackage{graphicx} may
% be used. the latter one loads the first and adds options
% in key value style for \cs{includegraphics}.
%
% Usually the drivers do not support all kind of graphics files.
% Other image types must be converted, before they become
% usuable. In case of driver \xprogram{dvips}, the graphics
% rule may contain a conversion rule. Then all that package
% \xpackage{graphics} must know is the bounding box, the command
% is passed to \xprogram{dvips} that calls it and embeds the
% converted image.
%
% However, \pdfTeX\ has its driver for PDF output already
% build in. It's graphics inclusion commands (\cs{pdfximage})
% does not allow the execution of external commands.
% Therefore commands in the last argument of
% \cs{DeclareGraphicsRule} were of no use.
% But external programs can be called within \pdfTeX.
% This feature is called ``shell escape'' or ``write 18''
% and must usually enabled explicitely because of security reasons.
% Now, this package \xpackage{epstopdf} hooks
% into package \xpackage{graphics}' code to catch
% that argument with the external command and executes it
% to convert the graphics file to a supported format and
% passes the control of graphics inclusion back to package
% \xpackage{graphics}.
%
% \subsection{Requirements}
%
% \begin{itemize}
% \item
%   The feature \cs{write18} must be enabled. This allows the
%   running of external programs during \TeX's compile run.
%   Keep in mind that this is a security risk. The feature is an
%   addition to \cs{TeX}. Mik\TeX, te\TeX, \TeX\ Live support it.
%   In Web2C based \TeX\ distributions (te\TeX, \TeX\ Live) it can be
%   enabled in the configuration file \xfile{texmf.cnf}:
%   \begin{quote}
%     |shell_escape = 1|
%   \end{quote}
%   Because of the security risk, it is better to do it on the
%   command line only:
%   \begin{quote}
%   \texttt{--shell-escape} (te\TeX, \TeX\ Live)\\
%   \texttt{--enable-write18} (MiK\TeX)
%   \end{quote}
%   Example:
%   \begin{quote}
%     |pdflatex -shell-escape test.tex|
%   \end{quote}
% \item
%   The program \xprogram{epstopdf} for the conversion from EPS
%   to PDF. However, other programs can
%   be used and configured by \cs{DeclareGraphicsRule}.
%   Example:
%   \begin{quote}
%   |\epstopdfDeclareGraphicsRule{.eps}{pdf}{.pdf}{%|\\
%   |  ps2pdf -dEPSCrop #1 \OutputFile|\\
%   |}|
%   \end{quote}
% \end{itemize}
%
% \subsection{Usage}
%
% The package is loaded after graphic\{s,x\}, e.g.:
% \begin{quote}
%   |\usepackage[pdftex]{graphicx}|\\
%   |\usepackage{epstopdf}|
% \end{quote}
% Now images with file name extension \xfile{.eps} are detected
% and supported using \cs{includegraphics}.
%
% If the graphics file name is explicitly specified with extension
% \xfile{.eps} the new rule for EPS files is called and
% the conversion performed. If option \xoption{update} is in force
% then the conversion step is dropped if the target file already
% exists and is not older then the EPS file.
%
% The situation is more complicate if the graphics file is
% given without file name extension. Then the \xpackage{graphics}
% package must search for a supported image file.
% The possible extensions are stored in the graphics extension
% list, that can be set by \cs{DeclareGraphicsExtensions}.
% The algorithm:
% \begin{quote}
%    \def\>{\mbox{}\quad}
%    function search( \meta{filebase} )\\
%    \>foreach \meta{ext} in \meta{graphics extensions}\\
%    \>\>foreach \meta{dir} in \meta{current directory}, \meta{\cs{graphicspath}}\\
%    \>\>\>\meta{file} := \meta{dir} + \meta{filebase} + \meta{ext}\\
%    \>\>\>if exist \meta{file}\\
%    \>\>\>\>return found\\
%    \>return not found
% \end{quote}
% Package \xpackage{epstopdf} puts \xfile{.eps} at the end of the
% graphics extension search list. This is the behaviour of option
% \xoption{append} that is enabled by default.
% That means, the conversion is called last
% unless a supported file type cannot be found earlier. This avoids
% unnecessary conversion steps that slow down the \LaTeX\ run.
% If you want to use option \xoption{update} and your \pdfTeX\ supports
% it, then an outdated PDF file also would be found earlier unless
% \xoption{suffix} is used that is the default since version 2.0.
%
% With an empty option \xoption{suffix} and option
% \xoption{prepend} there is a risk that an original PDF file is overwritten:
%
% If the original image file is the PDF file and there is also a generated
% EPS file, then the original PDF file can be regenerated (depending
% on the option settings) and the original PDF file gets lost.
% Therefore option \xoption{suffix} is introduced in version 1.9
% to create a separate name space for generated output files.
%
% \textbf{Note:} Usually the conversion program needs the exact location
% of the image file. Usually the current directory works.
% Also if the image file is found using \cs{graphicspath}, the
% location is known. However, if the image is somewhere
% in a directory of environment variable |TEXINPUTS|, then
% the package does not know the exact location and the
% conversion program will not find the image file unless it
% implements a search using |TEXINPUTS| (program \xprogram{kpsewhich}
% may be of help in this task).
%
% \subsection{Options}
%
% Options can be given as package options or later using:
% \begin{declcs}{epstopdfsetup} \M{key value list}
% \end{declcs}
% \LaTeX\ expands the option list before passing the option list
% to the package's option handling code. This can fail
% for option \xoption{suffix} if it contains some of the macros
% described below. Use \cs{epstopdfsetup} after the package is loaded.
% Or load package \xpackage{kvoptions-patch} before. This package
% is also loaded by option \xoption{patch} of package \xpackage{kvoptions}.
% \LaTeX's option code is redefined to respect key value options and
% let the values untouched.
%
% \begin{description}
% \def\It#1{\item[\xoption{#1}:]}^^A
% \It{update}
%   The conversion program is only called, if the target file
%   does not exist or is older than the source image file.
% \It{append}
%   Puts the extension \xfile{.eps} at the end of the graphics extension list
%   (default).
% \It{prepend}
%   Puts the extension \xfile{.eps} at the begin of the graphics extension list.
% \It{outdir}
%   The converted file may put in an other output directory.
%   The value of \xoption{outdir} must include the directory separator.
%   Example for the current directory:
%   \begin{quote}
%   |\epstopdfsetup{outdir=./}|
%   \end{quote}
%   For other directories ensure, that they can be found.
%   See \cs{graphicspath} or |TEXINPUTS|.
% \It{suffix}
%   This option takes a string that is put between the file name base
%   and the extension of the output file. Rationale: It can happen,
%   that a PDF file is the original file and the EPS file the
%   generated file. If now the package thinks, that the PDF file
%   is the generated file, it will `regnerate' it. But in reality
%   the original file is lost. Therefore I recommend to use this
%   option always to generate a separate name space for generated
%   files. Proposed value is \texttt{-generated} or \texttt{.generated}.
%   The suffix \texttt{.generated} will also work here without the need for
%   package \xpackage{grffile}).
%
%   Example:
%   \begin{quote}
%   |\epstopdfsetup{suffix=-generated}|\\
%   Then \xfile{foo.eps} is converted to \xfile{foo-generated.pdf}.
%   \end{quote}
%   \cs{SourceExt} can be used inside the suffix string.
%   It's will be replaced by the extension of the image source
%   file without the leading dot, for instance:
%   \begin{quote}
%   |\epstopdfsetup{suffix=-\SourceExt-converted-to}|\\
%   \xfile{foo.eps} $\Rightarrow$ \xfile{foo-eps-converted-to.pdf}
%   \end{quote}
%   See also the next option \xoption{prefersuffix} that
%   modifies the behaviour of option \xoption{suffix} in some
%   cases.
%
%   Default for \xoption{suffix} is `|-\SourceExt-converted-to|'.
%
% \It{prefersuffix}
%   If a suffix is set by option \xoption{suffix}, then there can be
%   two image file names that could be taken into account for inclusion:
%   A image file name with the suffix string inside
%   its name and a image file name without; e.g. for \xfile{foo.eps}
%   the names could be:
%   \begin{quote}
%   \xfile{foo-suffix.pdf}, \xfile{foo.pdf}
%   \end{quote}
%   If option \xoption{perfersuffix} is turned on, the
%   file \xfile{foo-suffix.pdf} and its generation is preferred
%   over using \xfile{foo.pdf}. Otherwise \xfile{foo.pdf} is included
%   without generating \xfile{foo-suffix.pdf}.
%   The default of option \xoption{prefersuffix} is \texttt{true}.
% \It{program@epstopdf}
%   The name for the conversion program from EPS to PDF,
%   default is `|epstopdf|'.
% \It{verbose}
%   It prints some information about the image in the \xfile{.log} file
%   (default).
% \end{description}
%
% \subsection{Configuration}
%
% \subsubsection{System configuration file \xfile{epstopdf-sys.cfg}}
%
% If \xfile{epstopdf-sys} exists it is loaded at the end of the
% package \xpackage{epstopdf-base} and before the user configuration
% file. It's intended for TeX distributors. Thus they could add
% additional conversion rules (e.g., .gif -> .png) or set options.
%
% \subsubsection{User configuration file \xfile{epstopdf.cfg}}
%
% A configuration file \xfile{epstopdf.cfg} is loaded at the end
% of the package if it exists. It can be used for changing
% the default option setting. Example:
% \begin{quote}
% |\epstopdfsetup{verbose=false}|
% \end{quote}
%
% \subsubsection{Conversion program}
%
% You can use \cs{DeclareGraphicsRule} in a similar way as the
% route via \xprogram{dvips} to specify the conversion
% command line.
% The conversion argument starts with a back tick, followed
% by the conversion command including parameters.
%
% The whole conversion argument should also be wrapped inside
% \cs{epstopdfcall}. This reduces the problem with packages
% (e.g. \xpackage{pst-pdf}) that use the conversion argument
% and expands it. Macros \cs{SourceFile}, \cs{OutputFile}, and
% \cs{SourceExt} are not defined outside \xpackage{epstopdf-base}'s
% \cs{Gin@setfile} and error messages because of undefined
% command names are the result. If \cs{epstopdfcall} detects that
% it is called outside \xpackage{epstopdf-base}'s \cs{Gin@setfile}
% then it replaces the conversion argument by
% package \xpackage{graphics}'s
% default, usually the image file.
%
% The following macros are available inside:
% \begin{description}
% \def\It#1{\item[\cs{#1}:]}
% \It{OutputFile}: output file name (with known path and extension)
% \It{SourceFile}: source file name (with known path and extension),
%                  usually the same as |#1|,
% \It{SourceExt}: source extension without leading dot.
% \end{description}
%
% \paragraph{Conversion from EPS to PDF.}
% Other programs than \xprogram{epstopdf} can be used to convert from
% EPS to PDF. Example that uses \xprogram{Ghostscript}:
% \begin{quote}
%   |\DeclareGraphicsRule{.eps}{pdf}{.pdf}{%|\\
%   |  \epstopdfcall{`ps2pdf -dEPSCrop #1 \noexpand\OutputFile}%|\\
%   |}|
% \end{quote}
% \cs{DeclareGraphicsRule} expands the argument, therefore \cs{noexpand}
% is necessary. As convenience package \xpackage{epstopdf-base} defines
% \cs{epstopdfDeclareGraphicsRule}. Then the conversion argument
% is not expanded, \cs{epstopdfcall} and the back tick are added:
% \begin{quote}
%   |\epstopdfDeclareGraphicsRule{.eps}{pdf}{.pdf}{%|\\
%   |  ps2pdf -dEPSCrop #1 \OutputFile|\\
%   |}|
% \end{quote}
%
% Also \cs{OutputFile} respects the setting of option \xoption{outdir}.
%
% \subsection{Other image formats}
%
% The support that package \xpackage{epstopdf} implements is not
% limited to EPS files. Other image conversions can be declared.
% The following example shows it for GIF images under Unix with
% ImageMagick's \xprogram{convert}:
%\begin{quote}
%\begin{verbatim}
%\epstopdfDeclareGraphicsRule{.gif}{png}{.png}{%
%  convert #1 \OutputFile
%}
%\end{verbatim}
%\end{quote}
% The file extension \xfile{.gif} can be added to the extension list that
% package \xpackage{graphics} searches if the file extension
% is not given in \cs{includegraphics}.
% The list can be set by \cs{GraphicsExtensions}.
% \begin{quote}
% |\AppendGraphicsExtensions{.gif}|\\
% or\\
% |\PrependGraphicsExtensions{.gif}|
% \end{quote}
%
% \StopEventually{
% }
%
% \section{Implementation}
%
%    \begin{macrocode}
%<*package>
%    \end{macrocode}
%
% \subsection{Wrapper package}
%
%    Reload check, especially if the package is not used with \LaTeX.
%    \begin{macrocode}
\begingroup\catcode61\catcode48\catcode32=10\relax%
  \catcode13=5 % ^^M
  \endlinechar=13 %
  \catcode35=6 % #
  \catcode39=12 % '
  \catcode44=12 % ,
  \catcode45=12 % -
  \catcode46=12 % .
  \catcode58=12 % :
  \catcode64=11 % @
  \catcode123=1 % {
  \catcode125=2 % }
  \expandafter\let\expandafter\x\csname ver@epstopdf.sty\endcsname
  \ifx\x\relax % plain-TeX, first loading
  \else
    \def\empty{}%
    \ifx\x\empty % LaTeX, first loading,
      % variable is initialized, but \ProvidesPackage not yet seen
    \else
      \expandafter\ifx\csname PackageInfo\endcsname\relax
        \def\x#1#2{%
          \immediate\write-1{Package #1 Info: #2.}%
        }%
      \else
        \def\x#1#2{\PackageInfo{#1}{#2, stopped}}%
      \fi
      \x{epstopdf}{The package is already loaded}%
      \aftergroup\endinput
    \fi
  \fi
\endgroup%
%    \end{macrocode}
%    Package identification:
%    \begin{macrocode}
\begingroup\catcode61\catcode48\catcode32=10\relax%
  \catcode13=5 % ^^M
  \endlinechar=13 %
  \catcode35=6 % #
  \catcode39=12 % '
  \catcode40=12 % (
  \catcode41=12 % )
  \catcode44=12 % ,
  \catcode45=12 % -
  \catcode46=12 % .
  \catcode47=12 % /
  \catcode58=12 % :
  \catcode64=11 % @
  \catcode91=12 % [
  \catcode93=12 % ]
  \catcode123=1 % {
  \catcode125=2 % }
  \expandafter\ifx\csname ProvidesPackage\endcsname\relax
    \def\x#1#2#3[#4]{\endgroup
      \immediate\write-1{Package: #3 #4}%
      \xdef#1{#4}%
    }%
  \else
    \def\x#1#2[#3]{\endgroup
      #2[{#3}]%
      \ifx#1\@undefined
        \xdef#1{#3}%
      \fi
      \ifx#1\relax
        \xdef#1{#3}%
      \fi
    }%
  \fi
\expandafter\x\csname ver@epstopdf.sty\endcsname
\ProvidesPackage{epstopdf}%
  [2016/05/15 v2.6 Conversion with epstopdf on the fly (HO)]%
%    \end{macrocode}
%    Larger catcode set because of configuration files needed.
%    \begin{macrocode}
\begingroup\catcode61\catcode48\catcode32=10\relax%
  \catcode13=5 % ^^M
  \endlinechar=13 %
  \catcode123=1 % {
  \catcode125=2 % }
  \catcode64=11 % @
  \def\x{\endgroup
    \expandafter\edef\csname ETE@AtEnd\endcsname{%
      \endlinechar=\the\endlinechar\relax
      \catcode13=\the\catcode13\relax
      \catcode32=\the\catcode32\relax
      \catcode35=\the\catcode35\relax
      \catcode61=\the\catcode61\relax
      \catcode64=\the\catcode64\relax
      \catcode123=\the\catcode123\relax
      \catcode125=\the\catcode125\relax
    }%
  }%
\x\catcode61\catcode48\catcode32=10\relax%
\catcode13=5 % ^^M
\endlinechar=13 %
\catcode35=6 % #
\catcode64=11 % @
\catcode123=1 % {
\catcode125=2 % }
\def\TMP@EnsureCode#1#2{%
  \edef\ETE@AtEnd{%
    \ETE@AtEnd
    \catcode#1=\the\catcode#1\relax
  }%
  \catcode#1=#2\relax
}
\TMP@EnsureCode{39}{12}% '
\TMP@EnsureCode{40}{12}% (
\TMP@EnsureCode{41}{12}% )
\TMP@EnsureCode{42}{12}% *
\TMP@EnsureCode{45}{12}% -
\TMP@EnsureCode{47}{12}% /
\TMP@EnsureCode{91}{12}% [
\TMP@EnsureCode{93}{12}% ]
\edef\ETE@AtEnd{\ETE@AtEnd\noexpand\endinput}
%    \end{macrocode}
%
%    \begin{macrocode}
\let\ETE@SavedAtEnd\ETE@AtEnd
\RequirePackage{epstopdf-base}[2016/05/15]
\let\ETE@AtEnd\ETE@SavedAtEnd
%    \end{macrocode}
%
% \subsubsection{Option handling}
%
%    \begin{macrocode}
\DeclareOption*{%
  \expandafter\epstopdfsetup\expandafter{\CurrentOption}%
}
\ProcessOptions*\relax
%    \end{macrocode}
%
%    \begin{macrocode}
\ETE@AtEnd%
%</package>
%    \end{macrocode}
%
% \subsection{Base package}
%
%    \begin{macrocode}
%<*base>
%    \end{macrocode}
%
% \subsection{Preparations}
%
% \subsubsection{Relead check and identification}
%
%    Reload check, especially if the package is not used with \LaTeX.
%    \begin{macrocode}
\begingroup\catcode61\catcode48\catcode32=10\relax%
  \catcode13=5 % ^^M
  \endlinechar=13 %
  \catcode35=6 % #
  \catcode39=12 % '
  \catcode44=12 % ,
  \catcode45=12 % -
  \catcode46=12 % .
  \catcode58=12 % :
  \catcode64=11 % @
  \catcode123=1 % {
  \catcode125=2 % }
  \expandafter\let\expandafter\x\csname ver@epstopdf-base.sty\endcsname
  \ifx\x\relax % plain-TeX, first loading
  \else
    \def\empty{}%
    \ifx\x\empty % LaTeX, first loading,
      % variable is initialized, but \ProvidesPackage not yet seen
    \else
      \expandafter\ifx\csname PackageInfo\endcsname\relax
        \def\x#1#2{%
          \immediate\write-1{Package #1 Info: #2.}%
        }%
      \else
        \def\x#1#2{\PackageInfo{#1}{#2, stopped}}%
      \fi
      \x{epstopdf-base}{The package is already loaded}%
      \aftergroup\endinput
    \fi
  \fi
\endgroup%
%    \end{macrocode}
%    Package identification:
%    \begin{macrocode}
\begingroup\catcode61\catcode48\catcode32=10\relax%
  \catcode13=5 % ^^M
  \endlinechar=13 %
  \catcode35=6 % #
  \catcode39=12 % '
  \catcode40=12 % (
  \catcode41=12 % )
  \catcode44=12 % ,
  \catcode45=12 % -
  \catcode46=12 % .
  \catcode47=12 % /
  \catcode58=12 % :
  \catcode64=11 % @
  \catcode91=12 % [
  \catcode93=12 % ]
  \catcode123=1 % {
  \catcode125=2 % }
  \expandafter\ifx\csname ProvidesPackage\endcsname\relax
    \def\x#1#2#3[#4]{\endgroup
      \immediate\write-1{Package: #3 #4}%
      \xdef#1{#4}%
    }%
  \else
    \def\x#1#2[#3]{\endgroup
      #2[{#3}]%
      \ifx#1\@undefined
        \xdef#1{#3}%
      \fi
      \ifx#1\relax
        \xdef#1{#3}%
      \fi
    }%
  \fi
\expandafter\x\csname ver@epstopdf-base.sty\endcsname
\ProvidesPackage{epstopdf-base}%
  [2016/05/15 v2.6 Base part for package epstopdf]%
%    \end{macrocode}
%
% \subsubsection{Catcodes}
%
%    \begin{macrocode}
\begingroup\catcode61\catcode48\catcode32=10\relax%
  \catcode13=5 % ^^M
  \endlinechar=13 %
  \catcode123=1 % {
  \catcode125=2 % }
  \catcode64=11 % @
  \def\x{\endgroup
    \expandafter\edef\csname ETE@AtEnd\endcsname{%
      \endlinechar=\the\endlinechar\relax
      \catcode13=\the\catcode13\relax
      \catcode32=\the\catcode32\relax
      \catcode35=\the\catcode35\relax
      \catcode61=\the\catcode61\relax
      \catcode64=\the\catcode64\relax
      \catcode123=\the\catcode123\relax
      \catcode125=\the\catcode125\relax
    }%
  }%
\x\catcode61\catcode48\catcode32=10\relax%
\catcode13=5 % ^^M
\endlinechar=13 %
\catcode35=6 % #
\catcode64=11 % @
\catcode123=1 % {
\catcode125=2 % }
\def\TMP@EnsureCode#1#2{%
  \edef\ETE@AtEnd{%
    \ETE@AtEnd
    \catcode#1=\the\catcode#1\relax
  }%
  \catcode#1=#2\relax
}
\TMP@EnsureCode{33}{12}% !
\TMP@EnsureCode{39}{12}% '
\TMP@EnsureCode{42}{12}% *
\TMP@EnsureCode{44}{12}% ,
\TMP@EnsureCode{45}{12}% -
\TMP@EnsureCode{46}{12}% .
\TMP@EnsureCode{47}{12}% /
\TMP@EnsureCode{58}{12}% :
\TMP@EnsureCode{60}{12}% <
\TMP@EnsureCode{62}{12}% >
\TMP@EnsureCode{96}{12}% `
\edef\ETE@AtEnd{\ETE@AtEnd\noexpand\endinput}
%    \end{macrocode}
%
% \subsubsection{Load packages}
%
%    \begin{macrocode}
\RequirePackage{infwarerr}[2007/09/09]
\RequirePackage{grfext}\relax
\RequirePackage{kvoptions}[2007/10/02]
\RequirePackage{pdftexcmds}[2007/11/11]
%    \end{macrocode}
%
% \subsection{Checks}
%
% Check, whether package graphics is loaded
% (also graphicx loads graphics).
% Because miniltx.tex does not know \cs{@ifpackageloaded}
% we test for \cs{Gin@setfile} instead.
%    \begin{macrocode}
\begingroup\expandafter\expandafter\expandafter\endgroup
\expandafter\ifx\csname Gin@setfile\endcsname\relax
  \@PackageWarningNoLine{epstopdf}{%
    No graphics package \string`graphic{s,x}\string' loaded%
  }%
  \newcommand*{\epstopdfsetup}[1]{}%
  \expandafter\ETE@AtEnd
\fi%
%    \end{macrocode}
%
% Check, whether pdftex.def is loaded. \cs{ver@pdftex.def} is not
% available with miniltx.tex, thus we test for \cs{Gin@driver}.
%    \begin{macrocode}
\begingroup
  \def\x{luatex.def}%
  \ifx\Gin@driver\x
  \else
  \def\x{pdftex.def}%
  \ifx\Gin@driver\x
  \else
    \@PackageWarningNoLine{epstopdf}{%
      Drivers other than `pdftex' and `luatex' are not supported%
    }%
    \endgroup
    \newcommand*{\epstopdfsetup}[1]{}%
    \expandafter\ETE@AtEnd
  \fi%
  \fi%
\endgroup
%    \end{macrocode}
%
% Check, whether the shell escape feature is enabled.
%
%    \begin{macrocode}
\begingroup
  \expandafter\ifx\csname pdf@shellescape\endcsname\relax
  \else
    \ifnum\pdf@shellescape>0 %
    \else
      \@PackageWarningNoLine{epstopdf}{%
        Shell escape feature is not enabled%
      }%
    \fi
  \fi
\endgroup
%    \end{macrocode}
%
% \subsection{Package loading}
%
%    \begin{macrocode}
%    \end{macrocode}
%
% \subsection{Options}
%
%    \begin{macrocode}
\SetupKeyvalOptions{family=ETE,prefix=ETE@}
\DeclareBoolOption{update}
\DeclareBoolOption{verbose}
\newif\ifETE@prepend
\DeclareVoidOption{prepend}{\ETE@prependtrue}
\DeclareVoidOption{append}{\ETE@prependfalse}
\DeclareStringOption{outdir}
\DeclareStringOption{suffix}
\DeclareBoolOption{prefersuffix}
\DeclareStringOption{program@epstopdf}
%    \end{macrocode}
%    Options \xoption{disable} and \xoption{enable} are for testing
%    only. Therefore they are not documented on purpose.
%    \begin{macrocode}
\DeclareBoolOption{disable}
\DeclareComplementaryOption{enable}{disable}
%    \end{macrocode}
%    \begin{macrocode}
\newcommand*{\epstopdfsetup}{\setkeys{ETE}}
%    \end{macrocode}
%
% \subsubsection{Default setting}
%
%    \begin{macrocode}
\epstopdfsetup{%
  verbose,%
  enable,%
  append,%
  update,%
  prefersuffix,%
  suffix=-\SourceExt-converted-to,%
  program@epstopdf=epstopdf%
}
%    \end{macrocode}
%
% \subsection{Make and verbose}
%
%    \begin{macrocode}
\begingroup\expandafter\expandafter\expandafter\endgroup
\expandafter\ifx\csname pdf@filemoddate\endcsname\relax
  \def\ETE@Make#1#2{%
    \ifETE@update
      \ETE@WarnModDate
    \fi
    \@firstofone
  }%
  \def\ETE@WarnModDate{%
    \@PackageWarningNoLine{epstopdf}{%
      \string\pdffilemoddate\space is not available,\MessageBreak
      option `update' will be ignored%
    }%
    \global\let\ETE@WarnModDate\relax
  }%
  \def\ETE@FileInfo#1#2{#1 file: <#2>}%
\else
  \def\ETE@Make#1#2{%
    \ifETE@update
      \ifnum\pdf@strcmp{\pdf@filemoddate{#1}}{\pdf@filemoddate{#2}}>0 %
        \expandafter\expandafter\expandafter\@firstofone
      \else
        \@PackageInfoNoLine{epstopdf}{%
          Output file is already uptodate%
        }%
        \expandafter\expandafter\expandafter\@gobble
      \fi
    \else
      \expandafter\@firstofone
    \fi
  }%
  \def\ETE@FileInfo#1#2{%
    #1 file: <#2>%
    \expandafter\expandafter\expandafter
    \ETE@Date\pdf@filemoddate{#2}\@nil
    \expandafter\expandafter\expandafter
    \ETE@Size\pdf@filesize{#2}\@nil
  }%
  \def\ETE@Date#1\@nil{%
    \ifx\\#1\\%
    \else
      \ETE@@Date#1\@nil
    \fi
  }%
  \def\ETE@@Date#1:#2#3#4#5#6#7#8#9{%
    \MessageBreak
    \@spaces\space\space\space date: #2#3#4#5-#6#7-#8#9 %
    \ETE@@Time
  }%
  \def\ETE@@Time#1#2#3#4#5#6#7\@nil{%
    #1#2:#3#4:#5#6%
  }%
  \def\ETE@Size#1\@nil{%
    \ifx\\#1\\%
    \else
      \MessageBreak
      \@spaces\space\space\space size: #1 bytes%
    \fi
  }%
\fi
%    \end{macrocode}
%
% \subsection{Adding conversion support}
%
%    Patch \cs{Gin@setfile} to execute |#3|, if it contains
%    a command.
%    \begin{macrocode}
\expandafter\ifx\csname ETE@OrgGin@setfile\endcsname\relax
  \let\ETE@OrgGin@setfile\Gin@setfile
\else
  \@PackageError{epstopdf}{%
    Command \string\ETE@OrgGin@setfile\space
    already defined.\MessageBreak
  }{%
    Probably some package has included the code of this package%
    \MessageBreak
    instead of using \string\RequirePackage{epstopdf}.%
    \MessageBreak
    \@ehc
  }%
\fi
\def\ETE@IfFileExists{%
  \begingroup\expandafter\expandafter\expandafter\endgroup
  \expandafter\ifx\csname grffile@IfFileExists\endcsname\relax
    \expandafter\IfFileExists
  \else
    \global\let\ETE@IfFileExists\grffile@IfFileExists
    \expandafter\grffile@IfFileExists
  \fi
}
\def\ETE@Skip#1#2\x\fi\fi{%
  \fi
  \fi
  \endgroup
  \fi
  \fi
  #1%
}
\newif\ifETE@InsideSetfile
\newcommand*{\epstopdfcall}[1]{%
  \ifETE@InsideSetfile
    \expandafter\@firstoftwo
  \else
    \expandafter\@secondoftwo
  \fi
  {`#1}%
  {\Gin@base\Gin@ext}%
}
\def\ETE@DefCommandLine#1{%
  \edef\CommandLine{\expandafter\fi\if`#1}%
}
\def\ETE@DefX#1{%
  \expandafter\expandafter\expandafter\def
  \expandafter\expandafter\expandafter\x
  \expandafter\expandafter\expandafter{%
    \expandafter\fi\if`#1\relax\else
  }%
}
\def\ETE@Gin@setfile#1#2#3{%
  \ifETE@disable
    \ETE@OrgGin@setfile{#1}{#2}{#3}%
  \else
    \begingroup
      \ETE@InsideSetfiletrue
      \ETE@DefX{#3}%
    \expandafter\endgroup
    \ifx\x\@empty
      \ETE@OrgGin@setfile{#1}{#2}{#3}%
    \else
      \begingroup
        \ETE@InsideSetfiletrue
        \def\GraphicsType{#1}%
        \def\GraphicsRead{#2}%
        \ifETE@prefersuffix
        \else
          \ifx\ETE@suffix\@empty
          \else
            \ETE@IfFileExists{\Gin@base\GraphicsRead}{%
              \ETE@Skip{%
                \ETE@OrgGin@setfile{#1}{#2}{\Gin@base#2}%
              }%
            }{%
              \let\next\relax
            }%
            \next
          \fi
        \fi
        \ifx\Gin@ext\relax
          \let\SourceExt\Gin@eext
          \def\SourceFile{\Gin@base\Gin@eext}%
        \else
          \let\SourceExt\Gin@ext
          \def\SourceFile{\Gin@base\Gin@ext}%
        \fi
        \edef\SourceExt{% remove dot
          \expandafter\@cdr\SourceExt\@empty\@nil
        }%
        \let\OutputDirectory\ETE@outdir
        \ifx\OutputDirectory\@empty
          \edef\OutputFile{\ETE@GenerateName{\Gin@base}{#2}}%
        \else
          \begingroup
            \filename@parse{\Gin@base#2}%
            \edef\x{\endgroup
              \def\noexpand\OutputFile{%
                \ETE@GenerateName{%
                  \OutputDirectory\filename@base
                }{#2}%
              }%
            }%
          \x
        \fi
        \ETE@DefCommandLine{#3}%
        \ifETE@verbose
          \@PackageInfo{epstopdf}{%
            \ETE@FileInfo{Source}\SourceFile\MessageBreak
            \ETE@FileInfo{Output}\OutputFile\MessageBreak
            Command: <\CommandLine>\MessageBreak
            \string\includegraphics
          }%
        \fi
        \ETE@Make\SourceFile\OutputFile{%
          \pdf@system{\CommandLine}%
          \ifETE@verbose
            \@PackageInfoNoLine{epstopdf}{%
              \ETE@FileInfo{Result}\OutputFile
            }%
          \fi
        }%
        \edef\x{\endgroup
          \ifx\OutputDirectory\@empty
          \else
            \def\noexpand\Gin@base{%
              \OutputDirectory\noexpand\filename@base
            }%
          \fi
          \ifx\ETE@suffix\@empty
          \else
            \edef\noexpand\Gin@base{%
              \noexpand\Gin@base\ETE@suffix
            }%
          \fi
          \noexpand\ETE@OrgGin@setfile{%
            \GraphicsType
          }{%
            \GraphicsRead
          }{%
            \OutputFile
          }%
        }%
      \x
    \fi
  \fi
}
\let\Gin@setfile\ETE@Gin@setfile
\def\ETE@GenerateName#1#2{%
  #1\ETE@suffix#2%
}
%    \end{macrocode}
%
% \subsection{Declare graphics rule}
%
%    \begin{macrocode}
\newcommand*{\epstopdfDeclareGraphicsRule}[4]{%
  \ifx\\#4\\%
    \@PackageError{epstopdf-base}{%
      Conversion command is missing%
    }\@ehc
  \else
    \begingroup
      \@ifundefined{Gin@rule@#1}{%
      }{%
        \@PackageInfo{epstopdf-base}{%
          Redefining graphics rule for `#1'%
        }%
      }%
    \endgroup
    \@namedef{Gin@rule@#1}##1{{#2}{#3}{\epstopdfcall{#4}}}%
  \fi
}
%    \end{macrocode}
%
% \cs{DeclareGraphicsRule} for .eps
%    \begin{macrocode}
\epstopdfDeclareGraphicsRule{.eps}{pdf}{.pdf}{%
  \ETE@epstopdf{#1}%
}
\def\ETE@epstopdf#1{%
  \ETE@program@epstopdf\space
  \ifcase\ifx\OutputDirectory\@empty
           \ifx\ETE@suffix\@empty
             1%
           \fi
         \fi
         0 %
    --outfile=\OutputFile\space
  \fi
  #1%
}
%    \end{macrocode}
%
%    \begin{macrocode}
\ifETE@prepend
  \expandafter\PrependGraphicsExtensions
\else
  \expandafter\AppendGraphicsExtensions
\fi
{.eps}
\let\ETE@prepend\@undefined
\DeclareVoidOption{prepend}{%
  \PrependGraphicsExtensions{.eps}%
}
\let\ETE@append\@undefined
\DeclareVoidOption{append}{%
  \AppendGraphicsExtensions{.eps}%
}
%    \end{macrocode}
%
%    \begin{macrocode}
\InputIfFileExists{epstopdf-sys.cfg}{}{}
\InputIfFileExists{epstopdf.cfg}{}{}
%    \end{macrocode}
%
%    \begin{macrocode}
\ETE@AtEnd%
%</base>
%    \end{macrocode}
%
% \section{Test}
%
% \subsection{Preface for standard catcode check}
%
%    \begin{macrocode}
%<*test1>
\input miniltx.tex\relax
\def\Gin@driver{pdftex.def}
\input graphicx.sty\relax
\resetatcatcode
%</test1>
%    \end{macrocode}
%
% \subsection{Catcode checks for loading}
%
%    \begin{macrocode}
%<*test1>
%    \end{macrocode}
%    \begin{macrocode}
\catcode`\{=1 %
\catcode`\}=2 %
\catcode`\#=6 %
\catcode`\@=11 %
\expandafter\ifx\csname count@\endcsname\relax
  \countdef\count@=255 %
\fi
\expandafter\ifx\csname @gobble\endcsname\relax
  \long\def\@gobble#1{}%
\fi
\expandafter\ifx\csname @firstofone\endcsname\relax
  \long\def\@firstofone#1{#1}%
\fi
\expandafter\ifx\csname loop\endcsname\relax
  \expandafter\@firstofone
\else
  \expandafter\@gobble
\fi
{%
  \def\loop#1\repeat{%
    \def\body{#1}%
    \iterate
  }%
  \def\iterate{%
    \body
      \let\next\iterate
    \else
      \let\next\relax
    \fi
    \next
  }%
  \let\repeat=\fi
}%
\def\RestoreCatcodes{}
\count@=0 %
\loop
  \edef\RestoreCatcodes{%
    \RestoreCatcodes
    \catcode\the\count@=\the\catcode\count@\relax
  }%
\ifnum\count@<255 %
  \advance\count@ 1 %
\repeat

\def\RangeCatcodeInvalid#1#2{%
  \count@=#1\relax
  \loop
    \catcode\count@=15 %
  \ifnum\count@<#2\relax
    \advance\count@ 1 %
  \repeat
}
\def\RangeCatcodeCheck#1#2#3{%
  \count@=#1\relax
  \loop
    \ifnum#3=\catcode\count@
    \else
      \errmessage{%
        Character \the\count@\space
        with wrong catcode \the\catcode\count@\space
        instead of \number#3%
      }%
    \fi
  \ifnum\count@<#2\relax
    \advance\count@ 1 %
  \repeat
}
\def\space{ }
\expandafter\ifx\csname LoadCommand\endcsname\relax
  \def\LoadCommand{\input epstopdf.sty\relax}%
\fi
\def\Test{%
  \RangeCatcodeInvalid{0}{47}%
  \RangeCatcodeInvalid{58}{64}%
  \RangeCatcodeInvalid{91}{96}%
  \RangeCatcodeInvalid{123}{255}%
  \catcode`\@=12 %
  \catcode`\\=0 %
  \catcode`\%=14 %
  \LoadCommand
  \RangeCatcodeCheck{0}{36}{15}%
  \RangeCatcodeCheck{37}{37}{14}%
  \RangeCatcodeCheck{38}{47}{15}%
  \RangeCatcodeCheck{48}{57}{12}%
  \RangeCatcodeCheck{58}{63}{15}%
  \RangeCatcodeCheck{64}{64}{12}%
  \RangeCatcodeCheck{65}{90}{11}%
  \RangeCatcodeCheck{91}{91}{15}%
  \RangeCatcodeCheck{92}{92}{0}%
  \RangeCatcodeCheck{93}{96}{15}%
  \RangeCatcodeCheck{97}{122}{11}%
  \RangeCatcodeCheck{123}{255}{15}%
  \RestoreCatcodes
}
\Test
\csname @@end\endcsname
\end
%    \end{macrocode}
%    \begin{macrocode}
%</test1>
%    \end{macrocode}
%
% \section{Installation}
%
% \subsection{Download}
%
% \paragraph{Package.} This package is available on
% CTAN\footnote{\url{http://ctan.org/pkg/epstopdf}}:
% \begin{description}
% \item[\CTAN{macros/latex/contrib/oberdiek/epstopdf.dtx}] The source file.
% \item[\CTAN{macros/latex/contrib/oberdiek/epstopdf.pdf}] Documentation.
% \end{description}
%
%
% \paragraph{Bundle.} All the packages of the bundle `oberdiek'
% are also available in a TDS compliant ZIP archive. There
% the packages are already unpacked and the documentation files
% are generated. The files and directories obey the TDS standard.
% \begin{description}
% \item[\CTAN{install/macros/latex/contrib/oberdiek.tds.zip}]
% \end{description}
% \emph{TDS} refers to the standard ``A Directory Structure
% for \TeX\ Files'' (\CTAN{tds/tds.pdf}). Directories
% with \xfile{texmf} in their name are usually organized this way.
%
% \subsection{Bundle installation}
%
% \paragraph{Unpacking.} Unpack the \xfile{oberdiek.tds.zip} in the
% TDS tree (also known as \xfile{texmf} tree) of your choice.
% Example (linux):
% \begin{quote}
%   |unzip oberdiek.tds.zip -d ~/texmf|
% \end{quote}
%
% \paragraph{Script installation.}
% Check the directory \xfile{TDS:scripts/oberdiek/} for
% scripts that need further installation steps.
% Package \xpackage{attachfile2} comes with the Perl script
% \xfile{pdfatfi.pl} that should be installed in such a way
% that it can be called as \texttt{pdfatfi}.
% Example (linux):
% \begin{quote}
%   |chmod +x scripts/oberdiek/pdfatfi.pl|\\
%   |cp scripts/oberdiek/pdfatfi.pl /usr/local/bin/|
% \end{quote}
%
% \subsection{Package installation}
%
% \paragraph{Unpacking.} The \xfile{.dtx} file is a self-extracting
% \docstrip\ archive. The files are extracted by running the
% \xfile{.dtx} through \plainTeX:
% \begin{quote}
%   \verb|tex epstopdf.dtx|
% \end{quote}
%
% \paragraph{TDS.} Now the different files must be moved into
% the different directories in your installation TDS tree
% (also known as \xfile{texmf} tree):
% \begin{quote}
% \def\t{^^A
% \begin{tabular}{@{}>{\ttfamily}l@{ $\rightarrow$ }>{\ttfamily}l@{}}
%   epstopdf.sty & tex/latex/oberdiek/epstopdf.sty\\
%   epstopdf-base.sty & tex/latex/oberdiek/epstopdf-base.sty\\
%   epstopdf.pdf & doc/latex/oberdiek/epstopdf.pdf\\
%   test/epstopdf-test1.tex & doc/latex/oberdiek/test/epstopdf-test1.tex\\
%   epstopdf.dtx & source/latex/oberdiek/epstopdf.dtx\\
% \end{tabular}^^A
% }^^A
% \sbox0{\t}^^A
% \ifdim\wd0>\linewidth
%   \begingroup
%     \advance\linewidth by\leftmargin
%     \advance\linewidth by\rightmargin
%   \edef\x{\endgroup
%     \def\noexpand\lw{\the\linewidth}^^A
%   }\x
%   \def\lwbox{^^A
%     \leavevmode
%     \hbox to \linewidth{^^A
%       \kern-\leftmargin\relax
%       \hss
%       \usebox0
%       \hss
%       \kern-\rightmargin\relax
%     }^^A
%   }^^A
%   \ifdim\wd0>\lw
%     \sbox0{\small\t}^^A
%     \ifdim\wd0>\linewidth
%       \ifdim\wd0>\lw
%         \sbox0{\footnotesize\t}^^A
%         \ifdim\wd0>\linewidth
%           \ifdim\wd0>\lw
%             \sbox0{\scriptsize\t}^^A
%             \ifdim\wd0>\linewidth
%               \ifdim\wd0>\lw
%                 \sbox0{\tiny\t}^^A
%                 \ifdim\wd0>\linewidth
%                   \lwbox
%                 \else
%                   \usebox0
%                 \fi
%               \else
%                 \lwbox
%               \fi
%             \else
%               \usebox0
%             \fi
%           \else
%             \lwbox
%           \fi
%         \else
%           \usebox0
%         \fi
%       \else
%         \lwbox
%       \fi
%     \else
%       \usebox0
%     \fi
%   \else
%     \lwbox
%   \fi
% \else
%   \usebox0
% \fi
% \end{quote}
% If you have a \xfile{docstrip.cfg} that configures and enables \docstrip's
% TDS installing feature, then some files can already be in the right
% place, see the documentation of \docstrip.
%
% \subsection{Refresh file name databases}
%
% If your \TeX~distribution
% (\teTeX, \mikTeX, \dots) relies on file name databases, you must refresh
% these. For example, \teTeX\ users run \verb|texhash| or
% \verb|mktexlsr|.
%
% \subsection{Some details for the interested}
%
% \paragraph{Attached source.}
%
% The PDF documentation on CTAN also includes the
% \xfile{.dtx} source file. It can be extracted by
% AcrobatReader 6 or higher. Another option is \textsf{pdftk},
% e.g. unpack the file into the current directory:
% \begin{quote}
%   \verb|pdftk epstopdf.pdf unpack_files output .|
% \end{quote}
%
% \paragraph{Unpacking with \LaTeX.}
% The \xfile{.dtx} chooses its action depending on the format:
% \begin{description}
% \item[\plainTeX:] Run \docstrip\ and extract the files.
% \item[\LaTeX:] Generate the documentation.
% \end{description}
% If you insist on using \LaTeX\ for \docstrip\ (really,
% \docstrip\ does not need \LaTeX), then inform the autodetect routine
% about your intention:
% \begin{quote}
%   \verb|latex \let\install=y% \iffalse meta-comment
%
% File: epstopdf.dtx
% Version: 2016/05/15 v2.6
% Info: Conversion with epstopdf on the fly
%
% Copyright (C) 2001, 2006-2010 by
%    Heiko Oberdiek <heiko.oberdiek at googlemail.com>
%
% This work may be distributed and/or modified under the
% conditions of the LaTeX Project Public License, either
% version 1.3c of this license or (at your option) any later
% version. This version of this license is in
%    http://www.latex-project.org/lppl/lppl-1-3c.txt
% and the latest version of this license is in
%    http://www.latex-project.org/lppl.txt
% and version 1.3 or later is part of all distributions of
% LaTeX version 2005/12/01 or later.
%
% This work has the LPPL maintenance status "maintained".
%
% This Current Maintainer of this work is Heiko Oberdiek.
%
% This work consists of the main source file epstopdf.dtx
% and the derived files
%    epstopdf.sty, epstopdf.pdf, epstopdf.ins, epstopdf.drv,
%    epstopdf-base.sty, epstopdf-test1.tex.
%
% Distribution:
%    CTAN:macros/latex/contrib/oberdiek/epstopdf.dtx
%    CTAN:macros/latex/contrib/oberdiek/epstopdf.pdf
%
% Unpacking:
%    (a) If epstopdf.ins is present:
%           tex epstopdf.ins
%    (b) Without epstopdf.ins:
%           tex epstopdf.dtx
%    (c) If you insist on using LaTeX
%           latex \let\install=y% \iffalse meta-comment
%
% File: epstopdf.dtx
% Version: 2016/05/15 v2.6
% Info: Conversion with epstopdf on the fly
%
% Copyright (C) 2001, 2006-2010 by
%    Heiko Oberdiek <heiko.oberdiek at googlemail.com>
%
% This work may be distributed and/or modified under the
% conditions of the LaTeX Project Public License, either
% version 1.3c of this license or (at your option) any later
% version. This version of this license is in
%    http://www.latex-project.org/lppl/lppl-1-3c.txt
% and the latest version of this license is in
%    http://www.latex-project.org/lppl.txt
% and version 1.3 or later is part of all distributions of
% LaTeX version 2005/12/01 or later.
%
% This work has the LPPL maintenance status "maintained".
%
% This Current Maintainer of this work is Heiko Oberdiek.
%
% This work consists of the main source file epstopdf.dtx
% and the derived files
%    epstopdf.sty, epstopdf.pdf, epstopdf.ins, epstopdf.drv,
%    epstopdf-base.sty, epstopdf-test1.tex.
%
% Distribution:
%    CTAN:macros/latex/contrib/oberdiek/epstopdf.dtx
%    CTAN:macros/latex/contrib/oberdiek/epstopdf.pdf
%
% Unpacking:
%    (a) If epstopdf.ins is present:
%           tex epstopdf.ins
%    (b) Without epstopdf.ins:
%           tex epstopdf.dtx
%    (c) If you insist on using LaTeX
%           latex \let\install=y\input{epstopdf.dtx}
%        (quote the arguments according to the demands of your shell)
%
% Documentation:
%    (a) If epstopdf.drv is present:
%           latex epstopdf.drv
%    (b) Without epstopdf.drv:
%           latex epstopdf.dtx; ...
%    The class ltxdoc loads the configuration file ltxdoc.cfg
%    if available. Here you can specify further options, e.g.
%    use A4 as paper format:
%       \PassOptionsToClass{a4paper}{article}
%
%    Programm calls to get the documentation (example):
%       pdflatex epstopdf.dtx
%       makeindex -s gind.ist epstopdf.idx
%       pdflatex epstopdf.dtx
%       makeindex -s gind.ist epstopdf.idx
%       pdflatex epstopdf.dtx
%
% Installation:
%    TDS:tex/latex/oberdiek/epstopdf.sty
%    TDS:tex/latex/oberdiek/epstopdf-base.sty
%    TDS:doc/latex/oberdiek/epstopdf.pdf
%    TDS:doc/latex/oberdiek/test/epstopdf-test1.tex
%    TDS:source/latex/oberdiek/epstopdf.dtx
%
%<*ignore>
\begingroup
  \catcode123=1 %
  \catcode125=2 %
  \def\x{LaTeX2e}%
\expandafter\endgroup
\ifcase 0\ifx\install y1\fi\expandafter
         \ifx\csname processbatchFile\endcsname\relax\else1\fi
         \ifx\fmtname\x\else 1\fi\relax
\else\csname fi\endcsname
%</ignore>
%<*install>
\input docstrip.tex
\Msg{************************************************************************}
\Msg{* Installation}
\Msg{* Package: epstopdf 2016/05/15 v2.6 Conversion with epstopdf on the fly (HO)}
\Msg{************************************************************************}

\keepsilent
\askforoverwritefalse

\let\MetaPrefix\relax
\preamble

This is a generated file.

Project: epstopdf
Version: 2016/05/15 v2.6

Copyright (C) 2001, 2006-2010 by
   Heiko Oberdiek <heiko.oberdiek at googlemail.com>

This work may be distributed and/or modified under the
conditions of the LaTeX Project Public License, either
version 1.3c of this license or (at your option) any later
version. This version of this license is in
   http://www.latex-project.org/lppl/lppl-1-3c.txt
and the latest version of this license is in
   http://www.latex-project.org/lppl.txt
and version 1.3 or later is part of all distributions of
LaTeX version 2005/12/01 or later.

This work has the LPPL maintenance status "maintained".

This Current Maintainer of this work is Heiko Oberdiek.

This work consists of the main source file epstopdf.dtx
and the derived files
   epstopdf.sty, epstopdf.pdf, epstopdf.ins, epstopdf.drv,
   epstopdf-base.sty, epstopdf-test1.tex.

\endpreamble
\let\MetaPrefix\DoubleperCent

\generate{%
  \file{epstopdf.ins}{\from{epstopdf.dtx}{install}}%
  \file{epstopdf.drv}{\from{epstopdf.dtx}{driver}}%
  \usedir{tex/latex/oberdiek}%
  \file{epstopdf.sty}{\from{epstopdf.dtx}{package}}%
  \file{epstopdf-base.sty}{\from{epstopdf.dtx}{base}}%
  \usedir{doc/latex/oberdiek/test}%
  \file{epstopdf-test1.tex}{\from{epstopdf.dtx}{test1}}%
  \nopreamble
  \nopostamble
  \usedir{source/latex/oberdiek/catalogue}%
  \file{epstopdf-pkg.xml}{\from{epstopdf.dtx}{catalogue}}%
}

\catcode32=13\relax% active space
\let =\space%
\Msg{************************************************************************}
\Msg{*}
\Msg{* To finish the installation you have to move the following}
\Msg{* files into a directory searched by TeX:}
\Msg{*}
\Msg{*     epstopdf.sty, epstopdf-base.sty}
\Msg{*}
\Msg{* To produce the documentation run the file `epstopdf.drv'}
\Msg{* through LaTeX.}
\Msg{*}
\Msg{* Happy TeXing!}
\Msg{*}
\Msg{************************************************************************}

\endbatchfile
%</install>
%<*ignore>
\fi
%</ignore>
%<*driver>
\NeedsTeXFormat{LaTeX2e}
\ProvidesFile{epstopdf.drv}%
  [2016/05/15 v2.6 Conversion with epstopdf on the fly (HO)]%
\documentclass{ltxdoc}
\usepackage{holtxdoc}[2011/11/22]
\begin{document}
  \DocInput{epstopdf.dtx}%
\end{document}
%</driver>
% \fi
%
%
% \CharacterTable
%  {Upper-case    \A\B\C\D\E\F\G\H\I\J\K\L\M\N\O\P\Q\R\S\T\U\V\W\X\Y\Z
%   Lower-case    \a\b\c\d\e\f\g\h\i\j\k\l\m\n\o\p\q\r\s\t\u\v\w\x\y\z
%   Digits        \0\1\2\3\4\5\6\7\8\9
%   Exclamation   \!     Double quote  \"     Hash (number) \#
%   Dollar        \$     Percent       \%     Ampersand     \&
%   Acute accent  \'     Left paren    \(     Right paren   \)
%   Asterisk      \*     Plus          \+     Comma         \,
%   Minus         \-     Point         \.     Solidus       \/
%   Colon         \:     Semicolon     \;     Less than     \<
%   Equals        \=     Greater than  \>     Question mark \?
%   Commercial at \@     Left bracket  \[     Backslash     \\
%   Right bracket \]     Circumflex    \^     Underscore    \_
%   Grave accent  \`     Left brace    \{     Vertical bar  \|
%   Right brace   \}     Tilde         \~}
%
% \GetFileInfo{epstopdf.drv}
%
% \title{The \xpackage{epstopdf} package}
% \date{2016/05/15 v2.6}
% \author{Heiko Oberdiek\thanks
% {Please report any issues at https://github.com/ho-tex/oberdiek/issues}\\
% \xemail{heiko.oberdiek at googlemail.com}}
%
% \maketitle
%
% \begin{abstract}
% This packages adds support of handling eps images
% to package \xpackage{graphics} or \xpackage{graphicx}
% with option \xoption{pdftex}. If an eps image is detected,
% epstopdf is automatically called to convert it to pdf format.
% \end{abstract}
%
% \tableofcontents
%
% \providecommand*{\xprogram}[1]{\textsf{#1}}
%
% \section{Documentation}
%
% \subsection{Introduction}
%
% \LaTeX\ provides its graphics bundle to include graphics files.
% Both packages \xpackage{graphics} or \xpackage{graphicx} may
% be used. the latter one loads the first and adds options
% in key value style for \cs{includegraphics}.
%
% Usually the drivers do not support all kind of graphics files.
% Other image types must be converted, before they become
% usuable. In case of driver \xprogram{dvips}, the graphics
% rule may contain a conversion rule. Then all that package
% \xpackage{graphics} must know is the bounding box, the command
% is passed to \xprogram{dvips} that calls it and embeds the
% converted image.
%
% However, \pdfTeX\ has its driver for PDF output already
% build in. It's graphics inclusion commands (\cs{pdfximage})
% does not allow the execution of external commands.
% Therefore commands in the last argument of
% \cs{DeclareGraphicsRule} were of no use.
% But external programs can be called within \pdfTeX.
% This feature is called ``shell escape'' or ``write 18''
% and must usually enabled explicitely because of security reasons.
% Now, this package \xpackage{epstopdf} hooks
% into package \xpackage{graphics}' code to catch
% that argument with the external command and executes it
% to convert the graphics file to a supported format and
% passes the control of graphics inclusion back to package
% \xpackage{graphics}.
%
% \subsection{Requirements}
%
% \begin{itemize}
% \item
%   The feature \cs{write18} must be enabled. This allows the
%   running of external programs during \TeX's compile run.
%   Keep in mind that this is a security risk. The feature is an
%   addition to \cs{TeX}. Mik\TeX, te\TeX, \TeX\ Live support it.
%   In Web2C based \TeX\ distributions (te\TeX, \TeX\ Live) it can be
%   enabled in the configuration file \xfile{texmf.cnf}:
%   \begin{quote}
%     |shell_escape = 1|
%   \end{quote}
%   Because of the security risk, it is better to do it on the
%   command line only:
%   \begin{quote}
%   \texttt{--shell-escape} (te\TeX, \TeX\ Live)\\
%   \texttt{--enable-write18} (MiK\TeX)
%   \end{quote}
%   Example:
%   \begin{quote}
%     |pdflatex -shell-escape test.tex|
%   \end{quote}
% \item
%   The program \xprogram{epstopdf} for the conversion from EPS
%   to PDF. However, other programs can
%   be used and configured by \cs{DeclareGraphicsRule}.
%   Example:
%   \begin{quote}
%   |\epstopdfDeclareGraphicsRule{.eps}{pdf}{.pdf}{%|\\
%   |  ps2pdf -dEPSCrop #1 \OutputFile|\\
%   |}|
%   \end{quote}
% \end{itemize}
%
% \subsection{Usage}
%
% The package is loaded after graphic\{s,x\}, e.g.:
% \begin{quote}
%   |\usepackage[pdftex]{graphicx}|\\
%   |\usepackage{epstopdf}|
% \end{quote}
% Now images with file name extension \xfile{.eps} are detected
% and supported using \cs{includegraphics}.
%
% If the graphics file name is explicitly specified with extension
% \xfile{.eps} the new rule for EPS files is called and
% the conversion performed. If option \xoption{update} is in force
% then the conversion step is dropped if the target file already
% exists and is not older then the EPS file.
%
% The situation is more complicate if the graphics file is
% given without file name extension. Then the \xpackage{graphics}
% package must search for a supported image file.
% The possible extensions are stored in the graphics extension
% list, that can be set by \cs{DeclareGraphicsExtensions}.
% The algorithm:
% \begin{quote}
%    \def\>{\mbox{}\quad}
%    function search( \meta{filebase} )\\
%    \>foreach \meta{ext} in \meta{graphics extensions}\\
%    \>\>foreach \meta{dir} in \meta{current directory}, \meta{\cs{graphicspath}}\\
%    \>\>\>\meta{file} := \meta{dir} + \meta{filebase} + \meta{ext}\\
%    \>\>\>if exist \meta{file}\\
%    \>\>\>\>return found\\
%    \>return not found
% \end{quote}
% Package \xpackage{epstopdf} puts \xfile{.eps} at the end of the
% graphics extension search list. This is the behaviour of option
% \xoption{append} that is enabled by default.
% That means, the conversion is called last
% unless a supported file type cannot be found earlier. This avoids
% unnecessary conversion steps that slow down the \LaTeX\ run.
% If you want to use option \xoption{update} and your \pdfTeX\ supports
% it, then an outdated PDF file also would be found earlier unless
% \xoption{suffix} is used that is the default since version 2.0.
%
% With an empty option \xoption{suffix} and option
% \xoption{prepend} there is a risk that an original PDF file is overwritten:
%
% If the original image file is the PDF file and there is also a generated
% EPS file, then the original PDF file can be regenerated (depending
% on the option settings) and the original PDF file gets lost.
% Therefore option \xoption{suffix} is introduced in version 1.9
% to create a separate name space for generated output files.
%
% \textbf{Note:} Usually the conversion program needs the exact location
% of the image file. Usually the current directory works.
% Also if the image file is found using \cs{graphicspath}, the
% location is known. However, if the image is somewhere
% in a directory of environment variable |TEXINPUTS|, then
% the package does not know the exact location and the
% conversion program will not find the image file unless it
% implements a search using |TEXINPUTS| (program \xprogram{kpsewhich}
% may be of help in this task).
%
% \subsection{Options}
%
% Options can be given as package options or later using:
% \begin{declcs}{epstopdfsetup} \M{key value list}
% \end{declcs}
% \LaTeX\ expands the option list before passing the option list
% to the package's option handling code. This can fail
% for option \xoption{suffix} if it contains some of the macros
% described below. Use \cs{epstopdfsetup} after the package is loaded.
% Or load package \xpackage{kvoptions-patch} before. This package
% is also loaded by option \xoption{patch} of package \xpackage{kvoptions}.
% \LaTeX's option code is redefined to respect key value options and
% let the values untouched.
%
% \begin{description}
% \def\It#1{\item[\xoption{#1}:]}^^A
% \It{update}
%   The conversion program is only called, if the target file
%   does not exist or is older than the source image file.
% \It{append}
%   Puts the extension \xfile{.eps} at the end of the graphics extension list
%   (default).
% \It{prepend}
%   Puts the extension \xfile{.eps} at the begin of the graphics extension list.
% \It{outdir}
%   The converted file may put in an other output directory.
%   The value of \xoption{outdir} must include the directory separator.
%   Example for the current directory:
%   \begin{quote}
%   |\epstopdfsetup{outdir=./}|
%   \end{quote}
%   For other directories ensure, that they can be found.
%   See \cs{graphicspath} or |TEXINPUTS|.
% \It{suffix}
%   This option takes a string that is put between the file name base
%   and the extension of the output file. Rationale: It can happen,
%   that a PDF file is the original file and the EPS file the
%   generated file. If now the package thinks, that the PDF file
%   is the generated file, it will `regnerate' it. But in reality
%   the original file is lost. Therefore I recommend to use this
%   option always to generate a separate name space for generated
%   files. Proposed value is \texttt{-generated} or \texttt{.generated}.
%   The suffix \texttt{.generated} will also work here without the need for
%   package \xpackage{grffile}).
%
%   Example:
%   \begin{quote}
%   |\epstopdfsetup{suffix=-generated}|\\
%   Then \xfile{foo.eps} is converted to \xfile{foo-generated.pdf}.
%   \end{quote}
%   \cs{SourceExt} can be used inside the suffix string.
%   It's will be replaced by the extension of the image source
%   file without the leading dot, for instance:
%   \begin{quote}
%   |\epstopdfsetup{suffix=-\SourceExt-converted-to}|\\
%   \xfile{foo.eps} $\Rightarrow$ \xfile{foo-eps-converted-to.pdf}
%   \end{quote}
%   See also the next option \xoption{prefersuffix} that
%   modifies the behaviour of option \xoption{suffix} in some
%   cases.
%
%   Default for \xoption{suffix} is `|-\SourceExt-converted-to|'.
%
% \It{prefersuffix}
%   If a suffix is set by option \xoption{suffix}, then there can be
%   two image file names that could be taken into account for inclusion:
%   A image file name with the suffix string inside
%   its name and a image file name without; e.g. for \xfile{foo.eps}
%   the names could be:
%   \begin{quote}
%   \xfile{foo-suffix.pdf}, \xfile{foo.pdf}
%   \end{quote}
%   If option \xoption{perfersuffix} is turned on, the
%   file \xfile{foo-suffix.pdf} and its generation is preferred
%   over using \xfile{foo.pdf}. Otherwise \xfile{foo.pdf} is included
%   without generating \xfile{foo-suffix.pdf}.
%   The default of option \xoption{prefersuffix} is \texttt{true}.
% \It{program@epstopdf}
%   The name for the conversion program from EPS to PDF,
%   default is `|epstopdf|'.
% \It{verbose}
%   It prints some information about the image in the \xfile{.log} file
%   (default).
% \end{description}
%
% \subsection{Configuration}
%
% \subsubsection{System configuration file \xfile{epstopdf-sys.cfg}}
%
% If \xfile{epstopdf-sys} exists it is loaded at the end of the
% package \xpackage{epstopdf-base} and before the user configuration
% file. It's intended for TeX distributors. Thus they could add
% additional conversion rules (e.g., .gif -> .png) or set options.
%
% \subsubsection{User configuration file \xfile{epstopdf.cfg}}
%
% A configuration file \xfile{epstopdf.cfg} is loaded at the end
% of the package if it exists. It can be used for changing
% the default option setting. Example:
% \begin{quote}
% |\epstopdfsetup{verbose=false}|
% \end{quote}
%
% \subsubsection{Conversion program}
%
% You can use \cs{DeclareGraphicsRule} in a similar way as the
% route via \xprogram{dvips} to specify the conversion
% command line.
% The conversion argument starts with a back tick, followed
% by the conversion command including parameters.
%
% The whole conversion argument should also be wrapped inside
% \cs{epstopdfcall}. This reduces the problem with packages
% (e.g. \xpackage{pst-pdf}) that use the conversion argument
% and expands it. Macros \cs{SourceFile}, \cs{OutputFile}, and
% \cs{SourceExt} are not defined outside \xpackage{epstopdf-base}'s
% \cs{Gin@setfile} and error messages because of undefined
% command names are the result. If \cs{epstopdfcall} detects that
% it is called outside \xpackage{epstopdf-base}'s \cs{Gin@setfile}
% then it replaces the conversion argument by
% package \xpackage{graphics}'s
% default, usually the image file.
%
% The following macros are available inside:
% \begin{description}
% \def\It#1{\item[\cs{#1}:]}
% \It{OutputFile}: output file name (with known path and extension)
% \It{SourceFile}: source file name (with known path and extension),
%                  usually the same as |#1|,
% \It{SourceExt}: source extension without leading dot.
% \end{description}
%
% \paragraph{Conversion from EPS to PDF.}
% Other programs than \xprogram{epstopdf} can be used to convert from
% EPS to PDF. Example that uses \xprogram{Ghostscript}:
% \begin{quote}
%   |\DeclareGraphicsRule{.eps}{pdf}{.pdf}{%|\\
%   |  \epstopdfcall{`ps2pdf -dEPSCrop #1 \noexpand\OutputFile}%|\\
%   |}|
% \end{quote}
% \cs{DeclareGraphicsRule} expands the argument, therefore \cs{noexpand}
% is necessary. As convenience package \xpackage{epstopdf-base} defines
% \cs{epstopdfDeclareGraphicsRule}. Then the conversion argument
% is not expanded, \cs{epstopdfcall} and the back tick are added:
% \begin{quote}
%   |\epstopdfDeclareGraphicsRule{.eps}{pdf}{.pdf}{%|\\
%   |  ps2pdf -dEPSCrop #1 \OutputFile|\\
%   |}|
% \end{quote}
%
% Also \cs{OutputFile} respects the setting of option \xoption{outdir}.
%
% \subsection{Other image formats}
%
% The support that package \xpackage{epstopdf} implements is not
% limited to EPS files. Other image conversions can be declared.
% The following example shows it for GIF images under Unix with
% ImageMagick's \xprogram{convert}:
%\begin{quote}
%\begin{verbatim}
%\epstopdfDeclareGraphicsRule{.gif}{png}{.png}{%
%  convert #1 \OutputFile
%}
%\end{verbatim}
%\end{quote}
% The file extension \xfile{.gif} can be added to the extension list that
% package \xpackage{graphics} searches if the file extension
% is not given in \cs{includegraphics}.
% The list can be set by \cs{GraphicsExtensions}.
% \begin{quote}
% |\AppendGraphicsExtensions{.gif}|\\
% or\\
% |\PrependGraphicsExtensions{.gif}|
% \end{quote}
%
% \StopEventually{
% }
%
% \section{Implementation}
%
%    \begin{macrocode}
%<*package>
%    \end{macrocode}
%
% \subsection{Wrapper package}
%
%    Reload check, especially if the package is not used with \LaTeX.
%    \begin{macrocode}
\begingroup\catcode61\catcode48\catcode32=10\relax%
  \catcode13=5 % ^^M
  \endlinechar=13 %
  \catcode35=6 % #
  \catcode39=12 % '
  \catcode44=12 % ,
  \catcode45=12 % -
  \catcode46=12 % .
  \catcode58=12 % :
  \catcode64=11 % @
  \catcode123=1 % {
  \catcode125=2 % }
  \expandafter\let\expandafter\x\csname ver@epstopdf.sty\endcsname
  \ifx\x\relax % plain-TeX, first loading
  \else
    \def\empty{}%
    \ifx\x\empty % LaTeX, first loading,
      % variable is initialized, but \ProvidesPackage not yet seen
    \else
      \expandafter\ifx\csname PackageInfo\endcsname\relax
        \def\x#1#2{%
          \immediate\write-1{Package #1 Info: #2.}%
        }%
      \else
        \def\x#1#2{\PackageInfo{#1}{#2, stopped}}%
      \fi
      \x{epstopdf}{The package is already loaded}%
      \aftergroup\endinput
    \fi
  \fi
\endgroup%
%    \end{macrocode}
%    Package identification:
%    \begin{macrocode}
\begingroup\catcode61\catcode48\catcode32=10\relax%
  \catcode13=5 % ^^M
  \endlinechar=13 %
  \catcode35=6 % #
  \catcode39=12 % '
  \catcode40=12 % (
  \catcode41=12 % )
  \catcode44=12 % ,
  \catcode45=12 % -
  \catcode46=12 % .
  \catcode47=12 % /
  \catcode58=12 % :
  \catcode64=11 % @
  \catcode91=12 % [
  \catcode93=12 % ]
  \catcode123=1 % {
  \catcode125=2 % }
  \expandafter\ifx\csname ProvidesPackage\endcsname\relax
    \def\x#1#2#3[#4]{\endgroup
      \immediate\write-1{Package: #3 #4}%
      \xdef#1{#4}%
    }%
  \else
    \def\x#1#2[#3]{\endgroup
      #2[{#3}]%
      \ifx#1\@undefined
        \xdef#1{#3}%
      \fi
      \ifx#1\relax
        \xdef#1{#3}%
      \fi
    }%
  \fi
\expandafter\x\csname ver@epstopdf.sty\endcsname
\ProvidesPackage{epstopdf}%
  [2016/05/15 v2.6 Conversion with epstopdf on the fly (HO)]%
%    \end{macrocode}
%    Larger catcode set because of configuration files needed.
%    \begin{macrocode}
\begingroup\catcode61\catcode48\catcode32=10\relax%
  \catcode13=5 % ^^M
  \endlinechar=13 %
  \catcode123=1 % {
  \catcode125=2 % }
  \catcode64=11 % @
  \def\x{\endgroup
    \expandafter\edef\csname ETE@AtEnd\endcsname{%
      \endlinechar=\the\endlinechar\relax
      \catcode13=\the\catcode13\relax
      \catcode32=\the\catcode32\relax
      \catcode35=\the\catcode35\relax
      \catcode61=\the\catcode61\relax
      \catcode64=\the\catcode64\relax
      \catcode123=\the\catcode123\relax
      \catcode125=\the\catcode125\relax
    }%
  }%
\x\catcode61\catcode48\catcode32=10\relax%
\catcode13=5 % ^^M
\endlinechar=13 %
\catcode35=6 % #
\catcode64=11 % @
\catcode123=1 % {
\catcode125=2 % }
\def\TMP@EnsureCode#1#2{%
  \edef\ETE@AtEnd{%
    \ETE@AtEnd
    \catcode#1=\the\catcode#1\relax
  }%
  \catcode#1=#2\relax
}
\TMP@EnsureCode{39}{12}% '
\TMP@EnsureCode{40}{12}% (
\TMP@EnsureCode{41}{12}% )
\TMP@EnsureCode{42}{12}% *
\TMP@EnsureCode{45}{12}% -
\TMP@EnsureCode{47}{12}% /
\TMP@EnsureCode{91}{12}% [
\TMP@EnsureCode{93}{12}% ]
\edef\ETE@AtEnd{\ETE@AtEnd\noexpand\endinput}
%    \end{macrocode}
%
%    \begin{macrocode}
\let\ETE@SavedAtEnd\ETE@AtEnd
\RequirePackage{epstopdf-base}[2016/05/15]
\let\ETE@AtEnd\ETE@SavedAtEnd
%    \end{macrocode}
%
% \subsubsection{Option handling}
%
%    \begin{macrocode}
\DeclareOption*{%
  \expandafter\epstopdfsetup\expandafter{\CurrentOption}%
}
\ProcessOptions*\relax
%    \end{macrocode}
%
%    \begin{macrocode}
\ETE@AtEnd%
%</package>
%    \end{macrocode}
%
% \subsection{Base package}
%
%    \begin{macrocode}
%<*base>
%    \end{macrocode}
%
% \subsection{Preparations}
%
% \subsubsection{Relead check and identification}
%
%    Reload check, especially if the package is not used with \LaTeX.
%    \begin{macrocode}
\begingroup\catcode61\catcode48\catcode32=10\relax%
  \catcode13=5 % ^^M
  \endlinechar=13 %
  \catcode35=6 % #
  \catcode39=12 % '
  \catcode44=12 % ,
  \catcode45=12 % -
  \catcode46=12 % .
  \catcode58=12 % :
  \catcode64=11 % @
  \catcode123=1 % {
  \catcode125=2 % }
  \expandafter\let\expandafter\x\csname ver@epstopdf-base.sty\endcsname
  \ifx\x\relax % plain-TeX, first loading
  \else
    \def\empty{}%
    \ifx\x\empty % LaTeX, first loading,
      % variable is initialized, but \ProvidesPackage not yet seen
    \else
      \expandafter\ifx\csname PackageInfo\endcsname\relax
        \def\x#1#2{%
          \immediate\write-1{Package #1 Info: #2.}%
        }%
      \else
        \def\x#1#2{\PackageInfo{#1}{#2, stopped}}%
      \fi
      \x{epstopdf-base}{The package is already loaded}%
      \aftergroup\endinput
    \fi
  \fi
\endgroup%
%    \end{macrocode}
%    Package identification:
%    \begin{macrocode}
\begingroup\catcode61\catcode48\catcode32=10\relax%
  \catcode13=5 % ^^M
  \endlinechar=13 %
  \catcode35=6 % #
  \catcode39=12 % '
  \catcode40=12 % (
  \catcode41=12 % )
  \catcode44=12 % ,
  \catcode45=12 % -
  \catcode46=12 % .
  \catcode47=12 % /
  \catcode58=12 % :
  \catcode64=11 % @
  \catcode91=12 % [
  \catcode93=12 % ]
  \catcode123=1 % {
  \catcode125=2 % }
  \expandafter\ifx\csname ProvidesPackage\endcsname\relax
    \def\x#1#2#3[#4]{\endgroup
      \immediate\write-1{Package: #3 #4}%
      \xdef#1{#4}%
    }%
  \else
    \def\x#1#2[#3]{\endgroup
      #2[{#3}]%
      \ifx#1\@undefined
        \xdef#1{#3}%
      \fi
      \ifx#1\relax
        \xdef#1{#3}%
      \fi
    }%
  \fi
\expandafter\x\csname ver@epstopdf-base.sty\endcsname
\ProvidesPackage{epstopdf-base}%
  [2016/05/15 v2.6 Base part for package epstopdf]%
%    \end{macrocode}
%
% \subsubsection{Catcodes}
%
%    \begin{macrocode}
\begingroup\catcode61\catcode48\catcode32=10\relax%
  \catcode13=5 % ^^M
  \endlinechar=13 %
  \catcode123=1 % {
  \catcode125=2 % }
  \catcode64=11 % @
  \def\x{\endgroup
    \expandafter\edef\csname ETE@AtEnd\endcsname{%
      \endlinechar=\the\endlinechar\relax
      \catcode13=\the\catcode13\relax
      \catcode32=\the\catcode32\relax
      \catcode35=\the\catcode35\relax
      \catcode61=\the\catcode61\relax
      \catcode64=\the\catcode64\relax
      \catcode123=\the\catcode123\relax
      \catcode125=\the\catcode125\relax
    }%
  }%
\x\catcode61\catcode48\catcode32=10\relax%
\catcode13=5 % ^^M
\endlinechar=13 %
\catcode35=6 % #
\catcode64=11 % @
\catcode123=1 % {
\catcode125=2 % }
\def\TMP@EnsureCode#1#2{%
  \edef\ETE@AtEnd{%
    \ETE@AtEnd
    \catcode#1=\the\catcode#1\relax
  }%
  \catcode#1=#2\relax
}
\TMP@EnsureCode{33}{12}% !
\TMP@EnsureCode{39}{12}% '
\TMP@EnsureCode{42}{12}% *
\TMP@EnsureCode{44}{12}% ,
\TMP@EnsureCode{45}{12}% -
\TMP@EnsureCode{46}{12}% .
\TMP@EnsureCode{47}{12}% /
\TMP@EnsureCode{58}{12}% :
\TMP@EnsureCode{60}{12}% <
\TMP@EnsureCode{62}{12}% >
\TMP@EnsureCode{96}{12}% `
\edef\ETE@AtEnd{\ETE@AtEnd\noexpand\endinput}
%    \end{macrocode}
%
% \subsubsection{Load packages}
%
%    \begin{macrocode}
\RequirePackage{infwarerr}[2007/09/09]
\RequirePackage{grfext}\relax
\RequirePackage{kvoptions}[2007/10/02]
\RequirePackage{pdftexcmds}[2007/11/11]
%    \end{macrocode}
%
% \subsection{Checks}
%
% Check, whether package graphics is loaded
% (also graphicx loads graphics).
% Because miniltx.tex does not know \cs{@ifpackageloaded}
% we test for \cs{Gin@setfile} instead.
%    \begin{macrocode}
\begingroup\expandafter\expandafter\expandafter\endgroup
\expandafter\ifx\csname Gin@setfile\endcsname\relax
  \@PackageWarningNoLine{epstopdf}{%
    No graphics package \string`graphic{s,x}\string' loaded%
  }%
  \newcommand*{\epstopdfsetup}[1]{}%
  \expandafter\ETE@AtEnd
\fi%
%    \end{macrocode}
%
% Check, whether pdftex.def is loaded. \cs{ver@pdftex.def} is not
% available with miniltx.tex, thus we test for \cs{Gin@driver}.
%    \begin{macrocode}
\begingroup
  \def\x{luatex.def}%
  \ifx\Gin@driver\x
  \else
  \def\x{pdftex.def}%
  \ifx\Gin@driver\x
  \else
    \@PackageWarningNoLine{epstopdf}{%
      Drivers other than `pdftex' and `luatex' are not supported%
    }%
    \endgroup
    \newcommand*{\epstopdfsetup}[1]{}%
    \expandafter\ETE@AtEnd
  \fi%
  \fi%
\endgroup
%    \end{macrocode}
%
% Check, whether the shell escape feature is enabled.
%
%    \begin{macrocode}
\begingroup
  \expandafter\ifx\csname pdf@shellescape\endcsname\relax
  \else
    \ifnum\pdf@shellescape>0 %
    \else
      \@PackageWarningNoLine{epstopdf}{%
        Shell escape feature is not enabled%
      }%
    \fi
  \fi
\endgroup
%    \end{macrocode}
%
% \subsection{Package loading}
%
%    \begin{macrocode}
%    \end{macrocode}
%
% \subsection{Options}
%
%    \begin{macrocode}
\SetupKeyvalOptions{family=ETE,prefix=ETE@}
\DeclareBoolOption{update}
\DeclareBoolOption{verbose}
\newif\ifETE@prepend
\DeclareVoidOption{prepend}{\ETE@prependtrue}
\DeclareVoidOption{append}{\ETE@prependfalse}
\DeclareStringOption{outdir}
\DeclareStringOption{suffix}
\DeclareBoolOption{prefersuffix}
\DeclareStringOption{program@epstopdf}
%    \end{macrocode}
%    Options \xoption{disable} and \xoption{enable} are for testing
%    only. Therefore they are not documented on purpose.
%    \begin{macrocode}
\DeclareBoolOption{disable}
\DeclareComplementaryOption{enable}{disable}
%    \end{macrocode}
%    \begin{macrocode}
\newcommand*{\epstopdfsetup}{\setkeys{ETE}}
%    \end{macrocode}
%
% \subsubsection{Default setting}
%
%    \begin{macrocode}
\epstopdfsetup{%
  verbose,%
  enable,%
  append,%
  update,%
  prefersuffix,%
  suffix=-\SourceExt-converted-to,%
  program@epstopdf=epstopdf%
}
%    \end{macrocode}
%
% \subsection{Make and verbose}
%
%    \begin{macrocode}
\begingroup\expandafter\expandafter\expandafter\endgroup
\expandafter\ifx\csname pdf@filemoddate\endcsname\relax
  \def\ETE@Make#1#2{%
    \ifETE@update
      \ETE@WarnModDate
    \fi
    \@firstofone
  }%
  \def\ETE@WarnModDate{%
    \@PackageWarningNoLine{epstopdf}{%
      \string\pdffilemoddate\space is not available,\MessageBreak
      option `update' will be ignored%
    }%
    \global\let\ETE@WarnModDate\relax
  }%
  \def\ETE@FileInfo#1#2{#1 file: <#2>}%
\else
  \def\ETE@Make#1#2{%
    \ifETE@update
      \ifnum\pdf@strcmp{\pdf@filemoddate{#1}}{\pdf@filemoddate{#2}}>0 %
        \expandafter\expandafter\expandafter\@firstofone
      \else
        \@PackageInfoNoLine{epstopdf}{%
          Output file is already uptodate%
        }%
        \expandafter\expandafter\expandafter\@gobble
      \fi
    \else
      \expandafter\@firstofone
    \fi
  }%
  \def\ETE@FileInfo#1#2{%
    #1 file: <#2>%
    \expandafter\expandafter\expandafter
    \ETE@Date\pdf@filemoddate{#2}\@nil
    \expandafter\expandafter\expandafter
    \ETE@Size\pdf@filesize{#2}\@nil
  }%
  \def\ETE@Date#1\@nil{%
    \ifx\\#1\\%
    \else
      \ETE@@Date#1\@nil
    \fi
  }%
  \def\ETE@@Date#1:#2#3#4#5#6#7#8#9{%
    \MessageBreak
    \@spaces\space\space\space date: #2#3#4#5-#6#7-#8#9 %
    \ETE@@Time
  }%
  \def\ETE@@Time#1#2#3#4#5#6#7\@nil{%
    #1#2:#3#4:#5#6%
  }%
  \def\ETE@Size#1\@nil{%
    \ifx\\#1\\%
    \else
      \MessageBreak
      \@spaces\space\space\space size: #1 bytes%
    \fi
  }%
\fi
%    \end{macrocode}
%
% \subsection{Adding conversion support}
%
%    Patch \cs{Gin@setfile} to execute |#3|, if it contains
%    a command.
%    \begin{macrocode}
\expandafter\ifx\csname ETE@OrgGin@setfile\endcsname\relax
  \let\ETE@OrgGin@setfile\Gin@setfile
\else
  \@PackageError{epstopdf}{%
    Command \string\ETE@OrgGin@setfile\space
    already defined.\MessageBreak
  }{%
    Probably some package has included the code of this package%
    \MessageBreak
    instead of using \string\RequirePackage{epstopdf}.%
    \MessageBreak
    \@ehc
  }%
\fi
\def\ETE@IfFileExists{%
  \begingroup\expandafter\expandafter\expandafter\endgroup
  \expandafter\ifx\csname grffile@IfFileExists\endcsname\relax
    \expandafter\IfFileExists
  \else
    \global\let\ETE@IfFileExists\grffile@IfFileExists
    \expandafter\grffile@IfFileExists
  \fi
}
\def\ETE@Skip#1#2\x\fi\fi{%
  \fi
  \fi
  \endgroup
  \fi
  \fi
  #1%
}
\newif\ifETE@InsideSetfile
\newcommand*{\epstopdfcall}[1]{%
  \ifETE@InsideSetfile
    \expandafter\@firstoftwo
  \else
    \expandafter\@secondoftwo
  \fi
  {`#1}%
  {\Gin@base\Gin@ext}%
}
\def\ETE@DefCommandLine#1{%
  \edef\CommandLine{\expandafter\fi\if`#1}%
}
\def\ETE@DefX#1{%
  \expandafter\expandafter\expandafter\def
  \expandafter\expandafter\expandafter\x
  \expandafter\expandafter\expandafter{%
    \expandafter\fi\if`#1\relax\else
  }%
}
\def\ETE@Gin@setfile#1#2#3{%
  \ifETE@disable
    \ETE@OrgGin@setfile{#1}{#2}{#3}%
  \else
    \begingroup
      \ETE@InsideSetfiletrue
      \ETE@DefX{#3}%
    \expandafter\endgroup
    \ifx\x\@empty
      \ETE@OrgGin@setfile{#1}{#2}{#3}%
    \else
      \begingroup
        \ETE@InsideSetfiletrue
        \def\GraphicsType{#1}%
        \def\GraphicsRead{#2}%
        \ifETE@prefersuffix
        \else
          \ifx\ETE@suffix\@empty
          \else
            \ETE@IfFileExists{\Gin@base\GraphicsRead}{%
              \ETE@Skip{%
                \ETE@OrgGin@setfile{#1}{#2}{\Gin@base#2}%
              }%
            }{%
              \let\next\relax
            }%
            \next
          \fi
        \fi
        \ifx\Gin@ext\relax
          \let\SourceExt\Gin@eext
          \def\SourceFile{\Gin@base\Gin@eext}%
        \else
          \let\SourceExt\Gin@ext
          \def\SourceFile{\Gin@base\Gin@ext}%
        \fi
        \edef\SourceExt{% remove dot
          \expandafter\@cdr\SourceExt\@empty\@nil
        }%
        \let\OutputDirectory\ETE@outdir
        \ifx\OutputDirectory\@empty
          \edef\OutputFile{\ETE@GenerateName{\Gin@base}{#2}}%
        \else
          \begingroup
            \filename@parse{\Gin@base#2}%
            \edef\x{\endgroup
              \def\noexpand\OutputFile{%
                \ETE@GenerateName{%
                  \OutputDirectory\filename@base
                }{#2}%
              }%
            }%
          \x
        \fi
        \ETE@DefCommandLine{#3}%
        \ifETE@verbose
          \@PackageInfo{epstopdf}{%
            \ETE@FileInfo{Source}\SourceFile\MessageBreak
            \ETE@FileInfo{Output}\OutputFile\MessageBreak
            Command: <\CommandLine>\MessageBreak
            \string\includegraphics
          }%
        \fi
        \ETE@Make\SourceFile\OutputFile{%
          \pdf@system{\CommandLine}%
          \ifETE@verbose
            \@PackageInfoNoLine{epstopdf}{%
              \ETE@FileInfo{Result}\OutputFile
            }%
          \fi
        }%
        \edef\x{\endgroup
          \ifx\OutputDirectory\@empty
          \else
            \def\noexpand\Gin@base{%
              \OutputDirectory\noexpand\filename@base
            }%
          \fi
          \ifx\ETE@suffix\@empty
          \else
            \edef\noexpand\Gin@base{%
              \noexpand\Gin@base\ETE@suffix
            }%
          \fi
          \noexpand\ETE@OrgGin@setfile{%
            \GraphicsType
          }{%
            \GraphicsRead
          }{%
            \OutputFile
          }%
        }%
      \x
    \fi
  \fi
}
\let\Gin@setfile\ETE@Gin@setfile
\def\ETE@GenerateName#1#2{%
  #1\ETE@suffix#2%
}
%    \end{macrocode}
%
% \subsection{Declare graphics rule}
%
%    \begin{macrocode}
\newcommand*{\epstopdfDeclareGraphicsRule}[4]{%
  \ifx\\#4\\%
    \@PackageError{epstopdf-base}{%
      Conversion command is missing%
    }\@ehc
  \else
    \begingroup
      \@ifundefined{Gin@rule@#1}{%
      }{%
        \@PackageInfo{epstopdf-base}{%
          Redefining graphics rule for `#1'%
        }%
      }%
    \endgroup
    \@namedef{Gin@rule@#1}##1{{#2}{#3}{\epstopdfcall{#4}}}%
  \fi
}
%    \end{macrocode}
%
% \cs{DeclareGraphicsRule} for .eps
%    \begin{macrocode}
\epstopdfDeclareGraphicsRule{.eps}{pdf}{.pdf}{%
  \ETE@epstopdf{#1}%
}
\def\ETE@epstopdf#1{%
  \ETE@program@epstopdf\space
  \ifcase\ifx\OutputDirectory\@empty
           \ifx\ETE@suffix\@empty
             1%
           \fi
         \fi
         0 %
    --outfile=\OutputFile\space
  \fi
  #1%
}
%    \end{macrocode}
%
%    \begin{macrocode}
\ifETE@prepend
  \expandafter\PrependGraphicsExtensions
\else
  \expandafter\AppendGraphicsExtensions
\fi
{.eps}
\let\ETE@prepend\@undefined
\DeclareVoidOption{prepend}{%
  \PrependGraphicsExtensions{.eps}%
}
\let\ETE@append\@undefined
\DeclareVoidOption{append}{%
  \AppendGraphicsExtensions{.eps}%
}
%    \end{macrocode}
%
%    \begin{macrocode}
\InputIfFileExists{epstopdf-sys.cfg}{}{}
\InputIfFileExists{epstopdf.cfg}{}{}
%    \end{macrocode}
%
%    \begin{macrocode}
\ETE@AtEnd%
%</base>
%    \end{macrocode}
%
% \section{Test}
%
% \subsection{Preface for standard catcode check}
%
%    \begin{macrocode}
%<*test1>
\input miniltx.tex\relax
\def\Gin@driver{pdftex.def}
\input graphicx.sty\relax
\resetatcatcode
%</test1>
%    \end{macrocode}
%
% \subsection{Catcode checks for loading}
%
%    \begin{macrocode}
%<*test1>
%    \end{macrocode}
%    \begin{macrocode}
\catcode`\{=1 %
\catcode`\}=2 %
\catcode`\#=6 %
\catcode`\@=11 %
\expandafter\ifx\csname count@\endcsname\relax
  \countdef\count@=255 %
\fi
\expandafter\ifx\csname @gobble\endcsname\relax
  \long\def\@gobble#1{}%
\fi
\expandafter\ifx\csname @firstofone\endcsname\relax
  \long\def\@firstofone#1{#1}%
\fi
\expandafter\ifx\csname loop\endcsname\relax
  \expandafter\@firstofone
\else
  \expandafter\@gobble
\fi
{%
  \def\loop#1\repeat{%
    \def\body{#1}%
    \iterate
  }%
  \def\iterate{%
    \body
      \let\next\iterate
    \else
      \let\next\relax
    \fi
    \next
  }%
  \let\repeat=\fi
}%
\def\RestoreCatcodes{}
\count@=0 %
\loop
  \edef\RestoreCatcodes{%
    \RestoreCatcodes
    \catcode\the\count@=\the\catcode\count@\relax
  }%
\ifnum\count@<255 %
  \advance\count@ 1 %
\repeat

\def\RangeCatcodeInvalid#1#2{%
  \count@=#1\relax
  \loop
    \catcode\count@=15 %
  \ifnum\count@<#2\relax
    \advance\count@ 1 %
  \repeat
}
\def\RangeCatcodeCheck#1#2#3{%
  \count@=#1\relax
  \loop
    \ifnum#3=\catcode\count@
    \else
      \errmessage{%
        Character \the\count@\space
        with wrong catcode \the\catcode\count@\space
        instead of \number#3%
      }%
    \fi
  \ifnum\count@<#2\relax
    \advance\count@ 1 %
  \repeat
}
\def\space{ }
\expandafter\ifx\csname LoadCommand\endcsname\relax
  \def\LoadCommand{\input epstopdf.sty\relax}%
\fi
\def\Test{%
  \RangeCatcodeInvalid{0}{47}%
  \RangeCatcodeInvalid{58}{64}%
  \RangeCatcodeInvalid{91}{96}%
  \RangeCatcodeInvalid{123}{255}%
  \catcode`\@=12 %
  \catcode`\\=0 %
  \catcode`\%=14 %
  \LoadCommand
  \RangeCatcodeCheck{0}{36}{15}%
  \RangeCatcodeCheck{37}{37}{14}%
  \RangeCatcodeCheck{38}{47}{15}%
  \RangeCatcodeCheck{48}{57}{12}%
  \RangeCatcodeCheck{58}{63}{15}%
  \RangeCatcodeCheck{64}{64}{12}%
  \RangeCatcodeCheck{65}{90}{11}%
  \RangeCatcodeCheck{91}{91}{15}%
  \RangeCatcodeCheck{92}{92}{0}%
  \RangeCatcodeCheck{93}{96}{15}%
  \RangeCatcodeCheck{97}{122}{11}%
  \RangeCatcodeCheck{123}{255}{15}%
  \RestoreCatcodes
}
\Test
\csname @@end\endcsname
\end
%    \end{macrocode}
%    \begin{macrocode}
%</test1>
%    \end{macrocode}
%
% \section{Installation}
%
% \subsection{Download}
%
% \paragraph{Package.} This package is available on
% CTAN\footnote{\url{http://ctan.org/pkg/epstopdf}}:
% \begin{description}
% \item[\CTAN{macros/latex/contrib/oberdiek/epstopdf.dtx}] The source file.
% \item[\CTAN{macros/latex/contrib/oberdiek/epstopdf.pdf}] Documentation.
% \end{description}
%
%
% \paragraph{Bundle.} All the packages of the bundle `oberdiek'
% are also available in a TDS compliant ZIP archive. There
% the packages are already unpacked and the documentation files
% are generated. The files and directories obey the TDS standard.
% \begin{description}
% \item[\CTAN{install/macros/latex/contrib/oberdiek.tds.zip}]
% \end{description}
% \emph{TDS} refers to the standard ``A Directory Structure
% for \TeX\ Files'' (\CTAN{tds/tds.pdf}). Directories
% with \xfile{texmf} in their name are usually organized this way.
%
% \subsection{Bundle installation}
%
% \paragraph{Unpacking.} Unpack the \xfile{oberdiek.tds.zip} in the
% TDS tree (also known as \xfile{texmf} tree) of your choice.
% Example (linux):
% \begin{quote}
%   |unzip oberdiek.tds.zip -d ~/texmf|
% \end{quote}
%
% \paragraph{Script installation.}
% Check the directory \xfile{TDS:scripts/oberdiek/} for
% scripts that need further installation steps.
% Package \xpackage{attachfile2} comes with the Perl script
% \xfile{pdfatfi.pl} that should be installed in such a way
% that it can be called as \texttt{pdfatfi}.
% Example (linux):
% \begin{quote}
%   |chmod +x scripts/oberdiek/pdfatfi.pl|\\
%   |cp scripts/oberdiek/pdfatfi.pl /usr/local/bin/|
% \end{quote}
%
% \subsection{Package installation}
%
% \paragraph{Unpacking.} The \xfile{.dtx} file is a self-extracting
% \docstrip\ archive. The files are extracted by running the
% \xfile{.dtx} through \plainTeX:
% \begin{quote}
%   \verb|tex epstopdf.dtx|
% \end{quote}
%
% \paragraph{TDS.} Now the different files must be moved into
% the different directories in your installation TDS tree
% (also known as \xfile{texmf} tree):
% \begin{quote}
% \def\t{^^A
% \begin{tabular}{@{}>{\ttfamily}l@{ $\rightarrow$ }>{\ttfamily}l@{}}
%   epstopdf.sty & tex/latex/oberdiek/epstopdf.sty\\
%   epstopdf-base.sty & tex/latex/oberdiek/epstopdf-base.sty\\
%   epstopdf.pdf & doc/latex/oberdiek/epstopdf.pdf\\
%   test/epstopdf-test1.tex & doc/latex/oberdiek/test/epstopdf-test1.tex\\
%   epstopdf.dtx & source/latex/oberdiek/epstopdf.dtx\\
% \end{tabular}^^A
% }^^A
% \sbox0{\t}^^A
% \ifdim\wd0>\linewidth
%   \begingroup
%     \advance\linewidth by\leftmargin
%     \advance\linewidth by\rightmargin
%   \edef\x{\endgroup
%     \def\noexpand\lw{\the\linewidth}^^A
%   }\x
%   \def\lwbox{^^A
%     \leavevmode
%     \hbox to \linewidth{^^A
%       \kern-\leftmargin\relax
%       \hss
%       \usebox0
%       \hss
%       \kern-\rightmargin\relax
%     }^^A
%   }^^A
%   \ifdim\wd0>\lw
%     \sbox0{\small\t}^^A
%     \ifdim\wd0>\linewidth
%       \ifdim\wd0>\lw
%         \sbox0{\footnotesize\t}^^A
%         \ifdim\wd0>\linewidth
%           \ifdim\wd0>\lw
%             \sbox0{\scriptsize\t}^^A
%             \ifdim\wd0>\linewidth
%               \ifdim\wd0>\lw
%                 \sbox0{\tiny\t}^^A
%                 \ifdim\wd0>\linewidth
%                   \lwbox
%                 \else
%                   \usebox0
%                 \fi
%               \else
%                 \lwbox
%               \fi
%             \else
%               \usebox0
%             \fi
%           \else
%             \lwbox
%           \fi
%         \else
%           \usebox0
%         \fi
%       \else
%         \lwbox
%       \fi
%     \else
%       \usebox0
%     \fi
%   \else
%     \lwbox
%   \fi
% \else
%   \usebox0
% \fi
% \end{quote}
% If you have a \xfile{docstrip.cfg} that configures and enables \docstrip's
% TDS installing feature, then some files can already be in the right
% place, see the documentation of \docstrip.
%
% \subsection{Refresh file name databases}
%
% If your \TeX~distribution
% (\teTeX, \mikTeX, \dots) relies on file name databases, you must refresh
% these. For example, \teTeX\ users run \verb|texhash| or
% \verb|mktexlsr|.
%
% \subsection{Some details for the interested}
%
% \paragraph{Attached source.}
%
% The PDF documentation on CTAN also includes the
% \xfile{.dtx} source file. It can be extracted by
% AcrobatReader 6 or higher. Another option is \textsf{pdftk},
% e.g. unpack the file into the current directory:
% \begin{quote}
%   \verb|pdftk epstopdf.pdf unpack_files output .|
% \end{quote}
%
% \paragraph{Unpacking with \LaTeX.}
% The \xfile{.dtx} chooses its action depending on the format:
% \begin{description}
% \item[\plainTeX:] Run \docstrip\ and extract the files.
% \item[\LaTeX:] Generate the documentation.
% \end{description}
% If you insist on using \LaTeX\ for \docstrip\ (really,
% \docstrip\ does not need \LaTeX), then inform the autodetect routine
% about your intention:
% \begin{quote}
%   \verb|latex \let\install=y\input{epstopdf.dtx}|
% \end{quote}
% Do not forget to quote the argument according to the demands
% of your shell.
%
% \paragraph{Generating the documentation.}
% You can use both the \xfile{.dtx} or the \xfile{.drv} to generate
% the documentation. The process can be configured by the
% configuration file \xfile{ltxdoc.cfg}. For instance, put this
% line into this file, if you want to have A4 as paper format:
% \begin{quote}
%   \verb|\PassOptionsToClass{a4paper}{article}|
% \end{quote}
% An example follows how to generate the
% documentation with pdf\LaTeX:
% \begin{quote}
%\begin{verbatim}
%pdflatex epstopdf.dtx
%makeindex -s gind.ist epstopdf.idx
%pdflatex epstopdf.dtx
%makeindex -s gind.ist epstopdf.idx
%pdflatex epstopdf.dtx
%\end{verbatim}
% \end{quote}
%
% \section{Catalogue}
%
% The following XML file can be used as source for the
% \href{http://mirror.ctan.org/help/Catalogue/catalogue.html}{\TeX\ Catalogue}.
% The elements \texttt{caption} and \texttt{description} are imported
% from the original XML file from the Catalogue.
% The name of the XML file in the Catalogue is \xfile{epstopdf-pkg.xml}.
%    \begin{macrocode}
%<*catalogue>
<?xml version='1.0' encoding='us-ascii'?>
<!DOCTYPE entry SYSTEM 'catalogue.dtd'>
<entry datestamp='$Date$' modifier='$Author$' id='epstopdf-pkg'>
  <name>epstopdf-pkg</name>
  <caption>Call epstopdf "on the fly"</caption>
  <authorref id='auth:oberdiek'/>
  <copyright owner='Heiko Oberdiek' year='2001,2006-2010'/>
  <license type='lppl1.3'/>
  <version number='2.6'/>
  <description>
    The package adds support for EPS files in the
    <xref refid='graphicx'>graphicx</xref> package when running under
    PDFTeX.  If an EPS graphic is detected, the package spawns a
    process to convert the EPS to PDF, using the script
    <xref refid='epstopdf'>epstopdf</xref>.  This of course requires
    that shell escape is enabled for the PDFTeX run.
    <p/>
    The package is part of the <xref refid='oberdiek'>oberdiek</xref>
    bundle.
  </description>
  <documentation details='Package documentation'
      href='ctan:/macros/latex/contrib/oberdiek/epstopdf.pdf'/>
  <ctan file='true' path='/macros/latex/contrib/oberdiek/epstopdf.dtx'/>
  <miktex location='oberdiek'/>
  <texlive location='oberdiek'/>
  <install path='/macros/latex/contrib/oberdiek/oberdiek.tds.zip'/>
</entry>
%</catalogue>
%    \end{macrocode}
%
% \begin{History}
%   \begin{Version}{2001/01/06 v1.0}
%   \item
%     First public version,
%     published in the \pdfTeX\ mailing list.
%   \end{Version}
%   \begin{Version}{2001/02/04 v1.1}
%   \item
%     Minor documentation update.
%   \item
%     CTAN.
%   \end{Version}
%   \begin{Version}{2006/02/20 v1.2}
%   \item
%     DTX framework.
%   \item
%     Compatibility for \xfile{miniltx.tex}.
%   \end{Version}
%   \begin{Version}{2006/08/26 v1.3}
%   \item
%     Check for \cs{write18} if available and print a warning
%     if the feature is not enabled.
%   \end{Version}
%   \begin{Version}{2007/04/26 v1.4}
%   \item
%     Documentation rewritten and extended.
%   \end{Version}
%   \begin{Version}{2007/10/02 v1.5}
%   \item
%     New option \xoption{update}: If the converted file exists, it will be only
%     converted if it is out of date.
%   \item
%     Updating the extension list is delegated to package \xpackage{grfext}.
%     Fine tuning is done by the new options \xoption{append}, \xoption{prepend}.
%   \item
%     New option \xoption{outdir} for changing the output directory.
%   \item
%     New option \xoption{verbose}.
%   \item
%     \cs{SourceFile} and \cs{OutputFile} introduced.
%   \item
%     Configuration file support added.
%   \end{Version}
%   \begin{Version}{2007/11/11 v1.6}
%   \item
%     Use of package \xpackage{pdftexcmds} for \LuaTeX\ support.
%   \end{Version}
%   \begin{Version}{2008/05/06 v1.7}
%   \item
%     Warning messages uses ``loaded'' instead of ``found''.
%   \end{Version}
%   \begin{Version}{2009/03/01 v1.8}
%   \item
%     Warning message for missing \xfile{pdftex.def} changed.
%   \end{Version}
%   \begin{Version}{2009/07/06 v1.9}
%   \item
%     Option \xoption{suffix} added.
%   \end{Version}
%   \begin{Version}{2009/07/07 v1.10}
%   \item
%     \cs{SourceExt} added.
%   \item
%     If option \xoption{suffix} is set, the inclusion of
%     an image without the suffix namespace is preferred over generating the
%     the image within the suffix namespace.
%   \end{Version}
%   \begin{Version}{2009/07/12 v2.0}
%   \item
%     New default settings.
%   \item
%     Package is split into \xpackage{epstopdf} that only takes package
%     options and loads \xpackage{epstopdf-base} that does the work.
%   \item
%     \cs{epstopdfDeclareGraphicsRule} and \cs{epstopdfcall} added.
%   \item
%     \xfile{epstopdf-sys.cfg} is loaded before \xfile{epstopdf.cfg}
%     if \xfile{epstopdf-sys.cfg} exists.
%   \end{Version}
%   \begin{Version}{2009/07/15 v2.1}
%   \item
%     Default setting: \xoption{verbose} is now turned on as the
%     documentation for v2.0 said.
%   \item
%     Documentation fixes.
%   \end{Version}
%   \begin{Version}{2009/07/16 v2.2}
%   \item
%     Fixed redefined \cs{Gin@setfile}.
%   \item
%     Documentation extended for package options.
%   \end{Version}
%   \begin{Version}{2009/09/24 v2.3}
%   \item
%     Bug fix for the case that both option \xoption{suffix} and
%     \xoption{outdir} are used.
%   \end{Version}
%   \begin{Version}{2009/10/17 v2.4}
%   \item
%     The name of the program `epstopdf' can be configured via
%     the new option \xoption{program@epstopdf}.
%   \end{Version}
%   \begin{Version}{2016/05/15 v2.5}
%   \item
%     Wording of warning message fixed (Karl Berry).
%   \item
%     \cs{ETE@Gin@setfile} added (Karl Berry).
%   \end{Version}
%   \begin{Version}{2016/05/15 v2.6}
%   \item
%     luaTeX compatibility
%   \end{Version}
% \end{History}
%
% \PrintIndex
%
% \Finale
\endinput

%        (quote the arguments according to the demands of your shell)
%
% Documentation:
%    (a) If epstopdf.drv is present:
%           latex epstopdf.drv
%    (b) Without epstopdf.drv:
%           latex epstopdf.dtx; ...
%    The class ltxdoc loads the configuration file ltxdoc.cfg
%    if available. Here you can specify further options, e.g.
%    use A4 as paper format:
%       \PassOptionsToClass{a4paper}{article}
%
%    Programm calls to get the documentation (example):
%       pdflatex epstopdf.dtx
%       makeindex -s gind.ist epstopdf.idx
%       pdflatex epstopdf.dtx
%       makeindex -s gind.ist epstopdf.idx
%       pdflatex epstopdf.dtx
%
% Installation:
%    TDS:tex/latex/oberdiek/epstopdf.sty
%    TDS:tex/latex/oberdiek/epstopdf-base.sty
%    TDS:doc/latex/oberdiek/epstopdf.pdf
%    TDS:doc/latex/oberdiek/test/epstopdf-test1.tex
%    TDS:source/latex/oberdiek/epstopdf.dtx
%
%<*ignore>
\begingroup
  \catcode123=1 %
  \catcode125=2 %
  \def\x{LaTeX2e}%
\expandafter\endgroup
\ifcase 0\ifx\install y1\fi\expandafter
         \ifx\csname processbatchFile\endcsname\relax\else1\fi
         \ifx\fmtname\x\else 1\fi\relax
\else\csname fi\endcsname
%</ignore>
%<*install>
\input docstrip.tex
\Msg{************************************************************************}
\Msg{* Installation}
\Msg{* Package: epstopdf 2016/05/15 v2.6 Conversion with epstopdf on the fly (HO)}
\Msg{************************************************************************}

\keepsilent
\askforoverwritefalse

\let\MetaPrefix\relax
\preamble

This is a generated file.

Project: epstopdf
Version: 2016/05/15 v2.6

Copyright (C) 2001, 2006-2010 by
   Heiko Oberdiek <heiko.oberdiek at googlemail.com>

This work may be distributed and/or modified under the
conditions of the LaTeX Project Public License, either
version 1.3c of this license or (at your option) any later
version. This version of this license is in
   http://www.latex-project.org/lppl/lppl-1-3c.txt
and the latest version of this license is in
   http://www.latex-project.org/lppl.txt
and version 1.3 or later is part of all distributions of
LaTeX version 2005/12/01 or later.

This work has the LPPL maintenance status "maintained".

This Current Maintainer of this work is Heiko Oberdiek.

This work consists of the main source file epstopdf.dtx
and the derived files
   epstopdf.sty, epstopdf.pdf, epstopdf.ins, epstopdf.drv,
   epstopdf-base.sty, epstopdf-test1.tex.

\endpreamble
\let\MetaPrefix\DoubleperCent

\generate{%
  \file{epstopdf.ins}{\from{epstopdf.dtx}{install}}%
  \file{epstopdf.drv}{\from{epstopdf.dtx}{driver}}%
  \usedir{tex/latex/oberdiek}%
  \file{epstopdf.sty}{\from{epstopdf.dtx}{package}}%
  \file{epstopdf-base.sty}{\from{epstopdf.dtx}{base}}%
  \usedir{doc/latex/oberdiek/test}%
  \file{epstopdf-test1.tex}{\from{epstopdf.dtx}{test1}}%
  \nopreamble
  \nopostamble
  \usedir{source/latex/oberdiek/catalogue}%
  \file{epstopdf-pkg.xml}{\from{epstopdf.dtx}{catalogue}}%
}

\catcode32=13\relax% active space
\let =\space%
\Msg{************************************************************************}
\Msg{*}
\Msg{* To finish the installation you have to move the following}
\Msg{* files into a directory searched by TeX:}
\Msg{*}
\Msg{*     epstopdf.sty, epstopdf-base.sty}
\Msg{*}
\Msg{* To produce the documentation run the file `epstopdf.drv'}
\Msg{* through LaTeX.}
\Msg{*}
\Msg{* Happy TeXing!}
\Msg{*}
\Msg{************************************************************************}

\endbatchfile
%</install>
%<*ignore>
\fi
%</ignore>
%<*driver>
\NeedsTeXFormat{LaTeX2e}
\ProvidesFile{epstopdf.drv}%
  [2016/05/15 v2.6 Conversion with epstopdf on the fly (HO)]%
\documentclass{ltxdoc}
\usepackage{holtxdoc}[2011/11/22]
\begin{document}
  \DocInput{epstopdf.dtx}%
\end{document}
%</driver>
% \fi
%
%
% \CharacterTable
%  {Upper-case    \A\B\C\D\E\F\G\H\I\J\K\L\M\N\O\P\Q\R\S\T\U\V\W\X\Y\Z
%   Lower-case    \a\b\c\d\e\f\g\h\i\j\k\l\m\n\o\p\q\r\s\t\u\v\w\x\y\z
%   Digits        \0\1\2\3\4\5\6\7\8\9
%   Exclamation   \!     Double quote  \"     Hash (number) \#
%   Dollar        \$     Percent       \%     Ampersand     \&
%   Acute accent  \'     Left paren    \(     Right paren   \)
%   Asterisk      \*     Plus          \+     Comma         \,
%   Minus         \-     Point         \.     Solidus       \/
%   Colon         \:     Semicolon     \;     Less than     \<
%   Equals        \=     Greater than  \>     Question mark \?
%   Commercial at \@     Left bracket  \[     Backslash     \\
%   Right bracket \]     Circumflex    \^     Underscore    \_
%   Grave accent  \`     Left brace    \{     Vertical bar  \|
%   Right brace   \}     Tilde         \~}
%
% \GetFileInfo{epstopdf.drv}
%
% \title{The \xpackage{epstopdf} package}
% \date{2016/05/15 v2.6}
% \author{Heiko Oberdiek\thanks
% {Please report any issues at https://github.com/ho-tex/oberdiek/issues}\\
% \xemail{heiko.oberdiek at googlemail.com}}
%
% \maketitle
%
% \begin{abstract}
% This packages adds support of handling eps images
% to package \xpackage{graphics} or \xpackage{graphicx}
% with option \xoption{pdftex}. If an eps image is detected,
% epstopdf is automatically called to convert it to pdf format.
% \end{abstract}
%
% \tableofcontents
%
% \providecommand*{\xprogram}[1]{\textsf{#1}}
%
% \section{Documentation}
%
% \subsection{Introduction}
%
% \LaTeX\ provides its graphics bundle to include graphics files.
% Both packages \xpackage{graphics} or \xpackage{graphicx} may
% be used. the latter one loads the first and adds options
% in key value style for \cs{includegraphics}.
%
% Usually the drivers do not support all kind of graphics files.
% Other image types must be converted, before they become
% usuable. In case of driver \xprogram{dvips}, the graphics
% rule may contain a conversion rule. Then all that package
% \xpackage{graphics} must know is the bounding box, the command
% is passed to \xprogram{dvips} that calls it and embeds the
% converted image.
%
% However, \pdfTeX\ has its driver for PDF output already
% build in. It's graphics inclusion commands (\cs{pdfximage})
% does not allow the execution of external commands.
% Therefore commands in the last argument of
% \cs{DeclareGraphicsRule} were of no use.
% But external programs can be called within \pdfTeX.
% This feature is called ``shell escape'' or ``write 18''
% and must usually enabled explicitely because of security reasons.
% Now, this package \xpackage{epstopdf} hooks
% into package \xpackage{graphics}' code to catch
% that argument with the external command and executes it
% to convert the graphics file to a supported format and
% passes the control of graphics inclusion back to package
% \xpackage{graphics}.
%
% \subsection{Requirements}
%
% \begin{itemize}
% \item
%   The feature \cs{write18} must be enabled. This allows the
%   running of external programs during \TeX's compile run.
%   Keep in mind that this is a security risk. The feature is an
%   addition to \cs{TeX}. Mik\TeX, te\TeX, \TeX\ Live support it.
%   In Web2C based \TeX\ distributions (te\TeX, \TeX\ Live) it can be
%   enabled in the configuration file \xfile{texmf.cnf}:
%   \begin{quote}
%     |shell_escape = 1|
%   \end{quote}
%   Because of the security risk, it is better to do it on the
%   command line only:
%   \begin{quote}
%   \texttt{--shell-escape} (te\TeX, \TeX\ Live)\\
%   \texttt{--enable-write18} (MiK\TeX)
%   \end{quote}
%   Example:
%   \begin{quote}
%     |pdflatex -shell-escape test.tex|
%   \end{quote}
% \item
%   The program \xprogram{epstopdf} for the conversion from EPS
%   to PDF. However, other programs can
%   be used and configured by \cs{DeclareGraphicsRule}.
%   Example:
%   \begin{quote}
%   |\epstopdfDeclareGraphicsRule{.eps}{pdf}{.pdf}{%|\\
%   |  ps2pdf -dEPSCrop #1 \OutputFile|\\
%   |}|
%   \end{quote}
% \end{itemize}
%
% \subsection{Usage}
%
% The package is loaded after graphic\{s,x\}, e.g.:
% \begin{quote}
%   |\usepackage[pdftex]{graphicx}|\\
%   |\usepackage{epstopdf}|
% \end{quote}
% Now images with file name extension \xfile{.eps} are detected
% and supported using \cs{includegraphics}.
%
% If the graphics file name is explicitly specified with extension
% \xfile{.eps} the new rule for EPS files is called and
% the conversion performed. If option \xoption{update} is in force
% then the conversion step is dropped if the target file already
% exists and is not older then the EPS file.
%
% The situation is more complicate if the graphics file is
% given without file name extension. Then the \xpackage{graphics}
% package must search for a supported image file.
% The possible extensions are stored in the graphics extension
% list, that can be set by \cs{DeclareGraphicsExtensions}.
% The algorithm:
% \begin{quote}
%    \def\>{\mbox{}\quad}
%    function search( \meta{filebase} )\\
%    \>foreach \meta{ext} in \meta{graphics extensions}\\
%    \>\>foreach \meta{dir} in \meta{current directory}, \meta{\cs{graphicspath}}\\
%    \>\>\>\meta{file} := \meta{dir} + \meta{filebase} + \meta{ext}\\
%    \>\>\>if exist \meta{file}\\
%    \>\>\>\>return found\\
%    \>return not found
% \end{quote}
% Package \xpackage{epstopdf} puts \xfile{.eps} at the end of the
% graphics extension search list. This is the behaviour of option
% \xoption{append} that is enabled by default.
% That means, the conversion is called last
% unless a supported file type cannot be found earlier. This avoids
% unnecessary conversion steps that slow down the \LaTeX\ run.
% If you want to use option \xoption{update} and your \pdfTeX\ supports
% it, then an outdated PDF file also would be found earlier unless
% \xoption{suffix} is used that is the default since version 2.0.
%
% With an empty option \xoption{suffix} and option
% \xoption{prepend} there is a risk that an original PDF file is overwritten:
%
% If the original image file is the PDF file and there is also a generated
% EPS file, then the original PDF file can be regenerated (depending
% on the option settings) and the original PDF file gets lost.
% Therefore option \xoption{suffix} is introduced in version 1.9
% to create a separate name space for generated output files.
%
% \textbf{Note:} Usually the conversion program needs the exact location
% of the image file. Usually the current directory works.
% Also if the image file is found using \cs{graphicspath}, the
% location is known. However, if the image is somewhere
% in a directory of environment variable |TEXINPUTS|, then
% the package does not know the exact location and the
% conversion program will not find the image file unless it
% implements a search using |TEXINPUTS| (program \xprogram{kpsewhich}
% may be of help in this task).
%
% \subsection{Options}
%
% Options can be given as package options or later using:
% \begin{declcs}{epstopdfsetup} \M{key value list}
% \end{declcs}
% \LaTeX\ expands the option list before passing the option list
% to the package's option handling code. This can fail
% for option \xoption{suffix} if it contains some of the macros
% described below. Use \cs{epstopdfsetup} after the package is loaded.
% Or load package \xpackage{kvoptions-patch} before. This package
% is also loaded by option \xoption{patch} of package \xpackage{kvoptions}.
% \LaTeX's option code is redefined to respect key value options and
% let the values untouched.
%
% \begin{description}
% \def\It#1{\item[\xoption{#1}:]}^^A
% \It{update}
%   The conversion program is only called, if the target file
%   does not exist or is older than the source image file.
% \It{append}
%   Puts the extension \xfile{.eps} at the end of the graphics extension list
%   (default).
% \It{prepend}
%   Puts the extension \xfile{.eps} at the begin of the graphics extension list.
% \It{outdir}
%   The converted file may put in an other output directory.
%   The value of \xoption{outdir} must include the directory separator.
%   Example for the current directory:
%   \begin{quote}
%   |\epstopdfsetup{outdir=./}|
%   \end{quote}
%   For other directories ensure, that they can be found.
%   See \cs{graphicspath} or |TEXINPUTS|.
% \It{suffix}
%   This option takes a string that is put between the file name base
%   and the extension of the output file. Rationale: It can happen,
%   that a PDF file is the original file and the EPS file the
%   generated file. If now the package thinks, that the PDF file
%   is the generated file, it will `regnerate' it. But in reality
%   the original file is lost. Therefore I recommend to use this
%   option always to generate a separate name space for generated
%   files. Proposed value is \texttt{-generated} or \texttt{.generated}.
%   The suffix \texttt{.generated} will also work here without the need for
%   package \xpackage{grffile}).
%
%   Example:
%   \begin{quote}
%   |\epstopdfsetup{suffix=-generated}|\\
%   Then \xfile{foo.eps} is converted to \xfile{foo-generated.pdf}.
%   \end{quote}
%   \cs{SourceExt} can be used inside the suffix string.
%   It's will be replaced by the extension of the image source
%   file without the leading dot, for instance:
%   \begin{quote}
%   |\epstopdfsetup{suffix=-\SourceExt-converted-to}|\\
%   \xfile{foo.eps} $\Rightarrow$ \xfile{foo-eps-converted-to.pdf}
%   \end{quote}
%   See also the next option \xoption{prefersuffix} that
%   modifies the behaviour of option \xoption{suffix} in some
%   cases.
%
%   Default for \xoption{suffix} is `|-\SourceExt-converted-to|'.
%
% \It{prefersuffix}
%   If a suffix is set by option \xoption{suffix}, then there can be
%   two image file names that could be taken into account for inclusion:
%   A image file name with the suffix string inside
%   its name and a image file name without; e.g. for \xfile{foo.eps}
%   the names could be:
%   \begin{quote}
%   \xfile{foo-suffix.pdf}, \xfile{foo.pdf}
%   \end{quote}
%   If option \xoption{perfersuffix} is turned on, the
%   file \xfile{foo-suffix.pdf} and its generation is preferred
%   over using \xfile{foo.pdf}. Otherwise \xfile{foo.pdf} is included
%   without generating \xfile{foo-suffix.pdf}.
%   The default of option \xoption{prefersuffix} is \texttt{true}.
% \It{program@epstopdf}
%   The name for the conversion program from EPS to PDF,
%   default is `|epstopdf|'.
% \It{verbose}
%   It prints some information about the image in the \xfile{.log} file
%   (default).
% \end{description}
%
% \subsection{Configuration}
%
% \subsubsection{System configuration file \xfile{epstopdf-sys.cfg}}
%
% If \xfile{epstopdf-sys} exists it is loaded at the end of the
% package \xpackage{epstopdf-base} and before the user configuration
% file. It's intended for TeX distributors. Thus they could add
% additional conversion rules (e.g., .gif -> .png) or set options.
%
% \subsubsection{User configuration file \xfile{epstopdf.cfg}}
%
% A configuration file \xfile{epstopdf.cfg} is loaded at the end
% of the package if it exists. It can be used for changing
% the default option setting. Example:
% \begin{quote}
% |\epstopdfsetup{verbose=false}|
% \end{quote}
%
% \subsubsection{Conversion program}
%
% You can use \cs{DeclareGraphicsRule} in a similar way as the
% route via \xprogram{dvips} to specify the conversion
% command line.
% The conversion argument starts with a back tick, followed
% by the conversion command including parameters.
%
% The whole conversion argument should also be wrapped inside
% \cs{epstopdfcall}. This reduces the problem with packages
% (e.g. \xpackage{pst-pdf}) that use the conversion argument
% and expands it. Macros \cs{SourceFile}, \cs{OutputFile}, and
% \cs{SourceExt} are not defined outside \xpackage{epstopdf-base}'s
% \cs{Gin@setfile} and error messages because of undefined
% command names are the result. If \cs{epstopdfcall} detects that
% it is called outside \xpackage{epstopdf-base}'s \cs{Gin@setfile}
% then it replaces the conversion argument by
% package \xpackage{graphics}'s
% default, usually the image file.
%
% The following macros are available inside:
% \begin{description}
% \def\It#1{\item[\cs{#1}:]}
% \It{OutputFile}: output file name (with known path and extension)
% \It{SourceFile}: source file name (with known path and extension),
%                  usually the same as |#1|,
% \It{SourceExt}: source extension without leading dot.
% \end{description}
%
% \paragraph{Conversion from EPS to PDF.}
% Other programs than \xprogram{epstopdf} can be used to convert from
% EPS to PDF. Example that uses \xprogram{Ghostscript}:
% \begin{quote}
%   |\DeclareGraphicsRule{.eps}{pdf}{.pdf}{%|\\
%   |  \epstopdfcall{`ps2pdf -dEPSCrop #1 \noexpand\OutputFile}%|\\
%   |}|
% \end{quote}
% \cs{DeclareGraphicsRule} expands the argument, therefore \cs{noexpand}
% is necessary. As convenience package \xpackage{epstopdf-base} defines
% \cs{epstopdfDeclareGraphicsRule}. Then the conversion argument
% is not expanded, \cs{epstopdfcall} and the back tick are added:
% \begin{quote}
%   |\epstopdfDeclareGraphicsRule{.eps}{pdf}{.pdf}{%|\\
%   |  ps2pdf -dEPSCrop #1 \OutputFile|\\
%   |}|
% \end{quote}
%
% Also \cs{OutputFile} respects the setting of option \xoption{outdir}.
%
% \subsection{Other image formats}
%
% The support that package \xpackage{epstopdf} implements is not
% limited to EPS files. Other image conversions can be declared.
% The following example shows it for GIF images under Unix with
% ImageMagick's \xprogram{convert}:
%\begin{quote}
%\begin{verbatim}
%\epstopdfDeclareGraphicsRule{.gif}{png}{.png}{%
%  convert #1 \OutputFile
%}
%\end{verbatim}
%\end{quote}
% The file extension \xfile{.gif} can be added to the extension list that
% package \xpackage{graphics} searches if the file extension
% is not given in \cs{includegraphics}.
% The list can be set by \cs{GraphicsExtensions}.
% \begin{quote}
% |\AppendGraphicsExtensions{.gif}|\\
% or\\
% |\PrependGraphicsExtensions{.gif}|
% \end{quote}
%
% \StopEventually{
% }
%
% \section{Implementation}
%
%    \begin{macrocode}
%<*package>
%    \end{macrocode}
%
% \subsection{Wrapper package}
%
%    Reload check, especially if the package is not used with \LaTeX.
%    \begin{macrocode}
\begingroup\catcode61\catcode48\catcode32=10\relax%
  \catcode13=5 % ^^M
  \endlinechar=13 %
  \catcode35=6 % #
  \catcode39=12 % '
  \catcode44=12 % ,
  \catcode45=12 % -
  \catcode46=12 % .
  \catcode58=12 % :
  \catcode64=11 % @
  \catcode123=1 % {
  \catcode125=2 % }
  \expandafter\let\expandafter\x\csname ver@epstopdf.sty\endcsname
  \ifx\x\relax % plain-TeX, first loading
  \else
    \def\empty{}%
    \ifx\x\empty % LaTeX, first loading,
      % variable is initialized, but \ProvidesPackage not yet seen
    \else
      \expandafter\ifx\csname PackageInfo\endcsname\relax
        \def\x#1#2{%
          \immediate\write-1{Package #1 Info: #2.}%
        }%
      \else
        \def\x#1#2{\PackageInfo{#1}{#2, stopped}}%
      \fi
      \x{epstopdf}{The package is already loaded}%
      \aftergroup\endinput
    \fi
  \fi
\endgroup%
%    \end{macrocode}
%    Package identification:
%    \begin{macrocode}
\begingroup\catcode61\catcode48\catcode32=10\relax%
  \catcode13=5 % ^^M
  \endlinechar=13 %
  \catcode35=6 % #
  \catcode39=12 % '
  \catcode40=12 % (
  \catcode41=12 % )
  \catcode44=12 % ,
  \catcode45=12 % -
  \catcode46=12 % .
  \catcode47=12 % /
  \catcode58=12 % :
  \catcode64=11 % @
  \catcode91=12 % [
  \catcode93=12 % ]
  \catcode123=1 % {
  \catcode125=2 % }
  \expandafter\ifx\csname ProvidesPackage\endcsname\relax
    \def\x#1#2#3[#4]{\endgroup
      \immediate\write-1{Package: #3 #4}%
      \xdef#1{#4}%
    }%
  \else
    \def\x#1#2[#3]{\endgroup
      #2[{#3}]%
      \ifx#1\@undefined
        \xdef#1{#3}%
      \fi
      \ifx#1\relax
        \xdef#1{#3}%
      \fi
    }%
  \fi
\expandafter\x\csname ver@epstopdf.sty\endcsname
\ProvidesPackage{epstopdf}%
  [2016/05/15 v2.6 Conversion with epstopdf on the fly (HO)]%
%    \end{macrocode}
%    Larger catcode set because of configuration files needed.
%    \begin{macrocode}
\begingroup\catcode61\catcode48\catcode32=10\relax%
  \catcode13=5 % ^^M
  \endlinechar=13 %
  \catcode123=1 % {
  \catcode125=2 % }
  \catcode64=11 % @
  \def\x{\endgroup
    \expandafter\edef\csname ETE@AtEnd\endcsname{%
      \endlinechar=\the\endlinechar\relax
      \catcode13=\the\catcode13\relax
      \catcode32=\the\catcode32\relax
      \catcode35=\the\catcode35\relax
      \catcode61=\the\catcode61\relax
      \catcode64=\the\catcode64\relax
      \catcode123=\the\catcode123\relax
      \catcode125=\the\catcode125\relax
    }%
  }%
\x\catcode61\catcode48\catcode32=10\relax%
\catcode13=5 % ^^M
\endlinechar=13 %
\catcode35=6 % #
\catcode64=11 % @
\catcode123=1 % {
\catcode125=2 % }
\def\TMP@EnsureCode#1#2{%
  \edef\ETE@AtEnd{%
    \ETE@AtEnd
    \catcode#1=\the\catcode#1\relax
  }%
  \catcode#1=#2\relax
}
\TMP@EnsureCode{39}{12}% '
\TMP@EnsureCode{40}{12}% (
\TMP@EnsureCode{41}{12}% )
\TMP@EnsureCode{42}{12}% *
\TMP@EnsureCode{45}{12}% -
\TMP@EnsureCode{47}{12}% /
\TMP@EnsureCode{91}{12}% [
\TMP@EnsureCode{93}{12}% ]
\edef\ETE@AtEnd{\ETE@AtEnd\noexpand\endinput}
%    \end{macrocode}
%
%    \begin{macrocode}
\let\ETE@SavedAtEnd\ETE@AtEnd
\RequirePackage{epstopdf-base}[2016/05/15]
\let\ETE@AtEnd\ETE@SavedAtEnd
%    \end{macrocode}
%
% \subsubsection{Option handling}
%
%    \begin{macrocode}
\DeclareOption*{%
  \expandafter\epstopdfsetup\expandafter{\CurrentOption}%
}
\ProcessOptions*\relax
%    \end{macrocode}
%
%    \begin{macrocode}
\ETE@AtEnd%
%</package>
%    \end{macrocode}
%
% \subsection{Base package}
%
%    \begin{macrocode}
%<*base>
%    \end{macrocode}
%
% \subsection{Preparations}
%
% \subsubsection{Relead check and identification}
%
%    Reload check, especially if the package is not used with \LaTeX.
%    \begin{macrocode}
\begingroup\catcode61\catcode48\catcode32=10\relax%
  \catcode13=5 % ^^M
  \endlinechar=13 %
  \catcode35=6 % #
  \catcode39=12 % '
  \catcode44=12 % ,
  \catcode45=12 % -
  \catcode46=12 % .
  \catcode58=12 % :
  \catcode64=11 % @
  \catcode123=1 % {
  \catcode125=2 % }
  \expandafter\let\expandafter\x\csname ver@epstopdf-base.sty\endcsname
  \ifx\x\relax % plain-TeX, first loading
  \else
    \def\empty{}%
    \ifx\x\empty % LaTeX, first loading,
      % variable is initialized, but \ProvidesPackage not yet seen
    \else
      \expandafter\ifx\csname PackageInfo\endcsname\relax
        \def\x#1#2{%
          \immediate\write-1{Package #1 Info: #2.}%
        }%
      \else
        \def\x#1#2{\PackageInfo{#1}{#2, stopped}}%
      \fi
      \x{epstopdf-base}{The package is already loaded}%
      \aftergroup\endinput
    \fi
  \fi
\endgroup%
%    \end{macrocode}
%    Package identification:
%    \begin{macrocode}
\begingroup\catcode61\catcode48\catcode32=10\relax%
  \catcode13=5 % ^^M
  \endlinechar=13 %
  \catcode35=6 % #
  \catcode39=12 % '
  \catcode40=12 % (
  \catcode41=12 % )
  \catcode44=12 % ,
  \catcode45=12 % -
  \catcode46=12 % .
  \catcode47=12 % /
  \catcode58=12 % :
  \catcode64=11 % @
  \catcode91=12 % [
  \catcode93=12 % ]
  \catcode123=1 % {
  \catcode125=2 % }
  \expandafter\ifx\csname ProvidesPackage\endcsname\relax
    \def\x#1#2#3[#4]{\endgroup
      \immediate\write-1{Package: #3 #4}%
      \xdef#1{#4}%
    }%
  \else
    \def\x#1#2[#3]{\endgroup
      #2[{#3}]%
      \ifx#1\@undefined
        \xdef#1{#3}%
      \fi
      \ifx#1\relax
        \xdef#1{#3}%
      \fi
    }%
  \fi
\expandafter\x\csname ver@epstopdf-base.sty\endcsname
\ProvidesPackage{epstopdf-base}%
  [2016/05/15 v2.6 Base part for package epstopdf]%
%    \end{macrocode}
%
% \subsubsection{Catcodes}
%
%    \begin{macrocode}
\begingroup\catcode61\catcode48\catcode32=10\relax%
  \catcode13=5 % ^^M
  \endlinechar=13 %
  \catcode123=1 % {
  \catcode125=2 % }
  \catcode64=11 % @
  \def\x{\endgroup
    \expandafter\edef\csname ETE@AtEnd\endcsname{%
      \endlinechar=\the\endlinechar\relax
      \catcode13=\the\catcode13\relax
      \catcode32=\the\catcode32\relax
      \catcode35=\the\catcode35\relax
      \catcode61=\the\catcode61\relax
      \catcode64=\the\catcode64\relax
      \catcode123=\the\catcode123\relax
      \catcode125=\the\catcode125\relax
    }%
  }%
\x\catcode61\catcode48\catcode32=10\relax%
\catcode13=5 % ^^M
\endlinechar=13 %
\catcode35=6 % #
\catcode64=11 % @
\catcode123=1 % {
\catcode125=2 % }
\def\TMP@EnsureCode#1#2{%
  \edef\ETE@AtEnd{%
    \ETE@AtEnd
    \catcode#1=\the\catcode#1\relax
  }%
  \catcode#1=#2\relax
}
\TMP@EnsureCode{33}{12}% !
\TMP@EnsureCode{39}{12}% '
\TMP@EnsureCode{42}{12}% *
\TMP@EnsureCode{44}{12}% ,
\TMP@EnsureCode{45}{12}% -
\TMP@EnsureCode{46}{12}% .
\TMP@EnsureCode{47}{12}% /
\TMP@EnsureCode{58}{12}% :
\TMP@EnsureCode{60}{12}% <
\TMP@EnsureCode{62}{12}% >
\TMP@EnsureCode{96}{12}% `
\edef\ETE@AtEnd{\ETE@AtEnd\noexpand\endinput}
%    \end{macrocode}
%
% \subsubsection{Load packages}
%
%    \begin{macrocode}
\RequirePackage{infwarerr}[2007/09/09]
\RequirePackage{grfext}\relax
\RequirePackage{kvoptions}[2007/10/02]
\RequirePackage{pdftexcmds}[2007/11/11]
%    \end{macrocode}
%
% \subsection{Checks}
%
% Check, whether package graphics is loaded
% (also graphicx loads graphics).
% Because miniltx.tex does not know \cs{@ifpackageloaded}
% we test for \cs{Gin@setfile} instead.
%    \begin{macrocode}
\begingroup\expandafter\expandafter\expandafter\endgroup
\expandafter\ifx\csname Gin@setfile\endcsname\relax
  \@PackageWarningNoLine{epstopdf}{%
    No graphics package \string`graphic{s,x}\string' loaded%
  }%
  \newcommand*{\epstopdfsetup}[1]{}%
  \expandafter\ETE@AtEnd
\fi%
%    \end{macrocode}
%
% Check, whether pdftex.def is loaded. \cs{ver@pdftex.def} is not
% available with miniltx.tex, thus we test for \cs{Gin@driver}.
%    \begin{macrocode}
\begingroup
  \def\x{luatex.def}%
  \ifx\Gin@driver\x
  \else
  \def\x{pdftex.def}%
  \ifx\Gin@driver\x
  \else
    \@PackageWarningNoLine{epstopdf}{%
      Drivers other than `pdftex' and `luatex' are not supported%
    }%
    \endgroup
    \newcommand*{\epstopdfsetup}[1]{}%
    \expandafter\ETE@AtEnd
  \fi%
  \fi%
\endgroup
%    \end{macrocode}
%
% Check, whether the shell escape feature is enabled.
%
%    \begin{macrocode}
\begingroup
  \expandafter\ifx\csname pdf@shellescape\endcsname\relax
  \else
    \ifnum\pdf@shellescape>0 %
    \else
      \@PackageWarningNoLine{epstopdf}{%
        Shell escape feature is not enabled%
      }%
    \fi
  \fi
\endgroup
%    \end{macrocode}
%
% \subsection{Package loading}
%
%    \begin{macrocode}
%    \end{macrocode}
%
% \subsection{Options}
%
%    \begin{macrocode}
\SetupKeyvalOptions{family=ETE,prefix=ETE@}
\DeclareBoolOption{update}
\DeclareBoolOption{verbose}
\newif\ifETE@prepend
\DeclareVoidOption{prepend}{\ETE@prependtrue}
\DeclareVoidOption{append}{\ETE@prependfalse}
\DeclareStringOption{outdir}
\DeclareStringOption{suffix}
\DeclareBoolOption{prefersuffix}
\DeclareStringOption{program@epstopdf}
%    \end{macrocode}
%    Options \xoption{disable} and \xoption{enable} are for testing
%    only. Therefore they are not documented on purpose.
%    \begin{macrocode}
\DeclareBoolOption{disable}
\DeclareComplementaryOption{enable}{disable}
%    \end{macrocode}
%    \begin{macrocode}
\newcommand*{\epstopdfsetup}{\setkeys{ETE}}
%    \end{macrocode}
%
% \subsubsection{Default setting}
%
%    \begin{macrocode}
\epstopdfsetup{%
  verbose,%
  enable,%
  append,%
  update,%
  prefersuffix,%
  suffix=-\SourceExt-converted-to,%
  program@epstopdf=epstopdf%
}
%    \end{macrocode}
%
% \subsection{Make and verbose}
%
%    \begin{macrocode}
\begingroup\expandafter\expandafter\expandafter\endgroup
\expandafter\ifx\csname pdf@filemoddate\endcsname\relax
  \def\ETE@Make#1#2{%
    \ifETE@update
      \ETE@WarnModDate
    \fi
    \@firstofone
  }%
  \def\ETE@WarnModDate{%
    \@PackageWarningNoLine{epstopdf}{%
      \string\pdffilemoddate\space is not available,\MessageBreak
      option `update' will be ignored%
    }%
    \global\let\ETE@WarnModDate\relax
  }%
  \def\ETE@FileInfo#1#2{#1 file: <#2>}%
\else
  \def\ETE@Make#1#2{%
    \ifETE@update
      \ifnum\pdf@strcmp{\pdf@filemoddate{#1}}{\pdf@filemoddate{#2}}>0 %
        \expandafter\expandafter\expandafter\@firstofone
      \else
        \@PackageInfoNoLine{epstopdf}{%
          Output file is already uptodate%
        }%
        \expandafter\expandafter\expandafter\@gobble
      \fi
    \else
      \expandafter\@firstofone
    \fi
  }%
  \def\ETE@FileInfo#1#2{%
    #1 file: <#2>%
    \expandafter\expandafter\expandafter
    \ETE@Date\pdf@filemoddate{#2}\@nil
    \expandafter\expandafter\expandafter
    \ETE@Size\pdf@filesize{#2}\@nil
  }%
  \def\ETE@Date#1\@nil{%
    \ifx\\#1\\%
    \else
      \ETE@@Date#1\@nil
    \fi
  }%
  \def\ETE@@Date#1:#2#3#4#5#6#7#8#9{%
    \MessageBreak
    \@spaces\space\space\space date: #2#3#4#5-#6#7-#8#9 %
    \ETE@@Time
  }%
  \def\ETE@@Time#1#2#3#4#5#6#7\@nil{%
    #1#2:#3#4:#5#6%
  }%
  \def\ETE@Size#1\@nil{%
    \ifx\\#1\\%
    \else
      \MessageBreak
      \@spaces\space\space\space size: #1 bytes%
    \fi
  }%
\fi
%    \end{macrocode}
%
% \subsection{Adding conversion support}
%
%    Patch \cs{Gin@setfile} to execute |#3|, if it contains
%    a command.
%    \begin{macrocode}
\expandafter\ifx\csname ETE@OrgGin@setfile\endcsname\relax
  \let\ETE@OrgGin@setfile\Gin@setfile
\else
  \@PackageError{epstopdf}{%
    Command \string\ETE@OrgGin@setfile\space
    already defined.\MessageBreak
  }{%
    Probably some package has included the code of this package%
    \MessageBreak
    instead of using \string\RequirePackage{epstopdf}.%
    \MessageBreak
    \@ehc
  }%
\fi
\def\ETE@IfFileExists{%
  \begingroup\expandafter\expandafter\expandafter\endgroup
  \expandafter\ifx\csname grffile@IfFileExists\endcsname\relax
    \expandafter\IfFileExists
  \else
    \global\let\ETE@IfFileExists\grffile@IfFileExists
    \expandafter\grffile@IfFileExists
  \fi
}
\def\ETE@Skip#1#2\x\fi\fi{%
  \fi
  \fi
  \endgroup
  \fi
  \fi
  #1%
}
\newif\ifETE@InsideSetfile
\newcommand*{\epstopdfcall}[1]{%
  \ifETE@InsideSetfile
    \expandafter\@firstoftwo
  \else
    \expandafter\@secondoftwo
  \fi
  {`#1}%
  {\Gin@base\Gin@ext}%
}
\def\ETE@DefCommandLine#1{%
  \edef\CommandLine{\expandafter\fi\if`#1}%
}
\def\ETE@DefX#1{%
  \expandafter\expandafter\expandafter\def
  \expandafter\expandafter\expandafter\x
  \expandafter\expandafter\expandafter{%
    \expandafter\fi\if`#1\relax\else
  }%
}
\def\ETE@Gin@setfile#1#2#3{%
  \ifETE@disable
    \ETE@OrgGin@setfile{#1}{#2}{#3}%
  \else
    \begingroup
      \ETE@InsideSetfiletrue
      \ETE@DefX{#3}%
    \expandafter\endgroup
    \ifx\x\@empty
      \ETE@OrgGin@setfile{#1}{#2}{#3}%
    \else
      \begingroup
        \ETE@InsideSetfiletrue
        \def\GraphicsType{#1}%
        \def\GraphicsRead{#2}%
        \ifETE@prefersuffix
        \else
          \ifx\ETE@suffix\@empty
          \else
            \ETE@IfFileExists{\Gin@base\GraphicsRead}{%
              \ETE@Skip{%
                \ETE@OrgGin@setfile{#1}{#2}{\Gin@base#2}%
              }%
            }{%
              \let\next\relax
            }%
            \next
          \fi
        \fi
        \ifx\Gin@ext\relax
          \let\SourceExt\Gin@eext
          \def\SourceFile{\Gin@base\Gin@eext}%
        \else
          \let\SourceExt\Gin@ext
          \def\SourceFile{\Gin@base\Gin@ext}%
        \fi
        \edef\SourceExt{% remove dot
          \expandafter\@cdr\SourceExt\@empty\@nil
        }%
        \let\OutputDirectory\ETE@outdir
        \ifx\OutputDirectory\@empty
          \edef\OutputFile{\ETE@GenerateName{\Gin@base}{#2}}%
        \else
          \begingroup
            \filename@parse{\Gin@base#2}%
            \edef\x{\endgroup
              \def\noexpand\OutputFile{%
                \ETE@GenerateName{%
                  \OutputDirectory\filename@base
                }{#2}%
              }%
            }%
          \x
        \fi
        \ETE@DefCommandLine{#3}%
        \ifETE@verbose
          \@PackageInfo{epstopdf}{%
            \ETE@FileInfo{Source}\SourceFile\MessageBreak
            \ETE@FileInfo{Output}\OutputFile\MessageBreak
            Command: <\CommandLine>\MessageBreak
            \string\includegraphics
          }%
        \fi
        \ETE@Make\SourceFile\OutputFile{%
          \pdf@system{\CommandLine}%
          \ifETE@verbose
            \@PackageInfoNoLine{epstopdf}{%
              \ETE@FileInfo{Result}\OutputFile
            }%
          \fi
        }%
        \edef\x{\endgroup
          \ifx\OutputDirectory\@empty
          \else
            \def\noexpand\Gin@base{%
              \OutputDirectory\noexpand\filename@base
            }%
          \fi
          \ifx\ETE@suffix\@empty
          \else
            \edef\noexpand\Gin@base{%
              \noexpand\Gin@base\ETE@suffix
            }%
          \fi
          \noexpand\ETE@OrgGin@setfile{%
            \GraphicsType
          }{%
            \GraphicsRead
          }{%
            \OutputFile
          }%
        }%
      \x
    \fi
  \fi
}
\let\Gin@setfile\ETE@Gin@setfile
\def\ETE@GenerateName#1#2{%
  #1\ETE@suffix#2%
}
%    \end{macrocode}
%
% \subsection{Declare graphics rule}
%
%    \begin{macrocode}
\newcommand*{\epstopdfDeclareGraphicsRule}[4]{%
  \ifx\\#4\\%
    \@PackageError{epstopdf-base}{%
      Conversion command is missing%
    }\@ehc
  \else
    \begingroup
      \@ifundefined{Gin@rule@#1}{%
      }{%
        \@PackageInfo{epstopdf-base}{%
          Redefining graphics rule for `#1'%
        }%
      }%
    \endgroup
    \@namedef{Gin@rule@#1}##1{{#2}{#3}{\epstopdfcall{#4}}}%
  \fi
}
%    \end{macrocode}
%
% \cs{DeclareGraphicsRule} for .eps
%    \begin{macrocode}
\epstopdfDeclareGraphicsRule{.eps}{pdf}{.pdf}{%
  \ETE@epstopdf{#1}%
}
\def\ETE@epstopdf#1{%
  \ETE@program@epstopdf\space
  \ifcase\ifx\OutputDirectory\@empty
           \ifx\ETE@suffix\@empty
             1%
           \fi
         \fi
         0 %
    --outfile=\OutputFile\space
  \fi
  #1%
}
%    \end{macrocode}
%
%    \begin{macrocode}
\ifETE@prepend
  \expandafter\PrependGraphicsExtensions
\else
  \expandafter\AppendGraphicsExtensions
\fi
{.eps}
\let\ETE@prepend\@undefined
\DeclareVoidOption{prepend}{%
  \PrependGraphicsExtensions{.eps}%
}
\let\ETE@append\@undefined
\DeclareVoidOption{append}{%
  \AppendGraphicsExtensions{.eps}%
}
%    \end{macrocode}
%
%    \begin{macrocode}
\InputIfFileExists{epstopdf-sys.cfg}{}{}
\InputIfFileExists{epstopdf.cfg}{}{}
%    \end{macrocode}
%
%    \begin{macrocode}
\ETE@AtEnd%
%</base>
%    \end{macrocode}
%
% \section{Test}
%
% \subsection{Preface for standard catcode check}
%
%    \begin{macrocode}
%<*test1>
\input miniltx.tex\relax
\def\Gin@driver{pdftex.def}
\input graphicx.sty\relax
\resetatcatcode
%</test1>
%    \end{macrocode}
%
% \subsection{Catcode checks for loading}
%
%    \begin{macrocode}
%<*test1>
%    \end{macrocode}
%    \begin{macrocode}
\catcode`\{=1 %
\catcode`\}=2 %
\catcode`\#=6 %
\catcode`\@=11 %
\expandafter\ifx\csname count@\endcsname\relax
  \countdef\count@=255 %
\fi
\expandafter\ifx\csname @gobble\endcsname\relax
  \long\def\@gobble#1{}%
\fi
\expandafter\ifx\csname @firstofone\endcsname\relax
  \long\def\@firstofone#1{#1}%
\fi
\expandafter\ifx\csname loop\endcsname\relax
  \expandafter\@firstofone
\else
  \expandafter\@gobble
\fi
{%
  \def\loop#1\repeat{%
    \def\body{#1}%
    \iterate
  }%
  \def\iterate{%
    \body
      \let\next\iterate
    \else
      \let\next\relax
    \fi
    \next
  }%
  \let\repeat=\fi
}%
\def\RestoreCatcodes{}
\count@=0 %
\loop
  \edef\RestoreCatcodes{%
    \RestoreCatcodes
    \catcode\the\count@=\the\catcode\count@\relax
  }%
\ifnum\count@<255 %
  \advance\count@ 1 %
\repeat

\def\RangeCatcodeInvalid#1#2{%
  \count@=#1\relax
  \loop
    \catcode\count@=15 %
  \ifnum\count@<#2\relax
    \advance\count@ 1 %
  \repeat
}
\def\RangeCatcodeCheck#1#2#3{%
  \count@=#1\relax
  \loop
    \ifnum#3=\catcode\count@
    \else
      \errmessage{%
        Character \the\count@\space
        with wrong catcode \the\catcode\count@\space
        instead of \number#3%
      }%
    \fi
  \ifnum\count@<#2\relax
    \advance\count@ 1 %
  \repeat
}
\def\space{ }
\expandafter\ifx\csname LoadCommand\endcsname\relax
  \def\LoadCommand{\input epstopdf.sty\relax}%
\fi
\def\Test{%
  \RangeCatcodeInvalid{0}{47}%
  \RangeCatcodeInvalid{58}{64}%
  \RangeCatcodeInvalid{91}{96}%
  \RangeCatcodeInvalid{123}{255}%
  \catcode`\@=12 %
  \catcode`\\=0 %
  \catcode`\%=14 %
  \LoadCommand
  \RangeCatcodeCheck{0}{36}{15}%
  \RangeCatcodeCheck{37}{37}{14}%
  \RangeCatcodeCheck{38}{47}{15}%
  \RangeCatcodeCheck{48}{57}{12}%
  \RangeCatcodeCheck{58}{63}{15}%
  \RangeCatcodeCheck{64}{64}{12}%
  \RangeCatcodeCheck{65}{90}{11}%
  \RangeCatcodeCheck{91}{91}{15}%
  \RangeCatcodeCheck{92}{92}{0}%
  \RangeCatcodeCheck{93}{96}{15}%
  \RangeCatcodeCheck{97}{122}{11}%
  \RangeCatcodeCheck{123}{255}{15}%
  \RestoreCatcodes
}
\Test
\csname @@end\endcsname
\end
%    \end{macrocode}
%    \begin{macrocode}
%</test1>
%    \end{macrocode}
%
% \section{Installation}
%
% \subsection{Download}
%
% \paragraph{Package.} This package is available on
% CTAN\footnote{\url{http://ctan.org/pkg/epstopdf}}:
% \begin{description}
% \item[\CTAN{macros/latex/contrib/oberdiek/epstopdf.dtx}] The source file.
% \item[\CTAN{macros/latex/contrib/oberdiek/epstopdf.pdf}] Documentation.
% \end{description}
%
%
% \paragraph{Bundle.} All the packages of the bundle `oberdiek'
% are also available in a TDS compliant ZIP archive. There
% the packages are already unpacked and the documentation files
% are generated. The files and directories obey the TDS standard.
% \begin{description}
% \item[\CTAN{install/macros/latex/contrib/oberdiek.tds.zip}]
% \end{description}
% \emph{TDS} refers to the standard ``A Directory Structure
% for \TeX\ Files'' (\CTAN{tds/tds.pdf}). Directories
% with \xfile{texmf} in their name are usually organized this way.
%
% \subsection{Bundle installation}
%
% \paragraph{Unpacking.} Unpack the \xfile{oberdiek.tds.zip} in the
% TDS tree (also known as \xfile{texmf} tree) of your choice.
% Example (linux):
% \begin{quote}
%   |unzip oberdiek.tds.zip -d ~/texmf|
% \end{quote}
%
% \paragraph{Script installation.}
% Check the directory \xfile{TDS:scripts/oberdiek/} for
% scripts that need further installation steps.
% Package \xpackage{attachfile2} comes with the Perl script
% \xfile{pdfatfi.pl} that should be installed in such a way
% that it can be called as \texttt{pdfatfi}.
% Example (linux):
% \begin{quote}
%   |chmod +x scripts/oberdiek/pdfatfi.pl|\\
%   |cp scripts/oberdiek/pdfatfi.pl /usr/local/bin/|
% \end{quote}
%
% \subsection{Package installation}
%
% \paragraph{Unpacking.} The \xfile{.dtx} file is a self-extracting
% \docstrip\ archive. The files are extracted by running the
% \xfile{.dtx} through \plainTeX:
% \begin{quote}
%   \verb|tex epstopdf.dtx|
% \end{quote}
%
% \paragraph{TDS.} Now the different files must be moved into
% the different directories in your installation TDS tree
% (also known as \xfile{texmf} tree):
% \begin{quote}
% \def\t{^^A
% \begin{tabular}{@{}>{\ttfamily}l@{ $\rightarrow$ }>{\ttfamily}l@{}}
%   epstopdf.sty & tex/latex/oberdiek/epstopdf.sty\\
%   epstopdf-base.sty & tex/latex/oberdiek/epstopdf-base.sty\\
%   epstopdf.pdf & doc/latex/oberdiek/epstopdf.pdf\\
%   test/epstopdf-test1.tex & doc/latex/oberdiek/test/epstopdf-test1.tex\\
%   epstopdf.dtx & source/latex/oberdiek/epstopdf.dtx\\
% \end{tabular}^^A
% }^^A
% \sbox0{\t}^^A
% \ifdim\wd0>\linewidth
%   \begingroup
%     \advance\linewidth by\leftmargin
%     \advance\linewidth by\rightmargin
%   \edef\x{\endgroup
%     \def\noexpand\lw{\the\linewidth}^^A
%   }\x
%   \def\lwbox{^^A
%     \leavevmode
%     \hbox to \linewidth{^^A
%       \kern-\leftmargin\relax
%       \hss
%       \usebox0
%       \hss
%       \kern-\rightmargin\relax
%     }^^A
%   }^^A
%   \ifdim\wd0>\lw
%     \sbox0{\small\t}^^A
%     \ifdim\wd0>\linewidth
%       \ifdim\wd0>\lw
%         \sbox0{\footnotesize\t}^^A
%         \ifdim\wd0>\linewidth
%           \ifdim\wd0>\lw
%             \sbox0{\scriptsize\t}^^A
%             \ifdim\wd0>\linewidth
%               \ifdim\wd0>\lw
%                 \sbox0{\tiny\t}^^A
%                 \ifdim\wd0>\linewidth
%                   \lwbox
%                 \else
%                   \usebox0
%                 \fi
%               \else
%                 \lwbox
%               \fi
%             \else
%               \usebox0
%             \fi
%           \else
%             \lwbox
%           \fi
%         \else
%           \usebox0
%         \fi
%       \else
%         \lwbox
%       \fi
%     \else
%       \usebox0
%     \fi
%   \else
%     \lwbox
%   \fi
% \else
%   \usebox0
% \fi
% \end{quote}
% If you have a \xfile{docstrip.cfg} that configures and enables \docstrip's
% TDS installing feature, then some files can already be in the right
% place, see the documentation of \docstrip.
%
% \subsection{Refresh file name databases}
%
% If your \TeX~distribution
% (\teTeX, \mikTeX, \dots) relies on file name databases, you must refresh
% these. For example, \teTeX\ users run \verb|texhash| or
% \verb|mktexlsr|.
%
% \subsection{Some details for the interested}
%
% \paragraph{Attached source.}
%
% The PDF documentation on CTAN also includes the
% \xfile{.dtx} source file. It can be extracted by
% AcrobatReader 6 or higher. Another option is \textsf{pdftk},
% e.g. unpack the file into the current directory:
% \begin{quote}
%   \verb|pdftk epstopdf.pdf unpack_files output .|
% \end{quote}
%
% \paragraph{Unpacking with \LaTeX.}
% The \xfile{.dtx} chooses its action depending on the format:
% \begin{description}
% \item[\plainTeX:] Run \docstrip\ and extract the files.
% \item[\LaTeX:] Generate the documentation.
% \end{description}
% If you insist on using \LaTeX\ for \docstrip\ (really,
% \docstrip\ does not need \LaTeX), then inform the autodetect routine
% about your intention:
% \begin{quote}
%   \verb|latex \let\install=y% \iffalse meta-comment
%
% File: epstopdf.dtx
% Version: 2016/05/15 v2.6
% Info: Conversion with epstopdf on the fly
%
% Copyright (C) 2001, 2006-2010 by
%    Heiko Oberdiek <heiko.oberdiek at googlemail.com>
%
% This work may be distributed and/or modified under the
% conditions of the LaTeX Project Public License, either
% version 1.3c of this license or (at your option) any later
% version. This version of this license is in
%    http://www.latex-project.org/lppl/lppl-1-3c.txt
% and the latest version of this license is in
%    http://www.latex-project.org/lppl.txt
% and version 1.3 or later is part of all distributions of
% LaTeX version 2005/12/01 or later.
%
% This work has the LPPL maintenance status "maintained".
%
% This Current Maintainer of this work is Heiko Oberdiek.
%
% This work consists of the main source file epstopdf.dtx
% and the derived files
%    epstopdf.sty, epstopdf.pdf, epstopdf.ins, epstopdf.drv,
%    epstopdf-base.sty, epstopdf-test1.tex.
%
% Distribution:
%    CTAN:macros/latex/contrib/oberdiek/epstopdf.dtx
%    CTAN:macros/latex/contrib/oberdiek/epstopdf.pdf
%
% Unpacking:
%    (a) If epstopdf.ins is present:
%           tex epstopdf.ins
%    (b) Without epstopdf.ins:
%           tex epstopdf.dtx
%    (c) If you insist on using LaTeX
%           latex \let\install=y\input{epstopdf.dtx}
%        (quote the arguments according to the demands of your shell)
%
% Documentation:
%    (a) If epstopdf.drv is present:
%           latex epstopdf.drv
%    (b) Without epstopdf.drv:
%           latex epstopdf.dtx; ...
%    The class ltxdoc loads the configuration file ltxdoc.cfg
%    if available. Here you can specify further options, e.g.
%    use A4 as paper format:
%       \PassOptionsToClass{a4paper}{article}
%
%    Programm calls to get the documentation (example):
%       pdflatex epstopdf.dtx
%       makeindex -s gind.ist epstopdf.idx
%       pdflatex epstopdf.dtx
%       makeindex -s gind.ist epstopdf.idx
%       pdflatex epstopdf.dtx
%
% Installation:
%    TDS:tex/latex/oberdiek/epstopdf.sty
%    TDS:tex/latex/oberdiek/epstopdf-base.sty
%    TDS:doc/latex/oberdiek/epstopdf.pdf
%    TDS:doc/latex/oberdiek/test/epstopdf-test1.tex
%    TDS:source/latex/oberdiek/epstopdf.dtx
%
%<*ignore>
\begingroup
  \catcode123=1 %
  \catcode125=2 %
  \def\x{LaTeX2e}%
\expandafter\endgroup
\ifcase 0\ifx\install y1\fi\expandafter
         \ifx\csname processbatchFile\endcsname\relax\else1\fi
         \ifx\fmtname\x\else 1\fi\relax
\else\csname fi\endcsname
%</ignore>
%<*install>
\input docstrip.tex
\Msg{************************************************************************}
\Msg{* Installation}
\Msg{* Package: epstopdf 2016/05/15 v2.6 Conversion with epstopdf on the fly (HO)}
\Msg{************************************************************************}

\keepsilent
\askforoverwritefalse

\let\MetaPrefix\relax
\preamble

This is a generated file.

Project: epstopdf
Version: 2016/05/15 v2.6

Copyright (C) 2001, 2006-2010 by
   Heiko Oberdiek <heiko.oberdiek at googlemail.com>

This work may be distributed and/or modified under the
conditions of the LaTeX Project Public License, either
version 1.3c of this license or (at your option) any later
version. This version of this license is in
   http://www.latex-project.org/lppl/lppl-1-3c.txt
and the latest version of this license is in
   http://www.latex-project.org/lppl.txt
and version 1.3 or later is part of all distributions of
LaTeX version 2005/12/01 or later.

This work has the LPPL maintenance status "maintained".

This Current Maintainer of this work is Heiko Oberdiek.

This work consists of the main source file epstopdf.dtx
and the derived files
   epstopdf.sty, epstopdf.pdf, epstopdf.ins, epstopdf.drv,
   epstopdf-base.sty, epstopdf-test1.tex.

\endpreamble
\let\MetaPrefix\DoubleperCent

\generate{%
  \file{epstopdf.ins}{\from{epstopdf.dtx}{install}}%
  \file{epstopdf.drv}{\from{epstopdf.dtx}{driver}}%
  \usedir{tex/latex/oberdiek}%
  \file{epstopdf.sty}{\from{epstopdf.dtx}{package}}%
  \file{epstopdf-base.sty}{\from{epstopdf.dtx}{base}}%
  \usedir{doc/latex/oberdiek/test}%
  \file{epstopdf-test1.tex}{\from{epstopdf.dtx}{test1}}%
  \nopreamble
  \nopostamble
  \usedir{source/latex/oberdiek/catalogue}%
  \file{epstopdf-pkg.xml}{\from{epstopdf.dtx}{catalogue}}%
}

\catcode32=13\relax% active space
\let =\space%
\Msg{************************************************************************}
\Msg{*}
\Msg{* To finish the installation you have to move the following}
\Msg{* files into a directory searched by TeX:}
\Msg{*}
\Msg{*     epstopdf.sty, epstopdf-base.sty}
\Msg{*}
\Msg{* To produce the documentation run the file `epstopdf.drv'}
\Msg{* through LaTeX.}
\Msg{*}
\Msg{* Happy TeXing!}
\Msg{*}
\Msg{************************************************************************}

\endbatchfile
%</install>
%<*ignore>
\fi
%</ignore>
%<*driver>
\NeedsTeXFormat{LaTeX2e}
\ProvidesFile{epstopdf.drv}%
  [2016/05/15 v2.6 Conversion with epstopdf on the fly (HO)]%
\documentclass{ltxdoc}
\usepackage{holtxdoc}[2011/11/22]
\begin{document}
  \DocInput{epstopdf.dtx}%
\end{document}
%</driver>
% \fi
%
%
% \CharacterTable
%  {Upper-case    \A\B\C\D\E\F\G\H\I\J\K\L\M\N\O\P\Q\R\S\T\U\V\W\X\Y\Z
%   Lower-case    \a\b\c\d\e\f\g\h\i\j\k\l\m\n\o\p\q\r\s\t\u\v\w\x\y\z
%   Digits        \0\1\2\3\4\5\6\7\8\9
%   Exclamation   \!     Double quote  \"     Hash (number) \#
%   Dollar        \$     Percent       \%     Ampersand     \&
%   Acute accent  \'     Left paren    \(     Right paren   \)
%   Asterisk      \*     Plus          \+     Comma         \,
%   Minus         \-     Point         \.     Solidus       \/
%   Colon         \:     Semicolon     \;     Less than     \<
%   Equals        \=     Greater than  \>     Question mark \?
%   Commercial at \@     Left bracket  \[     Backslash     \\
%   Right bracket \]     Circumflex    \^     Underscore    \_
%   Grave accent  \`     Left brace    \{     Vertical bar  \|
%   Right brace   \}     Tilde         \~}
%
% \GetFileInfo{epstopdf.drv}
%
% \title{The \xpackage{epstopdf} package}
% \date{2016/05/15 v2.6}
% \author{Heiko Oberdiek\thanks
% {Please report any issues at https://github.com/ho-tex/oberdiek/issues}\\
% \xemail{heiko.oberdiek at googlemail.com}}
%
% \maketitle
%
% \begin{abstract}
% This packages adds support of handling eps images
% to package \xpackage{graphics} or \xpackage{graphicx}
% with option \xoption{pdftex}. If an eps image is detected,
% epstopdf is automatically called to convert it to pdf format.
% \end{abstract}
%
% \tableofcontents
%
% \providecommand*{\xprogram}[1]{\textsf{#1}}
%
% \section{Documentation}
%
% \subsection{Introduction}
%
% \LaTeX\ provides its graphics bundle to include graphics files.
% Both packages \xpackage{graphics} or \xpackage{graphicx} may
% be used. the latter one loads the first and adds options
% in key value style for \cs{includegraphics}.
%
% Usually the drivers do not support all kind of graphics files.
% Other image types must be converted, before they become
% usuable. In case of driver \xprogram{dvips}, the graphics
% rule may contain a conversion rule. Then all that package
% \xpackage{graphics} must know is the bounding box, the command
% is passed to \xprogram{dvips} that calls it and embeds the
% converted image.
%
% However, \pdfTeX\ has its driver for PDF output already
% build in. It's graphics inclusion commands (\cs{pdfximage})
% does not allow the execution of external commands.
% Therefore commands in the last argument of
% \cs{DeclareGraphicsRule} were of no use.
% But external programs can be called within \pdfTeX.
% This feature is called ``shell escape'' or ``write 18''
% and must usually enabled explicitely because of security reasons.
% Now, this package \xpackage{epstopdf} hooks
% into package \xpackage{graphics}' code to catch
% that argument with the external command and executes it
% to convert the graphics file to a supported format and
% passes the control of graphics inclusion back to package
% \xpackage{graphics}.
%
% \subsection{Requirements}
%
% \begin{itemize}
% \item
%   The feature \cs{write18} must be enabled. This allows the
%   running of external programs during \TeX's compile run.
%   Keep in mind that this is a security risk. The feature is an
%   addition to \cs{TeX}. Mik\TeX, te\TeX, \TeX\ Live support it.
%   In Web2C based \TeX\ distributions (te\TeX, \TeX\ Live) it can be
%   enabled in the configuration file \xfile{texmf.cnf}:
%   \begin{quote}
%     |shell_escape = 1|
%   \end{quote}
%   Because of the security risk, it is better to do it on the
%   command line only:
%   \begin{quote}
%   \texttt{--shell-escape} (te\TeX, \TeX\ Live)\\
%   \texttt{--enable-write18} (MiK\TeX)
%   \end{quote}
%   Example:
%   \begin{quote}
%     |pdflatex -shell-escape test.tex|
%   \end{quote}
% \item
%   The program \xprogram{epstopdf} for the conversion from EPS
%   to PDF. However, other programs can
%   be used and configured by \cs{DeclareGraphicsRule}.
%   Example:
%   \begin{quote}
%   |\epstopdfDeclareGraphicsRule{.eps}{pdf}{.pdf}{%|\\
%   |  ps2pdf -dEPSCrop #1 \OutputFile|\\
%   |}|
%   \end{quote}
% \end{itemize}
%
% \subsection{Usage}
%
% The package is loaded after graphic\{s,x\}, e.g.:
% \begin{quote}
%   |\usepackage[pdftex]{graphicx}|\\
%   |\usepackage{epstopdf}|
% \end{quote}
% Now images with file name extension \xfile{.eps} are detected
% and supported using \cs{includegraphics}.
%
% If the graphics file name is explicitly specified with extension
% \xfile{.eps} the new rule for EPS files is called and
% the conversion performed. If option \xoption{update} is in force
% then the conversion step is dropped if the target file already
% exists and is not older then the EPS file.
%
% The situation is more complicate if the graphics file is
% given without file name extension. Then the \xpackage{graphics}
% package must search for a supported image file.
% The possible extensions are stored in the graphics extension
% list, that can be set by \cs{DeclareGraphicsExtensions}.
% The algorithm:
% \begin{quote}
%    \def\>{\mbox{}\quad}
%    function search( \meta{filebase} )\\
%    \>foreach \meta{ext} in \meta{graphics extensions}\\
%    \>\>foreach \meta{dir} in \meta{current directory}, \meta{\cs{graphicspath}}\\
%    \>\>\>\meta{file} := \meta{dir} + \meta{filebase} + \meta{ext}\\
%    \>\>\>if exist \meta{file}\\
%    \>\>\>\>return found\\
%    \>return not found
% \end{quote}
% Package \xpackage{epstopdf} puts \xfile{.eps} at the end of the
% graphics extension search list. This is the behaviour of option
% \xoption{append} that is enabled by default.
% That means, the conversion is called last
% unless a supported file type cannot be found earlier. This avoids
% unnecessary conversion steps that slow down the \LaTeX\ run.
% If you want to use option \xoption{update} and your \pdfTeX\ supports
% it, then an outdated PDF file also would be found earlier unless
% \xoption{suffix} is used that is the default since version 2.0.
%
% With an empty option \xoption{suffix} and option
% \xoption{prepend} there is a risk that an original PDF file is overwritten:
%
% If the original image file is the PDF file and there is also a generated
% EPS file, then the original PDF file can be regenerated (depending
% on the option settings) and the original PDF file gets lost.
% Therefore option \xoption{suffix} is introduced in version 1.9
% to create a separate name space for generated output files.
%
% \textbf{Note:} Usually the conversion program needs the exact location
% of the image file. Usually the current directory works.
% Also if the image file is found using \cs{graphicspath}, the
% location is known. However, if the image is somewhere
% in a directory of environment variable |TEXINPUTS|, then
% the package does not know the exact location and the
% conversion program will not find the image file unless it
% implements a search using |TEXINPUTS| (program \xprogram{kpsewhich}
% may be of help in this task).
%
% \subsection{Options}
%
% Options can be given as package options or later using:
% \begin{declcs}{epstopdfsetup} \M{key value list}
% \end{declcs}
% \LaTeX\ expands the option list before passing the option list
% to the package's option handling code. This can fail
% for option \xoption{suffix} if it contains some of the macros
% described below. Use \cs{epstopdfsetup} after the package is loaded.
% Or load package \xpackage{kvoptions-patch} before. This package
% is also loaded by option \xoption{patch} of package \xpackage{kvoptions}.
% \LaTeX's option code is redefined to respect key value options and
% let the values untouched.
%
% \begin{description}
% \def\It#1{\item[\xoption{#1}:]}^^A
% \It{update}
%   The conversion program is only called, if the target file
%   does not exist or is older than the source image file.
% \It{append}
%   Puts the extension \xfile{.eps} at the end of the graphics extension list
%   (default).
% \It{prepend}
%   Puts the extension \xfile{.eps} at the begin of the graphics extension list.
% \It{outdir}
%   The converted file may put in an other output directory.
%   The value of \xoption{outdir} must include the directory separator.
%   Example for the current directory:
%   \begin{quote}
%   |\epstopdfsetup{outdir=./}|
%   \end{quote}
%   For other directories ensure, that they can be found.
%   See \cs{graphicspath} or |TEXINPUTS|.
% \It{suffix}
%   This option takes a string that is put between the file name base
%   and the extension of the output file. Rationale: It can happen,
%   that a PDF file is the original file and the EPS file the
%   generated file. If now the package thinks, that the PDF file
%   is the generated file, it will `regnerate' it. But in reality
%   the original file is lost. Therefore I recommend to use this
%   option always to generate a separate name space for generated
%   files. Proposed value is \texttt{-generated} or \texttt{.generated}.
%   The suffix \texttt{.generated} will also work here without the need for
%   package \xpackage{grffile}).
%
%   Example:
%   \begin{quote}
%   |\epstopdfsetup{suffix=-generated}|\\
%   Then \xfile{foo.eps} is converted to \xfile{foo-generated.pdf}.
%   \end{quote}
%   \cs{SourceExt} can be used inside the suffix string.
%   It's will be replaced by the extension of the image source
%   file without the leading dot, for instance:
%   \begin{quote}
%   |\epstopdfsetup{suffix=-\SourceExt-converted-to}|\\
%   \xfile{foo.eps} $\Rightarrow$ \xfile{foo-eps-converted-to.pdf}
%   \end{quote}
%   See also the next option \xoption{prefersuffix} that
%   modifies the behaviour of option \xoption{suffix} in some
%   cases.
%
%   Default for \xoption{suffix} is `|-\SourceExt-converted-to|'.
%
% \It{prefersuffix}
%   If a suffix is set by option \xoption{suffix}, then there can be
%   two image file names that could be taken into account for inclusion:
%   A image file name with the suffix string inside
%   its name and a image file name without; e.g. for \xfile{foo.eps}
%   the names could be:
%   \begin{quote}
%   \xfile{foo-suffix.pdf}, \xfile{foo.pdf}
%   \end{quote}
%   If option \xoption{perfersuffix} is turned on, the
%   file \xfile{foo-suffix.pdf} and its generation is preferred
%   over using \xfile{foo.pdf}. Otherwise \xfile{foo.pdf} is included
%   without generating \xfile{foo-suffix.pdf}.
%   The default of option \xoption{prefersuffix} is \texttt{true}.
% \It{program@epstopdf}
%   The name for the conversion program from EPS to PDF,
%   default is `|epstopdf|'.
% \It{verbose}
%   It prints some information about the image in the \xfile{.log} file
%   (default).
% \end{description}
%
% \subsection{Configuration}
%
% \subsubsection{System configuration file \xfile{epstopdf-sys.cfg}}
%
% If \xfile{epstopdf-sys} exists it is loaded at the end of the
% package \xpackage{epstopdf-base} and before the user configuration
% file. It's intended for TeX distributors. Thus they could add
% additional conversion rules (e.g., .gif -> .png) or set options.
%
% \subsubsection{User configuration file \xfile{epstopdf.cfg}}
%
% A configuration file \xfile{epstopdf.cfg} is loaded at the end
% of the package if it exists. It can be used for changing
% the default option setting. Example:
% \begin{quote}
% |\epstopdfsetup{verbose=false}|
% \end{quote}
%
% \subsubsection{Conversion program}
%
% You can use \cs{DeclareGraphicsRule} in a similar way as the
% route via \xprogram{dvips} to specify the conversion
% command line.
% The conversion argument starts with a back tick, followed
% by the conversion command including parameters.
%
% The whole conversion argument should also be wrapped inside
% \cs{epstopdfcall}. This reduces the problem with packages
% (e.g. \xpackage{pst-pdf}) that use the conversion argument
% and expands it. Macros \cs{SourceFile}, \cs{OutputFile}, and
% \cs{SourceExt} are not defined outside \xpackage{epstopdf-base}'s
% \cs{Gin@setfile} and error messages because of undefined
% command names are the result. If \cs{epstopdfcall} detects that
% it is called outside \xpackage{epstopdf-base}'s \cs{Gin@setfile}
% then it replaces the conversion argument by
% package \xpackage{graphics}'s
% default, usually the image file.
%
% The following macros are available inside:
% \begin{description}
% \def\It#1{\item[\cs{#1}:]}
% \It{OutputFile}: output file name (with known path and extension)
% \It{SourceFile}: source file name (with known path and extension),
%                  usually the same as |#1|,
% \It{SourceExt}: source extension without leading dot.
% \end{description}
%
% \paragraph{Conversion from EPS to PDF.}
% Other programs than \xprogram{epstopdf} can be used to convert from
% EPS to PDF. Example that uses \xprogram{Ghostscript}:
% \begin{quote}
%   |\DeclareGraphicsRule{.eps}{pdf}{.pdf}{%|\\
%   |  \epstopdfcall{`ps2pdf -dEPSCrop #1 \noexpand\OutputFile}%|\\
%   |}|
% \end{quote}
% \cs{DeclareGraphicsRule} expands the argument, therefore \cs{noexpand}
% is necessary. As convenience package \xpackage{epstopdf-base} defines
% \cs{epstopdfDeclareGraphicsRule}. Then the conversion argument
% is not expanded, \cs{epstopdfcall} and the back tick are added:
% \begin{quote}
%   |\epstopdfDeclareGraphicsRule{.eps}{pdf}{.pdf}{%|\\
%   |  ps2pdf -dEPSCrop #1 \OutputFile|\\
%   |}|
% \end{quote}
%
% Also \cs{OutputFile} respects the setting of option \xoption{outdir}.
%
% \subsection{Other image formats}
%
% The support that package \xpackage{epstopdf} implements is not
% limited to EPS files. Other image conversions can be declared.
% The following example shows it for GIF images under Unix with
% ImageMagick's \xprogram{convert}:
%\begin{quote}
%\begin{verbatim}
%\epstopdfDeclareGraphicsRule{.gif}{png}{.png}{%
%  convert #1 \OutputFile
%}
%\end{verbatim}
%\end{quote}
% The file extension \xfile{.gif} can be added to the extension list that
% package \xpackage{graphics} searches if the file extension
% is not given in \cs{includegraphics}.
% The list can be set by \cs{GraphicsExtensions}.
% \begin{quote}
% |\AppendGraphicsExtensions{.gif}|\\
% or\\
% |\PrependGraphicsExtensions{.gif}|
% \end{quote}
%
% \StopEventually{
% }
%
% \section{Implementation}
%
%    \begin{macrocode}
%<*package>
%    \end{macrocode}
%
% \subsection{Wrapper package}
%
%    Reload check, especially if the package is not used with \LaTeX.
%    \begin{macrocode}
\begingroup\catcode61\catcode48\catcode32=10\relax%
  \catcode13=5 % ^^M
  \endlinechar=13 %
  \catcode35=6 % #
  \catcode39=12 % '
  \catcode44=12 % ,
  \catcode45=12 % -
  \catcode46=12 % .
  \catcode58=12 % :
  \catcode64=11 % @
  \catcode123=1 % {
  \catcode125=2 % }
  \expandafter\let\expandafter\x\csname ver@epstopdf.sty\endcsname
  \ifx\x\relax % plain-TeX, first loading
  \else
    \def\empty{}%
    \ifx\x\empty % LaTeX, first loading,
      % variable is initialized, but \ProvidesPackage not yet seen
    \else
      \expandafter\ifx\csname PackageInfo\endcsname\relax
        \def\x#1#2{%
          \immediate\write-1{Package #1 Info: #2.}%
        }%
      \else
        \def\x#1#2{\PackageInfo{#1}{#2, stopped}}%
      \fi
      \x{epstopdf}{The package is already loaded}%
      \aftergroup\endinput
    \fi
  \fi
\endgroup%
%    \end{macrocode}
%    Package identification:
%    \begin{macrocode}
\begingroup\catcode61\catcode48\catcode32=10\relax%
  \catcode13=5 % ^^M
  \endlinechar=13 %
  \catcode35=6 % #
  \catcode39=12 % '
  \catcode40=12 % (
  \catcode41=12 % )
  \catcode44=12 % ,
  \catcode45=12 % -
  \catcode46=12 % .
  \catcode47=12 % /
  \catcode58=12 % :
  \catcode64=11 % @
  \catcode91=12 % [
  \catcode93=12 % ]
  \catcode123=1 % {
  \catcode125=2 % }
  \expandafter\ifx\csname ProvidesPackage\endcsname\relax
    \def\x#1#2#3[#4]{\endgroup
      \immediate\write-1{Package: #3 #4}%
      \xdef#1{#4}%
    }%
  \else
    \def\x#1#2[#3]{\endgroup
      #2[{#3}]%
      \ifx#1\@undefined
        \xdef#1{#3}%
      \fi
      \ifx#1\relax
        \xdef#1{#3}%
      \fi
    }%
  \fi
\expandafter\x\csname ver@epstopdf.sty\endcsname
\ProvidesPackage{epstopdf}%
  [2016/05/15 v2.6 Conversion with epstopdf on the fly (HO)]%
%    \end{macrocode}
%    Larger catcode set because of configuration files needed.
%    \begin{macrocode}
\begingroup\catcode61\catcode48\catcode32=10\relax%
  \catcode13=5 % ^^M
  \endlinechar=13 %
  \catcode123=1 % {
  \catcode125=2 % }
  \catcode64=11 % @
  \def\x{\endgroup
    \expandafter\edef\csname ETE@AtEnd\endcsname{%
      \endlinechar=\the\endlinechar\relax
      \catcode13=\the\catcode13\relax
      \catcode32=\the\catcode32\relax
      \catcode35=\the\catcode35\relax
      \catcode61=\the\catcode61\relax
      \catcode64=\the\catcode64\relax
      \catcode123=\the\catcode123\relax
      \catcode125=\the\catcode125\relax
    }%
  }%
\x\catcode61\catcode48\catcode32=10\relax%
\catcode13=5 % ^^M
\endlinechar=13 %
\catcode35=6 % #
\catcode64=11 % @
\catcode123=1 % {
\catcode125=2 % }
\def\TMP@EnsureCode#1#2{%
  \edef\ETE@AtEnd{%
    \ETE@AtEnd
    \catcode#1=\the\catcode#1\relax
  }%
  \catcode#1=#2\relax
}
\TMP@EnsureCode{39}{12}% '
\TMP@EnsureCode{40}{12}% (
\TMP@EnsureCode{41}{12}% )
\TMP@EnsureCode{42}{12}% *
\TMP@EnsureCode{45}{12}% -
\TMP@EnsureCode{47}{12}% /
\TMP@EnsureCode{91}{12}% [
\TMP@EnsureCode{93}{12}% ]
\edef\ETE@AtEnd{\ETE@AtEnd\noexpand\endinput}
%    \end{macrocode}
%
%    \begin{macrocode}
\let\ETE@SavedAtEnd\ETE@AtEnd
\RequirePackage{epstopdf-base}[2016/05/15]
\let\ETE@AtEnd\ETE@SavedAtEnd
%    \end{macrocode}
%
% \subsubsection{Option handling}
%
%    \begin{macrocode}
\DeclareOption*{%
  \expandafter\epstopdfsetup\expandafter{\CurrentOption}%
}
\ProcessOptions*\relax
%    \end{macrocode}
%
%    \begin{macrocode}
\ETE@AtEnd%
%</package>
%    \end{macrocode}
%
% \subsection{Base package}
%
%    \begin{macrocode}
%<*base>
%    \end{macrocode}
%
% \subsection{Preparations}
%
% \subsubsection{Relead check and identification}
%
%    Reload check, especially if the package is not used with \LaTeX.
%    \begin{macrocode}
\begingroup\catcode61\catcode48\catcode32=10\relax%
  \catcode13=5 % ^^M
  \endlinechar=13 %
  \catcode35=6 % #
  \catcode39=12 % '
  \catcode44=12 % ,
  \catcode45=12 % -
  \catcode46=12 % .
  \catcode58=12 % :
  \catcode64=11 % @
  \catcode123=1 % {
  \catcode125=2 % }
  \expandafter\let\expandafter\x\csname ver@epstopdf-base.sty\endcsname
  \ifx\x\relax % plain-TeX, first loading
  \else
    \def\empty{}%
    \ifx\x\empty % LaTeX, first loading,
      % variable is initialized, but \ProvidesPackage not yet seen
    \else
      \expandafter\ifx\csname PackageInfo\endcsname\relax
        \def\x#1#2{%
          \immediate\write-1{Package #1 Info: #2.}%
        }%
      \else
        \def\x#1#2{\PackageInfo{#1}{#2, stopped}}%
      \fi
      \x{epstopdf-base}{The package is already loaded}%
      \aftergroup\endinput
    \fi
  \fi
\endgroup%
%    \end{macrocode}
%    Package identification:
%    \begin{macrocode}
\begingroup\catcode61\catcode48\catcode32=10\relax%
  \catcode13=5 % ^^M
  \endlinechar=13 %
  \catcode35=6 % #
  \catcode39=12 % '
  \catcode40=12 % (
  \catcode41=12 % )
  \catcode44=12 % ,
  \catcode45=12 % -
  \catcode46=12 % .
  \catcode47=12 % /
  \catcode58=12 % :
  \catcode64=11 % @
  \catcode91=12 % [
  \catcode93=12 % ]
  \catcode123=1 % {
  \catcode125=2 % }
  \expandafter\ifx\csname ProvidesPackage\endcsname\relax
    \def\x#1#2#3[#4]{\endgroup
      \immediate\write-1{Package: #3 #4}%
      \xdef#1{#4}%
    }%
  \else
    \def\x#1#2[#3]{\endgroup
      #2[{#3}]%
      \ifx#1\@undefined
        \xdef#1{#3}%
      \fi
      \ifx#1\relax
        \xdef#1{#3}%
      \fi
    }%
  \fi
\expandafter\x\csname ver@epstopdf-base.sty\endcsname
\ProvidesPackage{epstopdf-base}%
  [2016/05/15 v2.6 Base part for package epstopdf]%
%    \end{macrocode}
%
% \subsubsection{Catcodes}
%
%    \begin{macrocode}
\begingroup\catcode61\catcode48\catcode32=10\relax%
  \catcode13=5 % ^^M
  \endlinechar=13 %
  \catcode123=1 % {
  \catcode125=2 % }
  \catcode64=11 % @
  \def\x{\endgroup
    \expandafter\edef\csname ETE@AtEnd\endcsname{%
      \endlinechar=\the\endlinechar\relax
      \catcode13=\the\catcode13\relax
      \catcode32=\the\catcode32\relax
      \catcode35=\the\catcode35\relax
      \catcode61=\the\catcode61\relax
      \catcode64=\the\catcode64\relax
      \catcode123=\the\catcode123\relax
      \catcode125=\the\catcode125\relax
    }%
  }%
\x\catcode61\catcode48\catcode32=10\relax%
\catcode13=5 % ^^M
\endlinechar=13 %
\catcode35=6 % #
\catcode64=11 % @
\catcode123=1 % {
\catcode125=2 % }
\def\TMP@EnsureCode#1#2{%
  \edef\ETE@AtEnd{%
    \ETE@AtEnd
    \catcode#1=\the\catcode#1\relax
  }%
  \catcode#1=#2\relax
}
\TMP@EnsureCode{33}{12}% !
\TMP@EnsureCode{39}{12}% '
\TMP@EnsureCode{42}{12}% *
\TMP@EnsureCode{44}{12}% ,
\TMP@EnsureCode{45}{12}% -
\TMP@EnsureCode{46}{12}% .
\TMP@EnsureCode{47}{12}% /
\TMP@EnsureCode{58}{12}% :
\TMP@EnsureCode{60}{12}% <
\TMP@EnsureCode{62}{12}% >
\TMP@EnsureCode{96}{12}% `
\edef\ETE@AtEnd{\ETE@AtEnd\noexpand\endinput}
%    \end{macrocode}
%
% \subsubsection{Load packages}
%
%    \begin{macrocode}
\RequirePackage{infwarerr}[2007/09/09]
\RequirePackage{grfext}\relax
\RequirePackage{kvoptions}[2007/10/02]
\RequirePackage{pdftexcmds}[2007/11/11]
%    \end{macrocode}
%
% \subsection{Checks}
%
% Check, whether package graphics is loaded
% (also graphicx loads graphics).
% Because miniltx.tex does not know \cs{@ifpackageloaded}
% we test for \cs{Gin@setfile} instead.
%    \begin{macrocode}
\begingroup\expandafter\expandafter\expandafter\endgroup
\expandafter\ifx\csname Gin@setfile\endcsname\relax
  \@PackageWarningNoLine{epstopdf}{%
    No graphics package \string`graphic{s,x}\string' loaded%
  }%
  \newcommand*{\epstopdfsetup}[1]{}%
  \expandafter\ETE@AtEnd
\fi%
%    \end{macrocode}
%
% Check, whether pdftex.def is loaded. \cs{ver@pdftex.def} is not
% available with miniltx.tex, thus we test for \cs{Gin@driver}.
%    \begin{macrocode}
\begingroup
  \def\x{luatex.def}%
  \ifx\Gin@driver\x
  \else
  \def\x{pdftex.def}%
  \ifx\Gin@driver\x
  \else
    \@PackageWarningNoLine{epstopdf}{%
      Drivers other than `pdftex' and `luatex' are not supported%
    }%
    \endgroup
    \newcommand*{\epstopdfsetup}[1]{}%
    \expandafter\ETE@AtEnd
  \fi%
  \fi%
\endgroup
%    \end{macrocode}
%
% Check, whether the shell escape feature is enabled.
%
%    \begin{macrocode}
\begingroup
  \expandafter\ifx\csname pdf@shellescape\endcsname\relax
  \else
    \ifnum\pdf@shellescape>0 %
    \else
      \@PackageWarningNoLine{epstopdf}{%
        Shell escape feature is not enabled%
      }%
    \fi
  \fi
\endgroup
%    \end{macrocode}
%
% \subsection{Package loading}
%
%    \begin{macrocode}
%    \end{macrocode}
%
% \subsection{Options}
%
%    \begin{macrocode}
\SetupKeyvalOptions{family=ETE,prefix=ETE@}
\DeclareBoolOption{update}
\DeclareBoolOption{verbose}
\newif\ifETE@prepend
\DeclareVoidOption{prepend}{\ETE@prependtrue}
\DeclareVoidOption{append}{\ETE@prependfalse}
\DeclareStringOption{outdir}
\DeclareStringOption{suffix}
\DeclareBoolOption{prefersuffix}
\DeclareStringOption{program@epstopdf}
%    \end{macrocode}
%    Options \xoption{disable} and \xoption{enable} are for testing
%    only. Therefore they are not documented on purpose.
%    \begin{macrocode}
\DeclareBoolOption{disable}
\DeclareComplementaryOption{enable}{disable}
%    \end{macrocode}
%    \begin{macrocode}
\newcommand*{\epstopdfsetup}{\setkeys{ETE}}
%    \end{macrocode}
%
% \subsubsection{Default setting}
%
%    \begin{macrocode}
\epstopdfsetup{%
  verbose,%
  enable,%
  append,%
  update,%
  prefersuffix,%
  suffix=-\SourceExt-converted-to,%
  program@epstopdf=epstopdf%
}
%    \end{macrocode}
%
% \subsection{Make and verbose}
%
%    \begin{macrocode}
\begingroup\expandafter\expandafter\expandafter\endgroup
\expandafter\ifx\csname pdf@filemoddate\endcsname\relax
  \def\ETE@Make#1#2{%
    \ifETE@update
      \ETE@WarnModDate
    \fi
    \@firstofone
  }%
  \def\ETE@WarnModDate{%
    \@PackageWarningNoLine{epstopdf}{%
      \string\pdffilemoddate\space is not available,\MessageBreak
      option `update' will be ignored%
    }%
    \global\let\ETE@WarnModDate\relax
  }%
  \def\ETE@FileInfo#1#2{#1 file: <#2>}%
\else
  \def\ETE@Make#1#2{%
    \ifETE@update
      \ifnum\pdf@strcmp{\pdf@filemoddate{#1}}{\pdf@filemoddate{#2}}>0 %
        \expandafter\expandafter\expandafter\@firstofone
      \else
        \@PackageInfoNoLine{epstopdf}{%
          Output file is already uptodate%
        }%
        \expandafter\expandafter\expandafter\@gobble
      \fi
    \else
      \expandafter\@firstofone
    \fi
  }%
  \def\ETE@FileInfo#1#2{%
    #1 file: <#2>%
    \expandafter\expandafter\expandafter
    \ETE@Date\pdf@filemoddate{#2}\@nil
    \expandafter\expandafter\expandafter
    \ETE@Size\pdf@filesize{#2}\@nil
  }%
  \def\ETE@Date#1\@nil{%
    \ifx\\#1\\%
    \else
      \ETE@@Date#1\@nil
    \fi
  }%
  \def\ETE@@Date#1:#2#3#4#5#6#7#8#9{%
    \MessageBreak
    \@spaces\space\space\space date: #2#3#4#5-#6#7-#8#9 %
    \ETE@@Time
  }%
  \def\ETE@@Time#1#2#3#4#5#6#7\@nil{%
    #1#2:#3#4:#5#6%
  }%
  \def\ETE@Size#1\@nil{%
    \ifx\\#1\\%
    \else
      \MessageBreak
      \@spaces\space\space\space size: #1 bytes%
    \fi
  }%
\fi
%    \end{macrocode}
%
% \subsection{Adding conversion support}
%
%    Patch \cs{Gin@setfile} to execute |#3|, if it contains
%    a command.
%    \begin{macrocode}
\expandafter\ifx\csname ETE@OrgGin@setfile\endcsname\relax
  \let\ETE@OrgGin@setfile\Gin@setfile
\else
  \@PackageError{epstopdf}{%
    Command \string\ETE@OrgGin@setfile\space
    already defined.\MessageBreak
  }{%
    Probably some package has included the code of this package%
    \MessageBreak
    instead of using \string\RequirePackage{epstopdf}.%
    \MessageBreak
    \@ehc
  }%
\fi
\def\ETE@IfFileExists{%
  \begingroup\expandafter\expandafter\expandafter\endgroup
  \expandafter\ifx\csname grffile@IfFileExists\endcsname\relax
    \expandafter\IfFileExists
  \else
    \global\let\ETE@IfFileExists\grffile@IfFileExists
    \expandafter\grffile@IfFileExists
  \fi
}
\def\ETE@Skip#1#2\x\fi\fi{%
  \fi
  \fi
  \endgroup
  \fi
  \fi
  #1%
}
\newif\ifETE@InsideSetfile
\newcommand*{\epstopdfcall}[1]{%
  \ifETE@InsideSetfile
    \expandafter\@firstoftwo
  \else
    \expandafter\@secondoftwo
  \fi
  {`#1}%
  {\Gin@base\Gin@ext}%
}
\def\ETE@DefCommandLine#1{%
  \edef\CommandLine{\expandafter\fi\if`#1}%
}
\def\ETE@DefX#1{%
  \expandafter\expandafter\expandafter\def
  \expandafter\expandafter\expandafter\x
  \expandafter\expandafter\expandafter{%
    \expandafter\fi\if`#1\relax\else
  }%
}
\def\ETE@Gin@setfile#1#2#3{%
  \ifETE@disable
    \ETE@OrgGin@setfile{#1}{#2}{#3}%
  \else
    \begingroup
      \ETE@InsideSetfiletrue
      \ETE@DefX{#3}%
    \expandafter\endgroup
    \ifx\x\@empty
      \ETE@OrgGin@setfile{#1}{#2}{#3}%
    \else
      \begingroup
        \ETE@InsideSetfiletrue
        \def\GraphicsType{#1}%
        \def\GraphicsRead{#2}%
        \ifETE@prefersuffix
        \else
          \ifx\ETE@suffix\@empty
          \else
            \ETE@IfFileExists{\Gin@base\GraphicsRead}{%
              \ETE@Skip{%
                \ETE@OrgGin@setfile{#1}{#2}{\Gin@base#2}%
              }%
            }{%
              \let\next\relax
            }%
            \next
          \fi
        \fi
        \ifx\Gin@ext\relax
          \let\SourceExt\Gin@eext
          \def\SourceFile{\Gin@base\Gin@eext}%
        \else
          \let\SourceExt\Gin@ext
          \def\SourceFile{\Gin@base\Gin@ext}%
        \fi
        \edef\SourceExt{% remove dot
          \expandafter\@cdr\SourceExt\@empty\@nil
        }%
        \let\OutputDirectory\ETE@outdir
        \ifx\OutputDirectory\@empty
          \edef\OutputFile{\ETE@GenerateName{\Gin@base}{#2}}%
        \else
          \begingroup
            \filename@parse{\Gin@base#2}%
            \edef\x{\endgroup
              \def\noexpand\OutputFile{%
                \ETE@GenerateName{%
                  \OutputDirectory\filename@base
                }{#2}%
              }%
            }%
          \x
        \fi
        \ETE@DefCommandLine{#3}%
        \ifETE@verbose
          \@PackageInfo{epstopdf}{%
            \ETE@FileInfo{Source}\SourceFile\MessageBreak
            \ETE@FileInfo{Output}\OutputFile\MessageBreak
            Command: <\CommandLine>\MessageBreak
            \string\includegraphics
          }%
        \fi
        \ETE@Make\SourceFile\OutputFile{%
          \pdf@system{\CommandLine}%
          \ifETE@verbose
            \@PackageInfoNoLine{epstopdf}{%
              \ETE@FileInfo{Result}\OutputFile
            }%
          \fi
        }%
        \edef\x{\endgroup
          \ifx\OutputDirectory\@empty
          \else
            \def\noexpand\Gin@base{%
              \OutputDirectory\noexpand\filename@base
            }%
          \fi
          \ifx\ETE@suffix\@empty
          \else
            \edef\noexpand\Gin@base{%
              \noexpand\Gin@base\ETE@suffix
            }%
          \fi
          \noexpand\ETE@OrgGin@setfile{%
            \GraphicsType
          }{%
            \GraphicsRead
          }{%
            \OutputFile
          }%
        }%
      \x
    \fi
  \fi
}
\let\Gin@setfile\ETE@Gin@setfile
\def\ETE@GenerateName#1#2{%
  #1\ETE@suffix#2%
}
%    \end{macrocode}
%
% \subsection{Declare graphics rule}
%
%    \begin{macrocode}
\newcommand*{\epstopdfDeclareGraphicsRule}[4]{%
  \ifx\\#4\\%
    \@PackageError{epstopdf-base}{%
      Conversion command is missing%
    }\@ehc
  \else
    \begingroup
      \@ifundefined{Gin@rule@#1}{%
      }{%
        \@PackageInfo{epstopdf-base}{%
          Redefining graphics rule for `#1'%
        }%
      }%
    \endgroup
    \@namedef{Gin@rule@#1}##1{{#2}{#3}{\epstopdfcall{#4}}}%
  \fi
}
%    \end{macrocode}
%
% \cs{DeclareGraphicsRule} for .eps
%    \begin{macrocode}
\epstopdfDeclareGraphicsRule{.eps}{pdf}{.pdf}{%
  \ETE@epstopdf{#1}%
}
\def\ETE@epstopdf#1{%
  \ETE@program@epstopdf\space
  \ifcase\ifx\OutputDirectory\@empty
           \ifx\ETE@suffix\@empty
             1%
           \fi
         \fi
         0 %
    --outfile=\OutputFile\space
  \fi
  #1%
}
%    \end{macrocode}
%
%    \begin{macrocode}
\ifETE@prepend
  \expandafter\PrependGraphicsExtensions
\else
  \expandafter\AppendGraphicsExtensions
\fi
{.eps}
\let\ETE@prepend\@undefined
\DeclareVoidOption{prepend}{%
  \PrependGraphicsExtensions{.eps}%
}
\let\ETE@append\@undefined
\DeclareVoidOption{append}{%
  \AppendGraphicsExtensions{.eps}%
}
%    \end{macrocode}
%
%    \begin{macrocode}
\InputIfFileExists{epstopdf-sys.cfg}{}{}
\InputIfFileExists{epstopdf.cfg}{}{}
%    \end{macrocode}
%
%    \begin{macrocode}
\ETE@AtEnd%
%</base>
%    \end{macrocode}
%
% \section{Test}
%
% \subsection{Preface for standard catcode check}
%
%    \begin{macrocode}
%<*test1>
\input miniltx.tex\relax
\def\Gin@driver{pdftex.def}
\input graphicx.sty\relax
\resetatcatcode
%</test1>
%    \end{macrocode}
%
% \subsection{Catcode checks for loading}
%
%    \begin{macrocode}
%<*test1>
%    \end{macrocode}
%    \begin{macrocode}
\catcode`\{=1 %
\catcode`\}=2 %
\catcode`\#=6 %
\catcode`\@=11 %
\expandafter\ifx\csname count@\endcsname\relax
  \countdef\count@=255 %
\fi
\expandafter\ifx\csname @gobble\endcsname\relax
  \long\def\@gobble#1{}%
\fi
\expandafter\ifx\csname @firstofone\endcsname\relax
  \long\def\@firstofone#1{#1}%
\fi
\expandafter\ifx\csname loop\endcsname\relax
  \expandafter\@firstofone
\else
  \expandafter\@gobble
\fi
{%
  \def\loop#1\repeat{%
    \def\body{#1}%
    \iterate
  }%
  \def\iterate{%
    \body
      \let\next\iterate
    \else
      \let\next\relax
    \fi
    \next
  }%
  \let\repeat=\fi
}%
\def\RestoreCatcodes{}
\count@=0 %
\loop
  \edef\RestoreCatcodes{%
    \RestoreCatcodes
    \catcode\the\count@=\the\catcode\count@\relax
  }%
\ifnum\count@<255 %
  \advance\count@ 1 %
\repeat

\def\RangeCatcodeInvalid#1#2{%
  \count@=#1\relax
  \loop
    \catcode\count@=15 %
  \ifnum\count@<#2\relax
    \advance\count@ 1 %
  \repeat
}
\def\RangeCatcodeCheck#1#2#3{%
  \count@=#1\relax
  \loop
    \ifnum#3=\catcode\count@
    \else
      \errmessage{%
        Character \the\count@\space
        with wrong catcode \the\catcode\count@\space
        instead of \number#3%
      }%
    \fi
  \ifnum\count@<#2\relax
    \advance\count@ 1 %
  \repeat
}
\def\space{ }
\expandafter\ifx\csname LoadCommand\endcsname\relax
  \def\LoadCommand{\input epstopdf.sty\relax}%
\fi
\def\Test{%
  \RangeCatcodeInvalid{0}{47}%
  \RangeCatcodeInvalid{58}{64}%
  \RangeCatcodeInvalid{91}{96}%
  \RangeCatcodeInvalid{123}{255}%
  \catcode`\@=12 %
  \catcode`\\=0 %
  \catcode`\%=14 %
  \LoadCommand
  \RangeCatcodeCheck{0}{36}{15}%
  \RangeCatcodeCheck{37}{37}{14}%
  \RangeCatcodeCheck{38}{47}{15}%
  \RangeCatcodeCheck{48}{57}{12}%
  \RangeCatcodeCheck{58}{63}{15}%
  \RangeCatcodeCheck{64}{64}{12}%
  \RangeCatcodeCheck{65}{90}{11}%
  \RangeCatcodeCheck{91}{91}{15}%
  \RangeCatcodeCheck{92}{92}{0}%
  \RangeCatcodeCheck{93}{96}{15}%
  \RangeCatcodeCheck{97}{122}{11}%
  \RangeCatcodeCheck{123}{255}{15}%
  \RestoreCatcodes
}
\Test
\csname @@end\endcsname
\end
%    \end{macrocode}
%    \begin{macrocode}
%</test1>
%    \end{macrocode}
%
% \section{Installation}
%
% \subsection{Download}
%
% \paragraph{Package.} This package is available on
% CTAN\footnote{\url{http://ctan.org/pkg/epstopdf}}:
% \begin{description}
% \item[\CTAN{macros/latex/contrib/oberdiek/epstopdf.dtx}] The source file.
% \item[\CTAN{macros/latex/contrib/oberdiek/epstopdf.pdf}] Documentation.
% \end{description}
%
%
% \paragraph{Bundle.} All the packages of the bundle `oberdiek'
% are also available in a TDS compliant ZIP archive. There
% the packages are already unpacked and the documentation files
% are generated. The files and directories obey the TDS standard.
% \begin{description}
% \item[\CTAN{install/macros/latex/contrib/oberdiek.tds.zip}]
% \end{description}
% \emph{TDS} refers to the standard ``A Directory Structure
% for \TeX\ Files'' (\CTAN{tds/tds.pdf}). Directories
% with \xfile{texmf} in their name are usually organized this way.
%
% \subsection{Bundle installation}
%
% \paragraph{Unpacking.} Unpack the \xfile{oberdiek.tds.zip} in the
% TDS tree (also known as \xfile{texmf} tree) of your choice.
% Example (linux):
% \begin{quote}
%   |unzip oberdiek.tds.zip -d ~/texmf|
% \end{quote}
%
% \paragraph{Script installation.}
% Check the directory \xfile{TDS:scripts/oberdiek/} for
% scripts that need further installation steps.
% Package \xpackage{attachfile2} comes with the Perl script
% \xfile{pdfatfi.pl} that should be installed in such a way
% that it can be called as \texttt{pdfatfi}.
% Example (linux):
% \begin{quote}
%   |chmod +x scripts/oberdiek/pdfatfi.pl|\\
%   |cp scripts/oberdiek/pdfatfi.pl /usr/local/bin/|
% \end{quote}
%
% \subsection{Package installation}
%
% \paragraph{Unpacking.} The \xfile{.dtx} file is a self-extracting
% \docstrip\ archive. The files are extracted by running the
% \xfile{.dtx} through \plainTeX:
% \begin{quote}
%   \verb|tex epstopdf.dtx|
% \end{quote}
%
% \paragraph{TDS.} Now the different files must be moved into
% the different directories in your installation TDS tree
% (also known as \xfile{texmf} tree):
% \begin{quote}
% \def\t{^^A
% \begin{tabular}{@{}>{\ttfamily}l@{ $\rightarrow$ }>{\ttfamily}l@{}}
%   epstopdf.sty & tex/latex/oberdiek/epstopdf.sty\\
%   epstopdf-base.sty & tex/latex/oberdiek/epstopdf-base.sty\\
%   epstopdf.pdf & doc/latex/oberdiek/epstopdf.pdf\\
%   test/epstopdf-test1.tex & doc/latex/oberdiek/test/epstopdf-test1.tex\\
%   epstopdf.dtx & source/latex/oberdiek/epstopdf.dtx\\
% \end{tabular}^^A
% }^^A
% \sbox0{\t}^^A
% \ifdim\wd0>\linewidth
%   \begingroup
%     \advance\linewidth by\leftmargin
%     \advance\linewidth by\rightmargin
%   \edef\x{\endgroup
%     \def\noexpand\lw{\the\linewidth}^^A
%   }\x
%   \def\lwbox{^^A
%     \leavevmode
%     \hbox to \linewidth{^^A
%       \kern-\leftmargin\relax
%       \hss
%       \usebox0
%       \hss
%       \kern-\rightmargin\relax
%     }^^A
%   }^^A
%   \ifdim\wd0>\lw
%     \sbox0{\small\t}^^A
%     \ifdim\wd0>\linewidth
%       \ifdim\wd0>\lw
%         \sbox0{\footnotesize\t}^^A
%         \ifdim\wd0>\linewidth
%           \ifdim\wd0>\lw
%             \sbox0{\scriptsize\t}^^A
%             \ifdim\wd0>\linewidth
%               \ifdim\wd0>\lw
%                 \sbox0{\tiny\t}^^A
%                 \ifdim\wd0>\linewidth
%                   \lwbox
%                 \else
%                   \usebox0
%                 \fi
%               \else
%                 \lwbox
%               \fi
%             \else
%               \usebox0
%             \fi
%           \else
%             \lwbox
%           \fi
%         \else
%           \usebox0
%         \fi
%       \else
%         \lwbox
%       \fi
%     \else
%       \usebox0
%     \fi
%   \else
%     \lwbox
%   \fi
% \else
%   \usebox0
% \fi
% \end{quote}
% If you have a \xfile{docstrip.cfg} that configures and enables \docstrip's
% TDS installing feature, then some files can already be in the right
% place, see the documentation of \docstrip.
%
% \subsection{Refresh file name databases}
%
% If your \TeX~distribution
% (\teTeX, \mikTeX, \dots) relies on file name databases, you must refresh
% these. For example, \teTeX\ users run \verb|texhash| or
% \verb|mktexlsr|.
%
% \subsection{Some details for the interested}
%
% \paragraph{Attached source.}
%
% The PDF documentation on CTAN also includes the
% \xfile{.dtx} source file. It can be extracted by
% AcrobatReader 6 or higher. Another option is \textsf{pdftk},
% e.g. unpack the file into the current directory:
% \begin{quote}
%   \verb|pdftk epstopdf.pdf unpack_files output .|
% \end{quote}
%
% \paragraph{Unpacking with \LaTeX.}
% The \xfile{.dtx} chooses its action depending on the format:
% \begin{description}
% \item[\plainTeX:] Run \docstrip\ and extract the files.
% \item[\LaTeX:] Generate the documentation.
% \end{description}
% If you insist on using \LaTeX\ for \docstrip\ (really,
% \docstrip\ does not need \LaTeX), then inform the autodetect routine
% about your intention:
% \begin{quote}
%   \verb|latex \let\install=y\input{epstopdf.dtx}|
% \end{quote}
% Do not forget to quote the argument according to the demands
% of your shell.
%
% \paragraph{Generating the documentation.}
% You can use both the \xfile{.dtx} or the \xfile{.drv} to generate
% the documentation. The process can be configured by the
% configuration file \xfile{ltxdoc.cfg}. For instance, put this
% line into this file, if you want to have A4 as paper format:
% \begin{quote}
%   \verb|\PassOptionsToClass{a4paper}{article}|
% \end{quote}
% An example follows how to generate the
% documentation with pdf\LaTeX:
% \begin{quote}
%\begin{verbatim}
%pdflatex epstopdf.dtx
%makeindex -s gind.ist epstopdf.idx
%pdflatex epstopdf.dtx
%makeindex -s gind.ist epstopdf.idx
%pdflatex epstopdf.dtx
%\end{verbatim}
% \end{quote}
%
% \section{Catalogue}
%
% The following XML file can be used as source for the
% \href{http://mirror.ctan.org/help/Catalogue/catalogue.html}{\TeX\ Catalogue}.
% The elements \texttt{caption} and \texttt{description} are imported
% from the original XML file from the Catalogue.
% The name of the XML file in the Catalogue is \xfile{epstopdf-pkg.xml}.
%    \begin{macrocode}
%<*catalogue>
<?xml version='1.0' encoding='us-ascii'?>
<!DOCTYPE entry SYSTEM 'catalogue.dtd'>
<entry datestamp='$Date$' modifier='$Author$' id='epstopdf-pkg'>
  <name>epstopdf-pkg</name>
  <caption>Call epstopdf "on the fly"</caption>
  <authorref id='auth:oberdiek'/>
  <copyright owner='Heiko Oberdiek' year='2001,2006-2010'/>
  <license type='lppl1.3'/>
  <version number='2.6'/>
  <description>
    The package adds support for EPS files in the
    <xref refid='graphicx'>graphicx</xref> package when running under
    PDFTeX.  If an EPS graphic is detected, the package spawns a
    process to convert the EPS to PDF, using the script
    <xref refid='epstopdf'>epstopdf</xref>.  This of course requires
    that shell escape is enabled for the PDFTeX run.
    <p/>
    The package is part of the <xref refid='oberdiek'>oberdiek</xref>
    bundle.
  </description>
  <documentation details='Package documentation'
      href='ctan:/macros/latex/contrib/oberdiek/epstopdf.pdf'/>
  <ctan file='true' path='/macros/latex/contrib/oberdiek/epstopdf.dtx'/>
  <miktex location='oberdiek'/>
  <texlive location='oberdiek'/>
  <install path='/macros/latex/contrib/oberdiek/oberdiek.tds.zip'/>
</entry>
%</catalogue>
%    \end{macrocode}
%
% \begin{History}
%   \begin{Version}{2001/01/06 v1.0}
%   \item
%     First public version,
%     published in the \pdfTeX\ mailing list.
%   \end{Version}
%   \begin{Version}{2001/02/04 v1.1}
%   \item
%     Minor documentation update.
%   \item
%     CTAN.
%   \end{Version}
%   \begin{Version}{2006/02/20 v1.2}
%   \item
%     DTX framework.
%   \item
%     Compatibility for \xfile{miniltx.tex}.
%   \end{Version}
%   \begin{Version}{2006/08/26 v1.3}
%   \item
%     Check for \cs{write18} if available and print a warning
%     if the feature is not enabled.
%   \end{Version}
%   \begin{Version}{2007/04/26 v1.4}
%   \item
%     Documentation rewritten and extended.
%   \end{Version}
%   \begin{Version}{2007/10/02 v1.5}
%   \item
%     New option \xoption{update}: If the converted file exists, it will be only
%     converted if it is out of date.
%   \item
%     Updating the extension list is delegated to package \xpackage{grfext}.
%     Fine tuning is done by the new options \xoption{append}, \xoption{prepend}.
%   \item
%     New option \xoption{outdir} for changing the output directory.
%   \item
%     New option \xoption{verbose}.
%   \item
%     \cs{SourceFile} and \cs{OutputFile} introduced.
%   \item
%     Configuration file support added.
%   \end{Version}
%   \begin{Version}{2007/11/11 v1.6}
%   \item
%     Use of package \xpackage{pdftexcmds} for \LuaTeX\ support.
%   \end{Version}
%   \begin{Version}{2008/05/06 v1.7}
%   \item
%     Warning messages uses ``loaded'' instead of ``found''.
%   \end{Version}
%   \begin{Version}{2009/03/01 v1.8}
%   \item
%     Warning message for missing \xfile{pdftex.def} changed.
%   \end{Version}
%   \begin{Version}{2009/07/06 v1.9}
%   \item
%     Option \xoption{suffix} added.
%   \end{Version}
%   \begin{Version}{2009/07/07 v1.10}
%   \item
%     \cs{SourceExt} added.
%   \item
%     If option \xoption{suffix} is set, the inclusion of
%     an image without the suffix namespace is preferred over generating the
%     the image within the suffix namespace.
%   \end{Version}
%   \begin{Version}{2009/07/12 v2.0}
%   \item
%     New default settings.
%   \item
%     Package is split into \xpackage{epstopdf} that only takes package
%     options and loads \xpackage{epstopdf-base} that does the work.
%   \item
%     \cs{epstopdfDeclareGraphicsRule} and \cs{epstopdfcall} added.
%   \item
%     \xfile{epstopdf-sys.cfg} is loaded before \xfile{epstopdf.cfg}
%     if \xfile{epstopdf-sys.cfg} exists.
%   \end{Version}
%   \begin{Version}{2009/07/15 v2.1}
%   \item
%     Default setting: \xoption{verbose} is now turned on as the
%     documentation for v2.0 said.
%   \item
%     Documentation fixes.
%   \end{Version}
%   \begin{Version}{2009/07/16 v2.2}
%   \item
%     Fixed redefined \cs{Gin@setfile}.
%   \item
%     Documentation extended for package options.
%   \end{Version}
%   \begin{Version}{2009/09/24 v2.3}
%   \item
%     Bug fix for the case that both option \xoption{suffix} and
%     \xoption{outdir} are used.
%   \end{Version}
%   \begin{Version}{2009/10/17 v2.4}
%   \item
%     The name of the program `epstopdf' can be configured via
%     the new option \xoption{program@epstopdf}.
%   \end{Version}
%   \begin{Version}{2016/05/15 v2.5}
%   \item
%     Wording of warning message fixed (Karl Berry).
%   \item
%     \cs{ETE@Gin@setfile} added (Karl Berry).
%   \end{Version}
%   \begin{Version}{2016/05/15 v2.6}
%   \item
%     luaTeX compatibility
%   \end{Version}
% \end{History}
%
% \PrintIndex
%
% \Finale
\endinput
|
% \end{quote}
% Do not forget to quote the argument according to the demands
% of your shell.
%
% \paragraph{Generating the documentation.}
% You can use both the \xfile{.dtx} or the \xfile{.drv} to generate
% the documentation. The process can be configured by the
% configuration file \xfile{ltxdoc.cfg}. For instance, put this
% line into this file, if you want to have A4 as paper format:
% \begin{quote}
%   \verb|\PassOptionsToClass{a4paper}{article}|
% \end{quote}
% An example follows how to generate the
% documentation with pdf\LaTeX:
% \begin{quote}
%\begin{verbatim}
%pdflatex epstopdf.dtx
%makeindex -s gind.ist epstopdf.idx
%pdflatex epstopdf.dtx
%makeindex -s gind.ist epstopdf.idx
%pdflatex epstopdf.dtx
%\end{verbatim}
% \end{quote}
%
% \section{Catalogue}
%
% The following XML file can be used as source for the
% \href{http://mirror.ctan.org/help/Catalogue/catalogue.html}{\TeX\ Catalogue}.
% The elements \texttt{caption} and \texttt{description} are imported
% from the original XML file from the Catalogue.
% The name of the XML file in the Catalogue is \xfile{epstopdf-pkg.xml}.
%    \begin{macrocode}
%<*catalogue>
<?xml version='1.0' encoding='us-ascii'?>
<!DOCTYPE entry SYSTEM 'catalogue.dtd'>
<entry datestamp='$Date$' modifier='$Author$' id='epstopdf-pkg'>
  <name>epstopdf-pkg</name>
  <caption>Call epstopdf "on the fly"</caption>
  <authorref id='auth:oberdiek'/>
  <copyright owner='Heiko Oberdiek' year='2001,2006-2010'/>
  <license type='lppl1.3'/>
  <version number='2.6'/>
  <description>
    The package adds support for EPS files in the
    <xref refid='graphicx'>graphicx</xref> package when running under
    PDFTeX.  If an EPS graphic is detected, the package spawns a
    process to convert the EPS to PDF, using the script
    <xref refid='epstopdf'>epstopdf</xref>.  This of course requires
    that shell escape is enabled for the PDFTeX run.
    <p/>
    The package is part of the <xref refid='oberdiek'>oberdiek</xref>
    bundle.
  </description>
  <documentation details='Package documentation'
      href='ctan:/macros/latex/contrib/oberdiek/epstopdf.pdf'/>
  <ctan file='true' path='/macros/latex/contrib/oberdiek/epstopdf.dtx'/>
  <miktex location='oberdiek'/>
  <texlive location='oberdiek'/>
  <install path='/macros/latex/contrib/oberdiek/oberdiek.tds.zip'/>
</entry>
%</catalogue>
%    \end{macrocode}
%
% \begin{History}
%   \begin{Version}{2001/01/06 v1.0}
%   \item
%     First public version,
%     published in the \pdfTeX\ mailing list.
%   \end{Version}
%   \begin{Version}{2001/02/04 v1.1}
%   \item
%     Minor documentation update.
%   \item
%     CTAN.
%   \end{Version}
%   \begin{Version}{2006/02/20 v1.2}
%   \item
%     DTX framework.
%   \item
%     Compatibility for \xfile{miniltx.tex}.
%   \end{Version}
%   \begin{Version}{2006/08/26 v1.3}
%   \item
%     Check for \cs{write18} if available and print a warning
%     if the feature is not enabled.
%   \end{Version}
%   \begin{Version}{2007/04/26 v1.4}
%   \item
%     Documentation rewritten and extended.
%   \end{Version}
%   \begin{Version}{2007/10/02 v1.5}
%   \item
%     New option \xoption{update}: If the converted file exists, it will be only
%     converted if it is out of date.
%   \item
%     Updating the extension list is delegated to package \xpackage{grfext}.
%     Fine tuning is done by the new options \xoption{append}, \xoption{prepend}.
%   \item
%     New option \xoption{outdir} for changing the output directory.
%   \item
%     New option \xoption{verbose}.
%   \item
%     \cs{SourceFile} and \cs{OutputFile} introduced.
%   \item
%     Configuration file support added.
%   \end{Version}
%   \begin{Version}{2007/11/11 v1.6}
%   \item
%     Use of package \xpackage{pdftexcmds} for \LuaTeX\ support.
%   \end{Version}
%   \begin{Version}{2008/05/06 v1.7}
%   \item
%     Warning messages uses ``loaded'' instead of ``found''.
%   \end{Version}
%   \begin{Version}{2009/03/01 v1.8}
%   \item
%     Warning message for missing \xfile{pdftex.def} changed.
%   \end{Version}
%   \begin{Version}{2009/07/06 v1.9}
%   \item
%     Option \xoption{suffix} added.
%   \end{Version}
%   \begin{Version}{2009/07/07 v1.10}
%   \item
%     \cs{SourceExt} added.
%   \item
%     If option \xoption{suffix} is set, the inclusion of
%     an image without the suffix namespace is preferred over generating the
%     the image within the suffix namespace.
%   \end{Version}
%   \begin{Version}{2009/07/12 v2.0}
%   \item
%     New default settings.
%   \item
%     Package is split into \xpackage{epstopdf} that only takes package
%     options and loads \xpackage{epstopdf-base} that does the work.
%   \item
%     \cs{epstopdfDeclareGraphicsRule} and \cs{epstopdfcall} added.
%   \item
%     \xfile{epstopdf-sys.cfg} is loaded before \xfile{epstopdf.cfg}
%     if \xfile{epstopdf-sys.cfg} exists.
%   \end{Version}
%   \begin{Version}{2009/07/15 v2.1}
%   \item
%     Default setting: \xoption{verbose} is now turned on as the
%     documentation for v2.0 said.
%   \item
%     Documentation fixes.
%   \end{Version}
%   \begin{Version}{2009/07/16 v2.2}
%   \item
%     Fixed redefined \cs{Gin@setfile}.
%   \item
%     Documentation extended for package options.
%   \end{Version}
%   \begin{Version}{2009/09/24 v2.3}
%   \item
%     Bug fix for the case that both option \xoption{suffix} and
%     \xoption{outdir} are used.
%   \end{Version}
%   \begin{Version}{2009/10/17 v2.4}
%   \item
%     The name of the program `epstopdf' can be configured via
%     the new option \xoption{program@epstopdf}.
%   \end{Version}
%   \begin{Version}{2016/05/15 v2.5}
%   \item
%     Wording of warning message fixed (Karl Berry).
%   \item
%     \cs{ETE@Gin@setfile} added (Karl Berry).
%   \end{Version}
%   \begin{Version}{2016/05/15 v2.6}
%   \item
%     luaTeX compatibility
%   \end{Version}
% \end{History}
%
% \PrintIndex
%
% \Finale
\endinput
|
% \end{quote}
% Do not forget to quote the argument according to the demands
% of your shell.
%
% \paragraph{Generating the documentation.}
% You can use both the \xfile{.dtx} or the \xfile{.drv} to generate
% the documentation. The process can be configured by the
% configuration file \xfile{ltxdoc.cfg}. For instance, put this
% line into this file, if you want to have A4 as paper format:
% \begin{quote}
%   \verb|\PassOptionsToClass{a4paper}{article}|
% \end{quote}
% An example follows how to generate the
% documentation with pdf\LaTeX:
% \begin{quote}
%\begin{verbatim}
%pdflatex epstopdf.dtx
%makeindex -s gind.ist epstopdf.idx
%pdflatex epstopdf.dtx
%makeindex -s gind.ist epstopdf.idx
%pdflatex epstopdf.dtx
%\end{verbatim}
% \end{quote}
%
% \section{Catalogue}
%
% The following XML file can be used as source for the
% \href{http://mirror.ctan.org/help/Catalogue/catalogue.html}{\TeX\ Catalogue}.
% The elements \texttt{caption} and \texttt{description} are imported
% from the original XML file from the Catalogue.
% The name of the XML file in the Catalogue is \xfile{epstopdf-pkg.xml}.
%    \begin{macrocode}
%<*catalogue>
<?xml version='1.0' encoding='us-ascii'?>
<!DOCTYPE entry SYSTEM 'catalogue.dtd'>
<entry datestamp='$Date$' modifier='$Author$' id='epstopdf-pkg'>
  <name>epstopdf-pkg</name>
  <caption>Call epstopdf "on the fly"</caption>
  <authorref id='auth:oberdiek'/>
  <copyright owner='Heiko Oberdiek' year='2001,2006-2010'/>
  <license type='lppl1.3'/>
  <version number='2.6'/>
  <description>
    The package adds support for EPS files in the
    <xref refid='graphicx'>graphicx</xref> package when running under
    PDFTeX.  If an EPS graphic is detected, the package spawns a
    process to convert the EPS to PDF, using the script
    <xref refid='epstopdf'>epstopdf</xref>.  This of course requires
    that shell escape is enabled for the PDFTeX run.
    <p/>
    The package is part of the <xref refid='oberdiek'>oberdiek</xref>
    bundle.
  </description>
  <documentation details='Package documentation'
      href='ctan:/macros/latex/contrib/oberdiek/epstopdf.pdf'/>
  <ctan file='true' path='/macros/latex/contrib/oberdiek/epstopdf.dtx'/>
  <miktex location='oberdiek'/>
  <texlive location='oberdiek'/>
  <install path='/macros/latex/contrib/oberdiek/oberdiek.tds.zip'/>
</entry>
%</catalogue>
%    \end{macrocode}
%
% \begin{History}
%   \begin{Version}{2001/01/06 v1.0}
%   \item
%     First public version,
%     published in the \pdfTeX\ mailing list.
%   \end{Version}
%   \begin{Version}{2001/02/04 v1.1}
%   \item
%     Minor documentation update.
%   \item
%     CTAN.
%   \end{Version}
%   \begin{Version}{2006/02/20 v1.2}
%   \item
%     DTX framework.
%   \item
%     Compatibility for \xfile{miniltx.tex}.
%   \end{Version}
%   \begin{Version}{2006/08/26 v1.3}
%   \item
%     Check for \cs{write18} if available and print a warning
%     if the feature is not enabled.
%   \end{Version}
%   \begin{Version}{2007/04/26 v1.4}
%   \item
%     Documentation rewritten and extended.
%   \end{Version}
%   \begin{Version}{2007/10/02 v1.5}
%   \item
%     New option \xoption{update}: If the converted file exists, it will be only
%     converted if it is out of date.
%   \item
%     Updating the extension list is delegated to package \xpackage{grfext}.
%     Fine tuning is done by the new options \xoption{append}, \xoption{prepend}.
%   \item
%     New option \xoption{outdir} for changing the output directory.
%   \item
%     New option \xoption{verbose}.
%   \item
%     \cs{SourceFile} and \cs{OutputFile} introduced.
%   \item
%     Configuration file support added.
%   \end{Version}
%   \begin{Version}{2007/11/11 v1.6}
%   \item
%     Use of package \xpackage{pdftexcmds} for \LuaTeX\ support.
%   \end{Version}
%   \begin{Version}{2008/05/06 v1.7}
%   \item
%     Warning messages uses ``loaded'' instead of ``found''.
%   \end{Version}
%   \begin{Version}{2009/03/01 v1.8}
%   \item
%     Warning message for missing \xfile{pdftex.def} changed.
%   \end{Version}
%   \begin{Version}{2009/07/06 v1.9}
%   \item
%     Option \xoption{suffix} added.
%   \end{Version}
%   \begin{Version}{2009/07/07 v1.10}
%   \item
%     \cs{SourceExt} added.
%   \item
%     If option \xoption{suffix} is set, the inclusion of
%     an image without the suffix namespace is preferred over generating the
%     the image within the suffix namespace.
%   \end{Version}
%   \begin{Version}{2009/07/12 v2.0}
%   \item
%     New default settings.
%   \item
%     Package is split into \xpackage{epstopdf} that only takes package
%     options and loads \xpackage{epstopdf-base} that does the work.
%   \item
%     \cs{epstopdfDeclareGraphicsRule} and \cs{epstopdfcall} added.
%   \item
%     \xfile{epstopdf-sys.cfg} is loaded before \xfile{epstopdf.cfg}
%     if \xfile{epstopdf-sys.cfg} exists.
%   \end{Version}
%   \begin{Version}{2009/07/15 v2.1}
%   \item
%     Default setting: \xoption{verbose} is now turned on as the
%     documentation for v2.0 said.
%   \item
%     Documentation fixes.
%   \end{Version}
%   \begin{Version}{2009/07/16 v2.2}
%   \item
%     Fixed redefined \cs{Gin@setfile}.
%   \item
%     Documentation extended for package options.
%   \end{Version}
%   \begin{Version}{2009/09/24 v2.3}
%   \item
%     Bug fix for the case that both option \xoption{suffix} and
%     \xoption{outdir} are used.
%   \end{Version}
%   \begin{Version}{2009/10/17 v2.4}
%   \item
%     The name of the program `epstopdf' can be configured via
%     the new option \xoption{program@epstopdf}.
%   \end{Version}
%   \begin{Version}{2016/05/15 v2.5}
%   \item
%     Wording of warning message fixed (Karl Berry).
%   \item
%     \cs{ETE@Gin@setfile} added (Karl Berry).
%   \end{Version}
%   \begin{Version}{2016/05/15 v2.6}
%   \item
%     luaTeX compatibility
%   \end{Version}
% \end{History}
%
% \PrintIndex
%
% \Finale
\endinput
|
% \end{quote}
% Do not forget to quote the argument according to the demands
% of your shell.
%
% \paragraph{Generating the documentation.}
% You can use both the \xfile{.dtx} or the \xfile{.drv} to generate
% the documentation. The process can be configured by the
% configuration file \xfile{ltxdoc.cfg}. For instance, put this
% line into this file, if you want to have A4 as paper format:
% \begin{quote}
%   \verb|\PassOptionsToClass{a4paper}{article}|
% \end{quote}
% An example follows how to generate the
% documentation with pdf\LaTeX:
% \begin{quote}
%\begin{verbatim}
%pdflatex epstopdf.dtx
%makeindex -s gind.ist epstopdf.idx
%pdflatex epstopdf.dtx
%makeindex -s gind.ist epstopdf.idx
%pdflatex epstopdf.dtx
%\end{verbatim}
% \end{quote}
%
% \section{Catalogue}
%
% The following XML file can be used as source for the
% \href{http://mirror.ctan.org/help/Catalogue/catalogue.html}{\TeX\ Catalogue}.
% The elements \texttt{caption} and \texttt{description} are imported
% from the original XML file from the Catalogue.
% The name of the XML file in the Catalogue is \xfile{epstopdf-pkg.xml}.
%    \begin{macrocode}
%<*catalogue>
<?xml version='1.0' encoding='us-ascii'?>
<!DOCTYPE entry SYSTEM 'catalogue.dtd'>
<entry datestamp='$Date$' modifier='$Author$' id='epstopdf-pkg'>
  <name>epstopdf-pkg</name>
  <caption>Call epstopdf "on the fly"</caption>
  <authorref id='auth:oberdiek'/>
  <copyright owner='Heiko Oberdiek' year='2001,2006-2010'/>
  <license type='lppl1.3'/>
  <version number='2.6'/>
  <description>
    The package adds support for EPS files in the
    <xref refid='graphicx'>graphicx</xref> package when running under
    PDFTeX.  If an EPS graphic is detected, the package spawns a
    process to convert the EPS to PDF, using the script
    <xref refid='epstopdf'>epstopdf</xref>.  This of course requires
    that shell escape is enabled for the PDFTeX run.
    <p/>
    The package is part of the <xref refid='oberdiek'>oberdiek</xref>
    bundle.
  </description>
  <documentation details='Package documentation'
      href='ctan:/macros/latex/contrib/oberdiek/epstopdf.pdf'/>
  <ctan file='true' path='/macros/latex/contrib/oberdiek/epstopdf.dtx'/>
  <miktex location='oberdiek'/>
  <texlive location='oberdiek'/>
  <install path='/macros/latex/contrib/oberdiek/oberdiek.tds.zip'/>
</entry>
%</catalogue>
%    \end{macrocode}
%
% \begin{History}
%   \begin{Version}{2001/01/06 v1.0}
%   \item
%     First public version,
%     published in the \pdfTeX\ mailing list.
%   \end{Version}
%   \begin{Version}{2001/02/04 v1.1}
%   \item
%     Minor documentation update.
%   \item
%     CTAN.
%   \end{Version}
%   \begin{Version}{2006/02/20 v1.2}
%   \item
%     DTX framework.
%   \item
%     Compatibility for \xfile{miniltx.tex}.
%   \end{Version}
%   \begin{Version}{2006/08/26 v1.3}
%   \item
%     Check for \cs{write18} if available and print a warning
%     if the feature is not enabled.
%   \end{Version}
%   \begin{Version}{2007/04/26 v1.4}
%   \item
%     Documentation rewritten and extended.
%   \end{Version}
%   \begin{Version}{2007/10/02 v1.5}
%   \item
%     New option \xoption{update}: If the converted file exists, it will be only
%     converted if it is out of date.
%   \item
%     Updating the extension list is delegated to package \xpackage{grfext}.
%     Fine tuning is done by the new options \xoption{append}, \xoption{prepend}.
%   \item
%     New option \xoption{outdir} for changing the output directory.
%   \item
%     New option \xoption{verbose}.
%   \item
%     \cs{SourceFile} and \cs{OutputFile} introduced.
%   \item
%     Configuration file support added.
%   \end{Version}
%   \begin{Version}{2007/11/11 v1.6}
%   \item
%     Use of package \xpackage{pdftexcmds} for \LuaTeX\ support.
%   \end{Version}
%   \begin{Version}{2008/05/06 v1.7}
%   \item
%     Warning messages uses ``loaded'' instead of ``found''.
%   \end{Version}
%   \begin{Version}{2009/03/01 v1.8}
%   \item
%     Warning message for missing \xfile{pdftex.def} changed.
%   \end{Version}
%   \begin{Version}{2009/07/06 v1.9}
%   \item
%     Option \xoption{suffix} added.
%   \end{Version}
%   \begin{Version}{2009/07/07 v1.10}
%   \item
%     \cs{SourceExt} added.
%   \item
%     If option \xoption{suffix} is set, the inclusion of
%     an image without the suffix namespace is preferred over generating the
%     the image within the suffix namespace.
%   \end{Version}
%   \begin{Version}{2009/07/12 v2.0}
%   \item
%     New default settings.
%   \item
%     Package is split into \xpackage{epstopdf} that only takes package
%     options and loads \xpackage{epstopdf-base} that does the work.
%   \item
%     \cs{epstopdfDeclareGraphicsRule} and \cs{epstopdfcall} added.
%   \item
%     \xfile{epstopdf-sys.cfg} is loaded before \xfile{epstopdf.cfg}
%     if \xfile{epstopdf-sys.cfg} exists.
%   \end{Version}
%   \begin{Version}{2009/07/15 v2.1}
%   \item
%     Default setting: \xoption{verbose} is now turned on as the
%     documentation for v2.0 said.
%   \item
%     Documentation fixes.
%   \end{Version}
%   \begin{Version}{2009/07/16 v2.2}
%   \item
%     Fixed redefined \cs{Gin@setfile}.
%   \item
%     Documentation extended for package options.
%   \end{Version}
%   \begin{Version}{2009/09/24 v2.3}
%   \item
%     Bug fix for the case that both option \xoption{suffix} and
%     \xoption{outdir} are used.
%   \end{Version}
%   \begin{Version}{2009/10/17 v2.4}
%   \item
%     The name of the program `epstopdf' can be configured via
%     the new option \xoption{program@epstopdf}.
%   \end{Version}
%   \begin{Version}{2016/05/15 v2.5}
%   \item
%     Wording of warning message fixed (Karl Berry).
%   \item
%     \cs{ETE@Gin@setfile} added (Karl Berry).
%   \end{Version}
%   \begin{Version}{2016/05/15 v2.6}
%   \item
%     luaTeX compatibility
%   \end{Version}
% \end{History}
%
% \PrintIndex
%
% \Finale
\endinput
