% \iffalse meta-comment
%
% File: rerunfilecheck.dtx
% Version: 2016/05/16 v1.8
% Info: Rerun checks for auxiliary files
%
% Copyright (C) 2009-2011 by
%    Heiko Oberdiek <heiko.oberdiek at googlemail.com>
%    2016
%    https://github.com/ho-tex/oberdiek/issues
%
% This work may be distributed and/or modified under the
% conditions of the LaTeX Project Public License, either
% version 1.3c of this license or (at your option) any later
% version. This version of this license is in
%    http://www.latex-project.org/lppl/lppl-1-3c.txt
% and the latest version of this license is in
%    http://www.latex-project.org/lppl.txt
% and version 1.3 or later is part of all distributions of
% LaTeX version 2005/12/01 or later.
%
% This work has the LPPL maintenance status "maintained".
%
% This Current Maintainer of this work is Heiko Oberdiek.
%
% This work consists of the main source file rerunfilecheck.dtx
% and the derived files
%    rerunfilecheck.sty, rerunfilecheck.pdf, rerunfilecheck.ins,
%    rerunfilecheck.drv, rerunfilecheck-example.cfg,
%    rerunfilecheck-test1.tex.
%
% Distribution:
%    CTAN:macros/latex/contrib/oberdiek/rerunfilecheck.dtx
%    CTAN:macros/latex/contrib/oberdiek/rerunfilecheck.pdf
%
% Unpacking:
%    (a) If rerunfilecheck.ins is present:
%           tex rerunfilecheck.ins
%    (b) Without rerunfilecheck.ins:
%           tex rerunfilecheck.dtx
%    (c) If you insist on using LaTeX
%           latex \let\install=y% \iffalse meta-comment
%
% File: rerunfilecheck.dtx
% Version: 2016/05/16 v1.8
% Info: Rerun checks for auxiliary files
%
% Copyright (C) 2009-2011 by
%    Heiko Oberdiek <heiko.oberdiek at googlemail.com>
%    2016
%    https://github.com/ho-tex/oberdiek/issues
%
% This work may be distributed and/or modified under the
% conditions of the LaTeX Project Public License, either
% version 1.3c of this license or (at your option) any later
% version. This version of this license is in
%    http://www.latex-project.org/lppl/lppl-1-3c.txt
% and the latest version of this license is in
%    http://www.latex-project.org/lppl.txt
% and version 1.3 or later is part of all distributions of
% LaTeX version 2005/12/01 or later.
%
% This work has the LPPL maintenance status "maintained".
%
% This Current Maintainer of this work is Heiko Oberdiek.
%
% This work consists of the main source file rerunfilecheck.dtx
% and the derived files
%    rerunfilecheck.sty, rerunfilecheck.pdf, rerunfilecheck.ins,
%    rerunfilecheck.drv, rerunfilecheck-example.cfg,
%    rerunfilecheck-test1.tex.
%
% Distribution:
%    CTAN:macros/latex/contrib/oberdiek/rerunfilecheck.dtx
%    CTAN:macros/latex/contrib/oberdiek/rerunfilecheck.pdf
%
% Unpacking:
%    (a) If rerunfilecheck.ins is present:
%           tex rerunfilecheck.ins
%    (b) Without rerunfilecheck.ins:
%           tex rerunfilecheck.dtx
%    (c) If you insist on using LaTeX
%           latex \let\install=y% \iffalse meta-comment
%
% File: rerunfilecheck.dtx
% Version: 2016/05/16 v1.8
% Info: Rerun checks for auxiliary files
%
% Copyright (C) 2009-2011 by
%    Heiko Oberdiek <heiko.oberdiek at googlemail.com>
%    2016
%    https://github.com/ho-tex/oberdiek/issues
%
% This work may be distributed and/or modified under the
% conditions of the LaTeX Project Public License, either
% version 1.3c of this license or (at your option) any later
% version. This version of this license is in
%    http://www.latex-project.org/lppl/lppl-1-3c.txt
% and the latest version of this license is in
%    http://www.latex-project.org/lppl.txt
% and version 1.3 or later is part of all distributions of
% LaTeX version 2005/12/01 or later.
%
% This work has the LPPL maintenance status "maintained".
%
% This Current Maintainer of this work is Heiko Oberdiek.
%
% This work consists of the main source file rerunfilecheck.dtx
% and the derived files
%    rerunfilecheck.sty, rerunfilecheck.pdf, rerunfilecheck.ins,
%    rerunfilecheck.drv, rerunfilecheck-example.cfg,
%    rerunfilecheck-test1.tex.
%
% Distribution:
%    CTAN:macros/latex/contrib/oberdiek/rerunfilecheck.dtx
%    CTAN:macros/latex/contrib/oberdiek/rerunfilecheck.pdf
%
% Unpacking:
%    (a) If rerunfilecheck.ins is present:
%           tex rerunfilecheck.ins
%    (b) Without rerunfilecheck.ins:
%           tex rerunfilecheck.dtx
%    (c) If you insist on using LaTeX
%           latex \let\install=y% \iffalse meta-comment
%
% File: rerunfilecheck.dtx
% Version: 2016/05/16 v1.8
% Info: Rerun checks for auxiliary files
%
% Copyright (C) 2009-2011 by
%    Heiko Oberdiek <heiko.oberdiek at googlemail.com>
%    2016
%    https://github.com/ho-tex/oberdiek/issues
%
% This work may be distributed and/or modified under the
% conditions of the LaTeX Project Public License, either
% version 1.3c of this license or (at your option) any later
% version. This version of this license is in
%    http://www.latex-project.org/lppl/lppl-1-3c.txt
% and the latest version of this license is in
%    http://www.latex-project.org/lppl.txt
% and version 1.3 or later is part of all distributions of
% LaTeX version 2005/12/01 or later.
%
% This work has the LPPL maintenance status "maintained".
%
% This Current Maintainer of this work is Heiko Oberdiek.
%
% This work consists of the main source file rerunfilecheck.dtx
% and the derived files
%    rerunfilecheck.sty, rerunfilecheck.pdf, rerunfilecheck.ins,
%    rerunfilecheck.drv, rerunfilecheck-example.cfg,
%    rerunfilecheck-test1.tex.
%
% Distribution:
%    CTAN:macros/latex/contrib/oberdiek/rerunfilecheck.dtx
%    CTAN:macros/latex/contrib/oberdiek/rerunfilecheck.pdf
%
% Unpacking:
%    (a) If rerunfilecheck.ins is present:
%           tex rerunfilecheck.ins
%    (b) Without rerunfilecheck.ins:
%           tex rerunfilecheck.dtx
%    (c) If you insist on using LaTeX
%           latex \let\install=y\input{rerunfilecheck.dtx}
%        (quote the arguments according to the demands of your shell)
%
% Documentation:
%    (a) If rerunfilecheck.drv is present:
%           latex rerunfilecheck.drv
%    (b) Without rerunfilecheck.drv:
%           latex rerunfilecheck.dtx; ...
%    The class ltxdoc loads the configuration file ltxdoc.cfg
%    if available. Here you can specify further options, e.g.
%    use A4 as paper format:
%       \PassOptionsToClass{a4paper}{article}
%
%    Programm calls to get the documentation (example):
%       pdflatex rerunfilecheck.dtx
%       makeindex -s gind.ist rerunfilecheck.idx
%       pdflatex rerunfilecheck.dtx
%       makeindex -s gind.ist rerunfilecheck.idx
%       pdflatex rerunfilecheck.dtx
%
% Installation:
%    TDS:tex/latex/oberdiek/rerunfilecheck.sty
%    TDS:doc/latex/oberdiek/rerunfilecheck.pdf
%    TDS:doc/latex/oberdiek/rerunfilecheck-example.cfg
%    TDS:doc/latex/oberdiek/test/rerunfilecheck-test1.tex
%    TDS:source/latex/oberdiek/rerunfilecheck.dtx
%
%<*ignore>
\begingroup
  \catcode123=1 %
  \catcode125=2 %
  \def\x{LaTeX2e}%
\expandafter\endgroup
\ifcase 0\ifx\install y1\fi\expandafter
         \ifx\csname processbatchFile\endcsname\relax\else1\fi
         \ifx\fmtname\x\else 1\fi\relax
\else\csname fi\endcsname
%</ignore>
%<*install>
\input docstrip.tex
\Msg{************************************************************************}
\Msg{* Installation}
\Msg{* Package: rerunfilecheck 2016/05/16 v1.8 Rerun checks for auxiliary files (HO)}
\Msg{************************************************************************}

\keepsilent
\askforoverwritefalse

\let\MetaPrefix\relax
\preamble

This is a generated file.

Project: rerunfilecheck
Version: 2016/05/16 v1.8

Copyright (C) 2009-2011 by
   Heiko Oberdiek <heiko.oberdiek at googlemail.com>

This work may be distributed and/or modified under the
conditions of the LaTeX Project Public License, either
version 1.3c of this license or (at your option) any later
version. This version of this license is in
   http://www.latex-project.org/lppl/lppl-1-3c.txt
and the latest version of this license is in
   http://www.latex-project.org/lppl.txt
and version 1.3 or later is part of all distributions of
LaTeX version 2005/12/01 or later.

This work has the LPPL maintenance status "maintained".

This Current Maintainer of this work is Heiko Oberdiek.

This work consists of the main source file rerunfilecheck.dtx
and the derived files
   rerunfilecheck.sty, rerunfilecheck.pdf, rerunfilecheck.ins,
   rerunfilecheck.drv, rerunfilecheck-example.cfg,
   rerunfilecheck-test1.tex.

\endpreamble
\let\MetaPrefix\DoubleperCent

\generate{%
  \file{rerunfilecheck.ins}{\from{rerunfilecheck.dtx}{install}}%
  \file{rerunfilecheck.drv}{\from{rerunfilecheck.dtx}{driver}}%
  \usedir{tex/latex/oberdiek}%
  \file{rerunfilecheck.sty}{\from{rerunfilecheck.dtx}{package}}%
%  \usedir{doc/latex/oberdiek/test}%
%  \file{rerunfilecheck-test1.tex}{\from{rerunfilecheck.dtx}{test1}}%
  \nopreamble
  \nopostamble
  \usedir{doc/latex/oberdiek}%
  \file{rerunfilecheck-example.cfg}{\from{rerunfilecheck.dtx}{cfg}}%
  \usedir{source/latex/oberdiek/catalogue}%
  \file{rerunfilecheck.xml}{\from{rerunfilecheck.dtx}{catalogue}}%
}

\catcode32=13\relax% active space
\let =\space%
\Msg{************************************************************************}
\Msg{*}
\Msg{* To finish the installation you have to move the following}
\Msg{* file into a directory searched by TeX:}
\Msg{*}
\Msg{*     rerunfilecheck.sty}
\Msg{*}
\Msg{* To produce the documentation run the file `rerunfilecheck.drv'}
\Msg{* through LaTeX.}
\Msg{*}
\Msg{* Happy TeXing!}
\Msg{*}
\Msg{************************************************************************}

\endbatchfile
%</install>
%<*ignore>
\fi
%</ignore>
%<*driver>
\NeedsTeXFormat{LaTeX2e}
\ProvidesFile{rerunfilecheck.drv}%
  [2016/05/16 v1.8 Rerun checks for auxiliary files (HO)]%
\documentclass{ltxdoc}
\usepackage{holtxdoc}[2011/11/22]
\begin{document}
  \DocInput{rerunfilecheck.dtx}%
\end{document}
%</driver>
% \fi
%
%
% \CharacterTable
%  {Upper-case    \A\B\C\D\E\F\G\H\I\J\K\L\M\N\O\P\Q\R\S\T\U\V\W\X\Y\Z
%   Lower-case    \a\b\c\d\e\f\g\h\i\j\k\l\m\n\o\p\q\r\s\t\u\v\w\x\y\z
%   Digits        \0\1\2\3\4\5\6\7\8\9
%   Exclamation   \!     Double quote  \"     Hash (number) \#
%   Dollar        \$     Percent       \%     Ampersand     \&
%   Acute accent  \'     Left paren    \(     Right paren   \)
%   Asterisk      \*     Plus          \+     Comma         \,
%   Minus         \-     Point         \.     Solidus       \/
%   Colon         \:     Semicolon     \;     Less than     \<
%   Equals        \=     Greater than  \>     Question mark \?
%   Commercial at \@     Left bracket  \[     Backslash     \\
%   Right bracket \]     Circumflex    \^     Underscore    \_
%   Grave accent  \`     Left brace    \{     Vertical bar  \|
%   Right brace   \}     Tilde         \~}
%
% \GetFileInfo{rerunfilecheck.drv}
%
% \title{The \xpackage{rerunfilecheck} package}
% \date{2016/05/16 v1.8}
% \author{Heiko Oberdiek\thanks
% {Please report any issues at https://github.com/ho-tex/oberdiek/issues}\\
% \xemail{heiko.oberdiek at googlemail.com}}
%
% \maketitle
%
% \begin{abstract}
% The package provides additional rerun warnings if some
% auxiliary files have changed. It is based on MD5 checksum,
% provided by \pdfTeX.
% \end{abstract}
%
% \tableofcontents
%
% \section{Documentation}
%
% \LaTeX\ informs the user, when to run \LaTeX\ again, if the
% references have changed. It has the old references from the
% first reading of the \xext{aux} files already in memory, thus
% it compares them with the new version of the \xext{aux} file
% at the end of the document. However this rerun warnings are
% not given for the table of contents and other data stored
% in the \xext{aux} files or other auxiliary files.
% Usually many of these data as the table of contents is not keep
% in memory. If someone wants to detect changes, he has either
% to keep the data in memory. This does not scale well with huge
% documents. Or he copies the file before they are changed. Slow
% I/O operations cost time.
%
% Since version 1.30.0 \pdfTeX\ provides \cs{pdfmdfivesum}
% and \cs{pdffilesize}.
% These features are also available in \LuaTeX, provided by package
% \xpackage{pdftexcmds}.
% Thus this package \xpackage{rerunfilecheck} uses these features to
% detect file changes. This saves the packages from keeping
% the whole files in memory or in file copies.
% The drawback are different files with the same size and the
% same MD5 checksum (seldom, hopefully).
%
% \subsection{Options}
%
% All options are key value options of boolean type.
% No option or |true| turns an option on, |false| disables
% an option.
% \begin{description}
% \item[\xoption{mainaux}:]
%   Check the main \xext{aux} file.
% \item[\xoption{partaux}:]
%   Check the \xext{aux} files from \cs{include} files.
% \item[\xoption{starttoc}:]
%   Add the rerun checks in \cs{@starttoc} that is called by
%   \cs{tableofcontents}, \cs{listoffigures}, \dots
% \item[\xoption{index}, \xoption{glossary}:]
%   \LaTeX's original \cs{makeindex} and \cs{makeglossary} are
%   redefined to add the rerun checks. The options do not have
%   an effect, if \cs{makeindex}/\cs{makeglossary} are already
%   called or if a package or class had redefined or will
%   redefine them.
% \item[\xoption{aux}:]
%   This option turns all previous options on or off.
%   ``aux'' means auxiliary file.
% \end{description}
% The default for the options is |false|, because
% some internals must be redefined to insert the
% rerun checks.
% The options can be set in \cs{usepackage} or
% the configuration file \xfile{rerunfilecheck.cfg}.
% Global options are ignored (since 1.4).
%
% \begin{declcs}{RerunFileCheckSetup} \M{key value list}
% \end{declcs}
% Options can also be set using \cs{RerunFileCheckSetup}.
% Currently all options are disabled after the package
% is loaded. Thus \cs{RerunFileCheckSetup} makes sense in
% the configuration file only.
%
% Example for the configuration file:
%    \begin{macrocode}
%<*cfg>
\ProvidesFile{rerunfilecheck.cfg}[2016/05/16 Default configuration]%
\RerunFileCheckSetup{aux}
%</cfg>
%    \end{macrocode}
%
% \subsection{Interface for class/package authors}
%
% \begin{declcs}{RerunFileCheck} \M{file} \M{file closing action} \M{rerun warning}
% \end{declcs}
% If you want to add a rerun check, call \cs{RerunFileCheck} right before
% an output file is opened for writing. The macro first remembers
% the current checksum of \meta{file}.
% The file is checked again right before the end of the job.
% Macro \cs{AtVeryEndDocument} of package \xpackage{atveryend}
% is used to place the check after the main aux file is closed
% in |\end{document}|. Before reading the file again, it must be
% closed. Provide the code for closing in argument \meta{file closing action}.
% Do not forget \cs{immediate} before \cs{openout}. Otherwise the
% closing action would be delayed to the next shipout that never happens
% (the last page is already shipped out).
% If the file has changed, \cs{RerunFileCheck} informs the user with
% a warning that the file has changed and says the magic word ``Rerun''.
% If the last argument \meta{rerun warning} is not empty, then
% the rerun sentence is replaced by it. Usually the phrase
% ``to get something right'' is added.
% As example the relevant part of the redefined \cs{makeindex}
% is shown, see package code:
%\begin{quote}
%\begin{verbatim}
%\newwrite\@indexfile
%\RerunFileCheck{\jobname.idx}{%
%  \immediate\closeout\@indexfile
%}{%
%  Rerun LaTeX/makeindex to get index right%
%}%
%\immediate\openout\@indexfile=\jobname.idx %
%\end{verbatim}
%\end{quote}
%
% \StopEventually{
% }
%
% \section{Implementation}
%    \begin{macrocode}
%<*package>
%    \end{macrocode}
%
%    \begin{macrocode}
\begingroup\catcode61\catcode48\catcode32=10\relax%
  \catcode13=5 % ^^M
  \endlinechar=13 %
  \catcode123=1 % {
  \catcode125=2 % }
  \catcode64=11 % @
  \def\x{\endgroup
    \expandafter\edef\csname ReFiCh@AtEnd\endcsname{%
      \endlinechar=\the\endlinechar\relax
      \catcode13=\the\catcode13\relax
      \catcode32=\the\catcode32\relax
      \catcode35=\the\catcode35\relax
      \catcode61=\the\catcode61\relax
      \catcode64=\the\catcode64\relax
      \catcode123=\the\catcode123\relax
      \catcode125=\the\catcode125\relax
    }%
  }%
\x\catcode61\catcode48\catcode32=10\relax%
\catcode13=5 % ^^M
\endlinechar=13 %
\catcode35=6 % #
\catcode64=11 % @
\catcode123=1 % {
\catcode125=2 % }
\def\TMP@EnsureCode#1#2{%
  \edef\ReFiCh@AtEnd{%
    \ReFiCh@AtEnd
    \catcode#1=\the\catcode#1\relax
  }%
  \catcode#1=#2\relax
}
\TMP@EnsureCode{39}{12}% '
\TMP@EnsureCode{40}{12}% (
\TMP@EnsureCode{41}{12}% )
\TMP@EnsureCode{42}{12}% *
\TMP@EnsureCode{44}{12}% ,
\TMP@EnsureCode{46}{12}% .
\TMP@EnsureCode{47}{12}% /
\TMP@EnsureCode{58}{12}% :
\TMP@EnsureCode{59}{12}% ;
\TMP@EnsureCode{60}{12}% <
\TMP@EnsureCode{62}{12}% >
\TMP@EnsureCode{91}{12}% [
\TMP@EnsureCode{93}{12}% ]
\TMP@EnsureCode{96}{12}% `
\edef\ReFiCh@AtEnd{\ReFiCh@AtEnd\noexpand\endinput}
%    \end{macrocode}
%
%    Package identification.
%    \begin{macrocode}
\NeedsTeXFormat{LaTeX2e}
\ProvidesPackage{rerunfilecheck}%
  [2016/05/16 v1.8 Rerun checks for auxiliary files (HO)]
%    \end{macrocode}
%
% \subsection{Options}
%
%    \begin{macrocode}
\RequirePackage{kvoptions}[2010/02/22]
\SetupKeyvalOptions{%
  family=rerunfilecheck,%
  prefix=ReFiCh@%
}
%    \end{macrocode}
%    \begin{macro}{\RerunFileCheckSetup}
%    \begin{macrocode}
\newcommand*{\RerunFileCheckSetup}{%
  \setkeys{rerunfilecheck}%
}
%    \end{macrocode}
%    \end{macro}
%    \begin{macrocode}
\DeclareBoolOption{mainaux}
\DeclareBoolOption{partaux}
\DeclareBoolOption{starttoc}
\DeclareBoolOption{index}
\DeclareBoolOption{glossary}
\define@key{rerunfilecheck}{aux}[true]{%
  \RerunFileCheckSetup{%
    mainaux={#1},%
    partaux={#1},%
    starttoc={#1},%
    index={#1},%
    glossary={#1}%
  }%
}
%    \end{macrocode}
%    \begin{macrocode}
\InputIfFileExists{rerunfilecheck.cfg}{}{}
\ProcessLocalKeyvalOptions*
%    \end{macrocode}
%
%    \begin{macro}{\ReFiCh@DisableOption}
%    \begin{macrocode}
\def\ReFiCh@DisableOption{%
  \DisableKeyvalOption[%
    action=warning,%
    package=rerunfilecheck%
  ]{rerunfilecheck}%
}
%    \end{macrocode}
%    \end{macro}
%
% \subsection{Check for checksum feature}
%
%    \begin{macrocode}
\RequirePackage{infwarerr}[2007/09/09]
\RequirePackage{pdftexcmds}[2009/04/10]
%    \end{macrocode}
%    \begin{macrocode}
\begingroup\expandafter\expandafter\expandafter\endgroup
\expandafter\ifx\csname pdf@filemdfivesum\endcsname\relax
  \@PackageInfoNoLine{rerunfilecheck}{%
    Feature \string\pdfmdfivesum\space is not available\MessageBreak
    (e.g. pdfTeX or LuaTeX with package `pdftexcmds').\MessageBreak
    Therefore file contents cannot be checked efficiently\MessageBreak
    and the loading of the package is aborted%
  }%
  \newcommand*{\RerunFileCheck}[3]{}%
  \renewcommand*{\RerunFileCheckSetup}[1]{}%
  \expandafter\ReFiCh@AtEnd
\fi%
%    \end{macrocode}
%
% \subsection{Standard \xext{aux} files}
%
%    \begin{macrocode}
\ifReFiCh@partaux
  \let\ReFiCh@org@include\@include
  \def\@include#1 {%
    \if@filesw
      \RerunFileCheck{#1.aux}{}{}%
    \fi
    \ReFiCh@org@include{#1} %
  }%
\fi
\ifReFiCh@mainaux
  \AtBeginDocument{%
    \ReFiCh@mainauxfalse
  }%
  \ifReFiCh@mainaux
    \AtEndOfPackage{%
      \RerunFileCheck{\jobname.aux}{}{}%
    }%
  \else
    \if@filesw
      \@PackageWarningNoLine{rerunfilecheck}{%
        Main aux file check is disabled,\MessageBreak
        because the file is already opened.\MessageBreak
        Load the package before \string\begin{document}%
      }%
    \fi
  \fi
\fi
\ifReFiCh@starttoc
  \let\ReFiCh@org@starttoc\@starttoc
  \def\@starttoc#1{%
    \if@filesw
      \RerunFileCheck{\jobname.#1}{%
        \@ifundefined{tf@#1}{%
        }{%
          \immediate\closeout\csname tf@#1\endcsname
        }%
      }{}%
    \fi
    \ReFiCh@org@starttoc{#1}%
  }%
\fi
\ifReFiCh@index
  \ifx\makeindex\@empty
    \@PackageWarningNoLine{rerunfilecheck}{%
      Option `index' ignored,\MessageBreak
      because \string\makeindex\space has already been called%
    }%
  \else
    \def\ReFiCh@temp{%
      \newwrite\@indexfile
      \immediate\openout\@indexfile=\jobname.idx %
      \def\index{%
        \@bsphack
        \begingroup
        \@sanitize
        \@wrindex
      }%
      \typeout{Writing index file \jobname.idx}%
      \let\makeindex\@empty
    }%
    \ifx\ReFiCh@temp\makeindex
      \def\makeindex{%
        \newwrite\@indexfile
        \RerunFileCheck{\jobname.idx}{%
          \immediate\closeout\@indexfile
        }{%
          Rerun LaTeX/makeindex to get index right%
        }%
        \immediate\openout\@indexfile=\jobname.idx %
        \def\index{%
          \@bsphack
          \begingroup
          \@sanitize
          \@wrindex
        }%
        \typeout{Writing index file \jobname.idx}%
        \let\makeindex\@empty
      }%
    \else
      \@PackageInfoNoLine{rerunfilecheck}{%
        Option `index': unsupported version of \string\makeindex
      }%
    \fi
  \fi
\fi
\ifReFiCh@glossary
  \ifx\makeglossary\@empty
    \@PackageWarningNoLine{rerunfilecheck}{%
      Option `glossary' ignored,\MessageBreak
      because \string\makeglossary\space has already been called%
    }%
  \else
    \def\ReFiCh@temp{%
      \newwrite\@glossaryfile
      \immediate\openout\@glossaryfile=\jobname.glo %
      \def\glossary{%
        \@bsphack
        \begingroup
        \@sanitize
        \@wrglossary
      }%
      \typeout{Writing glossary file \jobname.glo }%
      \let\makeglossary\@empty
    }%
    \ifx\ReFiCh@temp\makeglossary
      \def\ReFiCh@temp{%
        \newwrite\@glossaryfile
        \RerunFileCheck{\jobname.glo}{%
          \immediate\closeout\@glossaryfile
        }{%
          Rerun LaTeX/makeindex to get glossary right%
        }%
        \immediate\openout\@glossaryfile=\jobname.glo %
        \def\glossary{%
          \@bsphack
          \begingroup
          \@sanitize
          \@wrglossary
        }%
        \typeout{Writing glossary file \jobname.glo}%
        \let\makeglossary\@empty
      }%
    \else
      \@PackageInfoNoLine{rerunfilecheck}{%
        Option `glossary': unsupported version of \string\makeglossary
      }%
    \fi
  \fi
\fi
\ReFiCh@DisableOption{mainaux}
\ReFiCh@DisableOption{partaux}
\ReFiCh@DisableOption{starttoc}
\ReFiCh@DisableOption{index}
\ReFiCh@DisableOption{glossary}
\ReFiCh@DisableOption{aux}
%    \end{macrocode}
%
% \subsection{Rerun check}
%
%    \begin{macrocode}
\RequirePackage{atveryend}[2016/05/16]
\RequirePackage{uniquecounter}[2009/12/18]
%    \end{macrocode}
%
%    \begin{macro}{\ReFiCh@CheckSum}
%    \begin{macrocode}
\begingroup\expandafter\expandafter\expandafter\endgroup
\expandafter\ifx\csname pdf@filesize\endcsname\relax
  \def\ReFiCh@CheckSum{%
    \pdf@filemdfivesum
  }%
\else
  \def\ReFiCh@CheckSum#1{%
    \pdf@filemdfivesum{#1}%
    \ReFiCh@Separator
    \pdf@filesize{#1}%
  }%
\fi
%    \end{macrocode}
%    \end{macro}
%    \begin{macro}{\ReFiCh@NoFile}
%    \begin{macrocode}
\def\ReFiCh@Separator{;}
%    \end{macrocode}
%    \end{macro}
%    \begin{macro}{\ReFiCh@NoFile}
%    \begin{macrocode}
\def\ReFiCh@NoFile{<no file>}
%    \end{macrocode}
%    \end{macro}
%
%    \begin{macrocode}
\UniqueCounterNew{rerunfilecheck}
%    \end{macrocode}
%
%    \begin{macro}{\RerunFileCheck}
%    \begin{macrocode}
\newcommand*{\RerunFileCheck}{%
  \UniqueCounterCall{rerunfilecheck}\ReFiCh@RerunFileCheck
}
%    \end{macrocode}
%    \end{macro}
%    \begin{macro}{\ReFiCh@RerunFileCheck}
%    \begin{macrocode}
\def\ReFiCh@RerunFileCheck#1{%
  \expandafter\ReFiCh@@RerunFileCheck\csname ReFiCh@#1\endcsname
}
%    \end{macrocode}
%    \end{macro}
%    \begin{macro}{\ReFiCh@Check}
%    \begin{macrocode}
\def\ReFiCh@Check#1#2#3{%
%  \IfFileExists{#3}{%
    #1\edef#2{\ReFiCh@CheckSum{#3}}%
    \ifx#2\ReFiCh@Separator
      #1\let#2\ReFiCh@NoFile
    \fi
%  }{%
%    #1\let#2\ReFiCh@NoFile
%  }%
}
%    \end{macrocode}
%    \end{macro}
%    \begin{macro}{\ReFiCh@@RerunFileCheck}
%    \begin{macrocode}
\def\ReFiCh@@RerunFileCheck#1#2#3#4{%
  \ReFiCh@Check\global#1{#2}%
  \AtEndAfterFileList{%
    \begingroup
      #3%
      \ReFiCh@Check{}\x{#2}%
      \ifx#1\x
        \@PackageInfoNoLine{rerunfilecheck}{%
          File `#2' has not changed.\MessageBreak
          Checksum: \x
        }%
      \else
        \ifnum
          \ReFiCh@IsAux#2\relax.aux\relax\@nil
          \ifx#1\ReFiCh@NoFile 1\else 0\fi
          \ifx\x\ReFiCh@AuxEmptyUnix 1%
          \else
            \ifx\x\ReFiCh@AuxEmptyDos 1\fi
          \fi
        =111 %
          \@PackageInfoNoLine{rerunfilecheck}{%
            File `#2' is empty .aux file.\MessageBreak
            Before: #1\MessageBreak
            After: \space\x
          }%
        \else
          \@PackageWarningNoLine{rerunfilecheck}{%
            File `#2' has changed.%
            \ifx\\#4\\%
              \space Rerun%
            \else
              \MessageBreak
              #4%
            \fi
          }%
          \@PackageInfoNoLine{rerunfilecheck}{%
            Checksums for `#2':\MessageBreak
            Before: #1\MessageBreak
            After: \space\x
          }%
        \fi
      \fi
    \endgroup
  }%
}
%    \end{macrocode}
%    \end{macro}
%    \begin{macrocode}
\def\ReFiCh@IsAux#1.aux\relax#2\@nil{%
  \ifx\hbox#2\hbox
    0%
  \else
    1%
  \fi
}
%    \end{macrocode}
%    \begin{macrocode}
\def\ReFiCh@AuxEmptyUnix{A94A2480D3289E625EEA47CD1B285758;8}%
\@onelevel@sanitize\ReFiCh@AuxEmptyUnix
%    \end{macrocode}
%    \begin{macrocode}
\def\ReFiCh@AuxEmptyDos{A62A15ECE803E2EBE94952FCC9933BC0;9}%
\@onelevel@sanitize\ReFiCh@AuxEmptyDos
%    \end{macrocode}
%
%    \begin{macrocode}
\ReFiCh@AtEnd%
%</package>
%    \end{macrocode}
%
% \section{Test}
%
%    \begin{macrocode}
%<*test1>
\def\LoadCommand{\RequirePackage{rerunfilecheck}[2016/05/16]}
%</test1>
%    \end{macrocode}
% \subsection{Catcode checks for loading}
%
%    \begin{macrocode}
%<*test1>
%    \end{macrocode}
%    \begin{macrocode}
\catcode`\{=1 %
\catcode`\}=2 %
\catcode`\#=6 %
\catcode`\@=11 %
\expandafter\ifx\csname count@\endcsname\relax
  \countdef\count@=255 %
\fi
\expandafter\ifx\csname @gobble\endcsname\relax
  \long\def\@gobble#1{}%
\fi
\expandafter\ifx\csname @firstofone\endcsname\relax
  \long\def\@firstofone#1{#1}%
\fi
\expandafter\ifx\csname loop\endcsname\relax
  \expandafter\@firstofone
\else
  \expandafter\@gobble
\fi
{%
  \def\loop#1\repeat{%
    \def\body{#1}%
    \iterate
  }%
  \def\iterate{%
    \body
      \let\next\iterate
    \else
      \let\next\relax
    \fi
    \next
  }%
  \let\repeat=\fi
}%
\def\RestoreCatcodes{}
\count@=0 %
\loop
  \edef\RestoreCatcodes{%
    \RestoreCatcodes
    \catcode\the\count@=\the\catcode\count@\relax
  }%
\ifnum\count@<255 %
  \advance\count@ 1 %
\repeat

\def\RangeCatcodeInvalid#1#2{%
  \count@=#1\relax
  \loop
    \catcode\count@=15 %
  \ifnum\count@<#2\relax
    \advance\count@ 1 %
  \repeat
}
\def\RangeCatcodeCheck#1#2#3{%
  \count@=#1\relax
  \loop
    \ifnum#3=\catcode\count@
    \else
      \errmessage{%
        Character \the\count@\space
        with wrong catcode \the\catcode\count@\space
        instead of \number#3%
      }%
    \fi
  \ifnum\count@<#2\relax
    \advance\count@ 1 %
  \repeat
}
\def\space{ }
\expandafter\ifx\csname LoadCommand\endcsname\relax
  \def\LoadCommand{\input rerunfilecheck.sty\relax}%
\fi
\def\Test{%
  \RangeCatcodeInvalid{0}{47}%
  \RangeCatcodeInvalid{58}{64}%
  \RangeCatcodeInvalid{91}{96}%
  \RangeCatcodeInvalid{123}{255}%
  \catcode`\@=12 %
  \catcode`\\=0 %
  \catcode`\%=14 %
  \LoadCommand
  \RangeCatcodeCheck{0}{36}{15}%
  \RangeCatcodeCheck{37}{37}{14}%
  \RangeCatcodeCheck{38}{47}{15}%
  \RangeCatcodeCheck{48}{57}{12}%
  \RangeCatcodeCheck{58}{63}{15}%
  \RangeCatcodeCheck{64}{64}{12}%
  \RangeCatcodeCheck{65}{90}{11}%
  \RangeCatcodeCheck{91}{91}{15}%
  \RangeCatcodeCheck{92}{92}{0}%
  \RangeCatcodeCheck{93}{96}{15}%
  \RangeCatcodeCheck{97}{122}{11}%
  \RangeCatcodeCheck{123}{255}{15}%
  \RestoreCatcodes
}
\Test
\csname @@end\endcsname
\end
%    \end{macrocode}
%    \begin{macrocode}
%</test1>
%    \end{macrocode}
%
% \section{Installation}
%
% \subsection{Download}
%
% \paragraph{Package.} This package is available on
% CTAN\footnote{\url{http://ctan.org/pkg/rerunfilecheck}}:
% \begin{description}
% \item[\CTAN{macros/latex/contrib/oberdiek/rerunfilecheck.dtx}] The source file.
% \item[\CTAN{macros/latex/contrib/oberdiek/rerunfilecheck.pdf}] Documentation.
% \end{description}
%
%
% \paragraph{Bundle.} All the packages of the bundle `oberdiek'
% are also available in a TDS compliant ZIP archive. There
% the packages are already unpacked and the documentation files
% are generated. The files and directories obey the TDS standard.
% \begin{description}
% \item[\CTAN{install/macros/latex/contrib/oberdiek.tds.zip}]
% \end{description}
% \emph{TDS} refers to the standard ``A Directory Structure
% for \TeX\ Files'' (\CTAN{tds/tds.pdf}). Directories
% with \xfile{texmf} in their name are usually organized this way.
%
% \subsection{Bundle installation}
%
% \paragraph{Unpacking.} Unpack the \xfile{oberdiek.tds.zip} in the
% TDS tree (also known as \xfile{texmf} tree) of your choice.
% Example (linux):
% \begin{quote}
%   |unzip oberdiek.tds.zip -d ~/texmf|
% \end{quote}
%
% \paragraph{Script installation.}
% Check the directory \xfile{TDS:scripts/oberdiek/} for
% scripts that need further installation steps.
% Package \xpackage{attachfile2} comes with the Perl script
% \xfile{pdfatfi.pl} that should be installed in such a way
% that it can be called as \texttt{pdfatfi}.
% Example (linux):
% \begin{quote}
%   |chmod +x scripts/oberdiek/pdfatfi.pl|\\
%   |cp scripts/oberdiek/pdfatfi.pl /usr/local/bin/|
% \end{quote}
%
% \subsection{Package installation}
%
% \paragraph{Unpacking.} The \xfile{.dtx} file is a self-extracting
% \docstrip\ archive. The files are extracted by running the
% \xfile{.dtx} through \plainTeX:
% \begin{quote}
%   \verb|tex rerunfilecheck.dtx|
% \end{quote}
%
% \paragraph{TDS.} Now the different files must be moved into
% the different directories in your installation TDS tree
% (also known as \xfile{texmf} tree):
% \begin{quote}
% \def\t{^^A
% \begin{tabular}{@{}>{\ttfamily}l@{ $\rightarrow$ }>{\ttfamily}l@{}}
%   rerunfilecheck.sty & tex/latex/oberdiek/rerunfilecheck.sty\\
%   rerunfilecheck.pdf & doc/latex/oberdiek/rerunfilecheck.pdf\\
%   rerunfilecheck-example.cfg & doc/latex/oberdiek/rerunfilecheck-example.cfg\\
%   test/rerunfilecheck-test1.tex & doc/latex/oberdiek/test/rerunfilecheck-test1.tex\\
%   rerunfilecheck.dtx & source/latex/oberdiek/rerunfilecheck.dtx\\
% \end{tabular}^^A
% }^^A
% \sbox0{\t}^^A
% \ifdim\wd0>\linewidth
%   \begingroup
%     \advance\linewidth by\leftmargin
%     \advance\linewidth by\rightmargin
%   \edef\x{\endgroup
%     \def\noexpand\lw{\the\linewidth}^^A
%   }\x
%   \def\lwbox{^^A
%     \leavevmode
%     \hbox to \linewidth{^^A
%       \kern-\leftmargin\relax
%       \hss
%       \usebox0
%       \hss
%       \kern-\rightmargin\relax
%     }^^A
%   }^^A
%   \ifdim\wd0>\lw
%     \sbox0{\small\t}^^A
%     \ifdim\wd0>\linewidth
%       \ifdim\wd0>\lw
%         \sbox0{\footnotesize\t}^^A
%         \ifdim\wd0>\linewidth
%           \ifdim\wd0>\lw
%             \sbox0{\scriptsize\t}^^A
%             \ifdim\wd0>\linewidth
%               \ifdim\wd0>\lw
%                 \sbox0{\tiny\t}^^A
%                 \ifdim\wd0>\linewidth
%                   \lwbox
%                 \else
%                   \usebox0
%                 \fi
%               \else
%                 \lwbox
%               \fi
%             \else
%               \usebox0
%             \fi
%           \else
%             \lwbox
%           \fi
%         \else
%           \usebox0
%         \fi
%       \else
%         \lwbox
%       \fi
%     \else
%       \usebox0
%     \fi
%   \else
%     \lwbox
%   \fi
% \else
%   \usebox0
% \fi
% \end{quote}
% If you have a \xfile{docstrip.cfg} that configures and enables \docstrip's
% TDS installing feature, then some files can already be in the right
% place, see the documentation of \docstrip.
%
% \subsection{Refresh file name databases}
%
% If your \TeX~distribution
% (\teTeX, \mikTeX, \dots) relies on file name databases, you must refresh
% these. For example, \teTeX\ users run \verb|texhash| or
% \verb|mktexlsr|.
%
% \subsection{Some details for the interested}
%
% \paragraph{Attached source.}
%
% The PDF documentation on CTAN also includes the
% \xfile{.dtx} source file. It can be extracted by
% AcrobatReader 6 or higher. Another option is \textsf{pdftk},
% e.g. unpack the file into the current directory:
% \begin{quote}
%   \verb|pdftk rerunfilecheck.pdf unpack_files output .|
% \end{quote}
%
% \paragraph{Unpacking with \LaTeX.}
% The \xfile{.dtx} chooses its action depending on the format:
% \begin{description}
% \item[\plainTeX:] Run \docstrip\ and extract the files.
% \item[\LaTeX:] Generate the documentation.
% \end{description}
% If you insist on using \LaTeX\ for \docstrip\ (really,
% \docstrip\ does not need \LaTeX), then inform the autodetect routine
% about your intention:
% \begin{quote}
%   \verb|latex \let\install=y\input{rerunfilecheck.dtx}|
% \end{quote}
% Do not forget to quote the argument according to the demands
% of your shell.
%
% \paragraph{Generating the documentation.}
% You can use both the \xfile{.dtx} or the \xfile{.drv} to generate
% the documentation. The process can be configured by the
% configuration file \xfile{ltxdoc.cfg}. For instance, put this
% line into this file, if you want to have A4 as paper format:
% \begin{quote}
%   \verb|\PassOptionsToClass{a4paper}{article}|
% \end{quote}
% An example follows how to generate the
% documentation with pdf\LaTeX:
% \begin{quote}
%\begin{verbatim}
%pdflatex rerunfilecheck.dtx
%makeindex -s gind.ist rerunfilecheck.idx
%pdflatex rerunfilecheck.dtx
%makeindex -s gind.ist rerunfilecheck.idx
%pdflatex rerunfilecheck.dtx
%\end{verbatim}
% \end{quote}
%
% \section{Catalogue}
%
% The following XML file can be used as source for the
% \href{http://mirror.ctan.org/help/Catalogue/catalogue.html}{\TeX\ Catalogue}.
% The elements \texttt{caption} and \texttt{description} are imported
% from the original XML file from the Catalogue.
% The name of the XML file in the Catalogue is \xfile{rerunfilecheck.xml}.
%    \begin{macrocode}
%<*catalogue>
<?xml version='1.0' encoding='us-ascii'?>
<!DOCTYPE entry SYSTEM 'catalogue.dtd'>
<entry datestamp='$Date$' modifier='$Author$' id='rerunfilecheck'>
  <name>rerunfilecheck</name>
  <caption>Checksum based rerun checks on auxiliary files.</caption>
  <authorref id='auth:oberdiek'/>
  <copyright owner='Heiko Oberdiek' year='2009-2011'/>
  <license type='lppl1.3'/>
  <version number='1.8'/>
  <description>
    The package provides additional rerun warnings if some
    auxiliary files have changed. It is based on MD5 checksum,
    provided by pdfTeX.
    <p/>
    The package is part of the <xref refid='oberdiek'>oberdiek</xref> bundle.
  </description>
  <documentation details='Package documentation'
      href='ctan:/macros/latex/contrib/oberdiek/rerunfilecheck.pdf'/>
  <ctan file='true' path='/macros/latex/contrib/oberdiek/rerunfilecheck.dtx'/>
  <miktex location='oberdiek'/>
  <texlive location='oberdiek'/>
  <install path='/macros/latex/contrib/oberdiek/oberdiek.tds.zip'/>
</entry>
%</catalogue>
%    \end{macrocode}
%
% \begin{History}
%   \begin{Version}{2009/12/10 v1.0}
%   \item
%     The first version.
%   \end{Version}
%   \begin{Version}{2009/12/12 v1.1}
%   \item
%     Short info shortened.
%   \end{Version}
%   \begin{Version}{2009/12/18 v1.2}
%   \item
%     Required date for package \xpackage{uniquecounter} updated
%     because of bug in this package.
%   \end{Version}
%   \begin{Version}{2010/01/25 v1.3}
%   \item
%     Moved from \texttt{TDS:*/generic/*} to \texttt{TDS:*/latex/*}.
%   \end{Version}
%   \begin{Version}{2010/02/22 v1.4}
%   \item
%     The options of this package are recognized only if they are
%     package options. Global options are ignored. This avoids
%     name clashes with class and other package options
%     (for example, class option `index=totoc').
%   \end{Version}
%   \begin{Version}{2010/03/15 v1.5}
%   \item
%     Call of \cs{pdffivesum} is wrapped in \cs{IfFileExists} to avoid
%     calls of \textsf{mktextex} if this feature is enabled.
%     However \cs{IfFileExists} has file name limitations.
%   \end{Version}
%   \begin{Version}{2010/03/16 v1.6}
%   \item
%     Reverted to version 1.4 and \cs{IfFileExists} wrapper of
%     version 1.5 is removed.
%   \end{Version}
%   \begin{Version}{2011/04/15 v1.7}
%   \item
%     Using \cs{AtEndAfterFileList} of package \xpackage{atveryend}
%     2011/04/15 v1.6 instead of \cs{AtVeryEndDocument}.
%   \end{Version}
%   \begin{Version}{2016/05/16 v1.8}
%   \item
%     Documentation updates.
%   \end{Version}
% \end{History}
%
% \PrintIndex
%
% \Finale
\endinput

%        (quote the arguments according to the demands of your shell)
%
% Documentation:
%    (a) If rerunfilecheck.drv is present:
%           latex rerunfilecheck.drv
%    (b) Without rerunfilecheck.drv:
%           latex rerunfilecheck.dtx; ...
%    The class ltxdoc loads the configuration file ltxdoc.cfg
%    if available. Here you can specify further options, e.g.
%    use A4 as paper format:
%       \PassOptionsToClass{a4paper}{article}
%
%    Programm calls to get the documentation (example):
%       pdflatex rerunfilecheck.dtx
%       makeindex -s gind.ist rerunfilecheck.idx
%       pdflatex rerunfilecheck.dtx
%       makeindex -s gind.ist rerunfilecheck.idx
%       pdflatex rerunfilecheck.dtx
%
% Installation:
%    TDS:tex/latex/oberdiek/rerunfilecheck.sty
%    TDS:doc/latex/oberdiek/rerunfilecheck.pdf
%    TDS:doc/latex/oberdiek/rerunfilecheck-example.cfg
%    TDS:doc/latex/oberdiek/test/rerunfilecheck-test1.tex
%    TDS:source/latex/oberdiek/rerunfilecheck.dtx
%
%<*ignore>
\begingroup
  \catcode123=1 %
  \catcode125=2 %
  \def\x{LaTeX2e}%
\expandafter\endgroup
\ifcase 0\ifx\install y1\fi\expandafter
         \ifx\csname processbatchFile\endcsname\relax\else1\fi
         \ifx\fmtname\x\else 1\fi\relax
\else\csname fi\endcsname
%</ignore>
%<*install>
\input docstrip.tex
\Msg{************************************************************************}
\Msg{* Installation}
\Msg{* Package: rerunfilecheck 2016/05/16 v1.8 Rerun checks for auxiliary files (HO)}
\Msg{************************************************************************}

\keepsilent
\askforoverwritefalse

\let\MetaPrefix\relax
\preamble

This is a generated file.

Project: rerunfilecheck
Version: 2016/05/16 v1.8

Copyright (C) 2009-2011 by
   Heiko Oberdiek <heiko.oberdiek at googlemail.com>

This work may be distributed and/or modified under the
conditions of the LaTeX Project Public License, either
version 1.3c of this license or (at your option) any later
version. This version of this license is in
   http://www.latex-project.org/lppl/lppl-1-3c.txt
and the latest version of this license is in
   http://www.latex-project.org/lppl.txt
and version 1.3 or later is part of all distributions of
LaTeX version 2005/12/01 or later.

This work has the LPPL maintenance status "maintained".

This Current Maintainer of this work is Heiko Oberdiek.

This work consists of the main source file rerunfilecheck.dtx
and the derived files
   rerunfilecheck.sty, rerunfilecheck.pdf, rerunfilecheck.ins,
   rerunfilecheck.drv, rerunfilecheck-example.cfg,
   rerunfilecheck-test1.tex.

\endpreamble
\let\MetaPrefix\DoubleperCent

\generate{%
  \file{rerunfilecheck.ins}{\from{rerunfilecheck.dtx}{install}}%
  \file{rerunfilecheck.drv}{\from{rerunfilecheck.dtx}{driver}}%
  \usedir{tex/latex/oberdiek}%
  \file{rerunfilecheck.sty}{\from{rerunfilecheck.dtx}{package}}%
%  \usedir{doc/latex/oberdiek/test}%
%  \file{rerunfilecheck-test1.tex}{\from{rerunfilecheck.dtx}{test1}}%
  \nopreamble
  \nopostamble
  \usedir{doc/latex/oberdiek}%
  \file{rerunfilecheck-example.cfg}{\from{rerunfilecheck.dtx}{cfg}}%
  \usedir{source/latex/oberdiek/catalogue}%
  \file{rerunfilecheck.xml}{\from{rerunfilecheck.dtx}{catalogue}}%
}

\catcode32=13\relax% active space
\let =\space%
\Msg{************************************************************************}
\Msg{*}
\Msg{* To finish the installation you have to move the following}
\Msg{* file into a directory searched by TeX:}
\Msg{*}
\Msg{*     rerunfilecheck.sty}
\Msg{*}
\Msg{* To produce the documentation run the file `rerunfilecheck.drv'}
\Msg{* through LaTeX.}
\Msg{*}
\Msg{* Happy TeXing!}
\Msg{*}
\Msg{************************************************************************}

\endbatchfile
%</install>
%<*ignore>
\fi
%</ignore>
%<*driver>
\NeedsTeXFormat{LaTeX2e}
\ProvidesFile{rerunfilecheck.drv}%
  [2016/05/16 v1.8 Rerun checks for auxiliary files (HO)]%
\documentclass{ltxdoc}
\usepackage{holtxdoc}[2011/11/22]
\begin{document}
  \DocInput{rerunfilecheck.dtx}%
\end{document}
%</driver>
% \fi
%
%
% \CharacterTable
%  {Upper-case    \A\B\C\D\E\F\G\H\I\J\K\L\M\N\O\P\Q\R\S\T\U\V\W\X\Y\Z
%   Lower-case    \a\b\c\d\e\f\g\h\i\j\k\l\m\n\o\p\q\r\s\t\u\v\w\x\y\z
%   Digits        \0\1\2\3\4\5\6\7\8\9
%   Exclamation   \!     Double quote  \"     Hash (number) \#
%   Dollar        \$     Percent       \%     Ampersand     \&
%   Acute accent  \'     Left paren    \(     Right paren   \)
%   Asterisk      \*     Plus          \+     Comma         \,
%   Minus         \-     Point         \.     Solidus       \/
%   Colon         \:     Semicolon     \;     Less than     \<
%   Equals        \=     Greater than  \>     Question mark \?
%   Commercial at \@     Left bracket  \[     Backslash     \\
%   Right bracket \]     Circumflex    \^     Underscore    \_
%   Grave accent  \`     Left brace    \{     Vertical bar  \|
%   Right brace   \}     Tilde         \~}
%
% \GetFileInfo{rerunfilecheck.drv}
%
% \title{The \xpackage{rerunfilecheck} package}
% \date{2016/05/16 v1.8}
% \author{Heiko Oberdiek\thanks
% {Please report any issues at https://github.com/ho-tex/oberdiek/issues}\\
% \xemail{heiko.oberdiek at googlemail.com}}
%
% \maketitle
%
% \begin{abstract}
% The package provides additional rerun warnings if some
% auxiliary files have changed. It is based on MD5 checksum,
% provided by \pdfTeX.
% \end{abstract}
%
% \tableofcontents
%
% \section{Documentation}
%
% \LaTeX\ informs the user, when to run \LaTeX\ again, if the
% references have changed. It has the old references from the
% first reading of the \xext{aux} files already in memory, thus
% it compares them with the new version of the \xext{aux} file
% at the end of the document. However this rerun warnings are
% not given for the table of contents and other data stored
% in the \xext{aux} files or other auxiliary files.
% Usually many of these data as the table of contents is not keep
% in memory. If someone wants to detect changes, he has either
% to keep the data in memory. This does not scale well with huge
% documents. Or he copies the file before they are changed. Slow
% I/O operations cost time.
%
% Since version 1.30.0 \pdfTeX\ provides \cs{pdfmdfivesum}
% and \cs{pdffilesize}.
% These features are also available in \LuaTeX, provided by package
% \xpackage{pdftexcmds}.
% Thus this package \xpackage{rerunfilecheck} uses these features to
% detect file changes. This saves the packages from keeping
% the whole files in memory or in file copies.
% The drawback are different files with the same size and the
% same MD5 checksum (seldom, hopefully).
%
% \subsection{Options}
%
% All options are key value options of boolean type.
% No option or |true| turns an option on, |false| disables
% an option.
% \begin{description}
% \item[\xoption{mainaux}:]
%   Check the main \xext{aux} file.
% \item[\xoption{partaux}:]
%   Check the \xext{aux} files from \cs{include} files.
% \item[\xoption{starttoc}:]
%   Add the rerun checks in \cs{@starttoc} that is called by
%   \cs{tableofcontents}, \cs{listoffigures}, \dots
% \item[\xoption{index}, \xoption{glossary}:]
%   \LaTeX's original \cs{makeindex} and \cs{makeglossary} are
%   redefined to add the rerun checks. The options do not have
%   an effect, if \cs{makeindex}/\cs{makeglossary} are already
%   called or if a package or class had redefined or will
%   redefine them.
% \item[\xoption{aux}:]
%   This option turns all previous options on or off.
%   ``aux'' means auxiliary file.
% \end{description}
% The default for the options is |false|, because
% some internals must be redefined to insert the
% rerun checks.
% The options can be set in \cs{usepackage} or
% the configuration file \xfile{rerunfilecheck.cfg}.
% Global options are ignored (since 1.4).
%
% \begin{declcs}{RerunFileCheckSetup} \M{key value list}
% \end{declcs}
% Options can also be set using \cs{RerunFileCheckSetup}.
% Currently all options are disabled after the package
% is loaded. Thus \cs{RerunFileCheckSetup} makes sense in
% the configuration file only.
%
% Example for the configuration file:
%    \begin{macrocode}
%<*cfg>
\ProvidesFile{rerunfilecheck.cfg}[2016/05/16 Default configuration]%
\RerunFileCheckSetup{aux}
%</cfg>
%    \end{macrocode}
%
% \subsection{Interface for class/package authors}
%
% \begin{declcs}{RerunFileCheck} \M{file} \M{file closing action} \M{rerun warning}
% \end{declcs}
% If you want to add a rerun check, call \cs{RerunFileCheck} right before
% an output file is opened for writing. The macro first remembers
% the current checksum of \meta{file}.
% The file is checked again right before the end of the job.
% Macro \cs{AtVeryEndDocument} of package \xpackage{atveryend}
% is used to place the check after the main aux file is closed
% in |\end{document}|. Before reading the file again, it must be
% closed. Provide the code for closing in argument \meta{file closing action}.
% Do not forget \cs{immediate} before \cs{openout}. Otherwise the
% closing action would be delayed to the next shipout that never happens
% (the last page is already shipped out).
% If the file has changed, \cs{RerunFileCheck} informs the user with
% a warning that the file has changed and says the magic word ``Rerun''.
% If the last argument \meta{rerun warning} is not empty, then
% the rerun sentence is replaced by it. Usually the phrase
% ``to get something right'' is added.
% As example the relevant part of the redefined \cs{makeindex}
% is shown, see package code:
%\begin{quote}
%\begin{verbatim}
%\newwrite\@indexfile
%\RerunFileCheck{\jobname.idx}{%
%  \immediate\closeout\@indexfile
%}{%
%  Rerun LaTeX/makeindex to get index right%
%}%
%\immediate\openout\@indexfile=\jobname.idx %
%\end{verbatim}
%\end{quote}
%
% \StopEventually{
% }
%
% \section{Implementation}
%    \begin{macrocode}
%<*package>
%    \end{macrocode}
%
%    \begin{macrocode}
\begingroup\catcode61\catcode48\catcode32=10\relax%
  \catcode13=5 % ^^M
  \endlinechar=13 %
  \catcode123=1 % {
  \catcode125=2 % }
  \catcode64=11 % @
  \def\x{\endgroup
    \expandafter\edef\csname ReFiCh@AtEnd\endcsname{%
      \endlinechar=\the\endlinechar\relax
      \catcode13=\the\catcode13\relax
      \catcode32=\the\catcode32\relax
      \catcode35=\the\catcode35\relax
      \catcode61=\the\catcode61\relax
      \catcode64=\the\catcode64\relax
      \catcode123=\the\catcode123\relax
      \catcode125=\the\catcode125\relax
    }%
  }%
\x\catcode61\catcode48\catcode32=10\relax%
\catcode13=5 % ^^M
\endlinechar=13 %
\catcode35=6 % #
\catcode64=11 % @
\catcode123=1 % {
\catcode125=2 % }
\def\TMP@EnsureCode#1#2{%
  \edef\ReFiCh@AtEnd{%
    \ReFiCh@AtEnd
    \catcode#1=\the\catcode#1\relax
  }%
  \catcode#1=#2\relax
}
\TMP@EnsureCode{39}{12}% '
\TMP@EnsureCode{40}{12}% (
\TMP@EnsureCode{41}{12}% )
\TMP@EnsureCode{42}{12}% *
\TMP@EnsureCode{44}{12}% ,
\TMP@EnsureCode{46}{12}% .
\TMP@EnsureCode{47}{12}% /
\TMP@EnsureCode{58}{12}% :
\TMP@EnsureCode{59}{12}% ;
\TMP@EnsureCode{60}{12}% <
\TMP@EnsureCode{62}{12}% >
\TMP@EnsureCode{91}{12}% [
\TMP@EnsureCode{93}{12}% ]
\TMP@EnsureCode{96}{12}% `
\edef\ReFiCh@AtEnd{\ReFiCh@AtEnd\noexpand\endinput}
%    \end{macrocode}
%
%    Package identification.
%    \begin{macrocode}
\NeedsTeXFormat{LaTeX2e}
\ProvidesPackage{rerunfilecheck}%
  [2016/05/16 v1.8 Rerun checks for auxiliary files (HO)]
%    \end{macrocode}
%
% \subsection{Options}
%
%    \begin{macrocode}
\RequirePackage{kvoptions}[2010/02/22]
\SetupKeyvalOptions{%
  family=rerunfilecheck,%
  prefix=ReFiCh@%
}
%    \end{macrocode}
%    \begin{macro}{\RerunFileCheckSetup}
%    \begin{macrocode}
\newcommand*{\RerunFileCheckSetup}{%
  \setkeys{rerunfilecheck}%
}
%    \end{macrocode}
%    \end{macro}
%    \begin{macrocode}
\DeclareBoolOption{mainaux}
\DeclareBoolOption{partaux}
\DeclareBoolOption{starttoc}
\DeclareBoolOption{index}
\DeclareBoolOption{glossary}
\define@key{rerunfilecheck}{aux}[true]{%
  \RerunFileCheckSetup{%
    mainaux={#1},%
    partaux={#1},%
    starttoc={#1},%
    index={#1},%
    glossary={#1}%
  }%
}
%    \end{macrocode}
%    \begin{macrocode}
\InputIfFileExists{rerunfilecheck.cfg}{}{}
\ProcessLocalKeyvalOptions*
%    \end{macrocode}
%
%    \begin{macro}{\ReFiCh@DisableOption}
%    \begin{macrocode}
\def\ReFiCh@DisableOption{%
  \DisableKeyvalOption[%
    action=warning,%
    package=rerunfilecheck%
  ]{rerunfilecheck}%
}
%    \end{macrocode}
%    \end{macro}
%
% \subsection{Check for checksum feature}
%
%    \begin{macrocode}
\RequirePackage{infwarerr}[2007/09/09]
\RequirePackage{pdftexcmds}[2009/04/10]
%    \end{macrocode}
%    \begin{macrocode}
\begingroup\expandafter\expandafter\expandafter\endgroup
\expandafter\ifx\csname pdf@filemdfivesum\endcsname\relax
  \@PackageInfoNoLine{rerunfilecheck}{%
    Feature \string\pdfmdfivesum\space is not available\MessageBreak
    (e.g. pdfTeX or LuaTeX with package `pdftexcmds').\MessageBreak
    Therefore file contents cannot be checked efficiently\MessageBreak
    and the loading of the package is aborted%
  }%
  \newcommand*{\RerunFileCheck}[3]{}%
  \renewcommand*{\RerunFileCheckSetup}[1]{}%
  \expandafter\ReFiCh@AtEnd
\fi%
%    \end{macrocode}
%
% \subsection{Standard \xext{aux} files}
%
%    \begin{macrocode}
\ifReFiCh@partaux
  \let\ReFiCh@org@include\@include
  \def\@include#1 {%
    \if@filesw
      \RerunFileCheck{#1.aux}{}{}%
    \fi
    \ReFiCh@org@include{#1} %
  }%
\fi
\ifReFiCh@mainaux
  \AtBeginDocument{%
    \ReFiCh@mainauxfalse
  }%
  \ifReFiCh@mainaux
    \AtEndOfPackage{%
      \RerunFileCheck{\jobname.aux}{}{}%
    }%
  \else
    \if@filesw
      \@PackageWarningNoLine{rerunfilecheck}{%
        Main aux file check is disabled,\MessageBreak
        because the file is already opened.\MessageBreak
        Load the package before \string\begin{document}%
      }%
    \fi
  \fi
\fi
\ifReFiCh@starttoc
  \let\ReFiCh@org@starttoc\@starttoc
  \def\@starttoc#1{%
    \if@filesw
      \RerunFileCheck{\jobname.#1}{%
        \@ifundefined{tf@#1}{%
        }{%
          \immediate\closeout\csname tf@#1\endcsname
        }%
      }{}%
    \fi
    \ReFiCh@org@starttoc{#1}%
  }%
\fi
\ifReFiCh@index
  \ifx\makeindex\@empty
    \@PackageWarningNoLine{rerunfilecheck}{%
      Option `index' ignored,\MessageBreak
      because \string\makeindex\space has already been called%
    }%
  \else
    \def\ReFiCh@temp{%
      \newwrite\@indexfile
      \immediate\openout\@indexfile=\jobname.idx %
      \def\index{%
        \@bsphack
        \begingroup
        \@sanitize
        \@wrindex
      }%
      \typeout{Writing index file \jobname.idx}%
      \let\makeindex\@empty
    }%
    \ifx\ReFiCh@temp\makeindex
      \def\makeindex{%
        \newwrite\@indexfile
        \RerunFileCheck{\jobname.idx}{%
          \immediate\closeout\@indexfile
        }{%
          Rerun LaTeX/makeindex to get index right%
        }%
        \immediate\openout\@indexfile=\jobname.idx %
        \def\index{%
          \@bsphack
          \begingroup
          \@sanitize
          \@wrindex
        }%
        \typeout{Writing index file \jobname.idx}%
        \let\makeindex\@empty
      }%
    \else
      \@PackageInfoNoLine{rerunfilecheck}{%
        Option `index': unsupported version of \string\makeindex
      }%
    \fi
  \fi
\fi
\ifReFiCh@glossary
  \ifx\makeglossary\@empty
    \@PackageWarningNoLine{rerunfilecheck}{%
      Option `glossary' ignored,\MessageBreak
      because \string\makeglossary\space has already been called%
    }%
  \else
    \def\ReFiCh@temp{%
      \newwrite\@glossaryfile
      \immediate\openout\@glossaryfile=\jobname.glo %
      \def\glossary{%
        \@bsphack
        \begingroup
        \@sanitize
        \@wrglossary
      }%
      \typeout{Writing glossary file \jobname.glo }%
      \let\makeglossary\@empty
    }%
    \ifx\ReFiCh@temp\makeglossary
      \def\ReFiCh@temp{%
        \newwrite\@glossaryfile
        \RerunFileCheck{\jobname.glo}{%
          \immediate\closeout\@glossaryfile
        }{%
          Rerun LaTeX/makeindex to get glossary right%
        }%
        \immediate\openout\@glossaryfile=\jobname.glo %
        \def\glossary{%
          \@bsphack
          \begingroup
          \@sanitize
          \@wrglossary
        }%
        \typeout{Writing glossary file \jobname.glo}%
        \let\makeglossary\@empty
      }%
    \else
      \@PackageInfoNoLine{rerunfilecheck}{%
        Option `glossary': unsupported version of \string\makeglossary
      }%
    \fi
  \fi
\fi
\ReFiCh@DisableOption{mainaux}
\ReFiCh@DisableOption{partaux}
\ReFiCh@DisableOption{starttoc}
\ReFiCh@DisableOption{index}
\ReFiCh@DisableOption{glossary}
\ReFiCh@DisableOption{aux}
%    \end{macrocode}
%
% \subsection{Rerun check}
%
%    \begin{macrocode}
\RequirePackage{atveryend}[2016/05/16]
\RequirePackage{uniquecounter}[2009/12/18]
%    \end{macrocode}
%
%    \begin{macro}{\ReFiCh@CheckSum}
%    \begin{macrocode}
\begingroup\expandafter\expandafter\expandafter\endgroup
\expandafter\ifx\csname pdf@filesize\endcsname\relax
  \def\ReFiCh@CheckSum{%
    \pdf@filemdfivesum
  }%
\else
  \def\ReFiCh@CheckSum#1{%
    \pdf@filemdfivesum{#1}%
    \ReFiCh@Separator
    \pdf@filesize{#1}%
  }%
\fi
%    \end{macrocode}
%    \end{macro}
%    \begin{macro}{\ReFiCh@NoFile}
%    \begin{macrocode}
\def\ReFiCh@Separator{;}
%    \end{macrocode}
%    \end{macro}
%    \begin{macro}{\ReFiCh@NoFile}
%    \begin{macrocode}
\def\ReFiCh@NoFile{<no file>}
%    \end{macrocode}
%    \end{macro}
%
%    \begin{macrocode}
\UniqueCounterNew{rerunfilecheck}
%    \end{macrocode}
%
%    \begin{macro}{\RerunFileCheck}
%    \begin{macrocode}
\newcommand*{\RerunFileCheck}{%
  \UniqueCounterCall{rerunfilecheck}\ReFiCh@RerunFileCheck
}
%    \end{macrocode}
%    \end{macro}
%    \begin{macro}{\ReFiCh@RerunFileCheck}
%    \begin{macrocode}
\def\ReFiCh@RerunFileCheck#1{%
  \expandafter\ReFiCh@@RerunFileCheck\csname ReFiCh@#1\endcsname
}
%    \end{macrocode}
%    \end{macro}
%    \begin{macro}{\ReFiCh@Check}
%    \begin{macrocode}
\def\ReFiCh@Check#1#2#3{%
%  \IfFileExists{#3}{%
    #1\edef#2{\ReFiCh@CheckSum{#3}}%
    \ifx#2\ReFiCh@Separator
      #1\let#2\ReFiCh@NoFile
    \fi
%  }{%
%    #1\let#2\ReFiCh@NoFile
%  }%
}
%    \end{macrocode}
%    \end{macro}
%    \begin{macro}{\ReFiCh@@RerunFileCheck}
%    \begin{macrocode}
\def\ReFiCh@@RerunFileCheck#1#2#3#4{%
  \ReFiCh@Check\global#1{#2}%
  \AtEndAfterFileList{%
    \begingroup
      #3%
      \ReFiCh@Check{}\x{#2}%
      \ifx#1\x
        \@PackageInfoNoLine{rerunfilecheck}{%
          File `#2' has not changed.\MessageBreak
          Checksum: \x
        }%
      \else
        \ifnum
          \ReFiCh@IsAux#2\relax.aux\relax\@nil
          \ifx#1\ReFiCh@NoFile 1\else 0\fi
          \ifx\x\ReFiCh@AuxEmptyUnix 1%
          \else
            \ifx\x\ReFiCh@AuxEmptyDos 1\fi
          \fi
        =111 %
          \@PackageInfoNoLine{rerunfilecheck}{%
            File `#2' is empty .aux file.\MessageBreak
            Before: #1\MessageBreak
            After: \space\x
          }%
        \else
          \@PackageWarningNoLine{rerunfilecheck}{%
            File `#2' has changed.%
            \ifx\\#4\\%
              \space Rerun%
            \else
              \MessageBreak
              #4%
            \fi
          }%
          \@PackageInfoNoLine{rerunfilecheck}{%
            Checksums for `#2':\MessageBreak
            Before: #1\MessageBreak
            After: \space\x
          }%
        \fi
      \fi
    \endgroup
  }%
}
%    \end{macrocode}
%    \end{macro}
%    \begin{macrocode}
\def\ReFiCh@IsAux#1.aux\relax#2\@nil{%
  \ifx\hbox#2\hbox
    0%
  \else
    1%
  \fi
}
%    \end{macrocode}
%    \begin{macrocode}
\def\ReFiCh@AuxEmptyUnix{A94A2480D3289E625EEA47CD1B285758;8}%
\@onelevel@sanitize\ReFiCh@AuxEmptyUnix
%    \end{macrocode}
%    \begin{macrocode}
\def\ReFiCh@AuxEmptyDos{A62A15ECE803E2EBE94952FCC9933BC0;9}%
\@onelevel@sanitize\ReFiCh@AuxEmptyDos
%    \end{macrocode}
%
%    \begin{macrocode}
\ReFiCh@AtEnd%
%</package>
%    \end{macrocode}
%
% \section{Test}
%
%    \begin{macrocode}
%<*test1>
\def\LoadCommand{\RequirePackage{rerunfilecheck}[2016/05/16]}
%</test1>
%    \end{macrocode}
% \subsection{Catcode checks for loading}
%
%    \begin{macrocode}
%<*test1>
%    \end{macrocode}
%    \begin{macrocode}
\catcode`\{=1 %
\catcode`\}=2 %
\catcode`\#=6 %
\catcode`\@=11 %
\expandafter\ifx\csname count@\endcsname\relax
  \countdef\count@=255 %
\fi
\expandafter\ifx\csname @gobble\endcsname\relax
  \long\def\@gobble#1{}%
\fi
\expandafter\ifx\csname @firstofone\endcsname\relax
  \long\def\@firstofone#1{#1}%
\fi
\expandafter\ifx\csname loop\endcsname\relax
  \expandafter\@firstofone
\else
  \expandafter\@gobble
\fi
{%
  \def\loop#1\repeat{%
    \def\body{#1}%
    \iterate
  }%
  \def\iterate{%
    \body
      \let\next\iterate
    \else
      \let\next\relax
    \fi
    \next
  }%
  \let\repeat=\fi
}%
\def\RestoreCatcodes{}
\count@=0 %
\loop
  \edef\RestoreCatcodes{%
    \RestoreCatcodes
    \catcode\the\count@=\the\catcode\count@\relax
  }%
\ifnum\count@<255 %
  \advance\count@ 1 %
\repeat

\def\RangeCatcodeInvalid#1#2{%
  \count@=#1\relax
  \loop
    \catcode\count@=15 %
  \ifnum\count@<#2\relax
    \advance\count@ 1 %
  \repeat
}
\def\RangeCatcodeCheck#1#2#3{%
  \count@=#1\relax
  \loop
    \ifnum#3=\catcode\count@
    \else
      \errmessage{%
        Character \the\count@\space
        with wrong catcode \the\catcode\count@\space
        instead of \number#3%
      }%
    \fi
  \ifnum\count@<#2\relax
    \advance\count@ 1 %
  \repeat
}
\def\space{ }
\expandafter\ifx\csname LoadCommand\endcsname\relax
  \def\LoadCommand{\input rerunfilecheck.sty\relax}%
\fi
\def\Test{%
  \RangeCatcodeInvalid{0}{47}%
  \RangeCatcodeInvalid{58}{64}%
  \RangeCatcodeInvalid{91}{96}%
  \RangeCatcodeInvalid{123}{255}%
  \catcode`\@=12 %
  \catcode`\\=0 %
  \catcode`\%=14 %
  \LoadCommand
  \RangeCatcodeCheck{0}{36}{15}%
  \RangeCatcodeCheck{37}{37}{14}%
  \RangeCatcodeCheck{38}{47}{15}%
  \RangeCatcodeCheck{48}{57}{12}%
  \RangeCatcodeCheck{58}{63}{15}%
  \RangeCatcodeCheck{64}{64}{12}%
  \RangeCatcodeCheck{65}{90}{11}%
  \RangeCatcodeCheck{91}{91}{15}%
  \RangeCatcodeCheck{92}{92}{0}%
  \RangeCatcodeCheck{93}{96}{15}%
  \RangeCatcodeCheck{97}{122}{11}%
  \RangeCatcodeCheck{123}{255}{15}%
  \RestoreCatcodes
}
\Test
\csname @@end\endcsname
\end
%    \end{macrocode}
%    \begin{macrocode}
%</test1>
%    \end{macrocode}
%
% \section{Installation}
%
% \subsection{Download}
%
% \paragraph{Package.} This package is available on
% CTAN\footnote{\url{http://ctan.org/pkg/rerunfilecheck}}:
% \begin{description}
% \item[\CTAN{macros/latex/contrib/oberdiek/rerunfilecheck.dtx}] The source file.
% \item[\CTAN{macros/latex/contrib/oberdiek/rerunfilecheck.pdf}] Documentation.
% \end{description}
%
%
% \paragraph{Bundle.} All the packages of the bundle `oberdiek'
% are also available in a TDS compliant ZIP archive. There
% the packages are already unpacked and the documentation files
% are generated. The files and directories obey the TDS standard.
% \begin{description}
% \item[\CTAN{install/macros/latex/contrib/oberdiek.tds.zip}]
% \end{description}
% \emph{TDS} refers to the standard ``A Directory Structure
% for \TeX\ Files'' (\CTAN{tds/tds.pdf}). Directories
% with \xfile{texmf} in their name are usually organized this way.
%
% \subsection{Bundle installation}
%
% \paragraph{Unpacking.} Unpack the \xfile{oberdiek.tds.zip} in the
% TDS tree (also known as \xfile{texmf} tree) of your choice.
% Example (linux):
% \begin{quote}
%   |unzip oberdiek.tds.zip -d ~/texmf|
% \end{quote}
%
% \paragraph{Script installation.}
% Check the directory \xfile{TDS:scripts/oberdiek/} for
% scripts that need further installation steps.
% Package \xpackage{attachfile2} comes with the Perl script
% \xfile{pdfatfi.pl} that should be installed in such a way
% that it can be called as \texttt{pdfatfi}.
% Example (linux):
% \begin{quote}
%   |chmod +x scripts/oberdiek/pdfatfi.pl|\\
%   |cp scripts/oberdiek/pdfatfi.pl /usr/local/bin/|
% \end{quote}
%
% \subsection{Package installation}
%
% \paragraph{Unpacking.} The \xfile{.dtx} file is a self-extracting
% \docstrip\ archive. The files are extracted by running the
% \xfile{.dtx} through \plainTeX:
% \begin{quote}
%   \verb|tex rerunfilecheck.dtx|
% \end{quote}
%
% \paragraph{TDS.} Now the different files must be moved into
% the different directories in your installation TDS tree
% (also known as \xfile{texmf} tree):
% \begin{quote}
% \def\t{^^A
% \begin{tabular}{@{}>{\ttfamily}l@{ $\rightarrow$ }>{\ttfamily}l@{}}
%   rerunfilecheck.sty & tex/latex/oberdiek/rerunfilecheck.sty\\
%   rerunfilecheck.pdf & doc/latex/oberdiek/rerunfilecheck.pdf\\
%   rerunfilecheck-example.cfg & doc/latex/oberdiek/rerunfilecheck-example.cfg\\
%   test/rerunfilecheck-test1.tex & doc/latex/oberdiek/test/rerunfilecheck-test1.tex\\
%   rerunfilecheck.dtx & source/latex/oberdiek/rerunfilecheck.dtx\\
% \end{tabular}^^A
% }^^A
% \sbox0{\t}^^A
% \ifdim\wd0>\linewidth
%   \begingroup
%     \advance\linewidth by\leftmargin
%     \advance\linewidth by\rightmargin
%   \edef\x{\endgroup
%     \def\noexpand\lw{\the\linewidth}^^A
%   }\x
%   \def\lwbox{^^A
%     \leavevmode
%     \hbox to \linewidth{^^A
%       \kern-\leftmargin\relax
%       \hss
%       \usebox0
%       \hss
%       \kern-\rightmargin\relax
%     }^^A
%   }^^A
%   \ifdim\wd0>\lw
%     \sbox0{\small\t}^^A
%     \ifdim\wd0>\linewidth
%       \ifdim\wd0>\lw
%         \sbox0{\footnotesize\t}^^A
%         \ifdim\wd0>\linewidth
%           \ifdim\wd0>\lw
%             \sbox0{\scriptsize\t}^^A
%             \ifdim\wd0>\linewidth
%               \ifdim\wd0>\lw
%                 \sbox0{\tiny\t}^^A
%                 \ifdim\wd0>\linewidth
%                   \lwbox
%                 \else
%                   \usebox0
%                 \fi
%               \else
%                 \lwbox
%               \fi
%             \else
%               \usebox0
%             \fi
%           \else
%             \lwbox
%           \fi
%         \else
%           \usebox0
%         \fi
%       \else
%         \lwbox
%       \fi
%     \else
%       \usebox0
%     \fi
%   \else
%     \lwbox
%   \fi
% \else
%   \usebox0
% \fi
% \end{quote}
% If you have a \xfile{docstrip.cfg} that configures and enables \docstrip's
% TDS installing feature, then some files can already be in the right
% place, see the documentation of \docstrip.
%
% \subsection{Refresh file name databases}
%
% If your \TeX~distribution
% (\teTeX, \mikTeX, \dots) relies on file name databases, you must refresh
% these. For example, \teTeX\ users run \verb|texhash| or
% \verb|mktexlsr|.
%
% \subsection{Some details for the interested}
%
% \paragraph{Attached source.}
%
% The PDF documentation on CTAN also includes the
% \xfile{.dtx} source file. It can be extracted by
% AcrobatReader 6 or higher. Another option is \textsf{pdftk},
% e.g. unpack the file into the current directory:
% \begin{quote}
%   \verb|pdftk rerunfilecheck.pdf unpack_files output .|
% \end{quote}
%
% \paragraph{Unpacking with \LaTeX.}
% The \xfile{.dtx} chooses its action depending on the format:
% \begin{description}
% \item[\plainTeX:] Run \docstrip\ and extract the files.
% \item[\LaTeX:] Generate the documentation.
% \end{description}
% If you insist on using \LaTeX\ for \docstrip\ (really,
% \docstrip\ does not need \LaTeX), then inform the autodetect routine
% about your intention:
% \begin{quote}
%   \verb|latex \let\install=y% \iffalse meta-comment
%
% File: rerunfilecheck.dtx
% Version: 2016/05/16 v1.8
% Info: Rerun checks for auxiliary files
%
% Copyright (C) 2009-2011 by
%    Heiko Oberdiek <heiko.oberdiek at googlemail.com>
%    2016
%    https://github.com/ho-tex/oberdiek/issues
%
% This work may be distributed and/or modified under the
% conditions of the LaTeX Project Public License, either
% version 1.3c of this license or (at your option) any later
% version. This version of this license is in
%    http://www.latex-project.org/lppl/lppl-1-3c.txt
% and the latest version of this license is in
%    http://www.latex-project.org/lppl.txt
% and version 1.3 or later is part of all distributions of
% LaTeX version 2005/12/01 or later.
%
% This work has the LPPL maintenance status "maintained".
%
% This Current Maintainer of this work is Heiko Oberdiek.
%
% This work consists of the main source file rerunfilecheck.dtx
% and the derived files
%    rerunfilecheck.sty, rerunfilecheck.pdf, rerunfilecheck.ins,
%    rerunfilecheck.drv, rerunfilecheck-example.cfg,
%    rerunfilecheck-test1.tex.
%
% Distribution:
%    CTAN:macros/latex/contrib/oberdiek/rerunfilecheck.dtx
%    CTAN:macros/latex/contrib/oberdiek/rerunfilecheck.pdf
%
% Unpacking:
%    (a) If rerunfilecheck.ins is present:
%           tex rerunfilecheck.ins
%    (b) Without rerunfilecheck.ins:
%           tex rerunfilecheck.dtx
%    (c) If you insist on using LaTeX
%           latex \let\install=y\input{rerunfilecheck.dtx}
%        (quote the arguments according to the demands of your shell)
%
% Documentation:
%    (a) If rerunfilecheck.drv is present:
%           latex rerunfilecheck.drv
%    (b) Without rerunfilecheck.drv:
%           latex rerunfilecheck.dtx; ...
%    The class ltxdoc loads the configuration file ltxdoc.cfg
%    if available. Here you can specify further options, e.g.
%    use A4 as paper format:
%       \PassOptionsToClass{a4paper}{article}
%
%    Programm calls to get the documentation (example):
%       pdflatex rerunfilecheck.dtx
%       makeindex -s gind.ist rerunfilecheck.idx
%       pdflatex rerunfilecheck.dtx
%       makeindex -s gind.ist rerunfilecheck.idx
%       pdflatex rerunfilecheck.dtx
%
% Installation:
%    TDS:tex/latex/oberdiek/rerunfilecheck.sty
%    TDS:doc/latex/oberdiek/rerunfilecheck.pdf
%    TDS:doc/latex/oberdiek/rerunfilecheck-example.cfg
%    TDS:doc/latex/oberdiek/test/rerunfilecheck-test1.tex
%    TDS:source/latex/oberdiek/rerunfilecheck.dtx
%
%<*ignore>
\begingroup
  \catcode123=1 %
  \catcode125=2 %
  \def\x{LaTeX2e}%
\expandafter\endgroup
\ifcase 0\ifx\install y1\fi\expandafter
         \ifx\csname processbatchFile\endcsname\relax\else1\fi
         \ifx\fmtname\x\else 1\fi\relax
\else\csname fi\endcsname
%</ignore>
%<*install>
\input docstrip.tex
\Msg{************************************************************************}
\Msg{* Installation}
\Msg{* Package: rerunfilecheck 2016/05/16 v1.8 Rerun checks for auxiliary files (HO)}
\Msg{************************************************************************}

\keepsilent
\askforoverwritefalse

\let\MetaPrefix\relax
\preamble

This is a generated file.

Project: rerunfilecheck
Version: 2016/05/16 v1.8

Copyright (C) 2009-2011 by
   Heiko Oberdiek <heiko.oberdiek at googlemail.com>

This work may be distributed and/or modified under the
conditions of the LaTeX Project Public License, either
version 1.3c of this license or (at your option) any later
version. This version of this license is in
   http://www.latex-project.org/lppl/lppl-1-3c.txt
and the latest version of this license is in
   http://www.latex-project.org/lppl.txt
and version 1.3 or later is part of all distributions of
LaTeX version 2005/12/01 or later.

This work has the LPPL maintenance status "maintained".

This Current Maintainer of this work is Heiko Oberdiek.

This work consists of the main source file rerunfilecheck.dtx
and the derived files
   rerunfilecheck.sty, rerunfilecheck.pdf, rerunfilecheck.ins,
   rerunfilecheck.drv, rerunfilecheck-example.cfg,
   rerunfilecheck-test1.tex.

\endpreamble
\let\MetaPrefix\DoubleperCent

\generate{%
  \file{rerunfilecheck.ins}{\from{rerunfilecheck.dtx}{install}}%
  \file{rerunfilecheck.drv}{\from{rerunfilecheck.dtx}{driver}}%
  \usedir{tex/latex/oberdiek}%
  \file{rerunfilecheck.sty}{\from{rerunfilecheck.dtx}{package}}%
%  \usedir{doc/latex/oberdiek/test}%
%  \file{rerunfilecheck-test1.tex}{\from{rerunfilecheck.dtx}{test1}}%
  \nopreamble
  \nopostamble
  \usedir{doc/latex/oberdiek}%
  \file{rerunfilecheck-example.cfg}{\from{rerunfilecheck.dtx}{cfg}}%
  \usedir{source/latex/oberdiek/catalogue}%
  \file{rerunfilecheck.xml}{\from{rerunfilecheck.dtx}{catalogue}}%
}

\catcode32=13\relax% active space
\let =\space%
\Msg{************************************************************************}
\Msg{*}
\Msg{* To finish the installation you have to move the following}
\Msg{* file into a directory searched by TeX:}
\Msg{*}
\Msg{*     rerunfilecheck.sty}
\Msg{*}
\Msg{* To produce the documentation run the file `rerunfilecheck.drv'}
\Msg{* through LaTeX.}
\Msg{*}
\Msg{* Happy TeXing!}
\Msg{*}
\Msg{************************************************************************}

\endbatchfile
%</install>
%<*ignore>
\fi
%</ignore>
%<*driver>
\NeedsTeXFormat{LaTeX2e}
\ProvidesFile{rerunfilecheck.drv}%
  [2016/05/16 v1.8 Rerun checks for auxiliary files (HO)]%
\documentclass{ltxdoc}
\usepackage{holtxdoc}[2011/11/22]
\begin{document}
  \DocInput{rerunfilecheck.dtx}%
\end{document}
%</driver>
% \fi
%
%
% \CharacterTable
%  {Upper-case    \A\B\C\D\E\F\G\H\I\J\K\L\M\N\O\P\Q\R\S\T\U\V\W\X\Y\Z
%   Lower-case    \a\b\c\d\e\f\g\h\i\j\k\l\m\n\o\p\q\r\s\t\u\v\w\x\y\z
%   Digits        \0\1\2\3\4\5\6\7\8\9
%   Exclamation   \!     Double quote  \"     Hash (number) \#
%   Dollar        \$     Percent       \%     Ampersand     \&
%   Acute accent  \'     Left paren    \(     Right paren   \)
%   Asterisk      \*     Plus          \+     Comma         \,
%   Minus         \-     Point         \.     Solidus       \/
%   Colon         \:     Semicolon     \;     Less than     \<
%   Equals        \=     Greater than  \>     Question mark \?
%   Commercial at \@     Left bracket  \[     Backslash     \\
%   Right bracket \]     Circumflex    \^     Underscore    \_
%   Grave accent  \`     Left brace    \{     Vertical bar  \|
%   Right brace   \}     Tilde         \~}
%
% \GetFileInfo{rerunfilecheck.drv}
%
% \title{The \xpackage{rerunfilecheck} package}
% \date{2016/05/16 v1.8}
% \author{Heiko Oberdiek\thanks
% {Please report any issues at https://github.com/ho-tex/oberdiek/issues}\\
% \xemail{heiko.oberdiek at googlemail.com}}
%
% \maketitle
%
% \begin{abstract}
% The package provides additional rerun warnings if some
% auxiliary files have changed. It is based on MD5 checksum,
% provided by \pdfTeX.
% \end{abstract}
%
% \tableofcontents
%
% \section{Documentation}
%
% \LaTeX\ informs the user, when to run \LaTeX\ again, if the
% references have changed. It has the old references from the
% first reading of the \xext{aux} files already in memory, thus
% it compares them with the new version of the \xext{aux} file
% at the end of the document. However this rerun warnings are
% not given for the table of contents and other data stored
% in the \xext{aux} files or other auxiliary files.
% Usually many of these data as the table of contents is not keep
% in memory. If someone wants to detect changes, he has either
% to keep the data in memory. This does not scale well with huge
% documents. Or he copies the file before they are changed. Slow
% I/O operations cost time.
%
% Since version 1.30.0 \pdfTeX\ provides \cs{pdfmdfivesum}
% and \cs{pdffilesize}.
% These features are also available in \LuaTeX, provided by package
% \xpackage{pdftexcmds}.
% Thus this package \xpackage{rerunfilecheck} uses these features to
% detect file changes. This saves the packages from keeping
% the whole files in memory or in file copies.
% The drawback are different files with the same size and the
% same MD5 checksum (seldom, hopefully).
%
% \subsection{Options}
%
% All options are key value options of boolean type.
% No option or |true| turns an option on, |false| disables
% an option.
% \begin{description}
% \item[\xoption{mainaux}:]
%   Check the main \xext{aux} file.
% \item[\xoption{partaux}:]
%   Check the \xext{aux} files from \cs{include} files.
% \item[\xoption{starttoc}:]
%   Add the rerun checks in \cs{@starttoc} that is called by
%   \cs{tableofcontents}, \cs{listoffigures}, \dots
% \item[\xoption{index}, \xoption{glossary}:]
%   \LaTeX's original \cs{makeindex} and \cs{makeglossary} are
%   redefined to add the rerun checks. The options do not have
%   an effect, if \cs{makeindex}/\cs{makeglossary} are already
%   called or if a package or class had redefined or will
%   redefine them.
% \item[\xoption{aux}:]
%   This option turns all previous options on or off.
%   ``aux'' means auxiliary file.
% \end{description}
% The default for the options is |false|, because
% some internals must be redefined to insert the
% rerun checks.
% The options can be set in \cs{usepackage} or
% the configuration file \xfile{rerunfilecheck.cfg}.
% Global options are ignored (since 1.4).
%
% \begin{declcs}{RerunFileCheckSetup} \M{key value list}
% \end{declcs}
% Options can also be set using \cs{RerunFileCheckSetup}.
% Currently all options are disabled after the package
% is loaded. Thus \cs{RerunFileCheckSetup} makes sense in
% the configuration file only.
%
% Example for the configuration file:
%    \begin{macrocode}
%<*cfg>
\ProvidesFile{rerunfilecheck.cfg}[2016/05/16 Default configuration]%
\RerunFileCheckSetup{aux}
%</cfg>
%    \end{macrocode}
%
% \subsection{Interface for class/package authors}
%
% \begin{declcs}{RerunFileCheck} \M{file} \M{file closing action} \M{rerun warning}
% \end{declcs}
% If you want to add a rerun check, call \cs{RerunFileCheck} right before
% an output file is opened for writing. The macro first remembers
% the current checksum of \meta{file}.
% The file is checked again right before the end of the job.
% Macro \cs{AtVeryEndDocument} of package \xpackage{atveryend}
% is used to place the check after the main aux file is closed
% in |\end{document}|. Before reading the file again, it must be
% closed. Provide the code for closing in argument \meta{file closing action}.
% Do not forget \cs{immediate} before \cs{openout}. Otherwise the
% closing action would be delayed to the next shipout that never happens
% (the last page is already shipped out).
% If the file has changed, \cs{RerunFileCheck} informs the user with
% a warning that the file has changed and says the magic word ``Rerun''.
% If the last argument \meta{rerun warning} is not empty, then
% the rerun sentence is replaced by it. Usually the phrase
% ``to get something right'' is added.
% As example the relevant part of the redefined \cs{makeindex}
% is shown, see package code:
%\begin{quote}
%\begin{verbatim}
%\newwrite\@indexfile
%\RerunFileCheck{\jobname.idx}{%
%  \immediate\closeout\@indexfile
%}{%
%  Rerun LaTeX/makeindex to get index right%
%}%
%\immediate\openout\@indexfile=\jobname.idx %
%\end{verbatim}
%\end{quote}
%
% \StopEventually{
% }
%
% \section{Implementation}
%    \begin{macrocode}
%<*package>
%    \end{macrocode}
%
%    \begin{macrocode}
\begingroup\catcode61\catcode48\catcode32=10\relax%
  \catcode13=5 % ^^M
  \endlinechar=13 %
  \catcode123=1 % {
  \catcode125=2 % }
  \catcode64=11 % @
  \def\x{\endgroup
    \expandafter\edef\csname ReFiCh@AtEnd\endcsname{%
      \endlinechar=\the\endlinechar\relax
      \catcode13=\the\catcode13\relax
      \catcode32=\the\catcode32\relax
      \catcode35=\the\catcode35\relax
      \catcode61=\the\catcode61\relax
      \catcode64=\the\catcode64\relax
      \catcode123=\the\catcode123\relax
      \catcode125=\the\catcode125\relax
    }%
  }%
\x\catcode61\catcode48\catcode32=10\relax%
\catcode13=5 % ^^M
\endlinechar=13 %
\catcode35=6 % #
\catcode64=11 % @
\catcode123=1 % {
\catcode125=2 % }
\def\TMP@EnsureCode#1#2{%
  \edef\ReFiCh@AtEnd{%
    \ReFiCh@AtEnd
    \catcode#1=\the\catcode#1\relax
  }%
  \catcode#1=#2\relax
}
\TMP@EnsureCode{39}{12}% '
\TMP@EnsureCode{40}{12}% (
\TMP@EnsureCode{41}{12}% )
\TMP@EnsureCode{42}{12}% *
\TMP@EnsureCode{44}{12}% ,
\TMP@EnsureCode{46}{12}% .
\TMP@EnsureCode{47}{12}% /
\TMP@EnsureCode{58}{12}% :
\TMP@EnsureCode{59}{12}% ;
\TMP@EnsureCode{60}{12}% <
\TMP@EnsureCode{62}{12}% >
\TMP@EnsureCode{91}{12}% [
\TMP@EnsureCode{93}{12}% ]
\TMP@EnsureCode{96}{12}% `
\edef\ReFiCh@AtEnd{\ReFiCh@AtEnd\noexpand\endinput}
%    \end{macrocode}
%
%    Package identification.
%    \begin{macrocode}
\NeedsTeXFormat{LaTeX2e}
\ProvidesPackage{rerunfilecheck}%
  [2016/05/16 v1.8 Rerun checks for auxiliary files (HO)]
%    \end{macrocode}
%
% \subsection{Options}
%
%    \begin{macrocode}
\RequirePackage{kvoptions}[2010/02/22]
\SetupKeyvalOptions{%
  family=rerunfilecheck,%
  prefix=ReFiCh@%
}
%    \end{macrocode}
%    \begin{macro}{\RerunFileCheckSetup}
%    \begin{macrocode}
\newcommand*{\RerunFileCheckSetup}{%
  \setkeys{rerunfilecheck}%
}
%    \end{macrocode}
%    \end{macro}
%    \begin{macrocode}
\DeclareBoolOption{mainaux}
\DeclareBoolOption{partaux}
\DeclareBoolOption{starttoc}
\DeclareBoolOption{index}
\DeclareBoolOption{glossary}
\define@key{rerunfilecheck}{aux}[true]{%
  \RerunFileCheckSetup{%
    mainaux={#1},%
    partaux={#1},%
    starttoc={#1},%
    index={#1},%
    glossary={#1}%
  }%
}
%    \end{macrocode}
%    \begin{macrocode}
\InputIfFileExists{rerunfilecheck.cfg}{}{}
\ProcessLocalKeyvalOptions*
%    \end{macrocode}
%
%    \begin{macro}{\ReFiCh@DisableOption}
%    \begin{macrocode}
\def\ReFiCh@DisableOption{%
  \DisableKeyvalOption[%
    action=warning,%
    package=rerunfilecheck%
  ]{rerunfilecheck}%
}
%    \end{macrocode}
%    \end{macro}
%
% \subsection{Check for checksum feature}
%
%    \begin{macrocode}
\RequirePackage{infwarerr}[2007/09/09]
\RequirePackage{pdftexcmds}[2009/04/10]
%    \end{macrocode}
%    \begin{macrocode}
\begingroup\expandafter\expandafter\expandafter\endgroup
\expandafter\ifx\csname pdf@filemdfivesum\endcsname\relax
  \@PackageInfoNoLine{rerunfilecheck}{%
    Feature \string\pdfmdfivesum\space is not available\MessageBreak
    (e.g. pdfTeX or LuaTeX with package `pdftexcmds').\MessageBreak
    Therefore file contents cannot be checked efficiently\MessageBreak
    and the loading of the package is aborted%
  }%
  \newcommand*{\RerunFileCheck}[3]{}%
  \renewcommand*{\RerunFileCheckSetup}[1]{}%
  \expandafter\ReFiCh@AtEnd
\fi%
%    \end{macrocode}
%
% \subsection{Standard \xext{aux} files}
%
%    \begin{macrocode}
\ifReFiCh@partaux
  \let\ReFiCh@org@include\@include
  \def\@include#1 {%
    \if@filesw
      \RerunFileCheck{#1.aux}{}{}%
    \fi
    \ReFiCh@org@include{#1} %
  }%
\fi
\ifReFiCh@mainaux
  \AtBeginDocument{%
    \ReFiCh@mainauxfalse
  }%
  \ifReFiCh@mainaux
    \AtEndOfPackage{%
      \RerunFileCheck{\jobname.aux}{}{}%
    }%
  \else
    \if@filesw
      \@PackageWarningNoLine{rerunfilecheck}{%
        Main aux file check is disabled,\MessageBreak
        because the file is already opened.\MessageBreak
        Load the package before \string\begin{document}%
      }%
    \fi
  \fi
\fi
\ifReFiCh@starttoc
  \let\ReFiCh@org@starttoc\@starttoc
  \def\@starttoc#1{%
    \if@filesw
      \RerunFileCheck{\jobname.#1}{%
        \@ifundefined{tf@#1}{%
        }{%
          \immediate\closeout\csname tf@#1\endcsname
        }%
      }{}%
    \fi
    \ReFiCh@org@starttoc{#1}%
  }%
\fi
\ifReFiCh@index
  \ifx\makeindex\@empty
    \@PackageWarningNoLine{rerunfilecheck}{%
      Option `index' ignored,\MessageBreak
      because \string\makeindex\space has already been called%
    }%
  \else
    \def\ReFiCh@temp{%
      \newwrite\@indexfile
      \immediate\openout\@indexfile=\jobname.idx %
      \def\index{%
        \@bsphack
        \begingroup
        \@sanitize
        \@wrindex
      }%
      \typeout{Writing index file \jobname.idx}%
      \let\makeindex\@empty
    }%
    \ifx\ReFiCh@temp\makeindex
      \def\makeindex{%
        \newwrite\@indexfile
        \RerunFileCheck{\jobname.idx}{%
          \immediate\closeout\@indexfile
        }{%
          Rerun LaTeX/makeindex to get index right%
        }%
        \immediate\openout\@indexfile=\jobname.idx %
        \def\index{%
          \@bsphack
          \begingroup
          \@sanitize
          \@wrindex
        }%
        \typeout{Writing index file \jobname.idx}%
        \let\makeindex\@empty
      }%
    \else
      \@PackageInfoNoLine{rerunfilecheck}{%
        Option `index': unsupported version of \string\makeindex
      }%
    \fi
  \fi
\fi
\ifReFiCh@glossary
  \ifx\makeglossary\@empty
    \@PackageWarningNoLine{rerunfilecheck}{%
      Option `glossary' ignored,\MessageBreak
      because \string\makeglossary\space has already been called%
    }%
  \else
    \def\ReFiCh@temp{%
      \newwrite\@glossaryfile
      \immediate\openout\@glossaryfile=\jobname.glo %
      \def\glossary{%
        \@bsphack
        \begingroup
        \@sanitize
        \@wrglossary
      }%
      \typeout{Writing glossary file \jobname.glo }%
      \let\makeglossary\@empty
    }%
    \ifx\ReFiCh@temp\makeglossary
      \def\ReFiCh@temp{%
        \newwrite\@glossaryfile
        \RerunFileCheck{\jobname.glo}{%
          \immediate\closeout\@glossaryfile
        }{%
          Rerun LaTeX/makeindex to get glossary right%
        }%
        \immediate\openout\@glossaryfile=\jobname.glo %
        \def\glossary{%
          \@bsphack
          \begingroup
          \@sanitize
          \@wrglossary
        }%
        \typeout{Writing glossary file \jobname.glo}%
        \let\makeglossary\@empty
      }%
    \else
      \@PackageInfoNoLine{rerunfilecheck}{%
        Option `glossary': unsupported version of \string\makeglossary
      }%
    \fi
  \fi
\fi
\ReFiCh@DisableOption{mainaux}
\ReFiCh@DisableOption{partaux}
\ReFiCh@DisableOption{starttoc}
\ReFiCh@DisableOption{index}
\ReFiCh@DisableOption{glossary}
\ReFiCh@DisableOption{aux}
%    \end{macrocode}
%
% \subsection{Rerun check}
%
%    \begin{macrocode}
\RequirePackage{atveryend}[2016/05/16]
\RequirePackage{uniquecounter}[2009/12/18]
%    \end{macrocode}
%
%    \begin{macro}{\ReFiCh@CheckSum}
%    \begin{macrocode}
\begingroup\expandafter\expandafter\expandafter\endgroup
\expandafter\ifx\csname pdf@filesize\endcsname\relax
  \def\ReFiCh@CheckSum{%
    \pdf@filemdfivesum
  }%
\else
  \def\ReFiCh@CheckSum#1{%
    \pdf@filemdfivesum{#1}%
    \ReFiCh@Separator
    \pdf@filesize{#1}%
  }%
\fi
%    \end{macrocode}
%    \end{macro}
%    \begin{macro}{\ReFiCh@NoFile}
%    \begin{macrocode}
\def\ReFiCh@Separator{;}
%    \end{macrocode}
%    \end{macro}
%    \begin{macro}{\ReFiCh@NoFile}
%    \begin{macrocode}
\def\ReFiCh@NoFile{<no file>}
%    \end{macrocode}
%    \end{macro}
%
%    \begin{macrocode}
\UniqueCounterNew{rerunfilecheck}
%    \end{macrocode}
%
%    \begin{macro}{\RerunFileCheck}
%    \begin{macrocode}
\newcommand*{\RerunFileCheck}{%
  \UniqueCounterCall{rerunfilecheck}\ReFiCh@RerunFileCheck
}
%    \end{macrocode}
%    \end{macro}
%    \begin{macro}{\ReFiCh@RerunFileCheck}
%    \begin{macrocode}
\def\ReFiCh@RerunFileCheck#1{%
  \expandafter\ReFiCh@@RerunFileCheck\csname ReFiCh@#1\endcsname
}
%    \end{macrocode}
%    \end{macro}
%    \begin{macro}{\ReFiCh@Check}
%    \begin{macrocode}
\def\ReFiCh@Check#1#2#3{%
%  \IfFileExists{#3}{%
    #1\edef#2{\ReFiCh@CheckSum{#3}}%
    \ifx#2\ReFiCh@Separator
      #1\let#2\ReFiCh@NoFile
    \fi
%  }{%
%    #1\let#2\ReFiCh@NoFile
%  }%
}
%    \end{macrocode}
%    \end{macro}
%    \begin{macro}{\ReFiCh@@RerunFileCheck}
%    \begin{macrocode}
\def\ReFiCh@@RerunFileCheck#1#2#3#4{%
  \ReFiCh@Check\global#1{#2}%
  \AtEndAfterFileList{%
    \begingroup
      #3%
      \ReFiCh@Check{}\x{#2}%
      \ifx#1\x
        \@PackageInfoNoLine{rerunfilecheck}{%
          File `#2' has not changed.\MessageBreak
          Checksum: \x
        }%
      \else
        \ifnum
          \ReFiCh@IsAux#2\relax.aux\relax\@nil
          \ifx#1\ReFiCh@NoFile 1\else 0\fi
          \ifx\x\ReFiCh@AuxEmptyUnix 1%
          \else
            \ifx\x\ReFiCh@AuxEmptyDos 1\fi
          \fi
        =111 %
          \@PackageInfoNoLine{rerunfilecheck}{%
            File `#2' is empty .aux file.\MessageBreak
            Before: #1\MessageBreak
            After: \space\x
          }%
        \else
          \@PackageWarningNoLine{rerunfilecheck}{%
            File `#2' has changed.%
            \ifx\\#4\\%
              \space Rerun%
            \else
              \MessageBreak
              #4%
            \fi
          }%
          \@PackageInfoNoLine{rerunfilecheck}{%
            Checksums for `#2':\MessageBreak
            Before: #1\MessageBreak
            After: \space\x
          }%
        \fi
      \fi
    \endgroup
  }%
}
%    \end{macrocode}
%    \end{macro}
%    \begin{macrocode}
\def\ReFiCh@IsAux#1.aux\relax#2\@nil{%
  \ifx\hbox#2\hbox
    0%
  \else
    1%
  \fi
}
%    \end{macrocode}
%    \begin{macrocode}
\def\ReFiCh@AuxEmptyUnix{A94A2480D3289E625EEA47CD1B285758;8}%
\@onelevel@sanitize\ReFiCh@AuxEmptyUnix
%    \end{macrocode}
%    \begin{macrocode}
\def\ReFiCh@AuxEmptyDos{A62A15ECE803E2EBE94952FCC9933BC0;9}%
\@onelevel@sanitize\ReFiCh@AuxEmptyDos
%    \end{macrocode}
%
%    \begin{macrocode}
\ReFiCh@AtEnd%
%</package>
%    \end{macrocode}
%
% \section{Test}
%
%    \begin{macrocode}
%<*test1>
\def\LoadCommand{\RequirePackage{rerunfilecheck}[2016/05/16]}
%</test1>
%    \end{macrocode}
% \subsection{Catcode checks for loading}
%
%    \begin{macrocode}
%<*test1>
%    \end{macrocode}
%    \begin{macrocode}
\catcode`\{=1 %
\catcode`\}=2 %
\catcode`\#=6 %
\catcode`\@=11 %
\expandafter\ifx\csname count@\endcsname\relax
  \countdef\count@=255 %
\fi
\expandafter\ifx\csname @gobble\endcsname\relax
  \long\def\@gobble#1{}%
\fi
\expandafter\ifx\csname @firstofone\endcsname\relax
  \long\def\@firstofone#1{#1}%
\fi
\expandafter\ifx\csname loop\endcsname\relax
  \expandafter\@firstofone
\else
  \expandafter\@gobble
\fi
{%
  \def\loop#1\repeat{%
    \def\body{#1}%
    \iterate
  }%
  \def\iterate{%
    \body
      \let\next\iterate
    \else
      \let\next\relax
    \fi
    \next
  }%
  \let\repeat=\fi
}%
\def\RestoreCatcodes{}
\count@=0 %
\loop
  \edef\RestoreCatcodes{%
    \RestoreCatcodes
    \catcode\the\count@=\the\catcode\count@\relax
  }%
\ifnum\count@<255 %
  \advance\count@ 1 %
\repeat

\def\RangeCatcodeInvalid#1#2{%
  \count@=#1\relax
  \loop
    \catcode\count@=15 %
  \ifnum\count@<#2\relax
    \advance\count@ 1 %
  \repeat
}
\def\RangeCatcodeCheck#1#2#3{%
  \count@=#1\relax
  \loop
    \ifnum#3=\catcode\count@
    \else
      \errmessage{%
        Character \the\count@\space
        with wrong catcode \the\catcode\count@\space
        instead of \number#3%
      }%
    \fi
  \ifnum\count@<#2\relax
    \advance\count@ 1 %
  \repeat
}
\def\space{ }
\expandafter\ifx\csname LoadCommand\endcsname\relax
  \def\LoadCommand{\input rerunfilecheck.sty\relax}%
\fi
\def\Test{%
  \RangeCatcodeInvalid{0}{47}%
  \RangeCatcodeInvalid{58}{64}%
  \RangeCatcodeInvalid{91}{96}%
  \RangeCatcodeInvalid{123}{255}%
  \catcode`\@=12 %
  \catcode`\\=0 %
  \catcode`\%=14 %
  \LoadCommand
  \RangeCatcodeCheck{0}{36}{15}%
  \RangeCatcodeCheck{37}{37}{14}%
  \RangeCatcodeCheck{38}{47}{15}%
  \RangeCatcodeCheck{48}{57}{12}%
  \RangeCatcodeCheck{58}{63}{15}%
  \RangeCatcodeCheck{64}{64}{12}%
  \RangeCatcodeCheck{65}{90}{11}%
  \RangeCatcodeCheck{91}{91}{15}%
  \RangeCatcodeCheck{92}{92}{0}%
  \RangeCatcodeCheck{93}{96}{15}%
  \RangeCatcodeCheck{97}{122}{11}%
  \RangeCatcodeCheck{123}{255}{15}%
  \RestoreCatcodes
}
\Test
\csname @@end\endcsname
\end
%    \end{macrocode}
%    \begin{macrocode}
%</test1>
%    \end{macrocode}
%
% \section{Installation}
%
% \subsection{Download}
%
% \paragraph{Package.} This package is available on
% CTAN\footnote{\url{http://ctan.org/pkg/rerunfilecheck}}:
% \begin{description}
% \item[\CTAN{macros/latex/contrib/oberdiek/rerunfilecheck.dtx}] The source file.
% \item[\CTAN{macros/latex/contrib/oberdiek/rerunfilecheck.pdf}] Documentation.
% \end{description}
%
%
% \paragraph{Bundle.} All the packages of the bundle `oberdiek'
% are also available in a TDS compliant ZIP archive. There
% the packages are already unpacked and the documentation files
% are generated. The files and directories obey the TDS standard.
% \begin{description}
% \item[\CTAN{install/macros/latex/contrib/oberdiek.tds.zip}]
% \end{description}
% \emph{TDS} refers to the standard ``A Directory Structure
% for \TeX\ Files'' (\CTAN{tds/tds.pdf}). Directories
% with \xfile{texmf} in their name are usually organized this way.
%
% \subsection{Bundle installation}
%
% \paragraph{Unpacking.} Unpack the \xfile{oberdiek.tds.zip} in the
% TDS tree (also known as \xfile{texmf} tree) of your choice.
% Example (linux):
% \begin{quote}
%   |unzip oberdiek.tds.zip -d ~/texmf|
% \end{quote}
%
% \paragraph{Script installation.}
% Check the directory \xfile{TDS:scripts/oberdiek/} for
% scripts that need further installation steps.
% Package \xpackage{attachfile2} comes with the Perl script
% \xfile{pdfatfi.pl} that should be installed in such a way
% that it can be called as \texttt{pdfatfi}.
% Example (linux):
% \begin{quote}
%   |chmod +x scripts/oberdiek/pdfatfi.pl|\\
%   |cp scripts/oberdiek/pdfatfi.pl /usr/local/bin/|
% \end{quote}
%
% \subsection{Package installation}
%
% \paragraph{Unpacking.} The \xfile{.dtx} file is a self-extracting
% \docstrip\ archive. The files are extracted by running the
% \xfile{.dtx} through \plainTeX:
% \begin{quote}
%   \verb|tex rerunfilecheck.dtx|
% \end{quote}
%
% \paragraph{TDS.} Now the different files must be moved into
% the different directories in your installation TDS tree
% (also known as \xfile{texmf} tree):
% \begin{quote}
% \def\t{^^A
% \begin{tabular}{@{}>{\ttfamily}l@{ $\rightarrow$ }>{\ttfamily}l@{}}
%   rerunfilecheck.sty & tex/latex/oberdiek/rerunfilecheck.sty\\
%   rerunfilecheck.pdf & doc/latex/oberdiek/rerunfilecheck.pdf\\
%   rerunfilecheck-example.cfg & doc/latex/oberdiek/rerunfilecheck-example.cfg\\
%   test/rerunfilecheck-test1.tex & doc/latex/oberdiek/test/rerunfilecheck-test1.tex\\
%   rerunfilecheck.dtx & source/latex/oberdiek/rerunfilecheck.dtx\\
% \end{tabular}^^A
% }^^A
% \sbox0{\t}^^A
% \ifdim\wd0>\linewidth
%   \begingroup
%     \advance\linewidth by\leftmargin
%     \advance\linewidth by\rightmargin
%   \edef\x{\endgroup
%     \def\noexpand\lw{\the\linewidth}^^A
%   }\x
%   \def\lwbox{^^A
%     \leavevmode
%     \hbox to \linewidth{^^A
%       \kern-\leftmargin\relax
%       \hss
%       \usebox0
%       \hss
%       \kern-\rightmargin\relax
%     }^^A
%   }^^A
%   \ifdim\wd0>\lw
%     \sbox0{\small\t}^^A
%     \ifdim\wd0>\linewidth
%       \ifdim\wd0>\lw
%         \sbox0{\footnotesize\t}^^A
%         \ifdim\wd0>\linewidth
%           \ifdim\wd0>\lw
%             \sbox0{\scriptsize\t}^^A
%             \ifdim\wd0>\linewidth
%               \ifdim\wd0>\lw
%                 \sbox0{\tiny\t}^^A
%                 \ifdim\wd0>\linewidth
%                   \lwbox
%                 \else
%                   \usebox0
%                 \fi
%               \else
%                 \lwbox
%               \fi
%             \else
%               \usebox0
%             \fi
%           \else
%             \lwbox
%           \fi
%         \else
%           \usebox0
%         \fi
%       \else
%         \lwbox
%       \fi
%     \else
%       \usebox0
%     \fi
%   \else
%     \lwbox
%   \fi
% \else
%   \usebox0
% \fi
% \end{quote}
% If you have a \xfile{docstrip.cfg} that configures and enables \docstrip's
% TDS installing feature, then some files can already be in the right
% place, see the documentation of \docstrip.
%
% \subsection{Refresh file name databases}
%
% If your \TeX~distribution
% (\teTeX, \mikTeX, \dots) relies on file name databases, you must refresh
% these. For example, \teTeX\ users run \verb|texhash| or
% \verb|mktexlsr|.
%
% \subsection{Some details for the interested}
%
% \paragraph{Attached source.}
%
% The PDF documentation on CTAN also includes the
% \xfile{.dtx} source file. It can be extracted by
% AcrobatReader 6 or higher. Another option is \textsf{pdftk},
% e.g. unpack the file into the current directory:
% \begin{quote}
%   \verb|pdftk rerunfilecheck.pdf unpack_files output .|
% \end{quote}
%
% \paragraph{Unpacking with \LaTeX.}
% The \xfile{.dtx} chooses its action depending on the format:
% \begin{description}
% \item[\plainTeX:] Run \docstrip\ and extract the files.
% \item[\LaTeX:] Generate the documentation.
% \end{description}
% If you insist on using \LaTeX\ for \docstrip\ (really,
% \docstrip\ does not need \LaTeX), then inform the autodetect routine
% about your intention:
% \begin{quote}
%   \verb|latex \let\install=y\input{rerunfilecheck.dtx}|
% \end{quote}
% Do not forget to quote the argument according to the demands
% of your shell.
%
% \paragraph{Generating the documentation.}
% You can use both the \xfile{.dtx} or the \xfile{.drv} to generate
% the documentation. The process can be configured by the
% configuration file \xfile{ltxdoc.cfg}. For instance, put this
% line into this file, if you want to have A4 as paper format:
% \begin{quote}
%   \verb|\PassOptionsToClass{a4paper}{article}|
% \end{quote}
% An example follows how to generate the
% documentation with pdf\LaTeX:
% \begin{quote}
%\begin{verbatim}
%pdflatex rerunfilecheck.dtx
%makeindex -s gind.ist rerunfilecheck.idx
%pdflatex rerunfilecheck.dtx
%makeindex -s gind.ist rerunfilecheck.idx
%pdflatex rerunfilecheck.dtx
%\end{verbatim}
% \end{quote}
%
% \section{Catalogue}
%
% The following XML file can be used as source for the
% \href{http://mirror.ctan.org/help/Catalogue/catalogue.html}{\TeX\ Catalogue}.
% The elements \texttt{caption} and \texttt{description} are imported
% from the original XML file from the Catalogue.
% The name of the XML file in the Catalogue is \xfile{rerunfilecheck.xml}.
%    \begin{macrocode}
%<*catalogue>
<?xml version='1.0' encoding='us-ascii'?>
<!DOCTYPE entry SYSTEM 'catalogue.dtd'>
<entry datestamp='$Date$' modifier='$Author$' id='rerunfilecheck'>
  <name>rerunfilecheck</name>
  <caption>Checksum based rerun checks on auxiliary files.</caption>
  <authorref id='auth:oberdiek'/>
  <copyright owner='Heiko Oberdiek' year='2009-2011'/>
  <license type='lppl1.3'/>
  <version number='1.8'/>
  <description>
    The package provides additional rerun warnings if some
    auxiliary files have changed. It is based on MD5 checksum,
    provided by pdfTeX.
    <p/>
    The package is part of the <xref refid='oberdiek'>oberdiek</xref> bundle.
  </description>
  <documentation details='Package documentation'
      href='ctan:/macros/latex/contrib/oberdiek/rerunfilecheck.pdf'/>
  <ctan file='true' path='/macros/latex/contrib/oberdiek/rerunfilecheck.dtx'/>
  <miktex location='oberdiek'/>
  <texlive location='oberdiek'/>
  <install path='/macros/latex/contrib/oberdiek/oberdiek.tds.zip'/>
</entry>
%</catalogue>
%    \end{macrocode}
%
% \begin{History}
%   \begin{Version}{2009/12/10 v1.0}
%   \item
%     The first version.
%   \end{Version}
%   \begin{Version}{2009/12/12 v1.1}
%   \item
%     Short info shortened.
%   \end{Version}
%   \begin{Version}{2009/12/18 v1.2}
%   \item
%     Required date for package \xpackage{uniquecounter} updated
%     because of bug in this package.
%   \end{Version}
%   \begin{Version}{2010/01/25 v1.3}
%   \item
%     Moved from \texttt{TDS:*/generic/*} to \texttt{TDS:*/latex/*}.
%   \end{Version}
%   \begin{Version}{2010/02/22 v1.4}
%   \item
%     The options of this package are recognized only if they are
%     package options. Global options are ignored. This avoids
%     name clashes with class and other package options
%     (for example, class option `index=totoc').
%   \end{Version}
%   \begin{Version}{2010/03/15 v1.5}
%   \item
%     Call of \cs{pdffivesum} is wrapped in \cs{IfFileExists} to avoid
%     calls of \textsf{mktextex} if this feature is enabled.
%     However \cs{IfFileExists} has file name limitations.
%   \end{Version}
%   \begin{Version}{2010/03/16 v1.6}
%   \item
%     Reverted to version 1.4 and \cs{IfFileExists} wrapper of
%     version 1.5 is removed.
%   \end{Version}
%   \begin{Version}{2011/04/15 v1.7}
%   \item
%     Using \cs{AtEndAfterFileList} of package \xpackage{atveryend}
%     2011/04/15 v1.6 instead of \cs{AtVeryEndDocument}.
%   \end{Version}
%   \begin{Version}{2016/05/16 v1.8}
%   \item
%     Documentation updates.
%   \end{Version}
% \end{History}
%
% \PrintIndex
%
% \Finale
\endinput
|
% \end{quote}
% Do not forget to quote the argument according to the demands
% of your shell.
%
% \paragraph{Generating the documentation.}
% You can use both the \xfile{.dtx} or the \xfile{.drv} to generate
% the documentation. The process can be configured by the
% configuration file \xfile{ltxdoc.cfg}. For instance, put this
% line into this file, if you want to have A4 as paper format:
% \begin{quote}
%   \verb|\PassOptionsToClass{a4paper}{article}|
% \end{quote}
% An example follows how to generate the
% documentation with pdf\LaTeX:
% \begin{quote}
%\begin{verbatim}
%pdflatex rerunfilecheck.dtx
%makeindex -s gind.ist rerunfilecheck.idx
%pdflatex rerunfilecheck.dtx
%makeindex -s gind.ist rerunfilecheck.idx
%pdflatex rerunfilecheck.dtx
%\end{verbatim}
% \end{quote}
%
% \section{Catalogue}
%
% The following XML file can be used as source for the
% \href{http://mirror.ctan.org/help/Catalogue/catalogue.html}{\TeX\ Catalogue}.
% The elements \texttt{caption} and \texttt{description} are imported
% from the original XML file from the Catalogue.
% The name of the XML file in the Catalogue is \xfile{rerunfilecheck.xml}.
%    \begin{macrocode}
%<*catalogue>
<?xml version='1.0' encoding='us-ascii'?>
<!DOCTYPE entry SYSTEM 'catalogue.dtd'>
<entry datestamp='$Date$' modifier='$Author$' id='rerunfilecheck'>
  <name>rerunfilecheck</name>
  <caption>Checksum based rerun checks on auxiliary files.</caption>
  <authorref id='auth:oberdiek'/>
  <copyright owner='Heiko Oberdiek' year='2009-2011'/>
  <license type='lppl1.3'/>
  <version number='1.8'/>
  <description>
    The package provides additional rerun warnings if some
    auxiliary files have changed. It is based on MD5 checksum,
    provided by pdfTeX.
    <p/>
    The package is part of the <xref refid='oberdiek'>oberdiek</xref> bundle.
  </description>
  <documentation details='Package documentation'
      href='ctan:/macros/latex/contrib/oberdiek/rerunfilecheck.pdf'/>
  <ctan file='true' path='/macros/latex/contrib/oberdiek/rerunfilecheck.dtx'/>
  <miktex location='oberdiek'/>
  <texlive location='oberdiek'/>
  <install path='/macros/latex/contrib/oberdiek/oberdiek.tds.zip'/>
</entry>
%</catalogue>
%    \end{macrocode}
%
% \begin{History}
%   \begin{Version}{2009/12/10 v1.0}
%   \item
%     The first version.
%   \end{Version}
%   \begin{Version}{2009/12/12 v1.1}
%   \item
%     Short info shortened.
%   \end{Version}
%   \begin{Version}{2009/12/18 v1.2}
%   \item
%     Required date for package \xpackage{uniquecounter} updated
%     because of bug in this package.
%   \end{Version}
%   \begin{Version}{2010/01/25 v1.3}
%   \item
%     Moved from \texttt{TDS:*/generic/*} to \texttt{TDS:*/latex/*}.
%   \end{Version}
%   \begin{Version}{2010/02/22 v1.4}
%   \item
%     The options of this package are recognized only if they are
%     package options. Global options are ignored. This avoids
%     name clashes with class and other package options
%     (for example, class option `index=totoc').
%   \end{Version}
%   \begin{Version}{2010/03/15 v1.5}
%   \item
%     Call of \cs{pdffivesum} is wrapped in \cs{IfFileExists} to avoid
%     calls of \textsf{mktextex} if this feature is enabled.
%     However \cs{IfFileExists} has file name limitations.
%   \end{Version}
%   \begin{Version}{2010/03/16 v1.6}
%   \item
%     Reverted to version 1.4 and \cs{IfFileExists} wrapper of
%     version 1.5 is removed.
%   \end{Version}
%   \begin{Version}{2011/04/15 v1.7}
%   \item
%     Using \cs{AtEndAfterFileList} of package \xpackage{atveryend}
%     2011/04/15 v1.6 instead of \cs{AtVeryEndDocument}.
%   \end{Version}
%   \begin{Version}{2016/05/16 v1.8}
%   \item
%     Documentation updates.
%   \end{Version}
% \end{History}
%
% \PrintIndex
%
% \Finale
\endinput

%        (quote the arguments according to the demands of your shell)
%
% Documentation:
%    (a) If rerunfilecheck.drv is present:
%           latex rerunfilecheck.drv
%    (b) Without rerunfilecheck.drv:
%           latex rerunfilecheck.dtx; ...
%    The class ltxdoc loads the configuration file ltxdoc.cfg
%    if available. Here you can specify further options, e.g.
%    use A4 as paper format:
%       \PassOptionsToClass{a4paper}{article}
%
%    Programm calls to get the documentation (example):
%       pdflatex rerunfilecheck.dtx
%       makeindex -s gind.ist rerunfilecheck.idx
%       pdflatex rerunfilecheck.dtx
%       makeindex -s gind.ist rerunfilecheck.idx
%       pdflatex rerunfilecheck.dtx
%
% Installation:
%    TDS:tex/latex/oberdiek/rerunfilecheck.sty
%    TDS:doc/latex/oberdiek/rerunfilecheck.pdf
%    TDS:doc/latex/oberdiek/rerunfilecheck-example.cfg
%    TDS:doc/latex/oberdiek/test/rerunfilecheck-test1.tex
%    TDS:source/latex/oberdiek/rerunfilecheck.dtx
%
%<*ignore>
\begingroup
  \catcode123=1 %
  \catcode125=2 %
  \def\x{LaTeX2e}%
\expandafter\endgroup
\ifcase 0\ifx\install y1\fi\expandafter
         \ifx\csname processbatchFile\endcsname\relax\else1\fi
         \ifx\fmtname\x\else 1\fi\relax
\else\csname fi\endcsname
%</ignore>
%<*install>
\input docstrip.tex
\Msg{************************************************************************}
\Msg{* Installation}
\Msg{* Package: rerunfilecheck 2016/05/16 v1.8 Rerun checks for auxiliary files (HO)}
\Msg{************************************************************************}

\keepsilent
\askforoverwritefalse

\let\MetaPrefix\relax
\preamble

This is a generated file.

Project: rerunfilecheck
Version: 2016/05/16 v1.8

Copyright (C) 2009-2011 by
   Heiko Oberdiek <heiko.oberdiek at googlemail.com>

This work may be distributed and/or modified under the
conditions of the LaTeX Project Public License, either
version 1.3c of this license or (at your option) any later
version. This version of this license is in
   http://www.latex-project.org/lppl/lppl-1-3c.txt
and the latest version of this license is in
   http://www.latex-project.org/lppl.txt
and version 1.3 or later is part of all distributions of
LaTeX version 2005/12/01 or later.

This work has the LPPL maintenance status "maintained".

This Current Maintainer of this work is Heiko Oberdiek.

This work consists of the main source file rerunfilecheck.dtx
and the derived files
   rerunfilecheck.sty, rerunfilecheck.pdf, rerunfilecheck.ins,
   rerunfilecheck.drv, rerunfilecheck-example.cfg,
   rerunfilecheck-test1.tex.

\endpreamble
\let\MetaPrefix\DoubleperCent

\generate{%
  \file{rerunfilecheck.ins}{\from{rerunfilecheck.dtx}{install}}%
  \file{rerunfilecheck.drv}{\from{rerunfilecheck.dtx}{driver}}%
  \usedir{tex/latex/oberdiek}%
  \file{rerunfilecheck.sty}{\from{rerunfilecheck.dtx}{package}}%
%  \usedir{doc/latex/oberdiek/test}%
%  \file{rerunfilecheck-test1.tex}{\from{rerunfilecheck.dtx}{test1}}%
  \nopreamble
  \nopostamble
  \usedir{doc/latex/oberdiek}%
  \file{rerunfilecheck-example.cfg}{\from{rerunfilecheck.dtx}{cfg}}%
  \usedir{source/latex/oberdiek/catalogue}%
  \file{rerunfilecheck.xml}{\from{rerunfilecheck.dtx}{catalogue}}%
}

\catcode32=13\relax% active space
\let =\space%
\Msg{************************************************************************}
\Msg{*}
\Msg{* To finish the installation you have to move the following}
\Msg{* file into a directory searched by TeX:}
\Msg{*}
\Msg{*     rerunfilecheck.sty}
\Msg{*}
\Msg{* To produce the documentation run the file `rerunfilecheck.drv'}
\Msg{* through LaTeX.}
\Msg{*}
\Msg{* Happy TeXing!}
\Msg{*}
\Msg{************************************************************************}

\endbatchfile
%</install>
%<*ignore>
\fi
%</ignore>
%<*driver>
\NeedsTeXFormat{LaTeX2e}
\ProvidesFile{rerunfilecheck.drv}%
  [2016/05/16 v1.8 Rerun checks for auxiliary files (HO)]%
\documentclass{ltxdoc}
\usepackage{holtxdoc}[2011/11/22]
\begin{document}
  \DocInput{rerunfilecheck.dtx}%
\end{document}
%</driver>
% \fi
%
%
% \CharacterTable
%  {Upper-case    \A\B\C\D\E\F\G\H\I\J\K\L\M\N\O\P\Q\R\S\T\U\V\W\X\Y\Z
%   Lower-case    \a\b\c\d\e\f\g\h\i\j\k\l\m\n\o\p\q\r\s\t\u\v\w\x\y\z
%   Digits        \0\1\2\3\4\5\6\7\8\9
%   Exclamation   \!     Double quote  \"     Hash (number) \#
%   Dollar        \$     Percent       \%     Ampersand     \&
%   Acute accent  \'     Left paren    \(     Right paren   \)
%   Asterisk      \*     Plus          \+     Comma         \,
%   Minus         \-     Point         \.     Solidus       \/
%   Colon         \:     Semicolon     \;     Less than     \<
%   Equals        \=     Greater than  \>     Question mark \?
%   Commercial at \@     Left bracket  \[     Backslash     \\
%   Right bracket \]     Circumflex    \^     Underscore    \_
%   Grave accent  \`     Left brace    \{     Vertical bar  \|
%   Right brace   \}     Tilde         \~}
%
% \GetFileInfo{rerunfilecheck.drv}
%
% \title{The \xpackage{rerunfilecheck} package}
% \date{2016/05/16 v1.8}
% \author{Heiko Oberdiek\thanks
% {Please report any issues at https://github.com/ho-tex/oberdiek/issues}\\
% \xemail{heiko.oberdiek at googlemail.com}}
%
% \maketitle
%
% \begin{abstract}
% The package provides additional rerun warnings if some
% auxiliary files have changed. It is based on MD5 checksum,
% provided by \pdfTeX.
% \end{abstract}
%
% \tableofcontents
%
% \section{Documentation}
%
% \LaTeX\ informs the user, when to run \LaTeX\ again, if the
% references have changed. It has the old references from the
% first reading of the \xext{aux} files already in memory, thus
% it compares them with the new version of the \xext{aux} file
% at the end of the document. However this rerun warnings are
% not given for the table of contents and other data stored
% in the \xext{aux} files or other auxiliary files.
% Usually many of these data as the table of contents is not keep
% in memory. If someone wants to detect changes, he has either
% to keep the data in memory. This does not scale well with huge
% documents. Or he copies the file before they are changed. Slow
% I/O operations cost time.
%
% Since version 1.30.0 \pdfTeX\ provides \cs{pdfmdfivesum}
% and \cs{pdffilesize}.
% These features are also available in \LuaTeX, provided by package
% \xpackage{pdftexcmds}.
% Thus this package \xpackage{rerunfilecheck} uses these features to
% detect file changes. This saves the packages from keeping
% the whole files in memory or in file copies.
% The drawback are different files with the same size and the
% same MD5 checksum (seldom, hopefully).
%
% \subsection{Options}
%
% All options are key value options of boolean type.
% No option or |true| turns an option on, |false| disables
% an option.
% \begin{description}
% \item[\xoption{mainaux}:]
%   Check the main \xext{aux} file.
% \item[\xoption{partaux}:]
%   Check the \xext{aux} files from \cs{include} files.
% \item[\xoption{starttoc}:]
%   Add the rerun checks in \cs{@starttoc} that is called by
%   \cs{tableofcontents}, \cs{listoffigures}, \dots
% \item[\xoption{index}, \xoption{glossary}:]
%   \LaTeX's original \cs{makeindex} and \cs{makeglossary} are
%   redefined to add the rerun checks. The options do not have
%   an effect, if \cs{makeindex}/\cs{makeglossary} are already
%   called or if a package or class had redefined or will
%   redefine them.
% \item[\xoption{aux}:]
%   This option turns all previous options on or off.
%   ``aux'' means auxiliary file.
% \end{description}
% The default for the options is |false|, because
% some internals must be redefined to insert the
% rerun checks.
% The options can be set in \cs{usepackage} or
% the configuration file \xfile{rerunfilecheck.cfg}.
% Global options are ignored (since 1.4).
%
% \begin{declcs}{RerunFileCheckSetup} \M{key value list}
% \end{declcs}
% Options can also be set using \cs{RerunFileCheckSetup}.
% Currently all options are disabled after the package
% is loaded. Thus \cs{RerunFileCheckSetup} makes sense in
% the configuration file only.
%
% Example for the configuration file:
%    \begin{macrocode}
%<*cfg>
\ProvidesFile{rerunfilecheck.cfg}[2016/05/16 Default configuration]%
\RerunFileCheckSetup{aux}
%</cfg>
%    \end{macrocode}
%
% \subsection{Interface for class/package authors}
%
% \begin{declcs}{RerunFileCheck} \M{file} \M{file closing action} \M{rerun warning}
% \end{declcs}
% If you want to add a rerun check, call \cs{RerunFileCheck} right before
% an output file is opened for writing. The macro first remembers
% the current checksum of \meta{file}.
% The file is checked again right before the end of the job.
% Macro \cs{AtVeryEndDocument} of package \xpackage{atveryend}
% is used to place the check after the main aux file is closed
% in |\end{document}|. Before reading the file again, it must be
% closed. Provide the code for closing in argument \meta{file closing action}.
% Do not forget \cs{immediate} before \cs{openout}. Otherwise the
% closing action would be delayed to the next shipout that never happens
% (the last page is already shipped out).
% If the file has changed, \cs{RerunFileCheck} informs the user with
% a warning that the file has changed and says the magic word ``Rerun''.
% If the last argument \meta{rerun warning} is not empty, then
% the rerun sentence is replaced by it. Usually the phrase
% ``to get something right'' is added.
% As example the relevant part of the redefined \cs{makeindex}
% is shown, see package code:
%\begin{quote}
%\begin{verbatim}
%\newwrite\@indexfile
%\RerunFileCheck{\jobname.idx}{%
%  \immediate\closeout\@indexfile
%}{%
%  Rerun LaTeX/makeindex to get index right%
%}%
%\immediate\openout\@indexfile=\jobname.idx %
%\end{verbatim}
%\end{quote}
%
% \StopEventually{
% }
%
% \section{Implementation}
%    \begin{macrocode}
%<*package>
%    \end{macrocode}
%
%    \begin{macrocode}
\begingroup\catcode61\catcode48\catcode32=10\relax%
  \catcode13=5 % ^^M
  \endlinechar=13 %
  \catcode123=1 % {
  \catcode125=2 % }
  \catcode64=11 % @
  \def\x{\endgroup
    \expandafter\edef\csname ReFiCh@AtEnd\endcsname{%
      \endlinechar=\the\endlinechar\relax
      \catcode13=\the\catcode13\relax
      \catcode32=\the\catcode32\relax
      \catcode35=\the\catcode35\relax
      \catcode61=\the\catcode61\relax
      \catcode64=\the\catcode64\relax
      \catcode123=\the\catcode123\relax
      \catcode125=\the\catcode125\relax
    }%
  }%
\x\catcode61\catcode48\catcode32=10\relax%
\catcode13=5 % ^^M
\endlinechar=13 %
\catcode35=6 % #
\catcode64=11 % @
\catcode123=1 % {
\catcode125=2 % }
\def\TMP@EnsureCode#1#2{%
  \edef\ReFiCh@AtEnd{%
    \ReFiCh@AtEnd
    \catcode#1=\the\catcode#1\relax
  }%
  \catcode#1=#2\relax
}
\TMP@EnsureCode{39}{12}% '
\TMP@EnsureCode{40}{12}% (
\TMP@EnsureCode{41}{12}% )
\TMP@EnsureCode{42}{12}% *
\TMP@EnsureCode{44}{12}% ,
\TMP@EnsureCode{46}{12}% .
\TMP@EnsureCode{47}{12}% /
\TMP@EnsureCode{58}{12}% :
\TMP@EnsureCode{59}{12}% ;
\TMP@EnsureCode{60}{12}% <
\TMP@EnsureCode{62}{12}% >
\TMP@EnsureCode{91}{12}% [
\TMP@EnsureCode{93}{12}% ]
\TMP@EnsureCode{96}{12}% `
\edef\ReFiCh@AtEnd{\ReFiCh@AtEnd\noexpand\endinput}
%    \end{macrocode}
%
%    Package identification.
%    \begin{macrocode}
\NeedsTeXFormat{LaTeX2e}
\ProvidesPackage{rerunfilecheck}%
  [2016/05/16 v1.8 Rerun checks for auxiliary files (HO)]
%    \end{macrocode}
%
% \subsection{Options}
%
%    \begin{macrocode}
\RequirePackage{kvoptions}[2010/02/22]
\SetupKeyvalOptions{%
  family=rerunfilecheck,%
  prefix=ReFiCh@%
}
%    \end{macrocode}
%    \begin{macro}{\RerunFileCheckSetup}
%    \begin{macrocode}
\newcommand*{\RerunFileCheckSetup}{%
  \setkeys{rerunfilecheck}%
}
%    \end{macrocode}
%    \end{macro}
%    \begin{macrocode}
\DeclareBoolOption{mainaux}
\DeclareBoolOption{partaux}
\DeclareBoolOption{starttoc}
\DeclareBoolOption{index}
\DeclareBoolOption{glossary}
\define@key{rerunfilecheck}{aux}[true]{%
  \RerunFileCheckSetup{%
    mainaux={#1},%
    partaux={#1},%
    starttoc={#1},%
    index={#1},%
    glossary={#1}%
  }%
}
%    \end{macrocode}
%    \begin{macrocode}
\InputIfFileExists{rerunfilecheck.cfg}{}{}
\ProcessLocalKeyvalOptions*
%    \end{macrocode}
%
%    \begin{macro}{\ReFiCh@DisableOption}
%    \begin{macrocode}
\def\ReFiCh@DisableOption{%
  \DisableKeyvalOption[%
    action=warning,%
    package=rerunfilecheck%
  ]{rerunfilecheck}%
}
%    \end{macrocode}
%    \end{macro}
%
% \subsection{Check for checksum feature}
%
%    \begin{macrocode}
\RequirePackage{infwarerr}[2007/09/09]
\RequirePackage{pdftexcmds}[2009/04/10]
%    \end{macrocode}
%    \begin{macrocode}
\begingroup\expandafter\expandafter\expandafter\endgroup
\expandafter\ifx\csname pdf@filemdfivesum\endcsname\relax
  \@PackageInfoNoLine{rerunfilecheck}{%
    Feature \string\pdfmdfivesum\space is not available\MessageBreak
    (e.g. pdfTeX or LuaTeX with package `pdftexcmds').\MessageBreak
    Therefore file contents cannot be checked efficiently\MessageBreak
    and the loading of the package is aborted%
  }%
  \newcommand*{\RerunFileCheck}[3]{}%
  \renewcommand*{\RerunFileCheckSetup}[1]{}%
  \expandafter\ReFiCh@AtEnd
\fi%
%    \end{macrocode}
%
% \subsection{Standard \xext{aux} files}
%
%    \begin{macrocode}
\ifReFiCh@partaux
  \let\ReFiCh@org@include\@include
  \def\@include#1 {%
    \if@filesw
      \RerunFileCheck{#1.aux}{}{}%
    \fi
    \ReFiCh@org@include{#1} %
  }%
\fi
\ifReFiCh@mainaux
  \AtBeginDocument{%
    \ReFiCh@mainauxfalse
  }%
  \ifReFiCh@mainaux
    \AtEndOfPackage{%
      \RerunFileCheck{\jobname.aux}{}{}%
    }%
  \else
    \if@filesw
      \@PackageWarningNoLine{rerunfilecheck}{%
        Main aux file check is disabled,\MessageBreak
        because the file is already opened.\MessageBreak
        Load the package before \string\begin{document}%
      }%
    \fi
  \fi
\fi
\ifReFiCh@starttoc
  \let\ReFiCh@org@starttoc\@starttoc
  \def\@starttoc#1{%
    \if@filesw
      \RerunFileCheck{\jobname.#1}{%
        \@ifundefined{tf@#1}{%
        }{%
          \immediate\closeout\csname tf@#1\endcsname
        }%
      }{}%
    \fi
    \ReFiCh@org@starttoc{#1}%
  }%
\fi
\ifReFiCh@index
  \ifx\makeindex\@empty
    \@PackageWarningNoLine{rerunfilecheck}{%
      Option `index' ignored,\MessageBreak
      because \string\makeindex\space has already been called%
    }%
  \else
    \def\ReFiCh@temp{%
      \newwrite\@indexfile
      \immediate\openout\@indexfile=\jobname.idx %
      \def\index{%
        \@bsphack
        \begingroup
        \@sanitize
        \@wrindex
      }%
      \typeout{Writing index file \jobname.idx}%
      \let\makeindex\@empty
    }%
    \ifx\ReFiCh@temp\makeindex
      \def\makeindex{%
        \newwrite\@indexfile
        \RerunFileCheck{\jobname.idx}{%
          \immediate\closeout\@indexfile
        }{%
          Rerun LaTeX/makeindex to get index right%
        }%
        \immediate\openout\@indexfile=\jobname.idx %
        \def\index{%
          \@bsphack
          \begingroup
          \@sanitize
          \@wrindex
        }%
        \typeout{Writing index file \jobname.idx}%
        \let\makeindex\@empty
      }%
    \else
      \@PackageInfoNoLine{rerunfilecheck}{%
        Option `index': unsupported version of \string\makeindex
      }%
    \fi
  \fi
\fi
\ifReFiCh@glossary
  \ifx\makeglossary\@empty
    \@PackageWarningNoLine{rerunfilecheck}{%
      Option `glossary' ignored,\MessageBreak
      because \string\makeglossary\space has already been called%
    }%
  \else
    \def\ReFiCh@temp{%
      \newwrite\@glossaryfile
      \immediate\openout\@glossaryfile=\jobname.glo %
      \def\glossary{%
        \@bsphack
        \begingroup
        \@sanitize
        \@wrglossary
      }%
      \typeout{Writing glossary file \jobname.glo }%
      \let\makeglossary\@empty
    }%
    \ifx\ReFiCh@temp\makeglossary
      \def\ReFiCh@temp{%
        \newwrite\@glossaryfile
        \RerunFileCheck{\jobname.glo}{%
          \immediate\closeout\@glossaryfile
        }{%
          Rerun LaTeX/makeindex to get glossary right%
        }%
        \immediate\openout\@glossaryfile=\jobname.glo %
        \def\glossary{%
          \@bsphack
          \begingroup
          \@sanitize
          \@wrglossary
        }%
        \typeout{Writing glossary file \jobname.glo}%
        \let\makeglossary\@empty
      }%
    \else
      \@PackageInfoNoLine{rerunfilecheck}{%
        Option `glossary': unsupported version of \string\makeglossary
      }%
    \fi
  \fi
\fi
\ReFiCh@DisableOption{mainaux}
\ReFiCh@DisableOption{partaux}
\ReFiCh@DisableOption{starttoc}
\ReFiCh@DisableOption{index}
\ReFiCh@DisableOption{glossary}
\ReFiCh@DisableOption{aux}
%    \end{macrocode}
%
% \subsection{Rerun check}
%
%    \begin{macrocode}
\RequirePackage{atveryend}[2016/05/16]
\RequirePackage{uniquecounter}[2009/12/18]
%    \end{macrocode}
%
%    \begin{macro}{\ReFiCh@CheckSum}
%    \begin{macrocode}
\begingroup\expandafter\expandafter\expandafter\endgroup
\expandafter\ifx\csname pdf@filesize\endcsname\relax
  \def\ReFiCh@CheckSum{%
    \pdf@filemdfivesum
  }%
\else
  \def\ReFiCh@CheckSum#1{%
    \pdf@filemdfivesum{#1}%
    \ReFiCh@Separator
    \pdf@filesize{#1}%
  }%
\fi
%    \end{macrocode}
%    \end{macro}
%    \begin{macro}{\ReFiCh@NoFile}
%    \begin{macrocode}
\def\ReFiCh@Separator{;}
%    \end{macrocode}
%    \end{macro}
%    \begin{macro}{\ReFiCh@NoFile}
%    \begin{macrocode}
\def\ReFiCh@NoFile{<no file>}
%    \end{macrocode}
%    \end{macro}
%
%    \begin{macrocode}
\UniqueCounterNew{rerunfilecheck}
%    \end{macrocode}
%
%    \begin{macro}{\RerunFileCheck}
%    \begin{macrocode}
\newcommand*{\RerunFileCheck}{%
  \UniqueCounterCall{rerunfilecheck}\ReFiCh@RerunFileCheck
}
%    \end{macrocode}
%    \end{macro}
%    \begin{macro}{\ReFiCh@RerunFileCheck}
%    \begin{macrocode}
\def\ReFiCh@RerunFileCheck#1{%
  \expandafter\ReFiCh@@RerunFileCheck\csname ReFiCh@#1\endcsname
}
%    \end{macrocode}
%    \end{macro}
%    \begin{macro}{\ReFiCh@Check}
%    \begin{macrocode}
\def\ReFiCh@Check#1#2#3{%
%  \IfFileExists{#3}{%
    #1\edef#2{\ReFiCh@CheckSum{#3}}%
    \ifx#2\ReFiCh@Separator
      #1\let#2\ReFiCh@NoFile
    \fi
%  }{%
%    #1\let#2\ReFiCh@NoFile
%  }%
}
%    \end{macrocode}
%    \end{macro}
%    \begin{macro}{\ReFiCh@@RerunFileCheck}
%    \begin{macrocode}
\def\ReFiCh@@RerunFileCheck#1#2#3#4{%
  \ReFiCh@Check\global#1{#2}%
  \AtEndAfterFileList{%
    \begingroup
      #3%
      \ReFiCh@Check{}\x{#2}%
      \ifx#1\x
        \@PackageInfoNoLine{rerunfilecheck}{%
          File `#2' has not changed.\MessageBreak
          Checksum: \x
        }%
      \else
        \ifnum
          \ReFiCh@IsAux#2\relax.aux\relax\@nil
          \ifx#1\ReFiCh@NoFile 1\else 0\fi
          \ifx\x\ReFiCh@AuxEmptyUnix 1%
          \else
            \ifx\x\ReFiCh@AuxEmptyDos 1\fi
          \fi
        =111 %
          \@PackageInfoNoLine{rerunfilecheck}{%
            File `#2' is empty .aux file.\MessageBreak
            Before: #1\MessageBreak
            After: \space\x
          }%
        \else
          \@PackageWarningNoLine{rerunfilecheck}{%
            File `#2' has changed.%
            \ifx\\#4\\%
              \space Rerun%
            \else
              \MessageBreak
              #4%
            \fi
          }%
          \@PackageInfoNoLine{rerunfilecheck}{%
            Checksums for `#2':\MessageBreak
            Before: #1\MessageBreak
            After: \space\x
          }%
        \fi
      \fi
    \endgroup
  }%
}
%    \end{macrocode}
%    \end{macro}
%    \begin{macrocode}
\def\ReFiCh@IsAux#1.aux\relax#2\@nil{%
  \ifx\hbox#2\hbox
    0%
  \else
    1%
  \fi
}
%    \end{macrocode}
%    \begin{macrocode}
\def\ReFiCh@AuxEmptyUnix{A94A2480D3289E625EEA47CD1B285758;8}%
\@onelevel@sanitize\ReFiCh@AuxEmptyUnix
%    \end{macrocode}
%    \begin{macrocode}
\def\ReFiCh@AuxEmptyDos{A62A15ECE803E2EBE94952FCC9933BC0;9}%
\@onelevel@sanitize\ReFiCh@AuxEmptyDos
%    \end{macrocode}
%
%    \begin{macrocode}
\ReFiCh@AtEnd%
%</package>
%    \end{macrocode}
%
% \section{Test}
%
%    \begin{macrocode}
%<*test1>
\def\LoadCommand{\RequirePackage{rerunfilecheck}[2016/05/16]}
%</test1>
%    \end{macrocode}
% \subsection{Catcode checks for loading}
%
%    \begin{macrocode}
%<*test1>
%    \end{macrocode}
%    \begin{macrocode}
\catcode`\{=1 %
\catcode`\}=2 %
\catcode`\#=6 %
\catcode`\@=11 %
\expandafter\ifx\csname count@\endcsname\relax
  \countdef\count@=255 %
\fi
\expandafter\ifx\csname @gobble\endcsname\relax
  \long\def\@gobble#1{}%
\fi
\expandafter\ifx\csname @firstofone\endcsname\relax
  \long\def\@firstofone#1{#1}%
\fi
\expandafter\ifx\csname loop\endcsname\relax
  \expandafter\@firstofone
\else
  \expandafter\@gobble
\fi
{%
  \def\loop#1\repeat{%
    \def\body{#1}%
    \iterate
  }%
  \def\iterate{%
    \body
      \let\next\iterate
    \else
      \let\next\relax
    \fi
    \next
  }%
  \let\repeat=\fi
}%
\def\RestoreCatcodes{}
\count@=0 %
\loop
  \edef\RestoreCatcodes{%
    \RestoreCatcodes
    \catcode\the\count@=\the\catcode\count@\relax
  }%
\ifnum\count@<255 %
  \advance\count@ 1 %
\repeat

\def\RangeCatcodeInvalid#1#2{%
  \count@=#1\relax
  \loop
    \catcode\count@=15 %
  \ifnum\count@<#2\relax
    \advance\count@ 1 %
  \repeat
}
\def\RangeCatcodeCheck#1#2#3{%
  \count@=#1\relax
  \loop
    \ifnum#3=\catcode\count@
    \else
      \errmessage{%
        Character \the\count@\space
        with wrong catcode \the\catcode\count@\space
        instead of \number#3%
      }%
    \fi
  \ifnum\count@<#2\relax
    \advance\count@ 1 %
  \repeat
}
\def\space{ }
\expandafter\ifx\csname LoadCommand\endcsname\relax
  \def\LoadCommand{\input rerunfilecheck.sty\relax}%
\fi
\def\Test{%
  \RangeCatcodeInvalid{0}{47}%
  \RangeCatcodeInvalid{58}{64}%
  \RangeCatcodeInvalid{91}{96}%
  \RangeCatcodeInvalid{123}{255}%
  \catcode`\@=12 %
  \catcode`\\=0 %
  \catcode`\%=14 %
  \LoadCommand
  \RangeCatcodeCheck{0}{36}{15}%
  \RangeCatcodeCheck{37}{37}{14}%
  \RangeCatcodeCheck{38}{47}{15}%
  \RangeCatcodeCheck{48}{57}{12}%
  \RangeCatcodeCheck{58}{63}{15}%
  \RangeCatcodeCheck{64}{64}{12}%
  \RangeCatcodeCheck{65}{90}{11}%
  \RangeCatcodeCheck{91}{91}{15}%
  \RangeCatcodeCheck{92}{92}{0}%
  \RangeCatcodeCheck{93}{96}{15}%
  \RangeCatcodeCheck{97}{122}{11}%
  \RangeCatcodeCheck{123}{255}{15}%
  \RestoreCatcodes
}
\Test
\csname @@end\endcsname
\end
%    \end{macrocode}
%    \begin{macrocode}
%</test1>
%    \end{macrocode}
%
% \section{Installation}
%
% \subsection{Download}
%
% \paragraph{Package.} This package is available on
% CTAN\footnote{\url{http://ctan.org/pkg/rerunfilecheck}}:
% \begin{description}
% \item[\CTAN{macros/latex/contrib/oberdiek/rerunfilecheck.dtx}] The source file.
% \item[\CTAN{macros/latex/contrib/oberdiek/rerunfilecheck.pdf}] Documentation.
% \end{description}
%
%
% \paragraph{Bundle.} All the packages of the bundle `oberdiek'
% are also available in a TDS compliant ZIP archive. There
% the packages are already unpacked and the documentation files
% are generated. The files and directories obey the TDS standard.
% \begin{description}
% \item[\CTAN{install/macros/latex/contrib/oberdiek.tds.zip}]
% \end{description}
% \emph{TDS} refers to the standard ``A Directory Structure
% for \TeX\ Files'' (\CTAN{tds/tds.pdf}). Directories
% with \xfile{texmf} in their name are usually organized this way.
%
% \subsection{Bundle installation}
%
% \paragraph{Unpacking.} Unpack the \xfile{oberdiek.tds.zip} in the
% TDS tree (also known as \xfile{texmf} tree) of your choice.
% Example (linux):
% \begin{quote}
%   |unzip oberdiek.tds.zip -d ~/texmf|
% \end{quote}
%
% \paragraph{Script installation.}
% Check the directory \xfile{TDS:scripts/oberdiek/} for
% scripts that need further installation steps.
% Package \xpackage{attachfile2} comes with the Perl script
% \xfile{pdfatfi.pl} that should be installed in such a way
% that it can be called as \texttt{pdfatfi}.
% Example (linux):
% \begin{quote}
%   |chmod +x scripts/oberdiek/pdfatfi.pl|\\
%   |cp scripts/oberdiek/pdfatfi.pl /usr/local/bin/|
% \end{quote}
%
% \subsection{Package installation}
%
% \paragraph{Unpacking.} The \xfile{.dtx} file is a self-extracting
% \docstrip\ archive. The files are extracted by running the
% \xfile{.dtx} through \plainTeX:
% \begin{quote}
%   \verb|tex rerunfilecheck.dtx|
% \end{quote}
%
% \paragraph{TDS.} Now the different files must be moved into
% the different directories in your installation TDS tree
% (also known as \xfile{texmf} tree):
% \begin{quote}
% \def\t{^^A
% \begin{tabular}{@{}>{\ttfamily}l@{ $\rightarrow$ }>{\ttfamily}l@{}}
%   rerunfilecheck.sty & tex/latex/oberdiek/rerunfilecheck.sty\\
%   rerunfilecheck.pdf & doc/latex/oberdiek/rerunfilecheck.pdf\\
%   rerunfilecheck-example.cfg & doc/latex/oberdiek/rerunfilecheck-example.cfg\\
%   test/rerunfilecheck-test1.tex & doc/latex/oberdiek/test/rerunfilecheck-test1.tex\\
%   rerunfilecheck.dtx & source/latex/oberdiek/rerunfilecheck.dtx\\
% \end{tabular}^^A
% }^^A
% \sbox0{\t}^^A
% \ifdim\wd0>\linewidth
%   \begingroup
%     \advance\linewidth by\leftmargin
%     \advance\linewidth by\rightmargin
%   \edef\x{\endgroup
%     \def\noexpand\lw{\the\linewidth}^^A
%   }\x
%   \def\lwbox{^^A
%     \leavevmode
%     \hbox to \linewidth{^^A
%       \kern-\leftmargin\relax
%       \hss
%       \usebox0
%       \hss
%       \kern-\rightmargin\relax
%     }^^A
%   }^^A
%   \ifdim\wd0>\lw
%     \sbox0{\small\t}^^A
%     \ifdim\wd0>\linewidth
%       \ifdim\wd0>\lw
%         \sbox0{\footnotesize\t}^^A
%         \ifdim\wd0>\linewidth
%           \ifdim\wd0>\lw
%             \sbox0{\scriptsize\t}^^A
%             \ifdim\wd0>\linewidth
%               \ifdim\wd0>\lw
%                 \sbox0{\tiny\t}^^A
%                 \ifdim\wd0>\linewidth
%                   \lwbox
%                 \else
%                   \usebox0
%                 \fi
%               \else
%                 \lwbox
%               \fi
%             \else
%               \usebox0
%             \fi
%           \else
%             \lwbox
%           \fi
%         \else
%           \usebox0
%         \fi
%       \else
%         \lwbox
%       \fi
%     \else
%       \usebox0
%     \fi
%   \else
%     \lwbox
%   \fi
% \else
%   \usebox0
% \fi
% \end{quote}
% If you have a \xfile{docstrip.cfg} that configures and enables \docstrip's
% TDS installing feature, then some files can already be in the right
% place, see the documentation of \docstrip.
%
% \subsection{Refresh file name databases}
%
% If your \TeX~distribution
% (\teTeX, \mikTeX, \dots) relies on file name databases, you must refresh
% these. For example, \teTeX\ users run \verb|texhash| or
% \verb|mktexlsr|.
%
% \subsection{Some details for the interested}
%
% \paragraph{Attached source.}
%
% The PDF documentation on CTAN also includes the
% \xfile{.dtx} source file. It can be extracted by
% AcrobatReader 6 or higher. Another option is \textsf{pdftk},
% e.g. unpack the file into the current directory:
% \begin{quote}
%   \verb|pdftk rerunfilecheck.pdf unpack_files output .|
% \end{quote}
%
% \paragraph{Unpacking with \LaTeX.}
% The \xfile{.dtx} chooses its action depending on the format:
% \begin{description}
% \item[\plainTeX:] Run \docstrip\ and extract the files.
% \item[\LaTeX:] Generate the documentation.
% \end{description}
% If you insist on using \LaTeX\ for \docstrip\ (really,
% \docstrip\ does not need \LaTeX), then inform the autodetect routine
% about your intention:
% \begin{quote}
%   \verb|latex \let\install=y% \iffalse meta-comment
%
% File: rerunfilecheck.dtx
% Version: 2016/05/16 v1.8
% Info: Rerun checks for auxiliary files
%
% Copyright (C) 2009-2011 by
%    Heiko Oberdiek <heiko.oberdiek at googlemail.com>
%    2016
%    https://github.com/ho-tex/oberdiek/issues
%
% This work may be distributed and/or modified under the
% conditions of the LaTeX Project Public License, either
% version 1.3c of this license or (at your option) any later
% version. This version of this license is in
%    http://www.latex-project.org/lppl/lppl-1-3c.txt
% and the latest version of this license is in
%    http://www.latex-project.org/lppl.txt
% and version 1.3 or later is part of all distributions of
% LaTeX version 2005/12/01 or later.
%
% This work has the LPPL maintenance status "maintained".
%
% This Current Maintainer of this work is Heiko Oberdiek.
%
% This work consists of the main source file rerunfilecheck.dtx
% and the derived files
%    rerunfilecheck.sty, rerunfilecheck.pdf, rerunfilecheck.ins,
%    rerunfilecheck.drv, rerunfilecheck-example.cfg,
%    rerunfilecheck-test1.tex.
%
% Distribution:
%    CTAN:macros/latex/contrib/oberdiek/rerunfilecheck.dtx
%    CTAN:macros/latex/contrib/oberdiek/rerunfilecheck.pdf
%
% Unpacking:
%    (a) If rerunfilecheck.ins is present:
%           tex rerunfilecheck.ins
%    (b) Without rerunfilecheck.ins:
%           tex rerunfilecheck.dtx
%    (c) If you insist on using LaTeX
%           latex \let\install=y% \iffalse meta-comment
%
% File: rerunfilecheck.dtx
% Version: 2016/05/16 v1.8
% Info: Rerun checks for auxiliary files
%
% Copyright (C) 2009-2011 by
%    Heiko Oberdiek <heiko.oberdiek at googlemail.com>
%    2016
%    https://github.com/ho-tex/oberdiek/issues
%
% This work may be distributed and/or modified under the
% conditions of the LaTeX Project Public License, either
% version 1.3c of this license or (at your option) any later
% version. This version of this license is in
%    http://www.latex-project.org/lppl/lppl-1-3c.txt
% and the latest version of this license is in
%    http://www.latex-project.org/lppl.txt
% and version 1.3 or later is part of all distributions of
% LaTeX version 2005/12/01 or later.
%
% This work has the LPPL maintenance status "maintained".
%
% This Current Maintainer of this work is Heiko Oberdiek.
%
% This work consists of the main source file rerunfilecheck.dtx
% and the derived files
%    rerunfilecheck.sty, rerunfilecheck.pdf, rerunfilecheck.ins,
%    rerunfilecheck.drv, rerunfilecheck-example.cfg,
%    rerunfilecheck-test1.tex.
%
% Distribution:
%    CTAN:macros/latex/contrib/oberdiek/rerunfilecheck.dtx
%    CTAN:macros/latex/contrib/oberdiek/rerunfilecheck.pdf
%
% Unpacking:
%    (a) If rerunfilecheck.ins is present:
%           tex rerunfilecheck.ins
%    (b) Without rerunfilecheck.ins:
%           tex rerunfilecheck.dtx
%    (c) If you insist on using LaTeX
%           latex \let\install=y\input{rerunfilecheck.dtx}
%        (quote the arguments according to the demands of your shell)
%
% Documentation:
%    (a) If rerunfilecheck.drv is present:
%           latex rerunfilecheck.drv
%    (b) Without rerunfilecheck.drv:
%           latex rerunfilecheck.dtx; ...
%    The class ltxdoc loads the configuration file ltxdoc.cfg
%    if available. Here you can specify further options, e.g.
%    use A4 as paper format:
%       \PassOptionsToClass{a4paper}{article}
%
%    Programm calls to get the documentation (example):
%       pdflatex rerunfilecheck.dtx
%       makeindex -s gind.ist rerunfilecheck.idx
%       pdflatex rerunfilecheck.dtx
%       makeindex -s gind.ist rerunfilecheck.idx
%       pdflatex rerunfilecheck.dtx
%
% Installation:
%    TDS:tex/latex/oberdiek/rerunfilecheck.sty
%    TDS:doc/latex/oberdiek/rerunfilecheck.pdf
%    TDS:doc/latex/oberdiek/rerunfilecheck-example.cfg
%    TDS:doc/latex/oberdiek/test/rerunfilecheck-test1.tex
%    TDS:source/latex/oberdiek/rerunfilecheck.dtx
%
%<*ignore>
\begingroup
  \catcode123=1 %
  \catcode125=2 %
  \def\x{LaTeX2e}%
\expandafter\endgroup
\ifcase 0\ifx\install y1\fi\expandafter
         \ifx\csname processbatchFile\endcsname\relax\else1\fi
         \ifx\fmtname\x\else 1\fi\relax
\else\csname fi\endcsname
%</ignore>
%<*install>
\input docstrip.tex
\Msg{************************************************************************}
\Msg{* Installation}
\Msg{* Package: rerunfilecheck 2016/05/16 v1.8 Rerun checks for auxiliary files (HO)}
\Msg{************************************************************************}

\keepsilent
\askforoverwritefalse

\let\MetaPrefix\relax
\preamble

This is a generated file.

Project: rerunfilecheck
Version: 2016/05/16 v1.8

Copyright (C) 2009-2011 by
   Heiko Oberdiek <heiko.oberdiek at googlemail.com>

This work may be distributed and/or modified under the
conditions of the LaTeX Project Public License, either
version 1.3c of this license or (at your option) any later
version. This version of this license is in
   http://www.latex-project.org/lppl/lppl-1-3c.txt
and the latest version of this license is in
   http://www.latex-project.org/lppl.txt
and version 1.3 or later is part of all distributions of
LaTeX version 2005/12/01 or later.

This work has the LPPL maintenance status "maintained".

This Current Maintainer of this work is Heiko Oberdiek.

This work consists of the main source file rerunfilecheck.dtx
and the derived files
   rerunfilecheck.sty, rerunfilecheck.pdf, rerunfilecheck.ins,
   rerunfilecheck.drv, rerunfilecheck-example.cfg,
   rerunfilecheck-test1.tex.

\endpreamble
\let\MetaPrefix\DoubleperCent

\generate{%
  \file{rerunfilecheck.ins}{\from{rerunfilecheck.dtx}{install}}%
  \file{rerunfilecheck.drv}{\from{rerunfilecheck.dtx}{driver}}%
  \usedir{tex/latex/oberdiek}%
  \file{rerunfilecheck.sty}{\from{rerunfilecheck.dtx}{package}}%
%  \usedir{doc/latex/oberdiek/test}%
%  \file{rerunfilecheck-test1.tex}{\from{rerunfilecheck.dtx}{test1}}%
  \nopreamble
  \nopostamble
  \usedir{doc/latex/oberdiek}%
  \file{rerunfilecheck-example.cfg}{\from{rerunfilecheck.dtx}{cfg}}%
  \usedir{source/latex/oberdiek/catalogue}%
  \file{rerunfilecheck.xml}{\from{rerunfilecheck.dtx}{catalogue}}%
}

\catcode32=13\relax% active space
\let =\space%
\Msg{************************************************************************}
\Msg{*}
\Msg{* To finish the installation you have to move the following}
\Msg{* file into a directory searched by TeX:}
\Msg{*}
\Msg{*     rerunfilecheck.sty}
\Msg{*}
\Msg{* To produce the documentation run the file `rerunfilecheck.drv'}
\Msg{* through LaTeX.}
\Msg{*}
\Msg{* Happy TeXing!}
\Msg{*}
\Msg{************************************************************************}

\endbatchfile
%</install>
%<*ignore>
\fi
%</ignore>
%<*driver>
\NeedsTeXFormat{LaTeX2e}
\ProvidesFile{rerunfilecheck.drv}%
  [2016/05/16 v1.8 Rerun checks for auxiliary files (HO)]%
\documentclass{ltxdoc}
\usepackage{holtxdoc}[2011/11/22]
\begin{document}
  \DocInput{rerunfilecheck.dtx}%
\end{document}
%</driver>
% \fi
%
%
% \CharacterTable
%  {Upper-case    \A\B\C\D\E\F\G\H\I\J\K\L\M\N\O\P\Q\R\S\T\U\V\W\X\Y\Z
%   Lower-case    \a\b\c\d\e\f\g\h\i\j\k\l\m\n\o\p\q\r\s\t\u\v\w\x\y\z
%   Digits        \0\1\2\3\4\5\6\7\8\9
%   Exclamation   \!     Double quote  \"     Hash (number) \#
%   Dollar        \$     Percent       \%     Ampersand     \&
%   Acute accent  \'     Left paren    \(     Right paren   \)
%   Asterisk      \*     Plus          \+     Comma         \,
%   Minus         \-     Point         \.     Solidus       \/
%   Colon         \:     Semicolon     \;     Less than     \<
%   Equals        \=     Greater than  \>     Question mark \?
%   Commercial at \@     Left bracket  \[     Backslash     \\
%   Right bracket \]     Circumflex    \^     Underscore    \_
%   Grave accent  \`     Left brace    \{     Vertical bar  \|
%   Right brace   \}     Tilde         \~}
%
% \GetFileInfo{rerunfilecheck.drv}
%
% \title{The \xpackage{rerunfilecheck} package}
% \date{2016/05/16 v1.8}
% \author{Heiko Oberdiek\thanks
% {Please report any issues at https://github.com/ho-tex/oberdiek/issues}\\
% \xemail{heiko.oberdiek at googlemail.com}}
%
% \maketitle
%
% \begin{abstract}
% The package provides additional rerun warnings if some
% auxiliary files have changed. It is based on MD5 checksum,
% provided by \pdfTeX.
% \end{abstract}
%
% \tableofcontents
%
% \section{Documentation}
%
% \LaTeX\ informs the user, when to run \LaTeX\ again, if the
% references have changed. It has the old references from the
% first reading of the \xext{aux} files already in memory, thus
% it compares them with the new version of the \xext{aux} file
% at the end of the document. However this rerun warnings are
% not given for the table of contents and other data stored
% in the \xext{aux} files or other auxiliary files.
% Usually many of these data as the table of contents is not keep
% in memory. If someone wants to detect changes, he has either
% to keep the data in memory. This does not scale well with huge
% documents. Or he copies the file before they are changed. Slow
% I/O operations cost time.
%
% Since version 1.30.0 \pdfTeX\ provides \cs{pdfmdfivesum}
% and \cs{pdffilesize}.
% These features are also available in \LuaTeX, provided by package
% \xpackage{pdftexcmds}.
% Thus this package \xpackage{rerunfilecheck} uses these features to
% detect file changes. This saves the packages from keeping
% the whole files in memory or in file copies.
% The drawback are different files with the same size and the
% same MD5 checksum (seldom, hopefully).
%
% \subsection{Options}
%
% All options are key value options of boolean type.
% No option or |true| turns an option on, |false| disables
% an option.
% \begin{description}
% \item[\xoption{mainaux}:]
%   Check the main \xext{aux} file.
% \item[\xoption{partaux}:]
%   Check the \xext{aux} files from \cs{include} files.
% \item[\xoption{starttoc}:]
%   Add the rerun checks in \cs{@starttoc} that is called by
%   \cs{tableofcontents}, \cs{listoffigures}, \dots
% \item[\xoption{index}, \xoption{glossary}:]
%   \LaTeX's original \cs{makeindex} and \cs{makeglossary} are
%   redefined to add the rerun checks. The options do not have
%   an effect, if \cs{makeindex}/\cs{makeglossary} are already
%   called or if a package or class had redefined or will
%   redefine them.
% \item[\xoption{aux}:]
%   This option turns all previous options on or off.
%   ``aux'' means auxiliary file.
% \end{description}
% The default for the options is |false|, because
% some internals must be redefined to insert the
% rerun checks.
% The options can be set in \cs{usepackage} or
% the configuration file \xfile{rerunfilecheck.cfg}.
% Global options are ignored (since 1.4).
%
% \begin{declcs}{RerunFileCheckSetup} \M{key value list}
% \end{declcs}
% Options can also be set using \cs{RerunFileCheckSetup}.
% Currently all options are disabled after the package
% is loaded. Thus \cs{RerunFileCheckSetup} makes sense in
% the configuration file only.
%
% Example for the configuration file:
%    \begin{macrocode}
%<*cfg>
\ProvidesFile{rerunfilecheck.cfg}[2016/05/16 Default configuration]%
\RerunFileCheckSetup{aux}
%</cfg>
%    \end{macrocode}
%
% \subsection{Interface for class/package authors}
%
% \begin{declcs}{RerunFileCheck} \M{file} \M{file closing action} \M{rerun warning}
% \end{declcs}
% If you want to add a rerun check, call \cs{RerunFileCheck} right before
% an output file is opened for writing. The macro first remembers
% the current checksum of \meta{file}.
% The file is checked again right before the end of the job.
% Macro \cs{AtVeryEndDocument} of package \xpackage{atveryend}
% is used to place the check after the main aux file is closed
% in |\end{document}|. Before reading the file again, it must be
% closed. Provide the code for closing in argument \meta{file closing action}.
% Do not forget \cs{immediate} before \cs{openout}. Otherwise the
% closing action would be delayed to the next shipout that never happens
% (the last page is already shipped out).
% If the file has changed, \cs{RerunFileCheck} informs the user with
% a warning that the file has changed and says the magic word ``Rerun''.
% If the last argument \meta{rerun warning} is not empty, then
% the rerun sentence is replaced by it. Usually the phrase
% ``to get something right'' is added.
% As example the relevant part of the redefined \cs{makeindex}
% is shown, see package code:
%\begin{quote}
%\begin{verbatim}
%\newwrite\@indexfile
%\RerunFileCheck{\jobname.idx}{%
%  \immediate\closeout\@indexfile
%}{%
%  Rerun LaTeX/makeindex to get index right%
%}%
%\immediate\openout\@indexfile=\jobname.idx %
%\end{verbatim}
%\end{quote}
%
% \StopEventually{
% }
%
% \section{Implementation}
%    \begin{macrocode}
%<*package>
%    \end{macrocode}
%
%    \begin{macrocode}
\begingroup\catcode61\catcode48\catcode32=10\relax%
  \catcode13=5 % ^^M
  \endlinechar=13 %
  \catcode123=1 % {
  \catcode125=2 % }
  \catcode64=11 % @
  \def\x{\endgroup
    \expandafter\edef\csname ReFiCh@AtEnd\endcsname{%
      \endlinechar=\the\endlinechar\relax
      \catcode13=\the\catcode13\relax
      \catcode32=\the\catcode32\relax
      \catcode35=\the\catcode35\relax
      \catcode61=\the\catcode61\relax
      \catcode64=\the\catcode64\relax
      \catcode123=\the\catcode123\relax
      \catcode125=\the\catcode125\relax
    }%
  }%
\x\catcode61\catcode48\catcode32=10\relax%
\catcode13=5 % ^^M
\endlinechar=13 %
\catcode35=6 % #
\catcode64=11 % @
\catcode123=1 % {
\catcode125=2 % }
\def\TMP@EnsureCode#1#2{%
  \edef\ReFiCh@AtEnd{%
    \ReFiCh@AtEnd
    \catcode#1=\the\catcode#1\relax
  }%
  \catcode#1=#2\relax
}
\TMP@EnsureCode{39}{12}% '
\TMP@EnsureCode{40}{12}% (
\TMP@EnsureCode{41}{12}% )
\TMP@EnsureCode{42}{12}% *
\TMP@EnsureCode{44}{12}% ,
\TMP@EnsureCode{46}{12}% .
\TMP@EnsureCode{47}{12}% /
\TMP@EnsureCode{58}{12}% :
\TMP@EnsureCode{59}{12}% ;
\TMP@EnsureCode{60}{12}% <
\TMP@EnsureCode{62}{12}% >
\TMP@EnsureCode{91}{12}% [
\TMP@EnsureCode{93}{12}% ]
\TMP@EnsureCode{96}{12}% `
\edef\ReFiCh@AtEnd{\ReFiCh@AtEnd\noexpand\endinput}
%    \end{macrocode}
%
%    Package identification.
%    \begin{macrocode}
\NeedsTeXFormat{LaTeX2e}
\ProvidesPackage{rerunfilecheck}%
  [2016/05/16 v1.8 Rerun checks for auxiliary files (HO)]
%    \end{macrocode}
%
% \subsection{Options}
%
%    \begin{macrocode}
\RequirePackage{kvoptions}[2010/02/22]
\SetupKeyvalOptions{%
  family=rerunfilecheck,%
  prefix=ReFiCh@%
}
%    \end{macrocode}
%    \begin{macro}{\RerunFileCheckSetup}
%    \begin{macrocode}
\newcommand*{\RerunFileCheckSetup}{%
  \setkeys{rerunfilecheck}%
}
%    \end{macrocode}
%    \end{macro}
%    \begin{macrocode}
\DeclareBoolOption{mainaux}
\DeclareBoolOption{partaux}
\DeclareBoolOption{starttoc}
\DeclareBoolOption{index}
\DeclareBoolOption{glossary}
\define@key{rerunfilecheck}{aux}[true]{%
  \RerunFileCheckSetup{%
    mainaux={#1},%
    partaux={#1},%
    starttoc={#1},%
    index={#1},%
    glossary={#1}%
  }%
}
%    \end{macrocode}
%    \begin{macrocode}
\InputIfFileExists{rerunfilecheck.cfg}{}{}
\ProcessLocalKeyvalOptions*
%    \end{macrocode}
%
%    \begin{macro}{\ReFiCh@DisableOption}
%    \begin{macrocode}
\def\ReFiCh@DisableOption{%
  \DisableKeyvalOption[%
    action=warning,%
    package=rerunfilecheck%
  ]{rerunfilecheck}%
}
%    \end{macrocode}
%    \end{macro}
%
% \subsection{Check for checksum feature}
%
%    \begin{macrocode}
\RequirePackage{infwarerr}[2007/09/09]
\RequirePackage{pdftexcmds}[2009/04/10]
%    \end{macrocode}
%    \begin{macrocode}
\begingroup\expandafter\expandafter\expandafter\endgroup
\expandafter\ifx\csname pdf@filemdfivesum\endcsname\relax
  \@PackageInfoNoLine{rerunfilecheck}{%
    Feature \string\pdfmdfivesum\space is not available\MessageBreak
    (e.g. pdfTeX or LuaTeX with package `pdftexcmds').\MessageBreak
    Therefore file contents cannot be checked efficiently\MessageBreak
    and the loading of the package is aborted%
  }%
  \newcommand*{\RerunFileCheck}[3]{}%
  \renewcommand*{\RerunFileCheckSetup}[1]{}%
  \expandafter\ReFiCh@AtEnd
\fi%
%    \end{macrocode}
%
% \subsection{Standard \xext{aux} files}
%
%    \begin{macrocode}
\ifReFiCh@partaux
  \let\ReFiCh@org@include\@include
  \def\@include#1 {%
    \if@filesw
      \RerunFileCheck{#1.aux}{}{}%
    \fi
    \ReFiCh@org@include{#1} %
  }%
\fi
\ifReFiCh@mainaux
  \AtBeginDocument{%
    \ReFiCh@mainauxfalse
  }%
  \ifReFiCh@mainaux
    \AtEndOfPackage{%
      \RerunFileCheck{\jobname.aux}{}{}%
    }%
  \else
    \if@filesw
      \@PackageWarningNoLine{rerunfilecheck}{%
        Main aux file check is disabled,\MessageBreak
        because the file is already opened.\MessageBreak
        Load the package before \string\begin{document}%
      }%
    \fi
  \fi
\fi
\ifReFiCh@starttoc
  \let\ReFiCh@org@starttoc\@starttoc
  \def\@starttoc#1{%
    \if@filesw
      \RerunFileCheck{\jobname.#1}{%
        \@ifundefined{tf@#1}{%
        }{%
          \immediate\closeout\csname tf@#1\endcsname
        }%
      }{}%
    \fi
    \ReFiCh@org@starttoc{#1}%
  }%
\fi
\ifReFiCh@index
  \ifx\makeindex\@empty
    \@PackageWarningNoLine{rerunfilecheck}{%
      Option `index' ignored,\MessageBreak
      because \string\makeindex\space has already been called%
    }%
  \else
    \def\ReFiCh@temp{%
      \newwrite\@indexfile
      \immediate\openout\@indexfile=\jobname.idx %
      \def\index{%
        \@bsphack
        \begingroup
        \@sanitize
        \@wrindex
      }%
      \typeout{Writing index file \jobname.idx}%
      \let\makeindex\@empty
    }%
    \ifx\ReFiCh@temp\makeindex
      \def\makeindex{%
        \newwrite\@indexfile
        \RerunFileCheck{\jobname.idx}{%
          \immediate\closeout\@indexfile
        }{%
          Rerun LaTeX/makeindex to get index right%
        }%
        \immediate\openout\@indexfile=\jobname.idx %
        \def\index{%
          \@bsphack
          \begingroup
          \@sanitize
          \@wrindex
        }%
        \typeout{Writing index file \jobname.idx}%
        \let\makeindex\@empty
      }%
    \else
      \@PackageInfoNoLine{rerunfilecheck}{%
        Option `index': unsupported version of \string\makeindex
      }%
    \fi
  \fi
\fi
\ifReFiCh@glossary
  \ifx\makeglossary\@empty
    \@PackageWarningNoLine{rerunfilecheck}{%
      Option `glossary' ignored,\MessageBreak
      because \string\makeglossary\space has already been called%
    }%
  \else
    \def\ReFiCh@temp{%
      \newwrite\@glossaryfile
      \immediate\openout\@glossaryfile=\jobname.glo %
      \def\glossary{%
        \@bsphack
        \begingroup
        \@sanitize
        \@wrglossary
      }%
      \typeout{Writing glossary file \jobname.glo }%
      \let\makeglossary\@empty
    }%
    \ifx\ReFiCh@temp\makeglossary
      \def\ReFiCh@temp{%
        \newwrite\@glossaryfile
        \RerunFileCheck{\jobname.glo}{%
          \immediate\closeout\@glossaryfile
        }{%
          Rerun LaTeX/makeindex to get glossary right%
        }%
        \immediate\openout\@glossaryfile=\jobname.glo %
        \def\glossary{%
          \@bsphack
          \begingroup
          \@sanitize
          \@wrglossary
        }%
        \typeout{Writing glossary file \jobname.glo}%
        \let\makeglossary\@empty
      }%
    \else
      \@PackageInfoNoLine{rerunfilecheck}{%
        Option `glossary': unsupported version of \string\makeglossary
      }%
    \fi
  \fi
\fi
\ReFiCh@DisableOption{mainaux}
\ReFiCh@DisableOption{partaux}
\ReFiCh@DisableOption{starttoc}
\ReFiCh@DisableOption{index}
\ReFiCh@DisableOption{glossary}
\ReFiCh@DisableOption{aux}
%    \end{macrocode}
%
% \subsection{Rerun check}
%
%    \begin{macrocode}
\RequirePackage{atveryend}[2016/05/16]
\RequirePackage{uniquecounter}[2009/12/18]
%    \end{macrocode}
%
%    \begin{macro}{\ReFiCh@CheckSum}
%    \begin{macrocode}
\begingroup\expandafter\expandafter\expandafter\endgroup
\expandafter\ifx\csname pdf@filesize\endcsname\relax
  \def\ReFiCh@CheckSum{%
    \pdf@filemdfivesum
  }%
\else
  \def\ReFiCh@CheckSum#1{%
    \pdf@filemdfivesum{#1}%
    \ReFiCh@Separator
    \pdf@filesize{#1}%
  }%
\fi
%    \end{macrocode}
%    \end{macro}
%    \begin{macro}{\ReFiCh@NoFile}
%    \begin{macrocode}
\def\ReFiCh@Separator{;}
%    \end{macrocode}
%    \end{macro}
%    \begin{macro}{\ReFiCh@NoFile}
%    \begin{macrocode}
\def\ReFiCh@NoFile{<no file>}
%    \end{macrocode}
%    \end{macro}
%
%    \begin{macrocode}
\UniqueCounterNew{rerunfilecheck}
%    \end{macrocode}
%
%    \begin{macro}{\RerunFileCheck}
%    \begin{macrocode}
\newcommand*{\RerunFileCheck}{%
  \UniqueCounterCall{rerunfilecheck}\ReFiCh@RerunFileCheck
}
%    \end{macrocode}
%    \end{macro}
%    \begin{macro}{\ReFiCh@RerunFileCheck}
%    \begin{macrocode}
\def\ReFiCh@RerunFileCheck#1{%
  \expandafter\ReFiCh@@RerunFileCheck\csname ReFiCh@#1\endcsname
}
%    \end{macrocode}
%    \end{macro}
%    \begin{macro}{\ReFiCh@Check}
%    \begin{macrocode}
\def\ReFiCh@Check#1#2#3{%
%  \IfFileExists{#3}{%
    #1\edef#2{\ReFiCh@CheckSum{#3}}%
    \ifx#2\ReFiCh@Separator
      #1\let#2\ReFiCh@NoFile
    \fi
%  }{%
%    #1\let#2\ReFiCh@NoFile
%  }%
}
%    \end{macrocode}
%    \end{macro}
%    \begin{macro}{\ReFiCh@@RerunFileCheck}
%    \begin{macrocode}
\def\ReFiCh@@RerunFileCheck#1#2#3#4{%
  \ReFiCh@Check\global#1{#2}%
  \AtEndAfterFileList{%
    \begingroup
      #3%
      \ReFiCh@Check{}\x{#2}%
      \ifx#1\x
        \@PackageInfoNoLine{rerunfilecheck}{%
          File `#2' has not changed.\MessageBreak
          Checksum: \x
        }%
      \else
        \ifnum
          \ReFiCh@IsAux#2\relax.aux\relax\@nil
          \ifx#1\ReFiCh@NoFile 1\else 0\fi
          \ifx\x\ReFiCh@AuxEmptyUnix 1%
          \else
            \ifx\x\ReFiCh@AuxEmptyDos 1\fi
          \fi
        =111 %
          \@PackageInfoNoLine{rerunfilecheck}{%
            File `#2' is empty .aux file.\MessageBreak
            Before: #1\MessageBreak
            After: \space\x
          }%
        \else
          \@PackageWarningNoLine{rerunfilecheck}{%
            File `#2' has changed.%
            \ifx\\#4\\%
              \space Rerun%
            \else
              \MessageBreak
              #4%
            \fi
          }%
          \@PackageInfoNoLine{rerunfilecheck}{%
            Checksums for `#2':\MessageBreak
            Before: #1\MessageBreak
            After: \space\x
          }%
        \fi
      \fi
    \endgroup
  }%
}
%    \end{macrocode}
%    \end{macro}
%    \begin{macrocode}
\def\ReFiCh@IsAux#1.aux\relax#2\@nil{%
  \ifx\hbox#2\hbox
    0%
  \else
    1%
  \fi
}
%    \end{macrocode}
%    \begin{macrocode}
\def\ReFiCh@AuxEmptyUnix{A94A2480D3289E625EEA47CD1B285758;8}%
\@onelevel@sanitize\ReFiCh@AuxEmptyUnix
%    \end{macrocode}
%    \begin{macrocode}
\def\ReFiCh@AuxEmptyDos{A62A15ECE803E2EBE94952FCC9933BC0;9}%
\@onelevel@sanitize\ReFiCh@AuxEmptyDos
%    \end{macrocode}
%
%    \begin{macrocode}
\ReFiCh@AtEnd%
%</package>
%    \end{macrocode}
%
% \section{Test}
%
%    \begin{macrocode}
%<*test1>
\def\LoadCommand{\RequirePackage{rerunfilecheck}[2016/05/16]}
%</test1>
%    \end{macrocode}
% \subsection{Catcode checks for loading}
%
%    \begin{macrocode}
%<*test1>
%    \end{macrocode}
%    \begin{macrocode}
\catcode`\{=1 %
\catcode`\}=2 %
\catcode`\#=6 %
\catcode`\@=11 %
\expandafter\ifx\csname count@\endcsname\relax
  \countdef\count@=255 %
\fi
\expandafter\ifx\csname @gobble\endcsname\relax
  \long\def\@gobble#1{}%
\fi
\expandafter\ifx\csname @firstofone\endcsname\relax
  \long\def\@firstofone#1{#1}%
\fi
\expandafter\ifx\csname loop\endcsname\relax
  \expandafter\@firstofone
\else
  \expandafter\@gobble
\fi
{%
  \def\loop#1\repeat{%
    \def\body{#1}%
    \iterate
  }%
  \def\iterate{%
    \body
      \let\next\iterate
    \else
      \let\next\relax
    \fi
    \next
  }%
  \let\repeat=\fi
}%
\def\RestoreCatcodes{}
\count@=0 %
\loop
  \edef\RestoreCatcodes{%
    \RestoreCatcodes
    \catcode\the\count@=\the\catcode\count@\relax
  }%
\ifnum\count@<255 %
  \advance\count@ 1 %
\repeat

\def\RangeCatcodeInvalid#1#2{%
  \count@=#1\relax
  \loop
    \catcode\count@=15 %
  \ifnum\count@<#2\relax
    \advance\count@ 1 %
  \repeat
}
\def\RangeCatcodeCheck#1#2#3{%
  \count@=#1\relax
  \loop
    \ifnum#3=\catcode\count@
    \else
      \errmessage{%
        Character \the\count@\space
        with wrong catcode \the\catcode\count@\space
        instead of \number#3%
      }%
    \fi
  \ifnum\count@<#2\relax
    \advance\count@ 1 %
  \repeat
}
\def\space{ }
\expandafter\ifx\csname LoadCommand\endcsname\relax
  \def\LoadCommand{\input rerunfilecheck.sty\relax}%
\fi
\def\Test{%
  \RangeCatcodeInvalid{0}{47}%
  \RangeCatcodeInvalid{58}{64}%
  \RangeCatcodeInvalid{91}{96}%
  \RangeCatcodeInvalid{123}{255}%
  \catcode`\@=12 %
  \catcode`\\=0 %
  \catcode`\%=14 %
  \LoadCommand
  \RangeCatcodeCheck{0}{36}{15}%
  \RangeCatcodeCheck{37}{37}{14}%
  \RangeCatcodeCheck{38}{47}{15}%
  \RangeCatcodeCheck{48}{57}{12}%
  \RangeCatcodeCheck{58}{63}{15}%
  \RangeCatcodeCheck{64}{64}{12}%
  \RangeCatcodeCheck{65}{90}{11}%
  \RangeCatcodeCheck{91}{91}{15}%
  \RangeCatcodeCheck{92}{92}{0}%
  \RangeCatcodeCheck{93}{96}{15}%
  \RangeCatcodeCheck{97}{122}{11}%
  \RangeCatcodeCheck{123}{255}{15}%
  \RestoreCatcodes
}
\Test
\csname @@end\endcsname
\end
%    \end{macrocode}
%    \begin{macrocode}
%</test1>
%    \end{macrocode}
%
% \section{Installation}
%
% \subsection{Download}
%
% \paragraph{Package.} This package is available on
% CTAN\footnote{\url{http://ctan.org/pkg/rerunfilecheck}}:
% \begin{description}
% \item[\CTAN{macros/latex/contrib/oberdiek/rerunfilecheck.dtx}] The source file.
% \item[\CTAN{macros/latex/contrib/oberdiek/rerunfilecheck.pdf}] Documentation.
% \end{description}
%
%
% \paragraph{Bundle.} All the packages of the bundle `oberdiek'
% are also available in a TDS compliant ZIP archive. There
% the packages are already unpacked and the documentation files
% are generated. The files and directories obey the TDS standard.
% \begin{description}
% \item[\CTAN{install/macros/latex/contrib/oberdiek.tds.zip}]
% \end{description}
% \emph{TDS} refers to the standard ``A Directory Structure
% for \TeX\ Files'' (\CTAN{tds/tds.pdf}). Directories
% with \xfile{texmf} in their name are usually organized this way.
%
% \subsection{Bundle installation}
%
% \paragraph{Unpacking.} Unpack the \xfile{oberdiek.tds.zip} in the
% TDS tree (also known as \xfile{texmf} tree) of your choice.
% Example (linux):
% \begin{quote}
%   |unzip oberdiek.tds.zip -d ~/texmf|
% \end{quote}
%
% \paragraph{Script installation.}
% Check the directory \xfile{TDS:scripts/oberdiek/} for
% scripts that need further installation steps.
% Package \xpackage{attachfile2} comes with the Perl script
% \xfile{pdfatfi.pl} that should be installed in such a way
% that it can be called as \texttt{pdfatfi}.
% Example (linux):
% \begin{quote}
%   |chmod +x scripts/oberdiek/pdfatfi.pl|\\
%   |cp scripts/oberdiek/pdfatfi.pl /usr/local/bin/|
% \end{quote}
%
% \subsection{Package installation}
%
% \paragraph{Unpacking.} The \xfile{.dtx} file is a self-extracting
% \docstrip\ archive. The files are extracted by running the
% \xfile{.dtx} through \plainTeX:
% \begin{quote}
%   \verb|tex rerunfilecheck.dtx|
% \end{quote}
%
% \paragraph{TDS.} Now the different files must be moved into
% the different directories in your installation TDS tree
% (also known as \xfile{texmf} tree):
% \begin{quote}
% \def\t{^^A
% \begin{tabular}{@{}>{\ttfamily}l@{ $\rightarrow$ }>{\ttfamily}l@{}}
%   rerunfilecheck.sty & tex/latex/oberdiek/rerunfilecheck.sty\\
%   rerunfilecheck.pdf & doc/latex/oberdiek/rerunfilecheck.pdf\\
%   rerunfilecheck-example.cfg & doc/latex/oberdiek/rerunfilecheck-example.cfg\\
%   test/rerunfilecheck-test1.tex & doc/latex/oberdiek/test/rerunfilecheck-test1.tex\\
%   rerunfilecheck.dtx & source/latex/oberdiek/rerunfilecheck.dtx\\
% \end{tabular}^^A
% }^^A
% \sbox0{\t}^^A
% \ifdim\wd0>\linewidth
%   \begingroup
%     \advance\linewidth by\leftmargin
%     \advance\linewidth by\rightmargin
%   \edef\x{\endgroup
%     \def\noexpand\lw{\the\linewidth}^^A
%   }\x
%   \def\lwbox{^^A
%     \leavevmode
%     \hbox to \linewidth{^^A
%       \kern-\leftmargin\relax
%       \hss
%       \usebox0
%       \hss
%       \kern-\rightmargin\relax
%     }^^A
%   }^^A
%   \ifdim\wd0>\lw
%     \sbox0{\small\t}^^A
%     \ifdim\wd0>\linewidth
%       \ifdim\wd0>\lw
%         \sbox0{\footnotesize\t}^^A
%         \ifdim\wd0>\linewidth
%           \ifdim\wd0>\lw
%             \sbox0{\scriptsize\t}^^A
%             \ifdim\wd0>\linewidth
%               \ifdim\wd0>\lw
%                 \sbox0{\tiny\t}^^A
%                 \ifdim\wd0>\linewidth
%                   \lwbox
%                 \else
%                   \usebox0
%                 \fi
%               \else
%                 \lwbox
%               \fi
%             \else
%               \usebox0
%             \fi
%           \else
%             \lwbox
%           \fi
%         \else
%           \usebox0
%         \fi
%       \else
%         \lwbox
%       \fi
%     \else
%       \usebox0
%     \fi
%   \else
%     \lwbox
%   \fi
% \else
%   \usebox0
% \fi
% \end{quote}
% If you have a \xfile{docstrip.cfg} that configures and enables \docstrip's
% TDS installing feature, then some files can already be in the right
% place, see the documentation of \docstrip.
%
% \subsection{Refresh file name databases}
%
% If your \TeX~distribution
% (\teTeX, \mikTeX, \dots) relies on file name databases, you must refresh
% these. For example, \teTeX\ users run \verb|texhash| or
% \verb|mktexlsr|.
%
% \subsection{Some details for the interested}
%
% \paragraph{Attached source.}
%
% The PDF documentation on CTAN also includes the
% \xfile{.dtx} source file. It can be extracted by
% AcrobatReader 6 or higher. Another option is \textsf{pdftk},
% e.g. unpack the file into the current directory:
% \begin{quote}
%   \verb|pdftk rerunfilecheck.pdf unpack_files output .|
% \end{quote}
%
% \paragraph{Unpacking with \LaTeX.}
% The \xfile{.dtx} chooses its action depending on the format:
% \begin{description}
% \item[\plainTeX:] Run \docstrip\ and extract the files.
% \item[\LaTeX:] Generate the documentation.
% \end{description}
% If you insist on using \LaTeX\ for \docstrip\ (really,
% \docstrip\ does not need \LaTeX), then inform the autodetect routine
% about your intention:
% \begin{quote}
%   \verb|latex \let\install=y\input{rerunfilecheck.dtx}|
% \end{quote}
% Do not forget to quote the argument according to the demands
% of your shell.
%
% \paragraph{Generating the documentation.}
% You can use both the \xfile{.dtx} or the \xfile{.drv} to generate
% the documentation. The process can be configured by the
% configuration file \xfile{ltxdoc.cfg}. For instance, put this
% line into this file, if you want to have A4 as paper format:
% \begin{quote}
%   \verb|\PassOptionsToClass{a4paper}{article}|
% \end{quote}
% An example follows how to generate the
% documentation with pdf\LaTeX:
% \begin{quote}
%\begin{verbatim}
%pdflatex rerunfilecheck.dtx
%makeindex -s gind.ist rerunfilecheck.idx
%pdflatex rerunfilecheck.dtx
%makeindex -s gind.ist rerunfilecheck.idx
%pdflatex rerunfilecheck.dtx
%\end{verbatim}
% \end{quote}
%
% \section{Catalogue}
%
% The following XML file can be used as source for the
% \href{http://mirror.ctan.org/help/Catalogue/catalogue.html}{\TeX\ Catalogue}.
% The elements \texttt{caption} and \texttt{description} are imported
% from the original XML file from the Catalogue.
% The name of the XML file in the Catalogue is \xfile{rerunfilecheck.xml}.
%    \begin{macrocode}
%<*catalogue>
<?xml version='1.0' encoding='us-ascii'?>
<!DOCTYPE entry SYSTEM 'catalogue.dtd'>
<entry datestamp='$Date$' modifier='$Author$' id='rerunfilecheck'>
  <name>rerunfilecheck</name>
  <caption>Checksum based rerun checks on auxiliary files.</caption>
  <authorref id='auth:oberdiek'/>
  <copyright owner='Heiko Oberdiek' year='2009-2011'/>
  <license type='lppl1.3'/>
  <version number='1.8'/>
  <description>
    The package provides additional rerun warnings if some
    auxiliary files have changed. It is based on MD5 checksum,
    provided by pdfTeX.
    <p/>
    The package is part of the <xref refid='oberdiek'>oberdiek</xref> bundle.
  </description>
  <documentation details='Package documentation'
      href='ctan:/macros/latex/contrib/oberdiek/rerunfilecheck.pdf'/>
  <ctan file='true' path='/macros/latex/contrib/oberdiek/rerunfilecheck.dtx'/>
  <miktex location='oberdiek'/>
  <texlive location='oberdiek'/>
  <install path='/macros/latex/contrib/oberdiek/oberdiek.tds.zip'/>
</entry>
%</catalogue>
%    \end{macrocode}
%
% \begin{History}
%   \begin{Version}{2009/12/10 v1.0}
%   \item
%     The first version.
%   \end{Version}
%   \begin{Version}{2009/12/12 v1.1}
%   \item
%     Short info shortened.
%   \end{Version}
%   \begin{Version}{2009/12/18 v1.2}
%   \item
%     Required date for package \xpackage{uniquecounter} updated
%     because of bug in this package.
%   \end{Version}
%   \begin{Version}{2010/01/25 v1.3}
%   \item
%     Moved from \texttt{TDS:*/generic/*} to \texttt{TDS:*/latex/*}.
%   \end{Version}
%   \begin{Version}{2010/02/22 v1.4}
%   \item
%     The options of this package are recognized only if they are
%     package options. Global options are ignored. This avoids
%     name clashes with class and other package options
%     (for example, class option `index=totoc').
%   \end{Version}
%   \begin{Version}{2010/03/15 v1.5}
%   \item
%     Call of \cs{pdffivesum} is wrapped in \cs{IfFileExists} to avoid
%     calls of \textsf{mktextex} if this feature is enabled.
%     However \cs{IfFileExists} has file name limitations.
%   \end{Version}
%   \begin{Version}{2010/03/16 v1.6}
%   \item
%     Reverted to version 1.4 and \cs{IfFileExists} wrapper of
%     version 1.5 is removed.
%   \end{Version}
%   \begin{Version}{2011/04/15 v1.7}
%   \item
%     Using \cs{AtEndAfterFileList} of package \xpackage{atveryend}
%     2011/04/15 v1.6 instead of \cs{AtVeryEndDocument}.
%   \end{Version}
%   \begin{Version}{2016/05/16 v1.8}
%   \item
%     Documentation updates.
%   \end{Version}
% \end{History}
%
% \PrintIndex
%
% \Finale
\endinput

%        (quote the arguments according to the demands of your shell)
%
% Documentation:
%    (a) If rerunfilecheck.drv is present:
%           latex rerunfilecheck.drv
%    (b) Without rerunfilecheck.drv:
%           latex rerunfilecheck.dtx; ...
%    The class ltxdoc loads the configuration file ltxdoc.cfg
%    if available. Here you can specify further options, e.g.
%    use A4 as paper format:
%       \PassOptionsToClass{a4paper}{article}
%
%    Programm calls to get the documentation (example):
%       pdflatex rerunfilecheck.dtx
%       makeindex -s gind.ist rerunfilecheck.idx
%       pdflatex rerunfilecheck.dtx
%       makeindex -s gind.ist rerunfilecheck.idx
%       pdflatex rerunfilecheck.dtx
%
% Installation:
%    TDS:tex/latex/oberdiek/rerunfilecheck.sty
%    TDS:doc/latex/oberdiek/rerunfilecheck.pdf
%    TDS:doc/latex/oberdiek/rerunfilecheck-example.cfg
%    TDS:doc/latex/oberdiek/test/rerunfilecheck-test1.tex
%    TDS:source/latex/oberdiek/rerunfilecheck.dtx
%
%<*ignore>
\begingroup
  \catcode123=1 %
  \catcode125=2 %
  \def\x{LaTeX2e}%
\expandafter\endgroup
\ifcase 0\ifx\install y1\fi\expandafter
         \ifx\csname processbatchFile\endcsname\relax\else1\fi
         \ifx\fmtname\x\else 1\fi\relax
\else\csname fi\endcsname
%</ignore>
%<*install>
\input docstrip.tex
\Msg{************************************************************************}
\Msg{* Installation}
\Msg{* Package: rerunfilecheck 2016/05/16 v1.8 Rerun checks for auxiliary files (HO)}
\Msg{************************************************************************}

\keepsilent
\askforoverwritefalse

\let\MetaPrefix\relax
\preamble

This is a generated file.

Project: rerunfilecheck
Version: 2016/05/16 v1.8

Copyright (C) 2009-2011 by
   Heiko Oberdiek <heiko.oberdiek at googlemail.com>

This work may be distributed and/or modified under the
conditions of the LaTeX Project Public License, either
version 1.3c of this license or (at your option) any later
version. This version of this license is in
   http://www.latex-project.org/lppl/lppl-1-3c.txt
and the latest version of this license is in
   http://www.latex-project.org/lppl.txt
and version 1.3 or later is part of all distributions of
LaTeX version 2005/12/01 or later.

This work has the LPPL maintenance status "maintained".

This Current Maintainer of this work is Heiko Oberdiek.

This work consists of the main source file rerunfilecheck.dtx
and the derived files
   rerunfilecheck.sty, rerunfilecheck.pdf, rerunfilecheck.ins,
   rerunfilecheck.drv, rerunfilecheck-example.cfg,
   rerunfilecheck-test1.tex.

\endpreamble
\let\MetaPrefix\DoubleperCent

\generate{%
  \file{rerunfilecheck.ins}{\from{rerunfilecheck.dtx}{install}}%
  \file{rerunfilecheck.drv}{\from{rerunfilecheck.dtx}{driver}}%
  \usedir{tex/latex/oberdiek}%
  \file{rerunfilecheck.sty}{\from{rerunfilecheck.dtx}{package}}%
%  \usedir{doc/latex/oberdiek/test}%
%  \file{rerunfilecheck-test1.tex}{\from{rerunfilecheck.dtx}{test1}}%
  \nopreamble
  \nopostamble
  \usedir{doc/latex/oberdiek}%
  \file{rerunfilecheck-example.cfg}{\from{rerunfilecheck.dtx}{cfg}}%
  \usedir{source/latex/oberdiek/catalogue}%
  \file{rerunfilecheck.xml}{\from{rerunfilecheck.dtx}{catalogue}}%
}

\catcode32=13\relax% active space
\let =\space%
\Msg{************************************************************************}
\Msg{*}
\Msg{* To finish the installation you have to move the following}
\Msg{* file into a directory searched by TeX:}
\Msg{*}
\Msg{*     rerunfilecheck.sty}
\Msg{*}
\Msg{* To produce the documentation run the file `rerunfilecheck.drv'}
\Msg{* through LaTeX.}
\Msg{*}
\Msg{* Happy TeXing!}
\Msg{*}
\Msg{************************************************************************}

\endbatchfile
%</install>
%<*ignore>
\fi
%</ignore>
%<*driver>
\NeedsTeXFormat{LaTeX2e}
\ProvidesFile{rerunfilecheck.drv}%
  [2016/05/16 v1.8 Rerun checks for auxiliary files (HO)]%
\documentclass{ltxdoc}
\usepackage{holtxdoc}[2011/11/22]
\begin{document}
  \DocInput{rerunfilecheck.dtx}%
\end{document}
%</driver>
% \fi
%
%
% \CharacterTable
%  {Upper-case    \A\B\C\D\E\F\G\H\I\J\K\L\M\N\O\P\Q\R\S\T\U\V\W\X\Y\Z
%   Lower-case    \a\b\c\d\e\f\g\h\i\j\k\l\m\n\o\p\q\r\s\t\u\v\w\x\y\z
%   Digits        \0\1\2\3\4\5\6\7\8\9
%   Exclamation   \!     Double quote  \"     Hash (number) \#
%   Dollar        \$     Percent       \%     Ampersand     \&
%   Acute accent  \'     Left paren    \(     Right paren   \)
%   Asterisk      \*     Plus          \+     Comma         \,
%   Minus         \-     Point         \.     Solidus       \/
%   Colon         \:     Semicolon     \;     Less than     \<
%   Equals        \=     Greater than  \>     Question mark \?
%   Commercial at \@     Left bracket  \[     Backslash     \\
%   Right bracket \]     Circumflex    \^     Underscore    \_
%   Grave accent  \`     Left brace    \{     Vertical bar  \|
%   Right brace   \}     Tilde         \~}
%
% \GetFileInfo{rerunfilecheck.drv}
%
% \title{The \xpackage{rerunfilecheck} package}
% \date{2016/05/16 v1.8}
% \author{Heiko Oberdiek\thanks
% {Please report any issues at https://github.com/ho-tex/oberdiek/issues}\\
% \xemail{heiko.oberdiek at googlemail.com}}
%
% \maketitle
%
% \begin{abstract}
% The package provides additional rerun warnings if some
% auxiliary files have changed. It is based on MD5 checksum,
% provided by \pdfTeX.
% \end{abstract}
%
% \tableofcontents
%
% \section{Documentation}
%
% \LaTeX\ informs the user, when to run \LaTeX\ again, if the
% references have changed. It has the old references from the
% first reading of the \xext{aux} files already in memory, thus
% it compares them with the new version of the \xext{aux} file
% at the end of the document. However this rerun warnings are
% not given for the table of contents and other data stored
% in the \xext{aux} files or other auxiliary files.
% Usually many of these data as the table of contents is not keep
% in memory. If someone wants to detect changes, he has either
% to keep the data in memory. This does not scale well with huge
% documents. Or he copies the file before they are changed. Slow
% I/O operations cost time.
%
% Since version 1.30.0 \pdfTeX\ provides \cs{pdfmdfivesum}
% and \cs{pdffilesize}.
% These features are also available in \LuaTeX, provided by package
% \xpackage{pdftexcmds}.
% Thus this package \xpackage{rerunfilecheck} uses these features to
% detect file changes. This saves the packages from keeping
% the whole files in memory or in file copies.
% The drawback are different files with the same size and the
% same MD5 checksum (seldom, hopefully).
%
% \subsection{Options}
%
% All options are key value options of boolean type.
% No option or |true| turns an option on, |false| disables
% an option.
% \begin{description}
% \item[\xoption{mainaux}:]
%   Check the main \xext{aux} file.
% \item[\xoption{partaux}:]
%   Check the \xext{aux} files from \cs{include} files.
% \item[\xoption{starttoc}:]
%   Add the rerun checks in \cs{@starttoc} that is called by
%   \cs{tableofcontents}, \cs{listoffigures}, \dots
% \item[\xoption{index}, \xoption{glossary}:]
%   \LaTeX's original \cs{makeindex} and \cs{makeglossary} are
%   redefined to add the rerun checks. The options do not have
%   an effect, if \cs{makeindex}/\cs{makeglossary} are already
%   called or if a package or class had redefined or will
%   redefine them.
% \item[\xoption{aux}:]
%   This option turns all previous options on or off.
%   ``aux'' means auxiliary file.
% \end{description}
% The default for the options is |false|, because
% some internals must be redefined to insert the
% rerun checks.
% The options can be set in \cs{usepackage} or
% the configuration file \xfile{rerunfilecheck.cfg}.
% Global options are ignored (since 1.4).
%
% \begin{declcs}{RerunFileCheckSetup} \M{key value list}
% \end{declcs}
% Options can also be set using \cs{RerunFileCheckSetup}.
% Currently all options are disabled after the package
% is loaded. Thus \cs{RerunFileCheckSetup} makes sense in
% the configuration file only.
%
% Example for the configuration file:
%    \begin{macrocode}
%<*cfg>
\ProvidesFile{rerunfilecheck.cfg}[2016/05/16 Default configuration]%
\RerunFileCheckSetup{aux}
%</cfg>
%    \end{macrocode}
%
% \subsection{Interface for class/package authors}
%
% \begin{declcs}{RerunFileCheck} \M{file} \M{file closing action} \M{rerun warning}
% \end{declcs}
% If you want to add a rerun check, call \cs{RerunFileCheck} right before
% an output file is opened for writing. The macro first remembers
% the current checksum of \meta{file}.
% The file is checked again right before the end of the job.
% Macro \cs{AtVeryEndDocument} of package \xpackage{atveryend}
% is used to place the check after the main aux file is closed
% in |\end{document}|. Before reading the file again, it must be
% closed. Provide the code for closing in argument \meta{file closing action}.
% Do not forget \cs{immediate} before \cs{openout}. Otherwise the
% closing action would be delayed to the next shipout that never happens
% (the last page is already shipped out).
% If the file has changed, \cs{RerunFileCheck} informs the user with
% a warning that the file has changed and says the magic word ``Rerun''.
% If the last argument \meta{rerun warning} is not empty, then
% the rerun sentence is replaced by it. Usually the phrase
% ``to get something right'' is added.
% As example the relevant part of the redefined \cs{makeindex}
% is shown, see package code:
%\begin{quote}
%\begin{verbatim}
%\newwrite\@indexfile
%\RerunFileCheck{\jobname.idx}{%
%  \immediate\closeout\@indexfile
%}{%
%  Rerun LaTeX/makeindex to get index right%
%}%
%\immediate\openout\@indexfile=\jobname.idx %
%\end{verbatim}
%\end{quote}
%
% \StopEventually{
% }
%
% \section{Implementation}
%    \begin{macrocode}
%<*package>
%    \end{macrocode}
%
%    \begin{macrocode}
\begingroup\catcode61\catcode48\catcode32=10\relax%
  \catcode13=5 % ^^M
  \endlinechar=13 %
  \catcode123=1 % {
  \catcode125=2 % }
  \catcode64=11 % @
  \def\x{\endgroup
    \expandafter\edef\csname ReFiCh@AtEnd\endcsname{%
      \endlinechar=\the\endlinechar\relax
      \catcode13=\the\catcode13\relax
      \catcode32=\the\catcode32\relax
      \catcode35=\the\catcode35\relax
      \catcode61=\the\catcode61\relax
      \catcode64=\the\catcode64\relax
      \catcode123=\the\catcode123\relax
      \catcode125=\the\catcode125\relax
    }%
  }%
\x\catcode61\catcode48\catcode32=10\relax%
\catcode13=5 % ^^M
\endlinechar=13 %
\catcode35=6 % #
\catcode64=11 % @
\catcode123=1 % {
\catcode125=2 % }
\def\TMP@EnsureCode#1#2{%
  \edef\ReFiCh@AtEnd{%
    \ReFiCh@AtEnd
    \catcode#1=\the\catcode#1\relax
  }%
  \catcode#1=#2\relax
}
\TMP@EnsureCode{39}{12}% '
\TMP@EnsureCode{40}{12}% (
\TMP@EnsureCode{41}{12}% )
\TMP@EnsureCode{42}{12}% *
\TMP@EnsureCode{44}{12}% ,
\TMP@EnsureCode{46}{12}% .
\TMP@EnsureCode{47}{12}% /
\TMP@EnsureCode{58}{12}% :
\TMP@EnsureCode{59}{12}% ;
\TMP@EnsureCode{60}{12}% <
\TMP@EnsureCode{62}{12}% >
\TMP@EnsureCode{91}{12}% [
\TMP@EnsureCode{93}{12}% ]
\TMP@EnsureCode{96}{12}% `
\edef\ReFiCh@AtEnd{\ReFiCh@AtEnd\noexpand\endinput}
%    \end{macrocode}
%
%    Package identification.
%    \begin{macrocode}
\NeedsTeXFormat{LaTeX2e}
\ProvidesPackage{rerunfilecheck}%
  [2016/05/16 v1.8 Rerun checks for auxiliary files (HO)]
%    \end{macrocode}
%
% \subsection{Options}
%
%    \begin{macrocode}
\RequirePackage{kvoptions}[2010/02/22]
\SetupKeyvalOptions{%
  family=rerunfilecheck,%
  prefix=ReFiCh@%
}
%    \end{macrocode}
%    \begin{macro}{\RerunFileCheckSetup}
%    \begin{macrocode}
\newcommand*{\RerunFileCheckSetup}{%
  \setkeys{rerunfilecheck}%
}
%    \end{macrocode}
%    \end{macro}
%    \begin{macrocode}
\DeclareBoolOption{mainaux}
\DeclareBoolOption{partaux}
\DeclareBoolOption{starttoc}
\DeclareBoolOption{index}
\DeclareBoolOption{glossary}
\define@key{rerunfilecheck}{aux}[true]{%
  \RerunFileCheckSetup{%
    mainaux={#1},%
    partaux={#1},%
    starttoc={#1},%
    index={#1},%
    glossary={#1}%
  }%
}
%    \end{macrocode}
%    \begin{macrocode}
\InputIfFileExists{rerunfilecheck.cfg}{}{}
\ProcessLocalKeyvalOptions*
%    \end{macrocode}
%
%    \begin{macro}{\ReFiCh@DisableOption}
%    \begin{macrocode}
\def\ReFiCh@DisableOption{%
  \DisableKeyvalOption[%
    action=warning,%
    package=rerunfilecheck%
  ]{rerunfilecheck}%
}
%    \end{macrocode}
%    \end{macro}
%
% \subsection{Check for checksum feature}
%
%    \begin{macrocode}
\RequirePackage{infwarerr}[2007/09/09]
\RequirePackage{pdftexcmds}[2009/04/10]
%    \end{macrocode}
%    \begin{macrocode}
\begingroup\expandafter\expandafter\expandafter\endgroup
\expandafter\ifx\csname pdf@filemdfivesum\endcsname\relax
  \@PackageInfoNoLine{rerunfilecheck}{%
    Feature \string\pdfmdfivesum\space is not available\MessageBreak
    (e.g. pdfTeX or LuaTeX with package `pdftexcmds').\MessageBreak
    Therefore file contents cannot be checked efficiently\MessageBreak
    and the loading of the package is aborted%
  }%
  \newcommand*{\RerunFileCheck}[3]{}%
  \renewcommand*{\RerunFileCheckSetup}[1]{}%
  \expandafter\ReFiCh@AtEnd
\fi%
%    \end{macrocode}
%
% \subsection{Standard \xext{aux} files}
%
%    \begin{macrocode}
\ifReFiCh@partaux
  \let\ReFiCh@org@include\@include
  \def\@include#1 {%
    \if@filesw
      \RerunFileCheck{#1.aux}{}{}%
    \fi
    \ReFiCh@org@include{#1} %
  }%
\fi
\ifReFiCh@mainaux
  \AtBeginDocument{%
    \ReFiCh@mainauxfalse
  }%
  \ifReFiCh@mainaux
    \AtEndOfPackage{%
      \RerunFileCheck{\jobname.aux}{}{}%
    }%
  \else
    \if@filesw
      \@PackageWarningNoLine{rerunfilecheck}{%
        Main aux file check is disabled,\MessageBreak
        because the file is already opened.\MessageBreak
        Load the package before \string\begin{document}%
      }%
    \fi
  \fi
\fi
\ifReFiCh@starttoc
  \let\ReFiCh@org@starttoc\@starttoc
  \def\@starttoc#1{%
    \if@filesw
      \RerunFileCheck{\jobname.#1}{%
        \@ifundefined{tf@#1}{%
        }{%
          \immediate\closeout\csname tf@#1\endcsname
        }%
      }{}%
    \fi
    \ReFiCh@org@starttoc{#1}%
  }%
\fi
\ifReFiCh@index
  \ifx\makeindex\@empty
    \@PackageWarningNoLine{rerunfilecheck}{%
      Option `index' ignored,\MessageBreak
      because \string\makeindex\space has already been called%
    }%
  \else
    \def\ReFiCh@temp{%
      \newwrite\@indexfile
      \immediate\openout\@indexfile=\jobname.idx %
      \def\index{%
        \@bsphack
        \begingroup
        \@sanitize
        \@wrindex
      }%
      \typeout{Writing index file \jobname.idx}%
      \let\makeindex\@empty
    }%
    \ifx\ReFiCh@temp\makeindex
      \def\makeindex{%
        \newwrite\@indexfile
        \RerunFileCheck{\jobname.idx}{%
          \immediate\closeout\@indexfile
        }{%
          Rerun LaTeX/makeindex to get index right%
        }%
        \immediate\openout\@indexfile=\jobname.idx %
        \def\index{%
          \@bsphack
          \begingroup
          \@sanitize
          \@wrindex
        }%
        \typeout{Writing index file \jobname.idx}%
        \let\makeindex\@empty
      }%
    \else
      \@PackageInfoNoLine{rerunfilecheck}{%
        Option `index': unsupported version of \string\makeindex
      }%
    \fi
  \fi
\fi
\ifReFiCh@glossary
  \ifx\makeglossary\@empty
    \@PackageWarningNoLine{rerunfilecheck}{%
      Option `glossary' ignored,\MessageBreak
      because \string\makeglossary\space has already been called%
    }%
  \else
    \def\ReFiCh@temp{%
      \newwrite\@glossaryfile
      \immediate\openout\@glossaryfile=\jobname.glo %
      \def\glossary{%
        \@bsphack
        \begingroup
        \@sanitize
        \@wrglossary
      }%
      \typeout{Writing glossary file \jobname.glo }%
      \let\makeglossary\@empty
    }%
    \ifx\ReFiCh@temp\makeglossary
      \def\ReFiCh@temp{%
        \newwrite\@glossaryfile
        \RerunFileCheck{\jobname.glo}{%
          \immediate\closeout\@glossaryfile
        }{%
          Rerun LaTeX/makeindex to get glossary right%
        }%
        \immediate\openout\@glossaryfile=\jobname.glo %
        \def\glossary{%
          \@bsphack
          \begingroup
          \@sanitize
          \@wrglossary
        }%
        \typeout{Writing glossary file \jobname.glo}%
        \let\makeglossary\@empty
      }%
    \else
      \@PackageInfoNoLine{rerunfilecheck}{%
        Option `glossary': unsupported version of \string\makeglossary
      }%
    \fi
  \fi
\fi
\ReFiCh@DisableOption{mainaux}
\ReFiCh@DisableOption{partaux}
\ReFiCh@DisableOption{starttoc}
\ReFiCh@DisableOption{index}
\ReFiCh@DisableOption{glossary}
\ReFiCh@DisableOption{aux}
%    \end{macrocode}
%
% \subsection{Rerun check}
%
%    \begin{macrocode}
\RequirePackage{atveryend}[2016/05/16]
\RequirePackage{uniquecounter}[2009/12/18]
%    \end{macrocode}
%
%    \begin{macro}{\ReFiCh@CheckSum}
%    \begin{macrocode}
\begingroup\expandafter\expandafter\expandafter\endgroup
\expandafter\ifx\csname pdf@filesize\endcsname\relax
  \def\ReFiCh@CheckSum{%
    \pdf@filemdfivesum
  }%
\else
  \def\ReFiCh@CheckSum#1{%
    \pdf@filemdfivesum{#1}%
    \ReFiCh@Separator
    \pdf@filesize{#1}%
  }%
\fi
%    \end{macrocode}
%    \end{macro}
%    \begin{macro}{\ReFiCh@NoFile}
%    \begin{macrocode}
\def\ReFiCh@Separator{;}
%    \end{macrocode}
%    \end{macro}
%    \begin{macro}{\ReFiCh@NoFile}
%    \begin{macrocode}
\def\ReFiCh@NoFile{<no file>}
%    \end{macrocode}
%    \end{macro}
%
%    \begin{macrocode}
\UniqueCounterNew{rerunfilecheck}
%    \end{macrocode}
%
%    \begin{macro}{\RerunFileCheck}
%    \begin{macrocode}
\newcommand*{\RerunFileCheck}{%
  \UniqueCounterCall{rerunfilecheck}\ReFiCh@RerunFileCheck
}
%    \end{macrocode}
%    \end{macro}
%    \begin{macro}{\ReFiCh@RerunFileCheck}
%    \begin{macrocode}
\def\ReFiCh@RerunFileCheck#1{%
  \expandafter\ReFiCh@@RerunFileCheck\csname ReFiCh@#1\endcsname
}
%    \end{macrocode}
%    \end{macro}
%    \begin{macro}{\ReFiCh@Check}
%    \begin{macrocode}
\def\ReFiCh@Check#1#2#3{%
%  \IfFileExists{#3}{%
    #1\edef#2{\ReFiCh@CheckSum{#3}}%
    \ifx#2\ReFiCh@Separator
      #1\let#2\ReFiCh@NoFile
    \fi
%  }{%
%    #1\let#2\ReFiCh@NoFile
%  }%
}
%    \end{macrocode}
%    \end{macro}
%    \begin{macro}{\ReFiCh@@RerunFileCheck}
%    \begin{macrocode}
\def\ReFiCh@@RerunFileCheck#1#2#3#4{%
  \ReFiCh@Check\global#1{#2}%
  \AtEndAfterFileList{%
    \begingroup
      #3%
      \ReFiCh@Check{}\x{#2}%
      \ifx#1\x
        \@PackageInfoNoLine{rerunfilecheck}{%
          File `#2' has not changed.\MessageBreak
          Checksum: \x
        }%
      \else
        \ifnum
          \ReFiCh@IsAux#2\relax.aux\relax\@nil
          \ifx#1\ReFiCh@NoFile 1\else 0\fi
          \ifx\x\ReFiCh@AuxEmptyUnix 1%
          \else
            \ifx\x\ReFiCh@AuxEmptyDos 1\fi
          \fi
        =111 %
          \@PackageInfoNoLine{rerunfilecheck}{%
            File `#2' is empty .aux file.\MessageBreak
            Before: #1\MessageBreak
            After: \space\x
          }%
        \else
          \@PackageWarningNoLine{rerunfilecheck}{%
            File `#2' has changed.%
            \ifx\\#4\\%
              \space Rerun%
            \else
              \MessageBreak
              #4%
            \fi
          }%
          \@PackageInfoNoLine{rerunfilecheck}{%
            Checksums for `#2':\MessageBreak
            Before: #1\MessageBreak
            After: \space\x
          }%
        \fi
      \fi
    \endgroup
  }%
}
%    \end{macrocode}
%    \end{macro}
%    \begin{macrocode}
\def\ReFiCh@IsAux#1.aux\relax#2\@nil{%
  \ifx\hbox#2\hbox
    0%
  \else
    1%
  \fi
}
%    \end{macrocode}
%    \begin{macrocode}
\def\ReFiCh@AuxEmptyUnix{A94A2480D3289E625EEA47CD1B285758;8}%
\@onelevel@sanitize\ReFiCh@AuxEmptyUnix
%    \end{macrocode}
%    \begin{macrocode}
\def\ReFiCh@AuxEmptyDos{A62A15ECE803E2EBE94952FCC9933BC0;9}%
\@onelevel@sanitize\ReFiCh@AuxEmptyDos
%    \end{macrocode}
%
%    \begin{macrocode}
\ReFiCh@AtEnd%
%</package>
%    \end{macrocode}
%
% \section{Test}
%
%    \begin{macrocode}
%<*test1>
\def\LoadCommand{\RequirePackage{rerunfilecheck}[2016/05/16]}
%</test1>
%    \end{macrocode}
% \subsection{Catcode checks for loading}
%
%    \begin{macrocode}
%<*test1>
%    \end{macrocode}
%    \begin{macrocode}
\catcode`\{=1 %
\catcode`\}=2 %
\catcode`\#=6 %
\catcode`\@=11 %
\expandafter\ifx\csname count@\endcsname\relax
  \countdef\count@=255 %
\fi
\expandafter\ifx\csname @gobble\endcsname\relax
  \long\def\@gobble#1{}%
\fi
\expandafter\ifx\csname @firstofone\endcsname\relax
  \long\def\@firstofone#1{#1}%
\fi
\expandafter\ifx\csname loop\endcsname\relax
  \expandafter\@firstofone
\else
  \expandafter\@gobble
\fi
{%
  \def\loop#1\repeat{%
    \def\body{#1}%
    \iterate
  }%
  \def\iterate{%
    \body
      \let\next\iterate
    \else
      \let\next\relax
    \fi
    \next
  }%
  \let\repeat=\fi
}%
\def\RestoreCatcodes{}
\count@=0 %
\loop
  \edef\RestoreCatcodes{%
    \RestoreCatcodes
    \catcode\the\count@=\the\catcode\count@\relax
  }%
\ifnum\count@<255 %
  \advance\count@ 1 %
\repeat

\def\RangeCatcodeInvalid#1#2{%
  \count@=#1\relax
  \loop
    \catcode\count@=15 %
  \ifnum\count@<#2\relax
    \advance\count@ 1 %
  \repeat
}
\def\RangeCatcodeCheck#1#2#3{%
  \count@=#1\relax
  \loop
    \ifnum#3=\catcode\count@
    \else
      \errmessage{%
        Character \the\count@\space
        with wrong catcode \the\catcode\count@\space
        instead of \number#3%
      }%
    \fi
  \ifnum\count@<#2\relax
    \advance\count@ 1 %
  \repeat
}
\def\space{ }
\expandafter\ifx\csname LoadCommand\endcsname\relax
  \def\LoadCommand{\input rerunfilecheck.sty\relax}%
\fi
\def\Test{%
  \RangeCatcodeInvalid{0}{47}%
  \RangeCatcodeInvalid{58}{64}%
  \RangeCatcodeInvalid{91}{96}%
  \RangeCatcodeInvalid{123}{255}%
  \catcode`\@=12 %
  \catcode`\\=0 %
  \catcode`\%=14 %
  \LoadCommand
  \RangeCatcodeCheck{0}{36}{15}%
  \RangeCatcodeCheck{37}{37}{14}%
  \RangeCatcodeCheck{38}{47}{15}%
  \RangeCatcodeCheck{48}{57}{12}%
  \RangeCatcodeCheck{58}{63}{15}%
  \RangeCatcodeCheck{64}{64}{12}%
  \RangeCatcodeCheck{65}{90}{11}%
  \RangeCatcodeCheck{91}{91}{15}%
  \RangeCatcodeCheck{92}{92}{0}%
  \RangeCatcodeCheck{93}{96}{15}%
  \RangeCatcodeCheck{97}{122}{11}%
  \RangeCatcodeCheck{123}{255}{15}%
  \RestoreCatcodes
}
\Test
\csname @@end\endcsname
\end
%    \end{macrocode}
%    \begin{macrocode}
%</test1>
%    \end{macrocode}
%
% \section{Installation}
%
% \subsection{Download}
%
% \paragraph{Package.} This package is available on
% CTAN\footnote{\url{http://ctan.org/pkg/rerunfilecheck}}:
% \begin{description}
% \item[\CTAN{macros/latex/contrib/oberdiek/rerunfilecheck.dtx}] The source file.
% \item[\CTAN{macros/latex/contrib/oberdiek/rerunfilecheck.pdf}] Documentation.
% \end{description}
%
%
% \paragraph{Bundle.} All the packages of the bundle `oberdiek'
% are also available in a TDS compliant ZIP archive. There
% the packages are already unpacked and the documentation files
% are generated. The files and directories obey the TDS standard.
% \begin{description}
% \item[\CTAN{install/macros/latex/contrib/oberdiek.tds.zip}]
% \end{description}
% \emph{TDS} refers to the standard ``A Directory Structure
% for \TeX\ Files'' (\CTAN{tds/tds.pdf}). Directories
% with \xfile{texmf} in their name are usually organized this way.
%
% \subsection{Bundle installation}
%
% \paragraph{Unpacking.} Unpack the \xfile{oberdiek.tds.zip} in the
% TDS tree (also known as \xfile{texmf} tree) of your choice.
% Example (linux):
% \begin{quote}
%   |unzip oberdiek.tds.zip -d ~/texmf|
% \end{quote}
%
% \paragraph{Script installation.}
% Check the directory \xfile{TDS:scripts/oberdiek/} for
% scripts that need further installation steps.
% Package \xpackage{attachfile2} comes with the Perl script
% \xfile{pdfatfi.pl} that should be installed in such a way
% that it can be called as \texttt{pdfatfi}.
% Example (linux):
% \begin{quote}
%   |chmod +x scripts/oberdiek/pdfatfi.pl|\\
%   |cp scripts/oberdiek/pdfatfi.pl /usr/local/bin/|
% \end{quote}
%
% \subsection{Package installation}
%
% \paragraph{Unpacking.} The \xfile{.dtx} file is a self-extracting
% \docstrip\ archive. The files are extracted by running the
% \xfile{.dtx} through \plainTeX:
% \begin{quote}
%   \verb|tex rerunfilecheck.dtx|
% \end{quote}
%
% \paragraph{TDS.} Now the different files must be moved into
% the different directories in your installation TDS tree
% (also known as \xfile{texmf} tree):
% \begin{quote}
% \def\t{^^A
% \begin{tabular}{@{}>{\ttfamily}l@{ $\rightarrow$ }>{\ttfamily}l@{}}
%   rerunfilecheck.sty & tex/latex/oberdiek/rerunfilecheck.sty\\
%   rerunfilecheck.pdf & doc/latex/oberdiek/rerunfilecheck.pdf\\
%   rerunfilecheck-example.cfg & doc/latex/oberdiek/rerunfilecheck-example.cfg\\
%   test/rerunfilecheck-test1.tex & doc/latex/oberdiek/test/rerunfilecheck-test1.tex\\
%   rerunfilecheck.dtx & source/latex/oberdiek/rerunfilecheck.dtx\\
% \end{tabular}^^A
% }^^A
% \sbox0{\t}^^A
% \ifdim\wd0>\linewidth
%   \begingroup
%     \advance\linewidth by\leftmargin
%     \advance\linewidth by\rightmargin
%   \edef\x{\endgroup
%     \def\noexpand\lw{\the\linewidth}^^A
%   }\x
%   \def\lwbox{^^A
%     \leavevmode
%     \hbox to \linewidth{^^A
%       \kern-\leftmargin\relax
%       \hss
%       \usebox0
%       \hss
%       \kern-\rightmargin\relax
%     }^^A
%   }^^A
%   \ifdim\wd0>\lw
%     \sbox0{\small\t}^^A
%     \ifdim\wd0>\linewidth
%       \ifdim\wd0>\lw
%         \sbox0{\footnotesize\t}^^A
%         \ifdim\wd0>\linewidth
%           \ifdim\wd0>\lw
%             \sbox0{\scriptsize\t}^^A
%             \ifdim\wd0>\linewidth
%               \ifdim\wd0>\lw
%                 \sbox0{\tiny\t}^^A
%                 \ifdim\wd0>\linewidth
%                   \lwbox
%                 \else
%                   \usebox0
%                 \fi
%               \else
%                 \lwbox
%               \fi
%             \else
%               \usebox0
%             \fi
%           \else
%             \lwbox
%           \fi
%         \else
%           \usebox0
%         \fi
%       \else
%         \lwbox
%       \fi
%     \else
%       \usebox0
%     \fi
%   \else
%     \lwbox
%   \fi
% \else
%   \usebox0
% \fi
% \end{quote}
% If you have a \xfile{docstrip.cfg} that configures and enables \docstrip's
% TDS installing feature, then some files can already be in the right
% place, see the documentation of \docstrip.
%
% \subsection{Refresh file name databases}
%
% If your \TeX~distribution
% (\teTeX, \mikTeX, \dots) relies on file name databases, you must refresh
% these. For example, \teTeX\ users run \verb|texhash| or
% \verb|mktexlsr|.
%
% \subsection{Some details for the interested}
%
% \paragraph{Attached source.}
%
% The PDF documentation on CTAN also includes the
% \xfile{.dtx} source file. It can be extracted by
% AcrobatReader 6 or higher. Another option is \textsf{pdftk},
% e.g. unpack the file into the current directory:
% \begin{quote}
%   \verb|pdftk rerunfilecheck.pdf unpack_files output .|
% \end{quote}
%
% \paragraph{Unpacking with \LaTeX.}
% The \xfile{.dtx} chooses its action depending on the format:
% \begin{description}
% \item[\plainTeX:] Run \docstrip\ and extract the files.
% \item[\LaTeX:] Generate the documentation.
% \end{description}
% If you insist on using \LaTeX\ for \docstrip\ (really,
% \docstrip\ does not need \LaTeX), then inform the autodetect routine
% about your intention:
% \begin{quote}
%   \verb|latex \let\install=y% \iffalse meta-comment
%
% File: rerunfilecheck.dtx
% Version: 2016/05/16 v1.8
% Info: Rerun checks for auxiliary files
%
% Copyright (C) 2009-2011 by
%    Heiko Oberdiek <heiko.oberdiek at googlemail.com>
%    2016
%    https://github.com/ho-tex/oberdiek/issues
%
% This work may be distributed and/or modified under the
% conditions of the LaTeX Project Public License, either
% version 1.3c of this license or (at your option) any later
% version. This version of this license is in
%    http://www.latex-project.org/lppl/lppl-1-3c.txt
% and the latest version of this license is in
%    http://www.latex-project.org/lppl.txt
% and version 1.3 or later is part of all distributions of
% LaTeX version 2005/12/01 or later.
%
% This work has the LPPL maintenance status "maintained".
%
% This Current Maintainer of this work is Heiko Oberdiek.
%
% This work consists of the main source file rerunfilecheck.dtx
% and the derived files
%    rerunfilecheck.sty, rerunfilecheck.pdf, rerunfilecheck.ins,
%    rerunfilecheck.drv, rerunfilecheck-example.cfg,
%    rerunfilecheck-test1.tex.
%
% Distribution:
%    CTAN:macros/latex/contrib/oberdiek/rerunfilecheck.dtx
%    CTAN:macros/latex/contrib/oberdiek/rerunfilecheck.pdf
%
% Unpacking:
%    (a) If rerunfilecheck.ins is present:
%           tex rerunfilecheck.ins
%    (b) Without rerunfilecheck.ins:
%           tex rerunfilecheck.dtx
%    (c) If you insist on using LaTeX
%           latex \let\install=y\input{rerunfilecheck.dtx}
%        (quote the arguments according to the demands of your shell)
%
% Documentation:
%    (a) If rerunfilecheck.drv is present:
%           latex rerunfilecheck.drv
%    (b) Without rerunfilecheck.drv:
%           latex rerunfilecheck.dtx; ...
%    The class ltxdoc loads the configuration file ltxdoc.cfg
%    if available. Here you can specify further options, e.g.
%    use A4 as paper format:
%       \PassOptionsToClass{a4paper}{article}
%
%    Programm calls to get the documentation (example):
%       pdflatex rerunfilecheck.dtx
%       makeindex -s gind.ist rerunfilecheck.idx
%       pdflatex rerunfilecheck.dtx
%       makeindex -s gind.ist rerunfilecheck.idx
%       pdflatex rerunfilecheck.dtx
%
% Installation:
%    TDS:tex/latex/oberdiek/rerunfilecheck.sty
%    TDS:doc/latex/oberdiek/rerunfilecheck.pdf
%    TDS:doc/latex/oberdiek/rerunfilecheck-example.cfg
%    TDS:doc/latex/oberdiek/test/rerunfilecheck-test1.tex
%    TDS:source/latex/oberdiek/rerunfilecheck.dtx
%
%<*ignore>
\begingroup
  \catcode123=1 %
  \catcode125=2 %
  \def\x{LaTeX2e}%
\expandafter\endgroup
\ifcase 0\ifx\install y1\fi\expandafter
         \ifx\csname processbatchFile\endcsname\relax\else1\fi
         \ifx\fmtname\x\else 1\fi\relax
\else\csname fi\endcsname
%</ignore>
%<*install>
\input docstrip.tex
\Msg{************************************************************************}
\Msg{* Installation}
\Msg{* Package: rerunfilecheck 2016/05/16 v1.8 Rerun checks for auxiliary files (HO)}
\Msg{************************************************************************}

\keepsilent
\askforoverwritefalse

\let\MetaPrefix\relax
\preamble

This is a generated file.

Project: rerunfilecheck
Version: 2016/05/16 v1.8

Copyright (C) 2009-2011 by
   Heiko Oberdiek <heiko.oberdiek at googlemail.com>

This work may be distributed and/or modified under the
conditions of the LaTeX Project Public License, either
version 1.3c of this license or (at your option) any later
version. This version of this license is in
   http://www.latex-project.org/lppl/lppl-1-3c.txt
and the latest version of this license is in
   http://www.latex-project.org/lppl.txt
and version 1.3 or later is part of all distributions of
LaTeX version 2005/12/01 or later.

This work has the LPPL maintenance status "maintained".

This Current Maintainer of this work is Heiko Oberdiek.

This work consists of the main source file rerunfilecheck.dtx
and the derived files
   rerunfilecheck.sty, rerunfilecheck.pdf, rerunfilecheck.ins,
   rerunfilecheck.drv, rerunfilecheck-example.cfg,
   rerunfilecheck-test1.tex.

\endpreamble
\let\MetaPrefix\DoubleperCent

\generate{%
  \file{rerunfilecheck.ins}{\from{rerunfilecheck.dtx}{install}}%
  \file{rerunfilecheck.drv}{\from{rerunfilecheck.dtx}{driver}}%
  \usedir{tex/latex/oberdiek}%
  \file{rerunfilecheck.sty}{\from{rerunfilecheck.dtx}{package}}%
%  \usedir{doc/latex/oberdiek/test}%
%  \file{rerunfilecheck-test1.tex}{\from{rerunfilecheck.dtx}{test1}}%
  \nopreamble
  \nopostamble
  \usedir{doc/latex/oberdiek}%
  \file{rerunfilecheck-example.cfg}{\from{rerunfilecheck.dtx}{cfg}}%
  \usedir{source/latex/oberdiek/catalogue}%
  \file{rerunfilecheck.xml}{\from{rerunfilecheck.dtx}{catalogue}}%
}

\catcode32=13\relax% active space
\let =\space%
\Msg{************************************************************************}
\Msg{*}
\Msg{* To finish the installation you have to move the following}
\Msg{* file into a directory searched by TeX:}
\Msg{*}
\Msg{*     rerunfilecheck.sty}
\Msg{*}
\Msg{* To produce the documentation run the file `rerunfilecheck.drv'}
\Msg{* through LaTeX.}
\Msg{*}
\Msg{* Happy TeXing!}
\Msg{*}
\Msg{************************************************************************}

\endbatchfile
%</install>
%<*ignore>
\fi
%</ignore>
%<*driver>
\NeedsTeXFormat{LaTeX2e}
\ProvidesFile{rerunfilecheck.drv}%
  [2016/05/16 v1.8 Rerun checks for auxiliary files (HO)]%
\documentclass{ltxdoc}
\usepackage{holtxdoc}[2011/11/22]
\begin{document}
  \DocInput{rerunfilecheck.dtx}%
\end{document}
%</driver>
% \fi
%
%
% \CharacterTable
%  {Upper-case    \A\B\C\D\E\F\G\H\I\J\K\L\M\N\O\P\Q\R\S\T\U\V\W\X\Y\Z
%   Lower-case    \a\b\c\d\e\f\g\h\i\j\k\l\m\n\o\p\q\r\s\t\u\v\w\x\y\z
%   Digits        \0\1\2\3\4\5\6\7\8\9
%   Exclamation   \!     Double quote  \"     Hash (number) \#
%   Dollar        \$     Percent       \%     Ampersand     \&
%   Acute accent  \'     Left paren    \(     Right paren   \)
%   Asterisk      \*     Plus          \+     Comma         \,
%   Minus         \-     Point         \.     Solidus       \/
%   Colon         \:     Semicolon     \;     Less than     \<
%   Equals        \=     Greater than  \>     Question mark \?
%   Commercial at \@     Left bracket  \[     Backslash     \\
%   Right bracket \]     Circumflex    \^     Underscore    \_
%   Grave accent  \`     Left brace    \{     Vertical bar  \|
%   Right brace   \}     Tilde         \~}
%
% \GetFileInfo{rerunfilecheck.drv}
%
% \title{The \xpackage{rerunfilecheck} package}
% \date{2016/05/16 v1.8}
% \author{Heiko Oberdiek\thanks
% {Please report any issues at https://github.com/ho-tex/oberdiek/issues}\\
% \xemail{heiko.oberdiek at googlemail.com}}
%
% \maketitle
%
% \begin{abstract}
% The package provides additional rerun warnings if some
% auxiliary files have changed. It is based on MD5 checksum,
% provided by \pdfTeX.
% \end{abstract}
%
% \tableofcontents
%
% \section{Documentation}
%
% \LaTeX\ informs the user, when to run \LaTeX\ again, if the
% references have changed. It has the old references from the
% first reading of the \xext{aux} files already in memory, thus
% it compares them with the new version of the \xext{aux} file
% at the end of the document. However this rerun warnings are
% not given for the table of contents and other data stored
% in the \xext{aux} files or other auxiliary files.
% Usually many of these data as the table of contents is not keep
% in memory. If someone wants to detect changes, he has either
% to keep the data in memory. This does not scale well with huge
% documents. Or he copies the file before they are changed. Slow
% I/O operations cost time.
%
% Since version 1.30.0 \pdfTeX\ provides \cs{pdfmdfivesum}
% and \cs{pdffilesize}.
% These features are also available in \LuaTeX, provided by package
% \xpackage{pdftexcmds}.
% Thus this package \xpackage{rerunfilecheck} uses these features to
% detect file changes. This saves the packages from keeping
% the whole files in memory or in file copies.
% The drawback are different files with the same size and the
% same MD5 checksum (seldom, hopefully).
%
% \subsection{Options}
%
% All options are key value options of boolean type.
% No option or |true| turns an option on, |false| disables
% an option.
% \begin{description}
% \item[\xoption{mainaux}:]
%   Check the main \xext{aux} file.
% \item[\xoption{partaux}:]
%   Check the \xext{aux} files from \cs{include} files.
% \item[\xoption{starttoc}:]
%   Add the rerun checks in \cs{@starttoc} that is called by
%   \cs{tableofcontents}, \cs{listoffigures}, \dots
% \item[\xoption{index}, \xoption{glossary}:]
%   \LaTeX's original \cs{makeindex} and \cs{makeglossary} are
%   redefined to add the rerun checks. The options do not have
%   an effect, if \cs{makeindex}/\cs{makeglossary} are already
%   called or if a package or class had redefined or will
%   redefine them.
% \item[\xoption{aux}:]
%   This option turns all previous options on or off.
%   ``aux'' means auxiliary file.
% \end{description}
% The default for the options is |false|, because
% some internals must be redefined to insert the
% rerun checks.
% The options can be set in \cs{usepackage} or
% the configuration file \xfile{rerunfilecheck.cfg}.
% Global options are ignored (since 1.4).
%
% \begin{declcs}{RerunFileCheckSetup} \M{key value list}
% \end{declcs}
% Options can also be set using \cs{RerunFileCheckSetup}.
% Currently all options are disabled after the package
% is loaded. Thus \cs{RerunFileCheckSetup} makes sense in
% the configuration file only.
%
% Example for the configuration file:
%    \begin{macrocode}
%<*cfg>
\ProvidesFile{rerunfilecheck.cfg}[2016/05/16 Default configuration]%
\RerunFileCheckSetup{aux}
%</cfg>
%    \end{macrocode}
%
% \subsection{Interface for class/package authors}
%
% \begin{declcs}{RerunFileCheck} \M{file} \M{file closing action} \M{rerun warning}
% \end{declcs}
% If you want to add a rerun check, call \cs{RerunFileCheck} right before
% an output file is opened for writing. The macro first remembers
% the current checksum of \meta{file}.
% The file is checked again right before the end of the job.
% Macro \cs{AtVeryEndDocument} of package \xpackage{atveryend}
% is used to place the check after the main aux file is closed
% in |\end{document}|. Before reading the file again, it must be
% closed. Provide the code for closing in argument \meta{file closing action}.
% Do not forget \cs{immediate} before \cs{openout}. Otherwise the
% closing action would be delayed to the next shipout that never happens
% (the last page is already shipped out).
% If the file has changed, \cs{RerunFileCheck} informs the user with
% a warning that the file has changed and says the magic word ``Rerun''.
% If the last argument \meta{rerun warning} is not empty, then
% the rerun sentence is replaced by it. Usually the phrase
% ``to get something right'' is added.
% As example the relevant part of the redefined \cs{makeindex}
% is shown, see package code:
%\begin{quote}
%\begin{verbatim}
%\newwrite\@indexfile
%\RerunFileCheck{\jobname.idx}{%
%  \immediate\closeout\@indexfile
%}{%
%  Rerun LaTeX/makeindex to get index right%
%}%
%\immediate\openout\@indexfile=\jobname.idx %
%\end{verbatim}
%\end{quote}
%
% \StopEventually{
% }
%
% \section{Implementation}
%    \begin{macrocode}
%<*package>
%    \end{macrocode}
%
%    \begin{macrocode}
\begingroup\catcode61\catcode48\catcode32=10\relax%
  \catcode13=5 % ^^M
  \endlinechar=13 %
  \catcode123=1 % {
  \catcode125=2 % }
  \catcode64=11 % @
  \def\x{\endgroup
    \expandafter\edef\csname ReFiCh@AtEnd\endcsname{%
      \endlinechar=\the\endlinechar\relax
      \catcode13=\the\catcode13\relax
      \catcode32=\the\catcode32\relax
      \catcode35=\the\catcode35\relax
      \catcode61=\the\catcode61\relax
      \catcode64=\the\catcode64\relax
      \catcode123=\the\catcode123\relax
      \catcode125=\the\catcode125\relax
    }%
  }%
\x\catcode61\catcode48\catcode32=10\relax%
\catcode13=5 % ^^M
\endlinechar=13 %
\catcode35=6 % #
\catcode64=11 % @
\catcode123=1 % {
\catcode125=2 % }
\def\TMP@EnsureCode#1#2{%
  \edef\ReFiCh@AtEnd{%
    \ReFiCh@AtEnd
    \catcode#1=\the\catcode#1\relax
  }%
  \catcode#1=#2\relax
}
\TMP@EnsureCode{39}{12}% '
\TMP@EnsureCode{40}{12}% (
\TMP@EnsureCode{41}{12}% )
\TMP@EnsureCode{42}{12}% *
\TMP@EnsureCode{44}{12}% ,
\TMP@EnsureCode{46}{12}% .
\TMP@EnsureCode{47}{12}% /
\TMP@EnsureCode{58}{12}% :
\TMP@EnsureCode{59}{12}% ;
\TMP@EnsureCode{60}{12}% <
\TMP@EnsureCode{62}{12}% >
\TMP@EnsureCode{91}{12}% [
\TMP@EnsureCode{93}{12}% ]
\TMP@EnsureCode{96}{12}% `
\edef\ReFiCh@AtEnd{\ReFiCh@AtEnd\noexpand\endinput}
%    \end{macrocode}
%
%    Package identification.
%    \begin{macrocode}
\NeedsTeXFormat{LaTeX2e}
\ProvidesPackage{rerunfilecheck}%
  [2016/05/16 v1.8 Rerun checks for auxiliary files (HO)]
%    \end{macrocode}
%
% \subsection{Options}
%
%    \begin{macrocode}
\RequirePackage{kvoptions}[2010/02/22]
\SetupKeyvalOptions{%
  family=rerunfilecheck,%
  prefix=ReFiCh@%
}
%    \end{macrocode}
%    \begin{macro}{\RerunFileCheckSetup}
%    \begin{macrocode}
\newcommand*{\RerunFileCheckSetup}{%
  \setkeys{rerunfilecheck}%
}
%    \end{macrocode}
%    \end{macro}
%    \begin{macrocode}
\DeclareBoolOption{mainaux}
\DeclareBoolOption{partaux}
\DeclareBoolOption{starttoc}
\DeclareBoolOption{index}
\DeclareBoolOption{glossary}
\define@key{rerunfilecheck}{aux}[true]{%
  \RerunFileCheckSetup{%
    mainaux={#1},%
    partaux={#1},%
    starttoc={#1},%
    index={#1},%
    glossary={#1}%
  }%
}
%    \end{macrocode}
%    \begin{macrocode}
\InputIfFileExists{rerunfilecheck.cfg}{}{}
\ProcessLocalKeyvalOptions*
%    \end{macrocode}
%
%    \begin{macro}{\ReFiCh@DisableOption}
%    \begin{macrocode}
\def\ReFiCh@DisableOption{%
  \DisableKeyvalOption[%
    action=warning,%
    package=rerunfilecheck%
  ]{rerunfilecheck}%
}
%    \end{macrocode}
%    \end{macro}
%
% \subsection{Check for checksum feature}
%
%    \begin{macrocode}
\RequirePackage{infwarerr}[2007/09/09]
\RequirePackage{pdftexcmds}[2009/04/10]
%    \end{macrocode}
%    \begin{macrocode}
\begingroup\expandafter\expandafter\expandafter\endgroup
\expandafter\ifx\csname pdf@filemdfivesum\endcsname\relax
  \@PackageInfoNoLine{rerunfilecheck}{%
    Feature \string\pdfmdfivesum\space is not available\MessageBreak
    (e.g. pdfTeX or LuaTeX with package `pdftexcmds').\MessageBreak
    Therefore file contents cannot be checked efficiently\MessageBreak
    and the loading of the package is aborted%
  }%
  \newcommand*{\RerunFileCheck}[3]{}%
  \renewcommand*{\RerunFileCheckSetup}[1]{}%
  \expandafter\ReFiCh@AtEnd
\fi%
%    \end{macrocode}
%
% \subsection{Standard \xext{aux} files}
%
%    \begin{macrocode}
\ifReFiCh@partaux
  \let\ReFiCh@org@include\@include
  \def\@include#1 {%
    \if@filesw
      \RerunFileCheck{#1.aux}{}{}%
    \fi
    \ReFiCh@org@include{#1} %
  }%
\fi
\ifReFiCh@mainaux
  \AtBeginDocument{%
    \ReFiCh@mainauxfalse
  }%
  \ifReFiCh@mainaux
    \AtEndOfPackage{%
      \RerunFileCheck{\jobname.aux}{}{}%
    }%
  \else
    \if@filesw
      \@PackageWarningNoLine{rerunfilecheck}{%
        Main aux file check is disabled,\MessageBreak
        because the file is already opened.\MessageBreak
        Load the package before \string\begin{document}%
      }%
    \fi
  \fi
\fi
\ifReFiCh@starttoc
  \let\ReFiCh@org@starttoc\@starttoc
  \def\@starttoc#1{%
    \if@filesw
      \RerunFileCheck{\jobname.#1}{%
        \@ifundefined{tf@#1}{%
        }{%
          \immediate\closeout\csname tf@#1\endcsname
        }%
      }{}%
    \fi
    \ReFiCh@org@starttoc{#1}%
  }%
\fi
\ifReFiCh@index
  \ifx\makeindex\@empty
    \@PackageWarningNoLine{rerunfilecheck}{%
      Option `index' ignored,\MessageBreak
      because \string\makeindex\space has already been called%
    }%
  \else
    \def\ReFiCh@temp{%
      \newwrite\@indexfile
      \immediate\openout\@indexfile=\jobname.idx %
      \def\index{%
        \@bsphack
        \begingroup
        \@sanitize
        \@wrindex
      }%
      \typeout{Writing index file \jobname.idx}%
      \let\makeindex\@empty
    }%
    \ifx\ReFiCh@temp\makeindex
      \def\makeindex{%
        \newwrite\@indexfile
        \RerunFileCheck{\jobname.idx}{%
          \immediate\closeout\@indexfile
        }{%
          Rerun LaTeX/makeindex to get index right%
        }%
        \immediate\openout\@indexfile=\jobname.idx %
        \def\index{%
          \@bsphack
          \begingroup
          \@sanitize
          \@wrindex
        }%
        \typeout{Writing index file \jobname.idx}%
        \let\makeindex\@empty
      }%
    \else
      \@PackageInfoNoLine{rerunfilecheck}{%
        Option `index': unsupported version of \string\makeindex
      }%
    \fi
  \fi
\fi
\ifReFiCh@glossary
  \ifx\makeglossary\@empty
    \@PackageWarningNoLine{rerunfilecheck}{%
      Option `glossary' ignored,\MessageBreak
      because \string\makeglossary\space has already been called%
    }%
  \else
    \def\ReFiCh@temp{%
      \newwrite\@glossaryfile
      \immediate\openout\@glossaryfile=\jobname.glo %
      \def\glossary{%
        \@bsphack
        \begingroup
        \@sanitize
        \@wrglossary
      }%
      \typeout{Writing glossary file \jobname.glo }%
      \let\makeglossary\@empty
    }%
    \ifx\ReFiCh@temp\makeglossary
      \def\ReFiCh@temp{%
        \newwrite\@glossaryfile
        \RerunFileCheck{\jobname.glo}{%
          \immediate\closeout\@glossaryfile
        }{%
          Rerun LaTeX/makeindex to get glossary right%
        }%
        \immediate\openout\@glossaryfile=\jobname.glo %
        \def\glossary{%
          \@bsphack
          \begingroup
          \@sanitize
          \@wrglossary
        }%
        \typeout{Writing glossary file \jobname.glo}%
        \let\makeglossary\@empty
      }%
    \else
      \@PackageInfoNoLine{rerunfilecheck}{%
        Option `glossary': unsupported version of \string\makeglossary
      }%
    \fi
  \fi
\fi
\ReFiCh@DisableOption{mainaux}
\ReFiCh@DisableOption{partaux}
\ReFiCh@DisableOption{starttoc}
\ReFiCh@DisableOption{index}
\ReFiCh@DisableOption{glossary}
\ReFiCh@DisableOption{aux}
%    \end{macrocode}
%
% \subsection{Rerun check}
%
%    \begin{macrocode}
\RequirePackage{atveryend}[2016/05/16]
\RequirePackage{uniquecounter}[2009/12/18]
%    \end{macrocode}
%
%    \begin{macro}{\ReFiCh@CheckSum}
%    \begin{macrocode}
\begingroup\expandafter\expandafter\expandafter\endgroup
\expandafter\ifx\csname pdf@filesize\endcsname\relax
  \def\ReFiCh@CheckSum{%
    \pdf@filemdfivesum
  }%
\else
  \def\ReFiCh@CheckSum#1{%
    \pdf@filemdfivesum{#1}%
    \ReFiCh@Separator
    \pdf@filesize{#1}%
  }%
\fi
%    \end{macrocode}
%    \end{macro}
%    \begin{macro}{\ReFiCh@NoFile}
%    \begin{macrocode}
\def\ReFiCh@Separator{;}
%    \end{macrocode}
%    \end{macro}
%    \begin{macro}{\ReFiCh@NoFile}
%    \begin{macrocode}
\def\ReFiCh@NoFile{<no file>}
%    \end{macrocode}
%    \end{macro}
%
%    \begin{macrocode}
\UniqueCounterNew{rerunfilecheck}
%    \end{macrocode}
%
%    \begin{macro}{\RerunFileCheck}
%    \begin{macrocode}
\newcommand*{\RerunFileCheck}{%
  \UniqueCounterCall{rerunfilecheck}\ReFiCh@RerunFileCheck
}
%    \end{macrocode}
%    \end{macro}
%    \begin{macro}{\ReFiCh@RerunFileCheck}
%    \begin{macrocode}
\def\ReFiCh@RerunFileCheck#1{%
  \expandafter\ReFiCh@@RerunFileCheck\csname ReFiCh@#1\endcsname
}
%    \end{macrocode}
%    \end{macro}
%    \begin{macro}{\ReFiCh@Check}
%    \begin{macrocode}
\def\ReFiCh@Check#1#2#3{%
%  \IfFileExists{#3}{%
    #1\edef#2{\ReFiCh@CheckSum{#3}}%
    \ifx#2\ReFiCh@Separator
      #1\let#2\ReFiCh@NoFile
    \fi
%  }{%
%    #1\let#2\ReFiCh@NoFile
%  }%
}
%    \end{macrocode}
%    \end{macro}
%    \begin{macro}{\ReFiCh@@RerunFileCheck}
%    \begin{macrocode}
\def\ReFiCh@@RerunFileCheck#1#2#3#4{%
  \ReFiCh@Check\global#1{#2}%
  \AtEndAfterFileList{%
    \begingroup
      #3%
      \ReFiCh@Check{}\x{#2}%
      \ifx#1\x
        \@PackageInfoNoLine{rerunfilecheck}{%
          File `#2' has not changed.\MessageBreak
          Checksum: \x
        }%
      \else
        \ifnum
          \ReFiCh@IsAux#2\relax.aux\relax\@nil
          \ifx#1\ReFiCh@NoFile 1\else 0\fi
          \ifx\x\ReFiCh@AuxEmptyUnix 1%
          \else
            \ifx\x\ReFiCh@AuxEmptyDos 1\fi
          \fi
        =111 %
          \@PackageInfoNoLine{rerunfilecheck}{%
            File `#2' is empty .aux file.\MessageBreak
            Before: #1\MessageBreak
            After: \space\x
          }%
        \else
          \@PackageWarningNoLine{rerunfilecheck}{%
            File `#2' has changed.%
            \ifx\\#4\\%
              \space Rerun%
            \else
              \MessageBreak
              #4%
            \fi
          }%
          \@PackageInfoNoLine{rerunfilecheck}{%
            Checksums for `#2':\MessageBreak
            Before: #1\MessageBreak
            After: \space\x
          }%
        \fi
      \fi
    \endgroup
  }%
}
%    \end{macrocode}
%    \end{macro}
%    \begin{macrocode}
\def\ReFiCh@IsAux#1.aux\relax#2\@nil{%
  \ifx\hbox#2\hbox
    0%
  \else
    1%
  \fi
}
%    \end{macrocode}
%    \begin{macrocode}
\def\ReFiCh@AuxEmptyUnix{A94A2480D3289E625EEA47CD1B285758;8}%
\@onelevel@sanitize\ReFiCh@AuxEmptyUnix
%    \end{macrocode}
%    \begin{macrocode}
\def\ReFiCh@AuxEmptyDos{A62A15ECE803E2EBE94952FCC9933BC0;9}%
\@onelevel@sanitize\ReFiCh@AuxEmptyDos
%    \end{macrocode}
%
%    \begin{macrocode}
\ReFiCh@AtEnd%
%</package>
%    \end{macrocode}
%
% \section{Test}
%
%    \begin{macrocode}
%<*test1>
\def\LoadCommand{\RequirePackage{rerunfilecheck}[2016/05/16]}
%</test1>
%    \end{macrocode}
% \subsection{Catcode checks for loading}
%
%    \begin{macrocode}
%<*test1>
%    \end{macrocode}
%    \begin{macrocode}
\catcode`\{=1 %
\catcode`\}=2 %
\catcode`\#=6 %
\catcode`\@=11 %
\expandafter\ifx\csname count@\endcsname\relax
  \countdef\count@=255 %
\fi
\expandafter\ifx\csname @gobble\endcsname\relax
  \long\def\@gobble#1{}%
\fi
\expandafter\ifx\csname @firstofone\endcsname\relax
  \long\def\@firstofone#1{#1}%
\fi
\expandafter\ifx\csname loop\endcsname\relax
  \expandafter\@firstofone
\else
  \expandafter\@gobble
\fi
{%
  \def\loop#1\repeat{%
    \def\body{#1}%
    \iterate
  }%
  \def\iterate{%
    \body
      \let\next\iterate
    \else
      \let\next\relax
    \fi
    \next
  }%
  \let\repeat=\fi
}%
\def\RestoreCatcodes{}
\count@=0 %
\loop
  \edef\RestoreCatcodes{%
    \RestoreCatcodes
    \catcode\the\count@=\the\catcode\count@\relax
  }%
\ifnum\count@<255 %
  \advance\count@ 1 %
\repeat

\def\RangeCatcodeInvalid#1#2{%
  \count@=#1\relax
  \loop
    \catcode\count@=15 %
  \ifnum\count@<#2\relax
    \advance\count@ 1 %
  \repeat
}
\def\RangeCatcodeCheck#1#2#3{%
  \count@=#1\relax
  \loop
    \ifnum#3=\catcode\count@
    \else
      \errmessage{%
        Character \the\count@\space
        with wrong catcode \the\catcode\count@\space
        instead of \number#3%
      }%
    \fi
  \ifnum\count@<#2\relax
    \advance\count@ 1 %
  \repeat
}
\def\space{ }
\expandafter\ifx\csname LoadCommand\endcsname\relax
  \def\LoadCommand{\input rerunfilecheck.sty\relax}%
\fi
\def\Test{%
  \RangeCatcodeInvalid{0}{47}%
  \RangeCatcodeInvalid{58}{64}%
  \RangeCatcodeInvalid{91}{96}%
  \RangeCatcodeInvalid{123}{255}%
  \catcode`\@=12 %
  \catcode`\\=0 %
  \catcode`\%=14 %
  \LoadCommand
  \RangeCatcodeCheck{0}{36}{15}%
  \RangeCatcodeCheck{37}{37}{14}%
  \RangeCatcodeCheck{38}{47}{15}%
  \RangeCatcodeCheck{48}{57}{12}%
  \RangeCatcodeCheck{58}{63}{15}%
  \RangeCatcodeCheck{64}{64}{12}%
  \RangeCatcodeCheck{65}{90}{11}%
  \RangeCatcodeCheck{91}{91}{15}%
  \RangeCatcodeCheck{92}{92}{0}%
  \RangeCatcodeCheck{93}{96}{15}%
  \RangeCatcodeCheck{97}{122}{11}%
  \RangeCatcodeCheck{123}{255}{15}%
  \RestoreCatcodes
}
\Test
\csname @@end\endcsname
\end
%    \end{macrocode}
%    \begin{macrocode}
%</test1>
%    \end{macrocode}
%
% \section{Installation}
%
% \subsection{Download}
%
% \paragraph{Package.} This package is available on
% CTAN\footnote{\url{http://ctan.org/pkg/rerunfilecheck}}:
% \begin{description}
% \item[\CTAN{macros/latex/contrib/oberdiek/rerunfilecheck.dtx}] The source file.
% \item[\CTAN{macros/latex/contrib/oberdiek/rerunfilecheck.pdf}] Documentation.
% \end{description}
%
%
% \paragraph{Bundle.} All the packages of the bundle `oberdiek'
% are also available in a TDS compliant ZIP archive. There
% the packages are already unpacked and the documentation files
% are generated. The files and directories obey the TDS standard.
% \begin{description}
% \item[\CTAN{install/macros/latex/contrib/oberdiek.tds.zip}]
% \end{description}
% \emph{TDS} refers to the standard ``A Directory Structure
% for \TeX\ Files'' (\CTAN{tds/tds.pdf}). Directories
% with \xfile{texmf} in their name are usually organized this way.
%
% \subsection{Bundle installation}
%
% \paragraph{Unpacking.} Unpack the \xfile{oberdiek.tds.zip} in the
% TDS tree (also known as \xfile{texmf} tree) of your choice.
% Example (linux):
% \begin{quote}
%   |unzip oberdiek.tds.zip -d ~/texmf|
% \end{quote}
%
% \paragraph{Script installation.}
% Check the directory \xfile{TDS:scripts/oberdiek/} for
% scripts that need further installation steps.
% Package \xpackage{attachfile2} comes with the Perl script
% \xfile{pdfatfi.pl} that should be installed in such a way
% that it can be called as \texttt{pdfatfi}.
% Example (linux):
% \begin{quote}
%   |chmod +x scripts/oberdiek/pdfatfi.pl|\\
%   |cp scripts/oberdiek/pdfatfi.pl /usr/local/bin/|
% \end{quote}
%
% \subsection{Package installation}
%
% \paragraph{Unpacking.} The \xfile{.dtx} file is a self-extracting
% \docstrip\ archive. The files are extracted by running the
% \xfile{.dtx} through \plainTeX:
% \begin{quote}
%   \verb|tex rerunfilecheck.dtx|
% \end{quote}
%
% \paragraph{TDS.} Now the different files must be moved into
% the different directories in your installation TDS tree
% (also known as \xfile{texmf} tree):
% \begin{quote}
% \def\t{^^A
% \begin{tabular}{@{}>{\ttfamily}l@{ $\rightarrow$ }>{\ttfamily}l@{}}
%   rerunfilecheck.sty & tex/latex/oberdiek/rerunfilecheck.sty\\
%   rerunfilecheck.pdf & doc/latex/oberdiek/rerunfilecheck.pdf\\
%   rerunfilecheck-example.cfg & doc/latex/oberdiek/rerunfilecheck-example.cfg\\
%   test/rerunfilecheck-test1.tex & doc/latex/oberdiek/test/rerunfilecheck-test1.tex\\
%   rerunfilecheck.dtx & source/latex/oberdiek/rerunfilecheck.dtx\\
% \end{tabular}^^A
% }^^A
% \sbox0{\t}^^A
% \ifdim\wd0>\linewidth
%   \begingroup
%     \advance\linewidth by\leftmargin
%     \advance\linewidth by\rightmargin
%   \edef\x{\endgroup
%     \def\noexpand\lw{\the\linewidth}^^A
%   }\x
%   \def\lwbox{^^A
%     \leavevmode
%     \hbox to \linewidth{^^A
%       \kern-\leftmargin\relax
%       \hss
%       \usebox0
%       \hss
%       \kern-\rightmargin\relax
%     }^^A
%   }^^A
%   \ifdim\wd0>\lw
%     \sbox0{\small\t}^^A
%     \ifdim\wd0>\linewidth
%       \ifdim\wd0>\lw
%         \sbox0{\footnotesize\t}^^A
%         \ifdim\wd0>\linewidth
%           \ifdim\wd0>\lw
%             \sbox0{\scriptsize\t}^^A
%             \ifdim\wd0>\linewidth
%               \ifdim\wd0>\lw
%                 \sbox0{\tiny\t}^^A
%                 \ifdim\wd0>\linewidth
%                   \lwbox
%                 \else
%                   \usebox0
%                 \fi
%               \else
%                 \lwbox
%               \fi
%             \else
%               \usebox0
%             \fi
%           \else
%             \lwbox
%           \fi
%         \else
%           \usebox0
%         \fi
%       \else
%         \lwbox
%       \fi
%     \else
%       \usebox0
%     \fi
%   \else
%     \lwbox
%   \fi
% \else
%   \usebox0
% \fi
% \end{quote}
% If you have a \xfile{docstrip.cfg} that configures and enables \docstrip's
% TDS installing feature, then some files can already be in the right
% place, see the documentation of \docstrip.
%
% \subsection{Refresh file name databases}
%
% If your \TeX~distribution
% (\teTeX, \mikTeX, \dots) relies on file name databases, you must refresh
% these. For example, \teTeX\ users run \verb|texhash| or
% \verb|mktexlsr|.
%
% \subsection{Some details for the interested}
%
% \paragraph{Attached source.}
%
% The PDF documentation on CTAN also includes the
% \xfile{.dtx} source file. It can be extracted by
% AcrobatReader 6 or higher. Another option is \textsf{pdftk},
% e.g. unpack the file into the current directory:
% \begin{quote}
%   \verb|pdftk rerunfilecheck.pdf unpack_files output .|
% \end{quote}
%
% \paragraph{Unpacking with \LaTeX.}
% The \xfile{.dtx} chooses its action depending on the format:
% \begin{description}
% \item[\plainTeX:] Run \docstrip\ and extract the files.
% \item[\LaTeX:] Generate the documentation.
% \end{description}
% If you insist on using \LaTeX\ for \docstrip\ (really,
% \docstrip\ does not need \LaTeX), then inform the autodetect routine
% about your intention:
% \begin{quote}
%   \verb|latex \let\install=y\input{rerunfilecheck.dtx}|
% \end{quote}
% Do not forget to quote the argument according to the demands
% of your shell.
%
% \paragraph{Generating the documentation.}
% You can use both the \xfile{.dtx} or the \xfile{.drv} to generate
% the documentation. The process can be configured by the
% configuration file \xfile{ltxdoc.cfg}. For instance, put this
% line into this file, if you want to have A4 as paper format:
% \begin{quote}
%   \verb|\PassOptionsToClass{a4paper}{article}|
% \end{quote}
% An example follows how to generate the
% documentation with pdf\LaTeX:
% \begin{quote}
%\begin{verbatim}
%pdflatex rerunfilecheck.dtx
%makeindex -s gind.ist rerunfilecheck.idx
%pdflatex rerunfilecheck.dtx
%makeindex -s gind.ist rerunfilecheck.idx
%pdflatex rerunfilecheck.dtx
%\end{verbatim}
% \end{quote}
%
% \section{Catalogue}
%
% The following XML file can be used as source for the
% \href{http://mirror.ctan.org/help/Catalogue/catalogue.html}{\TeX\ Catalogue}.
% The elements \texttt{caption} and \texttt{description} are imported
% from the original XML file from the Catalogue.
% The name of the XML file in the Catalogue is \xfile{rerunfilecheck.xml}.
%    \begin{macrocode}
%<*catalogue>
<?xml version='1.0' encoding='us-ascii'?>
<!DOCTYPE entry SYSTEM 'catalogue.dtd'>
<entry datestamp='$Date$' modifier='$Author$' id='rerunfilecheck'>
  <name>rerunfilecheck</name>
  <caption>Checksum based rerun checks on auxiliary files.</caption>
  <authorref id='auth:oberdiek'/>
  <copyright owner='Heiko Oberdiek' year='2009-2011'/>
  <license type='lppl1.3'/>
  <version number='1.8'/>
  <description>
    The package provides additional rerun warnings if some
    auxiliary files have changed. It is based on MD5 checksum,
    provided by pdfTeX.
    <p/>
    The package is part of the <xref refid='oberdiek'>oberdiek</xref> bundle.
  </description>
  <documentation details='Package documentation'
      href='ctan:/macros/latex/contrib/oberdiek/rerunfilecheck.pdf'/>
  <ctan file='true' path='/macros/latex/contrib/oberdiek/rerunfilecheck.dtx'/>
  <miktex location='oberdiek'/>
  <texlive location='oberdiek'/>
  <install path='/macros/latex/contrib/oberdiek/oberdiek.tds.zip'/>
</entry>
%</catalogue>
%    \end{macrocode}
%
% \begin{History}
%   \begin{Version}{2009/12/10 v1.0}
%   \item
%     The first version.
%   \end{Version}
%   \begin{Version}{2009/12/12 v1.1}
%   \item
%     Short info shortened.
%   \end{Version}
%   \begin{Version}{2009/12/18 v1.2}
%   \item
%     Required date for package \xpackage{uniquecounter} updated
%     because of bug in this package.
%   \end{Version}
%   \begin{Version}{2010/01/25 v1.3}
%   \item
%     Moved from \texttt{TDS:*/generic/*} to \texttt{TDS:*/latex/*}.
%   \end{Version}
%   \begin{Version}{2010/02/22 v1.4}
%   \item
%     The options of this package are recognized only if they are
%     package options. Global options are ignored. This avoids
%     name clashes with class and other package options
%     (for example, class option `index=totoc').
%   \end{Version}
%   \begin{Version}{2010/03/15 v1.5}
%   \item
%     Call of \cs{pdffivesum} is wrapped in \cs{IfFileExists} to avoid
%     calls of \textsf{mktextex} if this feature is enabled.
%     However \cs{IfFileExists} has file name limitations.
%   \end{Version}
%   \begin{Version}{2010/03/16 v1.6}
%   \item
%     Reverted to version 1.4 and \cs{IfFileExists} wrapper of
%     version 1.5 is removed.
%   \end{Version}
%   \begin{Version}{2011/04/15 v1.7}
%   \item
%     Using \cs{AtEndAfterFileList} of package \xpackage{atveryend}
%     2011/04/15 v1.6 instead of \cs{AtVeryEndDocument}.
%   \end{Version}
%   \begin{Version}{2016/05/16 v1.8}
%   \item
%     Documentation updates.
%   \end{Version}
% \end{History}
%
% \PrintIndex
%
% \Finale
\endinput
|
% \end{quote}
% Do not forget to quote the argument according to the demands
% of your shell.
%
% \paragraph{Generating the documentation.}
% You can use both the \xfile{.dtx} or the \xfile{.drv} to generate
% the documentation. The process can be configured by the
% configuration file \xfile{ltxdoc.cfg}. For instance, put this
% line into this file, if you want to have A4 as paper format:
% \begin{quote}
%   \verb|\PassOptionsToClass{a4paper}{article}|
% \end{quote}
% An example follows how to generate the
% documentation with pdf\LaTeX:
% \begin{quote}
%\begin{verbatim}
%pdflatex rerunfilecheck.dtx
%makeindex -s gind.ist rerunfilecheck.idx
%pdflatex rerunfilecheck.dtx
%makeindex -s gind.ist rerunfilecheck.idx
%pdflatex rerunfilecheck.dtx
%\end{verbatim}
% \end{quote}
%
% \section{Catalogue}
%
% The following XML file can be used as source for the
% \href{http://mirror.ctan.org/help/Catalogue/catalogue.html}{\TeX\ Catalogue}.
% The elements \texttt{caption} and \texttt{description} are imported
% from the original XML file from the Catalogue.
% The name of the XML file in the Catalogue is \xfile{rerunfilecheck.xml}.
%    \begin{macrocode}
%<*catalogue>
<?xml version='1.0' encoding='us-ascii'?>
<!DOCTYPE entry SYSTEM 'catalogue.dtd'>
<entry datestamp='$Date$' modifier='$Author$' id='rerunfilecheck'>
  <name>rerunfilecheck</name>
  <caption>Checksum based rerun checks on auxiliary files.</caption>
  <authorref id='auth:oberdiek'/>
  <copyright owner='Heiko Oberdiek' year='2009-2011'/>
  <license type='lppl1.3'/>
  <version number='1.8'/>
  <description>
    The package provides additional rerun warnings if some
    auxiliary files have changed. It is based on MD5 checksum,
    provided by pdfTeX.
    <p/>
    The package is part of the <xref refid='oberdiek'>oberdiek</xref> bundle.
  </description>
  <documentation details='Package documentation'
      href='ctan:/macros/latex/contrib/oberdiek/rerunfilecheck.pdf'/>
  <ctan file='true' path='/macros/latex/contrib/oberdiek/rerunfilecheck.dtx'/>
  <miktex location='oberdiek'/>
  <texlive location='oberdiek'/>
  <install path='/macros/latex/contrib/oberdiek/oberdiek.tds.zip'/>
</entry>
%</catalogue>
%    \end{macrocode}
%
% \begin{History}
%   \begin{Version}{2009/12/10 v1.0}
%   \item
%     The first version.
%   \end{Version}
%   \begin{Version}{2009/12/12 v1.1}
%   \item
%     Short info shortened.
%   \end{Version}
%   \begin{Version}{2009/12/18 v1.2}
%   \item
%     Required date for package \xpackage{uniquecounter} updated
%     because of bug in this package.
%   \end{Version}
%   \begin{Version}{2010/01/25 v1.3}
%   \item
%     Moved from \texttt{TDS:*/generic/*} to \texttt{TDS:*/latex/*}.
%   \end{Version}
%   \begin{Version}{2010/02/22 v1.4}
%   \item
%     The options of this package are recognized only if they are
%     package options. Global options are ignored. This avoids
%     name clashes with class and other package options
%     (for example, class option `index=totoc').
%   \end{Version}
%   \begin{Version}{2010/03/15 v1.5}
%   \item
%     Call of \cs{pdffivesum} is wrapped in \cs{IfFileExists} to avoid
%     calls of \textsf{mktextex} if this feature is enabled.
%     However \cs{IfFileExists} has file name limitations.
%   \end{Version}
%   \begin{Version}{2010/03/16 v1.6}
%   \item
%     Reverted to version 1.4 and \cs{IfFileExists} wrapper of
%     version 1.5 is removed.
%   \end{Version}
%   \begin{Version}{2011/04/15 v1.7}
%   \item
%     Using \cs{AtEndAfterFileList} of package \xpackage{atveryend}
%     2011/04/15 v1.6 instead of \cs{AtVeryEndDocument}.
%   \end{Version}
%   \begin{Version}{2016/05/16 v1.8}
%   \item
%     Documentation updates.
%   \end{Version}
% \end{History}
%
% \PrintIndex
%
% \Finale
\endinput
|
% \end{quote}
% Do not forget to quote the argument according to the demands
% of your shell.
%
% \paragraph{Generating the documentation.}
% You can use both the \xfile{.dtx} or the \xfile{.drv} to generate
% the documentation. The process can be configured by the
% configuration file \xfile{ltxdoc.cfg}. For instance, put this
% line into this file, if you want to have A4 as paper format:
% \begin{quote}
%   \verb|\PassOptionsToClass{a4paper}{article}|
% \end{quote}
% An example follows how to generate the
% documentation with pdf\LaTeX:
% \begin{quote}
%\begin{verbatim}
%pdflatex rerunfilecheck.dtx
%makeindex -s gind.ist rerunfilecheck.idx
%pdflatex rerunfilecheck.dtx
%makeindex -s gind.ist rerunfilecheck.idx
%pdflatex rerunfilecheck.dtx
%\end{verbatim}
% \end{quote}
%
% \section{Catalogue}
%
% The following XML file can be used as source for the
% \href{http://mirror.ctan.org/help/Catalogue/catalogue.html}{\TeX\ Catalogue}.
% The elements \texttt{caption} and \texttt{description} are imported
% from the original XML file from the Catalogue.
% The name of the XML file in the Catalogue is \xfile{rerunfilecheck.xml}.
%    \begin{macrocode}
%<*catalogue>
<?xml version='1.0' encoding='us-ascii'?>
<!DOCTYPE entry SYSTEM 'catalogue.dtd'>
<entry datestamp='$Date$' modifier='$Author$' id='rerunfilecheck'>
  <name>rerunfilecheck</name>
  <caption>Checksum based rerun checks on auxiliary files.</caption>
  <authorref id='auth:oberdiek'/>
  <copyright owner='Heiko Oberdiek' year='2009-2011'/>
  <license type='lppl1.3'/>
  <version number='1.8'/>
  <description>
    The package provides additional rerun warnings if some
    auxiliary files have changed. It is based on MD5 checksum,
    provided by pdfTeX.
    <p/>
    The package is part of the <xref refid='oberdiek'>oberdiek</xref> bundle.
  </description>
  <documentation details='Package documentation'
      href='ctan:/macros/latex/contrib/oberdiek/rerunfilecheck.pdf'/>
  <ctan file='true' path='/macros/latex/contrib/oberdiek/rerunfilecheck.dtx'/>
  <miktex location='oberdiek'/>
  <texlive location='oberdiek'/>
  <install path='/macros/latex/contrib/oberdiek/oberdiek.tds.zip'/>
</entry>
%</catalogue>
%    \end{macrocode}
%
% \begin{History}
%   \begin{Version}{2009/12/10 v1.0}
%   \item
%     The first version.
%   \end{Version}
%   \begin{Version}{2009/12/12 v1.1}
%   \item
%     Short info shortened.
%   \end{Version}
%   \begin{Version}{2009/12/18 v1.2}
%   \item
%     Required date for package \xpackage{uniquecounter} updated
%     because of bug in this package.
%   \end{Version}
%   \begin{Version}{2010/01/25 v1.3}
%   \item
%     Moved from \texttt{TDS:*/generic/*} to \texttt{TDS:*/latex/*}.
%   \end{Version}
%   \begin{Version}{2010/02/22 v1.4}
%   \item
%     The options of this package are recognized only if they are
%     package options. Global options are ignored. This avoids
%     name clashes with class and other package options
%     (for example, class option `index=totoc').
%   \end{Version}
%   \begin{Version}{2010/03/15 v1.5}
%   \item
%     Call of \cs{pdffivesum} is wrapped in \cs{IfFileExists} to avoid
%     calls of \textsf{mktextex} if this feature is enabled.
%     However \cs{IfFileExists} has file name limitations.
%   \end{Version}
%   \begin{Version}{2010/03/16 v1.6}
%   \item
%     Reverted to version 1.4 and \cs{IfFileExists} wrapper of
%     version 1.5 is removed.
%   \end{Version}
%   \begin{Version}{2011/04/15 v1.7}
%   \item
%     Using \cs{AtEndAfterFileList} of package \xpackage{atveryend}
%     2011/04/15 v1.6 instead of \cs{AtVeryEndDocument}.
%   \end{Version}
%   \begin{Version}{2016/05/16 v1.8}
%   \item
%     Documentation updates.
%   \end{Version}
% \end{History}
%
% \PrintIndex
%
% \Finale
\endinput

%        (quote the arguments according to the demands of your shell)
%
% Documentation:
%    (a) If rerunfilecheck.drv is present:
%           latex rerunfilecheck.drv
%    (b) Without rerunfilecheck.drv:
%           latex rerunfilecheck.dtx; ...
%    The class ltxdoc loads the configuration file ltxdoc.cfg
%    if available. Here you can specify further options, e.g.
%    use A4 as paper format:
%       \PassOptionsToClass{a4paper}{article}
%
%    Programm calls to get the documentation (example):
%       pdflatex rerunfilecheck.dtx
%       makeindex -s gind.ist rerunfilecheck.idx
%       pdflatex rerunfilecheck.dtx
%       makeindex -s gind.ist rerunfilecheck.idx
%       pdflatex rerunfilecheck.dtx
%
% Installation:
%    TDS:tex/latex/oberdiek/rerunfilecheck.sty
%    TDS:doc/latex/oberdiek/rerunfilecheck.pdf
%    TDS:doc/latex/oberdiek/rerunfilecheck-example.cfg
%    TDS:doc/latex/oberdiek/test/rerunfilecheck-test1.tex
%    TDS:source/latex/oberdiek/rerunfilecheck.dtx
%
%<*ignore>
\begingroup
  \catcode123=1 %
  \catcode125=2 %
  \def\x{LaTeX2e}%
\expandafter\endgroup
\ifcase 0\ifx\install y1\fi\expandafter
         \ifx\csname processbatchFile\endcsname\relax\else1\fi
         \ifx\fmtname\x\else 1\fi\relax
\else\csname fi\endcsname
%</ignore>
%<*install>
\input docstrip.tex
\Msg{************************************************************************}
\Msg{* Installation}
\Msg{* Package: rerunfilecheck 2016/05/16 v1.8 Rerun checks for auxiliary files (HO)}
\Msg{************************************************************************}

\keepsilent
\askforoverwritefalse

\let\MetaPrefix\relax
\preamble

This is a generated file.

Project: rerunfilecheck
Version: 2016/05/16 v1.8

Copyright (C) 2009-2011 by
   Heiko Oberdiek <heiko.oberdiek at googlemail.com>

This work may be distributed and/or modified under the
conditions of the LaTeX Project Public License, either
version 1.3c of this license or (at your option) any later
version. This version of this license is in
   http://www.latex-project.org/lppl/lppl-1-3c.txt
and the latest version of this license is in
   http://www.latex-project.org/lppl.txt
and version 1.3 or later is part of all distributions of
LaTeX version 2005/12/01 or later.

This work has the LPPL maintenance status "maintained".

This Current Maintainer of this work is Heiko Oberdiek.

This work consists of the main source file rerunfilecheck.dtx
and the derived files
   rerunfilecheck.sty, rerunfilecheck.pdf, rerunfilecheck.ins,
   rerunfilecheck.drv, rerunfilecheck-example.cfg,
   rerunfilecheck-test1.tex.

\endpreamble
\let\MetaPrefix\DoubleperCent

\generate{%
  \file{rerunfilecheck.ins}{\from{rerunfilecheck.dtx}{install}}%
  \file{rerunfilecheck.drv}{\from{rerunfilecheck.dtx}{driver}}%
  \usedir{tex/latex/oberdiek}%
  \file{rerunfilecheck.sty}{\from{rerunfilecheck.dtx}{package}}%
%  \usedir{doc/latex/oberdiek/test}%
%  \file{rerunfilecheck-test1.tex}{\from{rerunfilecheck.dtx}{test1}}%
  \nopreamble
  \nopostamble
  \usedir{doc/latex/oberdiek}%
  \file{rerunfilecheck-example.cfg}{\from{rerunfilecheck.dtx}{cfg}}%
%  \usedir{source/latex/oberdiek/catalogue}%
%  \file{rerunfilecheck.xml}{\from{rerunfilecheck.dtx}{catalogue}}%
}

\catcode32=13\relax% active space
\let =\space%
\Msg{************************************************************************}
\Msg{*}
\Msg{* To finish the installation you have to move the following}
\Msg{* file into a directory searched by TeX:}
\Msg{*}
\Msg{*     rerunfilecheck.sty}
\Msg{*}
\Msg{* To produce the documentation run the file `rerunfilecheck.drv'}
\Msg{* through LaTeX.}
\Msg{*}
\Msg{* Happy TeXing!}
\Msg{*}
\Msg{************************************************************************}

\endbatchfile
%</install>
%<*ignore>
\fi
%</ignore>
%<*driver>
\NeedsTeXFormat{LaTeX2e}
\ProvidesFile{rerunfilecheck.drv}%
  [2016/05/16 v1.8 Rerun checks for auxiliary files (HO)]%
\documentclass{ltxdoc}
\usepackage{holtxdoc}[2011/11/22]
\begin{document}
  \DocInput{rerunfilecheck.dtx}%
\end{document}
%</driver>
% \fi
%
%
% \CharacterTable
%  {Upper-case    \A\B\C\D\E\F\G\H\I\J\K\L\M\N\O\P\Q\R\S\T\U\V\W\X\Y\Z
%   Lower-case    \a\b\c\d\e\f\g\h\i\j\k\l\m\n\o\p\q\r\s\t\u\v\w\x\y\z
%   Digits        \0\1\2\3\4\5\6\7\8\9
%   Exclamation   \!     Double quote  \"     Hash (number) \#
%   Dollar        \$     Percent       \%     Ampersand     \&
%   Acute accent  \'     Left paren    \(     Right paren   \)
%   Asterisk      \*     Plus          \+     Comma         \,
%   Minus         \-     Point         \.     Solidus       \/
%   Colon         \:     Semicolon     \;     Less than     \<
%   Equals        \=     Greater than  \>     Question mark \?
%   Commercial at \@     Left bracket  \[     Backslash     \\
%   Right bracket \]     Circumflex    \^     Underscore    \_
%   Grave accent  \`     Left brace    \{     Vertical bar  \|
%   Right brace   \}     Tilde         \~}
%
% \GetFileInfo{rerunfilecheck.drv}
%
% \title{The \xpackage{rerunfilecheck} package}
% \date{2016/05/16 v1.8}
% \author{Heiko Oberdiek\thanks
% {Please report any issues at https://github.com/ho-tex/oberdiek/issues}\\
% \xemail{heiko.oberdiek at googlemail.com}}
%
% \maketitle
%
% \begin{abstract}
% The package provides additional rerun warnings if some
% auxiliary files have changed. It is based on MD5 checksum,
% provided by \pdfTeX.
% \end{abstract}
%
% \tableofcontents
%
% \section{Documentation}
%
% \LaTeX\ informs the user, when to run \LaTeX\ again, if the
% references have changed. It has the old references from the
% first reading of the \xext{aux} files already in memory, thus
% it compares them with the new version of the \xext{aux} file
% at the end of the document. However this rerun warnings are
% not given for the table of contents and other data stored
% in the \xext{aux} files or other auxiliary files.
% Usually many of these data as the table of contents is not keep
% in memory. If someone wants to detect changes, he has either
% to keep the data in memory. This does not scale well with huge
% documents. Or he copies the file before they are changed. Slow
% I/O operations cost time.
%
% Since version 1.30.0 \pdfTeX\ provides \cs{pdfmdfivesum}
% and \cs{pdffilesize}.
% These features are also available in \LuaTeX, provided by package
% \xpackage{pdftexcmds}.
% Thus this package \xpackage{rerunfilecheck} uses these features to
% detect file changes. This saves the packages from keeping
% the whole files in memory or in file copies.
% The drawback are different files with the same size and the
% same MD5 checksum (seldom, hopefully).
%
% \subsection{Options}
%
% All options are key value options of boolean type.
% No option or |true| turns an option on, |false| disables
% an option.
% \begin{description}
% \item[\xoption{mainaux}:]
%   Check the main \xext{aux} file.
% \item[\xoption{partaux}:]
%   Check the \xext{aux} files from \cs{include} files.
% \item[\xoption{starttoc}:]
%   Add the rerun checks in \cs{@starttoc} that is called by
%   \cs{tableofcontents}, \cs{listoffigures}, \dots
% \item[\xoption{index}, \xoption{glossary}:]
%   \LaTeX's original \cs{makeindex} and \cs{makeglossary} are
%   redefined to add the rerun checks. The options do not have
%   an effect, if \cs{makeindex}/\cs{makeglossary} are already
%   called or if a package or class had redefined or will
%   redefine them.
% \item[\xoption{aux}:]
%   This option turns all previous options on or off.
%   ``aux'' means auxiliary file.
% \end{description}
% The default for the options is |false|, because
% some internals must be redefined to insert the
% rerun checks.
% The options can be set in \cs{usepackage} or
% the configuration file \xfile{rerunfilecheck.cfg}.
% Global options are ignored (since 1.4).
%
% \begin{declcs}{RerunFileCheckSetup} \M{key value list}
% \end{declcs}
% Options can also be set using \cs{RerunFileCheckSetup}.
% Currently all options are disabled after the package
% is loaded. Thus \cs{RerunFileCheckSetup} makes sense in
% the configuration file only.
%
% Example for the configuration file:
%    \begin{macrocode}
%<*cfg>
\ProvidesFile{rerunfilecheck.cfg}[2016/05/16 Default configuration]%
\RerunFileCheckSetup{aux}
%</cfg>
%    \end{macrocode}
%
% \subsection{Interface for class/package authors}
%
% \begin{declcs}{RerunFileCheck} \M{file} \M{file closing action} \M{rerun warning}
% \end{declcs}
% If you want to add a rerun check, call \cs{RerunFileCheck} right before
% an output file is opened for writing. The macro first remembers
% the current checksum of \meta{file}.
% The file is checked again right before the end of the job.
% Macro \cs{AtVeryEndDocument} of package \xpackage{atveryend}
% is used to place the check after the main aux file is closed
% in |\end{document}|. Before reading the file again, it must be
% closed. Provide the code for closing in argument \meta{file closing action}.
% Do not forget \cs{immediate} before \cs{openout}. Otherwise the
% closing action would be delayed to the next shipout that never happens
% (the last page is already shipped out).
% If the file has changed, \cs{RerunFileCheck} informs the user with
% a warning that the file has changed and says the magic word ``Rerun''.
% If the last argument \meta{rerun warning} is not empty, then
% the rerun sentence is replaced by it. Usually the phrase
% ``to get something right'' is added.
% As example the relevant part of the redefined \cs{makeindex}
% is shown, see package code:
%\begin{quote}
%\begin{verbatim}
%\newwrite\@indexfile
%\RerunFileCheck{\jobname.idx}{%
%  \immediate\closeout\@indexfile
%}{%
%  Rerun LaTeX/makeindex to get index right%
%}%
%\immediate\openout\@indexfile=\jobname.idx %
%\end{verbatim}
%\end{quote}
%
% \StopEventually{
% }
%
% \section{Implementation}
%    \begin{macrocode}
%<*package>
%    \end{macrocode}
%
%    \begin{macrocode}
\begingroup\catcode61\catcode48\catcode32=10\relax%
  \catcode13=5 % ^^M
  \endlinechar=13 %
  \catcode123=1 % {
  \catcode125=2 % }
  \catcode64=11 % @
  \def\x{\endgroup
    \expandafter\edef\csname ReFiCh@AtEnd\endcsname{%
      \endlinechar=\the\endlinechar\relax
      \catcode13=\the\catcode13\relax
      \catcode32=\the\catcode32\relax
      \catcode35=\the\catcode35\relax
      \catcode61=\the\catcode61\relax
      \catcode64=\the\catcode64\relax
      \catcode123=\the\catcode123\relax
      \catcode125=\the\catcode125\relax
    }%
  }%
\x\catcode61\catcode48\catcode32=10\relax%
\catcode13=5 % ^^M
\endlinechar=13 %
\catcode35=6 % #
\catcode64=11 % @
\catcode123=1 % {
\catcode125=2 % }
\def\TMP@EnsureCode#1#2{%
  \edef\ReFiCh@AtEnd{%
    \ReFiCh@AtEnd
    \catcode#1=\the\catcode#1\relax
  }%
  \catcode#1=#2\relax
}
\TMP@EnsureCode{39}{12}% '
\TMP@EnsureCode{40}{12}% (
\TMP@EnsureCode{41}{12}% )
\TMP@EnsureCode{42}{12}% *
\TMP@EnsureCode{44}{12}% ,
\TMP@EnsureCode{46}{12}% .
\TMP@EnsureCode{47}{12}% /
\TMP@EnsureCode{58}{12}% :
\TMP@EnsureCode{59}{12}% ;
\TMP@EnsureCode{60}{12}% <
\TMP@EnsureCode{62}{12}% >
\TMP@EnsureCode{91}{12}% [
\TMP@EnsureCode{93}{12}% ]
\TMP@EnsureCode{96}{12}% `
\edef\ReFiCh@AtEnd{\ReFiCh@AtEnd\noexpand\endinput}
%    \end{macrocode}
%
%    Package identification.
%    \begin{macrocode}
\NeedsTeXFormat{LaTeX2e}
\ProvidesPackage{rerunfilecheck}%
  [2016/05/16 v1.8 Rerun checks for auxiliary files (HO)]
%    \end{macrocode}
%
% \subsection{Options}
%
%    \begin{macrocode}
\RequirePackage{kvoptions}[2010/02/22]
\SetupKeyvalOptions{%
  family=rerunfilecheck,%
  prefix=ReFiCh@%
}
%    \end{macrocode}
%    \begin{macro}{\RerunFileCheckSetup}
%    \begin{macrocode}
\newcommand*{\RerunFileCheckSetup}{%
  \setkeys{rerunfilecheck}%
}
%    \end{macrocode}
%    \end{macro}
%    \begin{macrocode}
\DeclareBoolOption{mainaux}
\DeclareBoolOption{partaux}
\DeclareBoolOption{starttoc}
\DeclareBoolOption{index}
\DeclareBoolOption{glossary}
\define@key{rerunfilecheck}{aux}[true]{%
  \RerunFileCheckSetup{%
    mainaux={#1},%
    partaux={#1},%
    starttoc={#1},%
    index={#1},%
    glossary={#1}%
  }%
}
%    \end{macrocode}
%    \begin{macrocode}
\InputIfFileExists{rerunfilecheck.cfg}{}{}
\ProcessLocalKeyvalOptions*
%    \end{macrocode}
%
%    \begin{macro}{\ReFiCh@DisableOption}
%    \begin{macrocode}
\def\ReFiCh@DisableOption{%
  \DisableKeyvalOption[%
    action=warning,%
    package=rerunfilecheck%
  ]{rerunfilecheck}%
}
%    \end{macrocode}
%    \end{macro}
%
% \subsection{Check for checksum feature}
%
%    \begin{macrocode}
\RequirePackage{infwarerr}[2007/09/09]
\RequirePackage{pdftexcmds}[2009/04/10]
%    \end{macrocode}
%    \begin{macrocode}
\begingroup\expandafter\expandafter\expandafter\endgroup
\expandafter\ifx\csname pdf@filemdfivesum\endcsname\relax
  \@PackageInfoNoLine{rerunfilecheck}{%
    Feature \string\pdfmdfivesum\space is not available\MessageBreak
    (e.g. pdfTeX or LuaTeX with package `pdftexcmds').\MessageBreak
    Therefore file contents cannot be checked efficiently\MessageBreak
    and the loading of the package is aborted%
  }%
  \newcommand*{\RerunFileCheck}[3]{}%
  \renewcommand*{\RerunFileCheckSetup}[1]{}%
  \expandafter\ReFiCh@AtEnd
\fi%
%    \end{macrocode}
%
% \subsection{Standard \xext{aux} files}
%
%    \begin{macrocode}
\ifReFiCh@partaux
  \let\ReFiCh@org@include\@include
  \def\@include#1 {%
    \if@filesw
      \RerunFileCheck{#1.aux}{}{}%
    \fi
    \ReFiCh@org@include{#1} %
  }%
\fi
\ifReFiCh@mainaux
  \AtBeginDocument{%
    \ReFiCh@mainauxfalse
  }%
  \ifReFiCh@mainaux
    \AtEndOfPackage{%
      \RerunFileCheck{\jobname.aux}{}{}%
    }%
  \else
    \if@filesw
      \@PackageWarningNoLine{rerunfilecheck}{%
        Main aux file check is disabled,\MessageBreak
        because the file is already opened.\MessageBreak
        Load the package before \string\begin{document}%
      }%
    \fi
  \fi
\fi
\ifReFiCh@starttoc
  \let\ReFiCh@org@starttoc\@starttoc
  \def\@starttoc#1{%
    \if@filesw
      \RerunFileCheck{\jobname.#1}{%
        \@ifundefined{tf@#1}{%
        }{%
          \immediate\closeout\csname tf@#1\endcsname
        }%
      }{}%
    \fi
    \ReFiCh@org@starttoc{#1}%
  }%
\fi
\ifReFiCh@index
  \ifx\makeindex\@empty
    \@PackageWarningNoLine{rerunfilecheck}{%
      Option `index' ignored,\MessageBreak
      because \string\makeindex\space has already been called%
    }%
  \else
    \def\ReFiCh@temp{%
      \newwrite\@indexfile
      \immediate\openout\@indexfile=\jobname.idx %
      \def\index{%
        \@bsphack
        \begingroup
        \@sanitize
        \@wrindex
      }%
      \typeout{Writing index file \jobname.idx}%
      \let\makeindex\@empty
    }%
    \ifx\ReFiCh@temp\makeindex
      \def\makeindex{%
        \newwrite\@indexfile
        \RerunFileCheck{\jobname.idx}{%
          \immediate\closeout\@indexfile
        }{%
          Rerun LaTeX/makeindex to get index right%
        }%
        \immediate\openout\@indexfile=\jobname.idx %
        \def\index{%
          \@bsphack
          \begingroup
          \@sanitize
          \@wrindex
        }%
        \typeout{Writing index file \jobname.idx}%
        \let\makeindex\@empty
      }%
    \else
      \@PackageInfoNoLine{rerunfilecheck}{%
        Option `index': unsupported version of \string\makeindex
      }%
    \fi
  \fi
\fi
\ifReFiCh@glossary
  \ifx\makeglossary\@empty
    \@PackageWarningNoLine{rerunfilecheck}{%
      Option `glossary' ignored,\MessageBreak
      because \string\makeglossary\space has already been called%
    }%
  \else
    \def\ReFiCh@temp{%
      \newwrite\@glossaryfile
      \immediate\openout\@glossaryfile=\jobname.glo %
      \def\glossary{%
        \@bsphack
        \begingroup
        \@sanitize
        \@wrglossary
      }%
      \typeout{Writing glossary file \jobname.glo }%
      \let\makeglossary\@empty
    }%
    \ifx\ReFiCh@temp\makeglossary
      \def\ReFiCh@temp{%
        \newwrite\@glossaryfile
        \RerunFileCheck{\jobname.glo}{%
          \immediate\closeout\@glossaryfile
        }{%
          Rerun LaTeX/makeindex to get glossary right%
        }%
        \immediate\openout\@glossaryfile=\jobname.glo %
        \def\glossary{%
          \@bsphack
          \begingroup
          \@sanitize
          \@wrglossary
        }%
        \typeout{Writing glossary file \jobname.glo}%
        \let\makeglossary\@empty
      }%
    \else
      \@PackageInfoNoLine{rerunfilecheck}{%
        Option `glossary': unsupported version of \string\makeglossary
      }%
    \fi
  \fi
\fi
\ReFiCh@DisableOption{mainaux}
\ReFiCh@DisableOption{partaux}
\ReFiCh@DisableOption{starttoc}
\ReFiCh@DisableOption{index}
\ReFiCh@DisableOption{glossary}
\ReFiCh@DisableOption{aux}
%    \end{macrocode}
%
% \subsection{Rerun check}
%
%    \begin{macrocode}
\RequirePackage{atveryend}[2016/05/16]
\RequirePackage{uniquecounter}[2009/12/18]
%    \end{macrocode}
%
%    \begin{macro}{\ReFiCh@CheckSum}
%    \begin{macrocode}
\begingroup\expandafter\expandafter\expandafter\endgroup
\expandafter\ifx\csname pdf@filesize\endcsname\relax
  \def\ReFiCh@CheckSum{%
    \pdf@filemdfivesum
  }%
\else
  \def\ReFiCh@CheckSum#1{%
    \pdf@filemdfivesum{#1}%
    \ReFiCh@Separator
    \pdf@filesize{#1}%
  }%
\fi
%    \end{macrocode}
%    \end{macro}
%    \begin{macro}{\ReFiCh@NoFile}
%    \begin{macrocode}
\def\ReFiCh@Separator{;}
%    \end{macrocode}
%    \end{macro}
%    \begin{macro}{\ReFiCh@NoFile}
%    \begin{macrocode}
\def\ReFiCh@NoFile{<no file>}
%    \end{macrocode}
%    \end{macro}
%
%    \begin{macrocode}
\UniqueCounterNew{rerunfilecheck}
%    \end{macrocode}
%
%    \begin{macro}{\RerunFileCheck}
%    \begin{macrocode}
\newcommand*{\RerunFileCheck}{%
  \UniqueCounterCall{rerunfilecheck}\ReFiCh@RerunFileCheck
}
%    \end{macrocode}
%    \end{macro}
%    \begin{macro}{\ReFiCh@RerunFileCheck}
%    \begin{macrocode}
\def\ReFiCh@RerunFileCheck#1{%
  \expandafter\ReFiCh@@RerunFileCheck\csname ReFiCh@#1\endcsname
}
%    \end{macrocode}
%    \end{macro}
%    \begin{macro}{\ReFiCh@Check}
%    \begin{macrocode}
\def\ReFiCh@Check#1#2#3{%
%  \IfFileExists{#3}{%
    #1\edef#2{\ReFiCh@CheckSum{#3}}%
    \ifx#2\ReFiCh@Separator
      #1\let#2\ReFiCh@NoFile
    \fi
%  }{%
%    #1\let#2\ReFiCh@NoFile
%  }%
}
%    \end{macrocode}
%    \end{macro}
%    \begin{macro}{\ReFiCh@@RerunFileCheck}
%    \begin{macrocode}
\def\ReFiCh@@RerunFileCheck#1#2#3#4{%
  \ReFiCh@Check\global#1{#2}%
  \AtEndAfterFileList{%
    \begingroup
      #3%
      \ReFiCh@Check{}\x{#2}%
      \ifx#1\x
        \@PackageInfoNoLine{rerunfilecheck}{%
          File `#2' has not changed.\MessageBreak
          Checksum: \x
        }%
      \else
        \ifnum
          \ReFiCh@IsAux#2\relax.aux\relax\@nil
          \ifx#1\ReFiCh@NoFile 1\else 0\fi
          \ifx\x\ReFiCh@AuxEmptyUnix 1%
          \else
            \ifx\x\ReFiCh@AuxEmptyDos 1\fi
          \fi
        =111 %
          \@PackageInfoNoLine{rerunfilecheck}{%
            File `#2' is empty .aux file.\MessageBreak
            Before: #1\MessageBreak
            After: \space\x
          }%
        \else
          \@PackageWarningNoLine{rerunfilecheck}{%
            File `#2' has changed.%
            \ifx\\#4\\%
              \space Rerun%
            \else
              \MessageBreak
              #4%
            \fi
          }%
          \@PackageInfoNoLine{rerunfilecheck}{%
            Checksums for `#2':\MessageBreak
            Before: #1\MessageBreak
            After: \space\x
          }%
        \fi
      \fi
    \endgroup
  }%
}
%    \end{macrocode}
%    \end{macro}
%    \begin{macrocode}
\def\ReFiCh@IsAux#1.aux\relax#2\@nil{%
  \ifx\hbox#2\hbox
    0%
  \else
    1%
  \fi
}
%    \end{macrocode}
%    \begin{macrocode}
\def\ReFiCh@AuxEmptyUnix{A94A2480D3289E625EEA47CD1B285758;8}%
\@onelevel@sanitize\ReFiCh@AuxEmptyUnix
%    \end{macrocode}
%    \begin{macrocode}
\def\ReFiCh@AuxEmptyDos{A62A15ECE803E2EBE94952FCC9933BC0;9}%
\@onelevel@sanitize\ReFiCh@AuxEmptyDos
%    \end{macrocode}
%
%    \begin{macrocode}
\ReFiCh@AtEnd%
%</package>
%    \end{macrocode}
%
% \section{Test}
%
%    \begin{macrocode}
%<*test1>
\def\LoadCommand{\RequirePackage{rerunfilecheck}[2016/05/16]}
%</test1>
%    \end{macrocode}
% \subsection{Catcode checks for loading}
%
%    \begin{macrocode}
%<*test1>
%    \end{macrocode}
%    \begin{macrocode}
\catcode`\{=1 %
\catcode`\}=2 %
\catcode`\#=6 %
\catcode`\@=11 %
\expandafter\ifx\csname count@\endcsname\relax
  \countdef\count@=255 %
\fi
\expandafter\ifx\csname @gobble\endcsname\relax
  \long\def\@gobble#1{}%
\fi
\expandafter\ifx\csname @firstofone\endcsname\relax
  \long\def\@firstofone#1{#1}%
\fi
\expandafter\ifx\csname loop\endcsname\relax
  \expandafter\@firstofone
\else
  \expandafter\@gobble
\fi
{%
  \def\loop#1\repeat{%
    \def\body{#1}%
    \iterate
  }%
  \def\iterate{%
    \body
      \let\next\iterate
    \else
      \let\next\relax
    \fi
    \next
  }%
  \let\repeat=\fi
}%
\def\RestoreCatcodes{}
\count@=0 %
\loop
  \edef\RestoreCatcodes{%
    \RestoreCatcodes
    \catcode\the\count@=\the\catcode\count@\relax
  }%
\ifnum\count@<255 %
  \advance\count@ 1 %
\repeat

\def\RangeCatcodeInvalid#1#2{%
  \count@=#1\relax
  \loop
    \catcode\count@=15 %
  \ifnum\count@<#2\relax
    \advance\count@ 1 %
  \repeat
}
\def\RangeCatcodeCheck#1#2#3{%
  \count@=#1\relax
  \loop
    \ifnum#3=\catcode\count@
    \else
      \errmessage{%
        Character \the\count@\space
        with wrong catcode \the\catcode\count@\space
        instead of \number#3%
      }%
    \fi
  \ifnum\count@<#2\relax
    \advance\count@ 1 %
  \repeat
}
\def\space{ }
\expandafter\ifx\csname LoadCommand\endcsname\relax
  \def\LoadCommand{\input rerunfilecheck.sty\relax}%
\fi
\def\Test{%
  \RangeCatcodeInvalid{0}{47}%
  \RangeCatcodeInvalid{58}{64}%
  \RangeCatcodeInvalid{91}{96}%
  \RangeCatcodeInvalid{123}{255}%
  \catcode`\@=12 %
  \catcode`\\=0 %
  \catcode`\%=14 %
  \LoadCommand
  \RangeCatcodeCheck{0}{36}{15}%
  \RangeCatcodeCheck{37}{37}{14}%
  \RangeCatcodeCheck{38}{47}{15}%
  \RangeCatcodeCheck{48}{57}{12}%
  \RangeCatcodeCheck{58}{63}{15}%
  \RangeCatcodeCheck{64}{64}{12}%
  \RangeCatcodeCheck{65}{90}{11}%
  \RangeCatcodeCheck{91}{91}{15}%
  \RangeCatcodeCheck{92}{92}{0}%
  \RangeCatcodeCheck{93}{96}{15}%
  \RangeCatcodeCheck{97}{122}{11}%
  \RangeCatcodeCheck{123}{255}{15}%
  \RestoreCatcodes
}
\Test
\csname @@end\endcsname
\end
%    \end{macrocode}
%    \begin{macrocode}
%</test1>
%    \end{macrocode}
%
% \section{Installation}
%
% \subsection{Download}
%
% \paragraph{Package.} This package is available on
% CTAN\footnote{\url{http://ctan.org/pkg/rerunfilecheck}}:
% \begin{description}
% \item[\CTAN{macros/latex/contrib/oberdiek/rerunfilecheck.dtx}] The source file.
% \item[\CTAN{macros/latex/contrib/oberdiek/rerunfilecheck.pdf}] Documentation.
% \end{description}
%
%
% \paragraph{Bundle.} All the packages of the bundle `oberdiek'
% are also available in a TDS compliant ZIP archive. There
% the packages are already unpacked and the documentation files
% are generated. The files and directories obey the TDS standard.
% \begin{description}
% \item[\CTAN{install/macros/latex/contrib/oberdiek.tds.zip}]
% \end{description}
% \emph{TDS} refers to the standard ``A Directory Structure
% for \TeX\ Files'' (\CTAN{tds/tds.pdf}). Directories
% with \xfile{texmf} in their name are usually organized this way.
%
% \subsection{Bundle installation}
%
% \paragraph{Unpacking.} Unpack the \xfile{oberdiek.tds.zip} in the
% TDS tree (also known as \xfile{texmf} tree) of your choice.
% Example (linux):
% \begin{quote}
%   |unzip oberdiek.tds.zip -d ~/texmf|
% \end{quote}
%
% \paragraph{Script installation.}
% Check the directory \xfile{TDS:scripts/oberdiek/} for
% scripts that need further installation steps.
% Package \xpackage{attachfile2} comes with the Perl script
% \xfile{pdfatfi.pl} that should be installed in such a way
% that it can be called as \texttt{pdfatfi}.
% Example (linux):
% \begin{quote}
%   |chmod +x scripts/oberdiek/pdfatfi.pl|\\
%   |cp scripts/oberdiek/pdfatfi.pl /usr/local/bin/|
% \end{quote}
%
% \subsection{Package installation}
%
% \paragraph{Unpacking.} The \xfile{.dtx} file is a self-extracting
% \docstrip\ archive. The files are extracted by running the
% \xfile{.dtx} through \plainTeX:
% \begin{quote}
%   \verb|tex rerunfilecheck.dtx|
% \end{quote}
%
% \paragraph{TDS.} Now the different files must be moved into
% the different directories in your installation TDS tree
% (also known as \xfile{texmf} tree):
% \begin{quote}
% \def\t{^^A
% \begin{tabular}{@{}>{\ttfamily}l@{ $\rightarrow$ }>{\ttfamily}l@{}}
%   rerunfilecheck.sty & tex/latex/oberdiek/rerunfilecheck.sty\\
%   rerunfilecheck.pdf & doc/latex/oberdiek/rerunfilecheck.pdf\\
%   rerunfilecheck-example.cfg & doc/latex/oberdiek/rerunfilecheck-example.cfg\\
%   test/rerunfilecheck-test1.tex & doc/latex/oberdiek/test/rerunfilecheck-test1.tex\\
%   rerunfilecheck.dtx & source/latex/oberdiek/rerunfilecheck.dtx\\
% \end{tabular}^^A
% }^^A
% \sbox0{\t}^^A
% \ifdim\wd0>\linewidth
%   \begingroup
%     \advance\linewidth by\leftmargin
%     \advance\linewidth by\rightmargin
%   \edef\x{\endgroup
%     \def\noexpand\lw{\the\linewidth}^^A
%   }\x
%   \def\lwbox{^^A
%     \leavevmode
%     \hbox to \linewidth{^^A
%       \kern-\leftmargin\relax
%       \hss
%       \usebox0
%       \hss
%       \kern-\rightmargin\relax
%     }^^A
%   }^^A
%   \ifdim\wd0>\lw
%     \sbox0{\small\t}^^A
%     \ifdim\wd0>\linewidth
%       \ifdim\wd0>\lw
%         \sbox0{\footnotesize\t}^^A
%         \ifdim\wd0>\linewidth
%           \ifdim\wd0>\lw
%             \sbox0{\scriptsize\t}^^A
%             \ifdim\wd0>\linewidth
%               \ifdim\wd0>\lw
%                 \sbox0{\tiny\t}^^A
%                 \ifdim\wd0>\linewidth
%                   \lwbox
%                 \else
%                   \usebox0
%                 \fi
%               \else
%                 \lwbox
%               \fi
%             \else
%               \usebox0
%             \fi
%           \else
%             \lwbox
%           \fi
%         \else
%           \usebox0
%         \fi
%       \else
%         \lwbox
%       \fi
%     \else
%       \usebox0
%     \fi
%   \else
%     \lwbox
%   \fi
% \else
%   \usebox0
% \fi
% \end{quote}
% If you have a \xfile{docstrip.cfg} that configures and enables \docstrip's
% TDS installing feature, then some files can already be in the right
% place, see the documentation of \docstrip.
%
% \subsection{Refresh file name databases}
%
% If your \TeX~distribution
% (\teTeX, \mikTeX, \dots) relies on file name databases, you must refresh
% these. For example, \teTeX\ users run \verb|texhash| or
% \verb|mktexlsr|.
%
% \subsection{Some details for the interested}
%
% \paragraph{Attached source.}
%
% The PDF documentation on CTAN also includes the
% \xfile{.dtx} source file. It can be extracted by
% AcrobatReader 6 or higher. Another option is \textsf{pdftk},
% e.g. unpack the file into the current directory:
% \begin{quote}
%   \verb|pdftk rerunfilecheck.pdf unpack_files output .|
% \end{quote}
%
% \paragraph{Unpacking with \LaTeX.}
% The \xfile{.dtx} chooses its action depending on the format:
% \begin{description}
% \item[\plainTeX:] Run \docstrip\ and extract the files.
% \item[\LaTeX:] Generate the documentation.
% \end{description}
% If you insist on using \LaTeX\ for \docstrip\ (really,
% \docstrip\ does not need \LaTeX), then inform the autodetect routine
% about your intention:
% \begin{quote}
%   \verb|latex \let\install=y% \iffalse meta-comment
%
% File: rerunfilecheck.dtx
% Version: 2016/05/16 v1.8
% Info: Rerun checks for auxiliary files
%
% Copyright (C) 2009-2011 by
%    Heiko Oberdiek <heiko.oberdiek at googlemail.com>
%    2016
%    https://github.com/ho-tex/oberdiek/issues
%
% This work may be distributed and/or modified under the
% conditions of the LaTeX Project Public License, either
% version 1.3c of this license or (at your option) any later
% version. This version of this license is in
%    http://www.latex-project.org/lppl/lppl-1-3c.txt
% and the latest version of this license is in
%    http://www.latex-project.org/lppl.txt
% and version 1.3 or later is part of all distributions of
% LaTeX version 2005/12/01 or later.
%
% This work has the LPPL maintenance status "maintained".
%
% This Current Maintainer of this work is Heiko Oberdiek.
%
% This work consists of the main source file rerunfilecheck.dtx
% and the derived files
%    rerunfilecheck.sty, rerunfilecheck.pdf, rerunfilecheck.ins,
%    rerunfilecheck.drv, rerunfilecheck-example.cfg,
%    rerunfilecheck-test1.tex.
%
% Distribution:
%    CTAN:macros/latex/contrib/oberdiek/rerunfilecheck.dtx
%    CTAN:macros/latex/contrib/oberdiek/rerunfilecheck.pdf
%
% Unpacking:
%    (a) If rerunfilecheck.ins is present:
%           tex rerunfilecheck.ins
%    (b) Without rerunfilecheck.ins:
%           tex rerunfilecheck.dtx
%    (c) If you insist on using LaTeX
%           latex \let\install=y% \iffalse meta-comment
%
% File: rerunfilecheck.dtx
% Version: 2016/05/16 v1.8
% Info: Rerun checks for auxiliary files
%
% Copyright (C) 2009-2011 by
%    Heiko Oberdiek <heiko.oberdiek at googlemail.com>
%    2016
%    https://github.com/ho-tex/oberdiek/issues
%
% This work may be distributed and/or modified under the
% conditions of the LaTeX Project Public License, either
% version 1.3c of this license or (at your option) any later
% version. This version of this license is in
%    http://www.latex-project.org/lppl/lppl-1-3c.txt
% and the latest version of this license is in
%    http://www.latex-project.org/lppl.txt
% and version 1.3 or later is part of all distributions of
% LaTeX version 2005/12/01 or later.
%
% This work has the LPPL maintenance status "maintained".
%
% This Current Maintainer of this work is Heiko Oberdiek.
%
% This work consists of the main source file rerunfilecheck.dtx
% and the derived files
%    rerunfilecheck.sty, rerunfilecheck.pdf, rerunfilecheck.ins,
%    rerunfilecheck.drv, rerunfilecheck-example.cfg,
%    rerunfilecheck-test1.tex.
%
% Distribution:
%    CTAN:macros/latex/contrib/oberdiek/rerunfilecheck.dtx
%    CTAN:macros/latex/contrib/oberdiek/rerunfilecheck.pdf
%
% Unpacking:
%    (a) If rerunfilecheck.ins is present:
%           tex rerunfilecheck.ins
%    (b) Without rerunfilecheck.ins:
%           tex rerunfilecheck.dtx
%    (c) If you insist on using LaTeX
%           latex \let\install=y% \iffalse meta-comment
%
% File: rerunfilecheck.dtx
% Version: 2016/05/16 v1.8
% Info: Rerun checks for auxiliary files
%
% Copyright (C) 2009-2011 by
%    Heiko Oberdiek <heiko.oberdiek at googlemail.com>
%    2016
%    https://github.com/ho-tex/oberdiek/issues
%
% This work may be distributed and/or modified under the
% conditions of the LaTeX Project Public License, either
% version 1.3c of this license or (at your option) any later
% version. This version of this license is in
%    http://www.latex-project.org/lppl/lppl-1-3c.txt
% and the latest version of this license is in
%    http://www.latex-project.org/lppl.txt
% and version 1.3 or later is part of all distributions of
% LaTeX version 2005/12/01 or later.
%
% This work has the LPPL maintenance status "maintained".
%
% This Current Maintainer of this work is Heiko Oberdiek.
%
% This work consists of the main source file rerunfilecheck.dtx
% and the derived files
%    rerunfilecheck.sty, rerunfilecheck.pdf, rerunfilecheck.ins,
%    rerunfilecheck.drv, rerunfilecheck-example.cfg,
%    rerunfilecheck-test1.tex.
%
% Distribution:
%    CTAN:macros/latex/contrib/oberdiek/rerunfilecheck.dtx
%    CTAN:macros/latex/contrib/oberdiek/rerunfilecheck.pdf
%
% Unpacking:
%    (a) If rerunfilecheck.ins is present:
%           tex rerunfilecheck.ins
%    (b) Without rerunfilecheck.ins:
%           tex rerunfilecheck.dtx
%    (c) If you insist on using LaTeX
%           latex \let\install=y\input{rerunfilecheck.dtx}
%        (quote the arguments according to the demands of your shell)
%
% Documentation:
%    (a) If rerunfilecheck.drv is present:
%           latex rerunfilecheck.drv
%    (b) Without rerunfilecheck.drv:
%           latex rerunfilecheck.dtx; ...
%    The class ltxdoc loads the configuration file ltxdoc.cfg
%    if available. Here you can specify further options, e.g.
%    use A4 as paper format:
%       \PassOptionsToClass{a4paper}{article}
%
%    Programm calls to get the documentation (example):
%       pdflatex rerunfilecheck.dtx
%       makeindex -s gind.ist rerunfilecheck.idx
%       pdflatex rerunfilecheck.dtx
%       makeindex -s gind.ist rerunfilecheck.idx
%       pdflatex rerunfilecheck.dtx
%
% Installation:
%    TDS:tex/latex/oberdiek/rerunfilecheck.sty
%    TDS:doc/latex/oberdiek/rerunfilecheck.pdf
%    TDS:doc/latex/oberdiek/rerunfilecheck-example.cfg
%    TDS:doc/latex/oberdiek/test/rerunfilecheck-test1.tex
%    TDS:source/latex/oberdiek/rerunfilecheck.dtx
%
%<*ignore>
\begingroup
  \catcode123=1 %
  \catcode125=2 %
  \def\x{LaTeX2e}%
\expandafter\endgroup
\ifcase 0\ifx\install y1\fi\expandafter
         \ifx\csname processbatchFile\endcsname\relax\else1\fi
         \ifx\fmtname\x\else 1\fi\relax
\else\csname fi\endcsname
%</ignore>
%<*install>
\input docstrip.tex
\Msg{************************************************************************}
\Msg{* Installation}
\Msg{* Package: rerunfilecheck 2016/05/16 v1.8 Rerun checks for auxiliary files (HO)}
\Msg{************************************************************************}

\keepsilent
\askforoverwritefalse

\let\MetaPrefix\relax
\preamble

This is a generated file.

Project: rerunfilecheck
Version: 2016/05/16 v1.8

Copyright (C) 2009-2011 by
   Heiko Oberdiek <heiko.oberdiek at googlemail.com>

This work may be distributed and/or modified under the
conditions of the LaTeX Project Public License, either
version 1.3c of this license or (at your option) any later
version. This version of this license is in
   http://www.latex-project.org/lppl/lppl-1-3c.txt
and the latest version of this license is in
   http://www.latex-project.org/lppl.txt
and version 1.3 or later is part of all distributions of
LaTeX version 2005/12/01 or later.

This work has the LPPL maintenance status "maintained".

This Current Maintainer of this work is Heiko Oberdiek.

This work consists of the main source file rerunfilecheck.dtx
and the derived files
   rerunfilecheck.sty, rerunfilecheck.pdf, rerunfilecheck.ins,
   rerunfilecheck.drv, rerunfilecheck-example.cfg,
   rerunfilecheck-test1.tex.

\endpreamble
\let\MetaPrefix\DoubleperCent

\generate{%
  \file{rerunfilecheck.ins}{\from{rerunfilecheck.dtx}{install}}%
  \file{rerunfilecheck.drv}{\from{rerunfilecheck.dtx}{driver}}%
  \usedir{tex/latex/oberdiek}%
  \file{rerunfilecheck.sty}{\from{rerunfilecheck.dtx}{package}}%
%  \usedir{doc/latex/oberdiek/test}%
%  \file{rerunfilecheck-test1.tex}{\from{rerunfilecheck.dtx}{test1}}%
  \nopreamble
  \nopostamble
  \usedir{doc/latex/oberdiek}%
  \file{rerunfilecheck-example.cfg}{\from{rerunfilecheck.dtx}{cfg}}%
  \usedir{source/latex/oberdiek/catalogue}%
  \file{rerunfilecheck.xml}{\from{rerunfilecheck.dtx}{catalogue}}%
}

\catcode32=13\relax% active space
\let =\space%
\Msg{************************************************************************}
\Msg{*}
\Msg{* To finish the installation you have to move the following}
\Msg{* file into a directory searched by TeX:}
\Msg{*}
\Msg{*     rerunfilecheck.sty}
\Msg{*}
\Msg{* To produce the documentation run the file `rerunfilecheck.drv'}
\Msg{* through LaTeX.}
\Msg{*}
\Msg{* Happy TeXing!}
\Msg{*}
\Msg{************************************************************************}

\endbatchfile
%</install>
%<*ignore>
\fi
%</ignore>
%<*driver>
\NeedsTeXFormat{LaTeX2e}
\ProvidesFile{rerunfilecheck.drv}%
  [2016/05/16 v1.8 Rerun checks for auxiliary files (HO)]%
\documentclass{ltxdoc}
\usepackage{holtxdoc}[2011/11/22]
\begin{document}
  \DocInput{rerunfilecheck.dtx}%
\end{document}
%</driver>
% \fi
%
%
% \CharacterTable
%  {Upper-case    \A\B\C\D\E\F\G\H\I\J\K\L\M\N\O\P\Q\R\S\T\U\V\W\X\Y\Z
%   Lower-case    \a\b\c\d\e\f\g\h\i\j\k\l\m\n\o\p\q\r\s\t\u\v\w\x\y\z
%   Digits        \0\1\2\3\4\5\6\7\8\9
%   Exclamation   \!     Double quote  \"     Hash (number) \#
%   Dollar        \$     Percent       \%     Ampersand     \&
%   Acute accent  \'     Left paren    \(     Right paren   \)
%   Asterisk      \*     Plus          \+     Comma         \,
%   Minus         \-     Point         \.     Solidus       \/
%   Colon         \:     Semicolon     \;     Less than     \<
%   Equals        \=     Greater than  \>     Question mark \?
%   Commercial at \@     Left bracket  \[     Backslash     \\
%   Right bracket \]     Circumflex    \^     Underscore    \_
%   Grave accent  \`     Left brace    \{     Vertical bar  \|
%   Right brace   \}     Tilde         \~}
%
% \GetFileInfo{rerunfilecheck.drv}
%
% \title{The \xpackage{rerunfilecheck} package}
% \date{2016/05/16 v1.8}
% \author{Heiko Oberdiek\thanks
% {Please report any issues at https://github.com/ho-tex/oberdiek/issues}\\
% \xemail{heiko.oberdiek at googlemail.com}}
%
% \maketitle
%
% \begin{abstract}
% The package provides additional rerun warnings if some
% auxiliary files have changed. It is based on MD5 checksum,
% provided by \pdfTeX.
% \end{abstract}
%
% \tableofcontents
%
% \section{Documentation}
%
% \LaTeX\ informs the user, when to run \LaTeX\ again, if the
% references have changed. It has the old references from the
% first reading of the \xext{aux} files already in memory, thus
% it compares them with the new version of the \xext{aux} file
% at the end of the document. However this rerun warnings are
% not given for the table of contents and other data stored
% in the \xext{aux} files or other auxiliary files.
% Usually many of these data as the table of contents is not keep
% in memory. If someone wants to detect changes, he has either
% to keep the data in memory. This does not scale well with huge
% documents. Or he copies the file before they are changed. Slow
% I/O operations cost time.
%
% Since version 1.30.0 \pdfTeX\ provides \cs{pdfmdfivesum}
% and \cs{pdffilesize}.
% These features are also available in \LuaTeX, provided by package
% \xpackage{pdftexcmds}.
% Thus this package \xpackage{rerunfilecheck} uses these features to
% detect file changes. This saves the packages from keeping
% the whole files in memory or in file copies.
% The drawback are different files with the same size and the
% same MD5 checksum (seldom, hopefully).
%
% \subsection{Options}
%
% All options are key value options of boolean type.
% No option or |true| turns an option on, |false| disables
% an option.
% \begin{description}
% \item[\xoption{mainaux}:]
%   Check the main \xext{aux} file.
% \item[\xoption{partaux}:]
%   Check the \xext{aux} files from \cs{include} files.
% \item[\xoption{starttoc}:]
%   Add the rerun checks in \cs{@starttoc} that is called by
%   \cs{tableofcontents}, \cs{listoffigures}, \dots
% \item[\xoption{index}, \xoption{glossary}:]
%   \LaTeX's original \cs{makeindex} and \cs{makeglossary} are
%   redefined to add the rerun checks. The options do not have
%   an effect, if \cs{makeindex}/\cs{makeglossary} are already
%   called or if a package or class had redefined or will
%   redefine them.
% \item[\xoption{aux}:]
%   This option turns all previous options on or off.
%   ``aux'' means auxiliary file.
% \end{description}
% The default for the options is |false|, because
% some internals must be redefined to insert the
% rerun checks.
% The options can be set in \cs{usepackage} or
% the configuration file \xfile{rerunfilecheck.cfg}.
% Global options are ignored (since 1.4).
%
% \begin{declcs}{RerunFileCheckSetup} \M{key value list}
% \end{declcs}
% Options can also be set using \cs{RerunFileCheckSetup}.
% Currently all options are disabled after the package
% is loaded. Thus \cs{RerunFileCheckSetup} makes sense in
% the configuration file only.
%
% Example for the configuration file:
%    \begin{macrocode}
%<*cfg>
\ProvidesFile{rerunfilecheck.cfg}[2016/05/16 Default configuration]%
\RerunFileCheckSetup{aux}
%</cfg>
%    \end{macrocode}
%
% \subsection{Interface for class/package authors}
%
% \begin{declcs}{RerunFileCheck} \M{file} \M{file closing action} \M{rerun warning}
% \end{declcs}
% If you want to add a rerun check, call \cs{RerunFileCheck} right before
% an output file is opened for writing. The macro first remembers
% the current checksum of \meta{file}.
% The file is checked again right before the end of the job.
% Macro \cs{AtVeryEndDocument} of package \xpackage{atveryend}
% is used to place the check after the main aux file is closed
% in |\end{document}|. Before reading the file again, it must be
% closed. Provide the code for closing in argument \meta{file closing action}.
% Do not forget \cs{immediate} before \cs{openout}. Otherwise the
% closing action would be delayed to the next shipout that never happens
% (the last page is already shipped out).
% If the file has changed, \cs{RerunFileCheck} informs the user with
% a warning that the file has changed and says the magic word ``Rerun''.
% If the last argument \meta{rerun warning} is not empty, then
% the rerun sentence is replaced by it. Usually the phrase
% ``to get something right'' is added.
% As example the relevant part of the redefined \cs{makeindex}
% is shown, see package code:
%\begin{quote}
%\begin{verbatim}
%\newwrite\@indexfile
%\RerunFileCheck{\jobname.idx}{%
%  \immediate\closeout\@indexfile
%}{%
%  Rerun LaTeX/makeindex to get index right%
%}%
%\immediate\openout\@indexfile=\jobname.idx %
%\end{verbatim}
%\end{quote}
%
% \StopEventually{
% }
%
% \section{Implementation}
%    \begin{macrocode}
%<*package>
%    \end{macrocode}
%
%    \begin{macrocode}
\begingroup\catcode61\catcode48\catcode32=10\relax%
  \catcode13=5 % ^^M
  \endlinechar=13 %
  \catcode123=1 % {
  \catcode125=2 % }
  \catcode64=11 % @
  \def\x{\endgroup
    \expandafter\edef\csname ReFiCh@AtEnd\endcsname{%
      \endlinechar=\the\endlinechar\relax
      \catcode13=\the\catcode13\relax
      \catcode32=\the\catcode32\relax
      \catcode35=\the\catcode35\relax
      \catcode61=\the\catcode61\relax
      \catcode64=\the\catcode64\relax
      \catcode123=\the\catcode123\relax
      \catcode125=\the\catcode125\relax
    }%
  }%
\x\catcode61\catcode48\catcode32=10\relax%
\catcode13=5 % ^^M
\endlinechar=13 %
\catcode35=6 % #
\catcode64=11 % @
\catcode123=1 % {
\catcode125=2 % }
\def\TMP@EnsureCode#1#2{%
  \edef\ReFiCh@AtEnd{%
    \ReFiCh@AtEnd
    \catcode#1=\the\catcode#1\relax
  }%
  \catcode#1=#2\relax
}
\TMP@EnsureCode{39}{12}% '
\TMP@EnsureCode{40}{12}% (
\TMP@EnsureCode{41}{12}% )
\TMP@EnsureCode{42}{12}% *
\TMP@EnsureCode{44}{12}% ,
\TMP@EnsureCode{46}{12}% .
\TMP@EnsureCode{47}{12}% /
\TMP@EnsureCode{58}{12}% :
\TMP@EnsureCode{59}{12}% ;
\TMP@EnsureCode{60}{12}% <
\TMP@EnsureCode{62}{12}% >
\TMP@EnsureCode{91}{12}% [
\TMP@EnsureCode{93}{12}% ]
\TMP@EnsureCode{96}{12}% `
\edef\ReFiCh@AtEnd{\ReFiCh@AtEnd\noexpand\endinput}
%    \end{macrocode}
%
%    Package identification.
%    \begin{macrocode}
\NeedsTeXFormat{LaTeX2e}
\ProvidesPackage{rerunfilecheck}%
  [2016/05/16 v1.8 Rerun checks for auxiliary files (HO)]
%    \end{macrocode}
%
% \subsection{Options}
%
%    \begin{macrocode}
\RequirePackage{kvoptions}[2010/02/22]
\SetupKeyvalOptions{%
  family=rerunfilecheck,%
  prefix=ReFiCh@%
}
%    \end{macrocode}
%    \begin{macro}{\RerunFileCheckSetup}
%    \begin{macrocode}
\newcommand*{\RerunFileCheckSetup}{%
  \setkeys{rerunfilecheck}%
}
%    \end{macrocode}
%    \end{macro}
%    \begin{macrocode}
\DeclareBoolOption{mainaux}
\DeclareBoolOption{partaux}
\DeclareBoolOption{starttoc}
\DeclareBoolOption{index}
\DeclareBoolOption{glossary}
\define@key{rerunfilecheck}{aux}[true]{%
  \RerunFileCheckSetup{%
    mainaux={#1},%
    partaux={#1},%
    starttoc={#1},%
    index={#1},%
    glossary={#1}%
  }%
}
%    \end{macrocode}
%    \begin{macrocode}
\InputIfFileExists{rerunfilecheck.cfg}{}{}
\ProcessLocalKeyvalOptions*
%    \end{macrocode}
%
%    \begin{macro}{\ReFiCh@DisableOption}
%    \begin{macrocode}
\def\ReFiCh@DisableOption{%
  \DisableKeyvalOption[%
    action=warning,%
    package=rerunfilecheck%
  ]{rerunfilecheck}%
}
%    \end{macrocode}
%    \end{macro}
%
% \subsection{Check for checksum feature}
%
%    \begin{macrocode}
\RequirePackage{infwarerr}[2007/09/09]
\RequirePackage{pdftexcmds}[2009/04/10]
%    \end{macrocode}
%    \begin{macrocode}
\begingroup\expandafter\expandafter\expandafter\endgroup
\expandafter\ifx\csname pdf@filemdfivesum\endcsname\relax
  \@PackageInfoNoLine{rerunfilecheck}{%
    Feature \string\pdfmdfivesum\space is not available\MessageBreak
    (e.g. pdfTeX or LuaTeX with package `pdftexcmds').\MessageBreak
    Therefore file contents cannot be checked efficiently\MessageBreak
    and the loading of the package is aborted%
  }%
  \newcommand*{\RerunFileCheck}[3]{}%
  \renewcommand*{\RerunFileCheckSetup}[1]{}%
  \expandafter\ReFiCh@AtEnd
\fi%
%    \end{macrocode}
%
% \subsection{Standard \xext{aux} files}
%
%    \begin{macrocode}
\ifReFiCh@partaux
  \let\ReFiCh@org@include\@include
  \def\@include#1 {%
    \if@filesw
      \RerunFileCheck{#1.aux}{}{}%
    \fi
    \ReFiCh@org@include{#1} %
  }%
\fi
\ifReFiCh@mainaux
  \AtBeginDocument{%
    \ReFiCh@mainauxfalse
  }%
  \ifReFiCh@mainaux
    \AtEndOfPackage{%
      \RerunFileCheck{\jobname.aux}{}{}%
    }%
  \else
    \if@filesw
      \@PackageWarningNoLine{rerunfilecheck}{%
        Main aux file check is disabled,\MessageBreak
        because the file is already opened.\MessageBreak
        Load the package before \string\begin{document}%
      }%
    \fi
  \fi
\fi
\ifReFiCh@starttoc
  \let\ReFiCh@org@starttoc\@starttoc
  \def\@starttoc#1{%
    \if@filesw
      \RerunFileCheck{\jobname.#1}{%
        \@ifundefined{tf@#1}{%
        }{%
          \immediate\closeout\csname tf@#1\endcsname
        }%
      }{}%
    \fi
    \ReFiCh@org@starttoc{#1}%
  }%
\fi
\ifReFiCh@index
  \ifx\makeindex\@empty
    \@PackageWarningNoLine{rerunfilecheck}{%
      Option `index' ignored,\MessageBreak
      because \string\makeindex\space has already been called%
    }%
  \else
    \def\ReFiCh@temp{%
      \newwrite\@indexfile
      \immediate\openout\@indexfile=\jobname.idx %
      \def\index{%
        \@bsphack
        \begingroup
        \@sanitize
        \@wrindex
      }%
      \typeout{Writing index file \jobname.idx}%
      \let\makeindex\@empty
    }%
    \ifx\ReFiCh@temp\makeindex
      \def\makeindex{%
        \newwrite\@indexfile
        \RerunFileCheck{\jobname.idx}{%
          \immediate\closeout\@indexfile
        }{%
          Rerun LaTeX/makeindex to get index right%
        }%
        \immediate\openout\@indexfile=\jobname.idx %
        \def\index{%
          \@bsphack
          \begingroup
          \@sanitize
          \@wrindex
        }%
        \typeout{Writing index file \jobname.idx}%
        \let\makeindex\@empty
      }%
    \else
      \@PackageInfoNoLine{rerunfilecheck}{%
        Option `index': unsupported version of \string\makeindex
      }%
    \fi
  \fi
\fi
\ifReFiCh@glossary
  \ifx\makeglossary\@empty
    \@PackageWarningNoLine{rerunfilecheck}{%
      Option `glossary' ignored,\MessageBreak
      because \string\makeglossary\space has already been called%
    }%
  \else
    \def\ReFiCh@temp{%
      \newwrite\@glossaryfile
      \immediate\openout\@glossaryfile=\jobname.glo %
      \def\glossary{%
        \@bsphack
        \begingroup
        \@sanitize
        \@wrglossary
      }%
      \typeout{Writing glossary file \jobname.glo }%
      \let\makeglossary\@empty
    }%
    \ifx\ReFiCh@temp\makeglossary
      \def\ReFiCh@temp{%
        \newwrite\@glossaryfile
        \RerunFileCheck{\jobname.glo}{%
          \immediate\closeout\@glossaryfile
        }{%
          Rerun LaTeX/makeindex to get glossary right%
        }%
        \immediate\openout\@glossaryfile=\jobname.glo %
        \def\glossary{%
          \@bsphack
          \begingroup
          \@sanitize
          \@wrglossary
        }%
        \typeout{Writing glossary file \jobname.glo}%
        \let\makeglossary\@empty
      }%
    \else
      \@PackageInfoNoLine{rerunfilecheck}{%
        Option `glossary': unsupported version of \string\makeglossary
      }%
    \fi
  \fi
\fi
\ReFiCh@DisableOption{mainaux}
\ReFiCh@DisableOption{partaux}
\ReFiCh@DisableOption{starttoc}
\ReFiCh@DisableOption{index}
\ReFiCh@DisableOption{glossary}
\ReFiCh@DisableOption{aux}
%    \end{macrocode}
%
% \subsection{Rerun check}
%
%    \begin{macrocode}
\RequirePackage{atveryend}[2016/05/16]
\RequirePackage{uniquecounter}[2009/12/18]
%    \end{macrocode}
%
%    \begin{macro}{\ReFiCh@CheckSum}
%    \begin{macrocode}
\begingroup\expandafter\expandafter\expandafter\endgroup
\expandafter\ifx\csname pdf@filesize\endcsname\relax
  \def\ReFiCh@CheckSum{%
    \pdf@filemdfivesum
  }%
\else
  \def\ReFiCh@CheckSum#1{%
    \pdf@filemdfivesum{#1}%
    \ReFiCh@Separator
    \pdf@filesize{#1}%
  }%
\fi
%    \end{macrocode}
%    \end{macro}
%    \begin{macro}{\ReFiCh@NoFile}
%    \begin{macrocode}
\def\ReFiCh@Separator{;}
%    \end{macrocode}
%    \end{macro}
%    \begin{macro}{\ReFiCh@NoFile}
%    \begin{macrocode}
\def\ReFiCh@NoFile{<no file>}
%    \end{macrocode}
%    \end{macro}
%
%    \begin{macrocode}
\UniqueCounterNew{rerunfilecheck}
%    \end{macrocode}
%
%    \begin{macro}{\RerunFileCheck}
%    \begin{macrocode}
\newcommand*{\RerunFileCheck}{%
  \UniqueCounterCall{rerunfilecheck}\ReFiCh@RerunFileCheck
}
%    \end{macrocode}
%    \end{macro}
%    \begin{macro}{\ReFiCh@RerunFileCheck}
%    \begin{macrocode}
\def\ReFiCh@RerunFileCheck#1{%
  \expandafter\ReFiCh@@RerunFileCheck\csname ReFiCh@#1\endcsname
}
%    \end{macrocode}
%    \end{macro}
%    \begin{macro}{\ReFiCh@Check}
%    \begin{macrocode}
\def\ReFiCh@Check#1#2#3{%
%  \IfFileExists{#3}{%
    #1\edef#2{\ReFiCh@CheckSum{#3}}%
    \ifx#2\ReFiCh@Separator
      #1\let#2\ReFiCh@NoFile
    \fi
%  }{%
%    #1\let#2\ReFiCh@NoFile
%  }%
}
%    \end{macrocode}
%    \end{macro}
%    \begin{macro}{\ReFiCh@@RerunFileCheck}
%    \begin{macrocode}
\def\ReFiCh@@RerunFileCheck#1#2#3#4{%
  \ReFiCh@Check\global#1{#2}%
  \AtEndAfterFileList{%
    \begingroup
      #3%
      \ReFiCh@Check{}\x{#2}%
      \ifx#1\x
        \@PackageInfoNoLine{rerunfilecheck}{%
          File `#2' has not changed.\MessageBreak
          Checksum: \x
        }%
      \else
        \ifnum
          \ReFiCh@IsAux#2\relax.aux\relax\@nil
          \ifx#1\ReFiCh@NoFile 1\else 0\fi
          \ifx\x\ReFiCh@AuxEmptyUnix 1%
          \else
            \ifx\x\ReFiCh@AuxEmptyDos 1\fi
          \fi
        =111 %
          \@PackageInfoNoLine{rerunfilecheck}{%
            File `#2' is empty .aux file.\MessageBreak
            Before: #1\MessageBreak
            After: \space\x
          }%
        \else
          \@PackageWarningNoLine{rerunfilecheck}{%
            File `#2' has changed.%
            \ifx\\#4\\%
              \space Rerun%
            \else
              \MessageBreak
              #4%
            \fi
          }%
          \@PackageInfoNoLine{rerunfilecheck}{%
            Checksums for `#2':\MessageBreak
            Before: #1\MessageBreak
            After: \space\x
          }%
        \fi
      \fi
    \endgroup
  }%
}
%    \end{macrocode}
%    \end{macro}
%    \begin{macrocode}
\def\ReFiCh@IsAux#1.aux\relax#2\@nil{%
  \ifx\hbox#2\hbox
    0%
  \else
    1%
  \fi
}
%    \end{macrocode}
%    \begin{macrocode}
\def\ReFiCh@AuxEmptyUnix{A94A2480D3289E625EEA47CD1B285758;8}%
\@onelevel@sanitize\ReFiCh@AuxEmptyUnix
%    \end{macrocode}
%    \begin{macrocode}
\def\ReFiCh@AuxEmptyDos{A62A15ECE803E2EBE94952FCC9933BC0;9}%
\@onelevel@sanitize\ReFiCh@AuxEmptyDos
%    \end{macrocode}
%
%    \begin{macrocode}
\ReFiCh@AtEnd%
%</package>
%    \end{macrocode}
%
% \section{Test}
%
%    \begin{macrocode}
%<*test1>
\def\LoadCommand{\RequirePackage{rerunfilecheck}[2016/05/16]}
%</test1>
%    \end{macrocode}
% \subsection{Catcode checks for loading}
%
%    \begin{macrocode}
%<*test1>
%    \end{macrocode}
%    \begin{macrocode}
\catcode`\{=1 %
\catcode`\}=2 %
\catcode`\#=6 %
\catcode`\@=11 %
\expandafter\ifx\csname count@\endcsname\relax
  \countdef\count@=255 %
\fi
\expandafter\ifx\csname @gobble\endcsname\relax
  \long\def\@gobble#1{}%
\fi
\expandafter\ifx\csname @firstofone\endcsname\relax
  \long\def\@firstofone#1{#1}%
\fi
\expandafter\ifx\csname loop\endcsname\relax
  \expandafter\@firstofone
\else
  \expandafter\@gobble
\fi
{%
  \def\loop#1\repeat{%
    \def\body{#1}%
    \iterate
  }%
  \def\iterate{%
    \body
      \let\next\iterate
    \else
      \let\next\relax
    \fi
    \next
  }%
  \let\repeat=\fi
}%
\def\RestoreCatcodes{}
\count@=0 %
\loop
  \edef\RestoreCatcodes{%
    \RestoreCatcodes
    \catcode\the\count@=\the\catcode\count@\relax
  }%
\ifnum\count@<255 %
  \advance\count@ 1 %
\repeat

\def\RangeCatcodeInvalid#1#2{%
  \count@=#1\relax
  \loop
    \catcode\count@=15 %
  \ifnum\count@<#2\relax
    \advance\count@ 1 %
  \repeat
}
\def\RangeCatcodeCheck#1#2#3{%
  \count@=#1\relax
  \loop
    \ifnum#3=\catcode\count@
    \else
      \errmessage{%
        Character \the\count@\space
        with wrong catcode \the\catcode\count@\space
        instead of \number#3%
      }%
    \fi
  \ifnum\count@<#2\relax
    \advance\count@ 1 %
  \repeat
}
\def\space{ }
\expandafter\ifx\csname LoadCommand\endcsname\relax
  \def\LoadCommand{\input rerunfilecheck.sty\relax}%
\fi
\def\Test{%
  \RangeCatcodeInvalid{0}{47}%
  \RangeCatcodeInvalid{58}{64}%
  \RangeCatcodeInvalid{91}{96}%
  \RangeCatcodeInvalid{123}{255}%
  \catcode`\@=12 %
  \catcode`\\=0 %
  \catcode`\%=14 %
  \LoadCommand
  \RangeCatcodeCheck{0}{36}{15}%
  \RangeCatcodeCheck{37}{37}{14}%
  \RangeCatcodeCheck{38}{47}{15}%
  \RangeCatcodeCheck{48}{57}{12}%
  \RangeCatcodeCheck{58}{63}{15}%
  \RangeCatcodeCheck{64}{64}{12}%
  \RangeCatcodeCheck{65}{90}{11}%
  \RangeCatcodeCheck{91}{91}{15}%
  \RangeCatcodeCheck{92}{92}{0}%
  \RangeCatcodeCheck{93}{96}{15}%
  \RangeCatcodeCheck{97}{122}{11}%
  \RangeCatcodeCheck{123}{255}{15}%
  \RestoreCatcodes
}
\Test
\csname @@end\endcsname
\end
%    \end{macrocode}
%    \begin{macrocode}
%</test1>
%    \end{macrocode}
%
% \section{Installation}
%
% \subsection{Download}
%
% \paragraph{Package.} This package is available on
% CTAN\footnote{\url{http://ctan.org/pkg/rerunfilecheck}}:
% \begin{description}
% \item[\CTAN{macros/latex/contrib/oberdiek/rerunfilecheck.dtx}] The source file.
% \item[\CTAN{macros/latex/contrib/oberdiek/rerunfilecheck.pdf}] Documentation.
% \end{description}
%
%
% \paragraph{Bundle.} All the packages of the bundle `oberdiek'
% are also available in a TDS compliant ZIP archive. There
% the packages are already unpacked and the documentation files
% are generated. The files and directories obey the TDS standard.
% \begin{description}
% \item[\CTAN{install/macros/latex/contrib/oberdiek.tds.zip}]
% \end{description}
% \emph{TDS} refers to the standard ``A Directory Structure
% for \TeX\ Files'' (\CTAN{tds/tds.pdf}). Directories
% with \xfile{texmf} in their name are usually organized this way.
%
% \subsection{Bundle installation}
%
% \paragraph{Unpacking.} Unpack the \xfile{oberdiek.tds.zip} in the
% TDS tree (also known as \xfile{texmf} tree) of your choice.
% Example (linux):
% \begin{quote}
%   |unzip oberdiek.tds.zip -d ~/texmf|
% \end{quote}
%
% \paragraph{Script installation.}
% Check the directory \xfile{TDS:scripts/oberdiek/} for
% scripts that need further installation steps.
% Package \xpackage{attachfile2} comes with the Perl script
% \xfile{pdfatfi.pl} that should be installed in such a way
% that it can be called as \texttt{pdfatfi}.
% Example (linux):
% \begin{quote}
%   |chmod +x scripts/oberdiek/pdfatfi.pl|\\
%   |cp scripts/oberdiek/pdfatfi.pl /usr/local/bin/|
% \end{quote}
%
% \subsection{Package installation}
%
% \paragraph{Unpacking.} The \xfile{.dtx} file is a self-extracting
% \docstrip\ archive. The files are extracted by running the
% \xfile{.dtx} through \plainTeX:
% \begin{quote}
%   \verb|tex rerunfilecheck.dtx|
% \end{quote}
%
% \paragraph{TDS.} Now the different files must be moved into
% the different directories in your installation TDS tree
% (also known as \xfile{texmf} tree):
% \begin{quote}
% \def\t{^^A
% \begin{tabular}{@{}>{\ttfamily}l@{ $\rightarrow$ }>{\ttfamily}l@{}}
%   rerunfilecheck.sty & tex/latex/oberdiek/rerunfilecheck.sty\\
%   rerunfilecheck.pdf & doc/latex/oberdiek/rerunfilecheck.pdf\\
%   rerunfilecheck-example.cfg & doc/latex/oberdiek/rerunfilecheck-example.cfg\\
%   test/rerunfilecheck-test1.tex & doc/latex/oberdiek/test/rerunfilecheck-test1.tex\\
%   rerunfilecheck.dtx & source/latex/oberdiek/rerunfilecheck.dtx\\
% \end{tabular}^^A
% }^^A
% \sbox0{\t}^^A
% \ifdim\wd0>\linewidth
%   \begingroup
%     \advance\linewidth by\leftmargin
%     \advance\linewidth by\rightmargin
%   \edef\x{\endgroup
%     \def\noexpand\lw{\the\linewidth}^^A
%   }\x
%   \def\lwbox{^^A
%     \leavevmode
%     \hbox to \linewidth{^^A
%       \kern-\leftmargin\relax
%       \hss
%       \usebox0
%       \hss
%       \kern-\rightmargin\relax
%     }^^A
%   }^^A
%   \ifdim\wd0>\lw
%     \sbox0{\small\t}^^A
%     \ifdim\wd0>\linewidth
%       \ifdim\wd0>\lw
%         \sbox0{\footnotesize\t}^^A
%         \ifdim\wd0>\linewidth
%           \ifdim\wd0>\lw
%             \sbox0{\scriptsize\t}^^A
%             \ifdim\wd0>\linewidth
%               \ifdim\wd0>\lw
%                 \sbox0{\tiny\t}^^A
%                 \ifdim\wd0>\linewidth
%                   \lwbox
%                 \else
%                   \usebox0
%                 \fi
%               \else
%                 \lwbox
%               \fi
%             \else
%               \usebox0
%             \fi
%           \else
%             \lwbox
%           \fi
%         \else
%           \usebox0
%         \fi
%       \else
%         \lwbox
%       \fi
%     \else
%       \usebox0
%     \fi
%   \else
%     \lwbox
%   \fi
% \else
%   \usebox0
% \fi
% \end{quote}
% If you have a \xfile{docstrip.cfg} that configures and enables \docstrip's
% TDS installing feature, then some files can already be in the right
% place, see the documentation of \docstrip.
%
% \subsection{Refresh file name databases}
%
% If your \TeX~distribution
% (\teTeX, \mikTeX, \dots) relies on file name databases, you must refresh
% these. For example, \teTeX\ users run \verb|texhash| or
% \verb|mktexlsr|.
%
% \subsection{Some details for the interested}
%
% \paragraph{Attached source.}
%
% The PDF documentation on CTAN also includes the
% \xfile{.dtx} source file. It can be extracted by
% AcrobatReader 6 or higher. Another option is \textsf{pdftk},
% e.g. unpack the file into the current directory:
% \begin{quote}
%   \verb|pdftk rerunfilecheck.pdf unpack_files output .|
% \end{quote}
%
% \paragraph{Unpacking with \LaTeX.}
% The \xfile{.dtx} chooses its action depending on the format:
% \begin{description}
% \item[\plainTeX:] Run \docstrip\ and extract the files.
% \item[\LaTeX:] Generate the documentation.
% \end{description}
% If you insist on using \LaTeX\ for \docstrip\ (really,
% \docstrip\ does not need \LaTeX), then inform the autodetect routine
% about your intention:
% \begin{quote}
%   \verb|latex \let\install=y\input{rerunfilecheck.dtx}|
% \end{quote}
% Do not forget to quote the argument according to the demands
% of your shell.
%
% \paragraph{Generating the documentation.}
% You can use both the \xfile{.dtx} or the \xfile{.drv} to generate
% the documentation. The process can be configured by the
% configuration file \xfile{ltxdoc.cfg}. For instance, put this
% line into this file, if you want to have A4 as paper format:
% \begin{quote}
%   \verb|\PassOptionsToClass{a4paper}{article}|
% \end{quote}
% An example follows how to generate the
% documentation with pdf\LaTeX:
% \begin{quote}
%\begin{verbatim}
%pdflatex rerunfilecheck.dtx
%makeindex -s gind.ist rerunfilecheck.idx
%pdflatex rerunfilecheck.dtx
%makeindex -s gind.ist rerunfilecheck.idx
%pdflatex rerunfilecheck.dtx
%\end{verbatim}
% \end{quote}
%
% \section{Catalogue}
%
% The following XML file can be used as source for the
% \href{http://mirror.ctan.org/help/Catalogue/catalogue.html}{\TeX\ Catalogue}.
% The elements \texttt{caption} and \texttt{description} are imported
% from the original XML file from the Catalogue.
% The name of the XML file in the Catalogue is \xfile{rerunfilecheck.xml}.
%    \begin{macrocode}
%<*catalogue>
<?xml version='1.0' encoding='us-ascii'?>
<!DOCTYPE entry SYSTEM 'catalogue.dtd'>
<entry datestamp='$Date$' modifier='$Author$' id='rerunfilecheck'>
  <name>rerunfilecheck</name>
  <caption>Checksum based rerun checks on auxiliary files.</caption>
  <authorref id='auth:oberdiek'/>
  <copyright owner='Heiko Oberdiek' year='2009-2011'/>
  <license type='lppl1.3'/>
  <version number='1.8'/>
  <description>
    The package provides additional rerun warnings if some
    auxiliary files have changed. It is based on MD5 checksum,
    provided by pdfTeX.
    <p/>
    The package is part of the <xref refid='oberdiek'>oberdiek</xref> bundle.
  </description>
  <documentation details='Package documentation'
      href='ctan:/macros/latex/contrib/oberdiek/rerunfilecheck.pdf'/>
  <ctan file='true' path='/macros/latex/contrib/oberdiek/rerunfilecheck.dtx'/>
  <miktex location='oberdiek'/>
  <texlive location='oberdiek'/>
  <install path='/macros/latex/contrib/oberdiek/oberdiek.tds.zip'/>
</entry>
%</catalogue>
%    \end{macrocode}
%
% \begin{History}
%   \begin{Version}{2009/12/10 v1.0}
%   \item
%     The first version.
%   \end{Version}
%   \begin{Version}{2009/12/12 v1.1}
%   \item
%     Short info shortened.
%   \end{Version}
%   \begin{Version}{2009/12/18 v1.2}
%   \item
%     Required date for package \xpackage{uniquecounter} updated
%     because of bug in this package.
%   \end{Version}
%   \begin{Version}{2010/01/25 v1.3}
%   \item
%     Moved from \texttt{TDS:*/generic/*} to \texttt{TDS:*/latex/*}.
%   \end{Version}
%   \begin{Version}{2010/02/22 v1.4}
%   \item
%     The options of this package are recognized only if they are
%     package options. Global options are ignored. This avoids
%     name clashes with class and other package options
%     (for example, class option `index=totoc').
%   \end{Version}
%   \begin{Version}{2010/03/15 v1.5}
%   \item
%     Call of \cs{pdffivesum} is wrapped in \cs{IfFileExists} to avoid
%     calls of \textsf{mktextex} if this feature is enabled.
%     However \cs{IfFileExists} has file name limitations.
%   \end{Version}
%   \begin{Version}{2010/03/16 v1.6}
%   \item
%     Reverted to version 1.4 and \cs{IfFileExists} wrapper of
%     version 1.5 is removed.
%   \end{Version}
%   \begin{Version}{2011/04/15 v1.7}
%   \item
%     Using \cs{AtEndAfterFileList} of package \xpackage{atveryend}
%     2011/04/15 v1.6 instead of \cs{AtVeryEndDocument}.
%   \end{Version}
%   \begin{Version}{2016/05/16 v1.8}
%   \item
%     Documentation updates.
%   \end{Version}
% \end{History}
%
% \PrintIndex
%
% \Finale
\endinput

%        (quote the arguments according to the demands of your shell)
%
% Documentation:
%    (a) If rerunfilecheck.drv is present:
%           latex rerunfilecheck.drv
%    (b) Without rerunfilecheck.drv:
%           latex rerunfilecheck.dtx; ...
%    The class ltxdoc loads the configuration file ltxdoc.cfg
%    if available. Here you can specify further options, e.g.
%    use A4 as paper format:
%       \PassOptionsToClass{a4paper}{article}
%
%    Programm calls to get the documentation (example):
%       pdflatex rerunfilecheck.dtx
%       makeindex -s gind.ist rerunfilecheck.idx
%       pdflatex rerunfilecheck.dtx
%       makeindex -s gind.ist rerunfilecheck.idx
%       pdflatex rerunfilecheck.dtx
%
% Installation:
%    TDS:tex/latex/oberdiek/rerunfilecheck.sty
%    TDS:doc/latex/oberdiek/rerunfilecheck.pdf
%    TDS:doc/latex/oberdiek/rerunfilecheck-example.cfg
%    TDS:doc/latex/oberdiek/test/rerunfilecheck-test1.tex
%    TDS:source/latex/oberdiek/rerunfilecheck.dtx
%
%<*ignore>
\begingroup
  \catcode123=1 %
  \catcode125=2 %
  \def\x{LaTeX2e}%
\expandafter\endgroup
\ifcase 0\ifx\install y1\fi\expandafter
         \ifx\csname processbatchFile\endcsname\relax\else1\fi
         \ifx\fmtname\x\else 1\fi\relax
\else\csname fi\endcsname
%</ignore>
%<*install>
\input docstrip.tex
\Msg{************************************************************************}
\Msg{* Installation}
\Msg{* Package: rerunfilecheck 2016/05/16 v1.8 Rerun checks for auxiliary files (HO)}
\Msg{************************************************************************}

\keepsilent
\askforoverwritefalse

\let\MetaPrefix\relax
\preamble

This is a generated file.

Project: rerunfilecheck
Version: 2016/05/16 v1.8

Copyright (C) 2009-2011 by
   Heiko Oberdiek <heiko.oberdiek at googlemail.com>

This work may be distributed and/or modified under the
conditions of the LaTeX Project Public License, either
version 1.3c of this license or (at your option) any later
version. This version of this license is in
   http://www.latex-project.org/lppl/lppl-1-3c.txt
and the latest version of this license is in
   http://www.latex-project.org/lppl.txt
and version 1.3 or later is part of all distributions of
LaTeX version 2005/12/01 or later.

This work has the LPPL maintenance status "maintained".

This Current Maintainer of this work is Heiko Oberdiek.

This work consists of the main source file rerunfilecheck.dtx
and the derived files
   rerunfilecheck.sty, rerunfilecheck.pdf, rerunfilecheck.ins,
   rerunfilecheck.drv, rerunfilecheck-example.cfg,
   rerunfilecheck-test1.tex.

\endpreamble
\let\MetaPrefix\DoubleperCent

\generate{%
  \file{rerunfilecheck.ins}{\from{rerunfilecheck.dtx}{install}}%
  \file{rerunfilecheck.drv}{\from{rerunfilecheck.dtx}{driver}}%
  \usedir{tex/latex/oberdiek}%
  \file{rerunfilecheck.sty}{\from{rerunfilecheck.dtx}{package}}%
%  \usedir{doc/latex/oberdiek/test}%
%  \file{rerunfilecheck-test1.tex}{\from{rerunfilecheck.dtx}{test1}}%
  \nopreamble
  \nopostamble
  \usedir{doc/latex/oberdiek}%
  \file{rerunfilecheck-example.cfg}{\from{rerunfilecheck.dtx}{cfg}}%
  \usedir{source/latex/oberdiek/catalogue}%
  \file{rerunfilecheck.xml}{\from{rerunfilecheck.dtx}{catalogue}}%
}

\catcode32=13\relax% active space
\let =\space%
\Msg{************************************************************************}
\Msg{*}
\Msg{* To finish the installation you have to move the following}
\Msg{* file into a directory searched by TeX:}
\Msg{*}
\Msg{*     rerunfilecheck.sty}
\Msg{*}
\Msg{* To produce the documentation run the file `rerunfilecheck.drv'}
\Msg{* through LaTeX.}
\Msg{*}
\Msg{* Happy TeXing!}
\Msg{*}
\Msg{************************************************************************}

\endbatchfile
%</install>
%<*ignore>
\fi
%</ignore>
%<*driver>
\NeedsTeXFormat{LaTeX2e}
\ProvidesFile{rerunfilecheck.drv}%
  [2016/05/16 v1.8 Rerun checks for auxiliary files (HO)]%
\documentclass{ltxdoc}
\usepackage{holtxdoc}[2011/11/22]
\begin{document}
  \DocInput{rerunfilecheck.dtx}%
\end{document}
%</driver>
% \fi
%
%
% \CharacterTable
%  {Upper-case    \A\B\C\D\E\F\G\H\I\J\K\L\M\N\O\P\Q\R\S\T\U\V\W\X\Y\Z
%   Lower-case    \a\b\c\d\e\f\g\h\i\j\k\l\m\n\o\p\q\r\s\t\u\v\w\x\y\z
%   Digits        \0\1\2\3\4\5\6\7\8\9
%   Exclamation   \!     Double quote  \"     Hash (number) \#
%   Dollar        \$     Percent       \%     Ampersand     \&
%   Acute accent  \'     Left paren    \(     Right paren   \)
%   Asterisk      \*     Plus          \+     Comma         \,
%   Minus         \-     Point         \.     Solidus       \/
%   Colon         \:     Semicolon     \;     Less than     \<
%   Equals        \=     Greater than  \>     Question mark \?
%   Commercial at \@     Left bracket  \[     Backslash     \\
%   Right bracket \]     Circumflex    \^     Underscore    \_
%   Grave accent  \`     Left brace    \{     Vertical bar  \|
%   Right brace   \}     Tilde         \~}
%
% \GetFileInfo{rerunfilecheck.drv}
%
% \title{The \xpackage{rerunfilecheck} package}
% \date{2016/05/16 v1.8}
% \author{Heiko Oberdiek\thanks
% {Please report any issues at https://github.com/ho-tex/oberdiek/issues}\\
% \xemail{heiko.oberdiek at googlemail.com}}
%
% \maketitle
%
% \begin{abstract}
% The package provides additional rerun warnings if some
% auxiliary files have changed. It is based on MD5 checksum,
% provided by \pdfTeX.
% \end{abstract}
%
% \tableofcontents
%
% \section{Documentation}
%
% \LaTeX\ informs the user, when to run \LaTeX\ again, if the
% references have changed. It has the old references from the
% first reading of the \xext{aux} files already in memory, thus
% it compares them with the new version of the \xext{aux} file
% at the end of the document. However this rerun warnings are
% not given for the table of contents and other data stored
% in the \xext{aux} files or other auxiliary files.
% Usually many of these data as the table of contents is not keep
% in memory. If someone wants to detect changes, he has either
% to keep the data in memory. This does not scale well with huge
% documents. Or he copies the file before they are changed. Slow
% I/O operations cost time.
%
% Since version 1.30.0 \pdfTeX\ provides \cs{pdfmdfivesum}
% and \cs{pdffilesize}.
% These features are also available in \LuaTeX, provided by package
% \xpackage{pdftexcmds}.
% Thus this package \xpackage{rerunfilecheck} uses these features to
% detect file changes. This saves the packages from keeping
% the whole files in memory or in file copies.
% The drawback are different files with the same size and the
% same MD5 checksum (seldom, hopefully).
%
% \subsection{Options}
%
% All options are key value options of boolean type.
% No option or |true| turns an option on, |false| disables
% an option.
% \begin{description}
% \item[\xoption{mainaux}:]
%   Check the main \xext{aux} file.
% \item[\xoption{partaux}:]
%   Check the \xext{aux} files from \cs{include} files.
% \item[\xoption{starttoc}:]
%   Add the rerun checks in \cs{@starttoc} that is called by
%   \cs{tableofcontents}, \cs{listoffigures}, \dots
% \item[\xoption{index}, \xoption{glossary}:]
%   \LaTeX's original \cs{makeindex} and \cs{makeglossary} are
%   redefined to add the rerun checks. The options do not have
%   an effect, if \cs{makeindex}/\cs{makeglossary} are already
%   called or if a package or class had redefined or will
%   redefine them.
% \item[\xoption{aux}:]
%   This option turns all previous options on or off.
%   ``aux'' means auxiliary file.
% \end{description}
% The default for the options is |false|, because
% some internals must be redefined to insert the
% rerun checks.
% The options can be set in \cs{usepackage} or
% the configuration file \xfile{rerunfilecheck.cfg}.
% Global options are ignored (since 1.4).
%
% \begin{declcs}{RerunFileCheckSetup} \M{key value list}
% \end{declcs}
% Options can also be set using \cs{RerunFileCheckSetup}.
% Currently all options are disabled after the package
% is loaded. Thus \cs{RerunFileCheckSetup} makes sense in
% the configuration file only.
%
% Example for the configuration file:
%    \begin{macrocode}
%<*cfg>
\ProvidesFile{rerunfilecheck.cfg}[2016/05/16 Default configuration]%
\RerunFileCheckSetup{aux}
%</cfg>
%    \end{macrocode}
%
% \subsection{Interface for class/package authors}
%
% \begin{declcs}{RerunFileCheck} \M{file} \M{file closing action} \M{rerun warning}
% \end{declcs}
% If you want to add a rerun check, call \cs{RerunFileCheck} right before
% an output file is opened for writing. The macro first remembers
% the current checksum of \meta{file}.
% The file is checked again right before the end of the job.
% Macro \cs{AtVeryEndDocument} of package \xpackage{atveryend}
% is used to place the check after the main aux file is closed
% in |\end{document}|. Before reading the file again, it must be
% closed. Provide the code for closing in argument \meta{file closing action}.
% Do not forget \cs{immediate} before \cs{openout}. Otherwise the
% closing action would be delayed to the next shipout that never happens
% (the last page is already shipped out).
% If the file has changed, \cs{RerunFileCheck} informs the user with
% a warning that the file has changed and says the magic word ``Rerun''.
% If the last argument \meta{rerun warning} is not empty, then
% the rerun sentence is replaced by it. Usually the phrase
% ``to get something right'' is added.
% As example the relevant part of the redefined \cs{makeindex}
% is shown, see package code:
%\begin{quote}
%\begin{verbatim}
%\newwrite\@indexfile
%\RerunFileCheck{\jobname.idx}{%
%  \immediate\closeout\@indexfile
%}{%
%  Rerun LaTeX/makeindex to get index right%
%}%
%\immediate\openout\@indexfile=\jobname.idx %
%\end{verbatim}
%\end{quote}
%
% \StopEventually{
% }
%
% \section{Implementation}
%    \begin{macrocode}
%<*package>
%    \end{macrocode}
%
%    \begin{macrocode}
\begingroup\catcode61\catcode48\catcode32=10\relax%
  \catcode13=5 % ^^M
  \endlinechar=13 %
  \catcode123=1 % {
  \catcode125=2 % }
  \catcode64=11 % @
  \def\x{\endgroup
    \expandafter\edef\csname ReFiCh@AtEnd\endcsname{%
      \endlinechar=\the\endlinechar\relax
      \catcode13=\the\catcode13\relax
      \catcode32=\the\catcode32\relax
      \catcode35=\the\catcode35\relax
      \catcode61=\the\catcode61\relax
      \catcode64=\the\catcode64\relax
      \catcode123=\the\catcode123\relax
      \catcode125=\the\catcode125\relax
    }%
  }%
\x\catcode61\catcode48\catcode32=10\relax%
\catcode13=5 % ^^M
\endlinechar=13 %
\catcode35=6 % #
\catcode64=11 % @
\catcode123=1 % {
\catcode125=2 % }
\def\TMP@EnsureCode#1#2{%
  \edef\ReFiCh@AtEnd{%
    \ReFiCh@AtEnd
    \catcode#1=\the\catcode#1\relax
  }%
  \catcode#1=#2\relax
}
\TMP@EnsureCode{39}{12}% '
\TMP@EnsureCode{40}{12}% (
\TMP@EnsureCode{41}{12}% )
\TMP@EnsureCode{42}{12}% *
\TMP@EnsureCode{44}{12}% ,
\TMP@EnsureCode{46}{12}% .
\TMP@EnsureCode{47}{12}% /
\TMP@EnsureCode{58}{12}% :
\TMP@EnsureCode{59}{12}% ;
\TMP@EnsureCode{60}{12}% <
\TMP@EnsureCode{62}{12}% >
\TMP@EnsureCode{91}{12}% [
\TMP@EnsureCode{93}{12}% ]
\TMP@EnsureCode{96}{12}% `
\edef\ReFiCh@AtEnd{\ReFiCh@AtEnd\noexpand\endinput}
%    \end{macrocode}
%
%    Package identification.
%    \begin{macrocode}
\NeedsTeXFormat{LaTeX2e}
\ProvidesPackage{rerunfilecheck}%
  [2016/05/16 v1.8 Rerun checks for auxiliary files (HO)]
%    \end{macrocode}
%
% \subsection{Options}
%
%    \begin{macrocode}
\RequirePackage{kvoptions}[2010/02/22]
\SetupKeyvalOptions{%
  family=rerunfilecheck,%
  prefix=ReFiCh@%
}
%    \end{macrocode}
%    \begin{macro}{\RerunFileCheckSetup}
%    \begin{macrocode}
\newcommand*{\RerunFileCheckSetup}{%
  \setkeys{rerunfilecheck}%
}
%    \end{macrocode}
%    \end{macro}
%    \begin{macrocode}
\DeclareBoolOption{mainaux}
\DeclareBoolOption{partaux}
\DeclareBoolOption{starttoc}
\DeclareBoolOption{index}
\DeclareBoolOption{glossary}
\define@key{rerunfilecheck}{aux}[true]{%
  \RerunFileCheckSetup{%
    mainaux={#1},%
    partaux={#1},%
    starttoc={#1},%
    index={#1},%
    glossary={#1}%
  }%
}
%    \end{macrocode}
%    \begin{macrocode}
\InputIfFileExists{rerunfilecheck.cfg}{}{}
\ProcessLocalKeyvalOptions*
%    \end{macrocode}
%
%    \begin{macro}{\ReFiCh@DisableOption}
%    \begin{macrocode}
\def\ReFiCh@DisableOption{%
  \DisableKeyvalOption[%
    action=warning,%
    package=rerunfilecheck%
  ]{rerunfilecheck}%
}
%    \end{macrocode}
%    \end{macro}
%
% \subsection{Check for checksum feature}
%
%    \begin{macrocode}
\RequirePackage{infwarerr}[2007/09/09]
\RequirePackage{pdftexcmds}[2009/04/10]
%    \end{macrocode}
%    \begin{macrocode}
\begingroup\expandafter\expandafter\expandafter\endgroup
\expandafter\ifx\csname pdf@filemdfivesum\endcsname\relax
  \@PackageInfoNoLine{rerunfilecheck}{%
    Feature \string\pdfmdfivesum\space is not available\MessageBreak
    (e.g. pdfTeX or LuaTeX with package `pdftexcmds').\MessageBreak
    Therefore file contents cannot be checked efficiently\MessageBreak
    and the loading of the package is aborted%
  }%
  \newcommand*{\RerunFileCheck}[3]{}%
  \renewcommand*{\RerunFileCheckSetup}[1]{}%
  \expandafter\ReFiCh@AtEnd
\fi%
%    \end{macrocode}
%
% \subsection{Standard \xext{aux} files}
%
%    \begin{macrocode}
\ifReFiCh@partaux
  \let\ReFiCh@org@include\@include
  \def\@include#1 {%
    \if@filesw
      \RerunFileCheck{#1.aux}{}{}%
    \fi
    \ReFiCh@org@include{#1} %
  }%
\fi
\ifReFiCh@mainaux
  \AtBeginDocument{%
    \ReFiCh@mainauxfalse
  }%
  \ifReFiCh@mainaux
    \AtEndOfPackage{%
      \RerunFileCheck{\jobname.aux}{}{}%
    }%
  \else
    \if@filesw
      \@PackageWarningNoLine{rerunfilecheck}{%
        Main aux file check is disabled,\MessageBreak
        because the file is already opened.\MessageBreak
        Load the package before \string\begin{document}%
      }%
    \fi
  \fi
\fi
\ifReFiCh@starttoc
  \let\ReFiCh@org@starttoc\@starttoc
  \def\@starttoc#1{%
    \if@filesw
      \RerunFileCheck{\jobname.#1}{%
        \@ifundefined{tf@#1}{%
        }{%
          \immediate\closeout\csname tf@#1\endcsname
        }%
      }{}%
    \fi
    \ReFiCh@org@starttoc{#1}%
  }%
\fi
\ifReFiCh@index
  \ifx\makeindex\@empty
    \@PackageWarningNoLine{rerunfilecheck}{%
      Option `index' ignored,\MessageBreak
      because \string\makeindex\space has already been called%
    }%
  \else
    \def\ReFiCh@temp{%
      \newwrite\@indexfile
      \immediate\openout\@indexfile=\jobname.idx %
      \def\index{%
        \@bsphack
        \begingroup
        \@sanitize
        \@wrindex
      }%
      \typeout{Writing index file \jobname.idx}%
      \let\makeindex\@empty
    }%
    \ifx\ReFiCh@temp\makeindex
      \def\makeindex{%
        \newwrite\@indexfile
        \RerunFileCheck{\jobname.idx}{%
          \immediate\closeout\@indexfile
        }{%
          Rerun LaTeX/makeindex to get index right%
        }%
        \immediate\openout\@indexfile=\jobname.idx %
        \def\index{%
          \@bsphack
          \begingroup
          \@sanitize
          \@wrindex
        }%
        \typeout{Writing index file \jobname.idx}%
        \let\makeindex\@empty
      }%
    \else
      \@PackageInfoNoLine{rerunfilecheck}{%
        Option `index': unsupported version of \string\makeindex
      }%
    \fi
  \fi
\fi
\ifReFiCh@glossary
  \ifx\makeglossary\@empty
    \@PackageWarningNoLine{rerunfilecheck}{%
      Option `glossary' ignored,\MessageBreak
      because \string\makeglossary\space has already been called%
    }%
  \else
    \def\ReFiCh@temp{%
      \newwrite\@glossaryfile
      \immediate\openout\@glossaryfile=\jobname.glo %
      \def\glossary{%
        \@bsphack
        \begingroup
        \@sanitize
        \@wrglossary
      }%
      \typeout{Writing glossary file \jobname.glo }%
      \let\makeglossary\@empty
    }%
    \ifx\ReFiCh@temp\makeglossary
      \def\ReFiCh@temp{%
        \newwrite\@glossaryfile
        \RerunFileCheck{\jobname.glo}{%
          \immediate\closeout\@glossaryfile
        }{%
          Rerun LaTeX/makeindex to get glossary right%
        }%
        \immediate\openout\@glossaryfile=\jobname.glo %
        \def\glossary{%
          \@bsphack
          \begingroup
          \@sanitize
          \@wrglossary
        }%
        \typeout{Writing glossary file \jobname.glo}%
        \let\makeglossary\@empty
      }%
    \else
      \@PackageInfoNoLine{rerunfilecheck}{%
        Option `glossary': unsupported version of \string\makeglossary
      }%
    \fi
  \fi
\fi
\ReFiCh@DisableOption{mainaux}
\ReFiCh@DisableOption{partaux}
\ReFiCh@DisableOption{starttoc}
\ReFiCh@DisableOption{index}
\ReFiCh@DisableOption{glossary}
\ReFiCh@DisableOption{aux}
%    \end{macrocode}
%
% \subsection{Rerun check}
%
%    \begin{macrocode}
\RequirePackage{atveryend}[2016/05/16]
\RequirePackage{uniquecounter}[2009/12/18]
%    \end{macrocode}
%
%    \begin{macro}{\ReFiCh@CheckSum}
%    \begin{macrocode}
\begingroup\expandafter\expandafter\expandafter\endgroup
\expandafter\ifx\csname pdf@filesize\endcsname\relax
  \def\ReFiCh@CheckSum{%
    \pdf@filemdfivesum
  }%
\else
  \def\ReFiCh@CheckSum#1{%
    \pdf@filemdfivesum{#1}%
    \ReFiCh@Separator
    \pdf@filesize{#1}%
  }%
\fi
%    \end{macrocode}
%    \end{macro}
%    \begin{macro}{\ReFiCh@NoFile}
%    \begin{macrocode}
\def\ReFiCh@Separator{;}
%    \end{macrocode}
%    \end{macro}
%    \begin{macro}{\ReFiCh@NoFile}
%    \begin{macrocode}
\def\ReFiCh@NoFile{<no file>}
%    \end{macrocode}
%    \end{macro}
%
%    \begin{macrocode}
\UniqueCounterNew{rerunfilecheck}
%    \end{macrocode}
%
%    \begin{macro}{\RerunFileCheck}
%    \begin{macrocode}
\newcommand*{\RerunFileCheck}{%
  \UniqueCounterCall{rerunfilecheck}\ReFiCh@RerunFileCheck
}
%    \end{macrocode}
%    \end{macro}
%    \begin{macro}{\ReFiCh@RerunFileCheck}
%    \begin{macrocode}
\def\ReFiCh@RerunFileCheck#1{%
  \expandafter\ReFiCh@@RerunFileCheck\csname ReFiCh@#1\endcsname
}
%    \end{macrocode}
%    \end{macro}
%    \begin{macro}{\ReFiCh@Check}
%    \begin{macrocode}
\def\ReFiCh@Check#1#2#3{%
%  \IfFileExists{#3}{%
    #1\edef#2{\ReFiCh@CheckSum{#3}}%
    \ifx#2\ReFiCh@Separator
      #1\let#2\ReFiCh@NoFile
    \fi
%  }{%
%    #1\let#2\ReFiCh@NoFile
%  }%
}
%    \end{macrocode}
%    \end{macro}
%    \begin{macro}{\ReFiCh@@RerunFileCheck}
%    \begin{macrocode}
\def\ReFiCh@@RerunFileCheck#1#2#3#4{%
  \ReFiCh@Check\global#1{#2}%
  \AtEndAfterFileList{%
    \begingroup
      #3%
      \ReFiCh@Check{}\x{#2}%
      \ifx#1\x
        \@PackageInfoNoLine{rerunfilecheck}{%
          File `#2' has not changed.\MessageBreak
          Checksum: \x
        }%
      \else
        \ifnum
          \ReFiCh@IsAux#2\relax.aux\relax\@nil
          \ifx#1\ReFiCh@NoFile 1\else 0\fi
          \ifx\x\ReFiCh@AuxEmptyUnix 1%
          \else
            \ifx\x\ReFiCh@AuxEmptyDos 1\fi
          \fi
        =111 %
          \@PackageInfoNoLine{rerunfilecheck}{%
            File `#2' is empty .aux file.\MessageBreak
            Before: #1\MessageBreak
            After: \space\x
          }%
        \else
          \@PackageWarningNoLine{rerunfilecheck}{%
            File `#2' has changed.%
            \ifx\\#4\\%
              \space Rerun%
            \else
              \MessageBreak
              #4%
            \fi
          }%
          \@PackageInfoNoLine{rerunfilecheck}{%
            Checksums for `#2':\MessageBreak
            Before: #1\MessageBreak
            After: \space\x
          }%
        \fi
      \fi
    \endgroup
  }%
}
%    \end{macrocode}
%    \end{macro}
%    \begin{macrocode}
\def\ReFiCh@IsAux#1.aux\relax#2\@nil{%
  \ifx\hbox#2\hbox
    0%
  \else
    1%
  \fi
}
%    \end{macrocode}
%    \begin{macrocode}
\def\ReFiCh@AuxEmptyUnix{A94A2480D3289E625EEA47CD1B285758;8}%
\@onelevel@sanitize\ReFiCh@AuxEmptyUnix
%    \end{macrocode}
%    \begin{macrocode}
\def\ReFiCh@AuxEmptyDos{A62A15ECE803E2EBE94952FCC9933BC0;9}%
\@onelevel@sanitize\ReFiCh@AuxEmptyDos
%    \end{macrocode}
%
%    \begin{macrocode}
\ReFiCh@AtEnd%
%</package>
%    \end{macrocode}
%
% \section{Test}
%
%    \begin{macrocode}
%<*test1>
\def\LoadCommand{\RequirePackage{rerunfilecheck}[2016/05/16]}
%</test1>
%    \end{macrocode}
% \subsection{Catcode checks for loading}
%
%    \begin{macrocode}
%<*test1>
%    \end{macrocode}
%    \begin{macrocode}
\catcode`\{=1 %
\catcode`\}=2 %
\catcode`\#=6 %
\catcode`\@=11 %
\expandafter\ifx\csname count@\endcsname\relax
  \countdef\count@=255 %
\fi
\expandafter\ifx\csname @gobble\endcsname\relax
  \long\def\@gobble#1{}%
\fi
\expandafter\ifx\csname @firstofone\endcsname\relax
  \long\def\@firstofone#1{#1}%
\fi
\expandafter\ifx\csname loop\endcsname\relax
  \expandafter\@firstofone
\else
  \expandafter\@gobble
\fi
{%
  \def\loop#1\repeat{%
    \def\body{#1}%
    \iterate
  }%
  \def\iterate{%
    \body
      \let\next\iterate
    \else
      \let\next\relax
    \fi
    \next
  }%
  \let\repeat=\fi
}%
\def\RestoreCatcodes{}
\count@=0 %
\loop
  \edef\RestoreCatcodes{%
    \RestoreCatcodes
    \catcode\the\count@=\the\catcode\count@\relax
  }%
\ifnum\count@<255 %
  \advance\count@ 1 %
\repeat

\def\RangeCatcodeInvalid#1#2{%
  \count@=#1\relax
  \loop
    \catcode\count@=15 %
  \ifnum\count@<#2\relax
    \advance\count@ 1 %
  \repeat
}
\def\RangeCatcodeCheck#1#2#3{%
  \count@=#1\relax
  \loop
    \ifnum#3=\catcode\count@
    \else
      \errmessage{%
        Character \the\count@\space
        with wrong catcode \the\catcode\count@\space
        instead of \number#3%
      }%
    \fi
  \ifnum\count@<#2\relax
    \advance\count@ 1 %
  \repeat
}
\def\space{ }
\expandafter\ifx\csname LoadCommand\endcsname\relax
  \def\LoadCommand{\input rerunfilecheck.sty\relax}%
\fi
\def\Test{%
  \RangeCatcodeInvalid{0}{47}%
  \RangeCatcodeInvalid{58}{64}%
  \RangeCatcodeInvalid{91}{96}%
  \RangeCatcodeInvalid{123}{255}%
  \catcode`\@=12 %
  \catcode`\\=0 %
  \catcode`\%=14 %
  \LoadCommand
  \RangeCatcodeCheck{0}{36}{15}%
  \RangeCatcodeCheck{37}{37}{14}%
  \RangeCatcodeCheck{38}{47}{15}%
  \RangeCatcodeCheck{48}{57}{12}%
  \RangeCatcodeCheck{58}{63}{15}%
  \RangeCatcodeCheck{64}{64}{12}%
  \RangeCatcodeCheck{65}{90}{11}%
  \RangeCatcodeCheck{91}{91}{15}%
  \RangeCatcodeCheck{92}{92}{0}%
  \RangeCatcodeCheck{93}{96}{15}%
  \RangeCatcodeCheck{97}{122}{11}%
  \RangeCatcodeCheck{123}{255}{15}%
  \RestoreCatcodes
}
\Test
\csname @@end\endcsname
\end
%    \end{macrocode}
%    \begin{macrocode}
%</test1>
%    \end{macrocode}
%
% \section{Installation}
%
% \subsection{Download}
%
% \paragraph{Package.} This package is available on
% CTAN\footnote{\url{http://ctan.org/pkg/rerunfilecheck}}:
% \begin{description}
% \item[\CTAN{macros/latex/contrib/oberdiek/rerunfilecheck.dtx}] The source file.
% \item[\CTAN{macros/latex/contrib/oberdiek/rerunfilecheck.pdf}] Documentation.
% \end{description}
%
%
% \paragraph{Bundle.} All the packages of the bundle `oberdiek'
% are also available in a TDS compliant ZIP archive. There
% the packages are already unpacked and the documentation files
% are generated. The files and directories obey the TDS standard.
% \begin{description}
% \item[\CTAN{install/macros/latex/contrib/oberdiek.tds.zip}]
% \end{description}
% \emph{TDS} refers to the standard ``A Directory Structure
% for \TeX\ Files'' (\CTAN{tds/tds.pdf}). Directories
% with \xfile{texmf} in their name are usually organized this way.
%
% \subsection{Bundle installation}
%
% \paragraph{Unpacking.} Unpack the \xfile{oberdiek.tds.zip} in the
% TDS tree (also known as \xfile{texmf} tree) of your choice.
% Example (linux):
% \begin{quote}
%   |unzip oberdiek.tds.zip -d ~/texmf|
% \end{quote}
%
% \paragraph{Script installation.}
% Check the directory \xfile{TDS:scripts/oberdiek/} for
% scripts that need further installation steps.
% Package \xpackage{attachfile2} comes with the Perl script
% \xfile{pdfatfi.pl} that should be installed in such a way
% that it can be called as \texttt{pdfatfi}.
% Example (linux):
% \begin{quote}
%   |chmod +x scripts/oberdiek/pdfatfi.pl|\\
%   |cp scripts/oberdiek/pdfatfi.pl /usr/local/bin/|
% \end{quote}
%
% \subsection{Package installation}
%
% \paragraph{Unpacking.} The \xfile{.dtx} file is a self-extracting
% \docstrip\ archive. The files are extracted by running the
% \xfile{.dtx} through \plainTeX:
% \begin{quote}
%   \verb|tex rerunfilecheck.dtx|
% \end{quote}
%
% \paragraph{TDS.} Now the different files must be moved into
% the different directories in your installation TDS tree
% (also known as \xfile{texmf} tree):
% \begin{quote}
% \def\t{^^A
% \begin{tabular}{@{}>{\ttfamily}l@{ $\rightarrow$ }>{\ttfamily}l@{}}
%   rerunfilecheck.sty & tex/latex/oberdiek/rerunfilecheck.sty\\
%   rerunfilecheck.pdf & doc/latex/oberdiek/rerunfilecheck.pdf\\
%   rerunfilecheck-example.cfg & doc/latex/oberdiek/rerunfilecheck-example.cfg\\
%   test/rerunfilecheck-test1.tex & doc/latex/oberdiek/test/rerunfilecheck-test1.tex\\
%   rerunfilecheck.dtx & source/latex/oberdiek/rerunfilecheck.dtx\\
% \end{tabular}^^A
% }^^A
% \sbox0{\t}^^A
% \ifdim\wd0>\linewidth
%   \begingroup
%     \advance\linewidth by\leftmargin
%     \advance\linewidth by\rightmargin
%   \edef\x{\endgroup
%     \def\noexpand\lw{\the\linewidth}^^A
%   }\x
%   \def\lwbox{^^A
%     \leavevmode
%     \hbox to \linewidth{^^A
%       \kern-\leftmargin\relax
%       \hss
%       \usebox0
%       \hss
%       \kern-\rightmargin\relax
%     }^^A
%   }^^A
%   \ifdim\wd0>\lw
%     \sbox0{\small\t}^^A
%     \ifdim\wd0>\linewidth
%       \ifdim\wd0>\lw
%         \sbox0{\footnotesize\t}^^A
%         \ifdim\wd0>\linewidth
%           \ifdim\wd0>\lw
%             \sbox0{\scriptsize\t}^^A
%             \ifdim\wd0>\linewidth
%               \ifdim\wd0>\lw
%                 \sbox0{\tiny\t}^^A
%                 \ifdim\wd0>\linewidth
%                   \lwbox
%                 \else
%                   \usebox0
%                 \fi
%               \else
%                 \lwbox
%               \fi
%             \else
%               \usebox0
%             \fi
%           \else
%             \lwbox
%           \fi
%         \else
%           \usebox0
%         \fi
%       \else
%         \lwbox
%       \fi
%     \else
%       \usebox0
%     \fi
%   \else
%     \lwbox
%   \fi
% \else
%   \usebox0
% \fi
% \end{quote}
% If you have a \xfile{docstrip.cfg} that configures and enables \docstrip's
% TDS installing feature, then some files can already be in the right
% place, see the documentation of \docstrip.
%
% \subsection{Refresh file name databases}
%
% If your \TeX~distribution
% (\teTeX, \mikTeX, \dots) relies on file name databases, you must refresh
% these. For example, \teTeX\ users run \verb|texhash| or
% \verb|mktexlsr|.
%
% \subsection{Some details for the interested}
%
% \paragraph{Attached source.}
%
% The PDF documentation on CTAN also includes the
% \xfile{.dtx} source file. It can be extracted by
% AcrobatReader 6 or higher. Another option is \textsf{pdftk},
% e.g. unpack the file into the current directory:
% \begin{quote}
%   \verb|pdftk rerunfilecheck.pdf unpack_files output .|
% \end{quote}
%
% \paragraph{Unpacking with \LaTeX.}
% The \xfile{.dtx} chooses its action depending on the format:
% \begin{description}
% \item[\plainTeX:] Run \docstrip\ and extract the files.
% \item[\LaTeX:] Generate the documentation.
% \end{description}
% If you insist on using \LaTeX\ for \docstrip\ (really,
% \docstrip\ does not need \LaTeX), then inform the autodetect routine
% about your intention:
% \begin{quote}
%   \verb|latex \let\install=y% \iffalse meta-comment
%
% File: rerunfilecheck.dtx
% Version: 2016/05/16 v1.8
% Info: Rerun checks for auxiliary files
%
% Copyright (C) 2009-2011 by
%    Heiko Oberdiek <heiko.oberdiek at googlemail.com>
%    2016
%    https://github.com/ho-tex/oberdiek/issues
%
% This work may be distributed and/or modified under the
% conditions of the LaTeX Project Public License, either
% version 1.3c of this license or (at your option) any later
% version. This version of this license is in
%    http://www.latex-project.org/lppl/lppl-1-3c.txt
% and the latest version of this license is in
%    http://www.latex-project.org/lppl.txt
% and version 1.3 or later is part of all distributions of
% LaTeX version 2005/12/01 or later.
%
% This work has the LPPL maintenance status "maintained".
%
% This Current Maintainer of this work is Heiko Oberdiek.
%
% This work consists of the main source file rerunfilecheck.dtx
% and the derived files
%    rerunfilecheck.sty, rerunfilecheck.pdf, rerunfilecheck.ins,
%    rerunfilecheck.drv, rerunfilecheck-example.cfg,
%    rerunfilecheck-test1.tex.
%
% Distribution:
%    CTAN:macros/latex/contrib/oberdiek/rerunfilecheck.dtx
%    CTAN:macros/latex/contrib/oberdiek/rerunfilecheck.pdf
%
% Unpacking:
%    (a) If rerunfilecheck.ins is present:
%           tex rerunfilecheck.ins
%    (b) Without rerunfilecheck.ins:
%           tex rerunfilecheck.dtx
%    (c) If you insist on using LaTeX
%           latex \let\install=y\input{rerunfilecheck.dtx}
%        (quote the arguments according to the demands of your shell)
%
% Documentation:
%    (a) If rerunfilecheck.drv is present:
%           latex rerunfilecheck.drv
%    (b) Without rerunfilecheck.drv:
%           latex rerunfilecheck.dtx; ...
%    The class ltxdoc loads the configuration file ltxdoc.cfg
%    if available. Here you can specify further options, e.g.
%    use A4 as paper format:
%       \PassOptionsToClass{a4paper}{article}
%
%    Programm calls to get the documentation (example):
%       pdflatex rerunfilecheck.dtx
%       makeindex -s gind.ist rerunfilecheck.idx
%       pdflatex rerunfilecheck.dtx
%       makeindex -s gind.ist rerunfilecheck.idx
%       pdflatex rerunfilecheck.dtx
%
% Installation:
%    TDS:tex/latex/oberdiek/rerunfilecheck.sty
%    TDS:doc/latex/oberdiek/rerunfilecheck.pdf
%    TDS:doc/latex/oberdiek/rerunfilecheck-example.cfg
%    TDS:doc/latex/oberdiek/test/rerunfilecheck-test1.tex
%    TDS:source/latex/oberdiek/rerunfilecheck.dtx
%
%<*ignore>
\begingroup
  \catcode123=1 %
  \catcode125=2 %
  \def\x{LaTeX2e}%
\expandafter\endgroup
\ifcase 0\ifx\install y1\fi\expandafter
         \ifx\csname processbatchFile\endcsname\relax\else1\fi
         \ifx\fmtname\x\else 1\fi\relax
\else\csname fi\endcsname
%</ignore>
%<*install>
\input docstrip.tex
\Msg{************************************************************************}
\Msg{* Installation}
\Msg{* Package: rerunfilecheck 2016/05/16 v1.8 Rerun checks for auxiliary files (HO)}
\Msg{************************************************************************}

\keepsilent
\askforoverwritefalse

\let\MetaPrefix\relax
\preamble

This is a generated file.

Project: rerunfilecheck
Version: 2016/05/16 v1.8

Copyright (C) 2009-2011 by
   Heiko Oberdiek <heiko.oberdiek at googlemail.com>

This work may be distributed and/or modified under the
conditions of the LaTeX Project Public License, either
version 1.3c of this license or (at your option) any later
version. This version of this license is in
   http://www.latex-project.org/lppl/lppl-1-3c.txt
and the latest version of this license is in
   http://www.latex-project.org/lppl.txt
and version 1.3 or later is part of all distributions of
LaTeX version 2005/12/01 or later.

This work has the LPPL maintenance status "maintained".

This Current Maintainer of this work is Heiko Oberdiek.

This work consists of the main source file rerunfilecheck.dtx
and the derived files
   rerunfilecheck.sty, rerunfilecheck.pdf, rerunfilecheck.ins,
   rerunfilecheck.drv, rerunfilecheck-example.cfg,
   rerunfilecheck-test1.tex.

\endpreamble
\let\MetaPrefix\DoubleperCent

\generate{%
  \file{rerunfilecheck.ins}{\from{rerunfilecheck.dtx}{install}}%
  \file{rerunfilecheck.drv}{\from{rerunfilecheck.dtx}{driver}}%
  \usedir{tex/latex/oberdiek}%
  \file{rerunfilecheck.sty}{\from{rerunfilecheck.dtx}{package}}%
%  \usedir{doc/latex/oberdiek/test}%
%  \file{rerunfilecheck-test1.tex}{\from{rerunfilecheck.dtx}{test1}}%
  \nopreamble
  \nopostamble
  \usedir{doc/latex/oberdiek}%
  \file{rerunfilecheck-example.cfg}{\from{rerunfilecheck.dtx}{cfg}}%
  \usedir{source/latex/oberdiek/catalogue}%
  \file{rerunfilecheck.xml}{\from{rerunfilecheck.dtx}{catalogue}}%
}

\catcode32=13\relax% active space
\let =\space%
\Msg{************************************************************************}
\Msg{*}
\Msg{* To finish the installation you have to move the following}
\Msg{* file into a directory searched by TeX:}
\Msg{*}
\Msg{*     rerunfilecheck.sty}
\Msg{*}
\Msg{* To produce the documentation run the file `rerunfilecheck.drv'}
\Msg{* through LaTeX.}
\Msg{*}
\Msg{* Happy TeXing!}
\Msg{*}
\Msg{************************************************************************}

\endbatchfile
%</install>
%<*ignore>
\fi
%</ignore>
%<*driver>
\NeedsTeXFormat{LaTeX2e}
\ProvidesFile{rerunfilecheck.drv}%
  [2016/05/16 v1.8 Rerun checks for auxiliary files (HO)]%
\documentclass{ltxdoc}
\usepackage{holtxdoc}[2011/11/22]
\begin{document}
  \DocInput{rerunfilecheck.dtx}%
\end{document}
%</driver>
% \fi
%
%
% \CharacterTable
%  {Upper-case    \A\B\C\D\E\F\G\H\I\J\K\L\M\N\O\P\Q\R\S\T\U\V\W\X\Y\Z
%   Lower-case    \a\b\c\d\e\f\g\h\i\j\k\l\m\n\o\p\q\r\s\t\u\v\w\x\y\z
%   Digits        \0\1\2\3\4\5\6\7\8\9
%   Exclamation   \!     Double quote  \"     Hash (number) \#
%   Dollar        \$     Percent       \%     Ampersand     \&
%   Acute accent  \'     Left paren    \(     Right paren   \)
%   Asterisk      \*     Plus          \+     Comma         \,
%   Minus         \-     Point         \.     Solidus       \/
%   Colon         \:     Semicolon     \;     Less than     \<
%   Equals        \=     Greater than  \>     Question mark \?
%   Commercial at \@     Left bracket  \[     Backslash     \\
%   Right bracket \]     Circumflex    \^     Underscore    \_
%   Grave accent  \`     Left brace    \{     Vertical bar  \|
%   Right brace   \}     Tilde         \~}
%
% \GetFileInfo{rerunfilecheck.drv}
%
% \title{The \xpackage{rerunfilecheck} package}
% \date{2016/05/16 v1.8}
% \author{Heiko Oberdiek\thanks
% {Please report any issues at https://github.com/ho-tex/oberdiek/issues}\\
% \xemail{heiko.oberdiek at googlemail.com}}
%
% \maketitle
%
% \begin{abstract}
% The package provides additional rerun warnings if some
% auxiliary files have changed. It is based on MD5 checksum,
% provided by \pdfTeX.
% \end{abstract}
%
% \tableofcontents
%
% \section{Documentation}
%
% \LaTeX\ informs the user, when to run \LaTeX\ again, if the
% references have changed. It has the old references from the
% first reading of the \xext{aux} files already in memory, thus
% it compares them with the new version of the \xext{aux} file
% at the end of the document. However this rerun warnings are
% not given for the table of contents and other data stored
% in the \xext{aux} files or other auxiliary files.
% Usually many of these data as the table of contents is not keep
% in memory. If someone wants to detect changes, he has either
% to keep the data in memory. This does not scale well with huge
% documents. Or he copies the file before they are changed. Slow
% I/O operations cost time.
%
% Since version 1.30.0 \pdfTeX\ provides \cs{pdfmdfivesum}
% and \cs{pdffilesize}.
% These features are also available in \LuaTeX, provided by package
% \xpackage{pdftexcmds}.
% Thus this package \xpackage{rerunfilecheck} uses these features to
% detect file changes. This saves the packages from keeping
% the whole files in memory or in file copies.
% The drawback are different files with the same size and the
% same MD5 checksum (seldom, hopefully).
%
% \subsection{Options}
%
% All options are key value options of boolean type.
% No option or |true| turns an option on, |false| disables
% an option.
% \begin{description}
% \item[\xoption{mainaux}:]
%   Check the main \xext{aux} file.
% \item[\xoption{partaux}:]
%   Check the \xext{aux} files from \cs{include} files.
% \item[\xoption{starttoc}:]
%   Add the rerun checks in \cs{@starttoc} that is called by
%   \cs{tableofcontents}, \cs{listoffigures}, \dots
% \item[\xoption{index}, \xoption{glossary}:]
%   \LaTeX's original \cs{makeindex} and \cs{makeglossary} are
%   redefined to add the rerun checks. The options do not have
%   an effect, if \cs{makeindex}/\cs{makeglossary} are already
%   called or if a package or class had redefined or will
%   redefine them.
% \item[\xoption{aux}:]
%   This option turns all previous options on or off.
%   ``aux'' means auxiliary file.
% \end{description}
% The default for the options is |false|, because
% some internals must be redefined to insert the
% rerun checks.
% The options can be set in \cs{usepackage} or
% the configuration file \xfile{rerunfilecheck.cfg}.
% Global options are ignored (since 1.4).
%
% \begin{declcs}{RerunFileCheckSetup} \M{key value list}
% \end{declcs}
% Options can also be set using \cs{RerunFileCheckSetup}.
% Currently all options are disabled after the package
% is loaded. Thus \cs{RerunFileCheckSetup} makes sense in
% the configuration file only.
%
% Example for the configuration file:
%    \begin{macrocode}
%<*cfg>
\ProvidesFile{rerunfilecheck.cfg}[2016/05/16 Default configuration]%
\RerunFileCheckSetup{aux}
%</cfg>
%    \end{macrocode}
%
% \subsection{Interface for class/package authors}
%
% \begin{declcs}{RerunFileCheck} \M{file} \M{file closing action} \M{rerun warning}
% \end{declcs}
% If you want to add a rerun check, call \cs{RerunFileCheck} right before
% an output file is opened for writing. The macro first remembers
% the current checksum of \meta{file}.
% The file is checked again right before the end of the job.
% Macro \cs{AtVeryEndDocument} of package \xpackage{atveryend}
% is used to place the check after the main aux file is closed
% in |\end{document}|. Before reading the file again, it must be
% closed. Provide the code for closing in argument \meta{file closing action}.
% Do not forget \cs{immediate} before \cs{openout}. Otherwise the
% closing action would be delayed to the next shipout that never happens
% (the last page is already shipped out).
% If the file has changed, \cs{RerunFileCheck} informs the user with
% a warning that the file has changed and says the magic word ``Rerun''.
% If the last argument \meta{rerun warning} is not empty, then
% the rerun sentence is replaced by it. Usually the phrase
% ``to get something right'' is added.
% As example the relevant part of the redefined \cs{makeindex}
% is shown, see package code:
%\begin{quote}
%\begin{verbatim}
%\newwrite\@indexfile
%\RerunFileCheck{\jobname.idx}{%
%  \immediate\closeout\@indexfile
%}{%
%  Rerun LaTeX/makeindex to get index right%
%}%
%\immediate\openout\@indexfile=\jobname.idx %
%\end{verbatim}
%\end{quote}
%
% \StopEventually{
% }
%
% \section{Implementation}
%    \begin{macrocode}
%<*package>
%    \end{macrocode}
%
%    \begin{macrocode}
\begingroup\catcode61\catcode48\catcode32=10\relax%
  \catcode13=5 % ^^M
  \endlinechar=13 %
  \catcode123=1 % {
  \catcode125=2 % }
  \catcode64=11 % @
  \def\x{\endgroup
    \expandafter\edef\csname ReFiCh@AtEnd\endcsname{%
      \endlinechar=\the\endlinechar\relax
      \catcode13=\the\catcode13\relax
      \catcode32=\the\catcode32\relax
      \catcode35=\the\catcode35\relax
      \catcode61=\the\catcode61\relax
      \catcode64=\the\catcode64\relax
      \catcode123=\the\catcode123\relax
      \catcode125=\the\catcode125\relax
    }%
  }%
\x\catcode61\catcode48\catcode32=10\relax%
\catcode13=5 % ^^M
\endlinechar=13 %
\catcode35=6 % #
\catcode64=11 % @
\catcode123=1 % {
\catcode125=2 % }
\def\TMP@EnsureCode#1#2{%
  \edef\ReFiCh@AtEnd{%
    \ReFiCh@AtEnd
    \catcode#1=\the\catcode#1\relax
  }%
  \catcode#1=#2\relax
}
\TMP@EnsureCode{39}{12}% '
\TMP@EnsureCode{40}{12}% (
\TMP@EnsureCode{41}{12}% )
\TMP@EnsureCode{42}{12}% *
\TMP@EnsureCode{44}{12}% ,
\TMP@EnsureCode{46}{12}% .
\TMP@EnsureCode{47}{12}% /
\TMP@EnsureCode{58}{12}% :
\TMP@EnsureCode{59}{12}% ;
\TMP@EnsureCode{60}{12}% <
\TMP@EnsureCode{62}{12}% >
\TMP@EnsureCode{91}{12}% [
\TMP@EnsureCode{93}{12}% ]
\TMP@EnsureCode{96}{12}% `
\edef\ReFiCh@AtEnd{\ReFiCh@AtEnd\noexpand\endinput}
%    \end{macrocode}
%
%    Package identification.
%    \begin{macrocode}
\NeedsTeXFormat{LaTeX2e}
\ProvidesPackage{rerunfilecheck}%
  [2016/05/16 v1.8 Rerun checks for auxiliary files (HO)]
%    \end{macrocode}
%
% \subsection{Options}
%
%    \begin{macrocode}
\RequirePackage{kvoptions}[2010/02/22]
\SetupKeyvalOptions{%
  family=rerunfilecheck,%
  prefix=ReFiCh@%
}
%    \end{macrocode}
%    \begin{macro}{\RerunFileCheckSetup}
%    \begin{macrocode}
\newcommand*{\RerunFileCheckSetup}{%
  \setkeys{rerunfilecheck}%
}
%    \end{macrocode}
%    \end{macro}
%    \begin{macrocode}
\DeclareBoolOption{mainaux}
\DeclareBoolOption{partaux}
\DeclareBoolOption{starttoc}
\DeclareBoolOption{index}
\DeclareBoolOption{glossary}
\define@key{rerunfilecheck}{aux}[true]{%
  \RerunFileCheckSetup{%
    mainaux={#1},%
    partaux={#1},%
    starttoc={#1},%
    index={#1},%
    glossary={#1}%
  }%
}
%    \end{macrocode}
%    \begin{macrocode}
\InputIfFileExists{rerunfilecheck.cfg}{}{}
\ProcessLocalKeyvalOptions*
%    \end{macrocode}
%
%    \begin{macro}{\ReFiCh@DisableOption}
%    \begin{macrocode}
\def\ReFiCh@DisableOption{%
  \DisableKeyvalOption[%
    action=warning,%
    package=rerunfilecheck%
  ]{rerunfilecheck}%
}
%    \end{macrocode}
%    \end{macro}
%
% \subsection{Check for checksum feature}
%
%    \begin{macrocode}
\RequirePackage{infwarerr}[2007/09/09]
\RequirePackage{pdftexcmds}[2009/04/10]
%    \end{macrocode}
%    \begin{macrocode}
\begingroup\expandafter\expandafter\expandafter\endgroup
\expandafter\ifx\csname pdf@filemdfivesum\endcsname\relax
  \@PackageInfoNoLine{rerunfilecheck}{%
    Feature \string\pdfmdfivesum\space is not available\MessageBreak
    (e.g. pdfTeX or LuaTeX with package `pdftexcmds').\MessageBreak
    Therefore file contents cannot be checked efficiently\MessageBreak
    and the loading of the package is aborted%
  }%
  \newcommand*{\RerunFileCheck}[3]{}%
  \renewcommand*{\RerunFileCheckSetup}[1]{}%
  \expandafter\ReFiCh@AtEnd
\fi%
%    \end{macrocode}
%
% \subsection{Standard \xext{aux} files}
%
%    \begin{macrocode}
\ifReFiCh@partaux
  \let\ReFiCh@org@include\@include
  \def\@include#1 {%
    \if@filesw
      \RerunFileCheck{#1.aux}{}{}%
    \fi
    \ReFiCh@org@include{#1} %
  }%
\fi
\ifReFiCh@mainaux
  \AtBeginDocument{%
    \ReFiCh@mainauxfalse
  }%
  \ifReFiCh@mainaux
    \AtEndOfPackage{%
      \RerunFileCheck{\jobname.aux}{}{}%
    }%
  \else
    \if@filesw
      \@PackageWarningNoLine{rerunfilecheck}{%
        Main aux file check is disabled,\MessageBreak
        because the file is already opened.\MessageBreak
        Load the package before \string\begin{document}%
      }%
    \fi
  \fi
\fi
\ifReFiCh@starttoc
  \let\ReFiCh@org@starttoc\@starttoc
  \def\@starttoc#1{%
    \if@filesw
      \RerunFileCheck{\jobname.#1}{%
        \@ifundefined{tf@#1}{%
        }{%
          \immediate\closeout\csname tf@#1\endcsname
        }%
      }{}%
    \fi
    \ReFiCh@org@starttoc{#1}%
  }%
\fi
\ifReFiCh@index
  \ifx\makeindex\@empty
    \@PackageWarningNoLine{rerunfilecheck}{%
      Option `index' ignored,\MessageBreak
      because \string\makeindex\space has already been called%
    }%
  \else
    \def\ReFiCh@temp{%
      \newwrite\@indexfile
      \immediate\openout\@indexfile=\jobname.idx %
      \def\index{%
        \@bsphack
        \begingroup
        \@sanitize
        \@wrindex
      }%
      \typeout{Writing index file \jobname.idx}%
      \let\makeindex\@empty
    }%
    \ifx\ReFiCh@temp\makeindex
      \def\makeindex{%
        \newwrite\@indexfile
        \RerunFileCheck{\jobname.idx}{%
          \immediate\closeout\@indexfile
        }{%
          Rerun LaTeX/makeindex to get index right%
        }%
        \immediate\openout\@indexfile=\jobname.idx %
        \def\index{%
          \@bsphack
          \begingroup
          \@sanitize
          \@wrindex
        }%
        \typeout{Writing index file \jobname.idx}%
        \let\makeindex\@empty
      }%
    \else
      \@PackageInfoNoLine{rerunfilecheck}{%
        Option `index': unsupported version of \string\makeindex
      }%
    \fi
  \fi
\fi
\ifReFiCh@glossary
  \ifx\makeglossary\@empty
    \@PackageWarningNoLine{rerunfilecheck}{%
      Option `glossary' ignored,\MessageBreak
      because \string\makeglossary\space has already been called%
    }%
  \else
    \def\ReFiCh@temp{%
      \newwrite\@glossaryfile
      \immediate\openout\@glossaryfile=\jobname.glo %
      \def\glossary{%
        \@bsphack
        \begingroup
        \@sanitize
        \@wrglossary
      }%
      \typeout{Writing glossary file \jobname.glo }%
      \let\makeglossary\@empty
    }%
    \ifx\ReFiCh@temp\makeglossary
      \def\ReFiCh@temp{%
        \newwrite\@glossaryfile
        \RerunFileCheck{\jobname.glo}{%
          \immediate\closeout\@glossaryfile
        }{%
          Rerun LaTeX/makeindex to get glossary right%
        }%
        \immediate\openout\@glossaryfile=\jobname.glo %
        \def\glossary{%
          \@bsphack
          \begingroup
          \@sanitize
          \@wrglossary
        }%
        \typeout{Writing glossary file \jobname.glo}%
        \let\makeglossary\@empty
      }%
    \else
      \@PackageInfoNoLine{rerunfilecheck}{%
        Option `glossary': unsupported version of \string\makeglossary
      }%
    \fi
  \fi
\fi
\ReFiCh@DisableOption{mainaux}
\ReFiCh@DisableOption{partaux}
\ReFiCh@DisableOption{starttoc}
\ReFiCh@DisableOption{index}
\ReFiCh@DisableOption{glossary}
\ReFiCh@DisableOption{aux}
%    \end{macrocode}
%
% \subsection{Rerun check}
%
%    \begin{macrocode}
\RequirePackage{atveryend}[2016/05/16]
\RequirePackage{uniquecounter}[2009/12/18]
%    \end{macrocode}
%
%    \begin{macro}{\ReFiCh@CheckSum}
%    \begin{macrocode}
\begingroup\expandafter\expandafter\expandafter\endgroup
\expandafter\ifx\csname pdf@filesize\endcsname\relax
  \def\ReFiCh@CheckSum{%
    \pdf@filemdfivesum
  }%
\else
  \def\ReFiCh@CheckSum#1{%
    \pdf@filemdfivesum{#1}%
    \ReFiCh@Separator
    \pdf@filesize{#1}%
  }%
\fi
%    \end{macrocode}
%    \end{macro}
%    \begin{macro}{\ReFiCh@NoFile}
%    \begin{macrocode}
\def\ReFiCh@Separator{;}
%    \end{macrocode}
%    \end{macro}
%    \begin{macro}{\ReFiCh@NoFile}
%    \begin{macrocode}
\def\ReFiCh@NoFile{<no file>}
%    \end{macrocode}
%    \end{macro}
%
%    \begin{macrocode}
\UniqueCounterNew{rerunfilecheck}
%    \end{macrocode}
%
%    \begin{macro}{\RerunFileCheck}
%    \begin{macrocode}
\newcommand*{\RerunFileCheck}{%
  \UniqueCounterCall{rerunfilecheck}\ReFiCh@RerunFileCheck
}
%    \end{macrocode}
%    \end{macro}
%    \begin{macro}{\ReFiCh@RerunFileCheck}
%    \begin{macrocode}
\def\ReFiCh@RerunFileCheck#1{%
  \expandafter\ReFiCh@@RerunFileCheck\csname ReFiCh@#1\endcsname
}
%    \end{macrocode}
%    \end{macro}
%    \begin{macro}{\ReFiCh@Check}
%    \begin{macrocode}
\def\ReFiCh@Check#1#2#3{%
%  \IfFileExists{#3}{%
    #1\edef#2{\ReFiCh@CheckSum{#3}}%
    \ifx#2\ReFiCh@Separator
      #1\let#2\ReFiCh@NoFile
    \fi
%  }{%
%    #1\let#2\ReFiCh@NoFile
%  }%
}
%    \end{macrocode}
%    \end{macro}
%    \begin{macro}{\ReFiCh@@RerunFileCheck}
%    \begin{macrocode}
\def\ReFiCh@@RerunFileCheck#1#2#3#4{%
  \ReFiCh@Check\global#1{#2}%
  \AtEndAfterFileList{%
    \begingroup
      #3%
      \ReFiCh@Check{}\x{#2}%
      \ifx#1\x
        \@PackageInfoNoLine{rerunfilecheck}{%
          File `#2' has not changed.\MessageBreak
          Checksum: \x
        }%
      \else
        \ifnum
          \ReFiCh@IsAux#2\relax.aux\relax\@nil
          \ifx#1\ReFiCh@NoFile 1\else 0\fi
          \ifx\x\ReFiCh@AuxEmptyUnix 1%
          \else
            \ifx\x\ReFiCh@AuxEmptyDos 1\fi
          \fi
        =111 %
          \@PackageInfoNoLine{rerunfilecheck}{%
            File `#2' is empty .aux file.\MessageBreak
            Before: #1\MessageBreak
            After: \space\x
          }%
        \else
          \@PackageWarningNoLine{rerunfilecheck}{%
            File `#2' has changed.%
            \ifx\\#4\\%
              \space Rerun%
            \else
              \MessageBreak
              #4%
            \fi
          }%
          \@PackageInfoNoLine{rerunfilecheck}{%
            Checksums for `#2':\MessageBreak
            Before: #1\MessageBreak
            After: \space\x
          }%
        \fi
      \fi
    \endgroup
  }%
}
%    \end{macrocode}
%    \end{macro}
%    \begin{macrocode}
\def\ReFiCh@IsAux#1.aux\relax#2\@nil{%
  \ifx\hbox#2\hbox
    0%
  \else
    1%
  \fi
}
%    \end{macrocode}
%    \begin{macrocode}
\def\ReFiCh@AuxEmptyUnix{A94A2480D3289E625EEA47CD1B285758;8}%
\@onelevel@sanitize\ReFiCh@AuxEmptyUnix
%    \end{macrocode}
%    \begin{macrocode}
\def\ReFiCh@AuxEmptyDos{A62A15ECE803E2EBE94952FCC9933BC0;9}%
\@onelevel@sanitize\ReFiCh@AuxEmptyDos
%    \end{macrocode}
%
%    \begin{macrocode}
\ReFiCh@AtEnd%
%</package>
%    \end{macrocode}
%
% \section{Test}
%
%    \begin{macrocode}
%<*test1>
\def\LoadCommand{\RequirePackage{rerunfilecheck}[2016/05/16]}
%</test1>
%    \end{macrocode}
% \subsection{Catcode checks for loading}
%
%    \begin{macrocode}
%<*test1>
%    \end{macrocode}
%    \begin{macrocode}
\catcode`\{=1 %
\catcode`\}=2 %
\catcode`\#=6 %
\catcode`\@=11 %
\expandafter\ifx\csname count@\endcsname\relax
  \countdef\count@=255 %
\fi
\expandafter\ifx\csname @gobble\endcsname\relax
  \long\def\@gobble#1{}%
\fi
\expandafter\ifx\csname @firstofone\endcsname\relax
  \long\def\@firstofone#1{#1}%
\fi
\expandafter\ifx\csname loop\endcsname\relax
  \expandafter\@firstofone
\else
  \expandafter\@gobble
\fi
{%
  \def\loop#1\repeat{%
    \def\body{#1}%
    \iterate
  }%
  \def\iterate{%
    \body
      \let\next\iterate
    \else
      \let\next\relax
    \fi
    \next
  }%
  \let\repeat=\fi
}%
\def\RestoreCatcodes{}
\count@=0 %
\loop
  \edef\RestoreCatcodes{%
    \RestoreCatcodes
    \catcode\the\count@=\the\catcode\count@\relax
  }%
\ifnum\count@<255 %
  \advance\count@ 1 %
\repeat

\def\RangeCatcodeInvalid#1#2{%
  \count@=#1\relax
  \loop
    \catcode\count@=15 %
  \ifnum\count@<#2\relax
    \advance\count@ 1 %
  \repeat
}
\def\RangeCatcodeCheck#1#2#3{%
  \count@=#1\relax
  \loop
    \ifnum#3=\catcode\count@
    \else
      \errmessage{%
        Character \the\count@\space
        with wrong catcode \the\catcode\count@\space
        instead of \number#3%
      }%
    \fi
  \ifnum\count@<#2\relax
    \advance\count@ 1 %
  \repeat
}
\def\space{ }
\expandafter\ifx\csname LoadCommand\endcsname\relax
  \def\LoadCommand{\input rerunfilecheck.sty\relax}%
\fi
\def\Test{%
  \RangeCatcodeInvalid{0}{47}%
  \RangeCatcodeInvalid{58}{64}%
  \RangeCatcodeInvalid{91}{96}%
  \RangeCatcodeInvalid{123}{255}%
  \catcode`\@=12 %
  \catcode`\\=0 %
  \catcode`\%=14 %
  \LoadCommand
  \RangeCatcodeCheck{0}{36}{15}%
  \RangeCatcodeCheck{37}{37}{14}%
  \RangeCatcodeCheck{38}{47}{15}%
  \RangeCatcodeCheck{48}{57}{12}%
  \RangeCatcodeCheck{58}{63}{15}%
  \RangeCatcodeCheck{64}{64}{12}%
  \RangeCatcodeCheck{65}{90}{11}%
  \RangeCatcodeCheck{91}{91}{15}%
  \RangeCatcodeCheck{92}{92}{0}%
  \RangeCatcodeCheck{93}{96}{15}%
  \RangeCatcodeCheck{97}{122}{11}%
  \RangeCatcodeCheck{123}{255}{15}%
  \RestoreCatcodes
}
\Test
\csname @@end\endcsname
\end
%    \end{macrocode}
%    \begin{macrocode}
%</test1>
%    \end{macrocode}
%
% \section{Installation}
%
% \subsection{Download}
%
% \paragraph{Package.} This package is available on
% CTAN\footnote{\url{http://ctan.org/pkg/rerunfilecheck}}:
% \begin{description}
% \item[\CTAN{macros/latex/contrib/oberdiek/rerunfilecheck.dtx}] The source file.
% \item[\CTAN{macros/latex/contrib/oberdiek/rerunfilecheck.pdf}] Documentation.
% \end{description}
%
%
% \paragraph{Bundle.} All the packages of the bundle `oberdiek'
% are also available in a TDS compliant ZIP archive. There
% the packages are already unpacked and the documentation files
% are generated. The files and directories obey the TDS standard.
% \begin{description}
% \item[\CTAN{install/macros/latex/contrib/oberdiek.tds.zip}]
% \end{description}
% \emph{TDS} refers to the standard ``A Directory Structure
% for \TeX\ Files'' (\CTAN{tds/tds.pdf}). Directories
% with \xfile{texmf} in their name are usually organized this way.
%
% \subsection{Bundle installation}
%
% \paragraph{Unpacking.} Unpack the \xfile{oberdiek.tds.zip} in the
% TDS tree (also known as \xfile{texmf} tree) of your choice.
% Example (linux):
% \begin{quote}
%   |unzip oberdiek.tds.zip -d ~/texmf|
% \end{quote}
%
% \paragraph{Script installation.}
% Check the directory \xfile{TDS:scripts/oberdiek/} for
% scripts that need further installation steps.
% Package \xpackage{attachfile2} comes with the Perl script
% \xfile{pdfatfi.pl} that should be installed in such a way
% that it can be called as \texttt{pdfatfi}.
% Example (linux):
% \begin{quote}
%   |chmod +x scripts/oberdiek/pdfatfi.pl|\\
%   |cp scripts/oberdiek/pdfatfi.pl /usr/local/bin/|
% \end{quote}
%
% \subsection{Package installation}
%
% \paragraph{Unpacking.} The \xfile{.dtx} file is a self-extracting
% \docstrip\ archive. The files are extracted by running the
% \xfile{.dtx} through \plainTeX:
% \begin{quote}
%   \verb|tex rerunfilecheck.dtx|
% \end{quote}
%
% \paragraph{TDS.} Now the different files must be moved into
% the different directories in your installation TDS tree
% (also known as \xfile{texmf} tree):
% \begin{quote}
% \def\t{^^A
% \begin{tabular}{@{}>{\ttfamily}l@{ $\rightarrow$ }>{\ttfamily}l@{}}
%   rerunfilecheck.sty & tex/latex/oberdiek/rerunfilecheck.sty\\
%   rerunfilecheck.pdf & doc/latex/oberdiek/rerunfilecheck.pdf\\
%   rerunfilecheck-example.cfg & doc/latex/oberdiek/rerunfilecheck-example.cfg\\
%   test/rerunfilecheck-test1.tex & doc/latex/oberdiek/test/rerunfilecheck-test1.tex\\
%   rerunfilecheck.dtx & source/latex/oberdiek/rerunfilecheck.dtx\\
% \end{tabular}^^A
% }^^A
% \sbox0{\t}^^A
% \ifdim\wd0>\linewidth
%   \begingroup
%     \advance\linewidth by\leftmargin
%     \advance\linewidth by\rightmargin
%   \edef\x{\endgroup
%     \def\noexpand\lw{\the\linewidth}^^A
%   }\x
%   \def\lwbox{^^A
%     \leavevmode
%     \hbox to \linewidth{^^A
%       \kern-\leftmargin\relax
%       \hss
%       \usebox0
%       \hss
%       \kern-\rightmargin\relax
%     }^^A
%   }^^A
%   \ifdim\wd0>\lw
%     \sbox0{\small\t}^^A
%     \ifdim\wd0>\linewidth
%       \ifdim\wd0>\lw
%         \sbox0{\footnotesize\t}^^A
%         \ifdim\wd0>\linewidth
%           \ifdim\wd0>\lw
%             \sbox0{\scriptsize\t}^^A
%             \ifdim\wd0>\linewidth
%               \ifdim\wd0>\lw
%                 \sbox0{\tiny\t}^^A
%                 \ifdim\wd0>\linewidth
%                   \lwbox
%                 \else
%                   \usebox0
%                 \fi
%               \else
%                 \lwbox
%               \fi
%             \else
%               \usebox0
%             \fi
%           \else
%             \lwbox
%           \fi
%         \else
%           \usebox0
%         \fi
%       \else
%         \lwbox
%       \fi
%     \else
%       \usebox0
%     \fi
%   \else
%     \lwbox
%   \fi
% \else
%   \usebox0
% \fi
% \end{quote}
% If you have a \xfile{docstrip.cfg} that configures and enables \docstrip's
% TDS installing feature, then some files can already be in the right
% place, see the documentation of \docstrip.
%
% \subsection{Refresh file name databases}
%
% If your \TeX~distribution
% (\teTeX, \mikTeX, \dots) relies on file name databases, you must refresh
% these. For example, \teTeX\ users run \verb|texhash| or
% \verb|mktexlsr|.
%
% \subsection{Some details for the interested}
%
% \paragraph{Attached source.}
%
% The PDF documentation on CTAN also includes the
% \xfile{.dtx} source file. It can be extracted by
% AcrobatReader 6 or higher. Another option is \textsf{pdftk},
% e.g. unpack the file into the current directory:
% \begin{quote}
%   \verb|pdftk rerunfilecheck.pdf unpack_files output .|
% \end{quote}
%
% \paragraph{Unpacking with \LaTeX.}
% The \xfile{.dtx} chooses its action depending on the format:
% \begin{description}
% \item[\plainTeX:] Run \docstrip\ and extract the files.
% \item[\LaTeX:] Generate the documentation.
% \end{description}
% If you insist on using \LaTeX\ for \docstrip\ (really,
% \docstrip\ does not need \LaTeX), then inform the autodetect routine
% about your intention:
% \begin{quote}
%   \verb|latex \let\install=y\input{rerunfilecheck.dtx}|
% \end{quote}
% Do not forget to quote the argument according to the demands
% of your shell.
%
% \paragraph{Generating the documentation.}
% You can use both the \xfile{.dtx} or the \xfile{.drv} to generate
% the documentation. The process can be configured by the
% configuration file \xfile{ltxdoc.cfg}. For instance, put this
% line into this file, if you want to have A4 as paper format:
% \begin{quote}
%   \verb|\PassOptionsToClass{a4paper}{article}|
% \end{quote}
% An example follows how to generate the
% documentation with pdf\LaTeX:
% \begin{quote}
%\begin{verbatim}
%pdflatex rerunfilecheck.dtx
%makeindex -s gind.ist rerunfilecheck.idx
%pdflatex rerunfilecheck.dtx
%makeindex -s gind.ist rerunfilecheck.idx
%pdflatex rerunfilecheck.dtx
%\end{verbatim}
% \end{quote}
%
% \section{Catalogue}
%
% The following XML file can be used as source for the
% \href{http://mirror.ctan.org/help/Catalogue/catalogue.html}{\TeX\ Catalogue}.
% The elements \texttt{caption} and \texttt{description} are imported
% from the original XML file from the Catalogue.
% The name of the XML file in the Catalogue is \xfile{rerunfilecheck.xml}.
%    \begin{macrocode}
%<*catalogue>
<?xml version='1.0' encoding='us-ascii'?>
<!DOCTYPE entry SYSTEM 'catalogue.dtd'>
<entry datestamp='$Date$' modifier='$Author$' id='rerunfilecheck'>
  <name>rerunfilecheck</name>
  <caption>Checksum based rerun checks on auxiliary files.</caption>
  <authorref id='auth:oberdiek'/>
  <copyright owner='Heiko Oberdiek' year='2009-2011'/>
  <license type='lppl1.3'/>
  <version number='1.8'/>
  <description>
    The package provides additional rerun warnings if some
    auxiliary files have changed. It is based on MD5 checksum,
    provided by pdfTeX.
    <p/>
    The package is part of the <xref refid='oberdiek'>oberdiek</xref> bundle.
  </description>
  <documentation details='Package documentation'
      href='ctan:/macros/latex/contrib/oberdiek/rerunfilecheck.pdf'/>
  <ctan file='true' path='/macros/latex/contrib/oberdiek/rerunfilecheck.dtx'/>
  <miktex location='oberdiek'/>
  <texlive location='oberdiek'/>
  <install path='/macros/latex/contrib/oberdiek/oberdiek.tds.zip'/>
</entry>
%</catalogue>
%    \end{macrocode}
%
% \begin{History}
%   \begin{Version}{2009/12/10 v1.0}
%   \item
%     The first version.
%   \end{Version}
%   \begin{Version}{2009/12/12 v1.1}
%   \item
%     Short info shortened.
%   \end{Version}
%   \begin{Version}{2009/12/18 v1.2}
%   \item
%     Required date for package \xpackage{uniquecounter} updated
%     because of bug in this package.
%   \end{Version}
%   \begin{Version}{2010/01/25 v1.3}
%   \item
%     Moved from \texttt{TDS:*/generic/*} to \texttt{TDS:*/latex/*}.
%   \end{Version}
%   \begin{Version}{2010/02/22 v1.4}
%   \item
%     The options of this package are recognized only if they are
%     package options. Global options are ignored. This avoids
%     name clashes with class and other package options
%     (for example, class option `index=totoc').
%   \end{Version}
%   \begin{Version}{2010/03/15 v1.5}
%   \item
%     Call of \cs{pdffivesum} is wrapped in \cs{IfFileExists} to avoid
%     calls of \textsf{mktextex} if this feature is enabled.
%     However \cs{IfFileExists} has file name limitations.
%   \end{Version}
%   \begin{Version}{2010/03/16 v1.6}
%   \item
%     Reverted to version 1.4 and \cs{IfFileExists} wrapper of
%     version 1.5 is removed.
%   \end{Version}
%   \begin{Version}{2011/04/15 v1.7}
%   \item
%     Using \cs{AtEndAfterFileList} of package \xpackage{atveryend}
%     2011/04/15 v1.6 instead of \cs{AtVeryEndDocument}.
%   \end{Version}
%   \begin{Version}{2016/05/16 v1.8}
%   \item
%     Documentation updates.
%   \end{Version}
% \end{History}
%
% \PrintIndex
%
% \Finale
\endinput
|
% \end{quote}
% Do not forget to quote the argument according to the demands
% of your shell.
%
% \paragraph{Generating the documentation.}
% You can use both the \xfile{.dtx} or the \xfile{.drv} to generate
% the documentation. The process can be configured by the
% configuration file \xfile{ltxdoc.cfg}. For instance, put this
% line into this file, if you want to have A4 as paper format:
% \begin{quote}
%   \verb|\PassOptionsToClass{a4paper}{article}|
% \end{quote}
% An example follows how to generate the
% documentation with pdf\LaTeX:
% \begin{quote}
%\begin{verbatim}
%pdflatex rerunfilecheck.dtx
%makeindex -s gind.ist rerunfilecheck.idx
%pdflatex rerunfilecheck.dtx
%makeindex -s gind.ist rerunfilecheck.idx
%pdflatex rerunfilecheck.dtx
%\end{verbatim}
% \end{quote}
%
% \section{Catalogue}
%
% The following XML file can be used as source for the
% \href{http://mirror.ctan.org/help/Catalogue/catalogue.html}{\TeX\ Catalogue}.
% The elements \texttt{caption} and \texttt{description} are imported
% from the original XML file from the Catalogue.
% The name of the XML file in the Catalogue is \xfile{rerunfilecheck.xml}.
%    \begin{macrocode}
%<*catalogue>
<?xml version='1.0' encoding='us-ascii'?>
<!DOCTYPE entry SYSTEM 'catalogue.dtd'>
<entry datestamp='$Date$' modifier='$Author$' id='rerunfilecheck'>
  <name>rerunfilecheck</name>
  <caption>Checksum based rerun checks on auxiliary files.</caption>
  <authorref id='auth:oberdiek'/>
  <copyright owner='Heiko Oberdiek' year='2009-2011'/>
  <license type='lppl1.3'/>
  <version number='1.8'/>
  <description>
    The package provides additional rerun warnings if some
    auxiliary files have changed. It is based on MD5 checksum,
    provided by pdfTeX.
    <p/>
    The package is part of the <xref refid='oberdiek'>oberdiek</xref> bundle.
  </description>
  <documentation details='Package documentation'
      href='ctan:/macros/latex/contrib/oberdiek/rerunfilecheck.pdf'/>
  <ctan file='true' path='/macros/latex/contrib/oberdiek/rerunfilecheck.dtx'/>
  <miktex location='oberdiek'/>
  <texlive location='oberdiek'/>
  <install path='/macros/latex/contrib/oberdiek/oberdiek.tds.zip'/>
</entry>
%</catalogue>
%    \end{macrocode}
%
% \begin{History}
%   \begin{Version}{2009/12/10 v1.0}
%   \item
%     The first version.
%   \end{Version}
%   \begin{Version}{2009/12/12 v1.1}
%   \item
%     Short info shortened.
%   \end{Version}
%   \begin{Version}{2009/12/18 v1.2}
%   \item
%     Required date for package \xpackage{uniquecounter} updated
%     because of bug in this package.
%   \end{Version}
%   \begin{Version}{2010/01/25 v1.3}
%   \item
%     Moved from \texttt{TDS:*/generic/*} to \texttt{TDS:*/latex/*}.
%   \end{Version}
%   \begin{Version}{2010/02/22 v1.4}
%   \item
%     The options of this package are recognized only if they are
%     package options. Global options are ignored. This avoids
%     name clashes with class and other package options
%     (for example, class option `index=totoc').
%   \end{Version}
%   \begin{Version}{2010/03/15 v1.5}
%   \item
%     Call of \cs{pdffivesum} is wrapped in \cs{IfFileExists} to avoid
%     calls of \textsf{mktextex} if this feature is enabled.
%     However \cs{IfFileExists} has file name limitations.
%   \end{Version}
%   \begin{Version}{2010/03/16 v1.6}
%   \item
%     Reverted to version 1.4 and \cs{IfFileExists} wrapper of
%     version 1.5 is removed.
%   \end{Version}
%   \begin{Version}{2011/04/15 v1.7}
%   \item
%     Using \cs{AtEndAfterFileList} of package \xpackage{atveryend}
%     2011/04/15 v1.6 instead of \cs{AtVeryEndDocument}.
%   \end{Version}
%   \begin{Version}{2016/05/16 v1.8}
%   \item
%     Documentation updates.
%   \end{Version}
% \end{History}
%
% \PrintIndex
%
% \Finale
\endinput

%        (quote the arguments according to the demands of your shell)
%
% Documentation:
%    (a) If rerunfilecheck.drv is present:
%           latex rerunfilecheck.drv
%    (b) Without rerunfilecheck.drv:
%           latex rerunfilecheck.dtx; ...
%    The class ltxdoc loads the configuration file ltxdoc.cfg
%    if available. Here you can specify further options, e.g.
%    use A4 as paper format:
%       \PassOptionsToClass{a4paper}{article}
%
%    Programm calls to get the documentation (example):
%       pdflatex rerunfilecheck.dtx
%       makeindex -s gind.ist rerunfilecheck.idx
%       pdflatex rerunfilecheck.dtx
%       makeindex -s gind.ist rerunfilecheck.idx
%       pdflatex rerunfilecheck.dtx
%
% Installation:
%    TDS:tex/latex/oberdiek/rerunfilecheck.sty
%    TDS:doc/latex/oberdiek/rerunfilecheck.pdf
%    TDS:doc/latex/oberdiek/rerunfilecheck-example.cfg
%    TDS:doc/latex/oberdiek/test/rerunfilecheck-test1.tex
%    TDS:source/latex/oberdiek/rerunfilecheck.dtx
%
%<*ignore>
\begingroup
  \catcode123=1 %
  \catcode125=2 %
  \def\x{LaTeX2e}%
\expandafter\endgroup
\ifcase 0\ifx\install y1\fi\expandafter
         \ifx\csname processbatchFile\endcsname\relax\else1\fi
         \ifx\fmtname\x\else 1\fi\relax
\else\csname fi\endcsname
%</ignore>
%<*install>
\input docstrip.tex
\Msg{************************************************************************}
\Msg{* Installation}
\Msg{* Package: rerunfilecheck 2016/05/16 v1.8 Rerun checks for auxiliary files (HO)}
\Msg{************************************************************************}

\keepsilent
\askforoverwritefalse

\let\MetaPrefix\relax
\preamble

This is a generated file.

Project: rerunfilecheck
Version: 2016/05/16 v1.8

Copyright (C) 2009-2011 by
   Heiko Oberdiek <heiko.oberdiek at googlemail.com>

This work may be distributed and/or modified under the
conditions of the LaTeX Project Public License, either
version 1.3c of this license or (at your option) any later
version. This version of this license is in
   http://www.latex-project.org/lppl/lppl-1-3c.txt
and the latest version of this license is in
   http://www.latex-project.org/lppl.txt
and version 1.3 or later is part of all distributions of
LaTeX version 2005/12/01 or later.

This work has the LPPL maintenance status "maintained".

This Current Maintainer of this work is Heiko Oberdiek.

This work consists of the main source file rerunfilecheck.dtx
and the derived files
   rerunfilecheck.sty, rerunfilecheck.pdf, rerunfilecheck.ins,
   rerunfilecheck.drv, rerunfilecheck-example.cfg,
   rerunfilecheck-test1.tex.

\endpreamble
\let\MetaPrefix\DoubleperCent

\generate{%
  \file{rerunfilecheck.ins}{\from{rerunfilecheck.dtx}{install}}%
  \file{rerunfilecheck.drv}{\from{rerunfilecheck.dtx}{driver}}%
  \usedir{tex/latex/oberdiek}%
  \file{rerunfilecheck.sty}{\from{rerunfilecheck.dtx}{package}}%
%  \usedir{doc/latex/oberdiek/test}%
%  \file{rerunfilecheck-test1.tex}{\from{rerunfilecheck.dtx}{test1}}%
  \nopreamble
  \nopostamble
  \usedir{doc/latex/oberdiek}%
  \file{rerunfilecheck-example.cfg}{\from{rerunfilecheck.dtx}{cfg}}%
  \usedir{source/latex/oberdiek/catalogue}%
  \file{rerunfilecheck.xml}{\from{rerunfilecheck.dtx}{catalogue}}%
}

\catcode32=13\relax% active space
\let =\space%
\Msg{************************************************************************}
\Msg{*}
\Msg{* To finish the installation you have to move the following}
\Msg{* file into a directory searched by TeX:}
\Msg{*}
\Msg{*     rerunfilecheck.sty}
\Msg{*}
\Msg{* To produce the documentation run the file `rerunfilecheck.drv'}
\Msg{* through LaTeX.}
\Msg{*}
\Msg{* Happy TeXing!}
\Msg{*}
\Msg{************************************************************************}

\endbatchfile
%</install>
%<*ignore>
\fi
%</ignore>
%<*driver>
\NeedsTeXFormat{LaTeX2e}
\ProvidesFile{rerunfilecheck.drv}%
  [2016/05/16 v1.8 Rerun checks for auxiliary files (HO)]%
\documentclass{ltxdoc}
\usepackage{holtxdoc}[2011/11/22]
\begin{document}
  \DocInput{rerunfilecheck.dtx}%
\end{document}
%</driver>
% \fi
%
%
% \CharacterTable
%  {Upper-case    \A\B\C\D\E\F\G\H\I\J\K\L\M\N\O\P\Q\R\S\T\U\V\W\X\Y\Z
%   Lower-case    \a\b\c\d\e\f\g\h\i\j\k\l\m\n\o\p\q\r\s\t\u\v\w\x\y\z
%   Digits        \0\1\2\3\4\5\6\7\8\9
%   Exclamation   \!     Double quote  \"     Hash (number) \#
%   Dollar        \$     Percent       \%     Ampersand     \&
%   Acute accent  \'     Left paren    \(     Right paren   \)
%   Asterisk      \*     Plus          \+     Comma         \,
%   Minus         \-     Point         \.     Solidus       \/
%   Colon         \:     Semicolon     \;     Less than     \<
%   Equals        \=     Greater than  \>     Question mark \?
%   Commercial at \@     Left bracket  \[     Backslash     \\
%   Right bracket \]     Circumflex    \^     Underscore    \_
%   Grave accent  \`     Left brace    \{     Vertical bar  \|
%   Right brace   \}     Tilde         \~}
%
% \GetFileInfo{rerunfilecheck.drv}
%
% \title{The \xpackage{rerunfilecheck} package}
% \date{2016/05/16 v1.8}
% \author{Heiko Oberdiek\thanks
% {Please report any issues at https://github.com/ho-tex/oberdiek/issues}\\
% \xemail{heiko.oberdiek at googlemail.com}}
%
% \maketitle
%
% \begin{abstract}
% The package provides additional rerun warnings if some
% auxiliary files have changed. It is based on MD5 checksum,
% provided by \pdfTeX.
% \end{abstract}
%
% \tableofcontents
%
% \section{Documentation}
%
% \LaTeX\ informs the user, when to run \LaTeX\ again, if the
% references have changed. It has the old references from the
% first reading of the \xext{aux} files already in memory, thus
% it compares them with the new version of the \xext{aux} file
% at the end of the document. However this rerun warnings are
% not given for the table of contents and other data stored
% in the \xext{aux} files or other auxiliary files.
% Usually many of these data as the table of contents is not keep
% in memory. If someone wants to detect changes, he has either
% to keep the data in memory. This does not scale well with huge
% documents. Or he copies the file before they are changed. Slow
% I/O operations cost time.
%
% Since version 1.30.0 \pdfTeX\ provides \cs{pdfmdfivesum}
% and \cs{pdffilesize}.
% These features are also available in \LuaTeX, provided by package
% \xpackage{pdftexcmds}.
% Thus this package \xpackage{rerunfilecheck} uses these features to
% detect file changes. This saves the packages from keeping
% the whole files in memory or in file copies.
% The drawback are different files with the same size and the
% same MD5 checksum (seldom, hopefully).
%
% \subsection{Options}
%
% All options are key value options of boolean type.
% No option or |true| turns an option on, |false| disables
% an option.
% \begin{description}
% \item[\xoption{mainaux}:]
%   Check the main \xext{aux} file.
% \item[\xoption{partaux}:]
%   Check the \xext{aux} files from \cs{include} files.
% \item[\xoption{starttoc}:]
%   Add the rerun checks in \cs{@starttoc} that is called by
%   \cs{tableofcontents}, \cs{listoffigures}, \dots
% \item[\xoption{index}, \xoption{glossary}:]
%   \LaTeX's original \cs{makeindex} and \cs{makeglossary} are
%   redefined to add the rerun checks. The options do not have
%   an effect, if \cs{makeindex}/\cs{makeglossary} are already
%   called or if a package or class had redefined or will
%   redefine them.
% \item[\xoption{aux}:]
%   This option turns all previous options on or off.
%   ``aux'' means auxiliary file.
% \end{description}
% The default for the options is |false|, because
% some internals must be redefined to insert the
% rerun checks.
% The options can be set in \cs{usepackage} or
% the configuration file \xfile{rerunfilecheck.cfg}.
% Global options are ignored (since 1.4).
%
% \begin{declcs}{RerunFileCheckSetup} \M{key value list}
% \end{declcs}
% Options can also be set using \cs{RerunFileCheckSetup}.
% Currently all options are disabled after the package
% is loaded. Thus \cs{RerunFileCheckSetup} makes sense in
% the configuration file only.
%
% Example for the configuration file:
%    \begin{macrocode}
%<*cfg>
\ProvidesFile{rerunfilecheck.cfg}[2016/05/16 Default configuration]%
\RerunFileCheckSetup{aux}
%</cfg>
%    \end{macrocode}
%
% \subsection{Interface for class/package authors}
%
% \begin{declcs}{RerunFileCheck} \M{file} \M{file closing action} \M{rerun warning}
% \end{declcs}
% If you want to add a rerun check, call \cs{RerunFileCheck} right before
% an output file is opened for writing. The macro first remembers
% the current checksum of \meta{file}.
% The file is checked again right before the end of the job.
% Macro \cs{AtVeryEndDocument} of package \xpackage{atveryend}
% is used to place the check after the main aux file is closed
% in |\end{document}|. Before reading the file again, it must be
% closed. Provide the code for closing in argument \meta{file closing action}.
% Do not forget \cs{immediate} before \cs{openout}. Otherwise the
% closing action would be delayed to the next shipout that never happens
% (the last page is already shipped out).
% If the file has changed, \cs{RerunFileCheck} informs the user with
% a warning that the file has changed and says the magic word ``Rerun''.
% If the last argument \meta{rerun warning} is not empty, then
% the rerun sentence is replaced by it. Usually the phrase
% ``to get something right'' is added.
% As example the relevant part of the redefined \cs{makeindex}
% is shown, see package code:
%\begin{quote}
%\begin{verbatim}
%\newwrite\@indexfile
%\RerunFileCheck{\jobname.idx}{%
%  \immediate\closeout\@indexfile
%}{%
%  Rerun LaTeX/makeindex to get index right%
%}%
%\immediate\openout\@indexfile=\jobname.idx %
%\end{verbatim}
%\end{quote}
%
% \StopEventually{
% }
%
% \section{Implementation}
%    \begin{macrocode}
%<*package>
%    \end{macrocode}
%
%    \begin{macrocode}
\begingroup\catcode61\catcode48\catcode32=10\relax%
  \catcode13=5 % ^^M
  \endlinechar=13 %
  \catcode123=1 % {
  \catcode125=2 % }
  \catcode64=11 % @
  \def\x{\endgroup
    \expandafter\edef\csname ReFiCh@AtEnd\endcsname{%
      \endlinechar=\the\endlinechar\relax
      \catcode13=\the\catcode13\relax
      \catcode32=\the\catcode32\relax
      \catcode35=\the\catcode35\relax
      \catcode61=\the\catcode61\relax
      \catcode64=\the\catcode64\relax
      \catcode123=\the\catcode123\relax
      \catcode125=\the\catcode125\relax
    }%
  }%
\x\catcode61\catcode48\catcode32=10\relax%
\catcode13=5 % ^^M
\endlinechar=13 %
\catcode35=6 % #
\catcode64=11 % @
\catcode123=1 % {
\catcode125=2 % }
\def\TMP@EnsureCode#1#2{%
  \edef\ReFiCh@AtEnd{%
    \ReFiCh@AtEnd
    \catcode#1=\the\catcode#1\relax
  }%
  \catcode#1=#2\relax
}
\TMP@EnsureCode{39}{12}% '
\TMP@EnsureCode{40}{12}% (
\TMP@EnsureCode{41}{12}% )
\TMP@EnsureCode{42}{12}% *
\TMP@EnsureCode{44}{12}% ,
\TMP@EnsureCode{46}{12}% .
\TMP@EnsureCode{47}{12}% /
\TMP@EnsureCode{58}{12}% :
\TMP@EnsureCode{59}{12}% ;
\TMP@EnsureCode{60}{12}% <
\TMP@EnsureCode{62}{12}% >
\TMP@EnsureCode{91}{12}% [
\TMP@EnsureCode{93}{12}% ]
\TMP@EnsureCode{96}{12}% `
\edef\ReFiCh@AtEnd{\ReFiCh@AtEnd\noexpand\endinput}
%    \end{macrocode}
%
%    Package identification.
%    \begin{macrocode}
\NeedsTeXFormat{LaTeX2e}
\ProvidesPackage{rerunfilecheck}%
  [2016/05/16 v1.8 Rerun checks for auxiliary files (HO)]
%    \end{macrocode}
%
% \subsection{Options}
%
%    \begin{macrocode}
\RequirePackage{kvoptions}[2010/02/22]
\SetupKeyvalOptions{%
  family=rerunfilecheck,%
  prefix=ReFiCh@%
}
%    \end{macrocode}
%    \begin{macro}{\RerunFileCheckSetup}
%    \begin{macrocode}
\newcommand*{\RerunFileCheckSetup}{%
  \setkeys{rerunfilecheck}%
}
%    \end{macrocode}
%    \end{macro}
%    \begin{macrocode}
\DeclareBoolOption{mainaux}
\DeclareBoolOption{partaux}
\DeclareBoolOption{starttoc}
\DeclareBoolOption{index}
\DeclareBoolOption{glossary}
\define@key{rerunfilecheck}{aux}[true]{%
  \RerunFileCheckSetup{%
    mainaux={#1},%
    partaux={#1},%
    starttoc={#1},%
    index={#1},%
    glossary={#1}%
  }%
}
%    \end{macrocode}
%    \begin{macrocode}
\InputIfFileExists{rerunfilecheck.cfg}{}{}
\ProcessLocalKeyvalOptions*
%    \end{macrocode}
%
%    \begin{macro}{\ReFiCh@DisableOption}
%    \begin{macrocode}
\def\ReFiCh@DisableOption{%
  \DisableKeyvalOption[%
    action=warning,%
    package=rerunfilecheck%
  ]{rerunfilecheck}%
}
%    \end{macrocode}
%    \end{macro}
%
% \subsection{Check for checksum feature}
%
%    \begin{macrocode}
\RequirePackage{infwarerr}[2007/09/09]
\RequirePackage{pdftexcmds}[2009/04/10]
%    \end{macrocode}
%    \begin{macrocode}
\begingroup\expandafter\expandafter\expandafter\endgroup
\expandafter\ifx\csname pdf@filemdfivesum\endcsname\relax
  \@PackageInfoNoLine{rerunfilecheck}{%
    Feature \string\pdfmdfivesum\space is not available\MessageBreak
    (e.g. pdfTeX or LuaTeX with package `pdftexcmds').\MessageBreak
    Therefore file contents cannot be checked efficiently\MessageBreak
    and the loading of the package is aborted%
  }%
  \newcommand*{\RerunFileCheck}[3]{}%
  \renewcommand*{\RerunFileCheckSetup}[1]{}%
  \expandafter\ReFiCh@AtEnd
\fi%
%    \end{macrocode}
%
% \subsection{Standard \xext{aux} files}
%
%    \begin{macrocode}
\ifReFiCh@partaux
  \let\ReFiCh@org@include\@include
  \def\@include#1 {%
    \if@filesw
      \RerunFileCheck{#1.aux}{}{}%
    \fi
    \ReFiCh@org@include{#1} %
  }%
\fi
\ifReFiCh@mainaux
  \AtBeginDocument{%
    \ReFiCh@mainauxfalse
  }%
  \ifReFiCh@mainaux
    \AtEndOfPackage{%
      \RerunFileCheck{\jobname.aux}{}{}%
    }%
  \else
    \if@filesw
      \@PackageWarningNoLine{rerunfilecheck}{%
        Main aux file check is disabled,\MessageBreak
        because the file is already opened.\MessageBreak
        Load the package before \string\begin{document}%
      }%
    \fi
  \fi
\fi
\ifReFiCh@starttoc
  \let\ReFiCh@org@starttoc\@starttoc
  \def\@starttoc#1{%
    \if@filesw
      \RerunFileCheck{\jobname.#1}{%
        \@ifundefined{tf@#1}{%
        }{%
          \immediate\closeout\csname tf@#1\endcsname
        }%
      }{}%
    \fi
    \ReFiCh@org@starttoc{#1}%
  }%
\fi
\ifReFiCh@index
  \ifx\makeindex\@empty
    \@PackageWarningNoLine{rerunfilecheck}{%
      Option `index' ignored,\MessageBreak
      because \string\makeindex\space has already been called%
    }%
  \else
    \def\ReFiCh@temp{%
      \newwrite\@indexfile
      \immediate\openout\@indexfile=\jobname.idx %
      \def\index{%
        \@bsphack
        \begingroup
        \@sanitize
        \@wrindex
      }%
      \typeout{Writing index file \jobname.idx}%
      \let\makeindex\@empty
    }%
    \ifx\ReFiCh@temp\makeindex
      \def\makeindex{%
        \newwrite\@indexfile
        \RerunFileCheck{\jobname.idx}{%
          \immediate\closeout\@indexfile
        }{%
          Rerun LaTeX/makeindex to get index right%
        }%
        \immediate\openout\@indexfile=\jobname.idx %
        \def\index{%
          \@bsphack
          \begingroup
          \@sanitize
          \@wrindex
        }%
        \typeout{Writing index file \jobname.idx}%
        \let\makeindex\@empty
      }%
    \else
      \@PackageInfoNoLine{rerunfilecheck}{%
        Option `index': unsupported version of \string\makeindex
      }%
    \fi
  \fi
\fi
\ifReFiCh@glossary
  \ifx\makeglossary\@empty
    \@PackageWarningNoLine{rerunfilecheck}{%
      Option `glossary' ignored,\MessageBreak
      because \string\makeglossary\space has already been called%
    }%
  \else
    \def\ReFiCh@temp{%
      \newwrite\@glossaryfile
      \immediate\openout\@glossaryfile=\jobname.glo %
      \def\glossary{%
        \@bsphack
        \begingroup
        \@sanitize
        \@wrglossary
      }%
      \typeout{Writing glossary file \jobname.glo }%
      \let\makeglossary\@empty
    }%
    \ifx\ReFiCh@temp\makeglossary
      \def\ReFiCh@temp{%
        \newwrite\@glossaryfile
        \RerunFileCheck{\jobname.glo}{%
          \immediate\closeout\@glossaryfile
        }{%
          Rerun LaTeX/makeindex to get glossary right%
        }%
        \immediate\openout\@glossaryfile=\jobname.glo %
        \def\glossary{%
          \@bsphack
          \begingroup
          \@sanitize
          \@wrglossary
        }%
        \typeout{Writing glossary file \jobname.glo}%
        \let\makeglossary\@empty
      }%
    \else
      \@PackageInfoNoLine{rerunfilecheck}{%
        Option `glossary': unsupported version of \string\makeglossary
      }%
    \fi
  \fi
\fi
\ReFiCh@DisableOption{mainaux}
\ReFiCh@DisableOption{partaux}
\ReFiCh@DisableOption{starttoc}
\ReFiCh@DisableOption{index}
\ReFiCh@DisableOption{glossary}
\ReFiCh@DisableOption{aux}
%    \end{macrocode}
%
% \subsection{Rerun check}
%
%    \begin{macrocode}
\RequirePackage{atveryend}[2016/05/16]
\RequirePackage{uniquecounter}[2009/12/18]
%    \end{macrocode}
%
%    \begin{macro}{\ReFiCh@CheckSum}
%    \begin{macrocode}
\begingroup\expandafter\expandafter\expandafter\endgroup
\expandafter\ifx\csname pdf@filesize\endcsname\relax
  \def\ReFiCh@CheckSum{%
    \pdf@filemdfivesum
  }%
\else
  \def\ReFiCh@CheckSum#1{%
    \pdf@filemdfivesum{#1}%
    \ReFiCh@Separator
    \pdf@filesize{#1}%
  }%
\fi
%    \end{macrocode}
%    \end{macro}
%    \begin{macro}{\ReFiCh@NoFile}
%    \begin{macrocode}
\def\ReFiCh@Separator{;}
%    \end{macrocode}
%    \end{macro}
%    \begin{macro}{\ReFiCh@NoFile}
%    \begin{macrocode}
\def\ReFiCh@NoFile{<no file>}
%    \end{macrocode}
%    \end{macro}
%
%    \begin{macrocode}
\UniqueCounterNew{rerunfilecheck}
%    \end{macrocode}
%
%    \begin{macro}{\RerunFileCheck}
%    \begin{macrocode}
\newcommand*{\RerunFileCheck}{%
  \UniqueCounterCall{rerunfilecheck}\ReFiCh@RerunFileCheck
}
%    \end{macrocode}
%    \end{macro}
%    \begin{macro}{\ReFiCh@RerunFileCheck}
%    \begin{macrocode}
\def\ReFiCh@RerunFileCheck#1{%
  \expandafter\ReFiCh@@RerunFileCheck\csname ReFiCh@#1\endcsname
}
%    \end{macrocode}
%    \end{macro}
%    \begin{macro}{\ReFiCh@Check}
%    \begin{macrocode}
\def\ReFiCh@Check#1#2#3{%
%  \IfFileExists{#3}{%
    #1\edef#2{\ReFiCh@CheckSum{#3}}%
    \ifx#2\ReFiCh@Separator
      #1\let#2\ReFiCh@NoFile
    \fi
%  }{%
%    #1\let#2\ReFiCh@NoFile
%  }%
}
%    \end{macrocode}
%    \end{macro}
%    \begin{macro}{\ReFiCh@@RerunFileCheck}
%    \begin{macrocode}
\def\ReFiCh@@RerunFileCheck#1#2#3#4{%
  \ReFiCh@Check\global#1{#2}%
  \AtEndAfterFileList{%
    \begingroup
      #3%
      \ReFiCh@Check{}\x{#2}%
      \ifx#1\x
        \@PackageInfoNoLine{rerunfilecheck}{%
          File `#2' has not changed.\MessageBreak
          Checksum: \x
        }%
      \else
        \ifnum
          \ReFiCh@IsAux#2\relax.aux\relax\@nil
          \ifx#1\ReFiCh@NoFile 1\else 0\fi
          \ifx\x\ReFiCh@AuxEmptyUnix 1%
          \else
            \ifx\x\ReFiCh@AuxEmptyDos 1\fi
          \fi
        =111 %
          \@PackageInfoNoLine{rerunfilecheck}{%
            File `#2' is empty .aux file.\MessageBreak
            Before: #1\MessageBreak
            After: \space\x
          }%
        \else
          \@PackageWarningNoLine{rerunfilecheck}{%
            File `#2' has changed.%
            \ifx\\#4\\%
              \space Rerun%
            \else
              \MessageBreak
              #4%
            \fi
          }%
          \@PackageInfoNoLine{rerunfilecheck}{%
            Checksums for `#2':\MessageBreak
            Before: #1\MessageBreak
            After: \space\x
          }%
        \fi
      \fi
    \endgroup
  }%
}
%    \end{macrocode}
%    \end{macro}
%    \begin{macrocode}
\def\ReFiCh@IsAux#1.aux\relax#2\@nil{%
  \ifx\hbox#2\hbox
    0%
  \else
    1%
  \fi
}
%    \end{macrocode}
%    \begin{macrocode}
\def\ReFiCh@AuxEmptyUnix{A94A2480D3289E625EEA47CD1B285758;8}%
\@onelevel@sanitize\ReFiCh@AuxEmptyUnix
%    \end{macrocode}
%    \begin{macrocode}
\def\ReFiCh@AuxEmptyDos{A62A15ECE803E2EBE94952FCC9933BC0;9}%
\@onelevel@sanitize\ReFiCh@AuxEmptyDos
%    \end{macrocode}
%
%    \begin{macrocode}
\ReFiCh@AtEnd%
%</package>
%    \end{macrocode}
%
% \section{Test}
%
%    \begin{macrocode}
%<*test1>
\def\LoadCommand{\RequirePackage{rerunfilecheck}[2016/05/16]}
%</test1>
%    \end{macrocode}
% \subsection{Catcode checks for loading}
%
%    \begin{macrocode}
%<*test1>
%    \end{macrocode}
%    \begin{macrocode}
\catcode`\{=1 %
\catcode`\}=2 %
\catcode`\#=6 %
\catcode`\@=11 %
\expandafter\ifx\csname count@\endcsname\relax
  \countdef\count@=255 %
\fi
\expandafter\ifx\csname @gobble\endcsname\relax
  \long\def\@gobble#1{}%
\fi
\expandafter\ifx\csname @firstofone\endcsname\relax
  \long\def\@firstofone#1{#1}%
\fi
\expandafter\ifx\csname loop\endcsname\relax
  \expandafter\@firstofone
\else
  \expandafter\@gobble
\fi
{%
  \def\loop#1\repeat{%
    \def\body{#1}%
    \iterate
  }%
  \def\iterate{%
    \body
      \let\next\iterate
    \else
      \let\next\relax
    \fi
    \next
  }%
  \let\repeat=\fi
}%
\def\RestoreCatcodes{}
\count@=0 %
\loop
  \edef\RestoreCatcodes{%
    \RestoreCatcodes
    \catcode\the\count@=\the\catcode\count@\relax
  }%
\ifnum\count@<255 %
  \advance\count@ 1 %
\repeat

\def\RangeCatcodeInvalid#1#2{%
  \count@=#1\relax
  \loop
    \catcode\count@=15 %
  \ifnum\count@<#2\relax
    \advance\count@ 1 %
  \repeat
}
\def\RangeCatcodeCheck#1#2#3{%
  \count@=#1\relax
  \loop
    \ifnum#3=\catcode\count@
    \else
      \errmessage{%
        Character \the\count@\space
        with wrong catcode \the\catcode\count@\space
        instead of \number#3%
      }%
    \fi
  \ifnum\count@<#2\relax
    \advance\count@ 1 %
  \repeat
}
\def\space{ }
\expandafter\ifx\csname LoadCommand\endcsname\relax
  \def\LoadCommand{\input rerunfilecheck.sty\relax}%
\fi
\def\Test{%
  \RangeCatcodeInvalid{0}{47}%
  \RangeCatcodeInvalid{58}{64}%
  \RangeCatcodeInvalid{91}{96}%
  \RangeCatcodeInvalid{123}{255}%
  \catcode`\@=12 %
  \catcode`\\=0 %
  \catcode`\%=14 %
  \LoadCommand
  \RangeCatcodeCheck{0}{36}{15}%
  \RangeCatcodeCheck{37}{37}{14}%
  \RangeCatcodeCheck{38}{47}{15}%
  \RangeCatcodeCheck{48}{57}{12}%
  \RangeCatcodeCheck{58}{63}{15}%
  \RangeCatcodeCheck{64}{64}{12}%
  \RangeCatcodeCheck{65}{90}{11}%
  \RangeCatcodeCheck{91}{91}{15}%
  \RangeCatcodeCheck{92}{92}{0}%
  \RangeCatcodeCheck{93}{96}{15}%
  \RangeCatcodeCheck{97}{122}{11}%
  \RangeCatcodeCheck{123}{255}{15}%
  \RestoreCatcodes
}
\Test
\csname @@end\endcsname
\end
%    \end{macrocode}
%    \begin{macrocode}
%</test1>
%    \end{macrocode}
%
% \section{Installation}
%
% \subsection{Download}
%
% \paragraph{Package.} This package is available on
% CTAN\footnote{\url{http://ctan.org/pkg/rerunfilecheck}}:
% \begin{description}
% \item[\CTAN{macros/latex/contrib/oberdiek/rerunfilecheck.dtx}] The source file.
% \item[\CTAN{macros/latex/contrib/oberdiek/rerunfilecheck.pdf}] Documentation.
% \end{description}
%
%
% \paragraph{Bundle.} All the packages of the bundle `oberdiek'
% are also available in a TDS compliant ZIP archive. There
% the packages are already unpacked and the documentation files
% are generated. The files and directories obey the TDS standard.
% \begin{description}
% \item[\CTAN{install/macros/latex/contrib/oberdiek.tds.zip}]
% \end{description}
% \emph{TDS} refers to the standard ``A Directory Structure
% for \TeX\ Files'' (\CTAN{tds/tds.pdf}). Directories
% with \xfile{texmf} in their name are usually organized this way.
%
% \subsection{Bundle installation}
%
% \paragraph{Unpacking.} Unpack the \xfile{oberdiek.tds.zip} in the
% TDS tree (also known as \xfile{texmf} tree) of your choice.
% Example (linux):
% \begin{quote}
%   |unzip oberdiek.tds.zip -d ~/texmf|
% \end{quote}
%
% \paragraph{Script installation.}
% Check the directory \xfile{TDS:scripts/oberdiek/} for
% scripts that need further installation steps.
% Package \xpackage{attachfile2} comes with the Perl script
% \xfile{pdfatfi.pl} that should be installed in such a way
% that it can be called as \texttt{pdfatfi}.
% Example (linux):
% \begin{quote}
%   |chmod +x scripts/oberdiek/pdfatfi.pl|\\
%   |cp scripts/oberdiek/pdfatfi.pl /usr/local/bin/|
% \end{quote}
%
% \subsection{Package installation}
%
% \paragraph{Unpacking.} The \xfile{.dtx} file is a self-extracting
% \docstrip\ archive. The files are extracted by running the
% \xfile{.dtx} through \plainTeX:
% \begin{quote}
%   \verb|tex rerunfilecheck.dtx|
% \end{quote}
%
% \paragraph{TDS.} Now the different files must be moved into
% the different directories in your installation TDS tree
% (also known as \xfile{texmf} tree):
% \begin{quote}
% \def\t{^^A
% \begin{tabular}{@{}>{\ttfamily}l@{ $\rightarrow$ }>{\ttfamily}l@{}}
%   rerunfilecheck.sty & tex/latex/oberdiek/rerunfilecheck.sty\\
%   rerunfilecheck.pdf & doc/latex/oberdiek/rerunfilecheck.pdf\\
%   rerunfilecheck-example.cfg & doc/latex/oberdiek/rerunfilecheck-example.cfg\\
%   test/rerunfilecheck-test1.tex & doc/latex/oberdiek/test/rerunfilecheck-test1.tex\\
%   rerunfilecheck.dtx & source/latex/oberdiek/rerunfilecheck.dtx\\
% \end{tabular}^^A
% }^^A
% \sbox0{\t}^^A
% \ifdim\wd0>\linewidth
%   \begingroup
%     \advance\linewidth by\leftmargin
%     \advance\linewidth by\rightmargin
%   \edef\x{\endgroup
%     \def\noexpand\lw{\the\linewidth}^^A
%   }\x
%   \def\lwbox{^^A
%     \leavevmode
%     \hbox to \linewidth{^^A
%       \kern-\leftmargin\relax
%       \hss
%       \usebox0
%       \hss
%       \kern-\rightmargin\relax
%     }^^A
%   }^^A
%   \ifdim\wd0>\lw
%     \sbox0{\small\t}^^A
%     \ifdim\wd0>\linewidth
%       \ifdim\wd0>\lw
%         \sbox0{\footnotesize\t}^^A
%         \ifdim\wd0>\linewidth
%           \ifdim\wd0>\lw
%             \sbox0{\scriptsize\t}^^A
%             \ifdim\wd0>\linewidth
%               \ifdim\wd0>\lw
%                 \sbox0{\tiny\t}^^A
%                 \ifdim\wd0>\linewidth
%                   \lwbox
%                 \else
%                   \usebox0
%                 \fi
%               \else
%                 \lwbox
%               \fi
%             \else
%               \usebox0
%             \fi
%           \else
%             \lwbox
%           \fi
%         \else
%           \usebox0
%         \fi
%       \else
%         \lwbox
%       \fi
%     \else
%       \usebox0
%     \fi
%   \else
%     \lwbox
%   \fi
% \else
%   \usebox0
% \fi
% \end{quote}
% If you have a \xfile{docstrip.cfg} that configures and enables \docstrip's
% TDS installing feature, then some files can already be in the right
% place, see the documentation of \docstrip.
%
% \subsection{Refresh file name databases}
%
% If your \TeX~distribution
% (\teTeX, \mikTeX, \dots) relies on file name databases, you must refresh
% these. For example, \teTeX\ users run \verb|texhash| or
% \verb|mktexlsr|.
%
% \subsection{Some details for the interested}
%
% \paragraph{Attached source.}
%
% The PDF documentation on CTAN also includes the
% \xfile{.dtx} source file. It can be extracted by
% AcrobatReader 6 or higher. Another option is \textsf{pdftk},
% e.g. unpack the file into the current directory:
% \begin{quote}
%   \verb|pdftk rerunfilecheck.pdf unpack_files output .|
% \end{quote}
%
% \paragraph{Unpacking with \LaTeX.}
% The \xfile{.dtx} chooses its action depending on the format:
% \begin{description}
% \item[\plainTeX:] Run \docstrip\ and extract the files.
% \item[\LaTeX:] Generate the documentation.
% \end{description}
% If you insist on using \LaTeX\ for \docstrip\ (really,
% \docstrip\ does not need \LaTeX), then inform the autodetect routine
% about your intention:
% \begin{quote}
%   \verb|latex \let\install=y% \iffalse meta-comment
%
% File: rerunfilecheck.dtx
% Version: 2016/05/16 v1.8
% Info: Rerun checks for auxiliary files
%
% Copyright (C) 2009-2011 by
%    Heiko Oberdiek <heiko.oberdiek at googlemail.com>
%    2016
%    https://github.com/ho-tex/oberdiek/issues
%
% This work may be distributed and/or modified under the
% conditions of the LaTeX Project Public License, either
% version 1.3c of this license or (at your option) any later
% version. This version of this license is in
%    http://www.latex-project.org/lppl/lppl-1-3c.txt
% and the latest version of this license is in
%    http://www.latex-project.org/lppl.txt
% and version 1.3 or later is part of all distributions of
% LaTeX version 2005/12/01 or later.
%
% This work has the LPPL maintenance status "maintained".
%
% This Current Maintainer of this work is Heiko Oberdiek.
%
% This work consists of the main source file rerunfilecheck.dtx
% and the derived files
%    rerunfilecheck.sty, rerunfilecheck.pdf, rerunfilecheck.ins,
%    rerunfilecheck.drv, rerunfilecheck-example.cfg,
%    rerunfilecheck-test1.tex.
%
% Distribution:
%    CTAN:macros/latex/contrib/oberdiek/rerunfilecheck.dtx
%    CTAN:macros/latex/contrib/oberdiek/rerunfilecheck.pdf
%
% Unpacking:
%    (a) If rerunfilecheck.ins is present:
%           tex rerunfilecheck.ins
%    (b) Without rerunfilecheck.ins:
%           tex rerunfilecheck.dtx
%    (c) If you insist on using LaTeX
%           latex \let\install=y% \iffalse meta-comment
%
% File: rerunfilecheck.dtx
% Version: 2016/05/16 v1.8
% Info: Rerun checks for auxiliary files
%
% Copyright (C) 2009-2011 by
%    Heiko Oberdiek <heiko.oberdiek at googlemail.com>
%    2016
%    https://github.com/ho-tex/oberdiek/issues
%
% This work may be distributed and/or modified under the
% conditions of the LaTeX Project Public License, either
% version 1.3c of this license or (at your option) any later
% version. This version of this license is in
%    http://www.latex-project.org/lppl/lppl-1-3c.txt
% and the latest version of this license is in
%    http://www.latex-project.org/lppl.txt
% and version 1.3 or later is part of all distributions of
% LaTeX version 2005/12/01 or later.
%
% This work has the LPPL maintenance status "maintained".
%
% This Current Maintainer of this work is Heiko Oberdiek.
%
% This work consists of the main source file rerunfilecheck.dtx
% and the derived files
%    rerunfilecheck.sty, rerunfilecheck.pdf, rerunfilecheck.ins,
%    rerunfilecheck.drv, rerunfilecheck-example.cfg,
%    rerunfilecheck-test1.tex.
%
% Distribution:
%    CTAN:macros/latex/contrib/oberdiek/rerunfilecheck.dtx
%    CTAN:macros/latex/contrib/oberdiek/rerunfilecheck.pdf
%
% Unpacking:
%    (a) If rerunfilecheck.ins is present:
%           tex rerunfilecheck.ins
%    (b) Without rerunfilecheck.ins:
%           tex rerunfilecheck.dtx
%    (c) If you insist on using LaTeX
%           latex \let\install=y\input{rerunfilecheck.dtx}
%        (quote the arguments according to the demands of your shell)
%
% Documentation:
%    (a) If rerunfilecheck.drv is present:
%           latex rerunfilecheck.drv
%    (b) Without rerunfilecheck.drv:
%           latex rerunfilecheck.dtx; ...
%    The class ltxdoc loads the configuration file ltxdoc.cfg
%    if available. Here you can specify further options, e.g.
%    use A4 as paper format:
%       \PassOptionsToClass{a4paper}{article}
%
%    Programm calls to get the documentation (example):
%       pdflatex rerunfilecheck.dtx
%       makeindex -s gind.ist rerunfilecheck.idx
%       pdflatex rerunfilecheck.dtx
%       makeindex -s gind.ist rerunfilecheck.idx
%       pdflatex rerunfilecheck.dtx
%
% Installation:
%    TDS:tex/latex/oberdiek/rerunfilecheck.sty
%    TDS:doc/latex/oberdiek/rerunfilecheck.pdf
%    TDS:doc/latex/oberdiek/rerunfilecheck-example.cfg
%    TDS:doc/latex/oberdiek/test/rerunfilecheck-test1.tex
%    TDS:source/latex/oberdiek/rerunfilecheck.dtx
%
%<*ignore>
\begingroup
  \catcode123=1 %
  \catcode125=2 %
  \def\x{LaTeX2e}%
\expandafter\endgroup
\ifcase 0\ifx\install y1\fi\expandafter
         \ifx\csname processbatchFile\endcsname\relax\else1\fi
         \ifx\fmtname\x\else 1\fi\relax
\else\csname fi\endcsname
%</ignore>
%<*install>
\input docstrip.tex
\Msg{************************************************************************}
\Msg{* Installation}
\Msg{* Package: rerunfilecheck 2016/05/16 v1.8 Rerun checks for auxiliary files (HO)}
\Msg{************************************************************************}

\keepsilent
\askforoverwritefalse

\let\MetaPrefix\relax
\preamble

This is a generated file.

Project: rerunfilecheck
Version: 2016/05/16 v1.8

Copyright (C) 2009-2011 by
   Heiko Oberdiek <heiko.oberdiek at googlemail.com>

This work may be distributed and/or modified under the
conditions of the LaTeX Project Public License, either
version 1.3c of this license or (at your option) any later
version. This version of this license is in
   http://www.latex-project.org/lppl/lppl-1-3c.txt
and the latest version of this license is in
   http://www.latex-project.org/lppl.txt
and version 1.3 or later is part of all distributions of
LaTeX version 2005/12/01 or later.

This work has the LPPL maintenance status "maintained".

This Current Maintainer of this work is Heiko Oberdiek.

This work consists of the main source file rerunfilecheck.dtx
and the derived files
   rerunfilecheck.sty, rerunfilecheck.pdf, rerunfilecheck.ins,
   rerunfilecheck.drv, rerunfilecheck-example.cfg,
   rerunfilecheck-test1.tex.

\endpreamble
\let\MetaPrefix\DoubleperCent

\generate{%
  \file{rerunfilecheck.ins}{\from{rerunfilecheck.dtx}{install}}%
  \file{rerunfilecheck.drv}{\from{rerunfilecheck.dtx}{driver}}%
  \usedir{tex/latex/oberdiek}%
  \file{rerunfilecheck.sty}{\from{rerunfilecheck.dtx}{package}}%
%  \usedir{doc/latex/oberdiek/test}%
%  \file{rerunfilecheck-test1.tex}{\from{rerunfilecheck.dtx}{test1}}%
  \nopreamble
  \nopostamble
  \usedir{doc/latex/oberdiek}%
  \file{rerunfilecheck-example.cfg}{\from{rerunfilecheck.dtx}{cfg}}%
  \usedir{source/latex/oberdiek/catalogue}%
  \file{rerunfilecheck.xml}{\from{rerunfilecheck.dtx}{catalogue}}%
}

\catcode32=13\relax% active space
\let =\space%
\Msg{************************************************************************}
\Msg{*}
\Msg{* To finish the installation you have to move the following}
\Msg{* file into a directory searched by TeX:}
\Msg{*}
\Msg{*     rerunfilecheck.sty}
\Msg{*}
\Msg{* To produce the documentation run the file `rerunfilecheck.drv'}
\Msg{* through LaTeX.}
\Msg{*}
\Msg{* Happy TeXing!}
\Msg{*}
\Msg{************************************************************************}

\endbatchfile
%</install>
%<*ignore>
\fi
%</ignore>
%<*driver>
\NeedsTeXFormat{LaTeX2e}
\ProvidesFile{rerunfilecheck.drv}%
  [2016/05/16 v1.8 Rerun checks for auxiliary files (HO)]%
\documentclass{ltxdoc}
\usepackage{holtxdoc}[2011/11/22]
\begin{document}
  \DocInput{rerunfilecheck.dtx}%
\end{document}
%</driver>
% \fi
%
%
% \CharacterTable
%  {Upper-case    \A\B\C\D\E\F\G\H\I\J\K\L\M\N\O\P\Q\R\S\T\U\V\W\X\Y\Z
%   Lower-case    \a\b\c\d\e\f\g\h\i\j\k\l\m\n\o\p\q\r\s\t\u\v\w\x\y\z
%   Digits        \0\1\2\3\4\5\6\7\8\9
%   Exclamation   \!     Double quote  \"     Hash (number) \#
%   Dollar        \$     Percent       \%     Ampersand     \&
%   Acute accent  \'     Left paren    \(     Right paren   \)
%   Asterisk      \*     Plus          \+     Comma         \,
%   Minus         \-     Point         \.     Solidus       \/
%   Colon         \:     Semicolon     \;     Less than     \<
%   Equals        \=     Greater than  \>     Question mark \?
%   Commercial at \@     Left bracket  \[     Backslash     \\
%   Right bracket \]     Circumflex    \^     Underscore    \_
%   Grave accent  \`     Left brace    \{     Vertical bar  \|
%   Right brace   \}     Tilde         \~}
%
% \GetFileInfo{rerunfilecheck.drv}
%
% \title{The \xpackage{rerunfilecheck} package}
% \date{2016/05/16 v1.8}
% \author{Heiko Oberdiek\thanks
% {Please report any issues at https://github.com/ho-tex/oberdiek/issues}\\
% \xemail{heiko.oberdiek at googlemail.com}}
%
% \maketitle
%
% \begin{abstract}
% The package provides additional rerun warnings if some
% auxiliary files have changed. It is based on MD5 checksum,
% provided by \pdfTeX.
% \end{abstract}
%
% \tableofcontents
%
% \section{Documentation}
%
% \LaTeX\ informs the user, when to run \LaTeX\ again, if the
% references have changed. It has the old references from the
% first reading of the \xext{aux} files already in memory, thus
% it compares them with the new version of the \xext{aux} file
% at the end of the document. However this rerun warnings are
% not given for the table of contents and other data stored
% in the \xext{aux} files or other auxiliary files.
% Usually many of these data as the table of contents is not keep
% in memory. If someone wants to detect changes, he has either
% to keep the data in memory. This does not scale well with huge
% documents. Or he copies the file before they are changed. Slow
% I/O operations cost time.
%
% Since version 1.30.0 \pdfTeX\ provides \cs{pdfmdfivesum}
% and \cs{pdffilesize}.
% These features are also available in \LuaTeX, provided by package
% \xpackage{pdftexcmds}.
% Thus this package \xpackage{rerunfilecheck} uses these features to
% detect file changes. This saves the packages from keeping
% the whole files in memory or in file copies.
% The drawback are different files with the same size and the
% same MD5 checksum (seldom, hopefully).
%
% \subsection{Options}
%
% All options are key value options of boolean type.
% No option or |true| turns an option on, |false| disables
% an option.
% \begin{description}
% \item[\xoption{mainaux}:]
%   Check the main \xext{aux} file.
% \item[\xoption{partaux}:]
%   Check the \xext{aux} files from \cs{include} files.
% \item[\xoption{starttoc}:]
%   Add the rerun checks in \cs{@starttoc} that is called by
%   \cs{tableofcontents}, \cs{listoffigures}, \dots
% \item[\xoption{index}, \xoption{glossary}:]
%   \LaTeX's original \cs{makeindex} and \cs{makeglossary} are
%   redefined to add the rerun checks. The options do not have
%   an effect, if \cs{makeindex}/\cs{makeglossary} are already
%   called or if a package or class had redefined or will
%   redefine them.
% \item[\xoption{aux}:]
%   This option turns all previous options on or off.
%   ``aux'' means auxiliary file.
% \end{description}
% The default for the options is |false|, because
% some internals must be redefined to insert the
% rerun checks.
% The options can be set in \cs{usepackage} or
% the configuration file \xfile{rerunfilecheck.cfg}.
% Global options are ignored (since 1.4).
%
% \begin{declcs}{RerunFileCheckSetup} \M{key value list}
% \end{declcs}
% Options can also be set using \cs{RerunFileCheckSetup}.
% Currently all options are disabled after the package
% is loaded. Thus \cs{RerunFileCheckSetup} makes sense in
% the configuration file only.
%
% Example for the configuration file:
%    \begin{macrocode}
%<*cfg>
\ProvidesFile{rerunfilecheck.cfg}[2016/05/16 Default configuration]%
\RerunFileCheckSetup{aux}
%</cfg>
%    \end{macrocode}
%
% \subsection{Interface for class/package authors}
%
% \begin{declcs}{RerunFileCheck} \M{file} \M{file closing action} \M{rerun warning}
% \end{declcs}
% If you want to add a rerun check, call \cs{RerunFileCheck} right before
% an output file is opened for writing. The macro first remembers
% the current checksum of \meta{file}.
% The file is checked again right before the end of the job.
% Macro \cs{AtVeryEndDocument} of package \xpackage{atveryend}
% is used to place the check after the main aux file is closed
% in |\end{document}|. Before reading the file again, it must be
% closed. Provide the code for closing in argument \meta{file closing action}.
% Do not forget \cs{immediate} before \cs{openout}. Otherwise the
% closing action would be delayed to the next shipout that never happens
% (the last page is already shipped out).
% If the file has changed, \cs{RerunFileCheck} informs the user with
% a warning that the file has changed and says the magic word ``Rerun''.
% If the last argument \meta{rerun warning} is not empty, then
% the rerun sentence is replaced by it. Usually the phrase
% ``to get something right'' is added.
% As example the relevant part of the redefined \cs{makeindex}
% is shown, see package code:
%\begin{quote}
%\begin{verbatim}
%\newwrite\@indexfile
%\RerunFileCheck{\jobname.idx}{%
%  \immediate\closeout\@indexfile
%}{%
%  Rerun LaTeX/makeindex to get index right%
%}%
%\immediate\openout\@indexfile=\jobname.idx %
%\end{verbatim}
%\end{quote}
%
% \StopEventually{
% }
%
% \section{Implementation}
%    \begin{macrocode}
%<*package>
%    \end{macrocode}
%
%    \begin{macrocode}
\begingroup\catcode61\catcode48\catcode32=10\relax%
  \catcode13=5 % ^^M
  \endlinechar=13 %
  \catcode123=1 % {
  \catcode125=2 % }
  \catcode64=11 % @
  \def\x{\endgroup
    \expandafter\edef\csname ReFiCh@AtEnd\endcsname{%
      \endlinechar=\the\endlinechar\relax
      \catcode13=\the\catcode13\relax
      \catcode32=\the\catcode32\relax
      \catcode35=\the\catcode35\relax
      \catcode61=\the\catcode61\relax
      \catcode64=\the\catcode64\relax
      \catcode123=\the\catcode123\relax
      \catcode125=\the\catcode125\relax
    }%
  }%
\x\catcode61\catcode48\catcode32=10\relax%
\catcode13=5 % ^^M
\endlinechar=13 %
\catcode35=6 % #
\catcode64=11 % @
\catcode123=1 % {
\catcode125=2 % }
\def\TMP@EnsureCode#1#2{%
  \edef\ReFiCh@AtEnd{%
    \ReFiCh@AtEnd
    \catcode#1=\the\catcode#1\relax
  }%
  \catcode#1=#2\relax
}
\TMP@EnsureCode{39}{12}% '
\TMP@EnsureCode{40}{12}% (
\TMP@EnsureCode{41}{12}% )
\TMP@EnsureCode{42}{12}% *
\TMP@EnsureCode{44}{12}% ,
\TMP@EnsureCode{46}{12}% .
\TMP@EnsureCode{47}{12}% /
\TMP@EnsureCode{58}{12}% :
\TMP@EnsureCode{59}{12}% ;
\TMP@EnsureCode{60}{12}% <
\TMP@EnsureCode{62}{12}% >
\TMP@EnsureCode{91}{12}% [
\TMP@EnsureCode{93}{12}% ]
\TMP@EnsureCode{96}{12}% `
\edef\ReFiCh@AtEnd{\ReFiCh@AtEnd\noexpand\endinput}
%    \end{macrocode}
%
%    Package identification.
%    \begin{macrocode}
\NeedsTeXFormat{LaTeX2e}
\ProvidesPackage{rerunfilecheck}%
  [2016/05/16 v1.8 Rerun checks for auxiliary files (HO)]
%    \end{macrocode}
%
% \subsection{Options}
%
%    \begin{macrocode}
\RequirePackage{kvoptions}[2010/02/22]
\SetupKeyvalOptions{%
  family=rerunfilecheck,%
  prefix=ReFiCh@%
}
%    \end{macrocode}
%    \begin{macro}{\RerunFileCheckSetup}
%    \begin{macrocode}
\newcommand*{\RerunFileCheckSetup}{%
  \setkeys{rerunfilecheck}%
}
%    \end{macrocode}
%    \end{macro}
%    \begin{macrocode}
\DeclareBoolOption{mainaux}
\DeclareBoolOption{partaux}
\DeclareBoolOption{starttoc}
\DeclareBoolOption{index}
\DeclareBoolOption{glossary}
\define@key{rerunfilecheck}{aux}[true]{%
  \RerunFileCheckSetup{%
    mainaux={#1},%
    partaux={#1},%
    starttoc={#1},%
    index={#1},%
    glossary={#1}%
  }%
}
%    \end{macrocode}
%    \begin{macrocode}
\InputIfFileExists{rerunfilecheck.cfg}{}{}
\ProcessLocalKeyvalOptions*
%    \end{macrocode}
%
%    \begin{macro}{\ReFiCh@DisableOption}
%    \begin{macrocode}
\def\ReFiCh@DisableOption{%
  \DisableKeyvalOption[%
    action=warning,%
    package=rerunfilecheck%
  ]{rerunfilecheck}%
}
%    \end{macrocode}
%    \end{macro}
%
% \subsection{Check for checksum feature}
%
%    \begin{macrocode}
\RequirePackage{infwarerr}[2007/09/09]
\RequirePackage{pdftexcmds}[2009/04/10]
%    \end{macrocode}
%    \begin{macrocode}
\begingroup\expandafter\expandafter\expandafter\endgroup
\expandafter\ifx\csname pdf@filemdfivesum\endcsname\relax
  \@PackageInfoNoLine{rerunfilecheck}{%
    Feature \string\pdfmdfivesum\space is not available\MessageBreak
    (e.g. pdfTeX or LuaTeX with package `pdftexcmds').\MessageBreak
    Therefore file contents cannot be checked efficiently\MessageBreak
    and the loading of the package is aborted%
  }%
  \newcommand*{\RerunFileCheck}[3]{}%
  \renewcommand*{\RerunFileCheckSetup}[1]{}%
  \expandafter\ReFiCh@AtEnd
\fi%
%    \end{macrocode}
%
% \subsection{Standard \xext{aux} files}
%
%    \begin{macrocode}
\ifReFiCh@partaux
  \let\ReFiCh@org@include\@include
  \def\@include#1 {%
    \if@filesw
      \RerunFileCheck{#1.aux}{}{}%
    \fi
    \ReFiCh@org@include{#1} %
  }%
\fi
\ifReFiCh@mainaux
  \AtBeginDocument{%
    \ReFiCh@mainauxfalse
  }%
  \ifReFiCh@mainaux
    \AtEndOfPackage{%
      \RerunFileCheck{\jobname.aux}{}{}%
    }%
  \else
    \if@filesw
      \@PackageWarningNoLine{rerunfilecheck}{%
        Main aux file check is disabled,\MessageBreak
        because the file is already opened.\MessageBreak
        Load the package before \string\begin{document}%
      }%
    \fi
  \fi
\fi
\ifReFiCh@starttoc
  \let\ReFiCh@org@starttoc\@starttoc
  \def\@starttoc#1{%
    \if@filesw
      \RerunFileCheck{\jobname.#1}{%
        \@ifundefined{tf@#1}{%
        }{%
          \immediate\closeout\csname tf@#1\endcsname
        }%
      }{}%
    \fi
    \ReFiCh@org@starttoc{#1}%
  }%
\fi
\ifReFiCh@index
  \ifx\makeindex\@empty
    \@PackageWarningNoLine{rerunfilecheck}{%
      Option `index' ignored,\MessageBreak
      because \string\makeindex\space has already been called%
    }%
  \else
    \def\ReFiCh@temp{%
      \newwrite\@indexfile
      \immediate\openout\@indexfile=\jobname.idx %
      \def\index{%
        \@bsphack
        \begingroup
        \@sanitize
        \@wrindex
      }%
      \typeout{Writing index file \jobname.idx}%
      \let\makeindex\@empty
    }%
    \ifx\ReFiCh@temp\makeindex
      \def\makeindex{%
        \newwrite\@indexfile
        \RerunFileCheck{\jobname.idx}{%
          \immediate\closeout\@indexfile
        }{%
          Rerun LaTeX/makeindex to get index right%
        }%
        \immediate\openout\@indexfile=\jobname.idx %
        \def\index{%
          \@bsphack
          \begingroup
          \@sanitize
          \@wrindex
        }%
        \typeout{Writing index file \jobname.idx}%
        \let\makeindex\@empty
      }%
    \else
      \@PackageInfoNoLine{rerunfilecheck}{%
        Option `index': unsupported version of \string\makeindex
      }%
    \fi
  \fi
\fi
\ifReFiCh@glossary
  \ifx\makeglossary\@empty
    \@PackageWarningNoLine{rerunfilecheck}{%
      Option `glossary' ignored,\MessageBreak
      because \string\makeglossary\space has already been called%
    }%
  \else
    \def\ReFiCh@temp{%
      \newwrite\@glossaryfile
      \immediate\openout\@glossaryfile=\jobname.glo %
      \def\glossary{%
        \@bsphack
        \begingroup
        \@sanitize
        \@wrglossary
      }%
      \typeout{Writing glossary file \jobname.glo }%
      \let\makeglossary\@empty
    }%
    \ifx\ReFiCh@temp\makeglossary
      \def\ReFiCh@temp{%
        \newwrite\@glossaryfile
        \RerunFileCheck{\jobname.glo}{%
          \immediate\closeout\@glossaryfile
        }{%
          Rerun LaTeX/makeindex to get glossary right%
        }%
        \immediate\openout\@glossaryfile=\jobname.glo %
        \def\glossary{%
          \@bsphack
          \begingroup
          \@sanitize
          \@wrglossary
        }%
        \typeout{Writing glossary file \jobname.glo}%
        \let\makeglossary\@empty
      }%
    \else
      \@PackageInfoNoLine{rerunfilecheck}{%
        Option `glossary': unsupported version of \string\makeglossary
      }%
    \fi
  \fi
\fi
\ReFiCh@DisableOption{mainaux}
\ReFiCh@DisableOption{partaux}
\ReFiCh@DisableOption{starttoc}
\ReFiCh@DisableOption{index}
\ReFiCh@DisableOption{glossary}
\ReFiCh@DisableOption{aux}
%    \end{macrocode}
%
% \subsection{Rerun check}
%
%    \begin{macrocode}
\RequirePackage{atveryend}[2016/05/16]
\RequirePackage{uniquecounter}[2009/12/18]
%    \end{macrocode}
%
%    \begin{macro}{\ReFiCh@CheckSum}
%    \begin{macrocode}
\begingroup\expandafter\expandafter\expandafter\endgroup
\expandafter\ifx\csname pdf@filesize\endcsname\relax
  \def\ReFiCh@CheckSum{%
    \pdf@filemdfivesum
  }%
\else
  \def\ReFiCh@CheckSum#1{%
    \pdf@filemdfivesum{#1}%
    \ReFiCh@Separator
    \pdf@filesize{#1}%
  }%
\fi
%    \end{macrocode}
%    \end{macro}
%    \begin{macro}{\ReFiCh@NoFile}
%    \begin{macrocode}
\def\ReFiCh@Separator{;}
%    \end{macrocode}
%    \end{macro}
%    \begin{macro}{\ReFiCh@NoFile}
%    \begin{macrocode}
\def\ReFiCh@NoFile{<no file>}
%    \end{macrocode}
%    \end{macro}
%
%    \begin{macrocode}
\UniqueCounterNew{rerunfilecheck}
%    \end{macrocode}
%
%    \begin{macro}{\RerunFileCheck}
%    \begin{macrocode}
\newcommand*{\RerunFileCheck}{%
  \UniqueCounterCall{rerunfilecheck}\ReFiCh@RerunFileCheck
}
%    \end{macrocode}
%    \end{macro}
%    \begin{macro}{\ReFiCh@RerunFileCheck}
%    \begin{macrocode}
\def\ReFiCh@RerunFileCheck#1{%
  \expandafter\ReFiCh@@RerunFileCheck\csname ReFiCh@#1\endcsname
}
%    \end{macrocode}
%    \end{macro}
%    \begin{macro}{\ReFiCh@Check}
%    \begin{macrocode}
\def\ReFiCh@Check#1#2#3{%
%  \IfFileExists{#3}{%
    #1\edef#2{\ReFiCh@CheckSum{#3}}%
    \ifx#2\ReFiCh@Separator
      #1\let#2\ReFiCh@NoFile
    \fi
%  }{%
%    #1\let#2\ReFiCh@NoFile
%  }%
}
%    \end{macrocode}
%    \end{macro}
%    \begin{macro}{\ReFiCh@@RerunFileCheck}
%    \begin{macrocode}
\def\ReFiCh@@RerunFileCheck#1#2#3#4{%
  \ReFiCh@Check\global#1{#2}%
  \AtEndAfterFileList{%
    \begingroup
      #3%
      \ReFiCh@Check{}\x{#2}%
      \ifx#1\x
        \@PackageInfoNoLine{rerunfilecheck}{%
          File `#2' has not changed.\MessageBreak
          Checksum: \x
        }%
      \else
        \ifnum
          \ReFiCh@IsAux#2\relax.aux\relax\@nil
          \ifx#1\ReFiCh@NoFile 1\else 0\fi
          \ifx\x\ReFiCh@AuxEmptyUnix 1%
          \else
            \ifx\x\ReFiCh@AuxEmptyDos 1\fi
          \fi
        =111 %
          \@PackageInfoNoLine{rerunfilecheck}{%
            File `#2' is empty .aux file.\MessageBreak
            Before: #1\MessageBreak
            After: \space\x
          }%
        \else
          \@PackageWarningNoLine{rerunfilecheck}{%
            File `#2' has changed.%
            \ifx\\#4\\%
              \space Rerun%
            \else
              \MessageBreak
              #4%
            \fi
          }%
          \@PackageInfoNoLine{rerunfilecheck}{%
            Checksums for `#2':\MessageBreak
            Before: #1\MessageBreak
            After: \space\x
          }%
        \fi
      \fi
    \endgroup
  }%
}
%    \end{macrocode}
%    \end{macro}
%    \begin{macrocode}
\def\ReFiCh@IsAux#1.aux\relax#2\@nil{%
  \ifx\hbox#2\hbox
    0%
  \else
    1%
  \fi
}
%    \end{macrocode}
%    \begin{macrocode}
\def\ReFiCh@AuxEmptyUnix{A94A2480D3289E625EEA47CD1B285758;8}%
\@onelevel@sanitize\ReFiCh@AuxEmptyUnix
%    \end{macrocode}
%    \begin{macrocode}
\def\ReFiCh@AuxEmptyDos{A62A15ECE803E2EBE94952FCC9933BC0;9}%
\@onelevel@sanitize\ReFiCh@AuxEmptyDos
%    \end{macrocode}
%
%    \begin{macrocode}
\ReFiCh@AtEnd%
%</package>
%    \end{macrocode}
%
% \section{Test}
%
%    \begin{macrocode}
%<*test1>
\def\LoadCommand{\RequirePackage{rerunfilecheck}[2016/05/16]}
%</test1>
%    \end{macrocode}
% \subsection{Catcode checks for loading}
%
%    \begin{macrocode}
%<*test1>
%    \end{macrocode}
%    \begin{macrocode}
\catcode`\{=1 %
\catcode`\}=2 %
\catcode`\#=6 %
\catcode`\@=11 %
\expandafter\ifx\csname count@\endcsname\relax
  \countdef\count@=255 %
\fi
\expandafter\ifx\csname @gobble\endcsname\relax
  \long\def\@gobble#1{}%
\fi
\expandafter\ifx\csname @firstofone\endcsname\relax
  \long\def\@firstofone#1{#1}%
\fi
\expandafter\ifx\csname loop\endcsname\relax
  \expandafter\@firstofone
\else
  \expandafter\@gobble
\fi
{%
  \def\loop#1\repeat{%
    \def\body{#1}%
    \iterate
  }%
  \def\iterate{%
    \body
      \let\next\iterate
    \else
      \let\next\relax
    \fi
    \next
  }%
  \let\repeat=\fi
}%
\def\RestoreCatcodes{}
\count@=0 %
\loop
  \edef\RestoreCatcodes{%
    \RestoreCatcodes
    \catcode\the\count@=\the\catcode\count@\relax
  }%
\ifnum\count@<255 %
  \advance\count@ 1 %
\repeat

\def\RangeCatcodeInvalid#1#2{%
  \count@=#1\relax
  \loop
    \catcode\count@=15 %
  \ifnum\count@<#2\relax
    \advance\count@ 1 %
  \repeat
}
\def\RangeCatcodeCheck#1#2#3{%
  \count@=#1\relax
  \loop
    \ifnum#3=\catcode\count@
    \else
      \errmessage{%
        Character \the\count@\space
        with wrong catcode \the\catcode\count@\space
        instead of \number#3%
      }%
    \fi
  \ifnum\count@<#2\relax
    \advance\count@ 1 %
  \repeat
}
\def\space{ }
\expandafter\ifx\csname LoadCommand\endcsname\relax
  \def\LoadCommand{\input rerunfilecheck.sty\relax}%
\fi
\def\Test{%
  \RangeCatcodeInvalid{0}{47}%
  \RangeCatcodeInvalid{58}{64}%
  \RangeCatcodeInvalid{91}{96}%
  \RangeCatcodeInvalid{123}{255}%
  \catcode`\@=12 %
  \catcode`\\=0 %
  \catcode`\%=14 %
  \LoadCommand
  \RangeCatcodeCheck{0}{36}{15}%
  \RangeCatcodeCheck{37}{37}{14}%
  \RangeCatcodeCheck{38}{47}{15}%
  \RangeCatcodeCheck{48}{57}{12}%
  \RangeCatcodeCheck{58}{63}{15}%
  \RangeCatcodeCheck{64}{64}{12}%
  \RangeCatcodeCheck{65}{90}{11}%
  \RangeCatcodeCheck{91}{91}{15}%
  \RangeCatcodeCheck{92}{92}{0}%
  \RangeCatcodeCheck{93}{96}{15}%
  \RangeCatcodeCheck{97}{122}{11}%
  \RangeCatcodeCheck{123}{255}{15}%
  \RestoreCatcodes
}
\Test
\csname @@end\endcsname
\end
%    \end{macrocode}
%    \begin{macrocode}
%</test1>
%    \end{macrocode}
%
% \section{Installation}
%
% \subsection{Download}
%
% \paragraph{Package.} This package is available on
% CTAN\footnote{\url{http://ctan.org/pkg/rerunfilecheck}}:
% \begin{description}
% \item[\CTAN{macros/latex/contrib/oberdiek/rerunfilecheck.dtx}] The source file.
% \item[\CTAN{macros/latex/contrib/oberdiek/rerunfilecheck.pdf}] Documentation.
% \end{description}
%
%
% \paragraph{Bundle.} All the packages of the bundle `oberdiek'
% are also available in a TDS compliant ZIP archive. There
% the packages are already unpacked and the documentation files
% are generated. The files and directories obey the TDS standard.
% \begin{description}
% \item[\CTAN{install/macros/latex/contrib/oberdiek.tds.zip}]
% \end{description}
% \emph{TDS} refers to the standard ``A Directory Structure
% for \TeX\ Files'' (\CTAN{tds/tds.pdf}). Directories
% with \xfile{texmf} in their name are usually organized this way.
%
% \subsection{Bundle installation}
%
% \paragraph{Unpacking.} Unpack the \xfile{oberdiek.tds.zip} in the
% TDS tree (also known as \xfile{texmf} tree) of your choice.
% Example (linux):
% \begin{quote}
%   |unzip oberdiek.tds.zip -d ~/texmf|
% \end{quote}
%
% \paragraph{Script installation.}
% Check the directory \xfile{TDS:scripts/oberdiek/} for
% scripts that need further installation steps.
% Package \xpackage{attachfile2} comes with the Perl script
% \xfile{pdfatfi.pl} that should be installed in such a way
% that it can be called as \texttt{pdfatfi}.
% Example (linux):
% \begin{quote}
%   |chmod +x scripts/oberdiek/pdfatfi.pl|\\
%   |cp scripts/oberdiek/pdfatfi.pl /usr/local/bin/|
% \end{quote}
%
% \subsection{Package installation}
%
% \paragraph{Unpacking.} The \xfile{.dtx} file is a self-extracting
% \docstrip\ archive. The files are extracted by running the
% \xfile{.dtx} through \plainTeX:
% \begin{quote}
%   \verb|tex rerunfilecheck.dtx|
% \end{quote}
%
% \paragraph{TDS.} Now the different files must be moved into
% the different directories in your installation TDS tree
% (also known as \xfile{texmf} tree):
% \begin{quote}
% \def\t{^^A
% \begin{tabular}{@{}>{\ttfamily}l@{ $\rightarrow$ }>{\ttfamily}l@{}}
%   rerunfilecheck.sty & tex/latex/oberdiek/rerunfilecheck.sty\\
%   rerunfilecheck.pdf & doc/latex/oberdiek/rerunfilecheck.pdf\\
%   rerunfilecheck-example.cfg & doc/latex/oberdiek/rerunfilecheck-example.cfg\\
%   test/rerunfilecheck-test1.tex & doc/latex/oberdiek/test/rerunfilecheck-test1.tex\\
%   rerunfilecheck.dtx & source/latex/oberdiek/rerunfilecheck.dtx\\
% \end{tabular}^^A
% }^^A
% \sbox0{\t}^^A
% \ifdim\wd0>\linewidth
%   \begingroup
%     \advance\linewidth by\leftmargin
%     \advance\linewidth by\rightmargin
%   \edef\x{\endgroup
%     \def\noexpand\lw{\the\linewidth}^^A
%   }\x
%   \def\lwbox{^^A
%     \leavevmode
%     \hbox to \linewidth{^^A
%       \kern-\leftmargin\relax
%       \hss
%       \usebox0
%       \hss
%       \kern-\rightmargin\relax
%     }^^A
%   }^^A
%   \ifdim\wd0>\lw
%     \sbox0{\small\t}^^A
%     \ifdim\wd0>\linewidth
%       \ifdim\wd0>\lw
%         \sbox0{\footnotesize\t}^^A
%         \ifdim\wd0>\linewidth
%           \ifdim\wd0>\lw
%             \sbox0{\scriptsize\t}^^A
%             \ifdim\wd0>\linewidth
%               \ifdim\wd0>\lw
%                 \sbox0{\tiny\t}^^A
%                 \ifdim\wd0>\linewidth
%                   \lwbox
%                 \else
%                   \usebox0
%                 \fi
%               \else
%                 \lwbox
%               \fi
%             \else
%               \usebox0
%             \fi
%           \else
%             \lwbox
%           \fi
%         \else
%           \usebox0
%         \fi
%       \else
%         \lwbox
%       \fi
%     \else
%       \usebox0
%     \fi
%   \else
%     \lwbox
%   \fi
% \else
%   \usebox0
% \fi
% \end{quote}
% If you have a \xfile{docstrip.cfg} that configures and enables \docstrip's
% TDS installing feature, then some files can already be in the right
% place, see the documentation of \docstrip.
%
% \subsection{Refresh file name databases}
%
% If your \TeX~distribution
% (\teTeX, \mikTeX, \dots) relies on file name databases, you must refresh
% these. For example, \teTeX\ users run \verb|texhash| or
% \verb|mktexlsr|.
%
% \subsection{Some details for the interested}
%
% \paragraph{Attached source.}
%
% The PDF documentation on CTAN also includes the
% \xfile{.dtx} source file. It can be extracted by
% AcrobatReader 6 or higher. Another option is \textsf{pdftk},
% e.g. unpack the file into the current directory:
% \begin{quote}
%   \verb|pdftk rerunfilecheck.pdf unpack_files output .|
% \end{quote}
%
% \paragraph{Unpacking with \LaTeX.}
% The \xfile{.dtx} chooses its action depending on the format:
% \begin{description}
% \item[\plainTeX:] Run \docstrip\ and extract the files.
% \item[\LaTeX:] Generate the documentation.
% \end{description}
% If you insist on using \LaTeX\ for \docstrip\ (really,
% \docstrip\ does not need \LaTeX), then inform the autodetect routine
% about your intention:
% \begin{quote}
%   \verb|latex \let\install=y\input{rerunfilecheck.dtx}|
% \end{quote}
% Do not forget to quote the argument according to the demands
% of your shell.
%
% \paragraph{Generating the documentation.}
% You can use both the \xfile{.dtx} or the \xfile{.drv} to generate
% the documentation. The process can be configured by the
% configuration file \xfile{ltxdoc.cfg}. For instance, put this
% line into this file, if you want to have A4 as paper format:
% \begin{quote}
%   \verb|\PassOptionsToClass{a4paper}{article}|
% \end{quote}
% An example follows how to generate the
% documentation with pdf\LaTeX:
% \begin{quote}
%\begin{verbatim}
%pdflatex rerunfilecheck.dtx
%makeindex -s gind.ist rerunfilecheck.idx
%pdflatex rerunfilecheck.dtx
%makeindex -s gind.ist rerunfilecheck.idx
%pdflatex rerunfilecheck.dtx
%\end{verbatim}
% \end{quote}
%
% \section{Catalogue}
%
% The following XML file can be used as source for the
% \href{http://mirror.ctan.org/help/Catalogue/catalogue.html}{\TeX\ Catalogue}.
% The elements \texttt{caption} and \texttt{description} are imported
% from the original XML file from the Catalogue.
% The name of the XML file in the Catalogue is \xfile{rerunfilecheck.xml}.
%    \begin{macrocode}
%<*catalogue>
<?xml version='1.0' encoding='us-ascii'?>
<!DOCTYPE entry SYSTEM 'catalogue.dtd'>
<entry datestamp='$Date$' modifier='$Author$' id='rerunfilecheck'>
  <name>rerunfilecheck</name>
  <caption>Checksum based rerun checks on auxiliary files.</caption>
  <authorref id='auth:oberdiek'/>
  <copyright owner='Heiko Oberdiek' year='2009-2011'/>
  <license type='lppl1.3'/>
  <version number='1.8'/>
  <description>
    The package provides additional rerun warnings if some
    auxiliary files have changed. It is based on MD5 checksum,
    provided by pdfTeX.
    <p/>
    The package is part of the <xref refid='oberdiek'>oberdiek</xref> bundle.
  </description>
  <documentation details='Package documentation'
      href='ctan:/macros/latex/contrib/oberdiek/rerunfilecheck.pdf'/>
  <ctan file='true' path='/macros/latex/contrib/oberdiek/rerunfilecheck.dtx'/>
  <miktex location='oberdiek'/>
  <texlive location='oberdiek'/>
  <install path='/macros/latex/contrib/oberdiek/oberdiek.tds.zip'/>
</entry>
%</catalogue>
%    \end{macrocode}
%
% \begin{History}
%   \begin{Version}{2009/12/10 v1.0}
%   \item
%     The first version.
%   \end{Version}
%   \begin{Version}{2009/12/12 v1.1}
%   \item
%     Short info shortened.
%   \end{Version}
%   \begin{Version}{2009/12/18 v1.2}
%   \item
%     Required date for package \xpackage{uniquecounter} updated
%     because of bug in this package.
%   \end{Version}
%   \begin{Version}{2010/01/25 v1.3}
%   \item
%     Moved from \texttt{TDS:*/generic/*} to \texttt{TDS:*/latex/*}.
%   \end{Version}
%   \begin{Version}{2010/02/22 v1.4}
%   \item
%     The options of this package are recognized only if they are
%     package options. Global options are ignored. This avoids
%     name clashes with class and other package options
%     (for example, class option `index=totoc').
%   \end{Version}
%   \begin{Version}{2010/03/15 v1.5}
%   \item
%     Call of \cs{pdffivesum} is wrapped in \cs{IfFileExists} to avoid
%     calls of \textsf{mktextex} if this feature is enabled.
%     However \cs{IfFileExists} has file name limitations.
%   \end{Version}
%   \begin{Version}{2010/03/16 v1.6}
%   \item
%     Reverted to version 1.4 and \cs{IfFileExists} wrapper of
%     version 1.5 is removed.
%   \end{Version}
%   \begin{Version}{2011/04/15 v1.7}
%   \item
%     Using \cs{AtEndAfterFileList} of package \xpackage{atveryend}
%     2011/04/15 v1.6 instead of \cs{AtVeryEndDocument}.
%   \end{Version}
%   \begin{Version}{2016/05/16 v1.8}
%   \item
%     Documentation updates.
%   \end{Version}
% \end{History}
%
% \PrintIndex
%
% \Finale
\endinput

%        (quote the arguments according to the demands of your shell)
%
% Documentation:
%    (a) If rerunfilecheck.drv is present:
%           latex rerunfilecheck.drv
%    (b) Without rerunfilecheck.drv:
%           latex rerunfilecheck.dtx; ...
%    The class ltxdoc loads the configuration file ltxdoc.cfg
%    if available. Here you can specify further options, e.g.
%    use A4 as paper format:
%       \PassOptionsToClass{a4paper}{article}
%
%    Programm calls to get the documentation (example):
%       pdflatex rerunfilecheck.dtx
%       makeindex -s gind.ist rerunfilecheck.idx
%       pdflatex rerunfilecheck.dtx
%       makeindex -s gind.ist rerunfilecheck.idx
%       pdflatex rerunfilecheck.dtx
%
% Installation:
%    TDS:tex/latex/oberdiek/rerunfilecheck.sty
%    TDS:doc/latex/oberdiek/rerunfilecheck.pdf
%    TDS:doc/latex/oberdiek/rerunfilecheck-example.cfg
%    TDS:doc/latex/oberdiek/test/rerunfilecheck-test1.tex
%    TDS:source/latex/oberdiek/rerunfilecheck.dtx
%
%<*ignore>
\begingroup
  \catcode123=1 %
  \catcode125=2 %
  \def\x{LaTeX2e}%
\expandafter\endgroup
\ifcase 0\ifx\install y1\fi\expandafter
         \ifx\csname processbatchFile\endcsname\relax\else1\fi
         \ifx\fmtname\x\else 1\fi\relax
\else\csname fi\endcsname
%</ignore>
%<*install>
\input docstrip.tex
\Msg{************************************************************************}
\Msg{* Installation}
\Msg{* Package: rerunfilecheck 2016/05/16 v1.8 Rerun checks for auxiliary files (HO)}
\Msg{************************************************************************}

\keepsilent
\askforoverwritefalse

\let\MetaPrefix\relax
\preamble

This is a generated file.

Project: rerunfilecheck
Version: 2016/05/16 v1.8

Copyright (C) 2009-2011 by
   Heiko Oberdiek <heiko.oberdiek at googlemail.com>

This work may be distributed and/or modified under the
conditions of the LaTeX Project Public License, either
version 1.3c of this license or (at your option) any later
version. This version of this license is in
   http://www.latex-project.org/lppl/lppl-1-3c.txt
and the latest version of this license is in
   http://www.latex-project.org/lppl.txt
and version 1.3 or later is part of all distributions of
LaTeX version 2005/12/01 or later.

This work has the LPPL maintenance status "maintained".

This Current Maintainer of this work is Heiko Oberdiek.

This work consists of the main source file rerunfilecheck.dtx
and the derived files
   rerunfilecheck.sty, rerunfilecheck.pdf, rerunfilecheck.ins,
   rerunfilecheck.drv, rerunfilecheck-example.cfg,
   rerunfilecheck-test1.tex.

\endpreamble
\let\MetaPrefix\DoubleperCent

\generate{%
  \file{rerunfilecheck.ins}{\from{rerunfilecheck.dtx}{install}}%
  \file{rerunfilecheck.drv}{\from{rerunfilecheck.dtx}{driver}}%
  \usedir{tex/latex/oberdiek}%
  \file{rerunfilecheck.sty}{\from{rerunfilecheck.dtx}{package}}%
%  \usedir{doc/latex/oberdiek/test}%
%  \file{rerunfilecheck-test1.tex}{\from{rerunfilecheck.dtx}{test1}}%
  \nopreamble
  \nopostamble
  \usedir{doc/latex/oberdiek}%
  \file{rerunfilecheck-example.cfg}{\from{rerunfilecheck.dtx}{cfg}}%
  \usedir{source/latex/oberdiek/catalogue}%
  \file{rerunfilecheck.xml}{\from{rerunfilecheck.dtx}{catalogue}}%
}

\catcode32=13\relax% active space
\let =\space%
\Msg{************************************************************************}
\Msg{*}
\Msg{* To finish the installation you have to move the following}
\Msg{* file into a directory searched by TeX:}
\Msg{*}
\Msg{*     rerunfilecheck.sty}
\Msg{*}
\Msg{* To produce the documentation run the file `rerunfilecheck.drv'}
\Msg{* through LaTeX.}
\Msg{*}
\Msg{* Happy TeXing!}
\Msg{*}
\Msg{************************************************************************}

\endbatchfile
%</install>
%<*ignore>
\fi
%</ignore>
%<*driver>
\NeedsTeXFormat{LaTeX2e}
\ProvidesFile{rerunfilecheck.drv}%
  [2016/05/16 v1.8 Rerun checks for auxiliary files (HO)]%
\documentclass{ltxdoc}
\usepackage{holtxdoc}[2011/11/22]
\begin{document}
  \DocInput{rerunfilecheck.dtx}%
\end{document}
%</driver>
% \fi
%
%
% \CharacterTable
%  {Upper-case    \A\B\C\D\E\F\G\H\I\J\K\L\M\N\O\P\Q\R\S\T\U\V\W\X\Y\Z
%   Lower-case    \a\b\c\d\e\f\g\h\i\j\k\l\m\n\o\p\q\r\s\t\u\v\w\x\y\z
%   Digits        \0\1\2\3\4\5\6\7\8\9
%   Exclamation   \!     Double quote  \"     Hash (number) \#
%   Dollar        \$     Percent       \%     Ampersand     \&
%   Acute accent  \'     Left paren    \(     Right paren   \)
%   Asterisk      \*     Plus          \+     Comma         \,
%   Minus         \-     Point         \.     Solidus       \/
%   Colon         \:     Semicolon     \;     Less than     \<
%   Equals        \=     Greater than  \>     Question mark \?
%   Commercial at \@     Left bracket  \[     Backslash     \\
%   Right bracket \]     Circumflex    \^     Underscore    \_
%   Grave accent  \`     Left brace    \{     Vertical bar  \|
%   Right brace   \}     Tilde         \~}
%
% \GetFileInfo{rerunfilecheck.drv}
%
% \title{The \xpackage{rerunfilecheck} package}
% \date{2016/05/16 v1.8}
% \author{Heiko Oberdiek\thanks
% {Please report any issues at https://github.com/ho-tex/oberdiek/issues}\\
% \xemail{heiko.oberdiek at googlemail.com}}
%
% \maketitle
%
% \begin{abstract}
% The package provides additional rerun warnings if some
% auxiliary files have changed. It is based on MD5 checksum,
% provided by \pdfTeX.
% \end{abstract}
%
% \tableofcontents
%
% \section{Documentation}
%
% \LaTeX\ informs the user, when to run \LaTeX\ again, if the
% references have changed. It has the old references from the
% first reading of the \xext{aux} files already in memory, thus
% it compares them with the new version of the \xext{aux} file
% at the end of the document. However this rerun warnings are
% not given for the table of contents and other data stored
% in the \xext{aux} files or other auxiliary files.
% Usually many of these data as the table of contents is not keep
% in memory. If someone wants to detect changes, he has either
% to keep the data in memory. This does not scale well with huge
% documents. Or he copies the file before they are changed. Slow
% I/O operations cost time.
%
% Since version 1.30.0 \pdfTeX\ provides \cs{pdfmdfivesum}
% and \cs{pdffilesize}.
% These features are also available in \LuaTeX, provided by package
% \xpackage{pdftexcmds}.
% Thus this package \xpackage{rerunfilecheck} uses these features to
% detect file changes. This saves the packages from keeping
% the whole files in memory or in file copies.
% The drawback are different files with the same size and the
% same MD5 checksum (seldom, hopefully).
%
% \subsection{Options}
%
% All options are key value options of boolean type.
% No option or |true| turns an option on, |false| disables
% an option.
% \begin{description}
% \item[\xoption{mainaux}:]
%   Check the main \xext{aux} file.
% \item[\xoption{partaux}:]
%   Check the \xext{aux} files from \cs{include} files.
% \item[\xoption{starttoc}:]
%   Add the rerun checks in \cs{@starttoc} that is called by
%   \cs{tableofcontents}, \cs{listoffigures}, \dots
% \item[\xoption{index}, \xoption{glossary}:]
%   \LaTeX's original \cs{makeindex} and \cs{makeglossary} are
%   redefined to add the rerun checks. The options do not have
%   an effect, if \cs{makeindex}/\cs{makeglossary} are already
%   called or if a package or class had redefined or will
%   redefine them.
% \item[\xoption{aux}:]
%   This option turns all previous options on or off.
%   ``aux'' means auxiliary file.
% \end{description}
% The default for the options is |false|, because
% some internals must be redefined to insert the
% rerun checks.
% The options can be set in \cs{usepackage} or
% the configuration file \xfile{rerunfilecheck.cfg}.
% Global options are ignored (since 1.4).
%
% \begin{declcs}{RerunFileCheckSetup} \M{key value list}
% \end{declcs}
% Options can also be set using \cs{RerunFileCheckSetup}.
% Currently all options are disabled after the package
% is loaded. Thus \cs{RerunFileCheckSetup} makes sense in
% the configuration file only.
%
% Example for the configuration file:
%    \begin{macrocode}
%<*cfg>
\ProvidesFile{rerunfilecheck.cfg}[2016/05/16 Default configuration]%
\RerunFileCheckSetup{aux}
%</cfg>
%    \end{macrocode}
%
% \subsection{Interface for class/package authors}
%
% \begin{declcs}{RerunFileCheck} \M{file} \M{file closing action} \M{rerun warning}
% \end{declcs}
% If you want to add a rerun check, call \cs{RerunFileCheck} right before
% an output file is opened for writing. The macro first remembers
% the current checksum of \meta{file}.
% The file is checked again right before the end of the job.
% Macro \cs{AtVeryEndDocument} of package \xpackage{atveryend}
% is used to place the check after the main aux file is closed
% in |\end{document}|. Before reading the file again, it must be
% closed. Provide the code for closing in argument \meta{file closing action}.
% Do not forget \cs{immediate} before \cs{openout}. Otherwise the
% closing action would be delayed to the next shipout that never happens
% (the last page is already shipped out).
% If the file has changed, \cs{RerunFileCheck} informs the user with
% a warning that the file has changed and says the magic word ``Rerun''.
% If the last argument \meta{rerun warning} is not empty, then
% the rerun sentence is replaced by it. Usually the phrase
% ``to get something right'' is added.
% As example the relevant part of the redefined \cs{makeindex}
% is shown, see package code:
%\begin{quote}
%\begin{verbatim}
%\newwrite\@indexfile
%\RerunFileCheck{\jobname.idx}{%
%  \immediate\closeout\@indexfile
%}{%
%  Rerun LaTeX/makeindex to get index right%
%}%
%\immediate\openout\@indexfile=\jobname.idx %
%\end{verbatim}
%\end{quote}
%
% \StopEventually{
% }
%
% \section{Implementation}
%    \begin{macrocode}
%<*package>
%    \end{macrocode}
%
%    \begin{macrocode}
\begingroup\catcode61\catcode48\catcode32=10\relax%
  \catcode13=5 % ^^M
  \endlinechar=13 %
  \catcode123=1 % {
  \catcode125=2 % }
  \catcode64=11 % @
  \def\x{\endgroup
    \expandafter\edef\csname ReFiCh@AtEnd\endcsname{%
      \endlinechar=\the\endlinechar\relax
      \catcode13=\the\catcode13\relax
      \catcode32=\the\catcode32\relax
      \catcode35=\the\catcode35\relax
      \catcode61=\the\catcode61\relax
      \catcode64=\the\catcode64\relax
      \catcode123=\the\catcode123\relax
      \catcode125=\the\catcode125\relax
    }%
  }%
\x\catcode61\catcode48\catcode32=10\relax%
\catcode13=5 % ^^M
\endlinechar=13 %
\catcode35=6 % #
\catcode64=11 % @
\catcode123=1 % {
\catcode125=2 % }
\def\TMP@EnsureCode#1#2{%
  \edef\ReFiCh@AtEnd{%
    \ReFiCh@AtEnd
    \catcode#1=\the\catcode#1\relax
  }%
  \catcode#1=#2\relax
}
\TMP@EnsureCode{39}{12}% '
\TMP@EnsureCode{40}{12}% (
\TMP@EnsureCode{41}{12}% )
\TMP@EnsureCode{42}{12}% *
\TMP@EnsureCode{44}{12}% ,
\TMP@EnsureCode{46}{12}% .
\TMP@EnsureCode{47}{12}% /
\TMP@EnsureCode{58}{12}% :
\TMP@EnsureCode{59}{12}% ;
\TMP@EnsureCode{60}{12}% <
\TMP@EnsureCode{62}{12}% >
\TMP@EnsureCode{91}{12}% [
\TMP@EnsureCode{93}{12}% ]
\TMP@EnsureCode{96}{12}% `
\edef\ReFiCh@AtEnd{\ReFiCh@AtEnd\noexpand\endinput}
%    \end{macrocode}
%
%    Package identification.
%    \begin{macrocode}
\NeedsTeXFormat{LaTeX2e}
\ProvidesPackage{rerunfilecheck}%
  [2016/05/16 v1.8 Rerun checks for auxiliary files (HO)]
%    \end{macrocode}
%
% \subsection{Options}
%
%    \begin{macrocode}
\RequirePackage{kvoptions}[2010/02/22]
\SetupKeyvalOptions{%
  family=rerunfilecheck,%
  prefix=ReFiCh@%
}
%    \end{macrocode}
%    \begin{macro}{\RerunFileCheckSetup}
%    \begin{macrocode}
\newcommand*{\RerunFileCheckSetup}{%
  \setkeys{rerunfilecheck}%
}
%    \end{macrocode}
%    \end{macro}
%    \begin{macrocode}
\DeclareBoolOption{mainaux}
\DeclareBoolOption{partaux}
\DeclareBoolOption{starttoc}
\DeclareBoolOption{index}
\DeclareBoolOption{glossary}
\define@key{rerunfilecheck}{aux}[true]{%
  \RerunFileCheckSetup{%
    mainaux={#1},%
    partaux={#1},%
    starttoc={#1},%
    index={#1},%
    glossary={#1}%
  }%
}
%    \end{macrocode}
%    \begin{macrocode}
\InputIfFileExists{rerunfilecheck.cfg}{}{}
\ProcessLocalKeyvalOptions*
%    \end{macrocode}
%
%    \begin{macro}{\ReFiCh@DisableOption}
%    \begin{macrocode}
\def\ReFiCh@DisableOption{%
  \DisableKeyvalOption[%
    action=warning,%
    package=rerunfilecheck%
  ]{rerunfilecheck}%
}
%    \end{macrocode}
%    \end{macro}
%
% \subsection{Check for checksum feature}
%
%    \begin{macrocode}
\RequirePackage{infwarerr}[2007/09/09]
\RequirePackage{pdftexcmds}[2009/04/10]
%    \end{macrocode}
%    \begin{macrocode}
\begingroup\expandafter\expandafter\expandafter\endgroup
\expandafter\ifx\csname pdf@filemdfivesum\endcsname\relax
  \@PackageInfoNoLine{rerunfilecheck}{%
    Feature \string\pdfmdfivesum\space is not available\MessageBreak
    (e.g. pdfTeX or LuaTeX with package `pdftexcmds').\MessageBreak
    Therefore file contents cannot be checked efficiently\MessageBreak
    and the loading of the package is aborted%
  }%
  \newcommand*{\RerunFileCheck}[3]{}%
  \renewcommand*{\RerunFileCheckSetup}[1]{}%
  \expandafter\ReFiCh@AtEnd
\fi%
%    \end{macrocode}
%
% \subsection{Standard \xext{aux} files}
%
%    \begin{macrocode}
\ifReFiCh@partaux
  \let\ReFiCh@org@include\@include
  \def\@include#1 {%
    \if@filesw
      \RerunFileCheck{#1.aux}{}{}%
    \fi
    \ReFiCh@org@include{#1} %
  }%
\fi
\ifReFiCh@mainaux
  \AtBeginDocument{%
    \ReFiCh@mainauxfalse
  }%
  \ifReFiCh@mainaux
    \AtEndOfPackage{%
      \RerunFileCheck{\jobname.aux}{}{}%
    }%
  \else
    \if@filesw
      \@PackageWarningNoLine{rerunfilecheck}{%
        Main aux file check is disabled,\MessageBreak
        because the file is already opened.\MessageBreak
        Load the package before \string\begin{document}%
      }%
    \fi
  \fi
\fi
\ifReFiCh@starttoc
  \let\ReFiCh@org@starttoc\@starttoc
  \def\@starttoc#1{%
    \if@filesw
      \RerunFileCheck{\jobname.#1}{%
        \@ifundefined{tf@#1}{%
        }{%
          \immediate\closeout\csname tf@#1\endcsname
        }%
      }{}%
    \fi
    \ReFiCh@org@starttoc{#1}%
  }%
\fi
\ifReFiCh@index
  \ifx\makeindex\@empty
    \@PackageWarningNoLine{rerunfilecheck}{%
      Option `index' ignored,\MessageBreak
      because \string\makeindex\space has already been called%
    }%
  \else
    \def\ReFiCh@temp{%
      \newwrite\@indexfile
      \immediate\openout\@indexfile=\jobname.idx %
      \def\index{%
        \@bsphack
        \begingroup
        \@sanitize
        \@wrindex
      }%
      \typeout{Writing index file \jobname.idx}%
      \let\makeindex\@empty
    }%
    \ifx\ReFiCh@temp\makeindex
      \def\makeindex{%
        \newwrite\@indexfile
        \RerunFileCheck{\jobname.idx}{%
          \immediate\closeout\@indexfile
        }{%
          Rerun LaTeX/makeindex to get index right%
        }%
        \immediate\openout\@indexfile=\jobname.idx %
        \def\index{%
          \@bsphack
          \begingroup
          \@sanitize
          \@wrindex
        }%
        \typeout{Writing index file \jobname.idx}%
        \let\makeindex\@empty
      }%
    \else
      \@PackageInfoNoLine{rerunfilecheck}{%
        Option `index': unsupported version of \string\makeindex
      }%
    \fi
  \fi
\fi
\ifReFiCh@glossary
  \ifx\makeglossary\@empty
    \@PackageWarningNoLine{rerunfilecheck}{%
      Option `glossary' ignored,\MessageBreak
      because \string\makeglossary\space has already been called%
    }%
  \else
    \def\ReFiCh@temp{%
      \newwrite\@glossaryfile
      \immediate\openout\@glossaryfile=\jobname.glo %
      \def\glossary{%
        \@bsphack
        \begingroup
        \@sanitize
        \@wrglossary
      }%
      \typeout{Writing glossary file \jobname.glo }%
      \let\makeglossary\@empty
    }%
    \ifx\ReFiCh@temp\makeglossary
      \def\ReFiCh@temp{%
        \newwrite\@glossaryfile
        \RerunFileCheck{\jobname.glo}{%
          \immediate\closeout\@glossaryfile
        }{%
          Rerun LaTeX/makeindex to get glossary right%
        }%
        \immediate\openout\@glossaryfile=\jobname.glo %
        \def\glossary{%
          \@bsphack
          \begingroup
          \@sanitize
          \@wrglossary
        }%
        \typeout{Writing glossary file \jobname.glo}%
        \let\makeglossary\@empty
      }%
    \else
      \@PackageInfoNoLine{rerunfilecheck}{%
        Option `glossary': unsupported version of \string\makeglossary
      }%
    \fi
  \fi
\fi
\ReFiCh@DisableOption{mainaux}
\ReFiCh@DisableOption{partaux}
\ReFiCh@DisableOption{starttoc}
\ReFiCh@DisableOption{index}
\ReFiCh@DisableOption{glossary}
\ReFiCh@DisableOption{aux}
%    \end{macrocode}
%
% \subsection{Rerun check}
%
%    \begin{macrocode}
\RequirePackage{atveryend}[2016/05/16]
\RequirePackage{uniquecounter}[2009/12/18]
%    \end{macrocode}
%
%    \begin{macro}{\ReFiCh@CheckSum}
%    \begin{macrocode}
\begingroup\expandafter\expandafter\expandafter\endgroup
\expandafter\ifx\csname pdf@filesize\endcsname\relax
  \def\ReFiCh@CheckSum{%
    \pdf@filemdfivesum
  }%
\else
  \def\ReFiCh@CheckSum#1{%
    \pdf@filemdfivesum{#1}%
    \ReFiCh@Separator
    \pdf@filesize{#1}%
  }%
\fi
%    \end{macrocode}
%    \end{macro}
%    \begin{macro}{\ReFiCh@NoFile}
%    \begin{macrocode}
\def\ReFiCh@Separator{;}
%    \end{macrocode}
%    \end{macro}
%    \begin{macro}{\ReFiCh@NoFile}
%    \begin{macrocode}
\def\ReFiCh@NoFile{<no file>}
%    \end{macrocode}
%    \end{macro}
%
%    \begin{macrocode}
\UniqueCounterNew{rerunfilecheck}
%    \end{macrocode}
%
%    \begin{macro}{\RerunFileCheck}
%    \begin{macrocode}
\newcommand*{\RerunFileCheck}{%
  \UniqueCounterCall{rerunfilecheck}\ReFiCh@RerunFileCheck
}
%    \end{macrocode}
%    \end{macro}
%    \begin{macro}{\ReFiCh@RerunFileCheck}
%    \begin{macrocode}
\def\ReFiCh@RerunFileCheck#1{%
  \expandafter\ReFiCh@@RerunFileCheck\csname ReFiCh@#1\endcsname
}
%    \end{macrocode}
%    \end{macro}
%    \begin{macro}{\ReFiCh@Check}
%    \begin{macrocode}
\def\ReFiCh@Check#1#2#3{%
%  \IfFileExists{#3}{%
    #1\edef#2{\ReFiCh@CheckSum{#3}}%
    \ifx#2\ReFiCh@Separator
      #1\let#2\ReFiCh@NoFile
    \fi
%  }{%
%    #1\let#2\ReFiCh@NoFile
%  }%
}
%    \end{macrocode}
%    \end{macro}
%    \begin{macro}{\ReFiCh@@RerunFileCheck}
%    \begin{macrocode}
\def\ReFiCh@@RerunFileCheck#1#2#3#4{%
  \ReFiCh@Check\global#1{#2}%
  \AtEndAfterFileList{%
    \begingroup
      #3%
      \ReFiCh@Check{}\x{#2}%
      \ifx#1\x
        \@PackageInfoNoLine{rerunfilecheck}{%
          File `#2' has not changed.\MessageBreak
          Checksum: \x
        }%
      \else
        \ifnum
          \ReFiCh@IsAux#2\relax.aux\relax\@nil
          \ifx#1\ReFiCh@NoFile 1\else 0\fi
          \ifx\x\ReFiCh@AuxEmptyUnix 1%
          \else
            \ifx\x\ReFiCh@AuxEmptyDos 1\fi
          \fi
        =111 %
          \@PackageInfoNoLine{rerunfilecheck}{%
            File `#2' is empty .aux file.\MessageBreak
            Before: #1\MessageBreak
            After: \space\x
          }%
        \else
          \@PackageWarningNoLine{rerunfilecheck}{%
            File `#2' has changed.%
            \ifx\\#4\\%
              \space Rerun%
            \else
              \MessageBreak
              #4%
            \fi
          }%
          \@PackageInfoNoLine{rerunfilecheck}{%
            Checksums for `#2':\MessageBreak
            Before: #1\MessageBreak
            After: \space\x
          }%
        \fi
      \fi
    \endgroup
  }%
}
%    \end{macrocode}
%    \end{macro}
%    \begin{macrocode}
\def\ReFiCh@IsAux#1.aux\relax#2\@nil{%
  \ifx\hbox#2\hbox
    0%
  \else
    1%
  \fi
}
%    \end{macrocode}
%    \begin{macrocode}
\def\ReFiCh@AuxEmptyUnix{A94A2480D3289E625EEA47CD1B285758;8}%
\@onelevel@sanitize\ReFiCh@AuxEmptyUnix
%    \end{macrocode}
%    \begin{macrocode}
\def\ReFiCh@AuxEmptyDos{A62A15ECE803E2EBE94952FCC9933BC0;9}%
\@onelevel@sanitize\ReFiCh@AuxEmptyDos
%    \end{macrocode}
%
%    \begin{macrocode}
\ReFiCh@AtEnd%
%</package>
%    \end{macrocode}
%
% \section{Test}
%
%    \begin{macrocode}
%<*test1>
\def\LoadCommand{\RequirePackage{rerunfilecheck}[2016/05/16]}
%</test1>
%    \end{macrocode}
% \subsection{Catcode checks for loading}
%
%    \begin{macrocode}
%<*test1>
%    \end{macrocode}
%    \begin{macrocode}
\catcode`\{=1 %
\catcode`\}=2 %
\catcode`\#=6 %
\catcode`\@=11 %
\expandafter\ifx\csname count@\endcsname\relax
  \countdef\count@=255 %
\fi
\expandafter\ifx\csname @gobble\endcsname\relax
  \long\def\@gobble#1{}%
\fi
\expandafter\ifx\csname @firstofone\endcsname\relax
  \long\def\@firstofone#1{#1}%
\fi
\expandafter\ifx\csname loop\endcsname\relax
  \expandafter\@firstofone
\else
  \expandafter\@gobble
\fi
{%
  \def\loop#1\repeat{%
    \def\body{#1}%
    \iterate
  }%
  \def\iterate{%
    \body
      \let\next\iterate
    \else
      \let\next\relax
    \fi
    \next
  }%
  \let\repeat=\fi
}%
\def\RestoreCatcodes{}
\count@=0 %
\loop
  \edef\RestoreCatcodes{%
    \RestoreCatcodes
    \catcode\the\count@=\the\catcode\count@\relax
  }%
\ifnum\count@<255 %
  \advance\count@ 1 %
\repeat

\def\RangeCatcodeInvalid#1#2{%
  \count@=#1\relax
  \loop
    \catcode\count@=15 %
  \ifnum\count@<#2\relax
    \advance\count@ 1 %
  \repeat
}
\def\RangeCatcodeCheck#1#2#3{%
  \count@=#1\relax
  \loop
    \ifnum#3=\catcode\count@
    \else
      \errmessage{%
        Character \the\count@\space
        with wrong catcode \the\catcode\count@\space
        instead of \number#3%
      }%
    \fi
  \ifnum\count@<#2\relax
    \advance\count@ 1 %
  \repeat
}
\def\space{ }
\expandafter\ifx\csname LoadCommand\endcsname\relax
  \def\LoadCommand{\input rerunfilecheck.sty\relax}%
\fi
\def\Test{%
  \RangeCatcodeInvalid{0}{47}%
  \RangeCatcodeInvalid{58}{64}%
  \RangeCatcodeInvalid{91}{96}%
  \RangeCatcodeInvalid{123}{255}%
  \catcode`\@=12 %
  \catcode`\\=0 %
  \catcode`\%=14 %
  \LoadCommand
  \RangeCatcodeCheck{0}{36}{15}%
  \RangeCatcodeCheck{37}{37}{14}%
  \RangeCatcodeCheck{38}{47}{15}%
  \RangeCatcodeCheck{48}{57}{12}%
  \RangeCatcodeCheck{58}{63}{15}%
  \RangeCatcodeCheck{64}{64}{12}%
  \RangeCatcodeCheck{65}{90}{11}%
  \RangeCatcodeCheck{91}{91}{15}%
  \RangeCatcodeCheck{92}{92}{0}%
  \RangeCatcodeCheck{93}{96}{15}%
  \RangeCatcodeCheck{97}{122}{11}%
  \RangeCatcodeCheck{123}{255}{15}%
  \RestoreCatcodes
}
\Test
\csname @@end\endcsname
\end
%    \end{macrocode}
%    \begin{macrocode}
%</test1>
%    \end{macrocode}
%
% \section{Installation}
%
% \subsection{Download}
%
% \paragraph{Package.} This package is available on
% CTAN\footnote{\url{http://ctan.org/pkg/rerunfilecheck}}:
% \begin{description}
% \item[\CTAN{macros/latex/contrib/oberdiek/rerunfilecheck.dtx}] The source file.
% \item[\CTAN{macros/latex/contrib/oberdiek/rerunfilecheck.pdf}] Documentation.
% \end{description}
%
%
% \paragraph{Bundle.} All the packages of the bundle `oberdiek'
% are also available in a TDS compliant ZIP archive. There
% the packages are already unpacked and the documentation files
% are generated. The files and directories obey the TDS standard.
% \begin{description}
% \item[\CTAN{install/macros/latex/contrib/oberdiek.tds.zip}]
% \end{description}
% \emph{TDS} refers to the standard ``A Directory Structure
% for \TeX\ Files'' (\CTAN{tds/tds.pdf}). Directories
% with \xfile{texmf} in their name are usually organized this way.
%
% \subsection{Bundle installation}
%
% \paragraph{Unpacking.} Unpack the \xfile{oberdiek.tds.zip} in the
% TDS tree (also known as \xfile{texmf} tree) of your choice.
% Example (linux):
% \begin{quote}
%   |unzip oberdiek.tds.zip -d ~/texmf|
% \end{quote}
%
% \paragraph{Script installation.}
% Check the directory \xfile{TDS:scripts/oberdiek/} for
% scripts that need further installation steps.
% Package \xpackage{attachfile2} comes with the Perl script
% \xfile{pdfatfi.pl} that should be installed in such a way
% that it can be called as \texttt{pdfatfi}.
% Example (linux):
% \begin{quote}
%   |chmod +x scripts/oberdiek/pdfatfi.pl|\\
%   |cp scripts/oberdiek/pdfatfi.pl /usr/local/bin/|
% \end{quote}
%
% \subsection{Package installation}
%
% \paragraph{Unpacking.} The \xfile{.dtx} file is a self-extracting
% \docstrip\ archive. The files are extracted by running the
% \xfile{.dtx} through \plainTeX:
% \begin{quote}
%   \verb|tex rerunfilecheck.dtx|
% \end{quote}
%
% \paragraph{TDS.} Now the different files must be moved into
% the different directories in your installation TDS tree
% (also known as \xfile{texmf} tree):
% \begin{quote}
% \def\t{^^A
% \begin{tabular}{@{}>{\ttfamily}l@{ $\rightarrow$ }>{\ttfamily}l@{}}
%   rerunfilecheck.sty & tex/latex/oberdiek/rerunfilecheck.sty\\
%   rerunfilecheck.pdf & doc/latex/oberdiek/rerunfilecheck.pdf\\
%   rerunfilecheck-example.cfg & doc/latex/oberdiek/rerunfilecheck-example.cfg\\
%   test/rerunfilecheck-test1.tex & doc/latex/oberdiek/test/rerunfilecheck-test1.tex\\
%   rerunfilecheck.dtx & source/latex/oberdiek/rerunfilecheck.dtx\\
% \end{tabular}^^A
% }^^A
% \sbox0{\t}^^A
% \ifdim\wd0>\linewidth
%   \begingroup
%     \advance\linewidth by\leftmargin
%     \advance\linewidth by\rightmargin
%   \edef\x{\endgroup
%     \def\noexpand\lw{\the\linewidth}^^A
%   }\x
%   \def\lwbox{^^A
%     \leavevmode
%     \hbox to \linewidth{^^A
%       \kern-\leftmargin\relax
%       \hss
%       \usebox0
%       \hss
%       \kern-\rightmargin\relax
%     }^^A
%   }^^A
%   \ifdim\wd0>\lw
%     \sbox0{\small\t}^^A
%     \ifdim\wd0>\linewidth
%       \ifdim\wd0>\lw
%         \sbox0{\footnotesize\t}^^A
%         \ifdim\wd0>\linewidth
%           \ifdim\wd0>\lw
%             \sbox0{\scriptsize\t}^^A
%             \ifdim\wd0>\linewidth
%               \ifdim\wd0>\lw
%                 \sbox0{\tiny\t}^^A
%                 \ifdim\wd0>\linewidth
%                   \lwbox
%                 \else
%                   \usebox0
%                 \fi
%               \else
%                 \lwbox
%               \fi
%             \else
%               \usebox0
%             \fi
%           \else
%             \lwbox
%           \fi
%         \else
%           \usebox0
%         \fi
%       \else
%         \lwbox
%       \fi
%     \else
%       \usebox0
%     \fi
%   \else
%     \lwbox
%   \fi
% \else
%   \usebox0
% \fi
% \end{quote}
% If you have a \xfile{docstrip.cfg} that configures and enables \docstrip's
% TDS installing feature, then some files can already be in the right
% place, see the documentation of \docstrip.
%
% \subsection{Refresh file name databases}
%
% If your \TeX~distribution
% (\teTeX, \mikTeX, \dots) relies on file name databases, you must refresh
% these. For example, \teTeX\ users run \verb|texhash| or
% \verb|mktexlsr|.
%
% \subsection{Some details for the interested}
%
% \paragraph{Attached source.}
%
% The PDF documentation on CTAN also includes the
% \xfile{.dtx} source file. It can be extracted by
% AcrobatReader 6 or higher. Another option is \textsf{pdftk},
% e.g. unpack the file into the current directory:
% \begin{quote}
%   \verb|pdftk rerunfilecheck.pdf unpack_files output .|
% \end{quote}
%
% \paragraph{Unpacking with \LaTeX.}
% The \xfile{.dtx} chooses its action depending on the format:
% \begin{description}
% \item[\plainTeX:] Run \docstrip\ and extract the files.
% \item[\LaTeX:] Generate the documentation.
% \end{description}
% If you insist on using \LaTeX\ for \docstrip\ (really,
% \docstrip\ does not need \LaTeX), then inform the autodetect routine
% about your intention:
% \begin{quote}
%   \verb|latex \let\install=y% \iffalse meta-comment
%
% File: rerunfilecheck.dtx
% Version: 2016/05/16 v1.8
% Info: Rerun checks for auxiliary files
%
% Copyright (C) 2009-2011 by
%    Heiko Oberdiek <heiko.oberdiek at googlemail.com>
%    2016
%    https://github.com/ho-tex/oberdiek/issues
%
% This work may be distributed and/or modified under the
% conditions of the LaTeX Project Public License, either
% version 1.3c of this license or (at your option) any later
% version. This version of this license is in
%    http://www.latex-project.org/lppl/lppl-1-3c.txt
% and the latest version of this license is in
%    http://www.latex-project.org/lppl.txt
% and version 1.3 or later is part of all distributions of
% LaTeX version 2005/12/01 or later.
%
% This work has the LPPL maintenance status "maintained".
%
% This Current Maintainer of this work is Heiko Oberdiek.
%
% This work consists of the main source file rerunfilecheck.dtx
% and the derived files
%    rerunfilecheck.sty, rerunfilecheck.pdf, rerunfilecheck.ins,
%    rerunfilecheck.drv, rerunfilecheck-example.cfg,
%    rerunfilecheck-test1.tex.
%
% Distribution:
%    CTAN:macros/latex/contrib/oberdiek/rerunfilecheck.dtx
%    CTAN:macros/latex/contrib/oberdiek/rerunfilecheck.pdf
%
% Unpacking:
%    (a) If rerunfilecheck.ins is present:
%           tex rerunfilecheck.ins
%    (b) Without rerunfilecheck.ins:
%           tex rerunfilecheck.dtx
%    (c) If you insist on using LaTeX
%           latex \let\install=y\input{rerunfilecheck.dtx}
%        (quote the arguments according to the demands of your shell)
%
% Documentation:
%    (a) If rerunfilecheck.drv is present:
%           latex rerunfilecheck.drv
%    (b) Without rerunfilecheck.drv:
%           latex rerunfilecheck.dtx; ...
%    The class ltxdoc loads the configuration file ltxdoc.cfg
%    if available. Here you can specify further options, e.g.
%    use A4 as paper format:
%       \PassOptionsToClass{a4paper}{article}
%
%    Programm calls to get the documentation (example):
%       pdflatex rerunfilecheck.dtx
%       makeindex -s gind.ist rerunfilecheck.idx
%       pdflatex rerunfilecheck.dtx
%       makeindex -s gind.ist rerunfilecheck.idx
%       pdflatex rerunfilecheck.dtx
%
% Installation:
%    TDS:tex/latex/oberdiek/rerunfilecheck.sty
%    TDS:doc/latex/oberdiek/rerunfilecheck.pdf
%    TDS:doc/latex/oberdiek/rerunfilecheck-example.cfg
%    TDS:doc/latex/oberdiek/test/rerunfilecheck-test1.tex
%    TDS:source/latex/oberdiek/rerunfilecheck.dtx
%
%<*ignore>
\begingroup
  \catcode123=1 %
  \catcode125=2 %
  \def\x{LaTeX2e}%
\expandafter\endgroup
\ifcase 0\ifx\install y1\fi\expandafter
         \ifx\csname processbatchFile\endcsname\relax\else1\fi
         \ifx\fmtname\x\else 1\fi\relax
\else\csname fi\endcsname
%</ignore>
%<*install>
\input docstrip.tex
\Msg{************************************************************************}
\Msg{* Installation}
\Msg{* Package: rerunfilecheck 2016/05/16 v1.8 Rerun checks for auxiliary files (HO)}
\Msg{************************************************************************}

\keepsilent
\askforoverwritefalse

\let\MetaPrefix\relax
\preamble

This is a generated file.

Project: rerunfilecheck
Version: 2016/05/16 v1.8

Copyright (C) 2009-2011 by
   Heiko Oberdiek <heiko.oberdiek at googlemail.com>

This work may be distributed and/or modified under the
conditions of the LaTeX Project Public License, either
version 1.3c of this license or (at your option) any later
version. This version of this license is in
   http://www.latex-project.org/lppl/lppl-1-3c.txt
and the latest version of this license is in
   http://www.latex-project.org/lppl.txt
and version 1.3 or later is part of all distributions of
LaTeX version 2005/12/01 or later.

This work has the LPPL maintenance status "maintained".

This Current Maintainer of this work is Heiko Oberdiek.

This work consists of the main source file rerunfilecheck.dtx
and the derived files
   rerunfilecheck.sty, rerunfilecheck.pdf, rerunfilecheck.ins,
   rerunfilecheck.drv, rerunfilecheck-example.cfg,
   rerunfilecheck-test1.tex.

\endpreamble
\let\MetaPrefix\DoubleperCent

\generate{%
  \file{rerunfilecheck.ins}{\from{rerunfilecheck.dtx}{install}}%
  \file{rerunfilecheck.drv}{\from{rerunfilecheck.dtx}{driver}}%
  \usedir{tex/latex/oberdiek}%
  \file{rerunfilecheck.sty}{\from{rerunfilecheck.dtx}{package}}%
%  \usedir{doc/latex/oberdiek/test}%
%  \file{rerunfilecheck-test1.tex}{\from{rerunfilecheck.dtx}{test1}}%
  \nopreamble
  \nopostamble
  \usedir{doc/latex/oberdiek}%
  \file{rerunfilecheck-example.cfg}{\from{rerunfilecheck.dtx}{cfg}}%
  \usedir{source/latex/oberdiek/catalogue}%
  \file{rerunfilecheck.xml}{\from{rerunfilecheck.dtx}{catalogue}}%
}

\catcode32=13\relax% active space
\let =\space%
\Msg{************************************************************************}
\Msg{*}
\Msg{* To finish the installation you have to move the following}
\Msg{* file into a directory searched by TeX:}
\Msg{*}
\Msg{*     rerunfilecheck.sty}
\Msg{*}
\Msg{* To produce the documentation run the file `rerunfilecheck.drv'}
\Msg{* through LaTeX.}
\Msg{*}
\Msg{* Happy TeXing!}
\Msg{*}
\Msg{************************************************************************}

\endbatchfile
%</install>
%<*ignore>
\fi
%</ignore>
%<*driver>
\NeedsTeXFormat{LaTeX2e}
\ProvidesFile{rerunfilecheck.drv}%
  [2016/05/16 v1.8 Rerun checks for auxiliary files (HO)]%
\documentclass{ltxdoc}
\usepackage{holtxdoc}[2011/11/22]
\begin{document}
  \DocInput{rerunfilecheck.dtx}%
\end{document}
%</driver>
% \fi
%
%
% \CharacterTable
%  {Upper-case    \A\B\C\D\E\F\G\H\I\J\K\L\M\N\O\P\Q\R\S\T\U\V\W\X\Y\Z
%   Lower-case    \a\b\c\d\e\f\g\h\i\j\k\l\m\n\o\p\q\r\s\t\u\v\w\x\y\z
%   Digits        \0\1\2\3\4\5\6\7\8\9
%   Exclamation   \!     Double quote  \"     Hash (number) \#
%   Dollar        \$     Percent       \%     Ampersand     \&
%   Acute accent  \'     Left paren    \(     Right paren   \)
%   Asterisk      \*     Plus          \+     Comma         \,
%   Minus         \-     Point         \.     Solidus       \/
%   Colon         \:     Semicolon     \;     Less than     \<
%   Equals        \=     Greater than  \>     Question mark \?
%   Commercial at \@     Left bracket  \[     Backslash     \\
%   Right bracket \]     Circumflex    \^     Underscore    \_
%   Grave accent  \`     Left brace    \{     Vertical bar  \|
%   Right brace   \}     Tilde         \~}
%
% \GetFileInfo{rerunfilecheck.drv}
%
% \title{The \xpackage{rerunfilecheck} package}
% \date{2016/05/16 v1.8}
% \author{Heiko Oberdiek\thanks
% {Please report any issues at https://github.com/ho-tex/oberdiek/issues}\\
% \xemail{heiko.oberdiek at googlemail.com}}
%
% \maketitle
%
% \begin{abstract}
% The package provides additional rerun warnings if some
% auxiliary files have changed. It is based on MD5 checksum,
% provided by \pdfTeX.
% \end{abstract}
%
% \tableofcontents
%
% \section{Documentation}
%
% \LaTeX\ informs the user, when to run \LaTeX\ again, if the
% references have changed. It has the old references from the
% first reading of the \xext{aux} files already in memory, thus
% it compares them with the new version of the \xext{aux} file
% at the end of the document. However this rerun warnings are
% not given for the table of contents and other data stored
% in the \xext{aux} files or other auxiliary files.
% Usually many of these data as the table of contents is not keep
% in memory. If someone wants to detect changes, he has either
% to keep the data in memory. This does not scale well with huge
% documents. Or he copies the file before they are changed. Slow
% I/O operations cost time.
%
% Since version 1.30.0 \pdfTeX\ provides \cs{pdfmdfivesum}
% and \cs{pdffilesize}.
% These features are also available in \LuaTeX, provided by package
% \xpackage{pdftexcmds}.
% Thus this package \xpackage{rerunfilecheck} uses these features to
% detect file changes. This saves the packages from keeping
% the whole files in memory or in file copies.
% The drawback are different files with the same size and the
% same MD5 checksum (seldom, hopefully).
%
% \subsection{Options}
%
% All options are key value options of boolean type.
% No option or |true| turns an option on, |false| disables
% an option.
% \begin{description}
% \item[\xoption{mainaux}:]
%   Check the main \xext{aux} file.
% \item[\xoption{partaux}:]
%   Check the \xext{aux} files from \cs{include} files.
% \item[\xoption{starttoc}:]
%   Add the rerun checks in \cs{@starttoc} that is called by
%   \cs{tableofcontents}, \cs{listoffigures}, \dots
% \item[\xoption{index}, \xoption{glossary}:]
%   \LaTeX's original \cs{makeindex} and \cs{makeglossary} are
%   redefined to add the rerun checks. The options do not have
%   an effect, if \cs{makeindex}/\cs{makeglossary} are already
%   called or if a package or class had redefined or will
%   redefine them.
% \item[\xoption{aux}:]
%   This option turns all previous options on or off.
%   ``aux'' means auxiliary file.
% \end{description}
% The default for the options is |false|, because
% some internals must be redefined to insert the
% rerun checks.
% The options can be set in \cs{usepackage} or
% the configuration file \xfile{rerunfilecheck.cfg}.
% Global options are ignored (since 1.4).
%
% \begin{declcs}{RerunFileCheckSetup} \M{key value list}
% \end{declcs}
% Options can also be set using \cs{RerunFileCheckSetup}.
% Currently all options are disabled after the package
% is loaded. Thus \cs{RerunFileCheckSetup} makes sense in
% the configuration file only.
%
% Example for the configuration file:
%    \begin{macrocode}
%<*cfg>
\ProvidesFile{rerunfilecheck.cfg}[2016/05/16 Default configuration]%
\RerunFileCheckSetup{aux}
%</cfg>
%    \end{macrocode}
%
% \subsection{Interface for class/package authors}
%
% \begin{declcs}{RerunFileCheck} \M{file} \M{file closing action} \M{rerun warning}
% \end{declcs}
% If you want to add a rerun check, call \cs{RerunFileCheck} right before
% an output file is opened for writing. The macro first remembers
% the current checksum of \meta{file}.
% The file is checked again right before the end of the job.
% Macro \cs{AtVeryEndDocument} of package \xpackage{atveryend}
% is used to place the check after the main aux file is closed
% in |\end{document}|. Before reading the file again, it must be
% closed. Provide the code for closing in argument \meta{file closing action}.
% Do not forget \cs{immediate} before \cs{openout}. Otherwise the
% closing action would be delayed to the next shipout that never happens
% (the last page is already shipped out).
% If the file has changed, \cs{RerunFileCheck} informs the user with
% a warning that the file has changed and says the magic word ``Rerun''.
% If the last argument \meta{rerun warning} is not empty, then
% the rerun sentence is replaced by it. Usually the phrase
% ``to get something right'' is added.
% As example the relevant part of the redefined \cs{makeindex}
% is shown, see package code:
%\begin{quote}
%\begin{verbatim}
%\newwrite\@indexfile
%\RerunFileCheck{\jobname.idx}{%
%  \immediate\closeout\@indexfile
%}{%
%  Rerun LaTeX/makeindex to get index right%
%}%
%\immediate\openout\@indexfile=\jobname.idx %
%\end{verbatim}
%\end{quote}
%
% \StopEventually{
% }
%
% \section{Implementation}
%    \begin{macrocode}
%<*package>
%    \end{macrocode}
%
%    \begin{macrocode}
\begingroup\catcode61\catcode48\catcode32=10\relax%
  \catcode13=5 % ^^M
  \endlinechar=13 %
  \catcode123=1 % {
  \catcode125=2 % }
  \catcode64=11 % @
  \def\x{\endgroup
    \expandafter\edef\csname ReFiCh@AtEnd\endcsname{%
      \endlinechar=\the\endlinechar\relax
      \catcode13=\the\catcode13\relax
      \catcode32=\the\catcode32\relax
      \catcode35=\the\catcode35\relax
      \catcode61=\the\catcode61\relax
      \catcode64=\the\catcode64\relax
      \catcode123=\the\catcode123\relax
      \catcode125=\the\catcode125\relax
    }%
  }%
\x\catcode61\catcode48\catcode32=10\relax%
\catcode13=5 % ^^M
\endlinechar=13 %
\catcode35=6 % #
\catcode64=11 % @
\catcode123=1 % {
\catcode125=2 % }
\def\TMP@EnsureCode#1#2{%
  \edef\ReFiCh@AtEnd{%
    \ReFiCh@AtEnd
    \catcode#1=\the\catcode#1\relax
  }%
  \catcode#1=#2\relax
}
\TMP@EnsureCode{39}{12}% '
\TMP@EnsureCode{40}{12}% (
\TMP@EnsureCode{41}{12}% )
\TMP@EnsureCode{42}{12}% *
\TMP@EnsureCode{44}{12}% ,
\TMP@EnsureCode{46}{12}% .
\TMP@EnsureCode{47}{12}% /
\TMP@EnsureCode{58}{12}% :
\TMP@EnsureCode{59}{12}% ;
\TMP@EnsureCode{60}{12}% <
\TMP@EnsureCode{62}{12}% >
\TMP@EnsureCode{91}{12}% [
\TMP@EnsureCode{93}{12}% ]
\TMP@EnsureCode{96}{12}% `
\edef\ReFiCh@AtEnd{\ReFiCh@AtEnd\noexpand\endinput}
%    \end{macrocode}
%
%    Package identification.
%    \begin{macrocode}
\NeedsTeXFormat{LaTeX2e}
\ProvidesPackage{rerunfilecheck}%
  [2016/05/16 v1.8 Rerun checks for auxiliary files (HO)]
%    \end{macrocode}
%
% \subsection{Options}
%
%    \begin{macrocode}
\RequirePackage{kvoptions}[2010/02/22]
\SetupKeyvalOptions{%
  family=rerunfilecheck,%
  prefix=ReFiCh@%
}
%    \end{macrocode}
%    \begin{macro}{\RerunFileCheckSetup}
%    \begin{macrocode}
\newcommand*{\RerunFileCheckSetup}{%
  \setkeys{rerunfilecheck}%
}
%    \end{macrocode}
%    \end{macro}
%    \begin{macrocode}
\DeclareBoolOption{mainaux}
\DeclareBoolOption{partaux}
\DeclareBoolOption{starttoc}
\DeclareBoolOption{index}
\DeclareBoolOption{glossary}
\define@key{rerunfilecheck}{aux}[true]{%
  \RerunFileCheckSetup{%
    mainaux={#1},%
    partaux={#1},%
    starttoc={#1},%
    index={#1},%
    glossary={#1}%
  }%
}
%    \end{macrocode}
%    \begin{macrocode}
\InputIfFileExists{rerunfilecheck.cfg}{}{}
\ProcessLocalKeyvalOptions*
%    \end{macrocode}
%
%    \begin{macro}{\ReFiCh@DisableOption}
%    \begin{macrocode}
\def\ReFiCh@DisableOption{%
  \DisableKeyvalOption[%
    action=warning,%
    package=rerunfilecheck%
  ]{rerunfilecheck}%
}
%    \end{macrocode}
%    \end{macro}
%
% \subsection{Check for checksum feature}
%
%    \begin{macrocode}
\RequirePackage{infwarerr}[2007/09/09]
\RequirePackage{pdftexcmds}[2009/04/10]
%    \end{macrocode}
%    \begin{macrocode}
\begingroup\expandafter\expandafter\expandafter\endgroup
\expandafter\ifx\csname pdf@filemdfivesum\endcsname\relax
  \@PackageInfoNoLine{rerunfilecheck}{%
    Feature \string\pdfmdfivesum\space is not available\MessageBreak
    (e.g. pdfTeX or LuaTeX with package `pdftexcmds').\MessageBreak
    Therefore file contents cannot be checked efficiently\MessageBreak
    and the loading of the package is aborted%
  }%
  \newcommand*{\RerunFileCheck}[3]{}%
  \renewcommand*{\RerunFileCheckSetup}[1]{}%
  \expandafter\ReFiCh@AtEnd
\fi%
%    \end{macrocode}
%
% \subsection{Standard \xext{aux} files}
%
%    \begin{macrocode}
\ifReFiCh@partaux
  \let\ReFiCh@org@include\@include
  \def\@include#1 {%
    \if@filesw
      \RerunFileCheck{#1.aux}{}{}%
    \fi
    \ReFiCh@org@include{#1} %
  }%
\fi
\ifReFiCh@mainaux
  \AtBeginDocument{%
    \ReFiCh@mainauxfalse
  }%
  \ifReFiCh@mainaux
    \AtEndOfPackage{%
      \RerunFileCheck{\jobname.aux}{}{}%
    }%
  \else
    \if@filesw
      \@PackageWarningNoLine{rerunfilecheck}{%
        Main aux file check is disabled,\MessageBreak
        because the file is already opened.\MessageBreak
        Load the package before \string\begin{document}%
      }%
    \fi
  \fi
\fi
\ifReFiCh@starttoc
  \let\ReFiCh@org@starttoc\@starttoc
  \def\@starttoc#1{%
    \if@filesw
      \RerunFileCheck{\jobname.#1}{%
        \@ifundefined{tf@#1}{%
        }{%
          \immediate\closeout\csname tf@#1\endcsname
        }%
      }{}%
    \fi
    \ReFiCh@org@starttoc{#1}%
  }%
\fi
\ifReFiCh@index
  \ifx\makeindex\@empty
    \@PackageWarningNoLine{rerunfilecheck}{%
      Option `index' ignored,\MessageBreak
      because \string\makeindex\space has already been called%
    }%
  \else
    \def\ReFiCh@temp{%
      \newwrite\@indexfile
      \immediate\openout\@indexfile=\jobname.idx %
      \def\index{%
        \@bsphack
        \begingroup
        \@sanitize
        \@wrindex
      }%
      \typeout{Writing index file \jobname.idx}%
      \let\makeindex\@empty
    }%
    \ifx\ReFiCh@temp\makeindex
      \def\makeindex{%
        \newwrite\@indexfile
        \RerunFileCheck{\jobname.idx}{%
          \immediate\closeout\@indexfile
        }{%
          Rerun LaTeX/makeindex to get index right%
        }%
        \immediate\openout\@indexfile=\jobname.idx %
        \def\index{%
          \@bsphack
          \begingroup
          \@sanitize
          \@wrindex
        }%
        \typeout{Writing index file \jobname.idx}%
        \let\makeindex\@empty
      }%
    \else
      \@PackageInfoNoLine{rerunfilecheck}{%
        Option `index': unsupported version of \string\makeindex
      }%
    \fi
  \fi
\fi
\ifReFiCh@glossary
  \ifx\makeglossary\@empty
    \@PackageWarningNoLine{rerunfilecheck}{%
      Option `glossary' ignored,\MessageBreak
      because \string\makeglossary\space has already been called%
    }%
  \else
    \def\ReFiCh@temp{%
      \newwrite\@glossaryfile
      \immediate\openout\@glossaryfile=\jobname.glo %
      \def\glossary{%
        \@bsphack
        \begingroup
        \@sanitize
        \@wrglossary
      }%
      \typeout{Writing glossary file \jobname.glo }%
      \let\makeglossary\@empty
    }%
    \ifx\ReFiCh@temp\makeglossary
      \def\ReFiCh@temp{%
        \newwrite\@glossaryfile
        \RerunFileCheck{\jobname.glo}{%
          \immediate\closeout\@glossaryfile
        }{%
          Rerun LaTeX/makeindex to get glossary right%
        }%
        \immediate\openout\@glossaryfile=\jobname.glo %
        \def\glossary{%
          \@bsphack
          \begingroup
          \@sanitize
          \@wrglossary
        }%
        \typeout{Writing glossary file \jobname.glo}%
        \let\makeglossary\@empty
      }%
    \else
      \@PackageInfoNoLine{rerunfilecheck}{%
        Option `glossary': unsupported version of \string\makeglossary
      }%
    \fi
  \fi
\fi
\ReFiCh@DisableOption{mainaux}
\ReFiCh@DisableOption{partaux}
\ReFiCh@DisableOption{starttoc}
\ReFiCh@DisableOption{index}
\ReFiCh@DisableOption{glossary}
\ReFiCh@DisableOption{aux}
%    \end{macrocode}
%
% \subsection{Rerun check}
%
%    \begin{macrocode}
\RequirePackage{atveryend}[2016/05/16]
\RequirePackage{uniquecounter}[2009/12/18]
%    \end{macrocode}
%
%    \begin{macro}{\ReFiCh@CheckSum}
%    \begin{macrocode}
\begingroup\expandafter\expandafter\expandafter\endgroup
\expandafter\ifx\csname pdf@filesize\endcsname\relax
  \def\ReFiCh@CheckSum{%
    \pdf@filemdfivesum
  }%
\else
  \def\ReFiCh@CheckSum#1{%
    \pdf@filemdfivesum{#1}%
    \ReFiCh@Separator
    \pdf@filesize{#1}%
  }%
\fi
%    \end{macrocode}
%    \end{macro}
%    \begin{macro}{\ReFiCh@NoFile}
%    \begin{macrocode}
\def\ReFiCh@Separator{;}
%    \end{macrocode}
%    \end{macro}
%    \begin{macro}{\ReFiCh@NoFile}
%    \begin{macrocode}
\def\ReFiCh@NoFile{<no file>}
%    \end{macrocode}
%    \end{macro}
%
%    \begin{macrocode}
\UniqueCounterNew{rerunfilecheck}
%    \end{macrocode}
%
%    \begin{macro}{\RerunFileCheck}
%    \begin{macrocode}
\newcommand*{\RerunFileCheck}{%
  \UniqueCounterCall{rerunfilecheck}\ReFiCh@RerunFileCheck
}
%    \end{macrocode}
%    \end{macro}
%    \begin{macro}{\ReFiCh@RerunFileCheck}
%    \begin{macrocode}
\def\ReFiCh@RerunFileCheck#1{%
  \expandafter\ReFiCh@@RerunFileCheck\csname ReFiCh@#1\endcsname
}
%    \end{macrocode}
%    \end{macro}
%    \begin{macro}{\ReFiCh@Check}
%    \begin{macrocode}
\def\ReFiCh@Check#1#2#3{%
%  \IfFileExists{#3}{%
    #1\edef#2{\ReFiCh@CheckSum{#3}}%
    \ifx#2\ReFiCh@Separator
      #1\let#2\ReFiCh@NoFile
    \fi
%  }{%
%    #1\let#2\ReFiCh@NoFile
%  }%
}
%    \end{macrocode}
%    \end{macro}
%    \begin{macro}{\ReFiCh@@RerunFileCheck}
%    \begin{macrocode}
\def\ReFiCh@@RerunFileCheck#1#2#3#4{%
  \ReFiCh@Check\global#1{#2}%
  \AtEndAfterFileList{%
    \begingroup
      #3%
      \ReFiCh@Check{}\x{#2}%
      \ifx#1\x
        \@PackageInfoNoLine{rerunfilecheck}{%
          File `#2' has not changed.\MessageBreak
          Checksum: \x
        }%
      \else
        \ifnum
          \ReFiCh@IsAux#2\relax.aux\relax\@nil
          \ifx#1\ReFiCh@NoFile 1\else 0\fi
          \ifx\x\ReFiCh@AuxEmptyUnix 1%
          \else
            \ifx\x\ReFiCh@AuxEmptyDos 1\fi
          \fi
        =111 %
          \@PackageInfoNoLine{rerunfilecheck}{%
            File `#2' is empty .aux file.\MessageBreak
            Before: #1\MessageBreak
            After: \space\x
          }%
        \else
          \@PackageWarningNoLine{rerunfilecheck}{%
            File `#2' has changed.%
            \ifx\\#4\\%
              \space Rerun%
            \else
              \MessageBreak
              #4%
            \fi
          }%
          \@PackageInfoNoLine{rerunfilecheck}{%
            Checksums for `#2':\MessageBreak
            Before: #1\MessageBreak
            After: \space\x
          }%
        \fi
      \fi
    \endgroup
  }%
}
%    \end{macrocode}
%    \end{macro}
%    \begin{macrocode}
\def\ReFiCh@IsAux#1.aux\relax#2\@nil{%
  \ifx\hbox#2\hbox
    0%
  \else
    1%
  \fi
}
%    \end{macrocode}
%    \begin{macrocode}
\def\ReFiCh@AuxEmptyUnix{A94A2480D3289E625EEA47CD1B285758;8}%
\@onelevel@sanitize\ReFiCh@AuxEmptyUnix
%    \end{macrocode}
%    \begin{macrocode}
\def\ReFiCh@AuxEmptyDos{A62A15ECE803E2EBE94952FCC9933BC0;9}%
\@onelevel@sanitize\ReFiCh@AuxEmptyDos
%    \end{macrocode}
%
%    \begin{macrocode}
\ReFiCh@AtEnd%
%</package>
%    \end{macrocode}
%
% \section{Test}
%
%    \begin{macrocode}
%<*test1>
\def\LoadCommand{\RequirePackage{rerunfilecheck}[2016/05/16]}
%</test1>
%    \end{macrocode}
% \subsection{Catcode checks for loading}
%
%    \begin{macrocode}
%<*test1>
%    \end{macrocode}
%    \begin{macrocode}
\catcode`\{=1 %
\catcode`\}=2 %
\catcode`\#=6 %
\catcode`\@=11 %
\expandafter\ifx\csname count@\endcsname\relax
  \countdef\count@=255 %
\fi
\expandafter\ifx\csname @gobble\endcsname\relax
  \long\def\@gobble#1{}%
\fi
\expandafter\ifx\csname @firstofone\endcsname\relax
  \long\def\@firstofone#1{#1}%
\fi
\expandafter\ifx\csname loop\endcsname\relax
  \expandafter\@firstofone
\else
  \expandafter\@gobble
\fi
{%
  \def\loop#1\repeat{%
    \def\body{#1}%
    \iterate
  }%
  \def\iterate{%
    \body
      \let\next\iterate
    \else
      \let\next\relax
    \fi
    \next
  }%
  \let\repeat=\fi
}%
\def\RestoreCatcodes{}
\count@=0 %
\loop
  \edef\RestoreCatcodes{%
    \RestoreCatcodes
    \catcode\the\count@=\the\catcode\count@\relax
  }%
\ifnum\count@<255 %
  \advance\count@ 1 %
\repeat

\def\RangeCatcodeInvalid#1#2{%
  \count@=#1\relax
  \loop
    \catcode\count@=15 %
  \ifnum\count@<#2\relax
    \advance\count@ 1 %
  \repeat
}
\def\RangeCatcodeCheck#1#2#3{%
  \count@=#1\relax
  \loop
    \ifnum#3=\catcode\count@
    \else
      \errmessage{%
        Character \the\count@\space
        with wrong catcode \the\catcode\count@\space
        instead of \number#3%
      }%
    \fi
  \ifnum\count@<#2\relax
    \advance\count@ 1 %
  \repeat
}
\def\space{ }
\expandafter\ifx\csname LoadCommand\endcsname\relax
  \def\LoadCommand{\input rerunfilecheck.sty\relax}%
\fi
\def\Test{%
  \RangeCatcodeInvalid{0}{47}%
  \RangeCatcodeInvalid{58}{64}%
  \RangeCatcodeInvalid{91}{96}%
  \RangeCatcodeInvalid{123}{255}%
  \catcode`\@=12 %
  \catcode`\\=0 %
  \catcode`\%=14 %
  \LoadCommand
  \RangeCatcodeCheck{0}{36}{15}%
  \RangeCatcodeCheck{37}{37}{14}%
  \RangeCatcodeCheck{38}{47}{15}%
  \RangeCatcodeCheck{48}{57}{12}%
  \RangeCatcodeCheck{58}{63}{15}%
  \RangeCatcodeCheck{64}{64}{12}%
  \RangeCatcodeCheck{65}{90}{11}%
  \RangeCatcodeCheck{91}{91}{15}%
  \RangeCatcodeCheck{92}{92}{0}%
  \RangeCatcodeCheck{93}{96}{15}%
  \RangeCatcodeCheck{97}{122}{11}%
  \RangeCatcodeCheck{123}{255}{15}%
  \RestoreCatcodes
}
\Test
\csname @@end\endcsname
\end
%    \end{macrocode}
%    \begin{macrocode}
%</test1>
%    \end{macrocode}
%
% \section{Installation}
%
% \subsection{Download}
%
% \paragraph{Package.} This package is available on
% CTAN\footnote{\url{http://ctan.org/pkg/rerunfilecheck}}:
% \begin{description}
% \item[\CTAN{macros/latex/contrib/oberdiek/rerunfilecheck.dtx}] The source file.
% \item[\CTAN{macros/latex/contrib/oberdiek/rerunfilecheck.pdf}] Documentation.
% \end{description}
%
%
% \paragraph{Bundle.} All the packages of the bundle `oberdiek'
% are also available in a TDS compliant ZIP archive. There
% the packages are already unpacked and the documentation files
% are generated. The files and directories obey the TDS standard.
% \begin{description}
% \item[\CTAN{install/macros/latex/contrib/oberdiek.tds.zip}]
% \end{description}
% \emph{TDS} refers to the standard ``A Directory Structure
% for \TeX\ Files'' (\CTAN{tds/tds.pdf}). Directories
% with \xfile{texmf} in their name are usually organized this way.
%
% \subsection{Bundle installation}
%
% \paragraph{Unpacking.} Unpack the \xfile{oberdiek.tds.zip} in the
% TDS tree (also known as \xfile{texmf} tree) of your choice.
% Example (linux):
% \begin{quote}
%   |unzip oberdiek.tds.zip -d ~/texmf|
% \end{quote}
%
% \paragraph{Script installation.}
% Check the directory \xfile{TDS:scripts/oberdiek/} for
% scripts that need further installation steps.
% Package \xpackage{attachfile2} comes with the Perl script
% \xfile{pdfatfi.pl} that should be installed in such a way
% that it can be called as \texttt{pdfatfi}.
% Example (linux):
% \begin{quote}
%   |chmod +x scripts/oberdiek/pdfatfi.pl|\\
%   |cp scripts/oberdiek/pdfatfi.pl /usr/local/bin/|
% \end{quote}
%
% \subsection{Package installation}
%
% \paragraph{Unpacking.} The \xfile{.dtx} file is a self-extracting
% \docstrip\ archive. The files are extracted by running the
% \xfile{.dtx} through \plainTeX:
% \begin{quote}
%   \verb|tex rerunfilecheck.dtx|
% \end{quote}
%
% \paragraph{TDS.} Now the different files must be moved into
% the different directories in your installation TDS tree
% (also known as \xfile{texmf} tree):
% \begin{quote}
% \def\t{^^A
% \begin{tabular}{@{}>{\ttfamily}l@{ $\rightarrow$ }>{\ttfamily}l@{}}
%   rerunfilecheck.sty & tex/latex/oberdiek/rerunfilecheck.sty\\
%   rerunfilecheck.pdf & doc/latex/oberdiek/rerunfilecheck.pdf\\
%   rerunfilecheck-example.cfg & doc/latex/oberdiek/rerunfilecheck-example.cfg\\
%   test/rerunfilecheck-test1.tex & doc/latex/oberdiek/test/rerunfilecheck-test1.tex\\
%   rerunfilecheck.dtx & source/latex/oberdiek/rerunfilecheck.dtx\\
% \end{tabular}^^A
% }^^A
% \sbox0{\t}^^A
% \ifdim\wd0>\linewidth
%   \begingroup
%     \advance\linewidth by\leftmargin
%     \advance\linewidth by\rightmargin
%   \edef\x{\endgroup
%     \def\noexpand\lw{\the\linewidth}^^A
%   }\x
%   \def\lwbox{^^A
%     \leavevmode
%     \hbox to \linewidth{^^A
%       \kern-\leftmargin\relax
%       \hss
%       \usebox0
%       \hss
%       \kern-\rightmargin\relax
%     }^^A
%   }^^A
%   \ifdim\wd0>\lw
%     \sbox0{\small\t}^^A
%     \ifdim\wd0>\linewidth
%       \ifdim\wd0>\lw
%         \sbox0{\footnotesize\t}^^A
%         \ifdim\wd0>\linewidth
%           \ifdim\wd0>\lw
%             \sbox0{\scriptsize\t}^^A
%             \ifdim\wd0>\linewidth
%               \ifdim\wd0>\lw
%                 \sbox0{\tiny\t}^^A
%                 \ifdim\wd0>\linewidth
%                   \lwbox
%                 \else
%                   \usebox0
%                 \fi
%               \else
%                 \lwbox
%               \fi
%             \else
%               \usebox0
%             \fi
%           \else
%             \lwbox
%           \fi
%         \else
%           \usebox0
%         \fi
%       \else
%         \lwbox
%       \fi
%     \else
%       \usebox0
%     \fi
%   \else
%     \lwbox
%   \fi
% \else
%   \usebox0
% \fi
% \end{quote}
% If you have a \xfile{docstrip.cfg} that configures and enables \docstrip's
% TDS installing feature, then some files can already be in the right
% place, see the documentation of \docstrip.
%
% \subsection{Refresh file name databases}
%
% If your \TeX~distribution
% (\teTeX, \mikTeX, \dots) relies on file name databases, you must refresh
% these. For example, \teTeX\ users run \verb|texhash| or
% \verb|mktexlsr|.
%
% \subsection{Some details for the interested}
%
% \paragraph{Attached source.}
%
% The PDF documentation on CTAN also includes the
% \xfile{.dtx} source file. It can be extracted by
% AcrobatReader 6 or higher. Another option is \textsf{pdftk},
% e.g. unpack the file into the current directory:
% \begin{quote}
%   \verb|pdftk rerunfilecheck.pdf unpack_files output .|
% \end{quote}
%
% \paragraph{Unpacking with \LaTeX.}
% The \xfile{.dtx} chooses its action depending on the format:
% \begin{description}
% \item[\plainTeX:] Run \docstrip\ and extract the files.
% \item[\LaTeX:] Generate the documentation.
% \end{description}
% If you insist on using \LaTeX\ for \docstrip\ (really,
% \docstrip\ does not need \LaTeX), then inform the autodetect routine
% about your intention:
% \begin{quote}
%   \verb|latex \let\install=y\input{rerunfilecheck.dtx}|
% \end{quote}
% Do not forget to quote the argument according to the demands
% of your shell.
%
% \paragraph{Generating the documentation.}
% You can use both the \xfile{.dtx} or the \xfile{.drv} to generate
% the documentation. The process can be configured by the
% configuration file \xfile{ltxdoc.cfg}. For instance, put this
% line into this file, if you want to have A4 as paper format:
% \begin{quote}
%   \verb|\PassOptionsToClass{a4paper}{article}|
% \end{quote}
% An example follows how to generate the
% documentation with pdf\LaTeX:
% \begin{quote}
%\begin{verbatim}
%pdflatex rerunfilecheck.dtx
%makeindex -s gind.ist rerunfilecheck.idx
%pdflatex rerunfilecheck.dtx
%makeindex -s gind.ist rerunfilecheck.idx
%pdflatex rerunfilecheck.dtx
%\end{verbatim}
% \end{quote}
%
% \section{Catalogue}
%
% The following XML file can be used as source for the
% \href{http://mirror.ctan.org/help/Catalogue/catalogue.html}{\TeX\ Catalogue}.
% The elements \texttt{caption} and \texttt{description} are imported
% from the original XML file from the Catalogue.
% The name of the XML file in the Catalogue is \xfile{rerunfilecheck.xml}.
%    \begin{macrocode}
%<*catalogue>
<?xml version='1.0' encoding='us-ascii'?>
<!DOCTYPE entry SYSTEM 'catalogue.dtd'>
<entry datestamp='$Date$' modifier='$Author$' id='rerunfilecheck'>
  <name>rerunfilecheck</name>
  <caption>Checksum based rerun checks on auxiliary files.</caption>
  <authorref id='auth:oberdiek'/>
  <copyright owner='Heiko Oberdiek' year='2009-2011'/>
  <license type='lppl1.3'/>
  <version number='1.8'/>
  <description>
    The package provides additional rerun warnings if some
    auxiliary files have changed. It is based on MD5 checksum,
    provided by pdfTeX.
    <p/>
    The package is part of the <xref refid='oberdiek'>oberdiek</xref> bundle.
  </description>
  <documentation details='Package documentation'
      href='ctan:/macros/latex/contrib/oberdiek/rerunfilecheck.pdf'/>
  <ctan file='true' path='/macros/latex/contrib/oberdiek/rerunfilecheck.dtx'/>
  <miktex location='oberdiek'/>
  <texlive location='oberdiek'/>
  <install path='/macros/latex/contrib/oberdiek/oberdiek.tds.zip'/>
</entry>
%</catalogue>
%    \end{macrocode}
%
% \begin{History}
%   \begin{Version}{2009/12/10 v1.0}
%   \item
%     The first version.
%   \end{Version}
%   \begin{Version}{2009/12/12 v1.1}
%   \item
%     Short info shortened.
%   \end{Version}
%   \begin{Version}{2009/12/18 v1.2}
%   \item
%     Required date for package \xpackage{uniquecounter} updated
%     because of bug in this package.
%   \end{Version}
%   \begin{Version}{2010/01/25 v1.3}
%   \item
%     Moved from \texttt{TDS:*/generic/*} to \texttt{TDS:*/latex/*}.
%   \end{Version}
%   \begin{Version}{2010/02/22 v1.4}
%   \item
%     The options of this package are recognized only if they are
%     package options. Global options are ignored. This avoids
%     name clashes with class and other package options
%     (for example, class option `index=totoc').
%   \end{Version}
%   \begin{Version}{2010/03/15 v1.5}
%   \item
%     Call of \cs{pdffivesum} is wrapped in \cs{IfFileExists} to avoid
%     calls of \textsf{mktextex} if this feature is enabled.
%     However \cs{IfFileExists} has file name limitations.
%   \end{Version}
%   \begin{Version}{2010/03/16 v1.6}
%   \item
%     Reverted to version 1.4 and \cs{IfFileExists} wrapper of
%     version 1.5 is removed.
%   \end{Version}
%   \begin{Version}{2011/04/15 v1.7}
%   \item
%     Using \cs{AtEndAfterFileList} of package \xpackage{atveryend}
%     2011/04/15 v1.6 instead of \cs{AtVeryEndDocument}.
%   \end{Version}
%   \begin{Version}{2016/05/16 v1.8}
%   \item
%     Documentation updates.
%   \end{Version}
% \end{History}
%
% \PrintIndex
%
% \Finale
\endinput
|
% \end{quote}
% Do not forget to quote the argument according to the demands
% of your shell.
%
% \paragraph{Generating the documentation.}
% You can use both the \xfile{.dtx} or the \xfile{.drv} to generate
% the documentation. The process can be configured by the
% configuration file \xfile{ltxdoc.cfg}. For instance, put this
% line into this file, if you want to have A4 as paper format:
% \begin{quote}
%   \verb|\PassOptionsToClass{a4paper}{article}|
% \end{quote}
% An example follows how to generate the
% documentation with pdf\LaTeX:
% \begin{quote}
%\begin{verbatim}
%pdflatex rerunfilecheck.dtx
%makeindex -s gind.ist rerunfilecheck.idx
%pdflatex rerunfilecheck.dtx
%makeindex -s gind.ist rerunfilecheck.idx
%pdflatex rerunfilecheck.dtx
%\end{verbatim}
% \end{quote}
%
% \section{Catalogue}
%
% The following XML file can be used as source for the
% \href{http://mirror.ctan.org/help/Catalogue/catalogue.html}{\TeX\ Catalogue}.
% The elements \texttt{caption} and \texttt{description} are imported
% from the original XML file from the Catalogue.
% The name of the XML file in the Catalogue is \xfile{rerunfilecheck.xml}.
%    \begin{macrocode}
%<*catalogue>
<?xml version='1.0' encoding='us-ascii'?>
<!DOCTYPE entry SYSTEM 'catalogue.dtd'>
<entry datestamp='$Date$' modifier='$Author$' id='rerunfilecheck'>
  <name>rerunfilecheck</name>
  <caption>Checksum based rerun checks on auxiliary files.</caption>
  <authorref id='auth:oberdiek'/>
  <copyright owner='Heiko Oberdiek' year='2009-2011'/>
  <license type='lppl1.3'/>
  <version number='1.8'/>
  <description>
    The package provides additional rerun warnings if some
    auxiliary files have changed. It is based on MD5 checksum,
    provided by pdfTeX.
    <p/>
    The package is part of the <xref refid='oberdiek'>oberdiek</xref> bundle.
  </description>
  <documentation details='Package documentation'
      href='ctan:/macros/latex/contrib/oberdiek/rerunfilecheck.pdf'/>
  <ctan file='true' path='/macros/latex/contrib/oberdiek/rerunfilecheck.dtx'/>
  <miktex location='oberdiek'/>
  <texlive location='oberdiek'/>
  <install path='/macros/latex/contrib/oberdiek/oberdiek.tds.zip'/>
</entry>
%</catalogue>
%    \end{macrocode}
%
% \begin{History}
%   \begin{Version}{2009/12/10 v1.0}
%   \item
%     The first version.
%   \end{Version}
%   \begin{Version}{2009/12/12 v1.1}
%   \item
%     Short info shortened.
%   \end{Version}
%   \begin{Version}{2009/12/18 v1.2}
%   \item
%     Required date for package \xpackage{uniquecounter} updated
%     because of bug in this package.
%   \end{Version}
%   \begin{Version}{2010/01/25 v1.3}
%   \item
%     Moved from \texttt{TDS:*/generic/*} to \texttt{TDS:*/latex/*}.
%   \end{Version}
%   \begin{Version}{2010/02/22 v1.4}
%   \item
%     The options of this package are recognized only if they are
%     package options. Global options are ignored. This avoids
%     name clashes with class and other package options
%     (for example, class option `index=totoc').
%   \end{Version}
%   \begin{Version}{2010/03/15 v1.5}
%   \item
%     Call of \cs{pdffivesum} is wrapped in \cs{IfFileExists} to avoid
%     calls of \textsf{mktextex} if this feature is enabled.
%     However \cs{IfFileExists} has file name limitations.
%   \end{Version}
%   \begin{Version}{2010/03/16 v1.6}
%   \item
%     Reverted to version 1.4 and \cs{IfFileExists} wrapper of
%     version 1.5 is removed.
%   \end{Version}
%   \begin{Version}{2011/04/15 v1.7}
%   \item
%     Using \cs{AtEndAfterFileList} of package \xpackage{atveryend}
%     2011/04/15 v1.6 instead of \cs{AtVeryEndDocument}.
%   \end{Version}
%   \begin{Version}{2016/05/16 v1.8}
%   \item
%     Documentation updates.
%   \end{Version}
% \end{History}
%
% \PrintIndex
%
% \Finale
\endinput
|
% \end{quote}
% Do not forget to quote the argument according to the demands
% of your shell.
%
% \paragraph{Generating the documentation.}
% You can use both the \xfile{.dtx} or the \xfile{.drv} to generate
% the documentation. The process can be configured by the
% configuration file \xfile{ltxdoc.cfg}. For instance, put this
% line into this file, if you want to have A4 as paper format:
% \begin{quote}
%   \verb|\PassOptionsToClass{a4paper}{article}|
% \end{quote}
% An example follows how to generate the
% documentation with pdf\LaTeX:
% \begin{quote}
%\begin{verbatim}
%pdflatex rerunfilecheck.dtx
%makeindex -s gind.ist rerunfilecheck.idx
%pdflatex rerunfilecheck.dtx
%makeindex -s gind.ist rerunfilecheck.idx
%pdflatex rerunfilecheck.dtx
%\end{verbatim}
% \end{quote}
%
% \section{Catalogue}
%
% The following XML file can be used as source for the
% \href{http://mirror.ctan.org/help/Catalogue/catalogue.html}{\TeX\ Catalogue}.
% The elements \texttt{caption} and \texttt{description} are imported
% from the original XML file from the Catalogue.
% The name of the XML file in the Catalogue is \xfile{rerunfilecheck.xml}.
%    \begin{macrocode}
%<*catalogue>
<?xml version='1.0' encoding='us-ascii'?>
<!DOCTYPE entry SYSTEM 'catalogue.dtd'>
<entry datestamp='$Date$' modifier='$Author$' id='rerunfilecheck'>
  <name>rerunfilecheck</name>
  <caption>Checksum based rerun checks on auxiliary files.</caption>
  <authorref id='auth:oberdiek'/>
  <copyright owner='Heiko Oberdiek' year='2009-2011'/>
  <license type='lppl1.3'/>
  <version number='1.8'/>
  <description>
    The package provides additional rerun warnings if some
    auxiliary files have changed. It is based on MD5 checksum,
    provided by pdfTeX.
    <p/>
    The package is part of the <xref refid='oberdiek'>oberdiek</xref> bundle.
  </description>
  <documentation details='Package documentation'
      href='ctan:/macros/latex/contrib/oberdiek/rerunfilecheck.pdf'/>
  <ctan file='true' path='/macros/latex/contrib/oberdiek/rerunfilecheck.dtx'/>
  <miktex location='oberdiek'/>
  <texlive location='oberdiek'/>
  <install path='/macros/latex/contrib/oberdiek/oberdiek.tds.zip'/>
</entry>
%</catalogue>
%    \end{macrocode}
%
% \begin{History}
%   \begin{Version}{2009/12/10 v1.0}
%   \item
%     The first version.
%   \end{Version}
%   \begin{Version}{2009/12/12 v1.1}
%   \item
%     Short info shortened.
%   \end{Version}
%   \begin{Version}{2009/12/18 v1.2}
%   \item
%     Required date for package \xpackage{uniquecounter} updated
%     because of bug in this package.
%   \end{Version}
%   \begin{Version}{2010/01/25 v1.3}
%   \item
%     Moved from \texttt{TDS:*/generic/*} to \texttt{TDS:*/latex/*}.
%   \end{Version}
%   \begin{Version}{2010/02/22 v1.4}
%   \item
%     The options of this package are recognized only if they are
%     package options. Global options are ignored. This avoids
%     name clashes with class and other package options
%     (for example, class option `index=totoc').
%   \end{Version}
%   \begin{Version}{2010/03/15 v1.5}
%   \item
%     Call of \cs{pdffivesum} is wrapped in \cs{IfFileExists} to avoid
%     calls of \textsf{mktextex} if this feature is enabled.
%     However \cs{IfFileExists} has file name limitations.
%   \end{Version}
%   \begin{Version}{2010/03/16 v1.6}
%   \item
%     Reverted to version 1.4 and \cs{IfFileExists} wrapper of
%     version 1.5 is removed.
%   \end{Version}
%   \begin{Version}{2011/04/15 v1.7}
%   \item
%     Using \cs{AtEndAfterFileList} of package \xpackage{atveryend}
%     2011/04/15 v1.6 instead of \cs{AtVeryEndDocument}.
%   \end{Version}
%   \begin{Version}{2016/05/16 v1.8}
%   \item
%     Documentation updates.
%   \end{Version}
% \end{History}
%
% \PrintIndex
%
% \Finale
\endinput
|
% \end{quote}
% Do not forget to quote the argument according to the demands
% of your shell.
%
% \paragraph{Generating the documentation.}
% You can use both the \xfile{.dtx} or the \xfile{.drv} to generate
% the documentation. The process can be configured by the
% configuration file \xfile{ltxdoc.cfg}. For instance, put this
% line into this file, if you want to have A4 as paper format:
% \begin{quote}
%   \verb|\PassOptionsToClass{a4paper}{article}|
% \end{quote}
% An example follows how to generate the
% documentation with pdf\LaTeX:
% \begin{quote}
%\begin{verbatim}
%pdflatex rerunfilecheck.dtx
%makeindex -s gind.ist rerunfilecheck.idx
%pdflatex rerunfilecheck.dtx
%makeindex -s gind.ist rerunfilecheck.idx
%pdflatex rerunfilecheck.dtx
%\end{verbatim}
% \end{quote}
%
% \section{Catalogue}
%
% The following XML file can be used as source for the
% \href{http://mirror.ctan.org/help/Catalogue/catalogue.html}{\TeX\ Catalogue}.
% The elements \texttt{caption} and \texttt{description} are imported
% from the original XML file from the Catalogue.
% The name of the XML file in the Catalogue is \xfile{rerunfilecheck.xml}.
%    \begin{macrocode}
%<*catalogue>
<?xml version='1.0' encoding='us-ascii'?>
<!DOCTYPE entry SYSTEM 'catalogue.dtd'>
<entry datestamp='$Date$' modifier='$Author$' id='rerunfilecheck'>
  <name>rerunfilecheck</name>
  <caption>Checksum based rerun checks on auxiliary files.</caption>
  <authorref id='auth:oberdiek'/>
  <copyright owner='Heiko Oberdiek' year='2009-2011'/>
  <license type='lppl1.3'/>
  <version number='1.8'/>
  <description>
    The package provides additional rerun warnings if some
    auxiliary files have changed. It is based on MD5 checksum,
    provided by pdfTeX.
    <p/>
    The package is part of the <xref refid='oberdiek'>oberdiek</xref> bundle.
  </description>
  <documentation details='Package documentation'
      href='ctan:/macros/latex/contrib/oberdiek/rerunfilecheck.pdf'/>
  <ctan file='true' path='/macros/latex/contrib/oberdiek/rerunfilecheck.dtx'/>
  <miktex location='oberdiek'/>
  <texlive location='oberdiek'/>
  <install path='/macros/latex/contrib/oberdiek/oberdiek.tds.zip'/>
</entry>
%</catalogue>
%    \end{macrocode}
%
% \begin{History}
%   \begin{Version}{2009/12/10 v1.0}
%   \item
%     The first version.
%   \end{Version}
%   \begin{Version}{2009/12/12 v1.1}
%   \item
%     Short info shortened.
%   \end{Version}
%   \begin{Version}{2009/12/18 v1.2}
%   \item
%     Required date for package \xpackage{uniquecounter} updated
%     because of bug in this package.
%   \end{Version}
%   \begin{Version}{2010/01/25 v1.3}
%   \item
%     Moved from \texttt{TDS:*/generic/*} to \texttt{TDS:*/latex/*}.
%   \end{Version}
%   \begin{Version}{2010/02/22 v1.4}
%   \item
%     The options of this package are recognized only if they are
%     package options. Global options are ignored. This avoids
%     name clashes with class and other package options
%     (for example, class option `index=totoc').
%   \end{Version}
%   \begin{Version}{2010/03/15 v1.5}
%   \item
%     Call of \cs{pdffivesum} is wrapped in \cs{IfFileExists} to avoid
%     calls of \textsf{mktextex} if this feature is enabled.
%     However \cs{IfFileExists} has file name limitations.
%   \end{Version}
%   \begin{Version}{2010/03/16 v1.6}
%   \item
%     Reverted to version 1.4 and \cs{IfFileExists} wrapper of
%     version 1.5 is removed.
%   \end{Version}
%   \begin{Version}{2011/04/15 v1.7}
%   \item
%     Using \cs{AtEndAfterFileList} of package \xpackage{atveryend}
%     2011/04/15 v1.6 instead of \cs{AtVeryEndDocument}.
%   \end{Version}
%   \begin{Version}{2016/05/16 v1.8}
%   \item
%     Documentation updates.
%   \end{Version}
% \end{History}
%
% \PrintIndex
%
% \Finale
\endinput
