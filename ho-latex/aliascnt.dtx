% \iffalse meta-comment
%
% File: aliascnt.dtx
% Version: 2018/09/07 v1.5
% Info: Alias counters
%
% Copyright (C) 2006, 2009 by
%    Heiko Oberdiek <heiko.oberdiek at googlemail.com>
%    2016
%    https://github.com/ho-tex/oberdiek/issues
%
% This work may be distributed and/or modified under the
% conditions of the LaTeX Project Public License, either
% version 1.3c of this license or (at your option) any later
% version. This version of this license is in
%    http://www.latex-project.org/lppl/lppl-1-3c.txt
% and the latest version of this license is in
%    http://www.latex-project.org/lppl.txt
% and version 1.3 or later is part of all distributions of
% LaTeX version 2005/12/01 or later.
%
% This work has the LPPL maintenance status "maintained".
%
% This Current Maintainer of this work is Heiko Oberdiek.
%
% This work consists of the main source file aliascnt.dtx
% and the derived files
%    aliascnt.sty, aliascnt.pdf, aliascnt.ins, aliascnt.drv.
%
% Distribution:
%    CTAN:macros/latex/contrib/oberdiek/aliascnt.dtx
%    CTAN:macros/latex/contrib/oberdiek/aliascnt.pdf
%
% Unpacking:
%    (a) If aliascnt.ins is present:
%           tex aliascnt.ins
%    (b) Without aliascnt.ins:
%           tex aliascnt.dtx
%    (c) If you insist on using LaTeX
%           latex \let\install=y% \iffalse meta-comment
%
% File: aliascnt.dtx
% Version: 2009/09/08 v1.3
% Info: Alias counters
%
% Copyright (C) 2006, 2009 by
%    Heiko Oberdiek <heiko.oberdiek at googlemail.com>
%
% This work may be distributed and/or modified under the
% conditions of the LaTeX Project Public License, either
% version 1.3c of this license or (at your option) any later
% version. This version of this license is in
%    http://www.latex-project.org/lppl/lppl-1-3c.txt
% and the latest version of this license is in
%    http://www.latex-project.org/lppl.txt
% and version 1.3 or later is part of all distributions of
% LaTeX version 2005/12/01 or later.
%
% This work has the LPPL maintenance status "maintained".
%
% This Current Maintainer of this work is Heiko Oberdiek.
%
% This work consists of the main source file aliascnt.dtx
% and the derived files
%    aliascnt.sty, aliascnt.pdf, aliascnt.ins, aliascnt.drv.
%
% Distribution:
%    CTAN:macros/latex/contrib/oberdiek/aliascnt.dtx
%    CTAN:macros/latex/contrib/oberdiek/aliascnt.pdf
%
% Unpacking:
%    (a) If aliascnt.ins is present:
%           tex aliascnt.ins
%    (b) Without aliascnt.ins:
%           tex aliascnt.dtx
%    (c) If you insist on using LaTeX
%           latex \let\install=y% \iffalse meta-comment
%
% File: aliascnt.dtx
% Version: 2009/09/08 v1.3
% Info: Alias counters
%
% Copyright (C) 2006, 2009 by
%    Heiko Oberdiek <heiko.oberdiek at googlemail.com>
%
% This work may be distributed and/or modified under the
% conditions of the LaTeX Project Public License, either
% version 1.3c of this license or (at your option) any later
% version. This version of this license is in
%    http://www.latex-project.org/lppl/lppl-1-3c.txt
% and the latest version of this license is in
%    http://www.latex-project.org/lppl.txt
% and version 1.3 or later is part of all distributions of
% LaTeX version 2005/12/01 or later.
%
% This work has the LPPL maintenance status "maintained".
%
% This Current Maintainer of this work is Heiko Oberdiek.
%
% This work consists of the main source file aliascnt.dtx
% and the derived files
%    aliascnt.sty, aliascnt.pdf, aliascnt.ins, aliascnt.drv.
%
% Distribution:
%    CTAN:macros/latex/contrib/oberdiek/aliascnt.dtx
%    CTAN:macros/latex/contrib/oberdiek/aliascnt.pdf
%
% Unpacking:
%    (a) If aliascnt.ins is present:
%           tex aliascnt.ins
%    (b) Without aliascnt.ins:
%           tex aliascnt.dtx
%    (c) If you insist on using LaTeX
%           latex \let\install=y% \iffalse meta-comment
%
% File: aliascnt.dtx
% Version: 2009/09/08 v1.3
% Info: Alias counters
%
% Copyright (C) 2006, 2009 by
%    Heiko Oberdiek <heiko.oberdiek at googlemail.com>
%
% This work may be distributed and/or modified under the
% conditions of the LaTeX Project Public License, either
% version 1.3c of this license or (at your option) any later
% version. This version of this license is in
%    http://www.latex-project.org/lppl/lppl-1-3c.txt
% and the latest version of this license is in
%    http://www.latex-project.org/lppl.txt
% and version 1.3 or later is part of all distributions of
% LaTeX version 2005/12/01 or later.
%
% This work has the LPPL maintenance status "maintained".
%
% This Current Maintainer of this work is Heiko Oberdiek.
%
% This work consists of the main source file aliascnt.dtx
% and the derived files
%    aliascnt.sty, aliascnt.pdf, aliascnt.ins, aliascnt.drv.
%
% Distribution:
%    CTAN:macros/latex/contrib/oberdiek/aliascnt.dtx
%    CTAN:macros/latex/contrib/oberdiek/aliascnt.pdf
%
% Unpacking:
%    (a) If aliascnt.ins is present:
%           tex aliascnt.ins
%    (b) Without aliascnt.ins:
%           tex aliascnt.dtx
%    (c) If you insist on using LaTeX
%           latex \let\install=y\input{aliascnt.dtx}
%        (quote the arguments according to the demands of your shell)
%
% Documentation:
%    (a) If aliascnt.drv is present:
%           latex aliascnt.drv
%    (b) Without aliascnt.drv:
%           latex aliascnt.dtx; ...
%    The class ltxdoc loads the configuration file ltxdoc.cfg
%    if available. Here you can specify further options, e.g.
%    use A4 as paper format:
%       \PassOptionsToClass{a4paper}{article}
%
%    Programm calls to get the documentation (example):
%       pdflatex aliascnt.dtx
%       makeindex -s gind.ist aliascnt.idx
%       pdflatex aliascnt.dtx
%       makeindex -s gind.ist aliascnt.idx
%       pdflatex aliascnt.dtx
%
% Installation:
%    TDS:tex/latex/oberdiek/aliascnt.sty
%    TDS:doc/latex/oberdiek/aliascnt.pdf
%    TDS:source/latex/oberdiek/aliascnt.dtx
%
%<*ignore>
\begingroup
  \catcode123=1 %
  \catcode125=2 %
  \def\x{LaTeX2e}%
\expandafter\endgroup
\ifcase 0\ifx\install y1\fi\expandafter
         \ifx\csname processbatchFile\endcsname\relax\else1\fi
         \ifx\fmtname\x\else 1\fi\relax
\else\csname fi\endcsname
%</ignore>
%<*install>
\input docstrip.tex
\Msg{************************************************************************}
\Msg{* Installation}
\Msg{* Package: aliascnt 2009/09/08 v1.3 Alias counters (HO)}
\Msg{************************************************************************}

\keepsilent
\askforoverwritefalse

\let\MetaPrefix\relax
\preamble

This is a generated file.

Project: aliascnt
Version: 2009/09/08 v1.3

Copyright (C) 2006, 2009 by
   Heiko Oberdiek <heiko.oberdiek at googlemail.com>

This work may be distributed and/or modified under the
conditions of the LaTeX Project Public License, either
version 1.3c of this license or (at your option) any later
version. This version of this license is in
   http://www.latex-project.org/lppl/lppl-1-3c.txt
and the latest version of this license is in
   http://www.latex-project.org/lppl.txt
and version 1.3 or later is part of all distributions of
LaTeX version 2005/12/01 or later.

This work has the LPPL maintenance status "maintained".

This Current Maintainer of this work is Heiko Oberdiek.

This work consists of the main source file aliascnt.dtx
and the derived files
   aliascnt.sty, aliascnt.pdf, aliascnt.ins, aliascnt.drv.

\endpreamble
\let\MetaPrefix\DoubleperCent

\generate{%
  \file{aliascnt.ins}{\from{aliascnt.dtx}{install}}%
  \file{aliascnt.drv}{\from{aliascnt.dtx}{driver}}%
  \usedir{tex/latex/oberdiek}%
  \file{aliascnt.sty}{\from{aliascnt.dtx}{package}}%
  \nopreamble
  \nopostamble
  \usedir{source/latex/oberdiek/catalogue}%
  \file{aliascnt.xml}{\from{aliascnt.dtx}{catalogue}}%
}

\catcode32=13\relax% active space
\let =\space%
\Msg{************************************************************************}
\Msg{*}
\Msg{* To finish the installation you have to move the following}
\Msg{* file into a directory searched by TeX:}
\Msg{*}
\Msg{*     aliascnt.sty}
\Msg{*}
\Msg{* To produce the documentation run the file `aliascnt.drv'}
\Msg{* through LaTeX.}
\Msg{*}
\Msg{* Happy TeXing!}
\Msg{*}
\Msg{************************************************************************}

\endbatchfile
%</install>
%<*ignore>
\fi
%</ignore>
%<*driver>
\NeedsTeXFormat{LaTeX2e}
\ProvidesFile{aliascnt.drv}%
  [2009/09/08 v1.3 Alias counters (HO)]%
\documentclass{ltxdoc}
\usepackage{holtxdoc}[2011/11/22]
\begin{document}
  \DocInput{aliascnt.dtx}%
\end{document}
%</driver>
% \fi
%
% \CheckSum{78}
%
% \CharacterTable
%  {Upper-case    \A\B\C\D\E\F\G\H\I\J\K\L\M\N\O\P\Q\R\S\T\U\V\W\X\Y\Z
%   Lower-case    \a\b\c\d\e\f\g\h\i\j\k\l\m\n\o\p\q\r\s\t\u\v\w\x\y\z
%   Digits        \0\1\2\3\4\5\6\7\8\9
%   Exclamation   \!     Double quote  \"     Hash (number) \#
%   Dollar        \$     Percent       \%     Ampersand     \&
%   Acute accent  \'     Left paren    \(     Right paren   \)
%   Asterisk      \*     Plus          \+     Comma         \,
%   Minus         \-     Point         \.     Solidus       \/
%   Colon         \:     Semicolon     \;     Less than     \<
%   Equals        \=     Greater than  \>     Question mark \?
%   Commercial at \@     Left bracket  \[     Backslash     \\
%   Right bracket \]     Circumflex    \^     Underscore    \_
%   Grave accent  \`     Left brace    \{     Vertical bar  \|
%   Right brace   \}     Tilde         \~}
%
% \GetFileInfo{aliascnt.drv}
%
% \title{The \xpackage{aliascnt} package}
% \date{2009/09/08 v1.3}
% \author{Heiko Oberdiek\\\xemail{heiko.oberdiek at googlemail.com}}
%
% \maketitle
%
% \begin{abstract}
% Package \xpackage{aliascnt} introduces \emph{alias counters} that
% share the same counter register and clear list.
% \end{abstract}
%
% \tableofcontents
%
% \section{User interface}
%
% \subsection{Introduction}
%
% There are features that rely on the name of counters. For
% example, \xpackage{hyperref}'s \cs{autoref} indirectly uses
% the counter name to determine which label text it puts in front
% of the reference number (\cite{hyperref}).
% In some circumstances this fail: several theorem environments
% are defined by \cs{newtheorem} that share the same counter.
%
% \subsection{Syntax}
%
% Macro names in user land contain the package name
% \texttt{aliascnt} in order to prevent name clashes.
%
% \newenvironment{desc}{^^A
%   \list{}{^^A
%     \setlength{\labelwidth}{0pt}^^A
%     \setlength{\itemindent}{-.5\marginparwidth}^^A
%     \setlength{\leftmargin}{0pt}^^A
%     \let\makelabel\desclabel
%   }^^A
% }{^^A
%   \endlist
% }
% \newcommand*{\desclabel}[1]{^^A
%   \hspace{\labelsep}^^A
%   \normalfont
%   #1^^A
% }
% \newcommand*{\itemcs}[2]{^^A
%   \item[^^A
%      \expandafter\SpecialUsageIndex\csname #1\endcsname
%      {\cs{#1}#2}^^A
%   ]\mbox{}\\*[.5ex]^^A
%   \ignorespaces
% }
% \begin{desc}
% \itemcs{newaliascnt}{\marg{ALIASCNT}\marg{BASECNT}}
%    An alias counter ALIASCNT is created that does not allocate
%    a new \TeX\ counter register. It shares the count register and
%    the clear list with counter BASECNT. If the value of either
%    the two registers is changed, the changes affects both.
% \itemcs{aliascntresetthe}{\marg{ALIASCNT}}
%    This fixes a problem with \cs{newtheorem} if it
%    is fooled by an alias counter with the same name:
%    \begin{quote}
%\begin{verbatim}
%\newtheorem{foo}{Foo}% counter "foo"
%\newaliascnt{bar}{foo}% alias counter "bar"
%\newtheorem{bar}[bar]{Bar}
%\aliascntresetthe{bar}
%\end{verbatim}
%    \end{quote}
% \end{desc}
%
% \StopEventually{
% }
%
% \section{Implementation}
%
% \subsection{Identification}
%
%    \begin{macrocode}
%<*package>
\NeedsTeXFormat{LaTeX2e}
\ProvidesPackage{aliascnt}%
  [2009/09/08 v1.3 Alias counters (HO)]%
%    \end{macrocode}
%
% \subsection{Create new alias counter}
%
%    \begin{macro}{\newaliascnt}
%    A new alias counter is set up by \cs{newaliascnt}.
%    The following properties are added for the new counter CNT:
%    \begin{description}
%    \item[\mdseries\cs{theH}\meta{CNT}:] Compatibility for \xpackage{hyperref}
%    \item[\mdseries\cs{AC@cnt@}\meta{CNT}:] Name of the referenced counter
%      in the definition.
%    \end{description}
%    \begin{macrocode}
\newcommand*{\newaliascnt}[2]{%
  \begingroup
    \def\AC@glet##1{%
      \global\expandafter\let\csname##1#1\expandafter\endcsname
        \csname##1#2\endcsname
    }%
    \@ifundefined{c@#2}{%
      \@nocounterr{#2}%
    }{%
      \expandafter\@ifdefinable\csname c@#1\endcsname{%
        \AC@glet{c@}%
        \AC@glet{the}%
        \AC@glet{theH}%
        \AC@glet{p@}%
        \expandafter\gdef\csname AC@cnt@#1\endcsname{#2}%
        \expandafter\gdef\csname cl@#1\expandafter\endcsname
        \expandafter{\csname cl@#2\endcsname}%
      }%
    }%
  \endgroup
}
%    \end{macrocode}
%    \end{macro}
%
%    \begin{macro}{\aliascntresetthe}
%    The \cs{the}\meta{CNT} macro is restored using the
%    main counter.
%    \begin{macrocode}
\newcommand*{\aliascntresetthe}[1]{%
  \@ifundefined{AC@cnt@#1}{%
    \PackageError{aliascnt}{%
      `#1' is not an alias counter%
    }\@ehc
  }{%
    \expandafter\let\csname the#1\expandafter\endcsname
      \csname the\csname AC@cnt@#1\endcsname\endcsname
  }%
}
%    \end{macrocode}
%    \end{macro}
%
% \subsection{Counter clear list}
%
%    The alias counters share the same register and clear list.
%    Therefore we must ensure that manipulations to the clear list
%    are done with the clear list macro of a real counter.
%    \begin{macro}{\AC@findrootcnt}
%    \cs{AC@findrootcnt} walks throught the aliasing relations
%    to find the base counter.
%    \begin{macrocode}
\newcommand*{\AC@findrootcnt}[1]{%
  \@ifundefined{AC@cnt@#1}{%
    #1%
  }{%
    \expandafter\AC@findrootcnt\csname AC@cnt@#1\endcsname
  }%
}
%    \end{macrocode}
%    \end{macro}
%
%    Clear lists are manipulated by \cs{@addtoreset} and
%    \cs{@removefromreset}. The latter one is provided by
%    the \xpackage{remreset} package (\cite{remreset}).
%
%    \begin{macro}{\AC@patch}
%    The same patch principle is applicable to both
%    \cs{@addtoreset} and \cs{@removefromreset}.
%    \begin{macrocode}
\def\AC@patch#1{%
  \expandafter\let\csname AC@org@#1reset\expandafter\endcsname
    \csname @#1reset\endcsname
  \expandafter\def\csname @#1reset\endcsname##1##2{%
    \csname AC@org@#1reset\endcsname{##1}{\AC@findrootcnt{##2}}%
  }%
}
%    \end{macrocode}
%    \end{macro}
%    If \xpackage{remreset} is not loaded we cannot delay
%    the patch to \cs{AtBeginDocumen}, because \cs{@removefromreset}
%    can be called in between. Therefore we force the loading of
%    the package.
%    \begin{macrocode}
\RequirePackage{remreset}
\AC@patch{addto}
\AC@patch{removefrom}
%    \end{macrocode}
%
%    \begin{macrocode}
%</package>
%    \end{macrocode}
%
% \section{Installation}
%
% \subsection{Download}
%
% \paragraph{Package.} This package is available on
% CTAN\footnote{\url{ftp://ftp.ctan.org/tex-archive/}}:
% \begin{description}
% \item[\CTAN{macros/latex/contrib/oberdiek/aliascnt.dtx}] The source file.
% \item[\CTAN{macros/latex/contrib/oberdiek/aliascnt.pdf}] Documentation.
% \end{description}
%
%
% \paragraph{Bundle.} All the packages of the bundle `oberdiek'
% are also available in a TDS compliant ZIP archive. There
% the packages are already unpacked and the documentation files
% are generated. The files and directories obey the TDS standard.
% \begin{description}
% \item[\CTAN{install/macros/latex/contrib/oberdiek.tds.zip}]
% \end{description}
% \emph{TDS} refers to the standard ``A Directory Structure
% for \TeX\ Files'' (\CTAN{tds/tds.pdf}). Directories
% with \xfile{texmf} in their name are usually organized this way.
%
% \subsection{Bundle installation}
%
% \paragraph{Unpacking.} Unpack the \xfile{oberdiek.tds.zip} in the
% TDS tree (also known as \xfile{texmf} tree) of your choice.
% Example (linux):
% \begin{quote}
%   |unzip oberdiek.tds.zip -d ~/texmf|
% \end{quote}
%
% \paragraph{Script installation.}
% Check the directory \xfile{TDS:scripts/oberdiek/} for
% scripts that need further installation steps.
% Package \xpackage{attachfile2} comes with the Perl script
% \xfile{pdfatfi.pl} that should be installed in such a way
% that it can be called as \texttt{pdfatfi}.
% Example (linux):
% \begin{quote}
%   |chmod +x scripts/oberdiek/pdfatfi.pl|\\
%   |cp scripts/oberdiek/pdfatfi.pl /usr/local/bin/|
% \end{quote}
%
% \subsection{Package installation}
%
% \paragraph{Unpacking.} The \xfile{.dtx} file is a self-extracting
% \docstrip\ archive. The files are extracted by running the
% \xfile{.dtx} through \plainTeX:
% \begin{quote}
%   \verb|tex aliascnt.dtx|
% \end{quote}
%
% \paragraph{TDS.} Now the different files must be moved into
% the different directories in your installation TDS tree
% (also known as \xfile{texmf} tree):
% \begin{quote}
% \def\t{^^A
% \begin{tabular}{@{}>{\ttfamily}l@{ $\rightarrow$ }>{\ttfamily}l@{}}
%   aliascnt.sty & tex/latex/oberdiek/aliascnt.sty\\
%   aliascnt.pdf & doc/latex/oberdiek/aliascnt.pdf\\
%   aliascnt.dtx & source/latex/oberdiek/aliascnt.dtx\\
% \end{tabular}^^A
% }^^A
% \sbox0{\t}^^A
% \ifdim\wd0>\linewidth
%   \begingroup
%     \advance\linewidth by\leftmargin
%     \advance\linewidth by\rightmargin
%   \edef\x{\endgroup
%     \def\noexpand\lw{\the\linewidth}^^A
%   }\x
%   \def\lwbox{^^A
%     \leavevmode
%     \hbox to \linewidth{^^A
%       \kern-\leftmargin\relax
%       \hss
%       \usebox0
%       \hss
%       \kern-\rightmargin\relax
%     }^^A
%   }^^A
%   \ifdim\wd0>\lw
%     \sbox0{\small\t}^^A
%     \ifdim\wd0>\linewidth
%       \ifdim\wd0>\lw
%         \sbox0{\footnotesize\t}^^A
%         \ifdim\wd0>\linewidth
%           \ifdim\wd0>\lw
%             \sbox0{\scriptsize\t}^^A
%             \ifdim\wd0>\linewidth
%               \ifdim\wd0>\lw
%                 \sbox0{\tiny\t}^^A
%                 \ifdim\wd0>\linewidth
%                   \lwbox
%                 \else
%                   \usebox0
%                 \fi
%               \else
%                 \lwbox
%               \fi
%             \else
%               \usebox0
%             \fi
%           \else
%             \lwbox
%           \fi
%         \else
%           \usebox0
%         \fi
%       \else
%         \lwbox
%       \fi
%     \else
%       \usebox0
%     \fi
%   \else
%     \lwbox
%   \fi
% \else
%   \usebox0
% \fi
% \end{quote}
% If you have a \xfile{docstrip.cfg} that configures and enables \docstrip's
% TDS installing feature, then some files can already be in the right
% place, see the documentation of \docstrip.
%
% \subsection{Refresh file name databases}
%
% If your \TeX~distribution
% (\teTeX, \mikTeX, \dots) relies on file name databases, you must refresh
% these. For example, \teTeX\ users run \verb|texhash| or
% \verb|mktexlsr|.
%
% \subsection{Some details for the interested}
%
% \paragraph{Attached source.}
%
% The PDF documentation on CTAN also includes the
% \xfile{.dtx} source file. It can be extracted by
% AcrobatReader 6 or higher. Another option is \textsf{pdftk},
% e.g. unpack the file into the current directory:
% \begin{quote}
%   \verb|pdftk aliascnt.pdf unpack_files output .|
% \end{quote}
%
% \paragraph{Unpacking with \LaTeX.}
% The \xfile{.dtx} chooses its action depending on the format:
% \begin{description}
% \item[\plainTeX:] Run \docstrip\ and extract the files.
% \item[\LaTeX:] Generate the documentation.
% \end{description}
% If you insist on using \LaTeX\ for \docstrip\ (really,
% \docstrip\ does not need \LaTeX), then inform the autodetect routine
% about your intention:
% \begin{quote}
%   \verb|latex \let\install=y\input{aliascnt.dtx}|
% \end{quote}
% Do not forget to quote the argument according to the demands
% of your shell.
%
% \paragraph{Generating the documentation.}
% You can use both the \xfile{.dtx} or the \xfile{.drv} to generate
% the documentation. The process can be configured by the
% configuration file \xfile{ltxdoc.cfg}. For instance, put this
% line into this file, if you want to have A4 as paper format:
% \begin{quote}
%   \verb|\PassOptionsToClass{a4paper}{article}|
% \end{quote}
% An example follows how to generate the
% documentation with pdf\LaTeX:
% \begin{quote}
%\begin{verbatim}
%pdflatex aliascnt.dtx
%makeindex -s gind.ist aliascnt.idx
%pdflatex aliascnt.dtx
%makeindex -s gind.ist aliascnt.idx
%pdflatex aliascnt.dtx
%\end{verbatim}
% \end{quote}
%
% \section{Catalogue}
%
% The following XML file can be used as source for the
% \href{http://mirror.ctan.org/help/Catalogue/catalogue.html}{\TeX\ Catalogue}.
% The elements \texttt{caption} and \texttt{description} are imported
% from the original XML file from the Catalogue.
% The name of the XML file in the Catalogue is \xfile{aliascnt.xml}.
%    \begin{macrocode}
%<*catalogue>
<?xml version='1.0' encoding='us-ascii'?>
<!DOCTYPE entry SYSTEM 'catalogue.dtd'>
<entry datestamp='$Date$' modifier='$Author$' id='aliascnt'>
  <name>aliascnt</name>
  <caption>Alias counters.</caption>
  <authorref id='auth:oberdiek'/>
  <copyright owner='Heiko Oberdiek' year='2006,2009'/>
  <license type='lppl1.3'/>
  <version number='1.3'/>
  <description>
    This package introduces aliases for counters, that
    share the same counter register and clear list.
    <p/>
    The package is part of the <xref refid='oberdiek'>oberdiek</xref>
    bundle.
  </description>
  <documentation details='Package documentation'
      href='ctan:/macros/latex/contrib/oberdiek/aliascnt.pdf'/>
  <ctan file='true' path='/macros/latex/contrib/oberdiek/aliascnt.dtx'/>
  <miktex location='oberdiek'/>
  <texlive location='oberdiek'/>
  <install path='/macros/latex/contrib/oberdiek/oberdiek.tds.zip'/>
</entry>
%</catalogue>
%    \end{macrocode}
%
% \section{Acknowledgement}
%
% \begin{description}
% \item[Ulrich Schwarz:] The package is based on his draft for
%   ``Die \TeX nische Kom\"odie'', see \cite{schwarz}.
% \end{description}
%
% \begin{thebibliography}{9}
%
% \bibitem{schwarz}
%   Ulrich Schwarz:
%   \textit{Was hinten herauskommt z\"ahlt: Counter Aliasing in \LaTeX},
%   \textit{Die \TeX nische Kom\"odie}, 3/2006, pages 8--14, Juli 2006.
%
% \bibitem{remreset}
%   David Carlisle: \textit{The \xpackage{remreset} package};
%   1997/09/28;
%   \CTAN{macros/latex/contrib/carlisle/remreset.sty}.
%
% \bibitem{hyperref}
%   Sebastian Rahtz, Heiko Oberdiek:
%   \textit{The \xpackage{hyperref} package};
%   2006/08/16 v6.75c;
%   \CTAN{macros/latex/contrib/hyperref/}.
%
% \end{thebibliography}
%
% \begin{History}
%   \begin{Version}{2006/02/20 v1.0}
%   \item
%     First version.
%   \end{Version}
%   \begin{Version}{2006/08/16 v1.1}
%   \item
%     Update of bibliography.
%   \end{Version}
%   \begin{Version}{2006/09/25 v1.2}
%   \item
%     Bug fix (\cs{aliascntresetthe}).
%   \end{Version}
%   \begin{Version}{2009/09/08 v1.3}
%   \item
%     Bug fix of \cs{@ifdefinable}'s use (thanks to Uwe L\"uck).
%   \end{Version}
% \end{History}
%
% \PrintIndex
%
% \Finale
\endinput

%        (quote the arguments according to the demands of your shell)
%
% Documentation:
%    (a) If aliascnt.drv is present:
%           latex aliascnt.drv
%    (b) Without aliascnt.drv:
%           latex aliascnt.dtx; ...
%    The class ltxdoc loads the configuration file ltxdoc.cfg
%    if available. Here you can specify further options, e.g.
%    use A4 as paper format:
%       \PassOptionsToClass{a4paper}{article}
%
%    Programm calls to get the documentation (example):
%       pdflatex aliascnt.dtx
%       makeindex -s gind.ist aliascnt.idx
%       pdflatex aliascnt.dtx
%       makeindex -s gind.ist aliascnt.idx
%       pdflatex aliascnt.dtx
%
% Installation:
%    TDS:tex/latex/oberdiek/aliascnt.sty
%    TDS:doc/latex/oberdiek/aliascnt.pdf
%    TDS:source/latex/oberdiek/aliascnt.dtx
%
%<*ignore>
\begingroup
  \catcode123=1 %
  \catcode125=2 %
  \def\x{LaTeX2e}%
\expandafter\endgroup
\ifcase 0\ifx\install y1\fi\expandafter
         \ifx\csname processbatchFile\endcsname\relax\else1\fi
         \ifx\fmtname\x\else 1\fi\relax
\else\csname fi\endcsname
%</ignore>
%<*install>
\input docstrip.tex
\Msg{************************************************************************}
\Msg{* Installation}
\Msg{* Package: aliascnt 2009/09/08 v1.3 Alias counters (HO)}
\Msg{************************************************************************}

\keepsilent
\askforoverwritefalse

\let\MetaPrefix\relax
\preamble

This is a generated file.

Project: aliascnt
Version: 2009/09/08 v1.3

Copyright (C) 2006, 2009 by
   Heiko Oberdiek <heiko.oberdiek at googlemail.com>

This work may be distributed and/or modified under the
conditions of the LaTeX Project Public License, either
version 1.3c of this license or (at your option) any later
version. This version of this license is in
   http://www.latex-project.org/lppl/lppl-1-3c.txt
and the latest version of this license is in
   http://www.latex-project.org/lppl.txt
and version 1.3 or later is part of all distributions of
LaTeX version 2005/12/01 or later.

This work has the LPPL maintenance status "maintained".

This Current Maintainer of this work is Heiko Oberdiek.

This work consists of the main source file aliascnt.dtx
and the derived files
   aliascnt.sty, aliascnt.pdf, aliascnt.ins, aliascnt.drv.

\endpreamble
\let\MetaPrefix\DoubleperCent

\generate{%
  \file{aliascnt.ins}{\from{aliascnt.dtx}{install}}%
  \file{aliascnt.drv}{\from{aliascnt.dtx}{driver}}%
  \usedir{tex/latex/oberdiek}%
  \file{aliascnt.sty}{\from{aliascnt.dtx}{package}}%
  \nopreamble
  \nopostamble
  \usedir{source/latex/oberdiek/catalogue}%
  \file{aliascnt.xml}{\from{aliascnt.dtx}{catalogue}}%
}

\catcode32=13\relax% active space
\let =\space%
\Msg{************************************************************************}
\Msg{*}
\Msg{* To finish the installation you have to move the following}
\Msg{* file into a directory searched by TeX:}
\Msg{*}
\Msg{*     aliascnt.sty}
\Msg{*}
\Msg{* To produce the documentation run the file `aliascnt.drv'}
\Msg{* through LaTeX.}
\Msg{*}
\Msg{* Happy TeXing!}
\Msg{*}
\Msg{************************************************************************}

\endbatchfile
%</install>
%<*ignore>
\fi
%</ignore>
%<*driver>
\NeedsTeXFormat{LaTeX2e}
\ProvidesFile{aliascnt.drv}%
  [2009/09/08 v1.3 Alias counters (HO)]%
\documentclass{ltxdoc}
\usepackage{holtxdoc}[2011/11/22]
\begin{document}
  \DocInput{aliascnt.dtx}%
\end{document}
%</driver>
% \fi
%
% \CheckSum{78}
%
% \CharacterTable
%  {Upper-case    \A\B\C\D\E\F\G\H\I\J\K\L\M\N\O\P\Q\R\S\T\U\V\W\X\Y\Z
%   Lower-case    \a\b\c\d\e\f\g\h\i\j\k\l\m\n\o\p\q\r\s\t\u\v\w\x\y\z
%   Digits        \0\1\2\3\4\5\6\7\8\9
%   Exclamation   \!     Double quote  \"     Hash (number) \#
%   Dollar        \$     Percent       \%     Ampersand     \&
%   Acute accent  \'     Left paren    \(     Right paren   \)
%   Asterisk      \*     Plus          \+     Comma         \,
%   Minus         \-     Point         \.     Solidus       \/
%   Colon         \:     Semicolon     \;     Less than     \<
%   Equals        \=     Greater than  \>     Question mark \?
%   Commercial at \@     Left bracket  \[     Backslash     \\
%   Right bracket \]     Circumflex    \^     Underscore    \_
%   Grave accent  \`     Left brace    \{     Vertical bar  \|
%   Right brace   \}     Tilde         \~}
%
% \GetFileInfo{aliascnt.drv}
%
% \title{The \xpackage{aliascnt} package}
% \date{2009/09/08 v1.3}
% \author{Heiko Oberdiek\\\xemail{heiko.oberdiek at googlemail.com}}
%
% \maketitle
%
% \begin{abstract}
% Package \xpackage{aliascnt} introduces \emph{alias counters} that
% share the same counter register and clear list.
% \end{abstract}
%
% \tableofcontents
%
% \section{User interface}
%
% \subsection{Introduction}
%
% There are features that rely on the name of counters. For
% example, \xpackage{hyperref}'s \cs{autoref} indirectly uses
% the counter name to determine which label text it puts in front
% of the reference number (\cite{hyperref}).
% In some circumstances this fail: several theorem environments
% are defined by \cs{newtheorem} that share the same counter.
%
% \subsection{Syntax}
%
% Macro names in user land contain the package name
% \texttt{aliascnt} in order to prevent name clashes.
%
% \newenvironment{desc}{^^A
%   \list{}{^^A
%     \setlength{\labelwidth}{0pt}^^A
%     \setlength{\itemindent}{-.5\marginparwidth}^^A
%     \setlength{\leftmargin}{0pt}^^A
%     \let\makelabel\desclabel
%   }^^A
% }{^^A
%   \endlist
% }
% \newcommand*{\desclabel}[1]{^^A
%   \hspace{\labelsep}^^A
%   \normalfont
%   #1^^A
% }
% \newcommand*{\itemcs}[2]{^^A
%   \item[^^A
%      \expandafter\SpecialUsageIndex\csname #1\endcsname
%      {\cs{#1}#2}^^A
%   ]\mbox{}\\*[.5ex]^^A
%   \ignorespaces
% }
% \begin{desc}
% \itemcs{newaliascnt}{\marg{ALIASCNT}\marg{BASECNT}}
%    An alias counter ALIASCNT is created that does not allocate
%    a new \TeX\ counter register. It shares the count register and
%    the clear list with counter BASECNT. If the value of either
%    the two registers is changed, the changes affects both.
% \itemcs{aliascntresetthe}{\marg{ALIASCNT}}
%    This fixes a problem with \cs{newtheorem} if it
%    is fooled by an alias counter with the same name:
%    \begin{quote}
%\begin{verbatim}
%\newtheorem{foo}{Foo}% counter "foo"
%\newaliascnt{bar}{foo}% alias counter "bar"
%\newtheorem{bar}[bar]{Bar}
%\aliascntresetthe{bar}
%\end{verbatim}
%    \end{quote}
% \end{desc}
%
% \StopEventually{
% }
%
% \section{Implementation}
%
% \subsection{Identification}
%
%    \begin{macrocode}
%<*package>
\NeedsTeXFormat{LaTeX2e}
\ProvidesPackage{aliascnt}%
  [2009/09/08 v1.3 Alias counters (HO)]%
%    \end{macrocode}
%
% \subsection{Create new alias counter}
%
%    \begin{macro}{\newaliascnt}
%    A new alias counter is set up by \cs{newaliascnt}.
%    The following properties are added for the new counter CNT:
%    \begin{description}
%    \item[\mdseries\cs{theH}\meta{CNT}:] Compatibility for \xpackage{hyperref}
%    \item[\mdseries\cs{AC@cnt@}\meta{CNT}:] Name of the referenced counter
%      in the definition.
%    \end{description}
%    \begin{macrocode}
\newcommand*{\newaliascnt}[2]{%
  \begingroup
    \def\AC@glet##1{%
      \global\expandafter\let\csname##1#1\expandafter\endcsname
        \csname##1#2\endcsname
    }%
    \@ifundefined{c@#2}{%
      \@nocounterr{#2}%
    }{%
      \expandafter\@ifdefinable\csname c@#1\endcsname{%
        \AC@glet{c@}%
        \AC@glet{the}%
        \AC@glet{theH}%
        \AC@glet{p@}%
        \expandafter\gdef\csname AC@cnt@#1\endcsname{#2}%
        \expandafter\gdef\csname cl@#1\expandafter\endcsname
        \expandafter{\csname cl@#2\endcsname}%
      }%
    }%
  \endgroup
}
%    \end{macrocode}
%    \end{macro}
%
%    \begin{macro}{\aliascntresetthe}
%    The \cs{the}\meta{CNT} macro is restored using the
%    main counter.
%    \begin{macrocode}
\newcommand*{\aliascntresetthe}[1]{%
  \@ifundefined{AC@cnt@#1}{%
    \PackageError{aliascnt}{%
      `#1' is not an alias counter%
    }\@ehc
  }{%
    \expandafter\let\csname the#1\expandafter\endcsname
      \csname the\csname AC@cnt@#1\endcsname\endcsname
  }%
}
%    \end{macrocode}
%    \end{macro}
%
% \subsection{Counter clear list}
%
%    The alias counters share the same register and clear list.
%    Therefore we must ensure that manipulations to the clear list
%    are done with the clear list macro of a real counter.
%    \begin{macro}{\AC@findrootcnt}
%    \cs{AC@findrootcnt} walks throught the aliasing relations
%    to find the base counter.
%    \begin{macrocode}
\newcommand*{\AC@findrootcnt}[1]{%
  \@ifundefined{AC@cnt@#1}{%
    #1%
  }{%
    \expandafter\AC@findrootcnt\csname AC@cnt@#1\endcsname
  }%
}
%    \end{macrocode}
%    \end{macro}
%
%    Clear lists are manipulated by \cs{@addtoreset} and
%    \cs{@removefromreset}. The latter one is provided by
%    the \xpackage{remreset} package (\cite{remreset}).
%
%    \begin{macro}{\AC@patch}
%    The same patch principle is applicable to both
%    \cs{@addtoreset} and \cs{@removefromreset}.
%    \begin{macrocode}
\def\AC@patch#1{%
  \expandafter\let\csname AC@org@#1reset\expandafter\endcsname
    \csname @#1reset\endcsname
  \expandafter\def\csname @#1reset\endcsname##1##2{%
    \csname AC@org@#1reset\endcsname{##1}{\AC@findrootcnt{##2}}%
  }%
}
%    \end{macrocode}
%    \end{macro}
%    If \xpackage{remreset} is not loaded we cannot delay
%    the patch to \cs{AtBeginDocumen}, because \cs{@removefromreset}
%    can be called in between. Therefore we force the loading of
%    the package.
%    \begin{macrocode}
\RequirePackage{remreset}
\AC@patch{addto}
\AC@patch{removefrom}
%    \end{macrocode}
%
%    \begin{macrocode}
%</package>
%    \end{macrocode}
%
% \section{Installation}
%
% \subsection{Download}
%
% \paragraph{Package.} This package is available on
% CTAN\footnote{\url{ftp://ftp.ctan.org/tex-archive/}}:
% \begin{description}
% \item[\CTAN{macros/latex/contrib/oberdiek/aliascnt.dtx}] The source file.
% \item[\CTAN{macros/latex/contrib/oberdiek/aliascnt.pdf}] Documentation.
% \end{description}
%
%
% \paragraph{Bundle.} All the packages of the bundle `oberdiek'
% are also available in a TDS compliant ZIP archive. There
% the packages are already unpacked and the documentation files
% are generated. The files and directories obey the TDS standard.
% \begin{description}
% \item[\CTAN{install/macros/latex/contrib/oberdiek.tds.zip}]
% \end{description}
% \emph{TDS} refers to the standard ``A Directory Structure
% for \TeX\ Files'' (\CTAN{tds/tds.pdf}). Directories
% with \xfile{texmf} in their name are usually organized this way.
%
% \subsection{Bundle installation}
%
% \paragraph{Unpacking.} Unpack the \xfile{oberdiek.tds.zip} in the
% TDS tree (also known as \xfile{texmf} tree) of your choice.
% Example (linux):
% \begin{quote}
%   |unzip oberdiek.tds.zip -d ~/texmf|
% \end{quote}
%
% \paragraph{Script installation.}
% Check the directory \xfile{TDS:scripts/oberdiek/} for
% scripts that need further installation steps.
% Package \xpackage{attachfile2} comes with the Perl script
% \xfile{pdfatfi.pl} that should be installed in such a way
% that it can be called as \texttt{pdfatfi}.
% Example (linux):
% \begin{quote}
%   |chmod +x scripts/oberdiek/pdfatfi.pl|\\
%   |cp scripts/oberdiek/pdfatfi.pl /usr/local/bin/|
% \end{quote}
%
% \subsection{Package installation}
%
% \paragraph{Unpacking.} The \xfile{.dtx} file is a self-extracting
% \docstrip\ archive. The files are extracted by running the
% \xfile{.dtx} through \plainTeX:
% \begin{quote}
%   \verb|tex aliascnt.dtx|
% \end{quote}
%
% \paragraph{TDS.} Now the different files must be moved into
% the different directories in your installation TDS tree
% (also known as \xfile{texmf} tree):
% \begin{quote}
% \def\t{^^A
% \begin{tabular}{@{}>{\ttfamily}l@{ $\rightarrow$ }>{\ttfamily}l@{}}
%   aliascnt.sty & tex/latex/oberdiek/aliascnt.sty\\
%   aliascnt.pdf & doc/latex/oberdiek/aliascnt.pdf\\
%   aliascnt.dtx & source/latex/oberdiek/aliascnt.dtx\\
% \end{tabular}^^A
% }^^A
% \sbox0{\t}^^A
% \ifdim\wd0>\linewidth
%   \begingroup
%     \advance\linewidth by\leftmargin
%     \advance\linewidth by\rightmargin
%   \edef\x{\endgroup
%     \def\noexpand\lw{\the\linewidth}^^A
%   }\x
%   \def\lwbox{^^A
%     \leavevmode
%     \hbox to \linewidth{^^A
%       \kern-\leftmargin\relax
%       \hss
%       \usebox0
%       \hss
%       \kern-\rightmargin\relax
%     }^^A
%   }^^A
%   \ifdim\wd0>\lw
%     \sbox0{\small\t}^^A
%     \ifdim\wd0>\linewidth
%       \ifdim\wd0>\lw
%         \sbox0{\footnotesize\t}^^A
%         \ifdim\wd0>\linewidth
%           \ifdim\wd0>\lw
%             \sbox0{\scriptsize\t}^^A
%             \ifdim\wd0>\linewidth
%               \ifdim\wd0>\lw
%                 \sbox0{\tiny\t}^^A
%                 \ifdim\wd0>\linewidth
%                   \lwbox
%                 \else
%                   \usebox0
%                 \fi
%               \else
%                 \lwbox
%               \fi
%             \else
%               \usebox0
%             \fi
%           \else
%             \lwbox
%           \fi
%         \else
%           \usebox0
%         \fi
%       \else
%         \lwbox
%       \fi
%     \else
%       \usebox0
%     \fi
%   \else
%     \lwbox
%   \fi
% \else
%   \usebox0
% \fi
% \end{quote}
% If you have a \xfile{docstrip.cfg} that configures and enables \docstrip's
% TDS installing feature, then some files can already be in the right
% place, see the documentation of \docstrip.
%
% \subsection{Refresh file name databases}
%
% If your \TeX~distribution
% (\teTeX, \mikTeX, \dots) relies on file name databases, you must refresh
% these. For example, \teTeX\ users run \verb|texhash| or
% \verb|mktexlsr|.
%
% \subsection{Some details for the interested}
%
% \paragraph{Attached source.}
%
% The PDF documentation on CTAN also includes the
% \xfile{.dtx} source file. It can be extracted by
% AcrobatReader 6 or higher. Another option is \textsf{pdftk},
% e.g. unpack the file into the current directory:
% \begin{quote}
%   \verb|pdftk aliascnt.pdf unpack_files output .|
% \end{quote}
%
% \paragraph{Unpacking with \LaTeX.}
% The \xfile{.dtx} chooses its action depending on the format:
% \begin{description}
% \item[\plainTeX:] Run \docstrip\ and extract the files.
% \item[\LaTeX:] Generate the documentation.
% \end{description}
% If you insist on using \LaTeX\ for \docstrip\ (really,
% \docstrip\ does not need \LaTeX), then inform the autodetect routine
% about your intention:
% \begin{quote}
%   \verb|latex \let\install=y% \iffalse meta-comment
%
% File: aliascnt.dtx
% Version: 2009/09/08 v1.3
% Info: Alias counters
%
% Copyright (C) 2006, 2009 by
%    Heiko Oberdiek <heiko.oberdiek at googlemail.com>
%
% This work may be distributed and/or modified under the
% conditions of the LaTeX Project Public License, either
% version 1.3c of this license or (at your option) any later
% version. This version of this license is in
%    http://www.latex-project.org/lppl/lppl-1-3c.txt
% and the latest version of this license is in
%    http://www.latex-project.org/lppl.txt
% and version 1.3 or later is part of all distributions of
% LaTeX version 2005/12/01 or later.
%
% This work has the LPPL maintenance status "maintained".
%
% This Current Maintainer of this work is Heiko Oberdiek.
%
% This work consists of the main source file aliascnt.dtx
% and the derived files
%    aliascnt.sty, aliascnt.pdf, aliascnt.ins, aliascnt.drv.
%
% Distribution:
%    CTAN:macros/latex/contrib/oberdiek/aliascnt.dtx
%    CTAN:macros/latex/contrib/oberdiek/aliascnt.pdf
%
% Unpacking:
%    (a) If aliascnt.ins is present:
%           tex aliascnt.ins
%    (b) Without aliascnt.ins:
%           tex aliascnt.dtx
%    (c) If you insist on using LaTeX
%           latex \let\install=y\input{aliascnt.dtx}
%        (quote the arguments according to the demands of your shell)
%
% Documentation:
%    (a) If aliascnt.drv is present:
%           latex aliascnt.drv
%    (b) Without aliascnt.drv:
%           latex aliascnt.dtx; ...
%    The class ltxdoc loads the configuration file ltxdoc.cfg
%    if available. Here you can specify further options, e.g.
%    use A4 as paper format:
%       \PassOptionsToClass{a4paper}{article}
%
%    Programm calls to get the documentation (example):
%       pdflatex aliascnt.dtx
%       makeindex -s gind.ist aliascnt.idx
%       pdflatex aliascnt.dtx
%       makeindex -s gind.ist aliascnt.idx
%       pdflatex aliascnt.dtx
%
% Installation:
%    TDS:tex/latex/oberdiek/aliascnt.sty
%    TDS:doc/latex/oberdiek/aliascnt.pdf
%    TDS:source/latex/oberdiek/aliascnt.dtx
%
%<*ignore>
\begingroup
  \catcode123=1 %
  \catcode125=2 %
  \def\x{LaTeX2e}%
\expandafter\endgroup
\ifcase 0\ifx\install y1\fi\expandafter
         \ifx\csname processbatchFile\endcsname\relax\else1\fi
         \ifx\fmtname\x\else 1\fi\relax
\else\csname fi\endcsname
%</ignore>
%<*install>
\input docstrip.tex
\Msg{************************************************************************}
\Msg{* Installation}
\Msg{* Package: aliascnt 2009/09/08 v1.3 Alias counters (HO)}
\Msg{************************************************************************}

\keepsilent
\askforoverwritefalse

\let\MetaPrefix\relax
\preamble

This is a generated file.

Project: aliascnt
Version: 2009/09/08 v1.3

Copyright (C) 2006, 2009 by
   Heiko Oberdiek <heiko.oberdiek at googlemail.com>

This work may be distributed and/or modified under the
conditions of the LaTeX Project Public License, either
version 1.3c of this license or (at your option) any later
version. This version of this license is in
   http://www.latex-project.org/lppl/lppl-1-3c.txt
and the latest version of this license is in
   http://www.latex-project.org/lppl.txt
and version 1.3 or later is part of all distributions of
LaTeX version 2005/12/01 or later.

This work has the LPPL maintenance status "maintained".

This Current Maintainer of this work is Heiko Oberdiek.

This work consists of the main source file aliascnt.dtx
and the derived files
   aliascnt.sty, aliascnt.pdf, aliascnt.ins, aliascnt.drv.

\endpreamble
\let\MetaPrefix\DoubleperCent

\generate{%
  \file{aliascnt.ins}{\from{aliascnt.dtx}{install}}%
  \file{aliascnt.drv}{\from{aliascnt.dtx}{driver}}%
  \usedir{tex/latex/oberdiek}%
  \file{aliascnt.sty}{\from{aliascnt.dtx}{package}}%
  \nopreamble
  \nopostamble
  \usedir{source/latex/oberdiek/catalogue}%
  \file{aliascnt.xml}{\from{aliascnt.dtx}{catalogue}}%
}

\catcode32=13\relax% active space
\let =\space%
\Msg{************************************************************************}
\Msg{*}
\Msg{* To finish the installation you have to move the following}
\Msg{* file into a directory searched by TeX:}
\Msg{*}
\Msg{*     aliascnt.sty}
\Msg{*}
\Msg{* To produce the documentation run the file `aliascnt.drv'}
\Msg{* through LaTeX.}
\Msg{*}
\Msg{* Happy TeXing!}
\Msg{*}
\Msg{************************************************************************}

\endbatchfile
%</install>
%<*ignore>
\fi
%</ignore>
%<*driver>
\NeedsTeXFormat{LaTeX2e}
\ProvidesFile{aliascnt.drv}%
  [2009/09/08 v1.3 Alias counters (HO)]%
\documentclass{ltxdoc}
\usepackage{holtxdoc}[2011/11/22]
\begin{document}
  \DocInput{aliascnt.dtx}%
\end{document}
%</driver>
% \fi
%
% \CheckSum{78}
%
% \CharacterTable
%  {Upper-case    \A\B\C\D\E\F\G\H\I\J\K\L\M\N\O\P\Q\R\S\T\U\V\W\X\Y\Z
%   Lower-case    \a\b\c\d\e\f\g\h\i\j\k\l\m\n\o\p\q\r\s\t\u\v\w\x\y\z
%   Digits        \0\1\2\3\4\5\6\7\8\9
%   Exclamation   \!     Double quote  \"     Hash (number) \#
%   Dollar        \$     Percent       \%     Ampersand     \&
%   Acute accent  \'     Left paren    \(     Right paren   \)
%   Asterisk      \*     Plus          \+     Comma         \,
%   Minus         \-     Point         \.     Solidus       \/
%   Colon         \:     Semicolon     \;     Less than     \<
%   Equals        \=     Greater than  \>     Question mark \?
%   Commercial at \@     Left bracket  \[     Backslash     \\
%   Right bracket \]     Circumflex    \^     Underscore    \_
%   Grave accent  \`     Left brace    \{     Vertical bar  \|
%   Right brace   \}     Tilde         \~}
%
% \GetFileInfo{aliascnt.drv}
%
% \title{The \xpackage{aliascnt} package}
% \date{2009/09/08 v1.3}
% \author{Heiko Oberdiek\\\xemail{heiko.oberdiek at googlemail.com}}
%
% \maketitle
%
% \begin{abstract}
% Package \xpackage{aliascnt} introduces \emph{alias counters} that
% share the same counter register and clear list.
% \end{abstract}
%
% \tableofcontents
%
% \section{User interface}
%
% \subsection{Introduction}
%
% There are features that rely on the name of counters. For
% example, \xpackage{hyperref}'s \cs{autoref} indirectly uses
% the counter name to determine which label text it puts in front
% of the reference number (\cite{hyperref}).
% In some circumstances this fail: several theorem environments
% are defined by \cs{newtheorem} that share the same counter.
%
% \subsection{Syntax}
%
% Macro names in user land contain the package name
% \texttt{aliascnt} in order to prevent name clashes.
%
% \newenvironment{desc}{^^A
%   \list{}{^^A
%     \setlength{\labelwidth}{0pt}^^A
%     \setlength{\itemindent}{-.5\marginparwidth}^^A
%     \setlength{\leftmargin}{0pt}^^A
%     \let\makelabel\desclabel
%   }^^A
% }{^^A
%   \endlist
% }
% \newcommand*{\desclabel}[1]{^^A
%   \hspace{\labelsep}^^A
%   \normalfont
%   #1^^A
% }
% \newcommand*{\itemcs}[2]{^^A
%   \item[^^A
%      \expandafter\SpecialUsageIndex\csname #1\endcsname
%      {\cs{#1}#2}^^A
%   ]\mbox{}\\*[.5ex]^^A
%   \ignorespaces
% }
% \begin{desc}
% \itemcs{newaliascnt}{\marg{ALIASCNT}\marg{BASECNT}}
%    An alias counter ALIASCNT is created that does not allocate
%    a new \TeX\ counter register. It shares the count register and
%    the clear list with counter BASECNT. If the value of either
%    the two registers is changed, the changes affects both.
% \itemcs{aliascntresetthe}{\marg{ALIASCNT}}
%    This fixes a problem with \cs{newtheorem} if it
%    is fooled by an alias counter with the same name:
%    \begin{quote}
%\begin{verbatim}
%\newtheorem{foo}{Foo}% counter "foo"
%\newaliascnt{bar}{foo}% alias counter "bar"
%\newtheorem{bar}[bar]{Bar}
%\aliascntresetthe{bar}
%\end{verbatim}
%    \end{quote}
% \end{desc}
%
% \StopEventually{
% }
%
% \section{Implementation}
%
% \subsection{Identification}
%
%    \begin{macrocode}
%<*package>
\NeedsTeXFormat{LaTeX2e}
\ProvidesPackage{aliascnt}%
  [2009/09/08 v1.3 Alias counters (HO)]%
%    \end{macrocode}
%
% \subsection{Create new alias counter}
%
%    \begin{macro}{\newaliascnt}
%    A new alias counter is set up by \cs{newaliascnt}.
%    The following properties are added for the new counter CNT:
%    \begin{description}
%    \item[\mdseries\cs{theH}\meta{CNT}:] Compatibility for \xpackage{hyperref}
%    \item[\mdseries\cs{AC@cnt@}\meta{CNT}:] Name of the referenced counter
%      in the definition.
%    \end{description}
%    \begin{macrocode}
\newcommand*{\newaliascnt}[2]{%
  \begingroup
    \def\AC@glet##1{%
      \global\expandafter\let\csname##1#1\expandafter\endcsname
        \csname##1#2\endcsname
    }%
    \@ifundefined{c@#2}{%
      \@nocounterr{#2}%
    }{%
      \expandafter\@ifdefinable\csname c@#1\endcsname{%
        \AC@glet{c@}%
        \AC@glet{the}%
        \AC@glet{theH}%
        \AC@glet{p@}%
        \expandafter\gdef\csname AC@cnt@#1\endcsname{#2}%
        \expandafter\gdef\csname cl@#1\expandafter\endcsname
        \expandafter{\csname cl@#2\endcsname}%
      }%
    }%
  \endgroup
}
%    \end{macrocode}
%    \end{macro}
%
%    \begin{macro}{\aliascntresetthe}
%    The \cs{the}\meta{CNT} macro is restored using the
%    main counter.
%    \begin{macrocode}
\newcommand*{\aliascntresetthe}[1]{%
  \@ifundefined{AC@cnt@#1}{%
    \PackageError{aliascnt}{%
      `#1' is not an alias counter%
    }\@ehc
  }{%
    \expandafter\let\csname the#1\expandafter\endcsname
      \csname the\csname AC@cnt@#1\endcsname\endcsname
  }%
}
%    \end{macrocode}
%    \end{macro}
%
% \subsection{Counter clear list}
%
%    The alias counters share the same register and clear list.
%    Therefore we must ensure that manipulations to the clear list
%    are done with the clear list macro of a real counter.
%    \begin{macro}{\AC@findrootcnt}
%    \cs{AC@findrootcnt} walks throught the aliasing relations
%    to find the base counter.
%    \begin{macrocode}
\newcommand*{\AC@findrootcnt}[1]{%
  \@ifundefined{AC@cnt@#1}{%
    #1%
  }{%
    \expandafter\AC@findrootcnt\csname AC@cnt@#1\endcsname
  }%
}
%    \end{macrocode}
%    \end{macro}
%
%    Clear lists are manipulated by \cs{@addtoreset} and
%    \cs{@removefromreset}. The latter one is provided by
%    the \xpackage{remreset} package (\cite{remreset}).
%
%    \begin{macro}{\AC@patch}
%    The same patch principle is applicable to both
%    \cs{@addtoreset} and \cs{@removefromreset}.
%    \begin{macrocode}
\def\AC@patch#1{%
  \expandafter\let\csname AC@org@#1reset\expandafter\endcsname
    \csname @#1reset\endcsname
  \expandafter\def\csname @#1reset\endcsname##1##2{%
    \csname AC@org@#1reset\endcsname{##1}{\AC@findrootcnt{##2}}%
  }%
}
%    \end{macrocode}
%    \end{macro}
%    If \xpackage{remreset} is not loaded we cannot delay
%    the patch to \cs{AtBeginDocumen}, because \cs{@removefromreset}
%    can be called in between. Therefore we force the loading of
%    the package.
%    \begin{macrocode}
\RequirePackage{remreset}
\AC@patch{addto}
\AC@patch{removefrom}
%    \end{macrocode}
%
%    \begin{macrocode}
%</package>
%    \end{macrocode}
%
% \section{Installation}
%
% \subsection{Download}
%
% \paragraph{Package.} This package is available on
% CTAN\footnote{\url{ftp://ftp.ctan.org/tex-archive/}}:
% \begin{description}
% \item[\CTAN{macros/latex/contrib/oberdiek/aliascnt.dtx}] The source file.
% \item[\CTAN{macros/latex/contrib/oberdiek/aliascnt.pdf}] Documentation.
% \end{description}
%
%
% \paragraph{Bundle.} All the packages of the bundle `oberdiek'
% are also available in a TDS compliant ZIP archive. There
% the packages are already unpacked and the documentation files
% are generated. The files and directories obey the TDS standard.
% \begin{description}
% \item[\CTAN{install/macros/latex/contrib/oberdiek.tds.zip}]
% \end{description}
% \emph{TDS} refers to the standard ``A Directory Structure
% for \TeX\ Files'' (\CTAN{tds/tds.pdf}). Directories
% with \xfile{texmf} in their name are usually organized this way.
%
% \subsection{Bundle installation}
%
% \paragraph{Unpacking.} Unpack the \xfile{oberdiek.tds.zip} in the
% TDS tree (also known as \xfile{texmf} tree) of your choice.
% Example (linux):
% \begin{quote}
%   |unzip oberdiek.tds.zip -d ~/texmf|
% \end{quote}
%
% \paragraph{Script installation.}
% Check the directory \xfile{TDS:scripts/oberdiek/} for
% scripts that need further installation steps.
% Package \xpackage{attachfile2} comes with the Perl script
% \xfile{pdfatfi.pl} that should be installed in such a way
% that it can be called as \texttt{pdfatfi}.
% Example (linux):
% \begin{quote}
%   |chmod +x scripts/oberdiek/pdfatfi.pl|\\
%   |cp scripts/oberdiek/pdfatfi.pl /usr/local/bin/|
% \end{quote}
%
% \subsection{Package installation}
%
% \paragraph{Unpacking.} The \xfile{.dtx} file is a self-extracting
% \docstrip\ archive. The files are extracted by running the
% \xfile{.dtx} through \plainTeX:
% \begin{quote}
%   \verb|tex aliascnt.dtx|
% \end{quote}
%
% \paragraph{TDS.} Now the different files must be moved into
% the different directories in your installation TDS tree
% (also known as \xfile{texmf} tree):
% \begin{quote}
% \def\t{^^A
% \begin{tabular}{@{}>{\ttfamily}l@{ $\rightarrow$ }>{\ttfamily}l@{}}
%   aliascnt.sty & tex/latex/oberdiek/aliascnt.sty\\
%   aliascnt.pdf & doc/latex/oberdiek/aliascnt.pdf\\
%   aliascnt.dtx & source/latex/oberdiek/aliascnt.dtx\\
% \end{tabular}^^A
% }^^A
% \sbox0{\t}^^A
% \ifdim\wd0>\linewidth
%   \begingroup
%     \advance\linewidth by\leftmargin
%     \advance\linewidth by\rightmargin
%   \edef\x{\endgroup
%     \def\noexpand\lw{\the\linewidth}^^A
%   }\x
%   \def\lwbox{^^A
%     \leavevmode
%     \hbox to \linewidth{^^A
%       \kern-\leftmargin\relax
%       \hss
%       \usebox0
%       \hss
%       \kern-\rightmargin\relax
%     }^^A
%   }^^A
%   \ifdim\wd0>\lw
%     \sbox0{\small\t}^^A
%     \ifdim\wd0>\linewidth
%       \ifdim\wd0>\lw
%         \sbox0{\footnotesize\t}^^A
%         \ifdim\wd0>\linewidth
%           \ifdim\wd0>\lw
%             \sbox0{\scriptsize\t}^^A
%             \ifdim\wd0>\linewidth
%               \ifdim\wd0>\lw
%                 \sbox0{\tiny\t}^^A
%                 \ifdim\wd0>\linewidth
%                   \lwbox
%                 \else
%                   \usebox0
%                 \fi
%               \else
%                 \lwbox
%               \fi
%             \else
%               \usebox0
%             \fi
%           \else
%             \lwbox
%           \fi
%         \else
%           \usebox0
%         \fi
%       \else
%         \lwbox
%       \fi
%     \else
%       \usebox0
%     \fi
%   \else
%     \lwbox
%   \fi
% \else
%   \usebox0
% \fi
% \end{quote}
% If you have a \xfile{docstrip.cfg} that configures and enables \docstrip's
% TDS installing feature, then some files can already be in the right
% place, see the documentation of \docstrip.
%
% \subsection{Refresh file name databases}
%
% If your \TeX~distribution
% (\teTeX, \mikTeX, \dots) relies on file name databases, you must refresh
% these. For example, \teTeX\ users run \verb|texhash| or
% \verb|mktexlsr|.
%
% \subsection{Some details for the interested}
%
% \paragraph{Attached source.}
%
% The PDF documentation on CTAN also includes the
% \xfile{.dtx} source file. It can be extracted by
% AcrobatReader 6 or higher. Another option is \textsf{pdftk},
% e.g. unpack the file into the current directory:
% \begin{quote}
%   \verb|pdftk aliascnt.pdf unpack_files output .|
% \end{quote}
%
% \paragraph{Unpacking with \LaTeX.}
% The \xfile{.dtx} chooses its action depending on the format:
% \begin{description}
% \item[\plainTeX:] Run \docstrip\ and extract the files.
% \item[\LaTeX:] Generate the documentation.
% \end{description}
% If you insist on using \LaTeX\ for \docstrip\ (really,
% \docstrip\ does not need \LaTeX), then inform the autodetect routine
% about your intention:
% \begin{quote}
%   \verb|latex \let\install=y\input{aliascnt.dtx}|
% \end{quote}
% Do not forget to quote the argument according to the demands
% of your shell.
%
% \paragraph{Generating the documentation.}
% You can use both the \xfile{.dtx} or the \xfile{.drv} to generate
% the documentation. The process can be configured by the
% configuration file \xfile{ltxdoc.cfg}. For instance, put this
% line into this file, if you want to have A4 as paper format:
% \begin{quote}
%   \verb|\PassOptionsToClass{a4paper}{article}|
% \end{quote}
% An example follows how to generate the
% documentation with pdf\LaTeX:
% \begin{quote}
%\begin{verbatim}
%pdflatex aliascnt.dtx
%makeindex -s gind.ist aliascnt.idx
%pdflatex aliascnt.dtx
%makeindex -s gind.ist aliascnt.idx
%pdflatex aliascnt.dtx
%\end{verbatim}
% \end{quote}
%
% \section{Catalogue}
%
% The following XML file can be used as source for the
% \href{http://mirror.ctan.org/help/Catalogue/catalogue.html}{\TeX\ Catalogue}.
% The elements \texttt{caption} and \texttt{description} are imported
% from the original XML file from the Catalogue.
% The name of the XML file in the Catalogue is \xfile{aliascnt.xml}.
%    \begin{macrocode}
%<*catalogue>
<?xml version='1.0' encoding='us-ascii'?>
<!DOCTYPE entry SYSTEM 'catalogue.dtd'>
<entry datestamp='$Date$' modifier='$Author$' id='aliascnt'>
  <name>aliascnt</name>
  <caption>Alias counters.</caption>
  <authorref id='auth:oberdiek'/>
  <copyright owner='Heiko Oberdiek' year='2006,2009'/>
  <license type='lppl1.3'/>
  <version number='1.3'/>
  <description>
    This package introduces aliases for counters, that
    share the same counter register and clear list.
    <p/>
    The package is part of the <xref refid='oberdiek'>oberdiek</xref>
    bundle.
  </description>
  <documentation details='Package documentation'
      href='ctan:/macros/latex/contrib/oberdiek/aliascnt.pdf'/>
  <ctan file='true' path='/macros/latex/contrib/oberdiek/aliascnt.dtx'/>
  <miktex location='oberdiek'/>
  <texlive location='oberdiek'/>
  <install path='/macros/latex/contrib/oberdiek/oberdiek.tds.zip'/>
</entry>
%</catalogue>
%    \end{macrocode}
%
% \section{Acknowledgement}
%
% \begin{description}
% \item[Ulrich Schwarz:] The package is based on his draft for
%   ``Die \TeX nische Kom\"odie'', see \cite{schwarz}.
% \end{description}
%
% \begin{thebibliography}{9}
%
% \bibitem{schwarz}
%   Ulrich Schwarz:
%   \textit{Was hinten herauskommt z\"ahlt: Counter Aliasing in \LaTeX},
%   \textit{Die \TeX nische Kom\"odie}, 3/2006, pages 8--14, Juli 2006.
%
% \bibitem{remreset}
%   David Carlisle: \textit{The \xpackage{remreset} package};
%   1997/09/28;
%   \CTAN{macros/latex/contrib/carlisle/remreset.sty}.
%
% \bibitem{hyperref}
%   Sebastian Rahtz, Heiko Oberdiek:
%   \textit{The \xpackage{hyperref} package};
%   2006/08/16 v6.75c;
%   \CTAN{macros/latex/contrib/hyperref/}.
%
% \end{thebibliography}
%
% \begin{History}
%   \begin{Version}{2006/02/20 v1.0}
%   \item
%     First version.
%   \end{Version}
%   \begin{Version}{2006/08/16 v1.1}
%   \item
%     Update of bibliography.
%   \end{Version}
%   \begin{Version}{2006/09/25 v1.2}
%   \item
%     Bug fix (\cs{aliascntresetthe}).
%   \end{Version}
%   \begin{Version}{2009/09/08 v1.3}
%   \item
%     Bug fix of \cs{@ifdefinable}'s use (thanks to Uwe L\"uck).
%   \end{Version}
% \end{History}
%
% \PrintIndex
%
% \Finale
\endinput
|
% \end{quote}
% Do not forget to quote the argument according to the demands
% of your shell.
%
% \paragraph{Generating the documentation.}
% You can use both the \xfile{.dtx} or the \xfile{.drv} to generate
% the documentation. The process can be configured by the
% configuration file \xfile{ltxdoc.cfg}. For instance, put this
% line into this file, if you want to have A4 as paper format:
% \begin{quote}
%   \verb|\PassOptionsToClass{a4paper}{article}|
% \end{quote}
% An example follows how to generate the
% documentation with pdf\LaTeX:
% \begin{quote}
%\begin{verbatim}
%pdflatex aliascnt.dtx
%makeindex -s gind.ist aliascnt.idx
%pdflatex aliascnt.dtx
%makeindex -s gind.ist aliascnt.idx
%pdflatex aliascnt.dtx
%\end{verbatim}
% \end{quote}
%
% \section{Catalogue}
%
% The following XML file can be used as source for the
% \href{http://mirror.ctan.org/help/Catalogue/catalogue.html}{\TeX\ Catalogue}.
% The elements \texttt{caption} and \texttt{description} are imported
% from the original XML file from the Catalogue.
% The name of the XML file in the Catalogue is \xfile{aliascnt.xml}.
%    \begin{macrocode}
%<*catalogue>
<?xml version='1.0' encoding='us-ascii'?>
<!DOCTYPE entry SYSTEM 'catalogue.dtd'>
<entry datestamp='$Date$' modifier='$Author$' id='aliascnt'>
  <name>aliascnt</name>
  <caption>Alias counters.</caption>
  <authorref id='auth:oberdiek'/>
  <copyright owner='Heiko Oberdiek' year='2006,2009'/>
  <license type='lppl1.3'/>
  <version number='1.3'/>
  <description>
    This package introduces aliases for counters, that
    share the same counter register and clear list.
    <p/>
    The package is part of the <xref refid='oberdiek'>oberdiek</xref>
    bundle.
  </description>
  <documentation details='Package documentation'
      href='ctan:/macros/latex/contrib/oberdiek/aliascnt.pdf'/>
  <ctan file='true' path='/macros/latex/contrib/oberdiek/aliascnt.dtx'/>
  <miktex location='oberdiek'/>
  <texlive location='oberdiek'/>
  <install path='/macros/latex/contrib/oberdiek/oberdiek.tds.zip'/>
</entry>
%</catalogue>
%    \end{macrocode}
%
% \section{Acknowledgement}
%
% \begin{description}
% \item[Ulrich Schwarz:] The package is based on his draft for
%   ``Die \TeX nische Kom\"odie'', see \cite{schwarz}.
% \end{description}
%
% \begin{thebibliography}{9}
%
% \bibitem{schwarz}
%   Ulrich Schwarz:
%   \textit{Was hinten herauskommt z\"ahlt: Counter Aliasing in \LaTeX},
%   \textit{Die \TeX nische Kom\"odie}, 3/2006, pages 8--14, Juli 2006.
%
% \bibitem{remreset}
%   David Carlisle: \textit{The \xpackage{remreset} package};
%   1997/09/28;
%   \CTAN{macros/latex/contrib/carlisle/remreset.sty}.
%
% \bibitem{hyperref}
%   Sebastian Rahtz, Heiko Oberdiek:
%   \textit{The \xpackage{hyperref} package};
%   2006/08/16 v6.75c;
%   \CTAN{macros/latex/contrib/hyperref/}.
%
% \end{thebibliography}
%
% \begin{History}
%   \begin{Version}{2006/02/20 v1.0}
%   \item
%     First version.
%   \end{Version}
%   \begin{Version}{2006/08/16 v1.1}
%   \item
%     Update of bibliography.
%   \end{Version}
%   \begin{Version}{2006/09/25 v1.2}
%   \item
%     Bug fix (\cs{aliascntresetthe}).
%   \end{Version}
%   \begin{Version}{2009/09/08 v1.3}
%   \item
%     Bug fix of \cs{@ifdefinable}'s use (thanks to Uwe L\"uck).
%   \end{Version}
% \end{History}
%
% \PrintIndex
%
% \Finale
\endinput

%        (quote the arguments according to the demands of your shell)
%
% Documentation:
%    (a) If aliascnt.drv is present:
%           latex aliascnt.drv
%    (b) Without aliascnt.drv:
%           latex aliascnt.dtx; ...
%    The class ltxdoc loads the configuration file ltxdoc.cfg
%    if available. Here you can specify further options, e.g.
%    use A4 as paper format:
%       \PassOptionsToClass{a4paper}{article}
%
%    Programm calls to get the documentation (example):
%       pdflatex aliascnt.dtx
%       makeindex -s gind.ist aliascnt.idx
%       pdflatex aliascnt.dtx
%       makeindex -s gind.ist aliascnt.idx
%       pdflatex aliascnt.dtx
%
% Installation:
%    TDS:tex/latex/oberdiek/aliascnt.sty
%    TDS:doc/latex/oberdiek/aliascnt.pdf
%    TDS:source/latex/oberdiek/aliascnt.dtx
%
%<*ignore>
\begingroup
  \catcode123=1 %
  \catcode125=2 %
  \def\x{LaTeX2e}%
\expandafter\endgroup
\ifcase 0\ifx\install y1\fi\expandafter
         \ifx\csname processbatchFile\endcsname\relax\else1\fi
         \ifx\fmtname\x\else 1\fi\relax
\else\csname fi\endcsname
%</ignore>
%<*install>
\input docstrip.tex
\Msg{************************************************************************}
\Msg{* Installation}
\Msg{* Package: aliascnt 2009/09/08 v1.3 Alias counters (HO)}
\Msg{************************************************************************}

\keepsilent
\askforoverwritefalse

\let\MetaPrefix\relax
\preamble

This is a generated file.

Project: aliascnt
Version: 2009/09/08 v1.3

Copyright (C) 2006, 2009 by
   Heiko Oberdiek <heiko.oberdiek at googlemail.com>

This work may be distributed and/or modified under the
conditions of the LaTeX Project Public License, either
version 1.3c of this license or (at your option) any later
version. This version of this license is in
   http://www.latex-project.org/lppl/lppl-1-3c.txt
and the latest version of this license is in
   http://www.latex-project.org/lppl.txt
and version 1.3 or later is part of all distributions of
LaTeX version 2005/12/01 or later.

This work has the LPPL maintenance status "maintained".

This Current Maintainer of this work is Heiko Oberdiek.

This work consists of the main source file aliascnt.dtx
and the derived files
   aliascnt.sty, aliascnt.pdf, aliascnt.ins, aliascnt.drv.

\endpreamble
\let\MetaPrefix\DoubleperCent

\generate{%
  \file{aliascnt.ins}{\from{aliascnt.dtx}{install}}%
  \file{aliascnt.drv}{\from{aliascnt.dtx}{driver}}%
  \usedir{tex/latex/oberdiek}%
  \file{aliascnt.sty}{\from{aliascnt.dtx}{package}}%
  \nopreamble
  \nopostamble
  \usedir{source/latex/oberdiek/catalogue}%
  \file{aliascnt.xml}{\from{aliascnt.dtx}{catalogue}}%
}

\catcode32=13\relax% active space
\let =\space%
\Msg{************************************************************************}
\Msg{*}
\Msg{* To finish the installation you have to move the following}
\Msg{* file into a directory searched by TeX:}
\Msg{*}
\Msg{*     aliascnt.sty}
\Msg{*}
\Msg{* To produce the documentation run the file `aliascnt.drv'}
\Msg{* through LaTeX.}
\Msg{*}
\Msg{* Happy TeXing!}
\Msg{*}
\Msg{************************************************************************}

\endbatchfile
%</install>
%<*ignore>
\fi
%</ignore>
%<*driver>
\NeedsTeXFormat{LaTeX2e}
\ProvidesFile{aliascnt.drv}%
  [2009/09/08 v1.3 Alias counters (HO)]%
\documentclass{ltxdoc}
\usepackage{holtxdoc}[2011/11/22]
\begin{document}
  \DocInput{aliascnt.dtx}%
\end{document}
%</driver>
% \fi
%
% \CheckSum{78}
%
% \CharacterTable
%  {Upper-case    \A\B\C\D\E\F\G\H\I\J\K\L\M\N\O\P\Q\R\S\T\U\V\W\X\Y\Z
%   Lower-case    \a\b\c\d\e\f\g\h\i\j\k\l\m\n\o\p\q\r\s\t\u\v\w\x\y\z
%   Digits        \0\1\2\3\4\5\6\7\8\9
%   Exclamation   \!     Double quote  \"     Hash (number) \#
%   Dollar        \$     Percent       \%     Ampersand     \&
%   Acute accent  \'     Left paren    \(     Right paren   \)
%   Asterisk      \*     Plus          \+     Comma         \,
%   Minus         \-     Point         \.     Solidus       \/
%   Colon         \:     Semicolon     \;     Less than     \<
%   Equals        \=     Greater than  \>     Question mark \?
%   Commercial at \@     Left bracket  \[     Backslash     \\
%   Right bracket \]     Circumflex    \^     Underscore    \_
%   Grave accent  \`     Left brace    \{     Vertical bar  \|
%   Right brace   \}     Tilde         \~}
%
% \GetFileInfo{aliascnt.drv}
%
% \title{The \xpackage{aliascnt} package}
% \date{2009/09/08 v1.3}
% \author{Heiko Oberdiek\\\xemail{heiko.oberdiek at googlemail.com}}
%
% \maketitle
%
% \begin{abstract}
% Package \xpackage{aliascnt} introduces \emph{alias counters} that
% share the same counter register and clear list.
% \end{abstract}
%
% \tableofcontents
%
% \section{User interface}
%
% \subsection{Introduction}
%
% There are features that rely on the name of counters. For
% example, \xpackage{hyperref}'s \cs{autoref} indirectly uses
% the counter name to determine which label text it puts in front
% of the reference number (\cite{hyperref}).
% In some circumstances this fail: several theorem environments
% are defined by \cs{newtheorem} that share the same counter.
%
% \subsection{Syntax}
%
% Macro names in user land contain the package name
% \texttt{aliascnt} in order to prevent name clashes.
%
% \newenvironment{desc}{^^A
%   \list{}{^^A
%     \setlength{\labelwidth}{0pt}^^A
%     \setlength{\itemindent}{-.5\marginparwidth}^^A
%     \setlength{\leftmargin}{0pt}^^A
%     \let\makelabel\desclabel
%   }^^A
% }{^^A
%   \endlist
% }
% \newcommand*{\desclabel}[1]{^^A
%   \hspace{\labelsep}^^A
%   \normalfont
%   #1^^A
% }
% \newcommand*{\itemcs}[2]{^^A
%   \item[^^A
%      \expandafter\SpecialUsageIndex\csname #1\endcsname
%      {\cs{#1}#2}^^A
%   ]\mbox{}\\*[.5ex]^^A
%   \ignorespaces
% }
% \begin{desc}
% \itemcs{newaliascnt}{\marg{ALIASCNT}\marg{BASECNT}}
%    An alias counter ALIASCNT is created that does not allocate
%    a new \TeX\ counter register. It shares the count register and
%    the clear list with counter BASECNT. If the value of either
%    the two registers is changed, the changes affects both.
% \itemcs{aliascntresetthe}{\marg{ALIASCNT}}
%    This fixes a problem with \cs{newtheorem} if it
%    is fooled by an alias counter with the same name:
%    \begin{quote}
%\begin{verbatim}
%\newtheorem{foo}{Foo}% counter "foo"
%\newaliascnt{bar}{foo}% alias counter "bar"
%\newtheorem{bar}[bar]{Bar}
%\aliascntresetthe{bar}
%\end{verbatim}
%    \end{quote}
% \end{desc}
%
% \StopEventually{
% }
%
% \section{Implementation}
%
% \subsection{Identification}
%
%    \begin{macrocode}
%<*package>
\NeedsTeXFormat{LaTeX2e}
\ProvidesPackage{aliascnt}%
  [2009/09/08 v1.3 Alias counters (HO)]%
%    \end{macrocode}
%
% \subsection{Create new alias counter}
%
%    \begin{macro}{\newaliascnt}
%    A new alias counter is set up by \cs{newaliascnt}.
%    The following properties are added for the new counter CNT:
%    \begin{description}
%    \item[\mdseries\cs{theH}\meta{CNT}:] Compatibility for \xpackage{hyperref}
%    \item[\mdseries\cs{AC@cnt@}\meta{CNT}:] Name of the referenced counter
%      in the definition.
%    \end{description}
%    \begin{macrocode}
\newcommand*{\newaliascnt}[2]{%
  \begingroup
    \def\AC@glet##1{%
      \global\expandafter\let\csname##1#1\expandafter\endcsname
        \csname##1#2\endcsname
    }%
    \@ifundefined{c@#2}{%
      \@nocounterr{#2}%
    }{%
      \expandafter\@ifdefinable\csname c@#1\endcsname{%
        \AC@glet{c@}%
        \AC@glet{the}%
        \AC@glet{theH}%
        \AC@glet{p@}%
        \expandafter\gdef\csname AC@cnt@#1\endcsname{#2}%
        \expandafter\gdef\csname cl@#1\expandafter\endcsname
        \expandafter{\csname cl@#2\endcsname}%
      }%
    }%
  \endgroup
}
%    \end{macrocode}
%    \end{macro}
%
%    \begin{macro}{\aliascntresetthe}
%    The \cs{the}\meta{CNT} macro is restored using the
%    main counter.
%    \begin{macrocode}
\newcommand*{\aliascntresetthe}[1]{%
  \@ifundefined{AC@cnt@#1}{%
    \PackageError{aliascnt}{%
      `#1' is not an alias counter%
    }\@ehc
  }{%
    \expandafter\let\csname the#1\expandafter\endcsname
      \csname the\csname AC@cnt@#1\endcsname\endcsname
  }%
}
%    \end{macrocode}
%    \end{macro}
%
% \subsection{Counter clear list}
%
%    The alias counters share the same register and clear list.
%    Therefore we must ensure that manipulations to the clear list
%    are done with the clear list macro of a real counter.
%    \begin{macro}{\AC@findrootcnt}
%    \cs{AC@findrootcnt} walks throught the aliasing relations
%    to find the base counter.
%    \begin{macrocode}
\newcommand*{\AC@findrootcnt}[1]{%
  \@ifundefined{AC@cnt@#1}{%
    #1%
  }{%
    \expandafter\AC@findrootcnt\csname AC@cnt@#1\endcsname
  }%
}
%    \end{macrocode}
%    \end{macro}
%
%    Clear lists are manipulated by \cs{@addtoreset} and
%    \cs{@removefromreset}. The latter one is provided by
%    the \xpackage{remreset} package (\cite{remreset}).
%
%    \begin{macro}{\AC@patch}
%    The same patch principle is applicable to both
%    \cs{@addtoreset} and \cs{@removefromreset}.
%    \begin{macrocode}
\def\AC@patch#1{%
  \expandafter\let\csname AC@org@#1reset\expandafter\endcsname
    \csname @#1reset\endcsname
  \expandafter\def\csname @#1reset\endcsname##1##2{%
    \csname AC@org@#1reset\endcsname{##1}{\AC@findrootcnt{##2}}%
  }%
}
%    \end{macrocode}
%    \end{macro}
%    If \xpackage{remreset} is not loaded we cannot delay
%    the patch to \cs{AtBeginDocumen}, because \cs{@removefromreset}
%    can be called in between. Therefore we force the loading of
%    the package.
%    \begin{macrocode}
\RequirePackage{remreset}
\AC@patch{addto}
\AC@patch{removefrom}
%    \end{macrocode}
%
%    \begin{macrocode}
%</package>
%    \end{macrocode}
%
% \section{Installation}
%
% \subsection{Download}
%
% \paragraph{Package.} This package is available on
% CTAN\footnote{\url{ftp://ftp.ctan.org/tex-archive/}}:
% \begin{description}
% \item[\CTAN{macros/latex/contrib/oberdiek/aliascnt.dtx}] The source file.
% \item[\CTAN{macros/latex/contrib/oberdiek/aliascnt.pdf}] Documentation.
% \end{description}
%
%
% \paragraph{Bundle.} All the packages of the bundle `oberdiek'
% are also available in a TDS compliant ZIP archive. There
% the packages are already unpacked and the documentation files
% are generated. The files and directories obey the TDS standard.
% \begin{description}
% \item[\CTAN{install/macros/latex/contrib/oberdiek.tds.zip}]
% \end{description}
% \emph{TDS} refers to the standard ``A Directory Structure
% for \TeX\ Files'' (\CTAN{tds/tds.pdf}). Directories
% with \xfile{texmf} in their name are usually organized this way.
%
% \subsection{Bundle installation}
%
% \paragraph{Unpacking.} Unpack the \xfile{oberdiek.tds.zip} in the
% TDS tree (also known as \xfile{texmf} tree) of your choice.
% Example (linux):
% \begin{quote}
%   |unzip oberdiek.tds.zip -d ~/texmf|
% \end{quote}
%
% \paragraph{Script installation.}
% Check the directory \xfile{TDS:scripts/oberdiek/} for
% scripts that need further installation steps.
% Package \xpackage{attachfile2} comes with the Perl script
% \xfile{pdfatfi.pl} that should be installed in such a way
% that it can be called as \texttt{pdfatfi}.
% Example (linux):
% \begin{quote}
%   |chmod +x scripts/oberdiek/pdfatfi.pl|\\
%   |cp scripts/oberdiek/pdfatfi.pl /usr/local/bin/|
% \end{quote}
%
% \subsection{Package installation}
%
% \paragraph{Unpacking.} The \xfile{.dtx} file is a self-extracting
% \docstrip\ archive. The files are extracted by running the
% \xfile{.dtx} through \plainTeX:
% \begin{quote}
%   \verb|tex aliascnt.dtx|
% \end{quote}
%
% \paragraph{TDS.} Now the different files must be moved into
% the different directories in your installation TDS tree
% (also known as \xfile{texmf} tree):
% \begin{quote}
% \def\t{^^A
% \begin{tabular}{@{}>{\ttfamily}l@{ $\rightarrow$ }>{\ttfamily}l@{}}
%   aliascnt.sty & tex/latex/oberdiek/aliascnt.sty\\
%   aliascnt.pdf & doc/latex/oberdiek/aliascnt.pdf\\
%   aliascnt.dtx & source/latex/oberdiek/aliascnt.dtx\\
% \end{tabular}^^A
% }^^A
% \sbox0{\t}^^A
% \ifdim\wd0>\linewidth
%   \begingroup
%     \advance\linewidth by\leftmargin
%     \advance\linewidth by\rightmargin
%   \edef\x{\endgroup
%     \def\noexpand\lw{\the\linewidth}^^A
%   }\x
%   \def\lwbox{^^A
%     \leavevmode
%     \hbox to \linewidth{^^A
%       \kern-\leftmargin\relax
%       \hss
%       \usebox0
%       \hss
%       \kern-\rightmargin\relax
%     }^^A
%   }^^A
%   \ifdim\wd0>\lw
%     \sbox0{\small\t}^^A
%     \ifdim\wd0>\linewidth
%       \ifdim\wd0>\lw
%         \sbox0{\footnotesize\t}^^A
%         \ifdim\wd0>\linewidth
%           \ifdim\wd0>\lw
%             \sbox0{\scriptsize\t}^^A
%             \ifdim\wd0>\linewidth
%               \ifdim\wd0>\lw
%                 \sbox0{\tiny\t}^^A
%                 \ifdim\wd0>\linewidth
%                   \lwbox
%                 \else
%                   \usebox0
%                 \fi
%               \else
%                 \lwbox
%               \fi
%             \else
%               \usebox0
%             \fi
%           \else
%             \lwbox
%           \fi
%         \else
%           \usebox0
%         \fi
%       \else
%         \lwbox
%       \fi
%     \else
%       \usebox0
%     \fi
%   \else
%     \lwbox
%   \fi
% \else
%   \usebox0
% \fi
% \end{quote}
% If you have a \xfile{docstrip.cfg} that configures and enables \docstrip's
% TDS installing feature, then some files can already be in the right
% place, see the documentation of \docstrip.
%
% \subsection{Refresh file name databases}
%
% If your \TeX~distribution
% (\teTeX, \mikTeX, \dots) relies on file name databases, you must refresh
% these. For example, \teTeX\ users run \verb|texhash| or
% \verb|mktexlsr|.
%
% \subsection{Some details for the interested}
%
% \paragraph{Attached source.}
%
% The PDF documentation on CTAN also includes the
% \xfile{.dtx} source file. It can be extracted by
% AcrobatReader 6 or higher. Another option is \textsf{pdftk},
% e.g. unpack the file into the current directory:
% \begin{quote}
%   \verb|pdftk aliascnt.pdf unpack_files output .|
% \end{quote}
%
% \paragraph{Unpacking with \LaTeX.}
% The \xfile{.dtx} chooses its action depending on the format:
% \begin{description}
% \item[\plainTeX:] Run \docstrip\ and extract the files.
% \item[\LaTeX:] Generate the documentation.
% \end{description}
% If you insist on using \LaTeX\ for \docstrip\ (really,
% \docstrip\ does not need \LaTeX), then inform the autodetect routine
% about your intention:
% \begin{quote}
%   \verb|latex \let\install=y% \iffalse meta-comment
%
% File: aliascnt.dtx
% Version: 2009/09/08 v1.3
% Info: Alias counters
%
% Copyright (C) 2006, 2009 by
%    Heiko Oberdiek <heiko.oberdiek at googlemail.com>
%
% This work may be distributed and/or modified under the
% conditions of the LaTeX Project Public License, either
% version 1.3c of this license or (at your option) any later
% version. This version of this license is in
%    http://www.latex-project.org/lppl/lppl-1-3c.txt
% and the latest version of this license is in
%    http://www.latex-project.org/lppl.txt
% and version 1.3 or later is part of all distributions of
% LaTeX version 2005/12/01 or later.
%
% This work has the LPPL maintenance status "maintained".
%
% This Current Maintainer of this work is Heiko Oberdiek.
%
% This work consists of the main source file aliascnt.dtx
% and the derived files
%    aliascnt.sty, aliascnt.pdf, aliascnt.ins, aliascnt.drv.
%
% Distribution:
%    CTAN:macros/latex/contrib/oberdiek/aliascnt.dtx
%    CTAN:macros/latex/contrib/oberdiek/aliascnt.pdf
%
% Unpacking:
%    (a) If aliascnt.ins is present:
%           tex aliascnt.ins
%    (b) Without aliascnt.ins:
%           tex aliascnt.dtx
%    (c) If you insist on using LaTeX
%           latex \let\install=y% \iffalse meta-comment
%
% File: aliascnt.dtx
% Version: 2009/09/08 v1.3
% Info: Alias counters
%
% Copyright (C) 2006, 2009 by
%    Heiko Oberdiek <heiko.oberdiek at googlemail.com>
%
% This work may be distributed and/or modified under the
% conditions of the LaTeX Project Public License, either
% version 1.3c of this license or (at your option) any later
% version. This version of this license is in
%    http://www.latex-project.org/lppl/lppl-1-3c.txt
% and the latest version of this license is in
%    http://www.latex-project.org/lppl.txt
% and version 1.3 or later is part of all distributions of
% LaTeX version 2005/12/01 or later.
%
% This work has the LPPL maintenance status "maintained".
%
% This Current Maintainer of this work is Heiko Oberdiek.
%
% This work consists of the main source file aliascnt.dtx
% and the derived files
%    aliascnt.sty, aliascnt.pdf, aliascnt.ins, aliascnt.drv.
%
% Distribution:
%    CTAN:macros/latex/contrib/oberdiek/aliascnt.dtx
%    CTAN:macros/latex/contrib/oberdiek/aliascnt.pdf
%
% Unpacking:
%    (a) If aliascnt.ins is present:
%           tex aliascnt.ins
%    (b) Without aliascnt.ins:
%           tex aliascnt.dtx
%    (c) If you insist on using LaTeX
%           latex \let\install=y\input{aliascnt.dtx}
%        (quote the arguments according to the demands of your shell)
%
% Documentation:
%    (a) If aliascnt.drv is present:
%           latex aliascnt.drv
%    (b) Without aliascnt.drv:
%           latex aliascnt.dtx; ...
%    The class ltxdoc loads the configuration file ltxdoc.cfg
%    if available. Here you can specify further options, e.g.
%    use A4 as paper format:
%       \PassOptionsToClass{a4paper}{article}
%
%    Programm calls to get the documentation (example):
%       pdflatex aliascnt.dtx
%       makeindex -s gind.ist aliascnt.idx
%       pdflatex aliascnt.dtx
%       makeindex -s gind.ist aliascnt.idx
%       pdflatex aliascnt.dtx
%
% Installation:
%    TDS:tex/latex/oberdiek/aliascnt.sty
%    TDS:doc/latex/oberdiek/aliascnt.pdf
%    TDS:source/latex/oberdiek/aliascnt.dtx
%
%<*ignore>
\begingroup
  \catcode123=1 %
  \catcode125=2 %
  \def\x{LaTeX2e}%
\expandafter\endgroup
\ifcase 0\ifx\install y1\fi\expandafter
         \ifx\csname processbatchFile\endcsname\relax\else1\fi
         \ifx\fmtname\x\else 1\fi\relax
\else\csname fi\endcsname
%</ignore>
%<*install>
\input docstrip.tex
\Msg{************************************************************************}
\Msg{* Installation}
\Msg{* Package: aliascnt 2009/09/08 v1.3 Alias counters (HO)}
\Msg{************************************************************************}

\keepsilent
\askforoverwritefalse

\let\MetaPrefix\relax
\preamble

This is a generated file.

Project: aliascnt
Version: 2009/09/08 v1.3

Copyright (C) 2006, 2009 by
   Heiko Oberdiek <heiko.oberdiek at googlemail.com>

This work may be distributed and/or modified under the
conditions of the LaTeX Project Public License, either
version 1.3c of this license or (at your option) any later
version. This version of this license is in
   http://www.latex-project.org/lppl/lppl-1-3c.txt
and the latest version of this license is in
   http://www.latex-project.org/lppl.txt
and version 1.3 or later is part of all distributions of
LaTeX version 2005/12/01 or later.

This work has the LPPL maintenance status "maintained".

This Current Maintainer of this work is Heiko Oberdiek.

This work consists of the main source file aliascnt.dtx
and the derived files
   aliascnt.sty, aliascnt.pdf, aliascnt.ins, aliascnt.drv.

\endpreamble
\let\MetaPrefix\DoubleperCent

\generate{%
  \file{aliascnt.ins}{\from{aliascnt.dtx}{install}}%
  \file{aliascnt.drv}{\from{aliascnt.dtx}{driver}}%
  \usedir{tex/latex/oberdiek}%
  \file{aliascnt.sty}{\from{aliascnt.dtx}{package}}%
  \nopreamble
  \nopostamble
  \usedir{source/latex/oberdiek/catalogue}%
  \file{aliascnt.xml}{\from{aliascnt.dtx}{catalogue}}%
}

\catcode32=13\relax% active space
\let =\space%
\Msg{************************************************************************}
\Msg{*}
\Msg{* To finish the installation you have to move the following}
\Msg{* file into a directory searched by TeX:}
\Msg{*}
\Msg{*     aliascnt.sty}
\Msg{*}
\Msg{* To produce the documentation run the file `aliascnt.drv'}
\Msg{* through LaTeX.}
\Msg{*}
\Msg{* Happy TeXing!}
\Msg{*}
\Msg{************************************************************************}

\endbatchfile
%</install>
%<*ignore>
\fi
%</ignore>
%<*driver>
\NeedsTeXFormat{LaTeX2e}
\ProvidesFile{aliascnt.drv}%
  [2009/09/08 v1.3 Alias counters (HO)]%
\documentclass{ltxdoc}
\usepackage{holtxdoc}[2011/11/22]
\begin{document}
  \DocInput{aliascnt.dtx}%
\end{document}
%</driver>
% \fi
%
% \CheckSum{78}
%
% \CharacterTable
%  {Upper-case    \A\B\C\D\E\F\G\H\I\J\K\L\M\N\O\P\Q\R\S\T\U\V\W\X\Y\Z
%   Lower-case    \a\b\c\d\e\f\g\h\i\j\k\l\m\n\o\p\q\r\s\t\u\v\w\x\y\z
%   Digits        \0\1\2\3\4\5\6\7\8\9
%   Exclamation   \!     Double quote  \"     Hash (number) \#
%   Dollar        \$     Percent       \%     Ampersand     \&
%   Acute accent  \'     Left paren    \(     Right paren   \)
%   Asterisk      \*     Plus          \+     Comma         \,
%   Minus         \-     Point         \.     Solidus       \/
%   Colon         \:     Semicolon     \;     Less than     \<
%   Equals        \=     Greater than  \>     Question mark \?
%   Commercial at \@     Left bracket  \[     Backslash     \\
%   Right bracket \]     Circumflex    \^     Underscore    \_
%   Grave accent  \`     Left brace    \{     Vertical bar  \|
%   Right brace   \}     Tilde         \~}
%
% \GetFileInfo{aliascnt.drv}
%
% \title{The \xpackage{aliascnt} package}
% \date{2009/09/08 v1.3}
% \author{Heiko Oberdiek\\\xemail{heiko.oberdiek at googlemail.com}}
%
% \maketitle
%
% \begin{abstract}
% Package \xpackage{aliascnt} introduces \emph{alias counters} that
% share the same counter register and clear list.
% \end{abstract}
%
% \tableofcontents
%
% \section{User interface}
%
% \subsection{Introduction}
%
% There are features that rely on the name of counters. For
% example, \xpackage{hyperref}'s \cs{autoref} indirectly uses
% the counter name to determine which label text it puts in front
% of the reference number (\cite{hyperref}).
% In some circumstances this fail: several theorem environments
% are defined by \cs{newtheorem} that share the same counter.
%
% \subsection{Syntax}
%
% Macro names in user land contain the package name
% \texttt{aliascnt} in order to prevent name clashes.
%
% \newenvironment{desc}{^^A
%   \list{}{^^A
%     \setlength{\labelwidth}{0pt}^^A
%     \setlength{\itemindent}{-.5\marginparwidth}^^A
%     \setlength{\leftmargin}{0pt}^^A
%     \let\makelabel\desclabel
%   }^^A
% }{^^A
%   \endlist
% }
% \newcommand*{\desclabel}[1]{^^A
%   \hspace{\labelsep}^^A
%   \normalfont
%   #1^^A
% }
% \newcommand*{\itemcs}[2]{^^A
%   \item[^^A
%      \expandafter\SpecialUsageIndex\csname #1\endcsname
%      {\cs{#1}#2}^^A
%   ]\mbox{}\\*[.5ex]^^A
%   \ignorespaces
% }
% \begin{desc}
% \itemcs{newaliascnt}{\marg{ALIASCNT}\marg{BASECNT}}
%    An alias counter ALIASCNT is created that does not allocate
%    a new \TeX\ counter register. It shares the count register and
%    the clear list with counter BASECNT. If the value of either
%    the two registers is changed, the changes affects both.
% \itemcs{aliascntresetthe}{\marg{ALIASCNT}}
%    This fixes a problem with \cs{newtheorem} if it
%    is fooled by an alias counter with the same name:
%    \begin{quote}
%\begin{verbatim}
%\newtheorem{foo}{Foo}% counter "foo"
%\newaliascnt{bar}{foo}% alias counter "bar"
%\newtheorem{bar}[bar]{Bar}
%\aliascntresetthe{bar}
%\end{verbatim}
%    \end{quote}
% \end{desc}
%
% \StopEventually{
% }
%
% \section{Implementation}
%
% \subsection{Identification}
%
%    \begin{macrocode}
%<*package>
\NeedsTeXFormat{LaTeX2e}
\ProvidesPackage{aliascnt}%
  [2009/09/08 v1.3 Alias counters (HO)]%
%    \end{macrocode}
%
% \subsection{Create new alias counter}
%
%    \begin{macro}{\newaliascnt}
%    A new alias counter is set up by \cs{newaliascnt}.
%    The following properties are added for the new counter CNT:
%    \begin{description}
%    \item[\mdseries\cs{theH}\meta{CNT}:] Compatibility for \xpackage{hyperref}
%    \item[\mdseries\cs{AC@cnt@}\meta{CNT}:] Name of the referenced counter
%      in the definition.
%    \end{description}
%    \begin{macrocode}
\newcommand*{\newaliascnt}[2]{%
  \begingroup
    \def\AC@glet##1{%
      \global\expandafter\let\csname##1#1\expandafter\endcsname
        \csname##1#2\endcsname
    }%
    \@ifundefined{c@#2}{%
      \@nocounterr{#2}%
    }{%
      \expandafter\@ifdefinable\csname c@#1\endcsname{%
        \AC@glet{c@}%
        \AC@glet{the}%
        \AC@glet{theH}%
        \AC@glet{p@}%
        \expandafter\gdef\csname AC@cnt@#1\endcsname{#2}%
        \expandafter\gdef\csname cl@#1\expandafter\endcsname
        \expandafter{\csname cl@#2\endcsname}%
      }%
    }%
  \endgroup
}
%    \end{macrocode}
%    \end{macro}
%
%    \begin{macro}{\aliascntresetthe}
%    The \cs{the}\meta{CNT} macro is restored using the
%    main counter.
%    \begin{macrocode}
\newcommand*{\aliascntresetthe}[1]{%
  \@ifundefined{AC@cnt@#1}{%
    \PackageError{aliascnt}{%
      `#1' is not an alias counter%
    }\@ehc
  }{%
    \expandafter\let\csname the#1\expandafter\endcsname
      \csname the\csname AC@cnt@#1\endcsname\endcsname
  }%
}
%    \end{macrocode}
%    \end{macro}
%
% \subsection{Counter clear list}
%
%    The alias counters share the same register and clear list.
%    Therefore we must ensure that manipulations to the clear list
%    are done with the clear list macro of a real counter.
%    \begin{macro}{\AC@findrootcnt}
%    \cs{AC@findrootcnt} walks throught the aliasing relations
%    to find the base counter.
%    \begin{macrocode}
\newcommand*{\AC@findrootcnt}[1]{%
  \@ifundefined{AC@cnt@#1}{%
    #1%
  }{%
    \expandafter\AC@findrootcnt\csname AC@cnt@#1\endcsname
  }%
}
%    \end{macrocode}
%    \end{macro}
%
%    Clear lists are manipulated by \cs{@addtoreset} and
%    \cs{@removefromreset}. The latter one is provided by
%    the \xpackage{remreset} package (\cite{remreset}).
%
%    \begin{macro}{\AC@patch}
%    The same patch principle is applicable to both
%    \cs{@addtoreset} and \cs{@removefromreset}.
%    \begin{macrocode}
\def\AC@patch#1{%
  \expandafter\let\csname AC@org@#1reset\expandafter\endcsname
    \csname @#1reset\endcsname
  \expandafter\def\csname @#1reset\endcsname##1##2{%
    \csname AC@org@#1reset\endcsname{##1}{\AC@findrootcnt{##2}}%
  }%
}
%    \end{macrocode}
%    \end{macro}
%    If \xpackage{remreset} is not loaded we cannot delay
%    the patch to \cs{AtBeginDocumen}, because \cs{@removefromreset}
%    can be called in between. Therefore we force the loading of
%    the package.
%    \begin{macrocode}
\RequirePackage{remreset}
\AC@patch{addto}
\AC@patch{removefrom}
%    \end{macrocode}
%
%    \begin{macrocode}
%</package>
%    \end{macrocode}
%
% \section{Installation}
%
% \subsection{Download}
%
% \paragraph{Package.} This package is available on
% CTAN\footnote{\url{ftp://ftp.ctan.org/tex-archive/}}:
% \begin{description}
% \item[\CTAN{macros/latex/contrib/oberdiek/aliascnt.dtx}] The source file.
% \item[\CTAN{macros/latex/contrib/oberdiek/aliascnt.pdf}] Documentation.
% \end{description}
%
%
% \paragraph{Bundle.} All the packages of the bundle `oberdiek'
% are also available in a TDS compliant ZIP archive. There
% the packages are already unpacked and the documentation files
% are generated. The files and directories obey the TDS standard.
% \begin{description}
% \item[\CTAN{install/macros/latex/contrib/oberdiek.tds.zip}]
% \end{description}
% \emph{TDS} refers to the standard ``A Directory Structure
% for \TeX\ Files'' (\CTAN{tds/tds.pdf}). Directories
% with \xfile{texmf} in their name are usually organized this way.
%
% \subsection{Bundle installation}
%
% \paragraph{Unpacking.} Unpack the \xfile{oberdiek.tds.zip} in the
% TDS tree (also known as \xfile{texmf} tree) of your choice.
% Example (linux):
% \begin{quote}
%   |unzip oberdiek.tds.zip -d ~/texmf|
% \end{quote}
%
% \paragraph{Script installation.}
% Check the directory \xfile{TDS:scripts/oberdiek/} for
% scripts that need further installation steps.
% Package \xpackage{attachfile2} comes with the Perl script
% \xfile{pdfatfi.pl} that should be installed in such a way
% that it can be called as \texttt{pdfatfi}.
% Example (linux):
% \begin{quote}
%   |chmod +x scripts/oberdiek/pdfatfi.pl|\\
%   |cp scripts/oberdiek/pdfatfi.pl /usr/local/bin/|
% \end{quote}
%
% \subsection{Package installation}
%
% \paragraph{Unpacking.} The \xfile{.dtx} file is a self-extracting
% \docstrip\ archive. The files are extracted by running the
% \xfile{.dtx} through \plainTeX:
% \begin{quote}
%   \verb|tex aliascnt.dtx|
% \end{quote}
%
% \paragraph{TDS.} Now the different files must be moved into
% the different directories in your installation TDS tree
% (also known as \xfile{texmf} tree):
% \begin{quote}
% \def\t{^^A
% \begin{tabular}{@{}>{\ttfamily}l@{ $\rightarrow$ }>{\ttfamily}l@{}}
%   aliascnt.sty & tex/latex/oberdiek/aliascnt.sty\\
%   aliascnt.pdf & doc/latex/oberdiek/aliascnt.pdf\\
%   aliascnt.dtx & source/latex/oberdiek/aliascnt.dtx\\
% \end{tabular}^^A
% }^^A
% \sbox0{\t}^^A
% \ifdim\wd0>\linewidth
%   \begingroup
%     \advance\linewidth by\leftmargin
%     \advance\linewidth by\rightmargin
%   \edef\x{\endgroup
%     \def\noexpand\lw{\the\linewidth}^^A
%   }\x
%   \def\lwbox{^^A
%     \leavevmode
%     \hbox to \linewidth{^^A
%       \kern-\leftmargin\relax
%       \hss
%       \usebox0
%       \hss
%       \kern-\rightmargin\relax
%     }^^A
%   }^^A
%   \ifdim\wd0>\lw
%     \sbox0{\small\t}^^A
%     \ifdim\wd0>\linewidth
%       \ifdim\wd0>\lw
%         \sbox0{\footnotesize\t}^^A
%         \ifdim\wd0>\linewidth
%           \ifdim\wd0>\lw
%             \sbox0{\scriptsize\t}^^A
%             \ifdim\wd0>\linewidth
%               \ifdim\wd0>\lw
%                 \sbox0{\tiny\t}^^A
%                 \ifdim\wd0>\linewidth
%                   \lwbox
%                 \else
%                   \usebox0
%                 \fi
%               \else
%                 \lwbox
%               \fi
%             \else
%               \usebox0
%             \fi
%           \else
%             \lwbox
%           \fi
%         \else
%           \usebox0
%         \fi
%       \else
%         \lwbox
%       \fi
%     \else
%       \usebox0
%     \fi
%   \else
%     \lwbox
%   \fi
% \else
%   \usebox0
% \fi
% \end{quote}
% If you have a \xfile{docstrip.cfg} that configures and enables \docstrip's
% TDS installing feature, then some files can already be in the right
% place, see the documentation of \docstrip.
%
% \subsection{Refresh file name databases}
%
% If your \TeX~distribution
% (\teTeX, \mikTeX, \dots) relies on file name databases, you must refresh
% these. For example, \teTeX\ users run \verb|texhash| or
% \verb|mktexlsr|.
%
% \subsection{Some details for the interested}
%
% \paragraph{Attached source.}
%
% The PDF documentation on CTAN also includes the
% \xfile{.dtx} source file. It can be extracted by
% AcrobatReader 6 or higher. Another option is \textsf{pdftk},
% e.g. unpack the file into the current directory:
% \begin{quote}
%   \verb|pdftk aliascnt.pdf unpack_files output .|
% \end{quote}
%
% \paragraph{Unpacking with \LaTeX.}
% The \xfile{.dtx} chooses its action depending on the format:
% \begin{description}
% \item[\plainTeX:] Run \docstrip\ and extract the files.
% \item[\LaTeX:] Generate the documentation.
% \end{description}
% If you insist on using \LaTeX\ for \docstrip\ (really,
% \docstrip\ does not need \LaTeX), then inform the autodetect routine
% about your intention:
% \begin{quote}
%   \verb|latex \let\install=y\input{aliascnt.dtx}|
% \end{quote}
% Do not forget to quote the argument according to the demands
% of your shell.
%
% \paragraph{Generating the documentation.}
% You can use both the \xfile{.dtx} or the \xfile{.drv} to generate
% the documentation. The process can be configured by the
% configuration file \xfile{ltxdoc.cfg}. For instance, put this
% line into this file, if you want to have A4 as paper format:
% \begin{quote}
%   \verb|\PassOptionsToClass{a4paper}{article}|
% \end{quote}
% An example follows how to generate the
% documentation with pdf\LaTeX:
% \begin{quote}
%\begin{verbatim}
%pdflatex aliascnt.dtx
%makeindex -s gind.ist aliascnt.idx
%pdflatex aliascnt.dtx
%makeindex -s gind.ist aliascnt.idx
%pdflatex aliascnt.dtx
%\end{verbatim}
% \end{quote}
%
% \section{Catalogue}
%
% The following XML file can be used as source for the
% \href{http://mirror.ctan.org/help/Catalogue/catalogue.html}{\TeX\ Catalogue}.
% The elements \texttt{caption} and \texttt{description} are imported
% from the original XML file from the Catalogue.
% The name of the XML file in the Catalogue is \xfile{aliascnt.xml}.
%    \begin{macrocode}
%<*catalogue>
<?xml version='1.0' encoding='us-ascii'?>
<!DOCTYPE entry SYSTEM 'catalogue.dtd'>
<entry datestamp='$Date$' modifier='$Author$' id='aliascnt'>
  <name>aliascnt</name>
  <caption>Alias counters.</caption>
  <authorref id='auth:oberdiek'/>
  <copyright owner='Heiko Oberdiek' year='2006,2009'/>
  <license type='lppl1.3'/>
  <version number='1.3'/>
  <description>
    This package introduces aliases for counters, that
    share the same counter register and clear list.
    <p/>
    The package is part of the <xref refid='oberdiek'>oberdiek</xref>
    bundle.
  </description>
  <documentation details='Package documentation'
      href='ctan:/macros/latex/contrib/oberdiek/aliascnt.pdf'/>
  <ctan file='true' path='/macros/latex/contrib/oberdiek/aliascnt.dtx'/>
  <miktex location='oberdiek'/>
  <texlive location='oberdiek'/>
  <install path='/macros/latex/contrib/oberdiek/oberdiek.tds.zip'/>
</entry>
%</catalogue>
%    \end{macrocode}
%
% \section{Acknowledgement}
%
% \begin{description}
% \item[Ulrich Schwarz:] The package is based on his draft for
%   ``Die \TeX nische Kom\"odie'', see \cite{schwarz}.
% \end{description}
%
% \begin{thebibliography}{9}
%
% \bibitem{schwarz}
%   Ulrich Schwarz:
%   \textit{Was hinten herauskommt z\"ahlt: Counter Aliasing in \LaTeX},
%   \textit{Die \TeX nische Kom\"odie}, 3/2006, pages 8--14, Juli 2006.
%
% \bibitem{remreset}
%   David Carlisle: \textit{The \xpackage{remreset} package};
%   1997/09/28;
%   \CTAN{macros/latex/contrib/carlisle/remreset.sty}.
%
% \bibitem{hyperref}
%   Sebastian Rahtz, Heiko Oberdiek:
%   \textit{The \xpackage{hyperref} package};
%   2006/08/16 v6.75c;
%   \CTAN{macros/latex/contrib/hyperref/}.
%
% \end{thebibliography}
%
% \begin{History}
%   \begin{Version}{2006/02/20 v1.0}
%   \item
%     First version.
%   \end{Version}
%   \begin{Version}{2006/08/16 v1.1}
%   \item
%     Update of bibliography.
%   \end{Version}
%   \begin{Version}{2006/09/25 v1.2}
%   \item
%     Bug fix (\cs{aliascntresetthe}).
%   \end{Version}
%   \begin{Version}{2009/09/08 v1.3}
%   \item
%     Bug fix of \cs{@ifdefinable}'s use (thanks to Uwe L\"uck).
%   \end{Version}
% \end{History}
%
% \PrintIndex
%
% \Finale
\endinput

%        (quote the arguments according to the demands of your shell)
%
% Documentation:
%    (a) If aliascnt.drv is present:
%           latex aliascnt.drv
%    (b) Without aliascnt.drv:
%           latex aliascnt.dtx; ...
%    The class ltxdoc loads the configuration file ltxdoc.cfg
%    if available. Here you can specify further options, e.g.
%    use A4 as paper format:
%       \PassOptionsToClass{a4paper}{article}
%
%    Programm calls to get the documentation (example):
%       pdflatex aliascnt.dtx
%       makeindex -s gind.ist aliascnt.idx
%       pdflatex aliascnt.dtx
%       makeindex -s gind.ist aliascnt.idx
%       pdflatex aliascnt.dtx
%
% Installation:
%    TDS:tex/latex/oberdiek/aliascnt.sty
%    TDS:doc/latex/oberdiek/aliascnt.pdf
%    TDS:source/latex/oberdiek/aliascnt.dtx
%
%<*ignore>
\begingroup
  \catcode123=1 %
  \catcode125=2 %
  \def\x{LaTeX2e}%
\expandafter\endgroup
\ifcase 0\ifx\install y1\fi\expandafter
         \ifx\csname processbatchFile\endcsname\relax\else1\fi
         \ifx\fmtname\x\else 1\fi\relax
\else\csname fi\endcsname
%</ignore>
%<*install>
\input docstrip.tex
\Msg{************************************************************************}
\Msg{* Installation}
\Msg{* Package: aliascnt 2009/09/08 v1.3 Alias counters (HO)}
\Msg{************************************************************************}

\keepsilent
\askforoverwritefalse

\let\MetaPrefix\relax
\preamble

This is a generated file.

Project: aliascnt
Version: 2009/09/08 v1.3

Copyright (C) 2006, 2009 by
   Heiko Oberdiek <heiko.oberdiek at googlemail.com>

This work may be distributed and/or modified under the
conditions of the LaTeX Project Public License, either
version 1.3c of this license or (at your option) any later
version. This version of this license is in
   http://www.latex-project.org/lppl/lppl-1-3c.txt
and the latest version of this license is in
   http://www.latex-project.org/lppl.txt
and version 1.3 or later is part of all distributions of
LaTeX version 2005/12/01 or later.

This work has the LPPL maintenance status "maintained".

This Current Maintainer of this work is Heiko Oberdiek.

This work consists of the main source file aliascnt.dtx
and the derived files
   aliascnt.sty, aliascnt.pdf, aliascnt.ins, aliascnt.drv.

\endpreamble
\let\MetaPrefix\DoubleperCent

\generate{%
  \file{aliascnt.ins}{\from{aliascnt.dtx}{install}}%
  \file{aliascnt.drv}{\from{aliascnt.dtx}{driver}}%
  \usedir{tex/latex/oberdiek}%
  \file{aliascnt.sty}{\from{aliascnt.dtx}{package}}%
  \nopreamble
  \nopostamble
  \usedir{source/latex/oberdiek/catalogue}%
  \file{aliascnt.xml}{\from{aliascnt.dtx}{catalogue}}%
}

\catcode32=13\relax% active space
\let =\space%
\Msg{************************************************************************}
\Msg{*}
\Msg{* To finish the installation you have to move the following}
\Msg{* file into a directory searched by TeX:}
\Msg{*}
\Msg{*     aliascnt.sty}
\Msg{*}
\Msg{* To produce the documentation run the file `aliascnt.drv'}
\Msg{* through LaTeX.}
\Msg{*}
\Msg{* Happy TeXing!}
\Msg{*}
\Msg{************************************************************************}

\endbatchfile
%</install>
%<*ignore>
\fi
%</ignore>
%<*driver>
\NeedsTeXFormat{LaTeX2e}
\ProvidesFile{aliascnt.drv}%
  [2009/09/08 v1.3 Alias counters (HO)]%
\documentclass{ltxdoc}
\usepackage{holtxdoc}[2011/11/22]
\begin{document}
  \DocInput{aliascnt.dtx}%
\end{document}
%</driver>
% \fi
%
% \CheckSum{78}
%
% \CharacterTable
%  {Upper-case    \A\B\C\D\E\F\G\H\I\J\K\L\M\N\O\P\Q\R\S\T\U\V\W\X\Y\Z
%   Lower-case    \a\b\c\d\e\f\g\h\i\j\k\l\m\n\o\p\q\r\s\t\u\v\w\x\y\z
%   Digits        \0\1\2\3\4\5\6\7\8\9
%   Exclamation   \!     Double quote  \"     Hash (number) \#
%   Dollar        \$     Percent       \%     Ampersand     \&
%   Acute accent  \'     Left paren    \(     Right paren   \)
%   Asterisk      \*     Plus          \+     Comma         \,
%   Minus         \-     Point         \.     Solidus       \/
%   Colon         \:     Semicolon     \;     Less than     \<
%   Equals        \=     Greater than  \>     Question mark \?
%   Commercial at \@     Left bracket  \[     Backslash     \\
%   Right bracket \]     Circumflex    \^     Underscore    \_
%   Grave accent  \`     Left brace    \{     Vertical bar  \|
%   Right brace   \}     Tilde         \~}
%
% \GetFileInfo{aliascnt.drv}
%
% \title{The \xpackage{aliascnt} package}
% \date{2009/09/08 v1.3}
% \author{Heiko Oberdiek\\\xemail{heiko.oberdiek at googlemail.com}}
%
% \maketitle
%
% \begin{abstract}
% Package \xpackage{aliascnt} introduces \emph{alias counters} that
% share the same counter register and clear list.
% \end{abstract}
%
% \tableofcontents
%
% \section{User interface}
%
% \subsection{Introduction}
%
% There are features that rely on the name of counters. For
% example, \xpackage{hyperref}'s \cs{autoref} indirectly uses
% the counter name to determine which label text it puts in front
% of the reference number (\cite{hyperref}).
% In some circumstances this fail: several theorem environments
% are defined by \cs{newtheorem} that share the same counter.
%
% \subsection{Syntax}
%
% Macro names in user land contain the package name
% \texttt{aliascnt} in order to prevent name clashes.
%
% \newenvironment{desc}{^^A
%   \list{}{^^A
%     \setlength{\labelwidth}{0pt}^^A
%     \setlength{\itemindent}{-.5\marginparwidth}^^A
%     \setlength{\leftmargin}{0pt}^^A
%     \let\makelabel\desclabel
%   }^^A
% }{^^A
%   \endlist
% }
% \newcommand*{\desclabel}[1]{^^A
%   \hspace{\labelsep}^^A
%   \normalfont
%   #1^^A
% }
% \newcommand*{\itemcs}[2]{^^A
%   \item[^^A
%      \expandafter\SpecialUsageIndex\csname #1\endcsname
%      {\cs{#1}#2}^^A
%   ]\mbox{}\\*[.5ex]^^A
%   \ignorespaces
% }
% \begin{desc}
% \itemcs{newaliascnt}{\marg{ALIASCNT}\marg{BASECNT}}
%    An alias counter ALIASCNT is created that does not allocate
%    a new \TeX\ counter register. It shares the count register and
%    the clear list with counter BASECNT. If the value of either
%    the two registers is changed, the changes affects both.
% \itemcs{aliascntresetthe}{\marg{ALIASCNT}}
%    This fixes a problem with \cs{newtheorem} if it
%    is fooled by an alias counter with the same name:
%    \begin{quote}
%\begin{verbatim}
%\newtheorem{foo}{Foo}% counter "foo"
%\newaliascnt{bar}{foo}% alias counter "bar"
%\newtheorem{bar}[bar]{Bar}
%\aliascntresetthe{bar}
%\end{verbatim}
%    \end{quote}
% \end{desc}
%
% \StopEventually{
% }
%
% \section{Implementation}
%
% \subsection{Identification}
%
%    \begin{macrocode}
%<*package>
\NeedsTeXFormat{LaTeX2e}
\ProvidesPackage{aliascnt}%
  [2009/09/08 v1.3 Alias counters (HO)]%
%    \end{macrocode}
%
% \subsection{Create new alias counter}
%
%    \begin{macro}{\newaliascnt}
%    A new alias counter is set up by \cs{newaliascnt}.
%    The following properties are added for the new counter CNT:
%    \begin{description}
%    \item[\mdseries\cs{theH}\meta{CNT}:] Compatibility for \xpackage{hyperref}
%    \item[\mdseries\cs{AC@cnt@}\meta{CNT}:] Name of the referenced counter
%      in the definition.
%    \end{description}
%    \begin{macrocode}
\newcommand*{\newaliascnt}[2]{%
  \begingroup
    \def\AC@glet##1{%
      \global\expandafter\let\csname##1#1\expandafter\endcsname
        \csname##1#2\endcsname
    }%
    \@ifundefined{c@#2}{%
      \@nocounterr{#2}%
    }{%
      \expandafter\@ifdefinable\csname c@#1\endcsname{%
        \AC@glet{c@}%
        \AC@glet{the}%
        \AC@glet{theH}%
        \AC@glet{p@}%
        \expandafter\gdef\csname AC@cnt@#1\endcsname{#2}%
        \expandafter\gdef\csname cl@#1\expandafter\endcsname
        \expandafter{\csname cl@#2\endcsname}%
      }%
    }%
  \endgroup
}
%    \end{macrocode}
%    \end{macro}
%
%    \begin{macro}{\aliascntresetthe}
%    The \cs{the}\meta{CNT} macro is restored using the
%    main counter.
%    \begin{macrocode}
\newcommand*{\aliascntresetthe}[1]{%
  \@ifundefined{AC@cnt@#1}{%
    \PackageError{aliascnt}{%
      `#1' is not an alias counter%
    }\@ehc
  }{%
    \expandafter\let\csname the#1\expandafter\endcsname
      \csname the\csname AC@cnt@#1\endcsname\endcsname
  }%
}
%    \end{macrocode}
%    \end{macro}
%
% \subsection{Counter clear list}
%
%    The alias counters share the same register and clear list.
%    Therefore we must ensure that manipulations to the clear list
%    are done with the clear list macro of a real counter.
%    \begin{macro}{\AC@findrootcnt}
%    \cs{AC@findrootcnt} walks throught the aliasing relations
%    to find the base counter.
%    \begin{macrocode}
\newcommand*{\AC@findrootcnt}[1]{%
  \@ifundefined{AC@cnt@#1}{%
    #1%
  }{%
    \expandafter\AC@findrootcnt\csname AC@cnt@#1\endcsname
  }%
}
%    \end{macrocode}
%    \end{macro}
%
%    Clear lists are manipulated by \cs{@addtoreset} and
%    \cs{@removefromreset}. The latter one is provided by
%    the \xpackage{remreset} package (\cite{remreset}).
%
%    \begin{macro}{\AC@patch}
%    The same patch principle is applicable to both
%    \cs{@addtoreset} and \cs{@removefromreset}.
%    \begin{macrocode}
\def\AC@patch#1{%
  \expandafter\let\csname AC@org@#1reset\expandafter\endcsname
    \csname @#1reset\endcsname
  \expandafter\def\csname @#1reset\endcsname##1##2{%
    \csname AC@org@#1reset\endcsname{##1}{\AC@findrootcnt{##2}}%
  }%
}
%    \end{macrocode}
%    \end{macro}
%    If \xpackage{remreset} is not loaded we cannot delay
%    the patch to \cs{AtBeginDocumen}, because \cs{@removefromreset}
%    can be called in between. Therefore we force the loading of
%    the package.
%    \begin{macrocode}
\RequirePackage{remreset}
\AC@patch{addto}
\AC@patch{removefrom}
%    \end{macrocode}
%
%    \begin{macrocode}
%</package>
%    \end{macrocode}
%
% \section{Installation}
%
% \subsection{Download}
%
% \paragraph{Package.} This package is available on
% CTAN\footnote{\url{ftp://ftp.ctan.org/tex-archive/}}:
% \begin{description}
% \item[\CTAN{macros/latex/contrib/oberdiek/aliascnt.dtx}] The source file.
% \item[\CTAN{macros/latex/contrib/oberdiek/aliascnt.pdf}] Documentation.
% \end{description}
%
%
% \paragraph{Bundle.} All the packages of the bundle `oberdiek'
% are also available in a TDS compliant ZIP archive. There
% the packages are already unpacked and the documentation files
% are generated. The files and directories obey the TDS standard.
% \begin{description}
% \item[\CTAN{install/macros/latex/contrib/oberdiek.tds.zip}]
% \end{description}
% \emph{TDS} refers to the standard ``A Directory Structure
% for \TeX\ Files'' (\CTAN{tds/tds.pdf}). Directories
% with \xfile{texmf} in their name are usually organized this way.
%
% \subsection{Bundle installation}
%
% \paragraph{Unpacking.} Unpack the \xfile{oberdiek.tds.zip} in the
% TDS tree (also known as \xfile{texmf} tree) of your choice.
% Example (linux):
% \begin{quote}
%   |unzip oberdiek.tds.zip -d ~/texmf|
% \end{quote}
%
% \paragraph{Script installation.}
% Check the directory \xfile{TDS:scripts/oberdiek/} for
% scripts that need further installation steps.
% Package \xpackage{attachfile2} comes with the Perl script
% \xfile{pdfatfi.pl} that should be installed in such a way
% that it can be called as \texttt{pdfatfi}.
% Example (linux):
% \begin{quote}
%   |chmod +x scripts/oberdiek/pdfatfi.pl|\\
%   |cp scripts/oberdiek/pdfatfi.pl /usr/local/bin/|
% \end{quote}
%
% \subsection{Package installation}
%
% \paragraph{Unpacking.} The \xfile{.dtx} file is a self-extracting
% \docstrip\ archive. The files are extracted by running the
% \xfile{.dtx} through \plainTeX:
% \begin{quote}
%   \verb|tex aliascnt.dtx|
% \end{quote}
%
% \paragraph{TDS.} Now the different files must be moved into
% the different directories in your installation TDS tree
% (also known as \xfile{texmf} tree):
% \begin{quote}
% \def\t{^^A
% \begin{tabular}{@{}>{\ttfamily}l@{ $\rightarrow$ }>{\ttfamily}l@{}}
%   aliascnt.sty & tex/latex/oberdiek/aliascnt.sty\\
%   aliascnt.pdf & doc/latex/oberdiek/aliascnt.pdf\\
%   aliascnt.dtx & source/latex/oberdiek/aliascnt.dtx\\
% \end{tabular}^^A
% }^^A
% \sbox0{\t}^^A
% \ifdim\wd0>\linewidth
%   \begingroup
%     \advance\linewidth by\leftmargin
%     \advance\linewidth by\rightmargin
%   \edef\x{\endgroup
%     \def\noexpand\lw{\the\linewidth}^^A
%   }\x
%   \def\lwbox{^^A
%     \leavevmode
%     \hbox to \linewidth{^^A
%       \kern-\leftmargin\relax
%       \hss
%       \usebox0
%       \hss
%       \kern-\rightmargin\relax
%     }^^A
%   }^^A
%   \ifdim\wd0>\lw
%     \sbox0{\small\t}^^A
%     \ifdim\wd0>\linewidth
%       \ifdim\wd0>\lw
%         \sbox0{\footnotesize\t}^^A
%         \ifdim\wd0>\linewidth
%           \ifdim\wd0>\lw
%             \sbox0{\scriptsize\t}^^A
%             \ifdim\wd0>\linewidth
%               \ifdim\wd0>\lw
%                 \sbox0{\tiny\t}^^A
%                 \ifdim\wd0>\linewidth
%                   \lwbox
%                 \else
%                   \usebox0
%                 \fi
%               \else
%                 \lwbox
%               \fi
%             \else
%               \usebox0
%             \fi
%           \else
%             \lwbox
%           \fi
%         \else
%           \usebox0
%         \fi
%       \else
%         \lwbox
%       \fi
%     \else
%       \usebox0
%     \fi
%   \else
%     \lwbox
%   \fi
% \else
%   \usebox0
% \fi
% \end{quote}
% If you have a \xfile{docstrip.cfg} that configures and enables \docstrip's
% TDS installing feature, then some files can already be in the right
% place, see the documentation of \docstrip.
%
% \subsection{Refresh file name databases}
%
% If your \TeX~distribution
% (\teTeX, \mikTeX, \dots) relies on file name databases, you must refresh
% these. For example, \teTeX\ users run \verb|texhash| or
% \verb|mktexlsr|.
%
% \subsection{Some details for the interested}
%
% \paragraph{Attached source.}
%
% The PDF documentation on CTAN also includes the
% \xfile{.dtx} source file. It can be extracted by
% AcrobatReader 6 or higher. Another option is \textsf{pdftk},
% e.g. unpack the file into the current directory:
% \begin{quote}
%   \verb|pdftk aliascnt.pdf unpack_files output .|
% \end{quote}
%
% \paragraph{Unpacking with \LaTeX.}
% The \xfile{.dtx} chooses its action depending on the format:
% \begin{description}
% \item[\plainTeX:] Run \docstrip\ and extract the files.
% \item[\LaTeX:] Generate the documentation.
% \end{description}
% If you insist on using \LaTeX\ for \docstrip\ (really,
% \docstrip\ does not need \LaTeX), then inform the autodetect routine
% about your intention:
% \begin{quote}
%   \verb|latex \let\install=y% \iffalse meta-comment
%
% File: aliascnt.dtx
% Version: 2009/09/08 v1.3
% Info: Alias counters
%
% Copyright (C) 2006, 2009 by
%    Heiko Oberdiek <heiko.oberdiek at googlemail.com>
%
% This work may be distributed and/or modified under the
% conditions of the LaTeX Project Public License, either
% version 1.3c of this license or (at your option) any later
% version. This version of this license is in
%    http://www.latex-project.org/lppl/lppl-1-3c.txt
% and the latest version of this license is in
%    http://www.latex-project.org/lppl.txt
% and version 1.3 or later is part of all distributions of
% LaTeX version 2005/12/01 or later.
%
% This work has the LPPL maintenance status "maintained".
%
% This Current Maintainer of this work is Heiko Oberdiek.
%
% This work consists of the main source file aliascnt.dtx
% and the derived files
%    aliascnt.sty, aliascnt.pdf, aliascnt.ins, aliascnt.drv.
%
% Distribution:
%    CTAN:macros/latex/contrib/oberdiek/aliascnt.dtx
%    CTAN:macros/latex/contrib/oberdiek/aliascnt.pdf
%
% Unpacking:
%    (a) If aliascnt.ins is present:
%           tex aliascnt.ins
%    (b) Without aliascnt.ins:
%           tex aliascnt.dtx
%    (c) If you insist on using LaTeX
%           latex \let\install=y\input{aliascnt.dtx}
%        (quote the arguments according to the demands of your shell)
%
% Documentation:
%    (a) If aliascnt.drv is present:
%           latex aliascnt.drv
%    (b) Without aliascnt.drv:
%           latex aliascnt.dtx; ...
%    The class ltxdoc loads the configuration file ltxdoc.cfg
%    if available. Here you can specify further options, e.g.
%    use A4 as paper format:
%       \PassOptionsToClass{a4paper}{article}
%
%    Programm calls to get the documentation (example):
%       pdflatex aliascnt.dtx
%       makeindex -s gind.ist aliascnt.idx
%       pdflatex aliascnt.dtx
%       makeindex -s gind.ist aliascnt.idx
%       pdflatex aliascnt.dtx
%
% Installation:
%    TDS:tex/latex/oberdiek/aliascnt.sty
%    TDS:doc/latex/oberdiek/aliascnt.pdf
%    TDS:source/latex/oberdiek/aliascnt.dtx
%
%<*ignore>
\begingroup
  \catcode123=1 %
  \catcode125=2 %
  \def\x{LaTeX2e}%
\expandafter\endgroup
\ifcase 0\ifx\install y1\fi\expandafter
         \ifx\csname processbatchFile\endcsname\relax\else1\fi
         \ifx\fmtname\x\else 1\fi\relax
\else\csname fi\endcsname
%</ignore>
%<*install>
\input docstrip.tex
\Msg{************************************************************************}
\Msg{* Installation}
\Msg{* Package: aliascnt 2009/09/08 v1.3 Alias counters (HO)}
\Msg{************************************************************************}

\keepsilent
\askforoverwritefalse

\let\MetaPrefix\relax
\preamble

This is a generated file.

Project: aliascnt
Version: 2009/09/08 v1.3

Copyright (C) 2006, 2009 by
   Heiko Oberdiek <heiko.oberdiek at googlemail.com>

This work may be distributed and/or modified under the
conditions of the LaTeX Project Public License, either
version 1.3c of this license or (at your option) any later
version. This version of this license is in
   http://www.latex-project.org/lppl/lppl-1-3c.txt
and the latest version of this license is in
   http://www.latex-project.org/lppl.txt
and version 1.3 or later is part of all distributions of
LaTeX version 2005/12/01 or later.

This work has the LPPL maintenance status "maintained".

This Current Maintainer of this work is Heiko Oberdiek.

This work consists of the main source file aliascnt.dtx
and the derived files
   aliascnt.sty, aliascnt.pdf, aliascnt.ins, aliascnt.drv.

\endpreamble
\let\MetaPrefix\DoubleperCent

\generate{%
  \file{aliascnt.ins}{\from{aliascnt.dtx}{install}}%
  \file{aliascnt.drv}{\from{aliascnt.dtx}{driver}}%
  \usedir{tex/latex/oberdiek}%
  \file{aliascnt.sty}{\from{aliascnt.dtx}{package}}%
  \nopreamble
  \nopostamble
  \usedir{source/latex/oberdiek/catalogue}%
  \file{aliascnt.xml}{\from{aliascnt.dtx}{catalogue}}%
}

\catcode32=13\relax% active space
\let =\space%
\Msg{************************************************************************}
\Msg{*}
\Msg{* To finish the installation you have to move the following}
\Msg{* file into a directory searched by TeX:}
\Msg{*}
\Msg{*     aliascnt.sty}
\Msg{*}
\Msg{* To produce the documentation run the file `aliascnt.drv'}
\Msg{* through LaTeX.}
\Msg{*}
\Msg{* Happy TeXing!}
\Msg{*}
\Msg{************************************************************************}

\endbatchfile
%</install>
%<*ignore>
\fi
%</ignore>
%<*driver>
\NeedsTeXFormat{LaTeX2e}
\ProvidesFile{aliascnt.drv}%
  [2009/09/08 v1.3 Alias counters (HO)]%
\documentclass{ltxdoc}
\usepackage{holtxdoc}[2011/11/22]
\begin{document}
  \DocInput{aliascnt.dtx}%
\end{document}
%</driver>
% \fi
%
% \CheckSum{78}
%
% \CharacterTable
%  {Upper-case    \A\B\C\D\E\F\G\H\I\J\K\L\M\N\O\P\Q\R\S\T\U\V\W\X\Y\Z
%   Lower-case    \a\b\c\d\e\f\g\h\i\j\k\l\m\n\o\p\q\r\s\t\u\v\w\x\y\z
%   Digits        \0\1\2\3\4\5\6\7\8\9
%   Exclamation   \!     Double quote  \"     Hash (number) \#
%   Dollar        \$     Percent       \%     Ampersand     \&
%   Acute accent  \'     Left paren    \(     Right paren   \)
%   Asterisk      \*     Plus          \+     Comma         \,
%   Minus         \-     Point         \.     Solidus       \/
%   Colon         \:     Semicolon     \;     Less than     \<
%   Equals        \=     Greater than  \>     Question mark \?
%   Commercial at \@     Left bracket  \[     Backslash     \\
%   Right bracket \]     Circumflex    \^     Underscore    \_
%   Grave accent  \`     Left brace    \{     Vertical bar  \|
%   Right brace   \}     Tilde         \~}
%
% \GetFileInfo{aliascnt.drv}
%
% \title{The \xpackage{aliascnt} package}
% \date{2009/09/08 v1.3}
% \author{Heiko Oberdiek\\\xemail{heiko.oberdiek at googlemail.com}}
%
% \maketitle
%
% \begin{abstract}
% Package \xpackage{aliascnt} introduces \emph{alias counters} that
% share the same counter register and clear list.
% \end{abstract}
%
% \tableofcontents
%
% \section{User interface}
%
% \subsection{Introduction}
%
% There are features that rely on the name of counters. For
% example, \xpackage{hyperref}'s \cs{autoref} indirectly uses
% the counter name to determine which label text it puts in front
% of the reference number (\cite{hyperref}).
% In some circumstances this fail: several theorem environments
% are defined by \cs{newtheorem} that share the same counter.
%
% \subsection{Syntax}
%
% Macro names in user land contain the package name
% \texttt{aliascnt} in order to prevent name clashes.
%
% \newenvironment{desc}{^^A
%   \list{}{^^A
%     \setlength{\labelwidth}{0pt}^^A
%     \setlength{\itemindent}{-.5\marginparwidth}^^A
%     \setlength{\leftmargin}{0pt}^^A
%     \let\makelabel\desclabel
%   }^^A
% }{^^A
%   \endlist
% }
% \newcommand*{\desclabel}[1]{^^A
%   \hspace{\labelsep}^^A
%   \normalfont
%   #1^^A
% }
% \newcommand*{\itemcs}[2]{^^A
%   \item[^^A
%      \expandafter\SpecialUsageIndex\csname #1\endcsname
%      {\cs{#1}#2}^^A
%   ]\mbox{}\\*[.5ex]^^A
%   \ignorespaces
% }
% \begin{desc}
% \itemcs{newaliascnt}{\marg{ALIASCNT}\marg{BASECNT}}
%    An alias counter ALIASCNT is created that does not allocate
%    a new \TeX\ counter register. It shares the count register and
%    the clear list with counter BASECNT. If the value of either
%    the two registers is changed, the changes affects both.
% \itemcs{aliascntresetthe}{\marg{ALIASCNT}}
%    This fixes a problem with \cs{newtheorem} if it
%    is fooled by an alias counter with the same name:
%    \begin{quote}
%\begin{verbatim}
%\newtheorem{foo}{Foo}% counter "foo"
%\newaliascnt{bar}{foo}% alias counter "bar"
%\newtheorem{bar}[bar]{Bar}
%\aliascntresetthe{bar}
%\end{verbatim}
%    \end{quote}
% \end{desc}
%
% \StopEventually{
% }
%
% \section{Implementation}
%
% \subsection{Identification}
%
%    \begin{macrocode}
%<*package>
\NeedsTeXFormat{LaTeX2e}
\ProvidesPackage{aliascnt}%
  [2009/09/08 v1.3 Alias counters (HO)]%
%    \end{macrocode}
%
% \subsection{Create new alias counter}
%
%    \begin{macro}{\newaliascnt}
%    A new alias counter is set up by \cs{newaliascnt}.
%    The following properties are added for the new counter CNT:
%    \begin{description}
%    \item[\mdseries\cs{theH}\meta{CNT}:] Compatibility for \xpackage{hyperref}
%    \item[\mdseries\cs{AC@cnt@}\meta{CNT}:] Name of the referenced counter
%      in the definition.
%    \end{description}
%    \begin{macrocode}
\newcommand*{\newaliascnt}[2]{%
  \begingroup
    \def\AC@glet##1{%
      \global\expandafter\let\csname##1#1\expandafter\endcsname
        \csname##1#2\endcsname
    }%
    \@ifundefined{c@#2}{%
      \@nocounterr{#2}%
    }{%
      \expandafter\@ifdefinable\csname c@#1\endcsname{%
        \AC@glet{c@}%
        \AC@glet{the}%
        \AC@glet{theH}%
        \AC@glet{p@}%
        \expandafter\gdef\csname AC@cnt@#1\endcsname{#2}%
        \expandafter\gdef\csname cl@#1\expandafter\endcsname
        \expandafter{\csname cl@#2\endcsname}%
      }%
    }%
  \endgroup
}
%    \end{macrocode}
%    \end{macro}
%
%    \begin{macro}{\aliascntresetthe}
%    The \cs{the}\meta{CNT} macro is restored using the
%    main counter.
%    \begin{macrocode}
\newcommand*{\aliascntresetthe}[1]{%
  \@ifundefined{AC@cnt@#1}{%
    \PackageError{aliascnt}{%
      `#1' is not an alias counter%
    }\@ehc
  }{%
    \expandafter\let\csname the#1\expandafter\endcsname
      \csname the\csname AC@cnt@#1\endcsname\endcsname
  }%
}
%    \end{macrocode}
%    \end{macro}
%
% \subsection{Counter clear list}
%
%    The alias counters share the same register and clear list.
%    Therefore we must ensure that manipulations to the clear list
%    are done with the clear list macro of a real counter.
%    \begin{macro}{\AC@findrootcnt}
%    \cs{AC@findrootcnt} walks throught the aliasing relations
%    to find the base counter.
%    \begin{macrocode}
\newcommand*{\AC@findrootcnt}[1]{%
  \@ifundefined{AC@cnt@#1}{%
    #1%
  }{%
    \expandafter\AC@findrootcnt\csname AC@cnt@#1\endcsname
  }%
}
%    \end{macrocode}
%    \end{macro}
%
%    Clear lists are manipulated by \cs{@addtoreset} and
%    \cs{@removefromreset}. The latter one is provided by
%    the \xpackage{remreset} package (\cite{remreset}).
%
%    \begin{macro}{\AC@patch}
%    The same patch principle is applicable to both
%    \cs{@addtoreset} and \cs{@removefromreset}.
%    \begin{macrocode}
\def\AC@patch#1{%
  \expandafter\let\csname AC@org@#1reset\expandafter\endcsname
    \csname @#1reset\endcsname
  \expandafter\def\csname @#1reset\endcsname##1##2{%
    \csname AC@org@#1reset\endcsname{##1}{\AC@findrootcnt{##2}}%
  }%
}
%    \end{macrocode}
%    \end{macro}
%    If \xpackage{remreset} is not loaded we cannot delay
%    the patch to \cs{AtBeginDocumen}, because \cs{@removefromreset}
%    can be called in between. Therefore we force the loading of
%    the package.
%    \begin{macrocode}
\RequirePackage{remreset}
\AC@patch{addto}
\AC@patch{removefrom}
%    \end{macrocode}
%
%    \begin{macrocode}
%</package>
%    \end{macrocode}
%
% \section{Installation}
%
% \subsection{Download}
%
% \paragraph{Package.} This package is available on
% CTAN\footnote{\url{ftp://ftp.ctan.org/tex-archive/}}:
% \begin{description}
% \item[\CTAN{macros/latex/contrib/oberdiek/aliascnt.dtx}] The source file.
% \item[\CTAN{macros/latex/contrib/oberdiek/aliascnt.pdf}] Documentation.
% \end{description}
%
%
% \paragraph{Bundle.} All the packages of the bundle `oberdiek'
% are also available in a TDS compliant ZIP archive. There
% the packages are already unpacked and the documentation files
% are generated. The files and directories obey the TDS standard.
% \begin{description}
% \item[\CTAN{install/macros/latex/contrib/oberdiek.tds.zip}]
% \end{description}
% \emph{TDS} refers to the standard ``A Directory Structure
% for \TeX\ Files'' (\CTAN{tds/tds.pdf}). Directories
% with \xfile{texmf} in their name are usually organized this way.
%
% \subsection{Bundle installation}
%
% \paragraph{Unpacking.} Unpack the \xfile{oberdiek.tds.zip} in the
% TDS tree (also known as \xfile{texmf} tree) of your choice.
% Example (linux):
% \begin{quote}
%   |unzip oberdiek.tds.zip -d ~/texmf|
% \end{quote}
%
% \paragraph{Script installation.}
% Check the directory \xfile{TDS:scripts/oberdiek/} for
% scripts that need further installation steps.
% Package \xpackage{attachfile2} comes with the Perl script
% \xfile{pdfatfi.pl} that should be installed in such a way
% that it can be called as \texttt{pdfatfi}.
% Example (linux):
% \begin{quote}
%   |chmod +x scripts/oberdiek/pdfatfi.pl|\\
%   |cp scripts/oberdiek/pdfatfi.pl /usr/local/bin/|
% \end{quote}
%
% \subsection{Package installation}
%
% \paragraph{Unpacking.} The \xfile{.dtx} file is a self-extracting
% \docstrip\ archive. The files are extracted by running the
% \xfile{.dtx} through \plainTeX:
% \begin{quote}
%   \verb|tex aliascnt.dtx|
% \end{quote}
%
% \paragraph{TDS.} Now the different files must be moved into
% the different directories in your installation TDS tree
% (also known as \xfile{texmf} tree):
% \begin{quote}
% \def\t{^^A
% \begin{tabular}{@{}>{\ttfamily}l@{ $\rightarrow$ }>{\ttfamily}l@{}}
%   aliascnt.sty & tex/latex/oberdiek/aliascnt.sty\\
%   aliascnt.pdf & doc/latex/oberdiek/aliascnt.pdf\\
%   aliascnt.dtx & source/latex/oberdiek/aliascnt.dtx\\
% \end{tabular}^^A
% }^^A
% \sbox0{\t}^^A
% \ifdim\wd0>\linewidth
%   \begingroup
%     \advance\linewidth by\leftmargin
%     \advance\linewidth by\rightmargin
%   \edef\x{\endgroup
%     \def\noexpand\lw{\the\linewidth}^^A
%   }\x
%   \def\lwbox{^^A
%     \leavevmode
%     \hbox to \linewidth{^^A
%       \kern-\leftmargin\relax
%       \hss
%       \usebox0
%       \hss
%       \kern-\rightmargin\relax
%     }^^A
%   }^^A
%   \ifdim\wd0>\lw
%     \sbox0{\small\t}^^A
%     \ifdim\wd0>\linewidth
%       \ifdim\wd0>\lw
%         \sbox0{\footnotesize\t}^^A
%         \ifdim\wd0>\linewidth
%           \ifdim\wd0>\lw
%             \sbox0{\scriptsize\t}^^A
%             \ifdim\wd0>\linewidth
%               \ifdim\wd0>\lw
%                 \sbox0{\tiny\t}^^A
%                 \ifdim\wd0>\linewidth
%                   \lwbox
%                 \else
%                   \usebox0
%                 \fi
%               \else
%                 \lwbox
%               \fi
%             \else
%               \usebox0
%             \fi
%           \else
%             \lwbox
%           \fi
%         \else
%           \usebox0
%         \fi
%       \else
%         \lwbox
%       \fi
%     \else
%       \usebox0
%     \fi
%   \else
%     \lwbox
%   \fi
% \else
%   \usebox0
% \fi
% \end{quote}
% If you have a \xfile{docstrip.cfg} that configures and enables \docstrip's
% TDS installing feature, then some files can already be in the right
% place, see the documentation of \docstrip.
%
% \subsection{Refresh file name databases}
%
% If your \TeX~distribution
% (\teTeX, \mikTeX, \dots) relies on file name databases, you must refresh
% these. For example, \teTeX\ users run \verb|texhash| or
% \verb|mktexlsr|.
%
% \subsection{Some details for the interested}
%
% \paragraph{Attached source.}
%
% The PDF documentation on CTAN also includes the
% \xfile{.dtx} source file. It can be extracted by
% AcrobatReader 6 or higher. Another option is \textsf{pdftk},
% e.g. unpack the file into the current directory:
% \begin{quote}
%   \verb|pdftk aliascnt.pdf unpack_files output .|
% \end{quote}
%
% \paragraph{Unpacking with \LaTeX.}
% The \xfile{.dtx} chooses its action depending on the format:
% \begin{description}
% \item[\plainTeX:] Run \docstrip\ and extract the files.
% \item[\LaTeX:] Generate the documentation.
% \end{description}
% If you insist on using \LaTeX\ for \docstrip\ (really,
% \docstrip\ does not need \LaTeX), then inform the autodetect routine
% about your intention:
% \begin{quote}
%   \verb|latex \let\install=y\input{aliascnt.dtx}|
% \end{quote}
% Do not forget to quote the argument according to the demands
% of your shell.
%
% \paragraph{Generating the documentation.}
% You can use both the \xfile{.dtx} or the \xfile{.drv} to generate
% the documentation. The process can be configured by the
% configuration file \xfile{ltxdoc.cfg}. For instance, put this
% line into this file, if you want to have A4 as paper format:
% \begin{quote}
%   \verb|\PassOptionsToClass{a4paper}{article}|
% \end{quote}
% An example follows how to generate the
% documentation with pdf\LaTeX:
% \begin{quote}
%\begin{verbatim}
%pdflatex aliascnt.dtx
%makeindex -s gind.ist aliascnt.idx
%pdflatex aliascnt.dtx
%makeindex -s gind.ist aliascnt.idx
%pdflatex aliascnt.dtx
%\end{verbatim}
% \end{quote}
%
% \section{Catalogue}
%
% The following XML file can be used as source for the
% \href{http://mirror.ctan.org/help/Catalogue/catalogue.html}{\TeX\ Catalogue}.
% The elements \texttt{caption} and \texttt{description} are imported
% from the original XML file from the Catalogue.
% The name of the XML file in the Catalogue is \xfile{aliascnt.xml}.
%    \begin{macrocode}
%<*catalogue>
<?xml version='1.0' encoding='us-ascii'?>
<!DOCTYPE entry SYSTEM 'catalogue.dtd'>
<entry datestamp='$Date$' modifier='$Author$' id='aliascnt'>
  <name>aliascnt</name>
  <caption>Alias counters.</caption>
  <authorref id='auth:oberdiek'/>
  <copyright owner='Heiko Oberdiek' year='2006,2009'/>
  <license type='lppl1.3'/>
  <version number='1.3'/>
  <description>
    This package introduces aliases for counters, that
    share the same counter register and clear list.
    <p/>
    The package is part of the <xref refid='oberdiek'>oberdiek</xref>
    bundle.
  </description>
  <documentation details='Package documentation'
      href='ctan:/macros/latex/contrib/oberdiek/aliascnt.pdf'/>
  <ctan file='true' path='/macros/latex/contrib/oberdiek/aliascnt.dtx'/>
  <miktex location='oberdiek'/>
  <texlive location='oberdiek'/>
  <install path='/macros/latex/contrib/oberdiek/oberdiek.tds.zip'/>
</entry>
%</catalogue>
%    \end{macrocode}
%
% \section{Acknowledgement}
%
% \begin{description}
% \item[Ulrich Schwarz:] The package is based on his draft for
%   ``Die \TeX nische Kom\"odie'', see \cite{schwarz}.
% \end{description}
%
% \begin{thebibliography}{9}
%
% \bibitem{schwarz}
%   Ulrich Schwarz:
%   \textit{Was hinten herauskommt z\"ahlt: Counter Aliasing in \LaTeX},
%   \textit{Die \TeX nische Kom\"odie}, 3/2006, pages 8--14, Juli 2006.
%
% \bibitem{remreset}
%   David Carlisle: \textit{The \xpackage{remreset} package};
%   1997/09/28;
%   \CTAN{macros/latex/contrib/carlisle/remreset.sty}.
%
% \bibitem{hyperref}
%   Sebastian Rahtz, Heiko Oberdiek:
%   \textit{The \xpackage{hyperref} package};
%   2006/08/16 v6.75c;
%   \CTAN{macros/latex/contrib/hyperref/}.
%
% \end{thebibliography}
%
% \begin{History}
%   \begin{Version}{2006/02/20 v1.0}
%   \item
%     First version.
%   \end{Version}
%   \begin{Version}{2006/08/16 v1.1}
%   \item
%     Update of bibliography.
%   \end{Version}
%   \begin{Version}{2006/09/25 v1.2}
%   \item
%     Bug fix (\cs{aliascntresetthe}).
%   \end{Version}
%   \begin{Version}{2009/09/08 v1.3}
%   \item
%     Bug fix of \cs{@ifdefinable}'s use (thanks to Uwe L\"uck).
%   \end{Version}
% \end{History}
%
% \PrintIndex
%
% \Finale
\endinput
|
% \end{quote}
% Do not forget to quote the argument according to the demands
% of your shell.
%
% \paragraph{Generating the documentation.}
% You can use both the \xfile{.dtx} or the \xfile{.drv} to generate
% the documentation. The process can be configured by the
% configuration file \xfile{ltxdoc.cfg}. For instance, put this
% line into this file, if you want to have A4 as paper format:
% \begin{quote}
%   \verb|\PassOptionsToClass{a4paper}{article}|
% \end{quote}
% An example follows how to generate the
% documentation with pdf\LaTeX:
% \begin{quote}
%\begin{verbatim}
%pdflatex aliascnt.dtx
%makeindex -s gind.ist aliascnt.idx
%pdflatex aliascnt.dtx
%makeindex -s gind.ist aliascnt.idx
%pdflatex aliascnt.dtx
%\end{verbatim}
% \end{quote}
%
% \section{Catalogue}
%
% The following XML file can be used as source for the
% \href{http://mirror.ctan.org/help/Catalogue/catalogue.html}{\TeX\ Catalogue}.
% The elements \texttt{caption} and \texttt{description} are imported
% from the original XML file from the Catalogue.
% The name of the XML file in the Catalogue is \xfile{aliascnt.xml}.
%    \begin{macrocode}
%<*catalogue>
<?xml version='1.0' encoding='us-ascii'?>
<!DOCTYPE entry SYSTEM 'catalogue.dtd'>
<entry datestamp='$Date$' modifier='$Author$' id='aliascnt'>
  <name>aliascnt</name>
  <caption>Alias counters.</caption>
  <authorref id='auth:oberdiek'/>
  <copyright owner='Heiko Oberdiek' year='2006,2009'/>
  <license type='lppl1.3'/>
  <version number='1.3'/>
  <description>
    This package introduces aliases for counters, that
    share the same counter register and clear list.
    <p/>
    The package is part of the <xref refid='oberdiek'>oberdiek</xref>
    bundle.
  </description>
  <documentation details='Package documentation'
      href='ctan:/macros/latex/contrib/oberdiek/aliascnt.pdf'/>
  <ctan file='true' path='/macros/latex/contrib/oberdiek/aliascnt.dtx'/>
  <miktex location='oberdiek'/>
  <texlive location='oberdiek'/>
  <install path='/macros/latex/contrib/oberdiek/oberdiek.tds.zip'/>
</entry>
%</catalogue>
%    \end{macrocode}
%
% \section{Acknowledgement}
%
% \begin{description}
% \item[Ulrich Schwarz:] The package is based on his draft for
%   ``Die \TeX nische Kom\"odie'', see \cite{schwarz}.
% \end{description}
%
% \begin{thebibliography}{9}
%
% \bibitem{schwarz}
%   Ulrich Schwarz:
%   \textit{Was hinten herauskommt z\"ahlt: Counter Aliasing in \LaTeX},
%   \textit{Die \TeX nische Kom\"odie}, 3/2006, pages 8--14, Juli 2006.
%
% \bibitem{remreset}
%   David Carlisle: \textit{The \xpackage{remreset} package};
%   1997/09/28;
%   \CTAN{macros/latex/contrib/carlisle/remreset.sty}.
%
% \bibitem{hyperref}
%   Sebastian Rahtz, Heiko Oberdiek:
%   \textit{The \xpackage{hyperref} package};
%   2006/08/16 v6.75c;
%   \CTAN{macros/latex/contrib/hyperref/}.
%
% \end{thebibliography}
%
% \begin{History}
%   \begin{Version}{2006/02/20 v1.0}
%   \item
%     First version.
%   \end{Version}
%   \begin{Version}{2006/08/16 v1.1}
%   \item
%     Update of bibliography.
%   \end{Version}
%   \begin{Version}{2006/09/25 v1.2}
%   \item
%     Bug fix (\cs{aliascntresetthe}).
%   \end{Version}
%   \begin{Version}{2009/09/08 v1.3}
%   \item
%     Bug fix of \cs{@ifdefinable}'s use (thanks to Uwe L\"uck).
%   \end{Version}
% \end{History}
%
% \PrintIndex
%
% \Finale
\endinput
|
% \end{quote}
% Do not forget to quote the argument according to the demands
% of your shell.
%
% \paragraph{Generating the documentation.}
% You can use both the \xfile{.dtx} or the \xfile{.drv} to generate
% the documentation. The process can be configured by the
% configuration file \xfile{ltxdoc.cfg}. For instance, put this
% line into this file, if you want to have A4 as paper format:
% \begin{quote}
%   \verb|\PassOptionsToClass{a4paper}{article}|
% \end{quote}
% An example follows how to generate the
% documentation with pdf\LaTeX:
% \begin{quote}
%\begin{verbatim}
%pdflatex aliascnt.dtx
%makeindex -s gind.ist aliascnt.idx
%pdflatex aliascnt.dtx
%makeindex -s gind.ist aliascnt.idx
%pdflatex aliascnt.dtx
%\end{verbatim}
% \end{quote}
%
% \section{Catalogue}
%
% The following XML file can be used as source for the
% \href{http://mirror.ctan.org/help/Catalogue/catalogue.html}{\TeX\ Catalogue}.
% The elements \texttt{caption} and \texttt{description} are imported
% from the original XML file from the Catalogue.
% The name of the XML file in the Catalogue is \xfile{aliascnt.xml}.
%    \begin{macrocode}
%<*catalogue>
<?xml version='1.0' encoding='us-ascii'?>
<!DOCTYPE entry SYSTEM 'catalogue.dtd'>
<entry datestamp='$Date$' modifier='$Author$' id='aliascnt'>
  <name>aliascnt</name>
  <caption>Alias counters.</caption>
  <authorref id='auth:oberdiek'/>
  <copyright owner='Heiko Oberdiek' year='2006,2009'/>
  <license type='lppl1.3'/>
  <version number='1.3'/>
  <description>
    This package introduces aliases for counters, that
    share the same counter register and clear list.
    <p/>
    The package is part of the <xref refid='oberdiek'>oberdiek</xref>
    bundle.
  </description>
  <documentation details='Package documentation'
      href='ctan:/macros/latex/contrib/oberdiek/aliascnt.pdf'/>
  <ctan file='true' path='/macros/latex/contrib/oberdiek/aliascnt.dtx'/>
  <miktex location='oberdiek'/>
  <texlive location='oberdiek'/>
  <install path='/macros/latex/contrib/oberdiek/oberdiek.tds.zip'/>
</entry>
%</catalogue>
%    \end{macrocode}
%
% \section{Acknowledgement}
%
% \begin{description}
% \item[Ulrich Schwarz:] The package is based on his draft for
%   ``Die \TeX nische Kom\"odie'', see \cite{schwarz}.
% \end{description}
%
% \begin{thebibliography}{9}
%
% \bibitem{schwarz}
%   Ulrich Schwarz:
%   \textit{Was hinten herauskommt z\"ahlt: Counter Aliasing in \LaTeX},
%   \textit{Die \TeX nische Kom\"odie}, 3/2006, pages 8--14, Juli 2006.
%
% \bibitem{remreset}
%   David Carlisle: \textit{The \xpackage{remreset} package};
%   1997/09/28;
%   \CTAN{macros/latex/contrib/carlisle/remreset.sty}.
%
% \bibitem{hyperref}
%   Sebastian Rahtz, Heiko Oberdiek:
%   \textit{The \xpackage{hyperref} package};
%   2006/08/16 v6.75c;
%   \CTAN{macros/latex/contrib/hyperref/}.
%
% \end{thebibliography}
%
% \begin{History}
%   \begin{Version}{2006/02/20 v1.0}
%   \item
%     First version.
%   \end{Version}
%   \begin{Version}{2006/08/16 v1.1}
%   \item
%     Update of bibliography.
%   \end{Version}
%   \begin{Version}{2006/09/25 v1.2}
%   \item
%     Bug fix (\cs{aliascntresetthe}).
%   \end{Version}
%   \begin{Version}{2009/09/08 v1.3}
%   \item
%     Bug fix of \cs{@ifdefinable}'s use (thanks to Uwe L\"uck).
%   \end{Version}
% \end{History}
%
% \PrintIndex
%
% \Finale
\endinput

%        (quote the arguments according to the demands of your shell)
%
% Documentation:
%    (a) If aliascnt.drv is present:
%           latex aliascnt.drv
%    (b) Without aliascnt.drv:
%           latex aliascnt.dtx; ...
%    The class ltxdoc loads the configuration file ltxdoc.cfg
%    if available. Here you can specify further options, e.g.
%    use A4 as paper format:
%       \PassOptionsToClass{a4paper}{article}
%
%    Programm calls to get the documentation (example):
%       pdflatex aliascnt.dtx
%       makeindex -s gind.ist aliascnt.idx
%       pdflatex aliascnt.dtx
%       makeindex -s gind.ist aliascnt.idx
%       pdflatex aliascnt.dtx
%
% Installation:
%    TDS:tex/latex/oberdiek/aliascnt.sty
%    TDS:doc/latex/oberdiek/aliascnt.pdf
%    TDS:source/latex/oberdiek/aliascnt.dtx
%
%<*ignore>
\begingroup
  \catcode123=1 %
  \catcode125=2 %
  \def\x{LaTeX2e}%
\expandafter\endgroup
\ifcase 0\ifx\install y1\fi\expandafter
         \ifx\csname processbatchFile\endcsname\relax\else1\fi
         \ifx\fmtname\x\else 1\fi\relax
\else\csname fi\endcsname
%</ignore>
%<*install>
\input docstrip.tex
\Msg{************************************************************************}
\Msg{* Installation}
\Msg{* Package: aliascnt 2018/09/07 v1.5 Alias counters (HO)}
\Msg{************************************************************************}

\keepsilent
\askforoverwritefalse

\let\MetaPrefix\relax
\preamble

This is a generated file.

Project: aliascnt
Version: 2018/09/07 v1.5

Copyright (C) 2006, 2009 by
   Heiko Oberdiek <heiko.oberdiek at googlemail.com>

This work may be distributed and/or modified under the
conditions of the LaTeX Project Public License, either
version 1.3c of this license or (at your option) any later
version. This version of this license is in
   http://www.latex-project.org/lppl/lppl-1-3c.txt
and the latest version of this license is in
   http://www.latex-project.org/lppl.txt
and version 1.3 or later is part of all distributions of
LaTeX version 2005/12/01 or later.

This work has the LPPL maintenance status "maintained".

This Current Maintainer of this work is Heiko Oberdiek.

This work consists of the main source file aliascnt.dtx
and the derived files
   aliascnt.sty, aliascnt.pdf, aliascnt.ins, aliascnt.drv.

\endpreamble
\let\MetaPrefix\DoubleperCent

\generate{%
  \file{aliascnt.ins}{\from{aliascnt.dtx}{install}}%
  \file{aliascnt.drv}{\from{aliascnt.dtx}{driver}}%
  \usedir{tex/latex/oberdiek}%
  \file{aliascnt.sty}{\from{aliascnt.dtx}{package}}%
  \nopreamble
  \nopostamble
%  \usedir{source/latex/oberdiek/catalogue}%
%  \file{aliascnt.xml}{\from{aliascnt.dtx}{catalogue}}%
}

\catcode32=13\relax% active space
\let =\space%
\Msg{************************************************************************}
\Msg{*}
\Msg{* To finish the installation you have to move the following}
\Msg{* file into a directory searched by TeX:}
\Msg{*}
\Msg{*     aliascnt.sty}
\Msg{*}
\Msg{* To produce the documentation run the file `aliascnt.drv'}
\Msg{* through LaTeX.}
\Msg{*}
\Msg{* Happy TeXing!}
\Msg{*}
\Msg{************************************************************************}

\endbatchfile
%</install>
%<*ignore>
\fi
%</ignore>
%<*driver>
\NeedsTeXFormat{LaTeX2e}
\ProvidesFile{aliascnt.drv}%
  [2018/09/07 v1.5 Alias counters (HO)]%
\documentclass{ltxdoc}
\usepackage{holtxdoc}[2011/11/22]
\begin{document}
  \DocInput{aliascnt.dtx}%
\end{document}
%</driver>
% \fi
%
%
% \CharacterTable
%  {Upper-case    \A\B\C\D\E\F\G\H\I\J\K\L\M\N\O\P\Q\R\S\T\U\V\W\X\Y\Z
%   Lower-case    \a\b\c\d\e\f\g\h\i\j\k\l\m\n\o\p\q\r\s\t\u\v\w\x\y\z
%   Digits        \0\1\2\3\4\5\6\7\8\9
%   Exclamation   \!     Double quote  \"     Hash (number) \#
%   Dollar        \$     Percent       \%     Ampersand     \&
%   Acute accent  \'     Left paren    \(     Right paren   \)
%   Asterisk      \*     Plus          \+     Comma         \,
%   Minus         \-     Point         \.     Solidus       \/
%   Colon         \:     Semicolon     \;     Less than     \<
%   Equals        \=     Greater than  \>     Question mark \?
%   Commercial at \@     Left bracket  \[     Backslash     \\
%   Right bracket \]     Circumflex    \^     Underscore    \_
%   Grave accent  \`     Left brace    \{     Vertical bar  \|
%   Right brace   \}     Tilde         \~}
%
% \GetFileInfo{aliascnt.drv}
%
% \title{The \xpackage{aliascnt} package}
% \date{2018/09/07 v1.5}
% \author{Heiko Oberdiek\thanks
% {Please report any issues at https://github.com/ho-tex/oberdiek/issues}\\
% \xemail{heiko.oberdiek at googlemail.com}}
%
% \maketitle
%
% \begin{abstract}
% Package \xpackage{aliascnt} introduces \emph{alias counters} that
% share the same counter register and clear list.
% \end{abstract}
%
% \tableofcontents
%
% \section{User interface}
%
% \subsection{Introduction}
%
% There are features that rely on the name of counters. For
% example, \xpackage{hyperref}'s \cs{autoref} indirectly uses
% the counter name to determine which label text it puts in front
% of the reference number (\cite{hyperref}).
% In some circumstances this fail: several theorem environments
% are defined by \cs{newtheorem} that share the same counter.
%
% \subsection{Syntax}
%
% Macro names in user land contain the package name
% \texttt{aliascnt} in order to prevent name clashes.
%
% \newenvironment{desc}{^^A
%   \list{}{^^A
%     \setlength{\labelwidth}{0pt}^^A
%     \setlength{\itemindent}{-.5\marginparwidth}^^A
%     \setlength{\leftmargin}{0pt}^^A
%     \let\makelabel\desclabel
%   }^^A
% }{^^A
%   \endlist
% }
% \newcommand*{\desclabel}[1]{^^A
%   \hspace{\labelsep}^^A
%   \normalfont
%   #1^^A
% }
% \newcommand*{\itemcs}[2]{^^A
%   \item[^^A
%      \expandafter\SpecialUsageIndex\csname #1\endcsname
%      {\cs{#1}#2}^^A
%   ]\mbox{}\\*[.5ex]^^A
%   \ignorespaces
% }
% \begin{desc}
% \itemcs{newaliascnt}{\marg{ALIASCNT}\marg{BASECNT}}
%    An alias counter ALIASCNT is created that does not allocate
%    a new \TeX\ counter register. It shares the count register and
%    the clear list with counter BASECNT. If the value of either
%    the two registers is changed, the changes affects both.
% \itemcs{aliascntresetthe}{\marg{ALIASCNT}}
%    This fixes a problem with \cs{newtheorem} if it
%    is fooled by an alias counter with the same name:
%    \begin{quote}
%\begin{verbatim}
%\newtheorem{foo}{Foo}% counter "foo"
%\newaliascnt{bar}{foo}% alias counter "bar"
%\newtheorem{bar}[bar]{Bar}
%\aliascntresetthe{bar}
%\end{verbatim}
%    \end{quote}
% \end{desc}
%
% \StopEventually{
% }
%
% \section{Implementation}
%
% \subsection{Identification}
%
%    \begin{macrocode}
%<*package>
\NeedsTeXFormat{LaTeX2e}
\ProvidesPackage{aliascnt}%
  [2018/09/07 v1.5 Alias counters (HO)]%
%    \end{macrocode}
%
% \subsection{Create new alias counter}
%
%    \begin{macro}{\newaliascnt}
%    A new alias counter is set up by \cs{newaliascnt}.
%    The following properties are added for the new counter CNT:
%    \begin{description}
%    \item[\mdseries\cs{theH}\meta{CNT}:] Compatibility for \xpackage{hyperref}
%    \item[\mdseries\cs{AC@cnt@}\meta{CNT}:] Name of the referenced counter
%      in the definition.
%    \end{description}
%    \begin{macrocode}
\newcommand*{\newaliascnt}[2]{%
  \begingroup
    \def\AC@glet##1{%
      \global\expandafter\let\csname##1#1\expandafter\endcsname
        \csname##1#2\endcsname
    }%
    \@ifundefined{c@#2}{%
      \@nocounterr{#2}%
    }{%
      \expandafter\@ifdefinable\csname c@#1\endcsname{%
        \AC@glet{c@}%
        \AC@glet{the}%
        \AC@glet{theH}%
        \AC@glet{p@}%
        \expandafter\gdef\csname AC@cnt@#1\endcsname{#2}%
        \expandafter\gdef\csname cl@#1\expandafter\endcsname
        \expandafter{\csname cl@#2\endcsname}%
      }%
    }%
  \endgroup
}
%    \end{macrocode}
%    \end{macro}
%
%    \begin{macro}{\aliascntresetthe}
%    The \cs{the}\meta{CNT} macro is restored using the
%    main counter.
%    \begin{macrocode}
\newcommand*{\aliascntresetthe}[1]{%
  \@ifundefined{AC@cnt@#1}{%
    \PackageError{aliascnt}{%
      `#1' is not an alias counter%
    }\@ehc
  }{%
    \expandafter\let\csname the#1\expandafter\endcsname
      \csname the\csname AC@cnt@#1\endcsname\endcsname
  }%
}
%    \end{macrocode}
%    \end{macro}
%
% \subsection{Counter clear list}
%
%    The alias counters share the same register and clear list.
%    Therefore we must ensure that manipulations to the clear list
%    are done with the clear list macro of a real counter.
%    \begin{macro}{\AC@findrootcnt}
%    \cs{AC@findrootcnt} walks throught the aliasing relations
%    to find the base counter.
%    \begin{macrocode}
\newcommand*{\AC@findrootcnt}[1]{%
  \@ifundefined{AC@cnt@#1}{%
    #1%
  }{%
    \expandafter\AC@findrootcnt\csname AC@cnt@#1\endcsname
  }%
}
%    \end{macrocode}
%    \end{macro}
%
%    Clear lists are manipulated by \cs{@addtoreset} and
%    \cs{@removefromreset}. The latter one is provided by
%    the \xpackage{remreset} package (\cite{remreset} for old latex formats).
%
%    \begin{macro}{\AC@patch}
%    The same patch principle is applicable to both
%    \cs{@addtoreset} and \cs{@removefromreset}.
%    \begin{macrocode}
\def\AC@patch#1{%
  \expandafter\let\csname AC@org@#1reset\expandafter\endcsname
    \csname @#1reset\endcsname
  \expandafter\def\csname @#1reset\endcsname##1##2{%
    \csname AC@org@#1reset\endcsname{##1}{\AC@findrootcnt{##2}}%
  }%
}
%    \end{macrocode}
%    \end{macro}
%    If \xpackage{remreset} is not loaded we cannot delay
%    the patch to \cs{AtBeginDocument}, because \cs{@removefromreset}
%    can be called in between. Therefore we force the loading of
%    the package.
%    \begin{macrocode}
\ifx\@removefromreset\@undefined
  \RequirePackage{remreset}  
\fi
\AC@patch{addto}
\AC@patch{removefrom}
%    \end{macrocode}
%
%    \begin{macrocode}
%</package>
%    \end{macrocode}
%
% \section{Installation}
%
% \subsection{Download}
%
% \paragraph{Package.} This package is available on
% CTAN\footnote{\url{http://ctan.org/pkg/aliascnt}}:
% \begin{description}
% \item[\CTAN{macros/latex/contrib/oberdiek/aliascnt.dtx}] The source file.
% \item[\CTAN{macros/latex/contrib/oberdiek/aliascnt.pdf}] Documentation.
% \end{description}
%
%
% \paragraph{Bundle.} All the packages of the bundle `oberdiek'
% are also available in a TDS compliant ZIP archive. There
% the packages are already unpacked and the documentation files
% are generated. The files and directories obey the TDS standard.
% \begin{description}
% \item[\CTAN{install/macros/latex/contrib/oberdiek.tds.zip}]
% \end{description}
% \emph{TDS} refers to the standard ``A Directory Structure
% for \TeX\ Files'' (\CTAN{tds/tds.pdf}). Directories
% with \xfile{texmf} in their name are usually organized this way.
%
% \subsection{Bundle installation}
%
% \paragraph{Unpacking.} Unpack the \xfile{oberdiek.tds.zip} in the
% TDS tree (also known as \xfile{texmf} tree) of your choice.
% Example (linux):
% \begin{quote}
%   |unzip oberdiek.tds.zip -d ~/texmf|
% \end{quote}
%
% \paragraph{Script installation.}
% Check the directory \xfile{TDS:scripts/oberdiek/} for
% scripts that need further installation steps.
% Package \xpackage{attachfile2} comes with the Perl script
% \xfile{pdfatfi.pl} that should be installed in such a way
% that it can be called as \texttt{pdfatfi}.
% Example (linux):
% \begin{quote}
%   |chmod +x scripts/oberdiek/pdfatfi.pl|\\
%   |cp scripts/oberdiek/pdfatfi.pl /usr/local/bin/|
% \end{quote}
%
% \subsection{Package installation}
%
% \paragraph{Unpacking.} The \xfile{.dtx} file is a self-extracting
% \docstrip\ archive. The files are extracted by running the
% \xfile{.dtx} through \plainTeX:
% \begin{quote}
%   \verb|tex aliascnt.dtx|
% \end{quote}
%
% \paragraph{TDS.} Now the different files must be moved into
% the different directories in your installation TDS tree
% (also known as \xfile{texmf} tree):
% \begin{quote}
% \def\t{^^A
% \begin{tabular}{@{}>{\ttfamily}l@{ $\rightarrow$ }>{\ttfamily}l@{}}
%   aliascnt.sty & tex/latex/oberdiek/aliascnt.sty\\
%   aliascnt.pdf & doc/latex/oberdiek/aliascnt.pdf\\
%   aliascnt.dtx & source/latex/oberdiek/aliascnt.dtx\\
% \end{tabular}^^A
% }^^A
% \sbox0{\t}^^A
% \ifdim\wd0>\linewidth
%   \begingroup
%     \advance\linewidth by\leftmargin
%     \advance\linewidth by\rightmargin
%   \edef\x{\endgroup
%     \def\noexpand\lw{\the\linewidth}^^A
%   }\x
%   \def\lwbox{^^A
%     \leavevmode
%     \hbox to \linewidth{^^A
%       \kern-\leftmargin\relax
%       \hss
%       \usebox0
%       \hss
%       \kern-\rightmargin\relax
%     }^^A
%   }^^A
%   \ifdim\wd0>\lw
%     \sbox0{\small\t}^^A
%     \ifdim\wd0>\linewidth
%       \ifdim\wd0>\lw
%         \sbox0{\footnotesize\t}^^A
%         \ifdim\wd0>\linewidth
%           \ifdim\wd0>\lw
%             \sbox0{\scriptsize\t}^^A
%             \ifdim\wd0>\linewidth
%               \ifdim\wd0>\lw
%                 \sbox0{\tiny\t}^^A
%                 \ifdim\wd0>\linewidth
%                   \lwbox
%                 \else
%                   \usebox0
%                 \fi
%               \else
%                 \lwbox
%               \fi
%             \else
%               \usebox0
%             \fi
%           \else
%             \lwbox
%           \fi
%         \else
%           \usebox0
%         \fi
%       \else
%         \lwbox
%       \fi
%     \else
%       \usebox0
%     \fi
%   \else
%     \lwbox
%   \fi
% \else
%   \usebox0
% \fi
% \end{quote}
% If you have a \xfile{docstrip.cfg} that configures and enables \docstrip's
% TDS installing feature, then some files can already be in the right
% place, see the documentation of \docstrip.
%
% \subsection{Refresh file name databases}
%
% If your \TeX~distribution
% (\teTeX, \mikTeX, \dots) relies on file name databases, you must refresh
% these. For example, \teTeX\ users run \verb|texhash| or
% \verb|mktexlsr|.
%
% \subsection{Some details for the interested}
%
% \paragraph{Attached source.}
%
% The PDF documentation on CTAN also includes the
% \xfile{.dtx} source file. It can be extracted by
% AcrobatReader 6 or higher. Another option is \textsf{pdftk},
% e.g. unpack the file into the current directory:
% \begin{quote}
%   \verb|pdftk aliascnt.pdf unpack_files output .|
% \end{quote}
%
% \paragraph{Unpacking with \LaTeX.}
% The \xfile{.dtx} chooses its action depending on the format:
% \begin{description}
% \item[\plainTeX:] Run \docstrip\ and extract the files.
% \item[\LaTeX:] Generate the documentation.
% \end{description}
% If you insist on using \LaTeX\ for \docstrip\ (really,
% \docstrip\ does not need \LaTeX), then inform the autodetect routine
% about your intention:
% \begin{quote}
%   \verb|latex \let\install=y% \iffalse meta-comment
%
% File: aliascnt.dtx
% Version: 2009/09/08 v1.3
% Info: Alias counters
%
% Copyright (C) 2006, 2009 by
%    Heiko Oberdiek <heiko.oberdiek at googlemail.com>
%
% This work may be distributed and/or modified under the
% conditions of the LaTeX Project Public License, either
% version 1.3c of this license or (at your option) any later
% version. This version of this license is in
%    http://www.latex-project.org/lppl/lppl-1-3c.txt
% and the latest version of this license is in
%    http://www.latex-project.org/lppl.txt
% and version 1.3 or later is part of all distributions of
% LaTeX version 2005/12/01 or later.
%
% This work has the LPPL maintenance status "maintained".
%
% This Current Maintainer of this work is Heiko Oberdiek.
%
% This work consists of the main source file aliascnt.dtx
% and the derived files
%    aliascnt.sty, aliascnt.pdf, aliascnt.ins, aliascnt.drv.
%
% Distribution:
%    CTAN:macros/latex/contrib/oberdiek/aliascnt.dtx
%    CTAN:macros/latex/contrib/oberdiek/aliascnt.pdf
%
% Unpacking:
%    (a) If aliascnt.ins is present:
%           tex aliascnt.ins
%    (b) Without aliascnt.ins:
%           tex aliascnt.dtx
%    (c) If you insist on using LaTeX
%           latex \let\install=y% \iffalse meta-comment
%
% File: aliascnt.dtx
% Version: 2009/09/08 v1.3
% Info: Alias counters
%
% Copyright (C) 2006, 2009 by
%    Heiko Oberdiek <heiko.oberdiek at googlemail.com>
%
% This work may be distributed and/or modified under the
% conditions of the LaTeX Project Public License, either
% version 1.3c of this license or (at your option) any later
% version. This version of this license is in
%    http://www.latex-project.org/lppl/lppl-1-3c.txt
% and the latest version of this license is in
%    http://www.latex-project.org/lppl.txt
% and version 1.3 or later is part of all distributions of
% LaTeX version 2005/12/01 or later.
%
% This work has the LPPL maintenance status "maintained".
%
% This Current Maintainer of this work is Heiko Oberdiek.
%
% This work consists of the main source file aliascnt.dtx
% and the derived files
%    aliascnt.sty, aliascnt.pdf, aliascnt.ins, aliascnt.drv.
%
% Distribution:
%    CTAN:macros/latex/contrib/oberdiek/aliascnt.dtx
%    CTAN:macros/latex/contrib/oberdiek/aliascnt.pdf
%
% Unpacking:
%    (a) If aliascnt.ins is present:
%           tex aliascnt.ins
%    (b) Without aliascnt.ins:
%           tex aliascnt.dtx
%    (c) If you insist on using LaTeX
%           latex \let\install=y% \iffalse meta-comment
%
% File: aliascnt.dtx
% Version: 2009/09/08 v1.3
% Info: Alias counters
%
% Copyright (C) 2006, 2009 by
%    Heiko Oberdiek <heiko.oberdiek at googlemail.com>
%
% This work may be distributed and/or modified under the
% conditions of the LaTeX Project Public License, either
% version 1.3c of this license or (at your option) any later
% version. This version of this license is in
%    http://www.latex-project.org/lppl/lppl-1-3c.txt
% and the latest version of this license is in
%    http://www.latex-project.org/lppl.txt
% and version 1.3 or later is part of all distributions of
% LaTeX version 2005/12/01 or later.
%
% This work has the LPPL maintenance status "maintained".
%
% This Current Maintainer of this work is Heiko Oberdiek.
%
% This work consists of the main source file aliascnt.dtx
% and the derived files
%    aliascnt.sty, aliascnt.pdf, aliascnt.ins, aliascnt.drv.
%
% Distribution:
%    CTAN:macros/latex/contrib/oberdiek/aliascnt.dtx
%    CTAN:macros/latex/contrib/oberdiek/aliascnt.pdf
%
% Unpacking:
%    (a) If aliascnt.ins is present:
%           tex aliascnt.ins
%    (b) Without aliascnt.ins:
%           tex aliascnt.dtx
%    (c) If you insist on using LaTeX
%           latex \let\install=y\input{aliascnt.dtx}
%        (quote the arguments according to the demands of your shell)
%
% Documentation:
%    (a) If aliascnt.drv is present:
%           latex aliascnt.drv
%    (b) Without aliascnt.drv:
%           latex aliascnt.dtx; ...
%    The class ltxdoc loads the configuration file ltxdoc.cfg
%    if available. Here you can specify further options, e.g.
%    use A4 as paper format:
%       \PassOptionsToClass{a4paper}{article}
%
%    Programm calls to get the documentation (example):
%       pdflatex aliascnt.dtx
%       makeindex -s gind.ist aliascnt.idx
%       pdflatex aliascnt.dtx
%       makeindex -s gind.ist aliascnt.idx
%       pdflatex aliascnt.dtx
%
% Installation:
%    TDS:tex/latex/oberdiek/aliascnt.sty
%    TDS:doc/latex/oberdiek/aliascnt.pdf
%    TDS:source/latex/oberdiek/aliascnt.dtx
%
%<*ignore>
\begingroup
  \catcode123=1 %
  \catcode125=2 %
  \def\x{LaTeX2e}%
\expandafter\endgroup
\ifcase 0\ifx\install y1\fi\expandafter
         \ifx\csname processbatchFile\endcsname\relax\else1\fi
         \ifx\fmtname\x\else 1\fi\relax
\else\csname fi\endcsname
%</ignore>
%<*install>
\input docstrip.tex
\Msg{************************************************************************}
\Msg{* Installation}
\Msg{* Package: aliascnt 2009/09/08 v1.3 Alias counters (HO)}
\Msg{************************************************************************}

\keepsilent
\askforoverwritefalse

\let\MetaPrefix\relax
\preamble

This is a generated file.

Project: aliascnt
Version: 2009/09/08 v1.3

Copyright (C) 2006, 2009 by
   Heiko Oberdiek <heiko.oberdiek at googlemail.com>

This work may be distributed and/or modified under the
conditions of the LaTeX Project Public License, either
version 1.3c of this license or (at your option) any later
version. This version of this license is in
   http://www.latex-project.org/lppl/lppl-1-3c.txt
and the latest version of this license is in
   http://www.latex-project.org/lppl.txt
and version 1.3 or later is part of all distributions of
LaTeX version 2005/12/01 or later.

This work has the LPPL maintenance status "maintained".

This Current Maintainer of this work is Heiko Oberdiek.

This work consists of the main source file aliascnt.dtx
and the derived files
   aliascnt.sty, aliascnt.pdf, aliascnt.ins, aliascnt.drv.

\endpreamble
\let\MetaPrefix\DoubleperCent

\generate{%
  \file{aliascnt.ins}{\from{aliascnt.dtx}{install}}%
  \file{aliascnt.drv}{\from{aliascnt.dtx}{driver}}%
  \usedir{tex/latex/oberdiek}%
  \file{aliascnt.sty}{\from{aliascnt.dtx}{package}}%
  \nopreamble
  \nopostamble
  \usedir{source/latex/oberdiek/catalogue}%
  \file{aliascnt.xml}{\from{aliascnt.dtx}{catalogue}}%
}

\catcode32=13\relax% active space
\let =\space%
\Msg{************************************************************************}
\Msg{*}
\Msg{* To finish the installation you have to move the following}
\Msg{* file into a directory searched by TeX:}
\Msg{*}
\Msg{*     aliascnt.sty}
\Msg{*}
\Msg{* To produce the documentation run the file `aliascnt.drv'}
\Msg{* through LaTeX.}
\Msg{*}
\Msg{* Happy TeXing!}
\Msg{*}
\Msg{************************************************************************}

\endbatchfile
%</install>
%<*ignore>
\fi
%</ignore>
%<*driver>
\NeedsTeXFormat{LaTeX2e}
\ProvidesFile{aliascnt.drv}%
  [2009/09/08 v1.3 Alias counters (HO)]%
\documentclass{ltxdoc}
\usepackage{holtxdoc}[2011/11/22]
\begin{document}
  \DocInput{aliascnt.dtx}%
\end{document}
%</driver>
% \fi
%
% \CheckSum{78}
%
% \CharacterTable
%  {Upper-case    \A\B\C\D\E\F\G\H\I\J\K\L\M\N\O\P\Q\R\S\T\U\V\W\X\Y\Z
%   Lower-case    \a\b\c\d\e\f\g\h\i\j\k\l\m\n\o\p\q\r\s\t\u\v\w\x\y\z
%   Digits        \0\1\2\3\4\5\6\7\8\9
%   Exclamation   \!     Double quote  \"     Hash (number) \#
%   Dollar        \$     Percent       \%     Ampersand     \&
%   Acute accent  \'     Left paren    \(     Right paren   \)
%   Asterisk      \*     Plus          \+     Comma         \,
%   Minus         \-     Point         \.     Solidus       \/
%   Colon         \:     Semicolon     \;     Less than     \<
%   Equals        \=     Greater than  \>     Question mark \?
%   Commercial at \@     Left bracket  \[     Backslash     \\
%   Right bracket \]     Circumflex    \^     Underscore    \_
%   Grave accent  \`     Left brace    \{     Vertical bar  \|
%   Right brace   \}     Tilde         \~}
%
% \GetFileInfo{aliascnt.drv}
%
% \title{The \xpackage{aliascnt} package}
% \date{2009/09/08 v1.3}
% \author{Heiko Oberdiek\\\xemail{heiko.oberdiek at googlemail.com}}
%
% \maketitle
%
% \begin{abstract}
% Package \xpackage{aliascnt} introduces \emph{alias counters} that
% share the same counter register and clear list.
% \end{abstract}
%
% \tableofcontents
%
% \section{User interface}
%
% \subsection{Introduction}
%
% There are features that rely on the name of counters. For
% example, \xpackage{hyperref}'s \cs{autoref} indirectly uses
% the counter name to determine which label text it puts in front
% of the reference number (\cite{hyperref}).
% In some circumstances this fail: several theorem environments
% are defined by \cs{newtheorem} that share the same counter.
%
% \subsection{Syntax}
%
% Macro names in user land contain the package name
% \texttt{aliascnt} in order to prevent name clashes.
%
% \newenvironment{desc}{^^A
%   \list{}{^^A
%     \setlength{\labelwidth}{0pt}^^A
%     \setlength{\itemindent}{-.5\marginparwidth}^^A
%     \setlength{\leftmargin}{0pt}^^A
%     \let\makelabel\desclabel
%   }^^A
% }{^^A
%   \endlist
% }
% \newcommand*{\desclabel}[1]{^^A
%   \hspace{\labelsep}^^A
%   \normalfont
%   #1^^A
% }
% \newcommand*{\itemcs}[2]{^^A
%   \item[^^A
%      \expandafter\SpecialUsageIndex\csname #1\endcsname
%      {\cs{#1}#2}^^A
%   ]\mbox{}\\*[.5ex]^^A
%   \ignorespaces
% }
% \begin{desc}
% \itemcs{newaliascnt}{\marg{ALIASCNT}\marg{BASECNT}}
%    An alias counter ALIASCNT is created that does not allocate
%    a new \TeX\ counter register. It shares the count register and
%    the clear list with counter BASECNT. If the value of either
%    the two registers is changed, the changes affects both.
% \itemcs{aliascntresetthe}{\marg{ALIASCNT}}
%    This fixes a problem with \cs{newtheorem} if it
%    is fooled by an alias counter with the same name:
%    \begin{quote}
%\begin{verbatim}
%\newtheorem{foo}{Foo}% counter "foo"
%\newaliascnt{bar}{foo}% alias counter "bar"
%\newtheorem{bar}[bar]{Bar}
%\aliascntresetthe{bar}
%\end{verbatim}
%    \end{quote}
% \end{desc}
%
% \StopEventually{
% }
%
% \section{Implementation}
%
% \subsection{Identification}
%
%    \begin{macrocode}
%<*package>
\NeedsTeXFormat{LaTeX2e}
\ProvidesPackage{aliascnt}%
  [2009/09/08 v1.3 Alias counters (HO)]%
%    \end{macrocode}
%
% \subsection{Create new alias counter}
%
%    \begin{macro}{\newaliascnt}
%    A new alias counter is set up by \cs{newaliascnt}.
%    The following properties are added for the new counter CNT:
%    \begin{description}
%    \item[\mdseries\cs{theH}\meta{CNT}:] Compatibility for \xpackage{hyperref}
%    \item[\mdseries\cs{AC@cnt@}\meta{CNT}:] Name of the referenced counter
%      in the definition.
%    \end{description}
%    \begin{macrocode}
\newcommand*{\newaliascnt}[2]{%
  \begingroup
    \def\AC@glet##1{%
      \global\expandafter\let\csname##1#1\expandafter\endcsname
        \csname##1#2\endcsname
    }%
    \@ifundefined{c@#2}{%
      \@nocounterr{#2}%
    }{%
      \expandafter\@ifdefinable\csname c@#1\endcsname{%
        \AC@glet{c@}%
        \AC@glet{the}%
        \AC@glet{theH}%
        \AC@glet{p@}%
        \expandafter\gdef\csname AC@cnt@#1\endcsname{#2}%
        \expandafter\gdef\csname cl@#1\expandafter\endcsname
        \expandafter{\csname cl@#2\endcsname}%
      }%
    }%
  \endgroup
}
%    \end{macrocode}
%    \end{macro}
%
%    \begin{macro}{\aliascntresetthe}
%    The \cs{the}\meta{CNT} macro is restored using the
%    main counter.
%    \begin{macrocode}
\newcommand*{\aliascntresetthe}[1]{%
  \@ifundefined{AC@cnt@#1}{%
    \PackageError{aliascnt}{%
      `#1' is not an alias counter%
    }\@ehc
  }{%
    \expandafter\let\csname the#1\expandafter\endcsname
      \csname the\csname AC@cnt@#1\endcsname\endcsname
  }%
}
%    \end{macrocode}
%    \end{macro}
%
% \subsection{Counter clear list}
%
%    The alias counters share the same register and clear list.
%    Therefore we must ensure that manipulations to the clear list
%    are done with the clear list macro of a real counter.
%    \begin{macro}{\AC@findrootcnt}
%    \cs{AC@findrootcnt} walks throught the aliasing relations
%    to find the base counter.
%    \begin{macrocode}
\newcommand*{\AC@findrootcnt}[1]{%
  \@ifundefined{AC@cnt@#1}{%
    #1%
  }{%
    \expandafter\AC@findrootcnt\csname AC@cnt@#1\endcsname
  }%
}
%    \end{macrocode}
%    \end{macro}
%
%    Clear lists are manipulated by \cs{@addtoreset} and
%    \cs{@removefromreset}. The latter one is provided by
%    the \xpackage{remreset} package (\cite{remreset}).
%
%    \begin{macro}{\AC@patch}
%    The same patch principle is applicable to both
%    \cs{@addtoreset} and \cs{@removefromreset}.
%    \begin{macrocode}
\def\AC@patch#1{%
  \expandafter\let\csname AC@org@#1reset\expandafter\endcsname
    \csname @#1reset\endcsname
  \expandafter\def\csname @#1reset\endcsname##1##2{%
    \csname AC@org@#1reset\endcsname{##1}{\AC@findrootcnt{##2}}%
  }%
}
%    \end{macrocode}
%    \end{macro}
%    If \xpackage{remreset} is not loaded we cannot delay
%    the patch to \cs{AtBeginDocumen}, because \cs{@removefromreset}
%    can be called in between. Therefore we force the loading of
%    the package.
%    \begin{macrocode}
\RequirePackage{remreset}
\AC@patch{addto}
\AC@patch{removefrom}
%    \end{macrocode}
%
%    \begin{macrocode}
%</package>
%    \end{macrocode}
%
% \section{Installation}
%
% \subsection{Download}
%
% \paragraph{Package.} This package is available on
% CTAN\footnote{\url{ftp://ftp.ctan.org/tex-archive/}}:
% \begin{description}
% \item[\CTAN{macros/latex/contrib/oberdiek/aliascnt.dtx}] The source file.
% \item[\CTAN{macros/latex/contrib/oberdiek/aliascnt.pdf}] Documentation.
% \end{description}
%
%
% \paragraph{Bundle.} All the packages of the bundle `oberdiek'
% are also available in a TDS compliant ZIP archive. There
% the packages are already unpacked and the documentation files
% are generated. The files and directories obey the TDS standard.
% \begin{description}
% \item[\CTAN{install/macros/latex/contrib/oberdiek.tds.zip}]
% \end{description}
% \emph{TDS} refers to the standard ``A Directory Structure
% for \TeX\ Files'' (\CTAN{tds/tds.pdf}). Directories
% with \xfile{texmf} in their name are usually organized this way.
%
% \subsection{Bundle installation}
%
% \paragraph{Unpacking.} Unpack the \xfile{oberdiek.tds.zip} in the
% TDS tree (also known as \xfile{texmf} tree) of your choice.
% Example (linux):
% \begin{quote}
%   |unzip oberdiek.tds.zip -d ~/texmf|
% \end{quote}
%
% \paragraph{Script installation.}
% Check the directory \xfile{TDS:scripts/oberdiek/} for
% scripts that need further installation steps.
% Package \xpackage{attachfile2} comes with the Perl script
% \xfile{pdfatfi.pl} that should be installed in such a way
% that it can be called as \texttt{pdfatfi}.
% Example (linux):
% \begin{quote}
%   |chmod +x scripts/oberdiek/pdfatfi.pl|\\
%   |cp scripts/oberdiek/pdfatfi.pl /usr/local/bin/|
% \end{quote}
%
% \subsection{Package installation}
%
% \paragraph{Unpacking.} The \xfile{.dtx} file is a self-extracting
% \docstrip\ archive. The files are extracted by running the
% \xfile{.dtx} through \plainTeX:
% \begin{quote}
%   \verb|tex aliascnt.dtx|
% \end{quote}
%
% \paragraph{TDS.} Now the different files must be moved into
% the different directories in your installation TDS tree
% (also known as \xfile{texmf} tree):
% \begin{quote}
% \def\t{^^A
% \begin{tabular}{@{}>{\ttfamily}l@{ $\rightarrow$ }>{\ttfamily}l@{}}
%   aliascnt.sty & tex/latex/oberdiek/aliascnt.sty\\
%   aliascnt.pdf & doc/latex/oberdiek/aliascnt.pdf\\
%   aliascnt.dtx & source/latex/oberdiek/aliascnt.dtx\\
% \end{tabular}^^A
% }^^A
% \sbox0{\t}^^A
% \ifdim\wd0>\linewidth
%   \begingroup
%     \advance\linewidth by\leftmargin
%     \advance\linewidth by\rightmargin
%   \edef\x{\endgroup
%     \def\noexpand\lw{\the\linewidth}^^A
%   }\x
%   \def\lwbox{^^A
%     \leavevmode
%     \hbox to \linewidth{^^A
%       \kern-\leftmargin\relax
%       \hss
%       \usebox0
%       \hss
%       \kern-\rightmargin\relax
%     }^^A
%   }^^A
%   \ifdim\wd0>\lw
%     \sbox0{\small\t}^^A
%     \ifdim\wd0>\linewidth
%       \ifdim\wd0>\lw
%         \sbox0{\footnotesize\t}^^A
%         \ifdim\wd0>\linewidth
%           \ifdim\wd0>\lw
%             \sbox0{\scriptsize\t}^^A
%             \ifdim\wd0>\linewidth
%               \ifdim\wd0>\lw
%                 \sbox0{\tiny\t}^^A
%                 \ifdim\wd0>\linewidth
%                   \lwbox
%                 \else
%                   \usebox0
%                 \fi
%               \else
%                 \lwbox
%               \fi
%             \else
%               \usebox0
%             \fi
%           \else
%             \lwbox
%           \fi
%         \else
%           \usebox0
%         \fi
%       \else
%         \lwbox
%       \fi
%     \else
%       \usebox0
%     \fi
%   \else
%     \lwbox
%   \fi
% \else
%   \usebox0
% \fi
% \end{quote}
% If you have a \xfile{docstrip.cfg} that configures and enables \docstrip's
% TDS installing feature, then some files can already be in the right
% place, see the documentation of \docstrip.
%
% \subsection{Refresh file name databases}
%
% If your \TeX~distribution
% (\teTeX, \mikTeX, \dots) relies on file name databases, you must refresh
% these. For example, \teTeX\ users run \verb|texhash| or
% \verb|mktexlsr|.
%
% \subsection{Some details for the interested}
%
% \paragraph{Attached source.}
%
% The PDF documentation on CTAN also includes the
% \xfile{.dtx} source file. It can be extracted by
% AcrobatReader 6 or higher. Another option is \textsf{pdftk},
% e.g. unpack the file into the current directory:
% \begin{quote}
%   \verb|pdftk aliascnt.pdf unpack_files output .|
% \end{quote}
%
% \paragraph{Unpacking with \LaTeX.}
% The \xfile{.dtx} chooses its action depending on the format:
% \begin{description}
% \item[\plainTeX:] Run \docstrip\ and extract the files.
% \item[\LaTeX:] Generate the documentation.
% \end{description}
% If you insist on using \LaTeX\ for \docstrip\ (really,
% \docstrip\ does not need \LaTeX), then inform the autodetect routine
% about your intention:
% \begin{quote}
%   \verb|latex \let\install=y\input{aliascnt.dtx}|
% \end{quote}
% Do not forget to quote the argument according to the demands
% of your shell.
%
% \paragraph{Generating the documentation.}
% You can use both the \xfile{.dtx} or the \xfile{.drv} to generate
% the documentation. The process can be configured by the
% configuration file \xfile{ltxdoc.cfg}. For instance, put this
% line into this file, if you want to have A4 as paper format:
% \begin{quote}
%   \verb|\PassOptionsToClass{a4paper}{article}|
% \end{quote}
% An example follows how to generate the
% documentation with pdf\LaTeX:
% \begin{quote}
%\begin{verbatim}
%pdflatex aliascnt.dtx
%makeindex -s gind.ist aliascnt.idx
%pdflatex aliascnt.dtx
%makeindex -s gind.ist aliascnt.idx
%pdflatex aliascnt.dtx
%\end{verbatim}
% \end{quote}
%
% \section{Catalogue}
%
% The following XML file can be used as source for the
% \href{http://mirror.ctan.org/help/Catalogue/catalogue.html}{\TeX\ Catalogue}.
% The elements \texttt{caption} and \texttt{description} are imported
% from the original XML file from the Catalogue.
% The name of the XML file in the Catalogue is \xfile{aliascnt.xml}.
%    \begin{macrocode}
%<*catalogue>
<?xml version='1.0' encoding='us-ascii'?>
<!DOCTYPE entry SYSTEM 'catalogue.dtd'>
<entry datestamp='$Date$' modifier='$Author$' id='aliascnt'>
  <name>aliascnt</name>
  <caption>Alias counters.</caption>
  <authorref id='auth:oberdiek'/>
  <copyright owner='Heiko Oberdiek' year='2006,2009'/>
  <license type='lppl1.3'/>
  <version number='1.3'/>
  <description>
    This package introduces aliases for counters, that
    share the same counter register and clear list.
    <p/>
    The package is part of the <xref refid='oberdiek'>oberdiek</xref>
    bundle.
  </description>
  <documentation details='Package documentation'
      href='ctan:/macros/latex/contrib/oberdiek/aliascnt.pdf'/>
  <ctan file='true' path='/macros/latex/contrib/oberdiek/aliascnt.dtx'/>
  <miktex location='oberdiek'/>
  <texlive location='oberdiek'/>
  <install path='/macros/latex/contrib/oberdiek/oberdiek.tds.zip'/>
</entry>
%</catalogue>
%    \end{macrocode}
%
% \section{Acknowledgement}
%
% \begin{description}
% \item[Ulrich Schwarz:] The package is based on his draft for
%   ``Die \TeX nische Kom\"odie'', see \cite{schwarz}.
% \end{description}
%
% \begin{thebibliography}{9}
%
% \bibitem{schwarz}
%   Ulrich Schwarz:
%   \textit{Was hinten herauskommt z\"ahlt: Counter Aliasing in \LaTeX},
%   \textit{Die \TeX nische Kom\"odie}, 3/2006, pages 8--14, Juli 2006.
%
% \bibitem{remreset}
%   David Carlisle: \textit{The \xpackage{remreset} package};
%   1997/09/28;
%   \CTAN{macros/latex/contrib/carlisle/remreset.sty}.
%
% \bibitem{hyperref}
%   Sebastian Rahtz, Heiko Oberdiek:
%   \textit{The \xpackage{hyperref} package};
%   2006/08/16 v6.75c;
%   \CTAN{macros/latex/contrib/hyperref/}.
%
% \end{thebibliography}
%
% \begin{History}
%   \begin{Version}{2006/02/20 v1.0}
%   \item
%     First version.
%   \end{Version}
%   \begin{Version}{2006/08/16 v1.1}
%   \item
%     Update of bibliography.
%   \end{Version}
%   \begin{Version}{2006/09/25 v1.2}
%   \item
%     Bug fix (\cs{aliascntresetthe}).
%   \end{Version}
%   \begin{Version}{2009/09/08 v1.3}
%   \item
%     Bug fix of \cs{@ifdefinable}'s use (thanks to Uwe L\"uck).
%   \end{Version}
% \end{History}
%
% \PrintIndex
%
% \Finale
\endinput

%        (quote the arguments according to the demands of your shell)
%
% Documentation:
%    (a) If aliascnt.drv is present:
%           latex aliascnt.drv
%    (b) Without aliascnt.drv:
%           latex aliascnt.dtx; ...
%    The class ltxdoc loads the configuration file ltxdoc.cfg
%    if available. Here you can specify further options, e.g.
%    use A4 as paper format:
%       \PassOptionsToClass{a4paper}{article}
%
%    Programm calls to get the documentation (example):
%       pdflatex aliascnt.dtx
%       makeindex -s gind.ist aliascnt.idx
%       pdflatex aliascnt.dtx
%       makeindex -s gind.ist aliascnt.idx
%       pdflatex aliascnt.dtx
%
% Installation:
%    TDS:tex/latex/oberdiek/aliascnt.sty
%    TDS:doc/latex/oberdiek/aliascnt.pdf
%    TDS:source/latex/oberdiek/aliascnt.dtx
%
%<*ignore>
\begingroup
  \catcode123=1 %
  \catcode125=2 %
  \def\x{LaTeX2e}%
\expandafter\endgroup
\ifcase 0\ifx\install y1\fi\expandafter
         \ifx\csname processbatchFile\endcsname\relax\else1\fi
         \ifx\fmtname\x\else 1\fi\relax
\else\csname fi\endcsname
%</ignore>
%<*install>
\input docstrip.tex
\Msg{************************************************************************}
\Msg{* Installation}
\Msg{* Package: aliascnt 2009/09/08 v1.3 Alias counters (HO)}
\Msg{************************************************************************}

\keepsilent
\askforoverwritefalse

\let\MetaPrefix\relax
\preamble

This is a generated file.

Project: aliascnt
Version: 2009/09/08 v1.3

Copyright (C) 2006, 2009 by
   Heiko Oberdiek <heiko.oberdiek at googlemail.com>

This work may be distributed and/or modified under the
conditions of the LaTeX Project Public License, either
version 1.3c of this license or (at your option) any later
version. This version of this license is in
   http://www.latex-project.org/lppl/lppl-1-3c.txt
and the latest version of this license is in
   http://www.latex-project.org/lppl.txt
and version 1.3 or later is part of all distributions of
LaTeX version 2005/12/01 or later.

This work has the LPPL maintenance status "maintained".

This Current Maintainer of this work is Heiko Oberdiek.

This work consists of the main source file aliascnt.dtx
and the derived files
   aliascnt.sty, aliascnt.pdf, aliascnt.ins, aliascnt.drv.

\endpreamble
\let\MetaPrefix\DoubleperCent

\generate{%
  \file{aliascnt.ins}{\from{aliascnt.dtx}{install}}%
  \file{aliascnt.drv}{\from{aliascnt.dtx}{driver}}%
  \usedir{tex/latex/oberdiek}%
  \file{aliascnt.sty}{\from{aliascnt.dtx}{package}}%
  \nopreamble
  \nopostamble
  \usedir{source/latex/oberdiek/catalogue}%
  \file{aliascnt.xml}{\from{aliascnt.dtx}{catalogue}}%
}

\catcode32=13\relax% active space
\let =\space%
\Msg{************************************************************************}
\Msg{*}
\Msg{* To finish the installation you have to move the following}
\Msg{* file into a directory searched by TeX:}
\Msg{*}
\Msg{*     aliascnt.sty}
\Msg{*}
\Msg{* To produce the documentation run the file `aliascnt.drv'}
\Msg{* through LaTeX.}
\Msg{*}
\Msg{* Happy TeXing!}
\Msg{*}
\Msg{************************************************************************}

\endbatchfile
%</install>
%<*ignore>
\fi
%</ignore>
%<*driver>
\NeedsTeXFormat{LaTeX2e}
\ProvidesFile{aliascnt.drv}%
  [2009/09/08 v1.3 Alias counters (HO)]%
\documentclass{ltxdoc}
\usepackage{holtxdoc}[2011/11/22]
\begin{document}
  \DocInput{aliascnt.dtx}%
\end{document}
%</driver>
% \fi
%
% \CheckSum{78}
%
% \CharacterTable
%  {Upper-case    \A\B\C\D\E\F\G\H\I\J\K\L\M\N\O\P\Q\R\S\T\U\V\W\X\Y\Z
%   Lower-case    \a\b\c\d\e\f\g\h\i\j\k\l\m\n\o\p\q\r\s\t\u\v\w\x\y\z
%   Digits        \0\1\2\3\4\5\6\7\8\9
%   Exclamation   \!     Double quote  \"     Hash (number) \#
%   Dollar        \$     Percent       \%     Ampersand     \&
%   Acute accent  \'     Left paren    \(     Right paren   \)
%   Asterisk      \*     Plus          \+     Comma         \,
%   Minus         \-     Point         \.     Solidus       \/
%   Colon         \:     Semicolon     \;     Less than     \<
%   Equals        \=     Greater than  \>     Question mark \?
%   Commercial at \@     Left bracket  \[     Backslash     \\
%   Right bracket \]     Circumflex    \^     Underscore    \_
%   Grave accent  \`     Left brace    \{     Vertical bar  \|
%   Right brace   \}     Tilde         \~}
%
% \GetFileInfo{aliascnt.drv}
%
% \title{The \xpackage{aliascnt} package}
% \date{2009/09/08 v1.3}
% \author{Heiko Oberdiek\\\xemail{heiko.oberdiek at googlemail.com}}
%
% \maketitle
%
% \begin{abstract}
% Package \xpackage{aliascnt} introduces \emph{alias counters} that
% share the same counter register and clear list.
% \end{abstract}
%
% \tableofcontents
%
% \section{User interface}
%
% \subsection{Introduction}
%
% There are features that rely on the name of counters. For
% example, \xpackage{hyperref}'s \cs{autoref} indirectly uses
% the counter name to determine which label text it puts in front
% of the reference number (\cite{hyperref}).
% In some circumstances this fail: several theorem environments
% are defined by \cs{newtheorem} that share the same counter.
%
% \subsection{Syntax}
%
% Macro names in user land contain the package name
% \texttt{aliascnt} in order to prevent name clashes.
%
% \newenvironment{desc}{^^A
%   \list{}{^^A
%     \setlength{\labelwidth}{0pt}^^A
%     \setlength{\itemindent}{-.5\marginparwidth}^^A
%     \setlength{\leftmargin}{0pt}^^A
%     \let\makelabel\desclabel
%   }^^A
% }{^^A
%   \endlist
% }
% \newcommand*{\desclabel}[1]{^^A
%   \hspace{\labelsep}^^A
%   \normalfont
%   #1^^A
% }
% \newcommand*{\itemcs}[2]{^^A
%   \item[^^A
%      \expandafter\SpecialUsageIndex\csname #1\endcsname
%      {\cs{#1}#2}^^A
%   ]\mbox{}\\*[.5ex]^^A
%   \ignorespaces
% }
% \begin{desc}
% \itemcs{newaliascnt}{\marg{ALIASCNT}\marg{BASECNT}}
%    An alias counter ALIASCNT is created that does not allocate
%    a new \TeX\ counter register. It shares the count register and
%    the clear list with counter BASECNT. If the value of either
%    the two registers is changed, the changes affects both.
% \itemcs{aliascntresetthe}{\marg{ALIASCNT}}
%    This fixes a problem with \cs{newtheorem} if it
%    is fooled by an alias counter with the same name:
%    \begin{quote}
%\begin{verbatim}
%\newtheorem{foo}{Foo}% counter "foo"
%\newaliascnt{bar}{foo}% alias counter "bar"
%\newtheorem{bar}[bar]{Bar}
%\aliascntresetthe{bar}
%\end{verbatim}
%    \end{quote}
% \end{desc}
%
% \StopEventually{
% }
%
% \section{Implementation}
%
% \subsection{Identification}
%
%    \begin{macrocode}
%<*package>
\NeedsTeXFormat{LaTeX2e}
\ProvidesPackage{aliascnt}%
  [2009/09/08 v1.3 Alias counters (HO)]%
%    \end{macrocode}
%
% \subsection{Create new alias counter}
%
%    \begin{macro}{\newaliascnt}
%    A new alias counter is set up by \cs{newaliascnt}.
%    The following properties are added for the new counter CNT:
%    \begin{description}
%    \item[\mdseries\cs{theH}\meta{CNT}:] Compatibility for \xpackage{hyperref}
%    \item[\mdseries\cs{AC@cnt@}\meta{CNT}:] Name of the referenced counter
%      in the definition.
%    \end{description}
%    \begin{macrocode}
\newcommand*{\newaliascnt}[2]{%
  \begingroup
    \def\AC@glet##1{%
      \global\expandafter\let\csname##1#1\expandafter\endcsname
        \csname##1#2\endcsname
    }%
    \@ifundefined{c@#2}{%
      \@nocounterr{#2}%
    }{%
      \expandafter\@ifdefinable\csname c@#1\endcsname{%
        \AC@glet{c@}%
        \AC@glet{the}%
        \AC@glet{theH}%
        \AC@glet{p@}%
        \expandafter\gdef\csname AC@cnt@#1\endcsname{#2}%
        \expandafter\gdef\csname cl@#1\expandafter\endcsname
        \expandafter{\csname cl@#2\endcsname}%
      }%
    }%
  \endgroup
}
%    \end{macrocode}
%    \end{macro}
%
%    \begin{macro}{\aliascntresetthe}
%    The \cs{the}\meta{CNT} macro is restored using the
%    main counter.
%    \begin{macrocode}
\newcommand*{\aliascntresetthe}[1]{%
  \@ifundefined{AC@cnt@#1}{%
    \PackageError{aliascnt}{%
      `#1' is not an alias counter%
    }\@ehc
  }{%
    \expandafter\let\csname the#1\expandafter\endcsname
      \csname the\csname AC@cnt@#1\endcsname\endcsname
  }%
}
%    \end{macrocode}
%    \end{macro}
%
% \subsection{Counter clear list}
%
%    The alias counters share the same register and clear list.
%    Therefore we must ensure that manipulations to the clear list
%    are done with the clear list macro of a real counter.
%    \begin{macro}{\AC@findrootcnt}
%    \cs{AC@findrootcnt} walks throught the aliasing relations
%    to find the base counter.
%    \begin{macrocode}
\newcommand*{\AC@findrootcnt}[1]{%
  \@ifundefined{AC@cnt@#1}{%
    #1%
  }{%
    \expandafter\AC@findrootcnt\csname AC@cnt@#1\endcsname
  }%
}
%    \end{macrocode}
%    \end{macro}
%
%    Clear lists are manipulated by \cs{@addtoreset} and
%    \cs{@removefromreset}. The latter one is provided by
%    the \xpackage{remreset} package (\cite{remreset}).
%
%    \begin{macro}{\AC@patch}
%    The same patch principle is applicable to both
%    \cs{@addtoreset} and \cs{@removefromreset}.
%    \begin{macrocode}
\def\AC@patch#1{%
  \expandafter\let\csname AC@org@#1reset\expandafter\endcsname
    \csname @#1reset\endcsname
  \expandafter\def\csname @#1reset\endcsname##1##2{%
    \csname AC@org@#1reset\endcsname{##1}{\AC@findrootcnt{##2}}%
  }%
}
%    \end{macrocode}
%    \end{macro}
%    If \xpackage{remreset} is not loaded we cannot delay
%    the patch to \cs{AtBeginDocumen}, because \cs{@removefromreset}
%    can be called in between. Therefore we force the loading of
%    the package.
%    \begin{macrocode}
\RequirePackage{remreset}
\AC@patch{addto}
\AC@patch{removefrom}
%    \end{macrocode}
%
%    \begin{macrocode}
%</package>
%    \end{macrocode}
%
% \section{Installation}
%
% \subsection{Download}
%
% \paragraph{Package.} This package is available on
% CTAN\footnote{\url{ftp://ftp.ctan.org/tex-archive/}}:
% \begin{description}
% \item[\CTAN{macros/latex/contrib/oberdiek/aliascnt.dtx}] The source file.
% \item[\CTAN{macros/latex/contrib/oberdiek/aliascnt.pdf}] Documentation.
% \end{description}
%
%
% \paragraph{Bundle.} All the packages of the bundle `oberdiek'
% are also available in a TDS compliant ZIP archive. There
% the packages are already unpacked and the documentation files
% are generated. The files and directories obey the TDS standard.
% \begin{description}
% \item[\CTAN{install/macros/latex/contrib/oberdiek.tds.zip}]
% \end{description}
% \emph{TDS} refers to the standard ``A Directory Structure
% for \TeX\ Files'' (\CTAN{tds/tds.pdf}). Directories
% with \xfile{texmf} in their name are usually organized this way.
%
% \subsection{Bundle installation}
%
% \paragraph{Unpacking.} Unpack the \xfile{oberdiek.tds.zip} in the
% TDS tree (also known as \xfile{texmf} tree) of your choice.
% Example (linux):
% \begin{quote}
%   |unzip oberdiek.tds.zip -d ~/texmf|
% \end{quote}
%
% \paragraph{Script installation.}
% Check the directory \xfile{TDS:scripts/oberdiek/} for
% scripts that need further installation steps.
% Package \xpackage{attachfile2} comes with the Perl script
% \xfile{pdfatfi.pl} that should be installed in such a way
% that it can be called as \texttt{pdfatfi}.
% Example (linux):
% \begin{quote}
%   |chmod +x scripts/oberdiek/pdfatfi.pl|\\
%   |cp scripts/oberdiek/pdfatfi.pl /usr/local/bin/|
% \end{quote}
%
% \subsection{Package installation}
%
% \paragraph{Unpacking.} The \xfile{.dtx} file is a self-extracting
% \docstrip\ archive. The files are extracted by running the
% \xfile{.dtx} through \plainTeX:
% \begin{quote}
%   \verb|tex aliascnt.dtx|
% \end{quote}
%
% \paragraph{TDS.} Now the different files must be moved into
% the different directories in your installation TDS tree
% (also known as \xfile{texmf} tree):
% \begin{quote}
% \def\t{^^A
% \begin{tabular}{@{}>{\ttfamily}l@{ $\rightarrow$ }>{\ttfamily}l@{}}
%   aliascnt.sty & tex/latex/oberdiek/aliascnt.sty\\
%   aliascnt.pdf & doc/latex/oberdiek/aliascnt.pdf\\
%   aliascnt.dtx & source/latex/oberdiek/aliascnt.dtx\\
% \end{tabular}^^A
% }^^A
% \sbox0{\t}^^A
% \ifdim\wd0>\linewidth
%   \begingroup
%     \advance\linewidth by\leftmargin
%     \advance\linewidth by\rightmargin
%   \edef\x{\endgroup
%     \def\noexpand\lw{\the\linewidth}^^A
%   }\x
%   \def\lwbox{^^A
%     \leavevmode
%     \hbox to \linewidth{^^A
%       \kern-\leftmargin\relax
%       \hss
%       \usebox0
%       \hss
%       \kern-\rightmargin\relax
%     }^^A
%   }^^A
%   \ifdim\wd0>\lw
%     \sbox0{\small\t}^^A
%     \ifdim\wd0>\linewidth
%       \ifdim\wd0>\lw
%         \sbox0{\footnotesize\t}^^A
%         \ifdim\wd0>\linewidth
%           \ifdim\wd0>\lw
%             \sbox0{\scriptsize\t}^^A
%             \ifdim\wd0>\linewidth
%               \ifdim\wd0>\lw
%                 \sbox0{\tiny\t}^^A
%                 \ifdim\wd0>\linewidth
%                   \lwbox
%                 \else
%                   \usebox0
%                 \fi
%               \else
%                 \lwbox
%               \fi
%             \else
%               \usebox0
%             \fi
%           \else
%             \lwbox
%           \fi
%         \else
%           \usebox0
%         \fi
%       \else
%         \lwbox
%       \fi
%     \else
%       \usebox0
%     \fi
%   \else
%     \lwbox
%   \fi
% \else
%   \usebox0
% \fi
% \end{quote}
% If you have a \xfile{docstrip.cfg} that configures and enables \docstrip's
% TDS installing feature, then some files can already be in the right
% place, see the documentation of \docstrip.
%
% \subsection{Refresh file name databases}
%
% If your \TeX~distribution
% (\teTeX, \mikTeX, \dots) relies on file name databases, you must refresh
% these. For example, \teTeX\ users run \verb|texhash| or
% \verb|mktexlsr|.
%
% \subsection{Some details for the interested}
%
% \paragraph{Attached source.}
%
% The PDF documentation on CTAN also includes the
% \xfile{.dtx} source file. It can be extracted by
% AcrobatReader 6 or higher. Another option is \textsf{pdftk},
% e.g. unpack the file into the current directory:
% \begin{quote}
%   \verb|pdftk aliascnt.pdf unpack_files output .|
% \end{quote}
%
% \paragraph{Unpacking with \LaTeX.}
% The \xfile{.dtx} chooses its action depending on the format:
% \begin{description}
% \item[\plainTeX:] Run \docstrip\ and extract the files.
% \item[\LaTeX:] Generate the documentation.
% \end{description}
% If you insist on using \LaTeX\ for \docstrip\ (really,
% \docstrip\ does not need \LaTeX), then inform the autodetect routine
% about your intention:
% \begin{quote}
%   \verb|latex \let\install=y% \iffalse meta-comment
%
% File: aliascnt.dtx
% Version: 2009/09/08 v1.3
% Info: Alias counters
%
% Copyright (C) 2006, 2009 by
%    Heiko Oberdiek <heiko.oberdiek at googlemail.com>
%
% This work may be distributed and/or modified under the
% conditions of the LaTeX Project Public License, either
% version 1.3c of this license or (at your option) any later
% version. This version of this license is in
%    http://www.latex-project.org/lppl/lppl-1-3c.txt
% and the latest version of this license is in
%    http://www.latex-project.org/lppl.txt
% and version 1.3 or later is part of all distributions of
% LaTeX version 2005/12/01 or later.
%
% This work has the LPPL maintenance status "maintained".
%
% This Current Maintainer of this work is Heiko Oberdiek.
%
% This work consists of the main source file aliascnt.dtx
% and the derived files
%    aliascnt.sty, aliascnt.pdf, aliascnt.ins, aliascnt.drv.
%
% Distribution:
%    CTAN:macros/latex/contrib/oberdiek/aliascnt.dtx
%    CTAN:macros/latex/contrib/oberdiek/aliascnt.pdf
%
% Unpacking:
%    (a) If aliascnt.ins is present:
%           tex aliascnt.ins
%    (b) Without aliascnt.ins:
%           tex aliascnt.dtx
%    (c) If you insist on using LaTeX
%           latex \let\install=y\input{aliascnt.dtx}
%        (quote the arguments according to the demands of your shell)
%
% Documentation:
%    (a) If aliascnt.drv is present:
%           latex aliascnt.drv
%    (b) Without aliascnt.drv:
%           latex aliascnt.dtx; ...
%    The class ltxdoc loads the configuration file ltxdoc.cfg
%    if available. Here you can specify further options, e.g.
%    use A4 as paper format:
%       \PassOptionsToClass{a4paper}{article}
%
%    Programm calls to get the documentation (example):
%       pdflatex aliascnt.dtx
%       makeindex -s gind.ist aliascnt.idx
%       pdflatex aliascnt.dtx
%       makeindex -s gind.ist aliascnt.idx
%       pdflatex aliascnt.dtx
%
% Installation:
%    TDS:tex/latex/oberdiek/aliascnt.sty
%    TDS:doc/latex/oberdiek/aliascnt.pdf
%    TDS:source/latex/oberdiek/aliascnt.dtx
%
%<*ignore>
\begingroup
  \catcode123=1 %
  \catcode125=2 %
  \def\x{LaTeX2e}%
\expandafter\endgroup
\ifcase 0\ifx\install y1\fi\expandafter
         \ifx\csname processbatchFile\endcsname\relax\else1\fi
         \ifx\fmtname\x\else 1\fi\relax
\else\csname fi\endcsname
%</ignore>
%<*install>
\input docstrip.tex
\Msg{************************************************************************}
\Msg{* Installation}
\Msg{* Package: aliascnt 2009/09/08 v1.3 Alias counters (HO)}
\Msg{************************************************************************}

\keepsilent
\askforoverwritefalse

\let\MetaPrefix\relax
\preamble

This is a generated file.

Project: aliascnt
Version: 2009/09/08 v1.3

Copyright (C) 2006, 2009 by
   Heiko Oberdiek <heiko.oberdiek at googlemail.com>

This work may be distributed and/or modified under the
conditions of the LaTeX Project Public License, either
version 1.3c of this license or (at your option) any later
version. This version of this license is in
   http://www.latex-project.org/lppl/lppl-1-3c.txt
and the latest version of this license is in
   http://www.latex-project.org/lppl.txt
and version 1.3 or later is part of all distributions of
LaTeX version 2005/12/01 or later.

This work has the LPPL maintenance status "maintained".

This Current Maintainer of this work is Heiko Oberdiek.

This work consists of the main source file aliascnt.dtx
and the derived files
   aliascnt.sty, aliascnt.pdf, aliascnt.ins, aliascnt.drv.

\endpreamble
\let\MetaPrefix\DoubleperCent

\generate{%
  \file{aliascnt.ins}{\from{aliascnt.dtx}{install}}%
  \file{aliascnt.drv}{\from{aliascnt.dtx}{driver}}%
  \usedir{tex/latex/oberdiek}%
  \file{aliascnt.sty}{\from{aliascnt.dtx}{package}}%
  \nopreamble
  \nopostamble
  \usedir{source/latex/oberdiek/catalogue}%
  \file{aliascnt.xml}{\from{aliascnt.dtx}{catalogue}}%
}

\catcode32=13\relax% active space
\let =\space%
\Msg{************************************************************************}
\Msg{*}
\Msg{* To finish the installation you have to move the following}
\Msg{* file into a directory searched by TeX:}
\Msg{*}
\Msg{*     aliascnt.sty}
\Msg{*}
\Msg{* To produce the documentation run the file `aliascnt.drv'}
\Msg{* through LaTeX.}
\Msg{*}
\Msg{* Happy TeXing!}
\Msg{*}
\Msg{************************************************************************}

\endbatchfile
%</install>
%<*ignore>
\fi
%</ignore>
%<*driver>
\NeedsTeXFormat{LaTeX2e}
\ProvidesFile{aliascnt.drv}%
  [2009/09/08 v1.3 Alias counters (HO)]%
\documentclass{ltxdoc}
\usepackage{holtxdoc}[2011/11/22]
\begin{document}
  \DocInput{aliascnt.dtx}%
\end{document}
%</driver>
% \fi
%
% \CheckSum{78}
%
% \CharacterTable
%  {Upper-case    \A\B\C\D\E\F\G\H\I\J\K\L\M\N\O\P\Q\R\S\T\U\V\W\X\Y\Z
%   Lower-case    \a\b\c\d\e\f\g\h\i\j\k\l\m\n\o\p\q\r\s\t\u\v\w\x\y\z
%   Digits        \0\1\2\3\4\5\6\7\8\9
%   Exclamation   \!     Double quote  \"     Hash (number) \#
%   Dollar        \$     Percent       \%     Ampersand     \&
%   Acute accent  \'     Left paren    \(     Right paren   \)
%   Asterisk      \*     Plus          \+     Comma         \,
%   Minus         \-     Point         \.     Solidus       \/
%   Colon         \:     Semicolon     \;     Less than     \<
%   Equals        \=     Greater than  \>     Question mark \?
%   Commercial at \@     Left bracket  \[     Backslash     \\
%   Right bracket \]     Circumflex    \^     Underscore    \_
%   Grave accent  \`     Left brace    \{     Vertical bar  \|
%   Right brace   \}     Tilde         \~}
%
% \GetFileInfo{aliascnt.drv}
%
% \title{The \xpackage{aliascnt} package}
% \date{2009/09/08 v1.3}
% \author{Heiko Oberdiek\\\xemail{heiko.oberdiek at googlemail.com}}
%
% \maketitle
%
% \begin{abstract}
% Package \xpackage{aliascnt} introduces \emph{alias counters} that
% share the same counter register and clear list.
% \end{abstract}
%
% \tableofcontents
%
% \section{User interface}
%
% \subsection{Introduction}
%
% There are features that rely on the name of counters. For
% example, \xpackage{hyperref}'s \cs{autoref} indirectly uses
% the counter name to determine which label text it puts in front
% of the reference number (\cite{hyperref}).
% In some circumstances this fail: several theorem environments
% are defined by \cs{newtheorem} that share the same counter.
%
% \subsection{Syntax}
%
% Macro names in user land contain the package name
% \texttt{aliascnt} in order to prevent name clashes.
%
% \newenvironment{desc}{^^A
%   \list{}{^^A
%     \setlength{\labelwidth}{0pt}^^A
%     \setlength{\itemindent}{-.5\marginparwidth}^^A
%     \setlength{\leftmargin}{0pt}^^A
%     \let\makelabel\desclabel
%   }^^A
% }{^^A
%   \endlist
% }
% \newcommand*{\desclabel}[1]{^^A
%   \hspace{\labelsep}^^A
%   \normalfont
%   #1^^A
% }
% \newcommand*{\itemcs}[2]{^^A
%   \item[^^A
%      \expandafter\SpecialUsageIndex\csname #1\endcsname
%      {\cs{#1}#2}^^A
%   ]\mbox{}\\*[.5ex]^^A
%   \ignorespaces
% }
% \begin{desc}
% \itemcs{newaliascnt}{\marg{ALIASCNT}\marg{BASECNT}}
%    An alias counter ALIASCNT is created that does not allocate
%    a new \TeX\ counter register. It shares the count register and
%    the clear list with counter BASECNT. If the value of either
%    the two registers is changed, the changes affects both.
% \itemcs{aliascntresetthe}{\marg{ALIASCNT}}
%    This fixes a problem with \cs{newtheorem} if it
%    is fooled by an alias counter with the same name:
%    \begin{quote}
%\begin{verbatim}
%\newtheorem{foo}{Foo}% counter "foo"
%\newaliascnt{bar}{foo}% alias counter "bar"
%\newtheorem{bar}[bar]{Bar}
%\aliascntresetthe{bar}
%\end{verbatim}
%    \end{quote}
% \end{desc}
%
% \StopEventually{
% }
%
% \section{Implementation}
%
% \subsection{Identification}
%
%    \begin{macrocode}
%<*package>
\NeedsTeXFormat{LaTeX2e}
\ProvidesPackage{aliascnt}%
  [2009/09/08 v1.3 Alias counters (HO)]%
%    \end{macrocode}
%
% \subsection{Create new alias counter}
%
%    \begin{macro}{\newaliascnt}
%    A new alias counter is set up by \cs{newaliascnt}.
%    The following properties are added for the new counter CNT:
%    \begin{description}
%    \item[\mdseries\cs{theH}\meta{CNT}:] Compatibility for \xpackage{hyperref}
%    \item[\mdseries\cs{AC@cnt@}\meta{CNT}:] Name of the referenced counter
%      in the definition.
%    \end{description}
%    \begin{macrocode}
\newcommand*{\newaliascnt}[2]{%
  \begingroup
    \def\AC@glet##1{%
      \global\expandafter\let\csname##1#1\expandafter\endcsname
        \csname##1#2\endcsname
    }%
    \@ifundefined{c@#2}{%
      \@nocounterr{#2}%
    }{%
      \expandafter\@ifdefinable\csname c@#1\endcsname{%
        \AC@glet{c@}%
        \AC@glet{the}%
        \AC@glet{theH}%
        \AC@glet{p@}%
        \expandafter\gdef\csname AC@cnt@#1\endcsname{#2}%
        \expandafter\gdef\csname cl@#1\expandafter\endcsname
        \expandafter{\csname cl@#2\endcsname}%
      }%
    }%
  \endgroup
}
%    \end{macrocode}
%    \end{macro}
%
%    \begin{macro}{\aliascntresetthe}
%    The \cs{the}\meta{CNT} macro is restored using the
%    main counter.
%    \begin{macrocode}
\newcommand*{\aliascntresetthe}[1]{%
  \@ifundefined{AC@cnt@#1}{%
    \PackageError{aliascnt}{%
      `#1' is not an alias counter%
    }\@ehc
  }{%
    \expandafter\let\csname the#1\expandafter\endcsname
      \csname the\csname AC@cnt@#1\endcsname\endcsname
  }%
}
%    \end{macrocode}
%    \end{macro}
%
% \subsection{Counter clear list}
%
%    The alias counters share the same register and clear list.
%    Therefore we must ensure that manipulations to the clear list
%    are done with the clear list macro of a real counter.
%    \begin{macro}{\AC@findrootcnt}
%    \cs{AC@findrootcnt} walks throught the aliasing relations
%    to find the base counter.
%    \begin{macrocode}
\newcommand*{\AC@findrootcnt}[1]{%
  \@ifundefined{AC@cnt@#1}{%
    #1%
  }{%
    \expandafter\AC@findrootcnt\csname AC@cnt@#1\endcsname
  }%
}
%    \end{macrocode}
%    \end{macro}
%
%    Clear lists are manipulated by \cs{@addtoreset} and
%    \cs{@removefromreset}. The latter one is provided by
%    the \xpackage{remreset} package (\cite{remreset}).
%
%    \begin{macro}{\AC@patch}
%    The same patch principle is applicable to both
%    \cs{@addtoreset} and \cs{@removefromreset}.
%    \begin{macrocode}
\def\AC@patch#1{%
  \expandafter\let\csname AC@org@#1reset\expandafter\endcsname
    \csname @#1reset\endcsname
  \expandafter\def\csname @#1reset\endcsname##1##2{%
    \csname AC@org@#1reset\endcsname{##1}{\AC@findrootcnt{##2}}%
  }%
}
%    \end{macrocode}
%    \end{macro}
%    If \xpackage{remreset} is not loaded we cannot delay
%    the patch to \cs{AtBeginDocumen}, because \cs{@removefromreset}
%    can be called in between. Therefore we force the loading of
%    the package.
%    \begin{macrocode}
\RequirePackage{remreset}
\AC@patch{addto}
\AC@patch{removefrom}
%    \end{macrocode}
%
%    \begin{macrocode}
%</package>
%    \end{macrocode}
%
% \section{Installation}
%
% \subsection{Download}
%
% \paragraph{Package.} This package is available on
% CTAN\footnote{\url{ftp://ftp.ctan.org/tex-archive/}}:
% \begin{description}
% \item[\CTAN{macros/latex/contrib/oberdiek/aliascnt.dtx}] The source file.
% \item[\CTAN{macros/latex/contrib/oberdiek/aliascnt.pdf}] Documentation.
% \end{description}
%
%
% \paragraph{Bundle.} All the packages of the bundle `oberdiek'
% are also available in a TDS compliant ZIP archive. There
% the packages are already unpacked and the documentation files
% are generated. The files and directories obey the TDS standard.
% \begin{description}
% \item[\CTAN{install/macros/latex/contrib/oberdiek.tds.zip}]
% \end{description}
% \emph{TDS} refers to the standard ``A Directory Structure
% for \TeX\ Files'' (\CTAN{tds/tds.pdf}). Directories
% with \xfile{texmf} in their name are usually organized this way.
%
% \subsection{Bundle installation}
%
% \paragraph{Unpacking.} Unpack the \xfile{oberdiek.tds.zip} in the
% TDS tree (also known as \xfile{texmf} tree) of your choice.
% Example (linux):
% \begin{quote}
%   |unzip oberdiek.tds.zip -d ~/texmf|
% \end{quote}
%
% \paragraph{Script installation.}
% Check the directory \xfile{TDS:scripts/oberdiek/} for
% scripts that need further installation steps.
% Package \xpackage{attachfile2} comes with the Perl script
% \xfile{pdfatfi.pl} that should be installed in such a way
% that it can be called as \texttt{pdfatfi}.
% Example (linux):
% \begin{quote}
%   |chmod +x scripts/oberdiek/pdfatfi.pl|\\
%   |cp scripts/oberdiek/pdfatfi.pl /usr/local/bin/|
% \end{quote}
%
% \subsection{Package installation}
%
% \paragraph{Unpacking.} The \xfile{.dtx} file is a self-extracting
% \docstrip\ archive. The files are extracted by running the
% \xfile{.dtx} through \plainTeX:
% \begin{quote}
%   \verb|tex aliascnt.dtx|
% \end{quote}
%
% \paragraph{TDS.} Now the different files must be moved into
% the different directories in your installation TDS tree
% (also known as \xfile{texmf} tree):
% \begin{quote}
% \def\t{^^A
% \begin{tabular}{@{}>{\ttfamily}l@{ $\rightarrow$ }>{\ttfamily}l@{}}
%   aliascnt.sty & tex/latex/oberdiek/aliascnt.sty\\
%   aliascnt.pdf & doc/latex/oberdiek/aliascnt.pdf\\
%   aliascnt.dtx & source/latex/oberdiek/aliascnt.dtx\\
% \end{tabular}^^A
% }^^A
% \sbox0{\t}^^A
% \ifdim\wd0>\linewidth
%   \begingroup
%     \advance\linewidth by\leftmargin
%     \advance\linewidth by\rightmargin
%   \edef\x{\endgroup
%     \def\noexpand\lw{\the\linewidth}^^A
%   }\x
%   \def\lwbox{^^A
%     \leavevmode
%     \hbox to \linewidth{^^A
%       \kern-\leftmargin\relax
%       \hss
%       \usebox0
%       \hss
%       \kern-\rightmargin\relax
%     }^^A
%   }^^A
%   \ifdim\wd0>\lw
%     \sbox0{\small\t}^^A
%     \ifdim\wd0>\linewidth
%       \ifdim\wd0>\lw
%         \sbox0{\footnotesize\t}^^A
%         \ifdim\wd0>\linewidth
%           \ifdim\wd0>\lw
%             \sbox0{\scriptsize\t}^^A
%             \ifdim\wd0>\linewidth
%               \ifdim\wd0>\lw
%                 \sbox0{\tiny\t}^^A
%                 \ifdim\wd0>\linewidth
%                   \lwbox
%                 \else
%                   \usebox0
%                 \fi
%               \else
%                 \lwbox
%               \fi
%             \else
%               \usebox0
%             \fi
%           \else
%             \lwbox
%           \fi
%         \else
%           \usebox0
%         \fi
%       \else
%         \lwbox
%       \fi
%     \else
%       \usebox0
%     \fi
%   \else
%     \lwbox
%   \fi
% \else
%   \usebox0
% \fi
% \end{quote}
% If you have a \xfile{docstrip.cfg} that configures and enables \docstrip's
% TDS installing feature, then some files can already be in the right
% place, see the documentation of \docstrip.
%
% \subsection{Refresh file name databases}
%
% If your \TeX~distribution
% (\teTeX, \mikTeX, \dots) relies on file name databases, you must refresh
% these. For example, \teTeX\ users run \verb|texhash| or
% \verb|mktexlsr|.
%
% \subsection{Some details for the interested}
%
% \paragraph{Attached source.}
%
% The PDF documentation on CTAN also includes the
% \xfile{.dtx} source file. It can be extracted by
% AcrobatReader 6 or higher. Another option is \textsf{pdftk},
% e.g. unpack the file into the current directory:
% \begin{quote}
%   \verb|pdftk aliascnt.pdf unpack_files output .|
% \end{quote}
%
% \paragraph{Unpacking with \LaTeX.}
% The \xfile{.dtx} chooses its action depending on the format:
% \begin{description}
% \item[\plainTeX:] Run \docstrip\ and extract the files.
% \item[\LaTeX:] Generate the documentation.
% \end{description}
% If you insist on using \LaTeX\ for \docstrip\ (really,
% \docstrip\ does not need \LaTeX), then inform the autodetect routine
% about your intention:
% \begin{quote}
%   \verb|latex \let\install=y\input{aliascnt.dtx}|
% \end{quote}
% Do not forget to quote the argument according to the demands
% of your shell.
%
% \paragraph{Generating the documentation.}
% You can use both the \xfile{.dtx} or the \xfile{.drv} to generate
% the documentation. The process can be configured by the
% configuration file \xfile{ltxdoc.cfg}. For instance, put this
% line into this file, if you want to have A4 as paper format:
% \begin{quote}
%   \verb|\PassOptionsToClass{a4paper}{article}|
% \end{quote}
% An example follows how to generate the
% documentation with pdf\LaTeX:
% \begin{quote}
%\begin{verbatim}
%pdflatex aliascnt.dtx
%makeindex -s gind.ist aliascnt.idx
%pdflatex aliascnt.dtx
%makeindex -s gind.ist aliascnt.idx
%pdflatex aliascnt.dtx
%\end{verbatim}
% \end{quote}
%
% \section{Catalogue}
%
% The following XML file can be used as source for the
% \href{http://mirror.ctan.org/help/Catalogue/catalogue.html}{\TeX\ Catalogue}.
% The elements \texttt{caption} and \texttt{description} are imported
% from the original XML file from the Catalogue.
% The name of the XML file in the Catalogue is \xfile{aliascnt.xml}.
%    \begin{macrocode}
%<*catalogue>
<?xml version='1.0' encoding='us-ascii'?>
<!DOCTYPE entry SYSTEM 'catalogue.dtd'>
<entry datestamp='$Date$' modifier='$Author$' id='aliascnt'>
  <name>aliascnt</name>
  <caption>Alias counters.</caption>
  <authorref id='auth:oberdiek'/>
  <copyright owner='Heiko Oberdiek' year='2006,2009'/>
  <license type='lppl1.3'/>
  <version number='1.3'/>
  <description>
    This package introduces aliases for counters, that
    share the same counter register and clear list.
    <p/>
    The package is part of the <xref refid='oberdiek'>oberdiek</xref>
    bundle.
  </description>
  <documentation details='Package documentation'
      href='ctan:/macros/latex/contrib/oberdiek/aliascnt.pdf'/>
  <ctan file='true' path='/macros/latex/contrib/oberdiek/aliascnt.dtx'/>
  <miktex location='oberdiek'/>
  <texlive location='oberdiek'/>
  <install path='/macros/latex/contrib/oberdiek/oberdiek.tds.zip'/>
</entry>
%</catalogue>
%    \end{macrocode}
%
% \section{Acknowledgement}
%
% \begin{description}
% \item[Ulrich Schwarz:] The package is based on his draft for
%   ``Die \TeX nische Kom\"odie'', see \cite{schwarz}.
% \end{description}
%
% \begin{thebibliography}{9}
%
% \bibitem{schwarz}
%   Ulrich Schwarz:
%   \textit{Was hinten herauskommt z\"ahlt: Counter Aliasing in \LaTeX},
%   \textit{Die \TeX nische Kom\"odie}, 3/2006, pages 8--14, Juli 2006.
%
% \bibitem{remreset}
%   David Carlisle: \textit{The \xpackage{remreset} package};
%   1997/09/28;
%   \CTAN{macros/latex/contrib/carlisle/remreset.sty}.
%
% \bibitem{hyperref}
%   Sebastian Rahtz, Heiko Oberdiek:
%   \textit{The \xpackage{hyperref} package};
%   2006/08/16 v6.75c;
%   \CTAN{macros/latex/contrib/hyperref/}.
%
% \end{thebibliography}
%
% \begin{History}
%   \begin{Version}{2006/02/20 v1.0}
%   \item
%     First version.
%   \end{Version}
%   \begin{Version}{2006/08/16 v1.1}
%   \item
%     Update of bibliography.
%   \end{Version}
%   \begin{Version}{2006/09/25 v1.2}
%   \item
%     Bug fix (\cs{aliascntresetthe}).
%   \end{Version}
%   \begin{Version}{2009/09/08 v1.3}
%   \item
%     Bug fix of \cs{@ifdefinable}'s use (thanks to Uwe L\"uck).
%   \end{Version}
% \end{History}
%
% \PrintIndex
%
% \Finale
\endinput
|
% \end{quote}
% Do not forget to quote the argument according to the demands
% of your shell.
%
% \paragraph{Generating the documentation.}
% You can use both the \xfile{.dtx} or the \xfile{.drv} to generate
% the documentation. The process can be configured by the
% configuration file \xfile{ltxdoc.cfg}. For instance, put this
% line into this file, if you want to have A4 as paper format:
% \begin{quote}
%   \verb|\PassOptionsToClass{a4paper}{article}|
% \end{quote}
% An example follows how to generate the
% documentation with pdf\LaTeX:
% \begin{quote}
%\begin{verbatim}
%pdflatex aliascnt.dtx
%makeindex -s gind.ist aliascnt.idx
%pdflatex aliascnt.dtx
%makeindex -s gind.ist aliascnt.idx
%pdflatex aliascnt.dtx
%\end{verbatim}
% \end{quote}
%
% \section{Catalogue}
%
% The following XML file can be used as source for the
% \href{http://mirror.ctan.org/help/Catalogue/catalogue.html}{\TeX\ Catalogue}.
% The elements \texttt{caption} and \texttt{description} are imported
% from the original XML file from the Catalogue.
% The name of the XML file in the Catalogue is \xfile{aliascnt.xml}.
%    \begin{macrocode}
%<*catalogue>
<?xml version='1.0' encoding='us-ascii'?>
<!DOCTYPE entry SYSTEM 'catalogue.dtd'>
<entry datestamp='$Date$' modifier='$Author$' id='aliascnt'>
  <name>aliascnt</name>
  <caption>Alias counters.</caption>
  <authorref id='auth:oberdiek'/>
  <copyright owner='Heiko Oberdiek' year='2006,2009'/>
  <license type='lppl1.3'/>
  <version number='1.3'/>
  <description>
    This package introduces aliases for counters, that
    share the same counter register and clear list.
    <p/>
    The package is part of the <xref refid='oberdiek'>oberdiek</xref>
    bundle.
  </description>
  <documentation details='Package documentation'
      href='ctan:/macros/latex/contrib/oberdiek/aliascnt.pdf'/>
  <ctan file='true' path='/macros/latex/contrib/oberdiek/aliascnt.dtx'/>
  <miktex location='oberdiek'/>
  <texlive location='oberdiek'/>
  <install path='/macros/latex/contrib/oberdiek/oberdiek.tds.zip'/>
</entry>
%</catalogue>
%    \end{macrocode}
%
% \section{Acknowledgement}
%
% \begin{description}
% \item[Ulrich Schwarz:] The package is based on his draft for
%   ``Die \TeX nische Kom\"odie'', see \cite{schwarz}.
% \end{description}
%
% \begin{thebibliography}{9}
%
% \bibitem{schwarz}
%   Ulrich Schwarz:
%   \textit{Was hinten herauskommt z\"ahlt: Counter Aliasing in \LaTeX},
%   \textit{Die \TeX nische Kom\"odie}, 3/2006, pages 8--14, Juli 2006.
%
% \bibitem{remreset}
%   David Carlisle: \textit{The \xpackage{remreset} package};
%   1997/09/28;
%   \CTAN{macros/latex/contrib/carlisle/remreset.sty}.
%
% \bibitem{hyperref}
%   Sebastian Rahtz, Heiko Oberdiek:
%   \textit{The \xpackage{hyperref} package};
%   2006/08/16 v6.75c;
%   \CTAN{macros/latex/contrib/hyperref/}.
%
% \end{thebibliography}
%
% \begin{History}
%   \begin{Version}{2006/02/20 v1.0}
%   \item
%     First version.
%   \end{Version}
%   \begin{Version}{2006/08/16 v1.1}
%   \item
%     Update of bibliography.
%   \end{Version}
%   \begin{Version}{2006/09/25 v1.2}
%   \item
%     Bug fix (\cs{aliascntresetthe}).
%   \end{Version}
%   \begin{Version}{2009/09/08 v1.3}
%   \item
%     Bug fix of \cs{@ifdefinable}'s use (thanks to Uwe L\"uck).
%   \end{Version}
% \end{History}
%
% \PrintIndex
%
% \Finale
\endinput

%        (quote the arguments according to the demands of your shell)
%
% Documentation:
%    (a) If aliascnt.drv is present:
%           latex aliascnt.drv
%    (b) Without aliascnt.drv:
%           latex aliascnt.dtx; ...
%    The class ltxdoc loads the configuration file ltxdoc.cfg
%    if available. Here you can specify further options, e.g.
%    use A4 as paper format:
%       \PassOptionsToClass{a4paper}{article}
%
%    Programm calls to get the documentation (example):
%       pdflatex aliascnt.dtx
%       makeindex -s gind.ist aliascnt.idx
%       pdflatex aliascnt.dtx
%       makeindex -s gind.ist aliascnt.idx
%       pdflatex aliascnt.dtx
%
% Installation:
%    TDS:tex/latex/oberdiek/aliascnt.sty
%    TDS:doc/latex/oberdiek/aliascnt.pdf
%    TDS:source/latex/oberdiek/aliascnt.dtx
%
%<*ignore>
\begingroup
  \catcode123=1 %
  \catcode125=2 %
  \def\x{LaTeX2e}%
\expandafter\endgroup
\ifcase 0\ifx\install y1\fi\expandafter
         \ifx\csname processbatchFile\endcsname\relax\else1\fi
         \ifx\fmtname\x\else 1\fi\relax
\else\csname fi\endcsname
%</ignore>
%<*install>
\input docstrip.tex
\Msg{************************************************************************}
\Msg{* Installation}
\Msg{* Package: aliascnt 2009/09/08 v1.3 Alias counters (HO)}
\Msg{************************************************************************}

\keepsilent
\askforoverwritefalse

\let\MetaPrefix\relax
\preamble

This is a generated file.

Project: aliascnt
Version: 2009/09/08 v1.3

Copyright (C) 2006, 2009 by
   Heiko Oberdiek <heiko.oberdiek at googlemail.com>

This work may be distributed and/or modified under the
conditions of the LaTeX Project Public License, either
version 1.3c of this license or (at your option) any later
version. This version of this license is in
   http://www.latex-project.org/lppl/lppl-1-3c.txt
and the latest version of this license is in
   http://www.latex-project.org/lppl.txt
and version 1.3 or later is part of all distributions of
LaTeX version 2005/12/01 or later.

This work has the LPPL maintenance status "maintained".

This Current Maintainer of this work is Heiko Oberdiek.

This work consists of the main source file aliascnt.dtx
and the derived files
   aliascnt.sty, aliascnt.pdf, aliascnt.ins, aliascnt.drv.

\endpreamble
\let\MetaPrefix\DoubleperCent

\generate{%
  \file{aliascnt.ins}{\from{aliascnt.dtx}{install}}%
  \file{aliascnt.drv}{\from{aliascnt.dtx}{driver}}%
  \usedir{tex/latex/oberdiek}%
  \file{aliascnt.sty}{\from{aliascnt.dtx}{package}}%
  \nopreamble
  \nopostamble
  \usedir{source/latex/oberdiek/catalogue}%
  \file{aliascnt.xml}{\from{aliascnt.dtx}{catalogue}}%
}

\catcode32=13\relax% active space
\let =\space%
\Msg{************************************************************************}
\Msg{*}
\Msg{* To finish the installation you have to move the following}
\Msg{* file into a directory searched by TeX:}
\Msg{*}
\Msg{*     aliascnt.sty}
\Msg{*}
\Msg{* To produce the documentation run the file `aliascnt.drv'}
\Msg{* through LaTeX.}
\Msg{*}
\Msg{* Happy TeXing!}
\Msg{*}
\Msg{************************************************************************}

\endbatchfile
%</install>
%<*ignore>
\fi
%</ignore>
%<*driver>
\NeedsTeXFormat{LaTeX2e}
\ProvidesFile{aliascnt.drv}%
  [2009/09/08 v1.3 Alias counters (HO)]%
\documentclass{ltxdoc}
\usepackage{holtxdoc}[2011/11/22]
\begin{document}
  \DocInput{aliascnt.dtx}%
\end{document}
%</driver>
% \fi
%
% \CheckSum{78}
%
% \CharacterTable
%  {Upper-case    \A\B\C\D\E\F\G\H\I\J\K\L\M\N\O\P\Q\R\S\T\U\V\W\X\Y\Z
%   Lower-case    \a\b\c\d\e\f\g\h\i\j\k\l\m\n\o\p\q\r\s\t\u\v\w\x\y\z
%   Digits        \0\1\2\3\4\5\6\7\8\9
%   Exclamation   \!     Double quote  \"     Hash (number) \#
%   Dollar        \$     Percent       \%     Ampersand     \&
%   Acute accent  \'     Left paren    \(     Right paren   \)
%   Asterisk      \*     Plus          \+     Comma         \,
%   Minus         \-     Point         \.     Solidus       \/
%   Colon         \:     Semicolon     \;     Less than     \<
%   Equals        \=     Greater than  \>     Question mark \?
%   Commercial at \@     Left bracket  \[     Backslash     \\
%   Right bracket \]     Circumflex    \^     Underscore    \_
%   Grave accent  \`     Left brace    \{     Vertical bar  \|
%   Right brace   \}     Tilde         \~}
%
% \GetFileInfo{aliascnt.drv}
%
% \title{The \xpackage{aliascnt} package}
% \date{2009/09/08 v1.3}
% \author{Heiko Oberdiek\\\xemail{heiko.oberdiek at googlemail.com}}
%
% \maketitle
%
% \begin{abstract}
% Package \xpackage{aliascnt} introduces \emph{alias counters} that
% share the same counter register and clear list.
% \end{abstract}
%
% \tableofcontents
%
% \section{User interface}
%
% \subsection{Introduction}
%
% There are features that rely on the name of counters. For
% example, \xpackage{hyperref}'s \cs{autoref} indirectly uses
% the counter name to determine which label text it puts in front
% of the reference number (\cite{hyperref}).
% In some circumstances this fail: several theorem environments
% are defined by \cs{newtheorem} that share the same counter.
%
% \subsection{Syntax}
%
% Macro names in user land contain the package name
% \texttt{aliascnt} in order to prevent name clashes.
%
% \newenvironment{desc}{^^A
%   \list{}{^^A
%     \setlength{\labelwidth}{0pt}^^A
%     \setlength{\itemindent}{-.5\marginparwidth}^^A
%     \setlength{\leftmargin}{0pt}^^A
%     \let\makelabel\desclabel
%   }^^A
% }{^^A
%   \endlist
% }
% \newcommand*{\desclabel}[1]{^^A
%   \hspace{\labelsep}^^A
%   \normalfont
%   #1^^A
% }
% \newcommand*{\itemcs}[2]{^^A
%   \item[^^A
%      \expandafter\SpecialUsageIndex\csname #1\endcsname
%      {\cs{#1}#2}^^A
%   ]\mbox{}\\*[.5ex]^^A
%   \ignorespaces
% }
% \begin{desc}
% \itemcs{newaliascnt}{\marg{ALIASCNT}\marg{BASECNT}}
%    An alias counter ALIASCNT is created that does not allocate
%    a new \TeX\ counter register. It shares the count register and
%    the clear list with counter BASECNT. If the value of either
%    the two registers is changed, the changes affects both.
% \itemcs{aliascntresetthe}{\marg{ALIASCNT}}
%    This fixes a problem with \cs{newtheorem} if it
%    is fooled by an alias counter with the same name:
%    \begin{quote}
%\begin{verbatim}
%\newtheorem{foo}{Foo}% counter "foo"
%\newaliascnt{bar}{foo}% alias counter "bar"
%\newtheorem{bar}[bar]{Bar}
%\aliascntresetthe{bar}
%\end{verbatim}
%    \end{quote}
% \end{desc}
%
% \StopEventually{
% }
%
% \section{Implementation}
%
% \subsection{Identification}
%
%    \begin{macrocode}
%<*package>
\NeedsTeXFormat{LaTeX2e}
\ProvidesPackage{aliascnt}%
  [2009/09/08 v1.3 Alias counters (HO)]%
%    \end{macrocode}
%
% \subsection{Create new alias counter}
%
%    \begin{macro}{\newaliascnt}
%    A new alias counter is set up by \cs{newaliascnt}.
%    The following properties are added for the new counter CNT:
%    \begin{description}
%    \item[\mdseries\cs{theH}\meta{CNT}:] Compatibility for \xpackage{hyperref}
%    \item[\mdseries\cs{AC@cnt@}\meta{CNT}:] Name of the referenced counter
%      in the definition.
%    \end{description}
%    \begin{macrocode}
\newcommand*{\newaliascnt}[2]{%
  \begingroup
    \def\AC@glet##1{%
      \global\expandafter\let\csname##1#1\expandafter\endcsname
        \csname##1#2\endcsname
    }%
    \@ifundefined{c@#2}{%
      \@nocounterr{#2}%
    }{%
      \expandafter\@ifdefinable\csname c@#1\endcsname{%
        \AC@glet{c@}%
        \AC@glet{the}%
        \AC@glet{theH}%
        \AC@glet{p@}%
        \expandafter\gdef\csname AC@cnt@#1\endcsname{#2}%
        \expandafter\gdef\csname cl@#1\expandafter\endcsname
        \expandafter{\csname cl@#2\endcsname}%
      }%
    }%
  \endgroup
}
%    \end{macrocode}
%    \end{macro}
%
%    \begin{macro}{\aliascntresetthe}
%    The \cs{the}\meta{CNT} macro is restored using the
%    main counter.
%    \begin{macrocode}
\newcommand*{\aliascntresetthe}[1]{%
  \@ifundefined{AC@cnt@#1}{%
    \PackageError{aliascnt}{%
      `#1' is not an alias counter%
    }\@ehc
  }{%
    \expandafter\let\csname the#1\expandafter\endcsname
      \csname the\csname AC@cnt@#1\endcsname\endcsname
  }%
}
%    \end{macrocode}
%    \end{macro}
%
% \subsection{Counter clear list}
%
%    The alias counters share the same register and clear list.
%    Therefore we must ensure that manipulations to the clear list
%    are done with the clear list macro of a real counter.
%    \begin{macro}{\AC@findrootcnt}
%    \cs{AC@findrootcnt} walks throught the aliasing relations
%    to find the base counter.
%    \begin{macrocode}
\newcommand*{\AC@findrootcnt}[1]{%
  \@ifundefined{AC@cnt@#1}{%
    #1%
  }{%
    \expandafter\AC@findrootcnt\csname AC@cnt@#1\endcsname
  }%
}
%    \end{macrocode}
%    \end{macro}
%
%    Clear lists are manipulated by \cs{@addtoreset} and
%    \cs{@removefromreset}. The latter one is provided by
%    the \xpackage{remreset} package (\cite{remreset}).
%
%    \begin{macro}{\AC@patch}
%    The same patch principle is applicable to both
%    \cs{@addtoreset} and \cs{@removefromreset}.
%    \begin{macrocode}
\def\AC@patch#1{%
  \expandafter\let\csname AC@org@#1reset\expandafter\endcsname
    \csname @#1reset\endcsname
  \expandafter\def\csname @#1reset\endcsname##1##2{%
    \csname AC@org@#1reset\endcsname{##1}{\AC@findrootcnt{##2}}%
  }%
}
%    \end{macrocode}
%    \end{macro}
%    If \xpackage{remreset} is not loaded we cannot delay
%    the patch to \cs{AtBeginDocumen}, because \cs{@removefromreset}
%    can be called in between. Therefore we force the loading of
%    the package.
%    \begin{macrocode}
\RequirePackage{remreset}
\AC@patch{addto}
\AC@patch{removefrom}
%    \end{macrocode}
%
%    \begin{macrocode}
%</package>
%    \end{macrocode}
%
% \section{Installation}
%
% \subsection{Download}
%
% \paragraph{Package.} This package is available on
% CTAN\footnote{\url{ftp://ftp.ctan.org/tex-archive/}}:
% \begin{description}
% \item[\CTAN{macros/latex/contrib/oberdiek/aliascnt.dtx}] The source file.
% \item[\CTAN{macros/latex/contrib/oberdiek/aliascnt.pdf}] Documentation.
% \end{description}
%
%
% \paragraph{Bundle.} All the packages of the bundle `oberdiek'
% are also available in a TDS compliant ZIP archive. There
% the packages are already unpacked and the documentation files
% are generated. The files and directories obey the TDS standard.
% \begin{description}
% \item[\CTAN{install/macros/latex/contrib/oberdiek.tds.zip}]
% \end{description}
% \emph{TDS} refers to the standard ``A Directory Structure
% for \TeX\ Files'' (\CTAN{tds/tds.pdf}). Directories
% with \xfile{texmf} in their name are usually organized this way.
%
% \subsection{Bundle installation}
%
% \paragraph{Unpacking.} Unpack the \xfile{oberdiek.tds.zip} in the
% TDS tree (also known as \xfile{texmf} tree) of your choice.
% Example (linux):
% \begin{quote}
%   |unzip oberdiek.tds.zip -d ~/texmf|
% \end{quote}
%
% \paragraph{Script installation.}
% Check the directory \xfile{TDS:scripts/oberdiek/} for
% scripts that need further installation steps.
% Package \xpackage{attachfile2} comes with the Perl script
% \xfile{pdfatfi.pl} that should be installed in such a way
% that it can be called as \texttt{pdfatfi}.
% Example (linux):
% \begin{quote}
%   |chmod +x scripts/oberdiek/pdfatfi.pl|\\
%   |cp scripts/oberdiek/pdfatfi.pl /usr/local/bin/|
% \end{quote}
%
% \subsection{Package installation}
%
% \paragraph{Unpacking.} The \xfile{.dtx} file is a self-extracting
% \docstrip\ archive. The files are extracted by running the
% \xfile{.dtx} through \plainTeX:
% \begin{quote}
%   \verb|tex aliascnt.dtx|
% \end{quote}
%
% \paragraph{TDS.} Now the different files must be moved into
% the different directories in your installation TDS tree
% (also known as \xfile{texmf} tree):
% \begin{quote}
% \def\t{^^A
% \begin{tabular}{@{}>{\ttfamily}l@{ $\rightarrow$ }>{\ttfamily}l@{}}
%   aliascnt.sty & tex/latex/oberdiek/aliascnt.sty\\
%   aliascnt.pdf & doc/latex/oberdiek/aliascnt.pdf\\
%   aliascnt.dtx & source/latex/oberdiek/aliascnt.dtx\\
% \end{tabular}^^A
% }^^A
% \sbox0{\t}^^A
% \ifdim\wd0>\linewidth
%   \begingroup
%     \advance\linewidth by\leftmargin
%     \advance\linewidth by\rightmargin
%   \edef\x{\endgroup
%     \def\noexpand\lw{\the\linewidth}^^A
%   }\x
%   \def\lwbox{^^A
%     \leavevmode
%     \hbox to \linewidth{^^A
%       \kern-\leftmargin\relax
%       \hss
%       \usebox0
%       \hss
%       \kern-\rightmargin\relax
%     }^^A
%   }^^A
%   \ifdim\wd0>\lw
%     \sbox0{\small\t}^^A
%     \ifdim\wd0>\linewidth
%       \ifdim\wd0>\lw
%         \sbox0{\footnotesize\t}^^A
%         \ifdim\wd0>\linewidth
%           \ifdim\wd0>\lw
%             \sbox0{\scriptsize\t}^^A
%             \ifdim\wd0>\linewidth
%               \ifdim\wd0>\lw
%                 \sbox0{\tiny\t}^^A
%                 \ifdim\wd0>\linewidth
%                   \lwbox
%                 \else
%                   \usebox0
%                 \fi
%               \else
%                 \lwbox
%               \fi
%             \else
%               \usebox0
%             \fi
%           \else
%             \lwbox
%           \fi
%         \else
%           \usebox0
%         \fi
%       \else
%         \lwbox
%       \fi
%     \else
%       \usebox0
%     \fi
%   \else
%     \lwbox
%   \fi
% \else
%   \usebox0
% \fi
% \end{quote}
% If you have a \xfile{docstrip.cfg} that configures and enables \docstrip's
% TDS installing feature, then some files can already be in the right
% place, see the documentation of \docstrip.
%
% \subsection{Refresh file name databases}
%
% If your \TeX~distribution
% (\teTeX, \mikTeX, \dots) relies on file name databases, you must refresh
% these. For example, \teTeX\ users run \verb|texhash| or
% \verb|mktexlsr|.
%
% \subsection{Some details for the interested}
%
% \paragraph{Attached source.}
%
% The PDF documentation on CTAN also includes the
% \xfile{.dtx} source file. It can be extracted by
% AcrobatReader 6 or higher. Another option is \textsf{pdftk},
% e.g. unpack the file into the current directory:
% \begin{quote}
%   \verb|pdftk aliascnt.pdf unpack_files output .|
% \end{quote}
%
% \paragraph{Unpacking with \LaTeX.}
% The \xfile{.dtx} chooses its action depending on the format:
% \begin{description}
% \item[\plainTeX:] Run \docstrip\ and extract the files.
% \item[\LaTeX:] Generate the documentation.
% \end{description}
% If you insist on using \LaTeX\ for \docstrip\ (really,
% \docstrip\ does not need \LaTeX), then inform the autodetect routine
% about your intention:
% \begin{quote}
%   \verb|latex \let\install=y% \iffalse meta-comment
%
% File: aliascnt.dtx
% Version: 2009/09/08 v1.3
% Info: Alias counters
%
% Copyright (C) 2006, 2009 by
%    Heiko Oberdiek <heiko.oberdiek at googlemail.com>
%
% This work may be distributed and/or modified under the
% conditions of the LaTeX Project Public License, either
% version 1.3c of this license or (at your option) any later
% version. This version of this license is in
%    http://www.latex-project.org/lppl/lppl-1-3c.txt
% and the latest version of this license is in
%    http://www.latex-project.org/lppl.txt
% and version 1.3 or later is part of all distributions of
% LaTeX version 2005/12/01 or later.
%
% This work has the LPPL maintenance status "maintained".
%
% This Current Maintainer of this work is Heiko Oberdiek.
%
% This work consists of the main source file aliascnt.dtx
% and the derived files
%    aliascnt.sty, aliascnt.pdf, aliascnt.ins, aliascnt.drv.
%
% Distribution:
%    CTAN:macros/latex/contrib/oberdiek/aliascnt.dtx
%    CTAN:macros/latex/contrib/oberdiek/aliascnt.pdf
%
% Unpacking:
%    (a) If aliascnt.ins is present:
%           tex aliascnt.ins
%    (b) Without aliascnt.ins:
%           tex aliascnt.dtx
%    (c) If you insist on using LaTeX
%           latex \let\install=y% \iffalse meta-comment
%
% File: aliascnt.dtx
% Version: 2009/09/08 v1.3
% Info: Alias counters
%
% Copyright (C) 2006, 2009 by
%    Heiko Oberdiek <heiko.oberdiek at googlemail.com>
%
% This work may be distributed and/or modified under the
% conditions of the LaTeX Project Public License, either
% version 1.3c of this license or (at your option) any later
% version. This version of this license is in
%    http://www.latex-project.org/lppl/lppl-1-3c.txt
% and the latest version of this license is in
%    http://www.latex-project.org/lppl.txt
% and version 1.3 or later is part of all distributions of
% LaTeX version 2005/12/01 or later.
%
% This work has the LPPL maintenance status "maintained".
%
% This Current Maintainer of this work is Heiko Oberdiek.
%
% This work consists of the main source file aliascnt.dtx
% and the derived files
%    aliascnt.sty, aliascnt.pdf, aliascnt.ins, aliascnt.drv.
%
% Distribution:
%    CTAN:macros/latex/contrib/oberdiek/aliascnt.dtx
%    CTAN:macros/latex/contrib/oberdiek/aliascnt.pdf
%
% Unpacking:
%    (a) If aliascnt.ins is present:
%           tex aliascnt.ins
%    (b) Without aliascnt.ins:
%           tex aliascnt.dtx
%    (c) If you insist on using LaTeX
%           latex \let\install=y\input{aliascnt.dtx}
%        (quote the arguments according to the demands of your shell)
%
% Documentation:
%    (a) If aliascnt.drv is present:
%           latex aliascnt.drv
%    (b) Without aliascnt.drv:
%           latex aliascnt.dtx; ...
%    The class ltxdoc loads the configuration file ltxdoc.cfg
%    if available. Here you can specify further options, e.g.
%    use A4 as paper format:
%       \PassOptionsToClass{a4paper}{article}
%
%    Programm calls to get the documentation (example):
%       pdflatex aliascnt.dtx
%       makeindex -s gind.ist aliascnt.idx
%       pdflatex aliascnt.dtx
%       makeindex -s gind.ist aliascnt.idx
%       pdflatex aliascnt.dtx
%
% Installation:
%    TDS:tex/latex/oberdiek/aliascnt.sty
%    TDS:doc/latex/oberdiek/aliascnt.pdf
%    TDS:source/latex/oberdiek/aliascnt.dtx
%
%<*ignore>
\begingroup
  \catcode123=1 %
  \catcode125=2 %
  \def\x{LaTeX2e}%
\expandafter\endgroup
\ifcase 0\ifx\install y1\fi\expandafter
         \ifx\csname processbatchFile\endcsname\relax\else1\fi
         \ifx\fmtname\x\else 1\fi\relax
\else\csname fi\endcsname
%</ignore>
%<*install>
\input docstrip.tex
\Msg{************************************************************************}
\Msg{* Installation}
\Msg{* Package: aliascnt 2009/09/08 v1.3 Alias counters (HO)}
\Msg{************************************************************************}

\keepsilent
\askforoverwritefalse

\let\MetaPrefix\relax
\preamble

This is a generated file.

Project: aliascnt
Version: 2009/09/08 v1.3

Copyright (C) 2006, 2009 by
   Heiko Oberdiek <heiko.oberdiek at googlemail.com>

This work may be distributed and/or modified under the
conditions of the LaTeX Project Public License, either
version 1.3c of this license or (at your option) any later
version. This version of this license is in
   http://www.latex-project.org/lppl/lppl-1-3c.txt
and the latest version of this license is in
   http://www.latex-project.org/lppl.txt
and version 1.3 or later is part of all distributions of
LaTeX version 2005/12/01 or later.

This work has the LPPL maintenance status "maintained".

This Current Maintainer of this work is Heiko Oberdiek.

This work consists of the main source file aliascnt.dtx
and the derived files
   aliascnt.sty, aliascnt.pdf, aliascnt.ins, aliascnt.drv.

\endpreamble
\let\MetaPrefix\DoubleperCent

\generate{%
  \file{aliascnt.ins}{\from{aliascnt.dtx}{install}}%
  \file{aliascnt.drv}{\from{aliascnt.dtx}{driver}}%
  \usedir{tex/latex/oberdiek}%
  \file{aliascnt.sty}{\from{aliascnt.dtx}{package}}%
  \nopreamble
  \nopostamble
  \usedir{source/latex/oberdiek/catalogue}%
  \file{aliascnt.xml}{\from{aliascnt.dtx}{catalogue}}%
}

\catcode32=13\relax% active space
\let =\space%
\Msg{************************************************************************}
\Msg{*}
\Msg{* To finish the installation you have to move the following}
\Msg{* file into a directory searched by TeX:}
\Msg{*}
\Msg{*     aliascnt.sty}
\Msg{*}
\Msg{* To produce the documentation run the file `aliascnt.drv'}
\Msg{* through LaTeX.}
\Msg{*}
\Msg{* Happy TeXing!}
\Msg{*}
\Msg{************************************************************************}

\endbatchfile
%</install>
%<*ignore>
\fi
%</ignore>
%<*driver>
\NeedsTeXFormat{LaTeX2e}
\ProvidesFile{aliascnt.drv}%
  [2009/09/08 v1.3 Alias counters (HO)]%
\documentclass{ltxdoc}
\usepackage{holtxdoc}[2011/11/22]
\begin{document}
  \DocInput{aliascnt.dtx}%
\end{document}
%</driver>
% \fi
%
% \CheckSum{78}
%
% \CharacterTable
%  {Upper-case    \A\B\C\D\E\F\G\H\I\J\K\L\M\N\O\P\Q\R\S\T\U\V\W\X\Y\Z
%   Lower-case    \a\b\c\d\e\f\g\h\i\j\k\l\m\n\o\p\q\r\s\t\u\v\w\x\y\z
%   Digits        \0\1\2\3\4\5\6\7\8\9
%   Exclamation   \!     Double quote  \"     Hash (number) \#
%   Dollar        \$     Percent       \%     Ampersand     \&
%   Acute accent  \'     Left paren    \(     Right paren   \)
%   Asterisk      \*     Plus          \+     Comma         \,
%   Minus         \-     Point         \.     Solidus       \/
%   Colon         \:     Semicolon     \;     Less than     \<
%   Equals        \=     Greater than  \>     Question mark \?
%   Commercial at \@     Left bracket  \[     Backslash     \\
%   Right bracket \]     Circumflex    \^     Underscore    \_
%   Grave accent  \`     Left brace    \{     Vertical bar  \|
%   Right brace   \}     Tilde         \~}
%
% \GetFileInfo{aliascnt.drv}
%
% \title{The \xpackage{aliascnt} package}
% \date{2009/09/08 v1.3}
% \author{Heiko Oberdiek\\\xemail{heiko.oberdiek at googlemail.com}}
%
% \maketitle
%
% \begin{abstract}
% Package \xpackage{aliascnt} introduces \emph{alias counters} that
% share the same counter register and clear list.
% \end{abstract}
%
% \tableofcontents
%
% \section{User interface}
%
% \subsection{Introduction}
%
% There are features that rely on the name of counters. For
% example, \xpackage{hyperref}'s \cs{autoref} indirectly uses
% the counter name to determine which label text it puts in front
% of the reference number (\cite{hyperref}).
% In some circumstances this fail: several theorem environments
% are defined by \cs{newtheorem} that share the same counter.
%
% \subsection{Syntax}
%
% Macro names in user land contain the package name
% \texttt{aliascnt} in order to prevent name clashes.
%
% \newenvironment{desc}{^^A
%   \list{}{^^A
%     \setlength{\labelwidth}{0pt}^^A
%     \setlength{\itemindent}{-.5\marginparwidth}^^A
%     \setlength{\leftmargin}{0pt}^^A
%     \let\makelabel\desclabel
%   }^^A
% }{^^A
%   \endlist
% }
% \newcommand*{\desclabel}[1]{^^A
%   \hspace{\labelsep}^^A
%   \normalfont
%   #1^^A
% }
% \newcommand*{\itemcs}[2]{^^A
%   \item[^^A
%      \expandafter\SpecialUsageIndex\csname #1\endcsname
%      {\cs{#1}#2}^^A
%   ]\mbox{}\\*[.5ex]^^A
%   \ignorespaces
% }
% \begin{desc}
% \itemcs{newaliascnt}{\marg{ALIASCNT}\marg{BASECNT}}
%    An alias counter ALIASCNT is created that does not allocate
%    a new \TeX\ counter register. It shares the count register and
%    the clear list with counter BASECNT. If the value of either
%    the two registers is changed, the changes affects both.
% \itemcs{aliascntresetthe}{\marg{ALIASCNT}}
%    This fixes a problem with \cs{newtheorem} if it
%    is fooled by an alias counter with the same name:
%    \begin{quote}
%\begin{verbatim}
%\newtheorem{foo}{Foo}% counter "foo"
%\newaliascnt{bar}{foo}% alias counter "bar"
%\newtheorem{bar}[bar]{Bar}
%\aliascntresetthe{bar}
%\end{verbatim}
%    \end{quote}
% \end{desc}
%
% \StopEventually{
% }
%
% \section{Implementation}
%
% \subsection{Identification}
%
%    \begin{macrocode}
%<*package>
\NeedsTeXFormat{LaTeX2e}
\ProvidesPackage{aliascnt}%
  [2009/09/08 v1.3 Alias counters (HO)]%
%    \end{macrocode}
%
% \subsection{Create new alias counter}
%
%    \begin{macro}{\newaliascnt}
%    A new alias counter is set up by \cs{newaliascnt}.
%    The following properties are added for the new counter CNT:
%    \begin{description}
%    \item[\mdseries\cs{theH}\meta{CNT}:] Compatibility for \xpackage{hyperref}
%    \item[\mdseries\cs{AC@cnt@}\meta{CNT}:] Name of the referenced counter
%      in the definition.
%    \end{description}
%    \begin{macrocode}
\newcommand*{\newaliascnt}[2]{%
  \begingroup
    \def\AC@glet##1{%
      \global\expandafter\let\csname##1#1\expandafter\endcsname
        \csname##1#2\endcsname
    }%
    \@ifundefined{c@#2}{%
      \@nocounterr{#2}%
    }{%
      \expandafter\@ifdefinable\csname c@#1\endcsname{%
        \AC@glet{c@}%
        \AC@glet{the}%
        \AC@glet{theH}%
        \AC@glet{p@}%
        \expandafter\gdef\csname AC@cnt@#1\endcsname{#2}%
        \expandafter\gdef\csname cl@#1\expandafter\endcsname
        \expandafter{\csname cl@#2\endcsname}%
      }%
    }%
  \endgroup
}
%    \end{macrocode}
%    \end{macro}
%
%    \begin{macro}{\aliascntresetthe}
%    The \cs{the}\meta{CNT} macro is restored using the
%    main counter.
%    \begin{macrocode}
\newcommand*{\aliascntresetthe}[1]{%
  \@ifundefined{AC@cnt@#1}{%
    \PackageError{aliascnt}{%
      `#1' is not an alias counter%
    }\@ehc
  }{%
    \expandafter\let\csname the#1\expandafter\endcsname
      \csname the\csname AC@cnt@#1\endcsname\endcsname
  }%
}
%    \end{macrocode}
%    \end{macro}
%
% \subsection{Counter clear list}
%
%    The alias counters share the same register and clear list.
%    Therefore we must ensure that manipulations to the clear list
%    are done with the clear list macro of a real counter.
%    \begin{macro}{\AC@findrootcnt}
%    \cs{AC@findrootcnt} walks throught the aliasing relations
%    to find the base counter.
%    \begin{macrocode}
\newcommand*{\AC@findrootcnt}[1]{%
  \@ifundefined{AC@cnt@#1}{%
    #1%
  }{%
    \expandafter\AC@findrootcnt\csname AC@cnt@#1\endcsname
  }%
}
%    \end{macrocode}
%    \end{macro}
%
%    Clear lists are manipulated by \cs{@addtoreset} and
%    \cs{@removefromreset}. The latter one is provided by
%    the \xpackage{remreset} package (\cite{remreset}).
%
%    \begin{macro}{\AC@patch}
%    The same patch principle is applicable to both
%    \cs{@addtoreset} and \cs{@removefromreset}.
%    \begin{macrocode}
\def\AC@patch#1{%
  \expandafter\let\csname AC@org@#1reset\expandafter\endcsname
    \csname @#1reset\endcsname
  \expandafter\def\csname @#1reset\endcsname##1##2{%
    \csname AC@org@#1reset\endcsname{##1}{\AC@findrootcnt{##2}}%
  }%
}
%    \end{macrocode}
%    \end{macro}
%    If \xpackage{remreset} is not loaded we cannot delay
%    the patch to \cs{AtBeginDocumen}, because \cs{@removefromreset}
%    can be called in between. Therefore we force the loading of
%    the package.
%    \begin{macrocode}
\RequirePackage{remreset}
\AC@patch{addto}
\AC@patch{removefrom}
%    \end{macrocode}
%
%    \begin{macrocode}
%</package>
%    \end{macrocode}
%
% \section{Installation}
%
% \subsection{Download}
%
% \paragraph{Package.} This package is available on
% CTAN\footnote{\url{ftp://ftp.ctan.org/tex-archive/}}:
% \begin{description}
% \item[\CTAN{macros/latex/contrib/oberdiek/aliascnt.dtx}] The source file.
% \item[\CTAN{macros/latex/contrib/oberdiek/aliascnt.pdf}] Documentation.
% \end{description}
%
%
% \paragraph{Bundle.} All the packages of the bundle `oberdiek'
% are also available in a TDS compliant ZIP archive. There
% the packages are already unpacked and the documentation files
% are generated. The files and directories obey the TDS standard.
% \begin{description}
% \item[\CTAN{install/macros/latex/contrib/oberdiek.tds.zip}]
% \end{description}
% \emph{TDS} refers to the standard ``A Directory Structure
% for \TeX\ Files'' (\CTAN{tds/tds.pdf}). Directories
% with \xfile{texmf} in their name are usually organized this way.
%
% \subsection{Bundle installation}
%
% \paragraph{Unpacking.} Unpack the \xfile{oberdiek.tds.zip} in the
% TDS tree (also known as \xfile{texmf} tree) of your choice.
% Example (linux):
% \begin{quote}
%   |unzip oberdiek.tds.zip -d ~/texmf|
% \end{quote}
%
% \paragraph{Script installation.}
% Check the directory \xfile{TDS:scripts/oberdiek/} for
% scripts that need further installation steps.
% Package \xpackage{attachfile2} comes with the Perl script
% \xfile{pdfatfi.pl} that should be installed in such a way
% that it can be called as \texttt{pdfatfi}.
% Example (linux):
% \begin{quote}
%   |chmod +x scripts/oberdiek/pdfatfi.pl|\\
%   |cp scripts/oberdiek/pdfatfi.pl /usr/local/bin/|
% \end{quote}
%
% \subsection{Package installation}
%
% \paragraph{Unpacking.} The \xfile{.dtx} file is a self-extracting
% \docstrip\ archive. The files are extracted by running the
% \xfile{.dtx} through \plainTeX:
% \begin{quote}
%   \verb|tex aliascnt.dtx|
% \end{quote}
%
% \paragraph{TDS.} Now the different files must be moved into
% the different directories in your installation TDS tree
% (also known as \xfile{texmf} tree):
% \begin{quote}
% \def\t{^^A
% \begin{tabular}{@{}>{\ttfamily}l@{ $\rightarrow$ }>{\ttfamily}l@{}}
%   aliascnt.sty & tex/latex/oberdiek/aliascnt.sty\\
%   aliascnt.pdf & doc/latex/oberdiek/aliascnt.pdf\\
%   aliascnt.dtx & source/latex/oberdiek/aliascnt.dtx\\
% \end{tabular}^^A
% }^^A
% \sbox0{\t}^^A
% \ifdim\wd0>\linewidth
%   \begingroup
%     \advance\linewidth by\leftmargin
%     \advance\linewidth by\rightmargin
%   \edef\x{\endgroup
%     \def\noexpand\lw{\the\linewidth}^^A
%   }\x
%   \def\lwbox{^^A
%     \leavevmode
%     \hbox to \linewidth{^^A
%       \kern-\leftmargin\relax
%       \hss
%       \usebox0
%       \hss
%       \kern-\rightmargin\relax
%     }^^A
%   }^^A
%   \ifdim\wd0>\lw
%     \sbox0{\small\t}^^A
%     \ifdim\wd0>\linewidth
%       \ifdim\wd0>\lw
%         \sbox0{\footnotesize\t}^^A
%         \ifdim\wd0>\linewidth
%           \ifdim\wd0>\lw
%             \sbox0{\scriptsize\t}^^A
%             \ifdim\wd0>\linewidth
%               \ifdim\wd0>\lw
%                 \sbox0{\tiny\t}^^A
%                 \ifdim\wd0>\linewidth
%                   \lwbox
%                 \else
%                   \usebox0
%                 \fi
%               \else
%                 \lwbox
%               \fi
%             \else
%               \usebox0
%             \fi
%           \else
%             \lwbox
%           \fi
%         \else
%           \usebox0
%         \fi
%       \else
%         \lwbox
%       \fi
%     \else
%       \usebox0
%     \fi
%   \else
%     \lwbox
%   \fi
% \else
%   \usebox0
% \fi
% \end{quote}
% If you have a \xfile{docstrip.cfg} that configures and enables \docstrip's
% TDS installing feature, then some files can already be in the right
% place, see the documentation of \docstrip.
%
% \subsection{Refresh file name databases}
%
% If your \TeX~distribution
% (\teTeX, \mikTeX, \dots) relies on file name databases, you must refresh
% these. For example, \teTeX\ users run \verb|texhash| or
% \verb|mktexlsr|.
%
% \subsection{Some details for the interested}
%
% \paragraph{Attached source.}
%
% The PDF documentation on CTAN also includes the
% \xfile{.dtx} source file. It can be extracted by
% AcrobatReader 6 or higher. Another option is \textsf{pdftk},
% e.g. unpack the file into the current directory:
% \begin{quote}
%   \verb|pdftk aliascnt.pdf unpack_files output .|
% \end{quote}
%
% \paragraph{Unpacking with \LaTeX.}
% The \xfile{.dtx} chooses its action depending on the format:
% \begin{description}
% \item[\plainTeX:] Run \docstrip\ and extract the files.
% \item[\LaTeX:] Generate the documentation.
% \end{description}
% If you insist on using \LaTeX\ for \docstrip\ (really,
% \docstrip\ does not need \LaTeX), then inform the autodetect routine
% about your intention:
% \begin{quote}
%   \verb|latex \let\install=y\input{aliascnt.dtx}|
% \end{quote}
% Do not forget to quote the argument according to the demands
% of your shell.
%
% \paragraph{Generating the documentation.}
% You can use both the \xfile{.dtx} or the \xfile{.drv} to generate
% the documentation. The process can be configured by the
% configuration file \xfile{ltxdoc.cfg}. For instance, put this
% line into this file, if you want to have A4 as paper format:
% \begin{quote}
%   \verb|\PassOptionsToClass{a4paper}{article}|
% \end{quote}
% An example follows how to generate the
% documentation with pdf\LaTeX:
% \begin{quote}
%\begin{verbatim}
%pdflatex aliascnt.dtx
%makeindex -s gind.ist aliascnt.idx
%pdflatex aliascnt.dtx
%makeindex -s gind.ist aliascnt.idx
%pdflatex aliascnt.dtx
%\end{verbatim}
% \end{quote}
%
% \section{Catalogue}
%
% The following XML file can be used as source for the
% \href{http://mirror.ctan.org/help/Catalogue/catalogue.html}{\TeX\ Catalogue}.
% The elements \texttt{caption} and \texttt{description} are imported
% from the original XML file from the Catalogue.
% The name of the XML file in the Catalogue is \xfile{aliascnt.xml}.
%    \begin{macrocode}
%<*catalogue>
<?xml version='1.0' encoding='us-ascii'?>
<!DOCTYPE entry SYSTEM 'catalogue.dtd'>
<entry datestamp='$Date$' modifier='$Author$' id='aliascnt'>
  <name>aliascnt</name>
  <caption>Alias counters.</caption>
  <authorref id='auth:oberdiek'/>
  <copyright owner='Heiko Oberdiek' year='2006,2009'/>
  <license type='lppl1.3'/>
  <version number='1.3'/>
  <description>
    This package introduces aliases for counters, that
    share the same counter register and clear list.
    <p/>
    The package is part of the <xref refid='oberdiek'>oberdiek</xref>
    bundle.
  </description>
  <documentation details='Package documentation'
      href='ctan:/macros/latex/contrib/oberdiek/aliascnt.pdf'/>
  <ctan file='true' path='/macros/latex/contrib/oberdiek/aliascnt.dtx'/>
  <miktex location='oberdiek'/>
  <texlive location='oberdiek'/>
  <install path='/macros/latex/contrib/oberdiek/oberdiek.tds.zip'/>
</entry>
%</catalogue>
%    \end{macrocode}
%
% \section{Acknowledgement}
%
% \begin{description}
% \item[Ulrich Schwarz:] The package is based on his draft for
%   ``Die \TeX nische Kom\"odie'', see \cite{schwarz}.
% \end{description}
%
% \begin{thebibliography}{9}
%
% \bibitem{schwarz}
%   Ulrich Schwarz:
%   \textit{Was hinten herauskommt z\"ahlt: Counter Aliasing in \LaTeX},
%   \textit{Die \TeX nische Kom\"odie}, 3/2006, pages 8--14, Juli 2006.
%
% \bibitem{remreset}
%   David Carlisle: \textit{The \xpackage{remreset} package};
%   1997/09/28;
%   \CTAN{macros/latex/contrib/carlisle/remreset.sty}.
%
% \bibitem{hyperref}
%   Sebastian Rahtz, Heiko Oberdiek:
%   \textit{The \xpackage{hyperref} package};
%   2006/08/16 v6.75c;
%   \CTAN{macros/latex/contrib/hyperref/}.
%
% \end{thebibliography}
%
% \begin{History}
%   \begin{Version}{2006/02/20 v1.0}
%   \item
%     First version.
%   \end{Version}
%   \begin{Version}{2006/08/16 v1.1}
%   \item
%     Update of bibliography.
%   \end{Version}
%   \begin{Version}{2006/09/25 v1.2}
%   \item
%     Bug fix (\cs{aliascntresetthe}).
%   \end{Version}
%   \begin{Version}{2009/09/08 v1.3}
%   \item
%     Bug fix of \cs{@ifdefinable}'s use (thanks to Uwe L\"uck).
%   \end{Version}
% \end{History}
%
% \PrintIndex
%
% \Finale
\endinput

%        (quote the arguments according to the demands of your shell)
%
% Documentation:
%    (a) If aliascnt.drv is present:
%           latex aliascnt.drv
%    (b) Without aliascnt.drv:
%           latex aliascnt.dtx; ...
%    The class ltxdoc loads the configuration file ltxdoc.cfg
%    if available. Here you can specify further options, e.g.
%    use A4 as paper format:
%       \PassOptionsToClass{a4paper}{article}
%
%    Programm calls to get the documentation (example):
%       pdflatex aliascnt.dtx
%       makeindex -s gind.ist aliascnt.idx
%       pdflatex aliascnt.dtx
%       makeindex -s gind.ist aliascnt.idx
%       pdflatex aliascnt.dtx
%
% Installation:
%    TDS:tex/latex/oberdiek/aliascnt.sty
%    TDS:doc/latex/oberdiek/aliascnt.pdf
%    TDS:source/latex/oberdiek/aliascnt.dtx
%
%<*ignore>
\begingroup
  \catcode123=1 %
  \catcode125=2 %
  \def\x{LaTeX2e}%
\expandafter\endgroup
\ifcase 0\ifx\install y1\fi\expandafter
         \ifx\csname processbatchFile\endcsname\relax\else1\fi
         \ifx\fmtname\x\else 1\fi\relax
\else\csname fi\endcsname
%</ignore>
%<*install>
\input docstrip.tex
\Msg{************************************************************************}
\Msg{* Installation}
\Msg{* Package: aliascnt 2009/09/08 v1.3 Alias counters (HO)}
\Msg{************************************************************************}

\keepsilent
\askforoverwritefalse

\let\MetaPrefix\relax
\preamble

This is a generated file.

Project: aliascnt
Version: 2009/09/08 v1.3

Copyright (C) 2006, 2009 by
   Heiko Oberdiek <heiko.oberdiek at googlemail.com>

This work may be distributed and/or modified under the
conditions of the LaTeX Project Public License, either
version 1.3c of this license or (at your option) any later
version. This version of this license is in
   http://www.latex-project.org/lppl/lppl-1-3c.txt
and the latest version of this license is in
   http://www.latex-project.org/lppl.txt
and version 1.3 or later is part of all distributions of
LaTeX version 2005/12/01 or later.

This work has the LPPL maintenance status "maintained".

This Current Maintainer of this work is Heiko Oberdiek.

This work consists of the main source file aliascnt.dtx
and the derived files
   aliascnt.sty, aliascnt.pdf, aliascnt.ins, aliascnt.drv.

\endpreamble
\let\MetaPrefix\DoubleperCent

\generate{%
  \file{aliascnt.ins}{\from{aliascnt.dtx}{install}}%
  \file{aliascnt.drv}{\from{aliascnt.dtx}{driver}}%
  \usedir{tex/latex/oberdiek}%
  \file{aliascnt.sty}{\from{aliascnt.dtx}{package}}%
  \nopreamble
  \nopostamble
  \usedir{source/latex/oberdiek/catalogue}%
  \file{aliascnt.xml}{\from{aliascnt.dtx}{catalogue}}%
}

\catcode32=13\relax% active space
\let =\space%
\Msg{************************************************************************}
\Msg{*}
\Msg{* To finish the installation you have to move the following}
\Msg{* file into a directory searched by TeX:}
\Msg{*}
\Msg{*     aliascnt.sty}
\Msg{*}
\Msg{* To produce the documentation run the file `aliascnt.drv'}
\Msg{* through LaTeX.}
\Msg{*}
\Msg{* Happy TeXing!}
\Msg{*}
\Msg{************************************************************************}

\endbatchfile
%</install>
%<*ignore>
\fi
%</ignore>
%<*driver>
\NeedsTeXFormat{LaTeX2e}
\ProvidesFile{aliascnt.drv}%
  [2009/09/08 v1.3 Alias counters (HO)]%
\documentclass{ltxdoc}
\usepackage{holtxdoc}[2011/11/22]
\begin{document}
  \DocInput{aliascnt.dtx}%
\end{document}
%</driver>
% \fi
%
% \CheckSum{78}
%
% \CharacterTable
%  {Upper-case    \A\B\C\D\E\F\G\H\I\J\K\L\M\N\O\P\Q\R\S\T\U\V\W\X\Y\Z
%   Lower-case    \a\b\c\d\e\f\g\h\i\j\k\l\m\n\o\p\q\r\s\t\u\v\w\x\y\z
%   Digits        \0\1\2\3\4\5\6\7\8\9
%   Exclamation   \!     Double quote  \"     Hash (number) \#
%   Dollar        \$     Percent       \%     Ampersand     \&
%   Acute accent  \'     Left paren    \(     Right paren   \)
%   Asterisk      \*     Plus          \+     Comma         \,
%   Minus         \-     Point         \.     Solidus       \/
%   Colon         \:     Semicolon     \;     Less than     \<
%   Equals        \=     Greater than  \>     Question mark \?
%   Commercial at \@     Left bracket  \[     Backslash     \\
%   Right bracket \]     Circumflex    \^     Underscore    \_
%   Grave accent  \`     Left brace    \{     Vertical bar  \|
%   Right brace   \}     Tilde         \~}
%
% \GetFileInfo{aliascnt.drv}
%
% \title{The \xpackage{aliascnt} package}
% \date{2009/09/08 v1.3}
% \author{Heiko Oberdiek\\\xemail{heiko.oberdiek at googlemail.com}}
%
% \maketitle
%
% \begin{abstract}
% Package \xpackage{aliascnt} introduces \emph{alias counters} that
% share the same counter register and clear list.
% \end{abstract}
%
% \tableofcontents
%
% \section{User interface}
%
% \subsection{Introduction}
%
% There are features that rely on the name of counters. For
% example, \xpackage{hyperref}'s \cs{autoref} indirectly uses
% the counter name to determine which label text it puts in front
% of the reference number (\cite{hyperref}).
% In some circumstances this fail: several theorem environments
% are defined by \cs{newtheorem} that share the same counter.
%
% \subsection{Syntax}
%
% Macro names in user land contain the package name
% \texttt{aliascnt} in order to prevent name clashes.
%
% \newenvironment{desc}{^^A
%   \list{}{^^A
%     \setlength{\labelwidth}{0pt}^^A
%     \setlength{\itemindent}{-.5\marginparwidth}^^A
%     \setlength{\leftmargin}{0pt}^^A
%     \let\makelabel\desclabel
%   }^^A
% }{^^A
%   \endlist
% }
% \newcommand*{\desclabel}[1]{^^A
%   \hspace{\labelsep}^^A
%   \normalfont
%   #1^^A
% }
% \newcommand*{\itemcs}[2]{^^A
%   \item[^^A
%      \expandafter\SpecialUsageIndex\csname #1\endcsname
%      {\cs{#1}#2}^^A
%   ]\mbox{}\\*[.5ex]^^A
%   \ignorespaces
% }
% \begin{desc}
% \itemcs{newaliascnt}{\marg{ALIASCNT}\marg{BASECNT}}
%    An alias counter ALIASCNT is created that does not allocate
%    a new \TeX\ counter register. It shares the count register and
%    the clear list with counter BASECNT. If the value of either
%    the two registers is changed, the changes affects both.
% \itemcs{aliascntresetthe}{\marg{ALIASCNT}}
%    This fixes a problem with \cs{newtheorem} if it
%    is fooled by an alias counter with the same name:
%    \begin{quote}
%\begin{verbatim}
%\newtheorem{foo}{Foo}% counter "foo"
%\newaliascnt{bar}{foo}% alias counter "bar"
%\newtheorem{bar}[bar]{Bar}
%\aliascntresetthe{bar}
%\end{verbatim}
%    \end{quote}
% \end{desc}
%
% \StopEventually{
% }
%
% \section{Implementation}
%
% \subsection{Identification}
%
%    \begin{macrocode}
%<*package>
\NeedsTeXFormat{LaTeX2e}
\ProvidesPackage{aliascnt}%
  [2009/09/08 v1.3 Alias counters (HO)]%
%    \end{macrocode}
%
% \subsection{Create new alias counter}
%
%    \begin{macro}{\newaliascnt}
%    A new alias counter is set up by \cs{newaliascnt}.
%    The following properties are added for the new counter CNT:
%    \begin{description}
%    \item[\mdseries\cs{theH}\meta{CNT}:] Compatibility for \xpackage{hyperref}
%    \item[\mdseries\cs{AC@cnt@}\meta{CNT}:] Name of the referenced counter
%      in the definition.
%    \end{description}
%    \begin{macrocode}
\newcommand*{\newaliascnt}[2]{%
  \begingroup
    \def\AC@glet##1{%
      \global\expandafter\let\csname##1#1\expandafter\endcsname
        \csname##1#2\endcsname
    }%
    \@ifundefined{c@#2}{%
      \@nocounterr{#2}%
    }{%
      \expandafter\@ifdefinable\csname c@#1\endcsname{%
        \AC@glet{c@}%
        \AC@glet{the}%
        \AC@glet{theH}%
        \AC@glet{p@}%
        \expandafter\gdef\csname AC@cnt@#1\endcsname{#2}%
        \expandafter\gdef\csname cl@#1\expandafter\endcsname
        \expandafter{\csname cl@#2\endcsname}%
      }%
    }%
  \endgroup
}
%    \end{macrocode}
%    \end{macro}
%
%    \begin{macro}{\aliascntresetthe}
%    The \cs{the}\meta{CNT} macro is restored using the
%    main counter.
%    \begin{macrocode}
\newcommand*{\aliascntresetthe}[1]{%
  \@ifundefined{AC@cnt@#1}{%
    \PackageError{aliascnt}{%
      `#1' is not an alias counter%
    }\@ehc
  }{%
    \expandafter\let\csname the#1\expandafter\endcsname
      \csname the\csname AC@cnt@#1\endcsname\endcsname
  }%
}
%    \end{macrocode}
%    \end{macro}
%
% \subsection{Counter clear list}
%
%    The alias counters share the same register and clear list.
%    Therefore we must ensure that manipulations to the clear list
%    are done with the clear list macro of a real counter.
%    \begin{macro}{\AC@findrootcnt}
%    \cs{AC@findrootcnt} walks throught the aliasing relations
%    to find the base counter.
%    \begin{macrocode}
\newcommand*{\AC@findrootcnt}[1]{%
  \@ifundefined{AC@cnt@#1}{%
    #1%
  }{%
    \expandafter\AC@findrootcnt\csname AC@cnt@#1\endcsname
  }%
}
%    \end{macrocode}
%    \end{macro}
%
%    Clear lists are manipulated by \cs{@addtoreset} and
%    \cs{@removefromreset}. The latter one is provided by
%    the \xpackage{remreset} package (\cite{remreset}).
%
%    \begin{macro}{\AC@patch}
%    The same patch principle is applicable to both
%    \cs{@addtoreset} and \cs{@removefromreset}.
%    \begin{macrocode}
\def\AC@patch#1{%
  \expandafter\let\csname AC@org@#1reset\expandafter\endcsname
    \csname @#1reset\endcsname
  \expandafter\def\csname @#1reset\endcsname##1##2{%
    \csname AC@org@#1reset\endcsname{##1}{\AC@findrootcnt{##2}}%
  }%
}
%    \end{macrocode}
%    \end{macro}
%    If \xpackage{remreset} is not loaded we cannot delay
%    the patch to \cs{AtBeginDocumen}, because \cs{@removefromreset}
%    can be called in between. Therefore we force the loading of
%    the package.
%    \begin{macrocode}
\RequirePackage{remreset}
\AC@patch{addto}
\AC@patch{removefrom}
%    \end{macrocode}
%
%    \begin{macrocode}
%</package>
%    \end{macrocode}
%
% \section{Installation}
%
% \subsection{Download}
%
% \paragraph{Package.} This package is available on
% CTAN\footnote{\url{ftp://ftp.ctan.org/tex-archive/}}:
% \begin{description}
% \item[\CTAN{macros/latex/contrib/oberdiek/aliascnt.dtx}] The source file.
% \item[\CTAN{macros/latex/contrib/oberdiek/aliascnt.pdf}] Documentation.
% \end{description}
%
%
% \paragraph{Bundle.} All the packages of the bundle `oberdiek'
% are also available in a TDS compliant ZIP archive. There
% the packages are already unpacked and the documentation files
% are generated. The files and directories obey the TDS standard.
% \begin{description}
% \item[\CTAN{install/macros/latex/contrib/oberdiek.tds.zip}]
% \end{description}
% \emph{TDS} refers to the standard ``A Directory Structure
% for \TeX\ Files'' (\CTAN{tds/tds.pdf}). Directories
% with \xfile{texmf} in their name are usually organized this way.
%
% \subsection{Bundle installation}
%
% \paragraph{Unpacking.} Unpack the \xfile{oberdiek.tds.zip} in the
% TDS tree (also known as \xfile{texmf} tree) of your choice.
% Example (linux):
% \begin{quote}
%   |unzip oberdiek.tds.zip -d ~/texmf|
% \end{quote}
%
% \paragraph{Script installation.}
% Check the directory \xfile{TDS:scripts/oberdiek/} for
% scripts that need further installation steps.
% Package \xpackage{attachfile2} comes with the Perl script
% \xfile{pdfatfi.pl} that should be installed in such a way
% that it can be called as \texttt{pdfatfi}.
% Example (linux):
% \begin{quote}
%   |chmod +x scripts/oberdiek/pdfatfi.pl|\\
%   |cp scripts/oberdiek/pdfatfi.pl /usr/local/bin/|
% \end{quote}
%
% \subsection{Package installation}
%
% \paragraph{Unpacking.} The \xfile{.dtx} file is a self-extracting
% \docstrip\ archive. The files are extracted by running the
% \xfile{.dtx} through \plainTeX:
% \begin{quote}
%   \verb|tex aliascnt.dtx|
% \end{quote}
%
% \paragraph{TDS.} Now the different files must be moved into
% the different directories in your installation TDS tree
% (also known as \xfile{texmf} tree):
% \begin{quote}
% \def\t{^^A
% \begin{tabular}{@{}>{\ttfamily}l@{ $\rightarrow$ }>{\ttfamily}l@{}}
%   aliascnt.sty & tex/latex/oberdiek/aliascnt.sty\\
%   aliascnt.pdf & doc/latex/oberdiek/aliascnt.pdf\\
%   aliascnt.dtx & source/latex/oberdiek/aliascnt.dtx\\
% \end{tabular}^^A
% }^^A
% \sbox0{\t}^^A
% \ifdim\wd0>\linewidth
%   \begingroup
%     \advance\linewidth by\leftmargin
%     \advance\linewidth by\rightmargin
%   \edef\x{\endgroup
%     \def\noexpand\lw{\the\linewidth}^^A
%   }\x
%   \def\lwbox{^^A
%     \leavevmode
%     \hbox to \linewidth{^^A
%       \kern-\leftmargin\relax
%       \hss
%       \usebox0
%       \hss
%       \kern-\rightmargin\relax
%     }^^A
%   }^^A
%   \ifdim\wd0>\lw
%     \sbox0{\small\t}^^A
%     \ifdim\wd0>\linewidth
%       \ifdim\wd0>\lw
%         \sbox0{\footnotesize\t}^^A
%         \ifdim\wd0>\linewidth
%           \ifdim\wd0>\lw
%             \sbox0{\scriptsize\t}^^A
%             \ifdim\wd0>\linewidth
%               \ifdim\wd0>\lw
%                 \sbox0{\tiny\t}^^A
%                 \ifdim\wd0>\linewidth
%                   \lwbox
%                 \else
%                   \usebox0
%                 \fi
%               \else
%                 \lwbox
%               \fi
%             \else
%               \usebox0
%             \fi
%           \else
%             \lwbox
%           \fi
%         \else
%           \usebox0
%         \fi
%       \else
%         \lwbox
%       \fi
%     \else
%       \usebox0
%     \fi
%   \else
%     \lwbox
%   \fi
% \else
%   \usebox0
% \fi
% \end{quote}
% If you have a \xfile{docstrip.cfg} that configures and enables \docstrip's
% TDS installing feature, then some files can already be in the right
% place, see the documentation of \docstrip.
%
% \subsection{Refresh file name databases}
%
% If your \TeX~distribution
% (\teTeX, \mikTeX, \dots) relies on file name databases, you must refresh
% these. For example, \teTeX\ users run \verb|texhash| or
% \verb|mktexlsr|.
%
% \subsection{Some details for the interested}
%
% \paragraph{Attached source.}
%
% The PDF documentation on CTAN also includes the
% \xfile{.dtx} source file. It can be extracted by
% AcrobatReader 6 or higher. Another option is \textsf{pdftk},
% e.g. unpack the file into the current directory:
% \begin{quote}
%   \verb|pdftk aliascnt.pdf unpack_files output .|
% \end{quote}
%
% \paragraph{Unpacking with \LaTeX.}
% The \xfile{.dtx} chooses its action depending on the format:
% \begin{description}
% \item[\plainTeX:] Run \docstrip\ and extract the files.
% \item[\LaTeX:] Generate the documentation.
% \end{description}
% If you insist on using \LaTeX\ for \docstrip\ (really,
% \docstrip\ does not need \LaTeX), then inform the autodetect routine
% about your intention:
% \begin{quote}
%   \verb|latex \let\install=y% \iffalse meta-comment
%
% File: aliascnt.dtx
% Version: 2009/09/08 v1.3
% Info: Alias counters
%
% Copyright (C) 2006, 2009 by
%    Heiko Oberdiek <heiko.oberdiek at googlemail.com>
%
% This work may be distributed and/or modified under the
% conditions of the LaTeX Project Public License, either
% version 1.3c of this license or (at your option) any later
% version. This version of this license is in
%    http://www.latex-project.org/lppl/lppl-1-3c.txt
% and the latest version of this license is in
%    http://www.latex-project.org/lppl.txt
% and version 1.3 or later is part of all distributions of
% LaTeX version 2005/12/01 or later.
%
% This work has the LPPL maintenance status "maintained".
%
% This Current Maintainer of this work is Heiko Oberdiek.
%
% This work consists of the main source file aliascnt.dtx
% and the derived files
%    aliascnt.sty, aliascnt.pdf, aliascnt.ins, aliascnt.drv.
%
% Distribution:
%    CTAN:macros/latex/contrib/oberdiek/aliascnt.dtx
%    CTAN:macros/latex/contrib/oberdiek/aliascnt.pdf
%
% Unpacking:
%    (a) If aliascnt.ins is present:
%           tex aliascnt.ins
%    (b) Without aliascnt.ins:
%           tex aliascnt.dtx
%    (c) If you insist on using LaTeX
%           latex \let\install=y\input{aliascnt.dtx}
%        (quote the arguments according to the demands of your shell)
%
% Documentation:
%    (a) If aliascnt.drv is present:
%           latex aliascnt.drv
%    (b) Without aliascnt.drv:
%           latex aliascnt.dtx; ...
%    The class ltxdoc loads the configuration file ltxdoc.cfg
%    if available. Here you can specify further options, e.g.
%    use A4 as paper format:
%       \PassOptionsToClass{a4paper}{article}
%
%    Programm calls to get the documentation (example):
%       pdflatex aliascnt.dtx
%       makeindex -s gind.ist aliascnt.idx
%       pdflatex aliascnt.dtx
%       makeindex -s gind.ist aliascnt.idx
%       pdflatex aliascnt.dtx
%
% Installation:
%    TDS:tex/latex/oberdiek/aliascnt.sty
%    TDS:doc/latex/oberdiek/aliascnt.pdf
%    TDS:source/latex/oberdiek/aliascnt.dtx
%
%<*ignore>
\begingroup
  \catcode123=1 %
  \catcode125=2 %
  \def\x{LaTeX2e}%
\expandafter\endgroup
\ifcase 0\ifx\install y1\fi\expandafter
         \ifx\csname processbatchFile\endcsname\relax\else1\fi
         \ifx\fmtname\x\else 1\fi\relax
\else\csname fi\endcsname
%</ignore>
%<*install>
\input docstrip.tex
\Msg{************************************************************************}
\Msg{* Installation}
\Msg{* Package: aliascnt 2009/09/08 v1.3 Alias counters (HO)}
\Msg{************************************************************************}

\keepsilent
\askforoverwritefalse

\let\MetaPrefix\relax
\preamble

This is a generated file.

Project: aliascnt
Version: 2009/09/08 v1.3

Copyright (C) 2006, 2009 by
   Heiko Oberdiek <heiko.oberdiek at googlemail.com>

This work may be distributed and/or modified under the
conditions of the LaTeX Project Public License, either
version 1.3c of this license or (at your option) any later
version. This version of this license is in
   http://www.latex-project.org/lppl/lppl-1-3c.txt
and the latest version of this license is in
   http://www.latex-project.org/lppl.txt
and version 1.3 or later is part of all distributions of
LaTeX version 2005/12/01 or later.

This work has the LPPL maintenance status "maintained".

This Current Maintainer of this work is Heiko Oberdiek.

This work consists of the main source file aliascnt.dtx
and the derived files
   aliascnt.sty, aliascnt.pdf, aliascnt.ins, aliascnt.drv.

\endpreamble
\let\MetaPrefix\DoubleperCent

\generate{%
  \file{aliascnt.ins}{\from{aliascnt.dtx}{install}}%
  \file{aliascnt.drv}{\from{aliascnt.dtx}{driver}}%
  \usedir{tex/latex/oberdiek}%
  \file{aliascnt.sty}{\from{aliascnt.dtx}{package}}%
  \nopreamble
  \nopostamble
  \usedir{source/latex/oberdiek/catalogue}%
  \file{aliascnt.xml}{\from{aliascnt.dtx}{catalogue}}%
}

\catcode32=13\relax% active space
\let =\space%
\Msg{************************************************************************}
\Msg{*}
\Msg{* To finish the installation you have to move the following}
\Msg{* file into a directory searched by TeX:}
\Msg{*}
\Msg{*     aliascnt.sty}
\Msg{*}
\Msg{* To produce the documentation run the file `aliascnt.drv'}
\Msg{* through LaTeX.}
\Msg{*}
\Msg{* Happy TeXing!}
\Msg{*}
\Msg{************************************************************************}

\endbatchfile
%</install>
%<*ignore>
\fi
%</ignore>
%<*driver>
\NeedsTeXFormat{LaTeX2e}
\ProvidesFile{aliascnt.drv}%
  [2009/09/08 v1.3 Alias counters (HO)]%
\documentclass{ltxdoc}
\usepackage{holtxdoc}[2011/11/22]
\begin{document}
  \DocInput{aliascnt.dtx}%
\end{document}
%</driver>
% \fi
%
% \CheckSum{78}
%
% \CharacterTable
%  {Upper-case    \A\B\C\D\E\F\G\H\I\J\K\L\M\N\O\P\Q\R\S\T\U\V\W\X\Y\Z
%   Lower-case    \a\b\c\d\e\f\g\h\i\j\k\l\m\n\o\p\q\r\s\t\u\v\w\x\y\z
%   Digits        \0\1\2\3\4\5\6\7\8\9
%   Exclamation   \!     Double quote  \"     Hash (number) \#
%   Dollar        \$     Percent       \%     Ampersand     \&
%   Acute accent  \'     Left paren    \(     Right paren   \)
%   Asterisk      \*     Plus          \+     Comma         \,
%   Minus         \-     Point         \.     Solidus       \/
%   Colon         \:     Semicolon     \;     Less than     \<
%   Equals        \=     Greater than  \>     Question mark \?
%   Commercial at \@     Left bracket  \[     Backslash     \\
%   Right bracket \]     Circumflex    \^     Underscore    \_
%   Grave accent  \`     Left brace    \{     Vertical bar  \|
%   Right brace   \}     Tilde         \~}
%
% \GetFileInfo{aliascnt.drv}
%
% \title{The \xpackage{aliascnt} package}
% \date{2009/09/08 v1.3}
% \author{Heiko Oberdiek\\\xemail{heiko.oberdiek at googlemail.com}}
%
% \maketitle
%
% \begin{abstract}
% Package \xpackage{aliascnt} introduces \emph{alias counters} that
% share the same counter register and clear list.
% \end{abstract}
%
% \tableofcontents
%
% \section{User interface}
%
% \subsection{Introduction}
%
% There are features that rely on the name of counters. For
% example, \xpackage{hyperref}'s \cs{autoref} indirectly uses
% the counter name to determine which label text it puts in front
% of the reference number (\cite{hyperref}).
% In some circumstances this fail: several theorem environments
% are defined by \cs{newtheorem} that share the same counter.
%
% \subsection{Syntax}
%
% Macro names in user land contain the package name
% \texttt{aliascnt} in order to prevent name clashes.
%
% \newenvironment{desc}{^^A
%   \list{}{^^A
%     \setlength{\labelwidth}{0pt}^^A
%     \setlength{\itemindent}{-.5\marginparwidth}^^A
%     \setlength{\leftmargin}{0pt}^^A
%     \let\makelabel\desclabel
%   }^^A
% }{^^A
%   \endlist
% }
% \newcommand*{\desclabel}[1]{^^A
%   \hspace{\labelsep}^^A
%   \normalfont
%   #1^^A
% }
% \newcommand*{\itemcs}[2]{^^A
%   \item[^^A
%      \expandafter\SpecialUsageIndex\csname #1\endcsname
%      {\cs{#1}#2}^^A
%   ]\mbox{}\\*[.5ex]^^A
%   \ignorespaces
% }
% \begin{desc}
% \itemcs{newaliascnt}{\marg{ALIASCNT}\marg{BASECNT}}
%    An alias counter ALIASCNT is created that does not allocate
%    a new \TeX\ counter register. It shares the count register and
%    the clear list with counter BASECNT. If the value of either
%    the two registers is changed, the changes affects both.
% \itemcs{aliascntresetthe}{\marg{ALIASCNT}}
%    This fixes a problem with \cs{newtheorem} if it
%    is fooled by an alias counter with the same name:
%    \begin{quote}
%\begin{verbatim}
%\newtheorem{foo}{Foo}% counter "foo"
%\newaliascnt{bar}{foo}% alias counter "bar"
%\newtheorem{bar}[bar]{Bar}
%\aliascntresetthe{bar}
%\end{verbatim}
%    \end{quote}
% \end{desc}
%
% \StopEventually{
% }
%
% \section{Implementation}
%
% \subsection{Identification}
%
%    \begin{macrocode}
%<*package>
\NeedsTeXFormat{LaTeX2e}
\ProvidesPackage{aliascnt}%
  [2009/09/08 v1.3 Alias counters (HO)]%
%    \end{macrocode}
%
% \subsection{Create new alias counter}
%
%    \begin{macro}{\newaliascnt}
%    A new alias counter is set up by \cs{newaliascnt}.
%    The following properties are added for the new counter CNT:
%    \begin{description}
%    \item[\mdseries\cs{theH}\meta{CNT}:] Compatibility for \xpackage{hyperref}
%    \item[\mdseries\cs{AC@cnt@}\meta{CNT}:] Name of the referenced counter
%      in the definition.
%    \end{description}
%    \begin{macrocode}
\newcommand*{\newaliascnt}[2]{%
  \begingroup
    \def\AC@glet##1{%
      \global\expandafter\let\csname##1#1\expandafter\endcsname
        \csname##1#2\endcsname
    }%
    \@ifundefined{c@#2}{%
      \@nocounterr{#2}%
    }{%
      \expandafter\@ifdefinable\csname c@#1\endcsname{%
        \AC@glet{c@}%
        \AC@glet{the}%
        \AC@glet{theH}%
        \AC@glet{p@}%
        \expandafter\gdef\csname AC@cnt@#1\endcsname{#2}%
        \expandafter\gdef\csname cl@#1\expandafter\endcsname
        \expandafter{\csname cl@#2\endcsname}%
      }%
    }%
  \endgroup
}
%    \end{macrocode}
%    \end{macro}
%
%    \begin{macro}{\aliascntresetthe}
%    The \cs{the}\meta{CNT} macro is restored using the
%    main counter.
%    \begin{macrocode}
\newcommand*{\aliascntresetthe}[1]{%
  \@ifundefined{AC@cnt@#1}{%
    \PackageError{aliascnt}{%
      `#1' is not an alias counter%
    }\@ehc
  }{%
    \expandafter\let\csname the#1\expandafter\endcsname
      \csname the\csname AC@cnt@#1\endcsname\endcsname
  }%
}
%    \end{macrocode}
%    \end{macro}
%
% \subsection{Counter clear list}
%
%    The alias counters share the same register and clear list.
%    Therefore we must ensure that manipulations to the clear list
%    are done with the clear list macro of a real counter.
%    \begin{macro}{\AC@findrootcnt}
%    \cs{AC@findrootcnt} walks throught the aliasing relations
%    to find the base counter.
%    \begin{macrocode}
\newcommand*{\AC@findrootcnt}[1]{%
  \@ifundefined{AC@cnt@#1}{%
    #1%
  }{%
    \expandafter\AC@findrootcnt\csname AC@cnt@#1\endcsname
  }%
}
%    \end{macrocode}
%    \end{macro}
%
%    Clear lists are manipulated by \cs{@addtoreset} and
%    \cs{@removefromreset}. The latter one is provided by
%    the \xpackage{remreset} package (\cite{remreset}).
%
%    \begin{macro}{\AC@patch}
%    The same patch principle is applicable to both
%    \cs{@addtoreset} and \cs{@removefromreset}.
%    \begin{macrocode}
\def\AC@patch#1{%
  \expandafter\let\csname AC@org@#1reset\expandafter\endcsname
    \csname @#1reset\endcsname
  \expandafter\def\csname @#1reset\endcsname##1##2{%
    \csname AC@org@#1reset\endcsname{##1}{\AC@findrootcnt{##2}}%
  }%
}
%    \end{macrocode}
%    \end{macro}
%    If \xpackage{remreset} is not loaded we cannot delay
%    the patch to \cs{AtBeginDocumen}, because \cs{@removefromreset}
%    can be called in between. Therefore we force the loading of
%    the package.
%    \begin{macrocode}
\RequirePackage{remreset}
\AC@patch{addto}
\AC@patch{removefrom}
%    \end{macrocode}
%
%    \begin{macrocode}
%</package>
%    \end{macrocode}
%
% \section{Installation}
%
% \subsection{Download}
%
% \paragraph{Package.} This package is available on
% CTAN\footnote{\url{ftp://ftp.ctan.org/tex-archive/}}:
% \begin{description}
% \item[\CTAN{macros/latex/contrib/oberdiek/aliascnt.dtx}] The source file.
% \item[\CTAN{macros/latex/contrib/oberdiek/aliascnt.pdf}] Documentation.
% \end{description}
%
%
% \paragraph{Bundle.} All the packages of the bundle `oberdiek'
% are also available in a TDS compliant ZIP archive. There
% the packages are already unpacked and the documentation files
% are generated. The files and directories obey the TDS standard.
% \begin{description}
% \item[\CTAN{install/macros/latex/contrib/oberdiek.tds.zip}]
% \end{description}
% \emph{TDS} refers to the standard ``A Directory Structure
% for \TeX\ Files'' (\CTAN{tds/tds.pdf}). Directories
% with \xfile{texmf} in their name are usually organized this way.
%
% \subsection{Bundle installation}
%
% \paragraph{Unpacking.} Unpack the \xfile{oberdiek.tds.zip} in the
% TDS tree (also known as \xfile{texmf} tree) of your choice.
% Example (linux):
% \begin{quote}
%   |unzip oberdiek.tds.zip -d ~/texmf|
% \end{quote}
%
% \paragraph{Script installation.}
% Check the directory \xfile{TDS:scripts/oberdiek/} for
% scripts that need further installation steps.
% Package \xpackage{attachfile2} comes with the Perl script
% \xfile{pdfatfi.pl} that should be installed in such a way
% that it can be called as \texttt{pdfatfi}.
% Example (linux):
% \begin{quote}
%   |chmod +x scripts/oberdiek/pdfatfi.pl|\\
%   |cp scripts/oberdiek/pdfatfi.pl /usr/local/bin/|
% \end{quote}
%
% \subsection{Package installation}
%
% \paragraph{Unpacking.} The \xfile{.dtx} file is a self-extracting
% \docstrip\ archive. The files are extracted by running the
% \xfile{.dtx} through \plainTeX:
% \begin{quote}
%   \verb|tex aliascnt.dtx|
% \end{quote}
%
% \paragraph{TDS.} Now the different files must be moved into
% the different directories in your installation TDS tree
% (also known as \xfile{texmf} tree):
% \begin{quote}
% \def\t{^^A
% \begin{tabular}{@{}>{\ttfamily}l@{ $\rightarrow$ }>{\ttfamily}l@{}}
%   aliascnt.sty & tex/latex/oberdiek/aliascnt.sty\\
%   aliascnt.pdf & doc/latex/oberdiek/aliascnt.pdf\\
%   aliascnt.dtx & source/latex/oberdiek/aliascnt.dtx\\
% \end{tabular}^^A
% }^^A
% \sbox0{\t}^^A
% \ifdim\wd0>\linewidth
%   \begingroup
%     \advance\linewidth by\leftmargin
%     \advance\linewidth by\rightmargin
%   \edef\x{\endgroup
%     \def\noexpand\lw{\the\linewidth}^^A
%   }\x
%   \def\lwbox{^^A
%     \leavevmode
%     \hbox to \linewidth{^^A
%       \kern-\leftmargin\relax
%       \hss
%       \usebox0
%       \hss
%       \kern-\rightmargin\relax
%     }^^A
%   }^^A
%   \ifdim\wd0>\lw
%     \sbox0{\small\t}^^A
%     \ifdim\wd0>\linewidth
%       \ifdim\wd0>\lw
%         \sbox0{\footnotesize\t}^^A
%         \ifdim\wd0>\linewidth
%           \ifdim\wd0>\lw
%             \sbox0{\scriptsize\t}^^A
%             \ifdim\wd0>\linewidth
%               \ifdim\wd0>\lw
%                 \sbox0{\tiny\t}^^A
%                 \ifdim\wd0>\linewidth
%                   \lwbox
%                 \else
%                   \usebox0
%                 \fi
%               \else
%                 \lwbox
%               \fi
%             \else
%               \usebox0
%             \fi
%           \else
%             \lwbox
%           \fi
%         \else
%           \usebox0
%         \fi
%       \else
%         \lwbox
%       \fi
%     \else
%       \usebox0
%     \fi
%   \else
%     \lwbox
%   \fi
% \else
%   \usebox0
% \fi
% \end{quote}
% If you have a \xfile{docstrip.cfg} that configures and enables \docstrip's
% TDS installing feature, then some files can already be in the right
% place, see the documentation of \docstrip.
%
% \subsection{Refresh file name databases}
%
% If your \TeX~distribution
% (\teTeX, \mikTeX, \dots) relies on file name databases, you must refresh
% these. For example, \teTeX\ users run \verb|texhash| or
% \verb|mktexlsr|.
%
% \subsection{Some details for the interested}
%
% \paragraph{Attached source.}
%
% The PDF documentation on CTAN also includes the
% \xfile{.dtx} source file. It can be extracted by
% AcrobatReader 6 or higher. Another option is \textsf{pdftk},
% e.g. unpack the file into the current directory:
% \begin{quote}
%   \verb|pdftk aliascnt.pdf unpack_files output .|
% \end{quote}
%
% \paragraph{Unpacking with \LaTeX.}
% The \xfile{.dtx} chooses its action depending on the format:
% \begin{description}
% \item[\plainTeX:] Run \docstrip\ and extract the files.
% \item[\LaTeX:] Generate the documentation.
% \end{description}
% If you insist on using \LaTeX\ for \docstrip\ (really,
% \docstrip\ does not need \LaTeX), then inform the autodetect routine
% about your intention:
% \begin{quote}
%   \verb|latex \let\install=y\input{aliascnt.dtx}|
% \end{quote}
% Do not forget to quote the argument according to the demands
% of your shell.
%
% \paragraph{Generating the documentation.}
% You can use both the \xfile{.dtx} or the \xfile{.drv} to generate
% the documentation. The process can be configured by the
% configuration file \xfile{ltxdoc.cfg}. For instance, put this
% line into this file, if you want to have A4 as paper format:
% \begin{quote}
%   \verb|\PassOptionsToClass{a4paper}{article}|
% \end{quote}
% An example follows how to generate the
% documentation with pdf\LaTeX:
% \begin{quote}
%\begin{verbatim}
%pdflatex aliascnt.dtx
%makeindex -s gind.ist aliascnt.idx
%pdflatex aliascnt.dtx
%makeindex -s gind.ist aliascnt.idx
%pdflatex aliascnt.dtx
%\end{verbatim}
% \end{quote}
%
% \section{Catalogue}
%
% The following XML file can be used as source for the
% \href{http://mirror.ctan.org/help/Catalogue/catalogue.html}{\TeX\ Catalogue}.
% The elements \texttt{caption} and \texttt{description} are imported
% from the original XML file from the Catalogue.
% The name of the XML file in the Catalogue is \xfile{aliascnt.xml}.
%    \begin{macrocode}
%<*catalogue>
<?xml version='1.0' encoding='us-ascii'?>
<!DOCTYPE entry SYSTEM 'catalogue.dtd'>
<entry datestamp='$Date$' modifier='$Author$' id='aliascnt'>
  <name>aliascnt</name>
  <caption>Alias counters.</caption>
  <authorref id='auth:oberdiek'/>
  <copyright owner='Heiko Oberdiek' year='2006,2009'/>
  <license type='lppl1.3'/>
  <version number='1.3'/>
  <description>
    This package introduces aliases for counters, that
    share the same counter register and clear list.
    <p/>
    The package is part of the <xref refid='oberdiek'>oberdiek</xref>
    bundle.
  </description>
  <documentation details='Package documentation'
      href='ctan:/macros/latex/contrib/oberdiek/aliascnt.pdf'/>
  <ctan file='true' path='/macros/latex/contrib/oberdiek/aliascnt.dtx'/>
  <miktex location='oberdiek'/>
  <texlive location='oberdiek'/>
  <install path='/macros/latex/contrib/oberdiek/oberdiek.tds.zip'/>
</entry>
%</catalogue>
%    \end{macrocode}
%
% \section{Acknowledgement}
%
% \begin{description}
% \item[Ulrich Schwarz:] The package is based on his draft for
%   ``Die \TeX nische Kom\"odie'', see \cite{schwarz}.
% \end{description}
%
% \begin{thebibliography}{9}
%
% \bibitem{schwarz}
%   Ulrich Schwarz:
%   \textit{Was hinten herauskommt z\"ahlt: Counter Aliasing in \LaTeX},
%   \textit{Die \TeX nische Kom\"odie}, 3/2006, pages 8--14, Juli 2006.
%
% \bibitem{remreset}
%   David Carlisle: \textit{The \xpackage{remreset} package};
%   1997/09/28;
%   \CTAN{macros/latex/contrib/carlisle/remreset.sty}.
%
% \bibitem{hyperref}
%   Sebastian Rahtz, Heiko Oberdiek:
%   \textit{The \xpackage{hyperref} package};
%   2006/08/16 v6.75c;
%   \CTAN{macros/latex/contrib/hyperref/}.
%
% \end{thebibliography}
%
% \begin{History}
%   \begin{Version}{2006/02/20 v1.0}
%   \item
%     First version.
%   \end{Version}
%   \begin{Version}{2006/08/16 v1.1}
%   \item
%     Update of bibliography.
%   \end{Version}
%   \begin{Version}{2006/09/25 v1.2}
%   \item
%     Bug fix (\cs{aliascntresetthe}).
%   \end{Version}
%   \begin{Version}{2009/09/08 v1.3}
%   \item
%     Bug fix of \cs{@ifdefinable}'s use (thanks to Uwe L\"uck).
%   \end{Version}
% \end{History}
%
% \PrintIndex
%
% \Finale
\endinput
|
% \end{quote}
% Do not forget to quote the argument according to the demands
% of your shell.
%
% \paragraph{Generating the documentation.}
% You can use both the \xfile{.dtx} or the \xfile{.drv} to generate
% the documentation. The process can be configured by the
% configuration file \xfile{ltxdoc.cfg}. For instance, put this
% line into this file, if you want to have A4 as paper format:
% \begin{quote}
%   \verb|\PassOptionsToClass{a4paper}{article}|
% \end{quote}
% An example follows how to generate the
% documentation with pdf\LaTeX:
% \begin{quote}
%\begin{verbatim}
%pdflatex aliascnt.dtx
%makeindex -s gind.ist aliascnt.idx
%pdflatex aliascnt.dtx
%makeindex -s gind.ist aliascnt.idx
%pdflatex aliascnt.dtx
%\end{verbatim}
% \end{quote}
%
% \section{Catalogue}
%
% The following XML file can be used as source for the
% \href{http://mirror.ctan.org/help/Catalogue/catalogue.html}{\TeX\ Catalogue}.
% The elements \texttt{caption} and \texttt{description} are imported
% from the original XML file from the Catalogue.
% The name of the XML file in the Catalogue is \xfile{aliascnt.xml}.
%    \begin{macrocode}
%<*catalogue>
<?xml version='1.0' encoding='us-ascii'?>
<!DOCTYPE entry SYSTEM 'catalogue.dtd'>
<entry datestamp='$Date$' modifier='$Author$' id='aliascnt'>
  <name>aliascnt</name>
  <caption>Alias counters.</caption>
  <authorref id='auth:oberdiek'/>
  <copyright owner='Heiko Oberdiek' year='2006,2009'/>
  <license type='lppl1.3'/>
  <version number='1.3'/>
  <description>
    This package introduces aliases for counters, that
    share the same counter register and clear list.
    <p/>
    The package is part of the <xref refid='oberdiek'>oberdiek</xref>
    bundle.
  </description>
  <documentation details='Package documentation'
      href='ctan:/macros/latex/contrib/oberdiek/aliascnt.pdf'/>
  <ctan file='true' path='/macros/latex/contrib/oberdiek/aliascnt.dtx'/>
  <miktex location='oberdiek'/>
  <texlive location='oberdiek'/>
  <install path='/macros/latex/contrib/oberdiek/oberdiek.tds.zip'/>
</entry>
%</catalogue>
%    \end{macrocode}
%
% \section{Acknowledgement}
%
% \begin{description}
% \item[Ulrich Schwarz:] The package is based on his draft for
%   ``Die \TeX nische Kom\"odie'', see \cite{schwarz}.
% \end{description}
%
% \begin{thebibliography}{9}
%
% \bibitem{schwarz}
%   Ulrich Schwarz:
%   \textit{Was hinten herauskommt z\"ahlt: Counter Aliasing in \LaTeX},
%   \textit{Die \TeX nische Kom\"odie}, 3/2006, pages 8--14, Juli 2006.
%
% \bibitem{remreset}
%   David Carlisle: \textit{The \xpackage{remreset} package};
%   1997/09/28;
%   \CTAN{macros/latex/contrib/carlisle/remreset.sty}.
%
% \bibitem{hyperref}
%   Sebastian Rahtz, Heiko Oberdiek:
%   \textit{The \xpackage{hyperref} package};
%   2006/08/16 v6.75c;
%   \CTAN{macros/latex/contrib/hyperref/}.
%
% \end{thebibliography}
%
% \begin{History}
%   \begin{Version}{2006/02/20 v1.0}
%   \item
%     First version.
%   \end{Version}
%   \begin{Version}{2006/08/16 v1.1}
%   \item
%     Update of bibliography.
%   \end{Version}
%   \begin{Version}{2006/09/25 v1.2}
%   \item
%     Bug fix (\cs{aliascntresetthe}).
%   \end{Version}
%   \begin{Version}{2009/09/08 v1.3}
%   \item
%     Bug fix of \cs{@ifdefinable}'s use (thanks to Uwe L\"uck).
%   \end{Version}
% \end{History}
%
% \PrintIndex
%
% \Finale
\endinput
|
% \end{quote}
% Do not forget to quote the argument according to the demands
% of your shell.
%
% \paragraph{Generating the documentation.}
% You can use both the \xfile{.dtx} or the \xfile{.drv} to generate
% the documentation. The process can be configured by the
% configuration file \xfile{ltxdoc.cfg}. For instance, put this
% line into this file, if you want to have A4 as paper format:
% \begin{quote}
%   \verb|\PassOptionsToClass{a4paper}{article}|
% \end{quote}
% An example follows how to generate the
% documentation with pdf\LaTeX:
% \begin{quote}
%\begin{verbatim}
%pdflatex aliascnt.dtx
%makeindex -s gind.ist aliascnt.idx
%pdflatex aliascnt.dtx
%makeindex -s gind.ist aliascnt.idx
%pdflatex aliascnt.dtx
%\end{verbatim}
% \end{quote}
%
% \section{Catalogue}
%
% The following XML file can be used as source for the
% \href{http://mirror.ctan.org/help/Catalogue/catalogue.html}{\TeX\ Catalogue}.
% The elements \texttt{caption} and \texttt{description} are imported
% from the original XML file from the Catalogue.
% The name of the XML file in the Catalogue is \xfile{aliascnt.xml}.
%    \begin{macrocode}
%<*catalogue>
<?xml version='1.0' encoding='us-ascii'?>
<!DOCTYPE entry SYSTEM 'catalogue.dtd'>
<entry datestamp='$Date$' modifier='$Author$' id='aliascnt'>
  <name>aliascnt</name>
  <caption>Alias counters.</caption>
  <authorref id='auth:oberdiek'/>
  <copyright owner='Heiko Oberdiek' year='2006,2009'/>
  <license type='lppl1.3'/>
  <version number='1.3'/>
  <description>
    This package introduces aliases for counters, that
    share the same counter register and clear list.
    <p/>
    The package is part of the <xref refid='oberdiek'>oberdiek</xref>
    bundle.
  </description>
  <documentation details='Package documentation'
      href='ctan:/macros/latex/contrib/oberdiek/aliascnt.pdf'/>
  <ctan file='true' path='/macros/latex/contrib/oberdiek/aliascnt.dtx'/>
  <miktex location='oberdiek'/>
  <texlive location='oberdiek'/>
  <install path='/macros/latex/contrib/oberdiek/oberdiek.tds.zip'/>
</entry>
%</catalogue>
%    \end{macrocode}
%
% \section{Acknowledgement}
%
% \begin{description}
% \item[Ulrich Schwarz:] The package is based on his draft for
%   ``Die \TeX nische Kom\"odie'', see \cite{schwarz}.
% \end{description}
%
% \begin{thebibliography}{9}
%
% \bibitem{schwarz}
%   Ulrich Schwarz:
%   \textit{Was hinten herauskommt z\"ahlt: Counter Aliasing in \LaTeX},
%   \textit{Die \TeX nische Kom\"odie}, 3/2006, pages 8--14, Juli 2006.
%
% \bibitem{remreset}
%   David Carlisle: \textit{The \xpackage{remreset} package};
%   1997/09/28;
%   \CTAN{macros/latex/contrib/carlisle/remreset.sty}.
%
% \bibitem{hyperref}
%   Sebastian Rahtz, Heiko Oberdiek:
%   \textit{The \xpackage{hyperref} package};
%   2006/08/16 v6.75c;
%   \CTAN{macros/latex/contrib/hyperref/}.
%
% \end{thebibliography}
%
% \begin{History}
%   \begin{Version}{2006/02/20 v1.0}
%   \item
%     First version.
%   \end{Version}
%   \begin{Version}{2006/08/16 v1.1}
%   \item
%     Update of bibliography.
%   \end{Version}
%   \begin{Version}{2006/09/25 v1.2}
%   \item
%     Bug fix (\cs{aliascntresetthe}).
%   \end{Version}
%   \begin{Version}{2009/09/08 v1.3}
%   \item
%     Bug fix of \cs{@ifdefinable}'s use (thanks to Uwe L\"uck).
%   \end{Version}
% \end{History}
%
% \PrintIndex
%
% \Finale
\endinput
|
% \end{quote}
% Do not forget to quote the argument according to the demands
% of your shell.
%
% \paragraph{Generating the documentation.}
% You can use both the \xfile{.dtx} or the \xfile{.drv} to generate
% the documentation. The process can be configured by the
% configuration file \xfile{ltxdoc.cfg}. For instance, put this
% line into this file, if you want to have A4 as paper format:
% \begin{quote}
%   \verb|\PassOptionsToClass{a4paper}{article}|
% \end{quote}
% An example follows how to generate the
% documentation with pdf\LaTeX:
% \begin{quote}
%\begin{verbatim}
%pdflatex aliascnt.dtx
%makeindex -s gind.ist aliascnt.idx
%pdflatex aliascnt.dtx
%makeindex -s gind.ist aliascnt.idx
%pdflatex aliascnt.dtx
%\end{verbatim}
% \end{quote}
%
% \section{Catalogue}
%
% The following XML file can be used as source for the
% \href{http://mirror.ctan.org/help/Catalogue/catalogue.html}{\TeX\ Catalogue}.
% The elements \texttt{caption} and \texttt{description} are imported
% from the original XML file from the Catalogue.
% The name of the XML file in the Catalogue is \xfile{aliascnt.xml}.
%    \begin{macrocode}
%<*catalogue>
<?xml version='1.0' encoding='us-ascii'?>
<!DOCTYPE entry SYSTEM 'catalogue.dtd'>
<entry datestamp='$Date$' modifier='$Author$' id='aliascnt'>
  <name>aliascnt</name>
  <caption>Alias counters.</caption>
  <authorref id='auth:oberdiek'/>
  <copyright owner='Heiko Oberdiek' year='2006,2009'/>
  <license type='lppl1.3'/>
  <version number='1.5'/>
  <description>
    This package introduces aliases for counters, that
    share the same counter register and clear list.
    <p/>
    The package is part of the <xref refid='oberdiek'>oberdiek</xref>
    bundle.
  </description>
  <documentation details='Package documentation'
      href='ctan:/macros/latex/contrib/oberdiek/aliascnt.pdf'/>
  <ctan file='true' path='/macros/latex/contrib/oberdiek/aliascnt.dtx'/>
  <miktex location='oberdiek'/>
  <texlive location='oberdiek'/>
  <install path='/macros/latex/contrib/oberdiek/oberdiek.tds.zip'/>
</entry>
%</catalogue>
%    \end{macrocode}
%
% \section{Acknowledgement}
%
% \begin{description}
% \item[Ulrich Schwarz:] The package is based on his draft for
%   ``Die \TeX nische Kom\"odie'', see \cite{schwarz}.
% \end{description}
%
% \begin{thebibliography}{9}
%
% \bibitem{schwarz}
%   Ulrich Schwarz:
%   \textit{Was hinten herauskommt z\"ahlt: Counter Aliasing in \LaTeX},
%   \textit{Die \TeX nische Kom\"odie}, 3/2006, pages 8--14, Juli 2006.
%
% \bibitem{remreset}
%   David Carlisle: \textit{The \xpackage{remreset} package};
%   1997/09/28;
%   \CTAN{macros/latex/contrib/carlisle/remreset.sty}.
%
% \bibitem{hyperref}
%   Sebastian Rahtz, Heiko Oberdiek:
%   \textit{The \xpackage{hyperref} package};
%   2006/08/16 v6.75c;
%   \CTAN{macros/latex/contrib/hyperref/}.
%
% \end{thebibliography}
%
% \begin{History}
%   \begin{Version}{2006/02/20 v1.0}
%   \item
%     First version.
%   \end{Version}
%   \begin{Version}{2006/08/16 v1.1}
%   \item
%     Update of bibliography.
%   \end{Version}
%   \begin{Version}{2006/09/25 v1.2}
%   \item
%     Bug fix (\cs{aliascntresetthe}).
%   \end{Version}
%   \begin{Version}{2009/09/08 v1.3}
%   \item
%     Bug fix of \cs{@ifdefinable}'s use (thanks to Uwe L\"uck).
%   \end{Version}
%   \begin{Version}{2016/05/16 v1.4}
%   \item
%     Documentation updates.
%   \end{Version}
%   \begin{Version}{2018/09/07 v1.5}
%   \item
%     Avoid loading obsolete remreset package..
%   \end{Version}
% \end{History}
%
% \PrintIndex
%
% \Finale
\endinput
