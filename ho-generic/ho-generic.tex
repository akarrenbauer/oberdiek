\NeedsTeXFormat{LaTeX2e}

\newcommand*{\Title}{Overview}
\newcommand*{\CTANdir}{macros/latex/contrib/ho-generic/}
\newcommand*{\CTANroot}{ftp://ftp.ctan.org/tex-archive/}
\newcommand*{\Subject}{CTAN:\CTANdir}
\newcommand*{\Author}{Heiko Oberdiek}
\newcommand*{\Email}{ho-tex at tug.org}
\newcommand*{\Date}{2018/11/30}

% Copyright (C) 2006-2018 by
%    Heiko Oberdiek <heiko.oberdiek at googlemail.com>
%
% This work may be distributed and/or modified under the
% conditions of the LaTeX Project Public License, either
% version 1.3 of this license or (at your option) any later
% version. The latest version of this license is in
%    http://www.latex-project.org/lppl.txt
% and version 1.3 or later is part of all distributions of
% LaTeX version 2003/12/01 or later.
%
% This work has the LPPL maintenance status "maintained".
%
% This Current Maintainer of this work is Heiko Oberdiek.
%
% This work consists of the overview "ho-generic.pdf", its source
% "ho-generic.tex", and the installation script "ho-generic.ins"
% for the projects in CTAN:macros/latex/contrib/ho-generic/.
%
\documentclass[a4paper,12pt]{article}

\usepackage{ifluatex}
\ifluatex
  \usepackage{fontspec}[2011/09/18]%
  \usepackage{unicode-math}[2011/09/19]%
  \setmathfont{latinmodern-math.otf}%
\fi

\usepackage[
  ignorehead,
  top=1in,
]{geometry}
\usepackage{longtable}
\usepackage{color}
\usepackage[ngerman,english]{babel}
\usepackage{hologo}
\usepackage{biblatex}% for internals in .toc files

\definecolor{link}{rgb}{1,0,0}% red
\definecolor{file}{rgb}{0,0,1}% blue
\definecolor{url}{cmyk}{0.1,1,0,0.1}

\definecolor{file}{rgb}{1,0,0}% red
\definecolor{url}{rgb}{0,0,1}% blue
\definecolor{link}{rgb}{0,0.75,0}%

\usepackage[
  colorlinks,
]{hyperref}[2006/02/12]
\hypersetup{
  pdftitle={CTAN:\CTANdir},
  pdfsubject={Package Overview},
  pdfauthor={\Author\ <\Email>},
  bookmarksnumbered,
  bookmarksopen,
  bookmarksopenlevel=2,
  bookmarksdepth=2,
  filecolor=file,
  urlcolor=url,
  linkcolor=link,
}
\usepackage{bookmark}
\usepackage{hypdestopt}
\setcounter{tocdepth}{1}
\setcounter{secnumdepth}{1}

\title{%
  \href{\CTANroot\CTANdir}{CTAN:\CTANdir}%
}
\author{%
  \Author\\
  \textless
  \href{mailto:\Email}{\texttt{\Email}}%
  \textgreater
}
\date{\Date}

\providecommand*{\pdfTeX}{pdf\TeX}
\providecommand*{\plainTeX}{\mbox{plain-\TeX}}
\providecommand*{\iniTeX}{\mbox{ini-\TeX}}
\providecommand*{\VTeX}{V\TeX}
\providecommand*{\eTeX}{$\csname m@th\endcsname\varepsilon$-\TeX}
\providecommand*{\LuaTeX}{%
  L\textsc{ua}\TeX
}
\newcommand*{\xpackage}[1]{\textsf{#1}}
\newcommand*{\xmodule}[1]{\textsf{#1}}
\newcommand*{\xfile}[1]{\texttt{#1}}
\newcommand*{\xext}[1]{\texttt{.#1}}
\newcommand*{\xoption}[1]{\textsf{#1}}
\newcommand*{\cs}[1]{\texttt{\textbackslash#1}}
\newcommand*{\meta}[1]{%
  \ensuremath\langle
  \textit{#1}%
  \ensuremath\rangle
}

\makeatletter
\g@addto@macro\abstract{\noindent\ignorespaces}

\newcommand*{\tocinclude}[1]{%
  \setcounter{tocdepth}{3}%
  \begingroup
    \makeatletter
    \def\@prj{#1}%
    \let\contentsline\foreign@contentsline
    \input{\@prj.toc}%
  \endgroup
}
\def\foreign@contentsline#1#2#3#4{%
  \ifx\\#4\\%
    \csname l@#1\endcsname{#2}{#3}%
  \else
    \ifHy@linktocpage
      \csname l@#1\endcsname{{#2}}{%
        \hyper@linkfile{#3}{\@prj.pdf}{#4}%
      }%
    \else
      \csname l@#1\endcsname{%
        \hyper@linkfile{#2}{\@prj.pdf}{#4}%
      }{#3}%
    \fi
  \fi
}%

\newenvironment{overview}{%
  \setlength{\tabcolsep}{0.8\tabcolsep}%
  \setlength{\LTleft}{0pt}%
  \longtable{@{}llll@{}}
}{%
  \endlongtable
}
\newcommand*{\entry}[4]{%
  \href{file:#1.pdf}{%
    \bfseries\xpackage{#1}%
  }%
  & #2%
  & v#3%
  & \href{\CTANroot\CTANdir #1.pdf}{[pdf]} %
    \href{\CTANroot\CTANdir #1.dtx}{[dtx]}
  \\*%
  \hyperref[{#1}]{\small (contents)}%
  &
  \multicolumn{2}{l}{%
    #4%
  }%
  \\%
}
\newcommand*{\entrysep}{1.5ex}

\newcommand*{\pkgsectformat}[1]{%
  \texorpdfstring{%
    \textcolor{link}{The} %
    \xpackage{#1} %
    \textcolor{link}{package}%
  }{#1}%
}

\makeatother

\begin{document}
\maketitle

\section{Overview}
\begin{overview}
\entry{alphalph}{2011/05/13}{2.4}{Convert numbers to letters}%
[\entrysep]
\entry{atbegshi}{2011/10/05}{1.16}{At begin shipout hook}%
[\entrysep]
\entry{bigintcalc}{2012/04/08}{1.3}{Expandable calculations on big integers}%
[\entrysep]
\entry{bitset}{2011/01/30}{1.1}{Handle bit-vector datatype}%
[\entrysep]
\entry{catchfile}{2011/03/01}{1.6}{Catch the contents of a file}%
[\entrysep]
\entry{embedfile}{2011/04/13}{2.6}{Embed files into PDF}%
[\entrysep]
\entry{engord}{2010/03/01}{1.8}{Provides English ordinal numbers}%
[\entrysep]
\entry{eolgrab}{2011/01/12}{1.0}{Catch arguments delimited by end of line}%
[\entrysep]
\entry{etexcmds}{2011/02/16}{1.5}{Avoid name clashes with \hologo{eTeX} commands}%
[\entrysep]
\entry{fibnum}{2012/04/08}{1.0}{Fibonacci numbers}%
[\entrysep]
\entry{flags}{2007/09/30}{0.4}{Setting/clearing of flags in bit fields}%
[\entrysep]
\entry{gettitlestring}{2010/12/03}{1.4}{Cleanup title references}%
[\entrysep]
\entry{hologo}{2012/04/26}{1.10}{A logo collection with bookmark support}%
[\entrysep]
\entry{hyphsubst}{2008/06/09}{0.2}{Substitute hyphenation patterns}%
[\entrysep]
\entry{infwarerr}{2010/04/08}{1.3}{Providing info/warning/error messages}%
[\entrysep]
\entry{intcalc}{2007/09/27}{1.1}{Expandable calculations with integers}%
[\entrysep]
\entry{ltxcmds}{2011/11/09}{1.22}{\hologo{LaTeX} kernel commands for general use}%
[\entrysep]
\entry{magicnum}{2011/04/10}{1.4}{Magic numbers}%
[\entrysep]
\entry{mleftright}{2010/09/25}{1.0}{Math left/right delim.\@ as open/close}%
[\entrysep]
\entry{pdfcrypt}{2007/04/26}{1.0}{Allows the setting of PDF encryption}%
[\entrysep]
\entry{pdfescape}{2011/11/25}{1.13}{Implements \hologo{pdfTeX}'s escape features}%
[\entrysep]
\entry{pdfrender}{2010/01/28}{1.2}{Access to some PDF graphics parameters}%
[\entrysep]
\entry{pdftexcmds}{2011/11/29}{0.20}{Utility functions of \hologo{pdfTeX} for \hologo{LuaTeX}}%
[\entrysep]
\entry{protecteddef}{2011/01/31}{1.0}{Define protected commands}%
[\entrysep]
\entry{rotchiffre}{2010/11/12}{1.0}{Perform simple rotation ciphers}%
[\entrysep]
\entry{selinput}{2007/09/09}{1.2}{Semi-automatic input encoding detection}%
[\entrysep]
\entry{setouterhbox}{2007/09/09}{1.7}{Set hbox in outer horizontal mode}%
[\entrysep]
\entry{stringenc}{2011/12/02}{1.10}{Convert strings between diff.\@ encodings}%
[\entrysep]
\entry{thepdfnumber}{2011/11/24}{1.0}{Print PDF numbers with minimal digits}%
[\entrysep]
\entry{uniquecounter}{2011/01/30}{1.2}{Provide unlimited unique counter}%
\end{overview}

\section{Packages}
\hypersetup{bookmarksnumbered=false}
\subsection{\pkgsectformat{alphalph}}
\label{alphalph}
\begin{abstract}
The package provides methods to represent numbers with a limited
set of symbols. Both \hologo{LaTeX} and \hologo{plainTeX} are supported.
\end{abstract}
\tocinclude{alphalph}

\newpage
\subsection{\pkgsectformat{atbegshi}}
\label{atbegshi}
\begin{abstract}
This package is a modern reimplementation of package \xpackage{everyshi}
without the burden of compatibility. It makes use of \eTeX's if available.
Both \LaTeX\ and \plainTeX\ are supported.
\end{abstract}
\tocinclude{atbegshi}

\newpage
\subsection{\pkgsectformat{bigintcalc}}
\label{bigintcalc}
\begin{abstract}
This package provides expandable arithmetic operations
with big integers that can exceed \TeX's number limits.
\end{abstract}
\tocinclude{bigintcalc}

\newpage
\subsection{\pkgsectformat{bitset}}
\label{bitset}
\begin{abstract}
This package defines and implements the data type bit set,
a vector of bits. The size of the vector may grow dynamically.
Individual bits can be manipulated.
\end{abstract}
\tocinclude{bitset}


\newpage
\subsection{\pkgsectformat{catchfile}}
\label{catchfile}
\begin{abstract}
This package catches the contents of a file and puts it in a macro.
It requires \eTeX. Both \LaTeX\ and \plainTeX\ are supported.
\end{abstract}
\tocinclude{catchfile}

\newpage
\subsection{\pkgsectformat{embedfile}}
\label{embedfile}
\begin{abstract}
This package embeds files to a PDF document.
Currently the only supported driver is \pdfTeX\ $>=$ 1.30 in PDF mode.
\end{abstract}
\tocinclude{embedfile}

\newpage
\subsection{\pkgsectformat{engord}}
\label{engord}
\begin{abstract}
The package generates the suffix of English ordinal numbers.
It can be used with plain and \LaTeX\ formats.
\end{abstract}
\tocinclude{engord}


\newpage
\subsection{\pkgsectformat{eolgrab}}
\label{eolgrab}
\begin{abstract}
This package implements a generic argument grabber
to catch an argument that is delimited by the line end.
\end{abstract}
\tocinclude{eolgrab}
\newpage
\subsection{\pkgsectformat{etexcmds}}
\label{etexcmds}
\begin{abstract}
New primitive commands are introduced in \eTeX. Sometimes the
names collide with existing macros. This package solves this
name clashes by adding a prefix to \eTeX's commands. For example,
\eTeX's \cs{unexpanded} is provided as \cs{etex@unexpanded}.
\end{abstract}
\tocinclude{etexcmds}

\newpage
\subsection{\pkgsectformat{fibnum}}
\label{fibnum}
\begin{abstract}
The package \xpackage{fibnum} provides expandable fibonacci
numbers for both \hologo{LaTeX} and \hologo{plainTeX}.
\end{abstract}
\tocinclude{fibnum}

\newpage
\subsection{\pkgsectformat{gettitlestring}}
\label{gettitlestring}
\begin{abstract}
The \LaTeX\ package addresses packages that are dealing with
references to titles (\cs{section}, \cs{caption}, \dots).
The package tries to remove \cs{label} and other
commands from title strings.
\end{abstract}
\tocinclude{gettitlestring}

\newpage
\subsection{\pkgsectformat{hologo}}
\label{hologo}
\begin{abstract}
This package starts a collection of logos with support for bookmarks
strings.
\end{abstract}
\tocinclude{hologo}
\newpage
\subsection{\pkgsectformat{hyphsubst}}
\label{hyphsubst}
\begin{abstract}
A \TeX\ format file may include alternative hyphenation patterns
for a language with a different name. If the naming convention
follows \xpackage{babel's} rules, then the hyphenation patterns
for a language can be replaced by the alternative hyphenation patterns,
provided in the format file.
\end{abstract}
\tocinclude{hyphsubst}
\newpage
\subsection{\pkgsectformat{infwarerr}}
\label{infwarerr}
\begin{abstract}
This package provides a complete set of macros for informations,
warnings and error messages with support for \plainTeX.
\end{abstract}
\tocinclude{infwarerr}
\newpage
\subsection{\pkgsectformat{intcalc}}
\label{intcalc}
\begin{abstract}
This package provides expandable arithmetic operations
with integers.
\end{abstract}
\tocinclude{intcalc}


\newpage
\subsection{\pkgsectformat{ltxcmds}}
\label{ltxcmds}
\begin{abstract}
The package \xpackage{ltxcmds} exports some utility macros
from the \LaTeX\ kernel into a separate namespace and
also provides them for other formats such as plain-\TeX.
\end{abstract}
\tocinclude{ltxcmds}

\newpage
\subsection{\pkgsectformat{magicnum}}
\label{magicnum}
\begin{abstract}
This packages allows to access magic numbers by a hierarchical
name system.
\end{abstract}
\tocinclude{magicnum}

\newpage
\subsection{\pkgsectformat{mleftright}}
\label{mleftright}
\begin{abstract}
\TeX\ sets subformulas by \cs{left} and \cs{right} as inner formulas
with additional surrounding spaces in some situations. This package
provides \cs{mleft} and \cs{mright} that call \cs{left} and \cs{right},
but the delimiters will act as normal \cs{mathopen} and \cs{mathclose}
delimiters without the additional space of an inner formula.
\end{abstract}
\tocinclude{mleftright}


\newpage
\subsection{\pkgsectformat{pdfcrypt}}
\label{pdfcrypt}
\begin{abstract}
This package supports the setting of pdf encryption options
for \VTeX\ and some older versions of \pdfTeX.
\end{abstract}
\tocinclude{pdfcrypt}

\newpage
\subsection{\pkgsectformat{pdfescape}}
\label{pdfescape}
\begin{abstract}
This package implements \pdfTeX's escape features
(\cs{pdfescapehex}, \cs{pdfunescapehex}, \cs{pdfescapename},
\cs{pdfescapestring}) using \TeX\ or \eTeX.
\end{abstract}
\tocinclude{pdfescape}


\newpage
\subsection{\pkgsectformat{pdfrender}}
\label{pdfrender}
\begin{abstract}
The PDF format has some graphics parameter like
line width or text rendering mode. This package
provides an interface for setting these parameters.
\end{abstract}
\tocinclude{pdfrender}

\newpage
\subsection{\pkgsectformat{pdftexcmds}}
\label{pdftexcmds}
\begin{abstract}
\hologo{LuaTeX} provides most of the commands of \hologo{pdfTeX} 1.40. However
a number of utility functions are removed. This package tries to fill
the gap and implements some of the missing primitive using Lua.
\end{abstract}
\tocinclude{pdftexcmds}


\newpage
\subsection{\pkgsectformat{protecteddef}}
\label{protecteddef}
\begin{abstract}
This packages provides \cs{ProtectedDef} for defining
robust macros for both \hologo{plainTeX} and \hologo{LaTeX}.
First \hologo{eTeX}'s \cs{protected} is tried, then
\hologo{LaTeX}'s \cs{DeclareRobustCommand} is used.
Otherwise the macro is not made robust.
\end{abstract}
\tocinclude{protecteddef}


\newpage
\subsection{\pkgsectformat{rotchiffre}}
\label{rotchiffre}
\begin{abstract}
This package implements chiffres ROT13 with its variants
ROT5, ROT18, and ROT47.
\end{abstract}
\tocinclude{rotchiffre}

\newpage
\subsection{\pkgsectformat{selinput}}
\label{selinput}
\begin{abstract}
This package selects the input encoding by specifying between
input characters and their glyph names.
\end{abstract}
\tocinclude{selinput}

\newpage
\subsection{\pkgsectformat{setouterhbox}}
\label{setouterhbox}
\begin{abstract}
If math stuff is set in an \cs{hbox}, then TeX
performs some optimization and omits the implicite
penalties \cs{binoppenalty} and \cs{relpenalty}.
This packages tries to put stuff into an \cs{hbox}
without getting lost of those penalties.
\end{abstract}
\tocinclude{setouterhbox}

\newpage
\subsection{\pkgsectformat{stringenc}}
\label{stringenc}
\begin{abstract}
This package provides \cs{StringEncodingConvert} for converting
a string between different encodings.
Both \LaTeX\ and \plainTeX\ are supported.
\end{abstract}
\tocinclude{stringenc}

\newpage
\subsection{\pkgsectformat{thepdfnumber}}
\label{thepdfnumber}
\begin{abstract}
The package converts real numbers to a minimal representation
that is stripped from leading or trailing zeros,
plus signs and decimal point if not necessary.
\end{abstract}
\tocinclude{thepdfnumber}


\newpage
\subsection{\pkgsectformat{uniquecounter}}
\label{uniquecounter}
\begin{abstract}
This package provides a kind of counter that provides unique
number values. Several counters can be created by different names.
The numeric values are not limited.
\end{abstract}
\tocinclude{uniquecounter}

\end{document}
