% \iffalse meta-comment
%
% File: thepdfnumber.dtx
% Version: 2016/05/16 v1.1
% Info: Print PDF numbers with minimal digits
%
% Copyright (C) 2011 by
%    Heiko Oberdiek <heiko.oberdiek at googlemail.com>
%    2016
%    https://github.com/ho-tex/oberdiek/issues
%
% This work may be distributed and/or modified under the
% conditions of the LaTeX Project Public License, either
% version 1.3c of this license or (at your option) any later
% version. This version of this license is in
%    http://www.latex-project.org/lppl/lppl-1-3c.txt
% and the latest version of this license is in
%    http://www.latex-project.org/lppl.txt
% and version 1.3 or later is part of all distributions of
% LaTeX version 2005/12/01 or later.
%
% This work has the LPPL maintenance status "maintained".
%
% This Current Maintainer of this work is Heiko Oberdiek.
%
% The Base Interpreter refers to any `TeX-Format',
% because some files are installed in TDS:tex/generic//.
%
% This work consists of the main source file thepdfnumber.dtx
% and the derived files
%    thepdfnumber.sty, thepdfnumber.pdf, thepdfnumber.ins, thepdfnumber.drv,
%    thepdfnumber-test1.tex, thepdfnumber-test2.tex,
%    thepdfnumber-test3.tex.
%
% Distribution:
%    CTAN:macros/latex/contrib/oberdiek/thepdfnumber.dtx
%    CTAN:macros/latex/contrib/oberdiek/thepdfnumber.pdf
%
% Unpacking:
%    (a) If thepdfnumber.ins is present:
%           tex thepdfnumber.ins
%    (b) Without thepdfnumber.ins:
%           tex thepdfnumber.dtx
%    (c) If you insist on using LaTeX
%           latex \let\install=y% \iffalse meta-comment
%
% File: thepdfnumber.dtx
% Version: 2016/05/16 v1.1
% Info: Print PDF numbers with minimal digits
%
% Copyright (C) 2011 by
%    Heiko Oberdiek <heiko.oberdiek at googlemail.com>
%    2016
%    https://github.com/ho-tex/oberdiek/issues
%
% This work may be distributed and/or modified under the
% conditions of the LaTeX Project Public License, either
% version 1.3c of this license or (at your option) any later
% version. This version of this license is in
%    http://www.latex-project.org/lppl/lppl-1-3c.txt
% and the latest version of this license is in
%    http://www.latex-project.org/lppl.txt
% and version 1.3 or later is part of all distributions of
% LaTeX version 2005/12/01 or later.
%
% This work has the LPPL maintenance status "maintained".
%
% This Current Maintainer of this work is Heiko Oberdiek.
%
% The Base Interpreter refers to any `TeX-Format',
% because some files are installed in TDS:tex/generic//.
%
% This work consists of the main source file thepdfnumber.dtx
% and the derived files
%    thepdfnumber.sty, thepdfnumber.pdf, thepdfnumber.ins, thepdfnumber.drv,
%    thepdfnumber-test1.tex, thepdfnumber-test2.tex,
%    thepdfnumber-test3.tex.
%
% Distribution:
%    CTAN:macros/latex/contrib/oberdiek/thepdfnumber.dtx
%    CTAN:macros/latex/contrib/oberdiek/thepdfnumber.pdf
%
% Unpacking:
%    (a) If thepdfnumber.ins is present:
%           tex thepdfnumber.ins
%    (b) Without thepdfnumber.ins:
%           tex thepdfnumber.dtx
%    (c) If you insist on using LaTeX
%           latex \let\install=y% \iffalse meta-comment
%
% File: thepdfnumber.dtx
% Version: 2016/05/16 v1.1
% Info: Print PDF numbers with minimal digits
%
% Copyright (C) 2011 by
%    Heiko Oberdiek <heiko.oberdiek at googlemail.com>
%    2016
%    https://github.com/ho-tex/oberdiek/issues
%
% This work may be distributed and/or modified under the
% conditions of the LaTeX Project Public License, either
% version 1.3c of this license or (at your option) any later
% version. This version of this license is in
%    http://www.latex-project.org/lppl/lppl-1-3c.txt
% and the latest version of this license is in
%    http://www.latex-project.org/lppl.txt
% and version 1.3 or later is part of all distributions of
% LaTeX version 2005/12/01 or later.
%
% This work has the LPPL maintenance status "maintained".
%
% This Current Maintainer of this work is Heiko Oberdiek.
%
% The Base Interpreter refers to any `TeX-Format',
% because some files are installed in TDS:tex/generic//.
%
% This work consists of the main source file thepdfnumber.dtx
% and the derived files
%    thepdfnumber.sty, thepdfnumber.pdf, thepdfnumber.ins, thepdfnumber.drv,
%    thepdfnumber-test1.tex, thepdfnumber-test2.tex,
%    thepdfnumber-test3.tex.
%
% Distribution:
%    CTAN:macros/latex/contrib/oberdiek/thepdfnumber.dtx
%    CTAN:macros/latex/contrib/oberdiek/thepdfnumber.pdf
%
% Unpacking:
%    (a) If thepdfnumber.ins is present:
%           tex thepdfnumber.ins
%    (b) Without thepdfnumber.ins:
%           tex thepdfnumber.dtx
%    (c) If you insist on using LaTeX
%           latex \let\install=y% \iffalse meta-comment
%
% File: thepdfnumber.dtx
% Version: 2016/05/16 v1.1
% Info: Print PDF numbers with minimal digits
%
% Copyright (C) 2011 by
%    Heiko Oberdiek <heiko.oberdiek at googlemail.com>
%    2016
%    https://github.com/ho-tex/oberdiek/issues
%
% This work may be distributed and/or modified under the
% conditions of the LaTeX Project Public License, either
% version 1.3c of this license or (at your option) any later
% version. This version of this license is in
%    http://www.latex-project.org/lppl/lppl-1-3c.txt
% and the latest version of this license is in
%    http://www.latex-project.org/lppl.txt
% and version 1.3 or later is part of all distributions of
% LaTeX version 2005/12/01 or later.
%
% This work has the LPPL maintenance status "maintained".
%
% This Current Maintainer of this work is Heiko Oberdiek.
%
% The Base Interpreter refers to any `TeX-Format',
% because some files are installed in TDS:tex/generic//.
%
% This work consists of the main source file thepdfnumber.dtx
% and the derived files
%    thepdfnumber.sty, thepdfnumber.pdf, thepdfnumber.ins, thepdfnumber.drv,
%    thepdfnumber-test1.tex, thepdfnumber-test2.tex,
%    thepdfnumber-test3.tex.
%
% Distribution:
%    CTAN:macros/latex/contrib/oberdiek/thepdfnumber.dtx
%    CTAN:macros/latex/contrib/oberdiek/thepdfnumber.pdf
%
% Unpacking:
%    (a) If thepdfnumber.ins is present:
%           tex thepdfnumber.ins
%    (b) Without thepdfnumber.ins:
%           tex thepdfnumber.dtx
%    (c) If you insist on using LaTeX
%           latex \let\install=y\input{thepdfnumber.dtx}
%        (quote the arguments according to the demands of your shell)
%
% Documentation:
%    (a) If thepdfnumber.drv is present:
%           latex thepdfnumber.drv
%    (b) Without thepdfnumber.drv:
%           latex thepdfnumber.dtx; ...
%    The class ltxdoc loads the configuration file ltxdoc.cfg
%    if available. Here you can specify further options, e.g.
%    use A4 as paper format:
%       \PassOptionsToClass{a4paper}{article}
%
%    Programm calls to get the documentation (example):
%       pdflatex thepdfnumber.dtx
%       makeindex -s gind.ist thepdfnumber.idx
%       pdflatex thepdfnumber.dtx
%       makeindex -s gind.ist thepdfnumber.idx
%       pdflatex thepdfnumber.dtx
%
% Installation:
%    TDS:tex/generic/oberdiek/thepdfnumber.sty
%    TDS:doc/latex/oberdiek/thepdfnumber.pdf
%    TDS:doc/latex/oberdiek/test/thepdfnumber-test1.tex
%    TDS:doc/latex/oberdiek/test/thepdfnumber-test2.tex
%    TDS:doc/latex/oberdiek/test/thepdfnumber-test3.tex
%    TDS:source/latex/oberdiek/thepdfnumber.dtx
%
%<*ignore>
\begingroup
  \catcode123=1 %
  \catcode125=2 %
  \def\x{LaTeX2e}%
\expandafter\endgroup
\ifcase 0\ifx\install y1\fi\expandafter
         \ifx\csname processbatchFile\endcsname\relax\else1\fi
         \ifx\fmtname\x\else 1\fi\relax
\else\csname fi\endcsname
%</ignore>
%<*install>
\input docstrip.tex
\Msg{************************************************************************}
\Msg{* Installation}
\Msg{* Package: thepdfnumber 2016/05/16 v1.1 Print PDF numbers with minimal digits (HO)}
\Msg{************************************************************************}

\keepsilent
\askforoverwritefalse

\let\MetaPrefix\relax
\preamble

This is a generated file.

Project: thepdfnumber
Version: 2016/05/16 v1.1

Copyright (C) 2011 by
   Heiko Oberdiek <heiko.oberdiek at googlemail.com>

This work may be distributed and/or modified under the
conditions of the LaTeX Project Public License, either
version 1.3c of this license or (at your option) any later
version. This version of this license is in
   http://www.latex-project.org/lppl/lppl-1-3c.txt
and the latest version of this license is in
   http://www.latex-project.org/lppl.txt
and version 1.3 or later is part of all distributions of
LaTeX version 2005/12/01 or later.

This work has the LPPL maintenance status "maintained".

This Current Maintainer of this work is Heiko Oberdiek.

The Base Interpreter refers to any `TeX-Format',
because some files are installed in TDS:tex/generic//.

This work consists of the main source file thepdfnumber.dtx
and the derived files
   thepdfnumber.sty, thepdfnumber.pdf, thepdfnumber.ins, thepdfnumber.drv,
   thepdfnumber-test1.tex, thepdfnumber-test2.tex,
   thepdfnumber-test3.tex.

\endpreamble
\let\MetaPrefix\DoubleperCent

\generate{%
  \file{thepdfnumber.ins}{\from{thepdfnumber.dtx}{install}}%
  \file{thepdfnumber.drv}{\from{thepdfnumber.dtx}{driver}}%
  \usedir{tex/generic/oberdiek}%
  \file{thepdfnumber.sty}{\from{thepdfnumber.dtx}{package}}%
  \usedir{doc/latex/oberdiek/test}%
  \file{thepdfnumber-test1.tex}{\from{thepdfnumber.dtx}{test1}}%
  \file{thepdfnumber-test2.tex}{\from{thepdfnumber.dtx}{test2}}%
  \file{thepdfnumber-test3.tex}{\from{thepdfnumber.dtx}{test2,active}}%
}

\catcode32=13\relax% active space
\let =\space%
\Msg{************************************************************************}
\Msg{*}
\Msg{* To finish the installation you have to move the following}
\Msg{* file into a directory searched by TeX:}
\Msg{*}
\Msg{*     thepdfnumber.sty}
\Msg{*}
\Msg{* To produce the documentation run the file `thepdfnumber.drv'}
\Msg{* through LaTeX.}
\Msg{*}
\Msg{* Happy TeXing!}
\Msg{*}
\Msg{************************************************************************}

\endbatchfile
%</install>
%<*ignore>
\fi
%</ignore>
%<*driver>
\NeedsTeXFormat{LaTeX2e}
\ProvidesFile{thepdfnumber.drv}%
  [2016/05/16 v1.1 Print PDF numbers with minimal digits (HO)]%
\documentclass{ltxdoc}
\usepackage{holtxdoc}[2011/11/22]
\usepackage{csquotes}
\usepackage[
  backend=bibtex,
  bibencoding=ascii,
  alldates=iso8601,
]{biblatex}[2011/11/13]
\bibliography{oberdiek-source}
\begin{document}
  \DocInput{thepdfnumber.dtx}%
\end{document}
%</driver>
% \fi
%
%
% \CharacterTable
%  {Upper-case    \A\B\C\D\E\F\G\H\I\J\K\L\M\N\O\P\Q\R\S\T\U\V\W\X\Y\Z
%   Lower-case    \a\b\c\d\e\f\g\h\i\j\k\l\m\n\o\p\q\r\s\t\u\v\w\x\y\z
%   Digits        \0\1\2\3\4\5\6\7\8\9
%   Exclamation   \!     Double quote  \"     Hash (number) \#
%   Dollar        \$     Percent       \%     Ampersand     \&
%   Acute accent  \'     Left paren    \(     Right paren   \)
%   Asterisk      \*     Plus          \+     Comma         \,
%   Minus         \-     Point         \.     Solidus       \/
%   Colon         \:     Semicolon     \;     Less than     \<
%   Equals        \=     Greater than  \>     Question mark \?
%   Commercial at \@     Left bracket  \[     Backslash     \\
%   Right bracket \]     Circumflex    \^     Underscore    \_
%   Grave accent  \`     Left brace    \{     Vertical bar  \|
%   Right brace   \}     Tilde         \~}
%
% \GetFileInfo{thepdfnumber.drv}
%
% \title{The \xpackage{thepdfnumber} package}
% \date{2016/05/16 v1.1}
% \author{Heiko Oberdiek\thanks
% {Please report any issues at https://github.com/ho-tex/oberdiek/issues}\\
% \xemail{heiko.oberdiek at googlemail.com}}
%
% \maketitle
%
% \begin{abstract}
% The package converts real numbers to a minimal representation
% that is stripped from leading or trailing zeros,
% plus signs and decimal point if not necessary.
% \end{abstract}
%
% \tableofcontents
%
% \section{Documentation}
%
% \subsection{Introduction}
%
% Dealing with the PDF format, there is sometimes the need to write
% some low level PDF stuff. In case of numbers, the numbers can
% arise from user input (e.g. color or transparency specifications)
% or can be calculated. For example, \hologo{LaTeX}'s \cs{strip@pt}
% makes a good job to output a real number. It automatically suppresses
% the decimal part if the number is an integer. However it leaves
% a leading zero for numbers greater zero and smaller one.
% Thus the package provides macros that can be used with different
% formats, even with \hologo{iniTeX} and generates numbers that
% are valid numbers of the PDF format and whose length is minimal.
%
% \subsection{Usage}
%
% The package \xpackage{thepdfnumber} can be used with \hologo{LaTeX},
% \hologo{plainTeX} or even with \hologo{iniTeX}:
% \begin{quote}
%   |\RequirePackage{thepdfnumber}| \textit{\% \hologo{LaTeX}}\\
%   |\input thepdfnumber.sty| \textit{\% \hologo{plainTeX}/\hologo{iniTeX}}
% \end{quote}
% The package does not need and have package options.
%
% \subsection{User macros}
%
% All user macros are expandable in exact two expansion steps.
%
% \begin{declcs}{thepdfnumber} \M{number}
% \end{declcs}
% Macro \cs{thepdfnumber} takes a number
% as argument and expands to a minimal representation of that number.
% Some examples:
% \begin{quote}
% \begin{tabular}{r@{.: }>{\ttfamily}l@{ $\rightarrow$\ }>{\ttfamily}l@{}}
%   1& +123 & 123\\
%   2& --123 & 123\\
%   3& -01 & -1\\
%   4& 0045 & 45\\
%   5& 1.0 & 1\\
%   6& 1.20 & 1.2\\
%   7& 0.0 & 0\\
%   8& 0.78 & .78\\
%   9& +012.340 & 12.34\\
% \end{tabular}
% \end{quote}
% It reduces the length of the number representation:
% \begin{itemize}
% \item The signs are collapsed and only one minus sign
%   is output if the number is negative (see examples 1, 2, 3, 9).
% \item Leading zeros are removed (4, 8, 9) unless the number is zero (7).
% \item The decimal part is omitted, if the number is an integer (5, 7).
% \item Trailing zeros from the decimal part are stripped (5, 6, 7, 9).
% \end{itemize}
% The resulting number representation can be catched with one of the
% following Perl regular expressions:
% \begin{itemize}
% \item |^0$| (zero)
% \item |^-?[1-9][0-9]*$| (integer)
% \item |^-?[0-9]*\.[0-9]*[1-9]$| (real)
% \end{itemize}
% This is a valid numeric object of the PDF specification \cite[``7.3.3~Numeric Objects'']{pdfspec-iso32000-1}.
%
% \begin{declcs}{thepdfnumberNormZeroOne}
% \end{declcs}
% There are various places in the PDF specification where the
% number is in the domain 0.0 upto 1.0. Macro thepdfnumberNormZeroOne
% automatically adjusts the number to fit into that range. Negative
% numbers are mapped to |0| and numbers greater than one are
% replaced by |1|. Thus the result fits one of the following regular
% expressions:
% \begin{itemize}
% \item |^0$|
% \item |^\.[0-9]*[1-9]$|
% \item |^1$|
% \end{itemize}
% Examples:
% \begin{quote}
% \begin{tabular}{@{}>{\ttfamily}l@{ $\rightarrow$\ }>{\ttfamily}l@{}}
%   -456 & 0\\
%   -0.001 & 0\\
%   0.0 & 0\\
%   0.010 & .01\\
%   0.456 & .456\\
%   1.0 & 1\\
%   01.001 & 1\\
%   4 & 1\\
% \end{tabular}
% \end{quote}
%
% \subsection{Input number}
%
% The user macros expect a number as argument. The number can either
% be given explicitely or as macro that expands in one step
% to an explicite number, because the first token of the argument
% is expanded once.
%
% The explicite number consists of
% \begin{itemize}
% \item optional signs `\texttt{+}' and `\texttt{-}',
% \item digits `\texttt{0}' upto `\texttt{9}' and
% \item an optional dot `\texttt{.}'.
% \end{itemize}
% All tokens must have catcode 12 (other), the default catcodes
% for these characters in \hologo{LaTeX}, \hologo{plainTeX} or
% \hologo{iniTeX}. As Perl regular expression the number is expected
% in one of the following forms:
% \begin{itemize}
% \item |^[+-]*[0-9]+$|
% \item |^[+-]*[0-9]*\.[0-9]*$|
% \end{itemize}
% At least one digit or the dot must be present.
%
% \subsection{Error handling}
%
% The package is not intended for validating numbers or to
% decide if an argument is a number. Therefore it is an usage
% error to use the user macros with arguments that are not
% explicite numbers as specified in the previous sections.
% Nevertheless some error conditions are sometimes recognized.
% Errors are given in form of an undefined command sequence.
% It is the only way to notify \hologo{TeX} in expandable context.
% Expanding to some error text would invalidate the output.
% Currently the following errors are thrown:
% \begin{description}
% \item[\cs{thepdfnumber@ErrorEndMarker}:]
%   Internally the argument parsing uses an end marker that
%   is never called directly. If it is called with valid
%   user input, then this is a bug. Otherwise it means the user input
%   contains nasty stuff.
% \item[\cs{thepdfnumber@ErrorUnexpectedEnd}:] The macros expect at least one digit
%   or the dot, otherwise if the argument is empty or only contains
%   signs, then this error is called.
% \item[\cs{thepdfnumber@ErrorInvalidToken}:] It is called if the number
%   contains other tokens than signs, digits or the dot or the
%   token at the wrong place (e.g. a sign after a digit).
%   In case of \cs{thepdfnumberNormZeroOne} this error condition might
%   not always be detected, because the number parsing might stop
%   at an early point, when the result is already clear (e.g.
%   if the number is negative or will be greater than one).
% \item[Improper alphabetic constant:] This error might be
%   thrown by \hologo{TeX}, if the number contains
%   command tokens instead of characters.
% \end{description}
%
% \StopEventually{
% }
%
% \section{Implementation}
%
%    \begin{macrocode}
%<*package>
%    \end{macrocode}
%
% \subsection{Reload check and package identification}
%    Reload check, especially if the package is not used with \LaTeX.
%    \begin{macrocode}
\begingroup\catcode61\catcode48\catcode32=10\relax%
  \catcode13=5 % ^^M
  \endlinechar=13 %
  \catcode35=6 % #
  \catcode39=12 % '
  \catcode44=12 % ,
  \catcode45=12 % -
  \catcode46=12 % .
  \catcode58=12 % :
  \catcode64=11 % @
  \catcode123=1 % {
  \catcode125=2 % }
  \expandafter\let\expandafter\x\csname ver@thepdfnumber.sty\endcsname
  \ifx\x\relax % plain-TeX, first loading
  \else
    \def\empty{}%
    \ifx\x\empty % LaTeX, first loading,
      % variable is initialized, but \ProvidesPackage not yet seen
    \else
      \expandafter\ifx\csname PackageInfo\endcsname\relax
        \def\x#1#2{%
          \immediate\write-1{Package #1 Info: #2.}%
        }%
      \else
        \def\x#1#2{\PackageInfo{#1}{#2, stopped}}%
      \fi
      \x{thepdfnumber}{The package is already loaded}%
      \aftergroup\endinput
    \fi
  \fi
\endgroup%
%    \end{macrocode}
%    Package identification:
%    \begin{macrocode}
\begingroup\catcode61\catcode48\catcode32=10\relax%
  \catcode13=5 % ^^M
  \endlinechar=13 %
  \catcode35=6 % #
  \catcode39=12 % '
  \catcode40=12 % (
  \catcode41=12 % )
  \catcode44=12 % ,
  \catcode45=12 % -
  \catcode46=12 % .
  \catcode47=12 % /
  \catcode58=12 % :
  \catcode64=11 % @
  \catcode91=12 % [
  \catcode93=12 % ]
  \catcode123=1 % {
  \catcode125=2 % }
  \expandafter\ifx\csname ProvidesPackage\endcsname\relax
    \def\x#1#2#3[#4]{\endgroup
      \immediate\write-1{Package: #3 #4}%
      \xdef#1{#4}%
    }%
  \else
    \def\x#1#2[#3]{\endgroup
      #2[{#3}]%
      \ifx#1\@undefined
        \xdef#1{#3}%
      \fi
      \ifx#1\relax
        \xdef#1{#3}%
      \fi
    }%
  \fi
\expandafter\x\csname ver@thepdfnumber.sty\endcsname
\ProvidesPackage{thepdfnumber}%
  [2016/05/16 v1.1 Print PDF numbers with minimal digits (HO)]%
%    \end{macrocode}
%
% \subsection{Catcodes}
%
%    \begin{macrocode}
\begingroup\catcode61\catcode48\catcode32=10\relax%
  \catcode13=5 % ^^M
  \endlinechar=13 %
  \catcode123=1 % {
  \catcode125=2 % }
  \catcode64=11 % @
  \def\x{\endgroup
    \expandafter\edef\csname ThPdNu@AtEnd\endcsname{%
      \endlinechar=\the\endlinechar\relax
      \catcode13=\the\catcode13\relax
      \catcode32=\the\catcode32\relax
      \catcode35=\the\catcode35\relax
      \catcode61=\the\catcode61\relax
      \catcode64=\the\catcode64\relax
      \catcode123=\the\catcode123\relax
      \catcode125=\the\catcode125\relax
    }%
  }%
\x\catcode61\catcode48\catcode32=10\relax%
\catcode13=5 % ^^M
\endlinechar=13 %
\catcode35=6 % #
\catcode64=11 % @
\catcode123=1 % {
\catcode125=2 % }
\def\TMP@EnsureCode#1#2{%
  \edef\ThPdNu@AtEnd{%
    \ThPdNu@AtEnd
    \catcode#1=\the\catcode#1\relax
  }%
  \catcode#1=#2\relax
}
\TMP@EnsureCode{33}{12}% !
\TMP@EnsureCode{36}{3}% $
\TMP@EnsureCode{38}{4}% &
\TMP@EnsureCode{42}{12}% *
\TMP@EnsureCode{43}{12}% +
\TMP@EnsureCode{45}{12}% -
\TMP@EnsureCode{46}{12}% .
\TMP@EnsureCode{60}{12}% <
\TMP@EnsureCode{62}{12}% >
\TMP@EnsureCode{96}{12}% `
\edef\ThPdNu@AtEnd{\ThPdNu@AtEnd\noexpand\endinput}
%    \end{macrocode}
%
% \subsection{Helper macros}
%
%    \begin{macro}{\ThPdNu@FIN}
%    \begin{macrocode}
\def\ThPdNu@FIN{\thepdfnumber@ErrorEndMarker}
%    \end{macrocode}
%    \end{macro}
%
%    \begin{macro}{\ThPdNu@space}
%    \begin{macrocode}
\def\ThPdNu@space{ }
%    \end{macrocode}
%    \end{macro}
%
%    \begin{macro}{\ThPdNu@zero}
%    \begin{macrocode}
\chardef\ThPdNu@zero=0 %
%    \end{macrocode}
%    \end{macro}
%
%    \begin{macro}{\ThPdNu@one}
%    \begin{macrocode}
\chardef\ThPdNu@one=1 %
%    \end{macrocode}
%    \end{macro}
%
%    \begin{macro}{\ThPdNu@firstoftwo}
%    \begin{macrocode}
\long\def\ThPdNu@firstoftwo#1#2{#1}
%    \end{macrocode}
%    \end{macro}
%
%    \begin{macro}{\ThPdNu@secondoftwo}
%    \begin{macrocode}
\long\def\ThPdNu@secondoftwo#1#2{#2}
%    \end{macrocode}
%    \end{macro}
%
% \subsection{Detect \hologo{eTeX}}
%
%    \begin{macrocode}
\begingroup\expandafter\expandafter\expandafter\endgroup
\expandafter\ifx\csname detokenize\endcsname\relax
  \catcode`\&=14 %
  \catcode`\$=9 %
\else
  \catcode`\&=9 %
  \catcode`\$=14 %
\fi
%    \end{macrocode}
%
% \subsection{User macro \cs{thepdfnumber}}
%
%    \begin{macro}{\thepdfnumber}
%    \begin{macrocode}
\def\thepdfnumber#1{%
  \romannumeral
& \iftrue\expandafter\ThPdNu@State@Plus\expandafter\fi
& \detokenize\expandafter{#1}%
& \ThPdNu@FIN
$ \ifx\ThPdNu@FIN#1\ThPdNu@FIN
$   \expandafter\ThPdNu@firstoftwo
$ \else
$   \expandafter\ThPdNu@secondoftwo
$ \fi
$ {%
$   \ThPdNu@zero
$   0\thepdfnumber@ErrorUnexpectedEnd
$ }{%
$   \iftrue\expandafter\ThPdNu@State@Plus\expandafter\fi#1\ThPdNu@FIN
$ }%
}
%    \end{macrocode}
%    \end{macro}
%
% \subsubsection{State definitions for sign}
%
%    \begin{macro}{\ThPdNu@State@Plus}
%    \begin{macrocode}
\def\ThPdNu@State@Plus#1\fi#2{%
  \fi
  \ifcase\ifx\ThPdNu@FIN#2%
           0%
&        \else\ifx-#2%
$        \else\ifnum`#2=45 % -
           1%
         \else\ifx0#2%
           2%
         \else\ifnum`#2>48 %
           \ifnum`#2<58 %
             3%
           \else
             9%
           \fi
&        \else\ifx.#2%
$        \else\ifnum`#2=46 % .
           4%
&        \else\ifx+#2%
$        \else\ifnum`#2=43 % +
           5%
         \else
           9%
         \fi\fi\fi\fi\fi\fi\ThPdNu@space
    \expandafter\ThPdNu@zero
    \expandafter0%
    \expandafter\thepdfnumber@ErrorUnexpectedEnd
  \or
    \ThPdNu@State@Minus
  \or
    \ThPdNu@State@SkipZeros!%
  \or
    \ThPdNu@State@Int!#2!%
  \or
    \ThPdNu@State@Dot!\ThPdNu@zero*\ThPdNu@zero!!%
  \or
    \ThPdNu@State@Plus
  \else
    \ThPdNu@ReturnError{0}%
  \fi
}
%    \end{macrocode}
%    \end{macro}
%
%    \begin{macro}{\ThPdNu@State@Minus}
%    \begin{macrocode}
\def\ThPdNu@State@Minus#1\fi#2{%
  \fi
  \ifcase\ifx\ThPdNu@FIN#2%
           0%
         \else\ifx0#2%
           1%
         \else\ifnum`#2>48 %
           \ifnum`#2<58 %
             2%
           \else
             9%
           \fi
&        \else\ifx.#2%
$        \else\ifnum`#2=46 % .
           3%
&        \else\ifx-#2%
$        \else\ifnum`#2=45 % -
           4%
&        \else\ifx+#2%
$        \else\ifnum`#2=43 % +
           5%
         \else
           9%
         \fi\fi\fi\fi\fi\fi\ThPdNu@space
    \expandafter\ThPdNu@zero
    \expandafter0%
    \expandafter\thepdfnumber@ErrorUnexpectedEnd
  \or
    \ThPdNu@State@SkipZeros-!%
  \or
    \ThPdNu@State@Int-!#2!%
  \or
    \ThPdNu@State@Dot-!\ThPdNu@zero*\ThPdNu@zero!!%
  \or
    \ThPdNu@State@Plus
  \or
    \ThPdNu@State@Minus
  \else
    \ThPdNu@ReturnError{0}%
  \fi
}
%    \end{macrocode}
%    \end{macro}
%
%    \begin{macro}{\ThPdNu@ReturnError}
%    \begin{macrocode}
\def\ThPdNu@ReturnError#1#2\fi#3\ThPdNu@FIN{%
  \fi
  \ThPdNu@zero
  #1%
  \thepdfnumber@ErrorInvalidToken
}
%    \end{macrocode}
%    \end{macro}
%
% \subsubsection{State definitions for integer part}
%
%    \begin{macro}{\ThPdNu@State@SkipZeros}
%    \begin{macrocode}
\def\ThPdNu@State@SkipZeros#1!#2\fi#3{%
  \fi
  \ifcase\ifx\ThPdNu@FIN#3%
           0%
         \else\ifx0#3%
           1%
         \else\ifnum`#3>48 %
           \ifnum`#3<58 %
             2%
           \else
             9%
           \fi
&        \else\ifx.#3%
$        \else\ifnum`#3=46 % .
           3%
         \else
           9%
         \fi\fi\fi\fi\ThPdNu@space
    \expandafter\ThPdNu@zero
    \expandafter0%
  \or
   \ThPdNu@State@SkipZeros#1!%
  \or
    \ThPdNu@State@Int#1!#3!%
  \or
    \ThPdNu@State@Dot#1!\ThPdNu@zero*\ThPdNu@zero!!%
  \else
    \ThPdNu@ReturnError{0}%
  \fi
}
%    \end{macrocode}
%    \end{macro}
%
%    \begin{macro}{\ThPdNu@State@Int}
%    \begin{macrocode}
\def\ThPdNu@State@Int#1!#2!#3\fi#4{%
  \fi
  \ifcase\ifx\ThPdNu@FIN#4%
           0%
         \else\ifnum`#4>47 %
           \ifnum`#4<58 %
             1%
           \else
             9%
           \fi
&        \else\ifx.#4%
$        \else\ifnum`#4=46 % .
           2%
         \else
           9%
         \fi\fi\fi\ThPdNu@space
    \ThPdNu@ReturnInt{#1#2}%
  \or
    \ThPdNu@State@Int#1!#2#4!%
  \or
    \ThPdNu@State@Dot#1!\ThPdNu@one#2*\ThPdNu@zero!!%
  \else
    \ThPdNu@ReturnError{#1#2}%
  \fi
}
%    \end{macrocode}
%    \end{macro}
%
%    \begin{macro}{\ThPdNu@ReturnInt}
%    \begin{macrocode}
\def\ThPdNu@ReturnInt#1#2\fi{%
  \fi
  \ThPdNu@zero
  #1%
}
%    \end{macrocode}
%    \end{macro}
%
% \subsubsection{State definitions for decimal digits}
%
%    \begin{macro}{\ThPdNu@State@Dot}
%    \begin{macrocode}
\def\ThPdNu@State@Dot#1*#2#3!#4!#5\fi#6{%
  \fi
  \ifcase\ifx\ThPdNu@FIN#6%
           0%
         \else\ifnum`#6>48 %
           \ifnum`#6<58 %
             1%
           \else
             9%
           \fi
         \else\ifx0#6%
           2%
         \else
           9%
         \fi\fi\fi\ThPdNu@space
    \ThPdNu@ReturnNumber#1*#2#3!%
  \or
    \ThPdNu@State@Dot#1*\ThPdNu@one#3#4#6!!%
  \or
    \ThPdNu@State@DotZero#1*#2#3!#4#6!%
  \else
    \ThPdNu@ReturnNumberInvalid#1*#2#3!%
  \fi
}
%    \end{macrocode}
%    \end{macro}
%
%    \begin{macro}{\ThPdNu@State@DotZero}
%    \begin{macrocode}
\def\ThPdNu@State@DotZero#1*#2#3!#4!#5\fi#6{%
  \fi
  \ifcase\ifx\ThPdNu@FIN#6%
           0%
         \else\ifnum`#6>48 %
           \ifnum`#6<58 %
             1%
           \else
             9%
           \fi
         \else\ifx0#6%
           2%
         \else
           9%
         \fi\fi\fi\ThPdNu@space
    \ThPdNu@ReturnNumber#1*#2#3!%
  \or
    \ThPdNu@State@Dot#1*\ThPdNu@one#3#4#6!!%
  \or
    \ThPdNu@State@DotZero#1*#2#3!#4#6!%
  \else
    \ThPdNu@ReturnNumber#1*#2#3!%
  \fi
}
%    \end{macrocode}
%    \end{macro}
%
%    \begin{macro}{\ThPdNu@ReturnNumber}
%    \begin{macrocode}
\def\ThPdNu@ReturnNumber#1!#2#3*#4#5!#6\fi{%
  \fi
  \ifcase#2%
    \expandafter\ThPdNu@firstoftwo
  \else
    \expandafter\ThPdNu@secondoftwo
  \fi
  {%
    \ifcase#4%
      \expandafter\ThPdNu@firstoftwo
    \else
      \expandafter\ThPdNu@secondoftwo
    \fi
    {\ThPdNu@zero 0}%
    {\ThPdNu@zero #1.#5}%
  }{%
    \ifcase#4%
      \expandafter\ThPdNu@firstoftwo
    \else
      \expandafter\ThPdNu@secondoftwo
    \fi
    {\ThPdNu@zero #1#3}%
    {\ThPdNu@zero #1#3.#5}%
  }%
}
%    \end{macrocode}
%    \end{macro}
%
%    \begin{macro}{\ThPdNu@ReturnNumberInvalid}
%    \begin{macrocode}
\def\ThPdNu@ReturnNumberInvalid#1*#2!#3\fi#4\ThPdNu@FIN{%
  \fi
  \iftrue\ThPdNu@ReturnNumber#1*#2!\fi
  \thepdfnumber@ErrorInvalidToken
}
%    \end{macrocode}
%    \end{macro}
%
% \subsection{Norm macro}
%
%    \begin{macro}{\thepdfnumberNormZeroOne}
%    \begin{macrocode}
\def\thepdfnumberNormZeroOne#1{%
  \romannumeral
& \iftrue\expandafter\ThPbNu@StateN@Plus\expandafter\fi
& \detokenize\expandafter{#1}%
& \ThPdNu@FIN
$ \ifx\ThPdNu@FIN#1\ThPdNu@FIN
$   \expandafter\ThPdNu@firstoftwo
$ \else
$   \expandafter\ThPdNu@secondoftwo
$ \fi
$ {%
$   \ThPdNu@zero
$   0\thepdfnumber@ErrorUnexpectedEnd
$ }{%
$   \iftrue\expandafter\ThPbNu@StateN@Plus\expandafter\fi#1\ThPdNu@FIN
$ }%
}
%    \end{macrocode}
%    \end{macro}
%
% \subsubsection{State definitions for sign}
%
%    \begin{macro}{\ThPbNu@StateN@Plus}
%    \begin{macrocode}
\def\ThPbNu@StateN@Plus#1\fi#2{%
  \fi
  \ifcase\ifx\ThPdNu@FIN#2%
           0%
&        \else\ifx-#2%
$        \else\ifnum`#2=45 % -
           1%
         \else\ifx0#2%
           2%
         \else\ifnum`#2>48 %
           \ifnum`#2<58 %
             3%
           \else
             9%
           \fi
&        \else\ifx.#2%
$        \else\ifnum`#2=46 % .
           4%
&        \else\ifx+#2%
$        \else\ifnum`#2=43 % +
           5%
         \else
           9%
         \fi\fi\fi\fi\fi\fi\ThPdNu@space
    \expandafter\ThPdNu@zero
    \expandafter0%
    \expandafter\thepdfnumber@ErrorUnexpectedEnd
  \or
    \ThPbNu@StateN@Minus
  \or
    \ThPbNu@StateN@SkipZeros
  \or
    \ThPdNu@ReturnAndSkip{1}%
  \or
    \ThPbNu@StateN@Dot\ThPdNu@zero!!%
  \or
    \ThPbNu@StateN@Plus
  \else
    \ThPdNu@ReturnError{0}%
  \fi
}
%    \end{macrocode}
%    \end{macro}
%
%    \begin{macro}{\ThPbNu@StateN@Minus}
%    \begin{macrocode}
\def\ThPbNu@StateN@Minus#1\fi#2{%
  \fi
  \ifcase\ifx\ThPdNu@FIN#2%
           0%
         \else\ifnum`#2>47 %
           \ifnum`#2<58 %
             1%
           \else
             9%
           \fi
&        \else\ifx.#2%
$        \else\ifnum`#2=46 % .
           1%
&        \else\ifx-#2%
$        \else\ifnum`#2=45 % -
           2%
&        \else\ifx+#2%
$        \else\ifnum`#2=43 % +
           3%
         \else
           9%
         \fi\fi\fi\fi\fi\ThPdNu@space
    \expandafter\ThPdNu@zero
    \expandafter0%
    \expandafter\thepdfnumber@ErrorUnexpectedEnd
  \or
    \ThPdNu@ReturnAndSkip{0}%
  \or
    \ThPbNu@StateN@Plus
  \or
    \ThPbNu@StateN@Minus
  \else
    \ThPdNu@ReturnError{0}%
  \fi
}
%    \end{macrocode}
%    \end{macro}
%
%    \begin{macro}{\ThPdNu@ReturnAndSkip}
%    \begin{macrocode}
\def\ThPdNu@ReturnAndSkip#1#2\fi#3\ThPdNu@FIN{%
  \fi
  \ThPdNu@zero
  #1%
}
%    \end{macrocode}
%    \end{macro}
%
% \subsubsection{State definitions for integer part}
%
%    \begin{macro}{\ThPbNu@StateN@SkipZeros}
%    \begin{macrocode}
\def\ThPbNu@StateN@SkipZeros#1\fi#2{%
  \fi
  \ifcase\ifx\ThPdNu@FIN#2%
           0%
         \else\ifx0#2%
           1%
         \else\ifnum`#2>48 %
           \ifnum`#2<58 %
             2%
           \else
             9%
           \fi
&        \else\ifx.#2%
$        \else\ifnum`#2=46 % .
           3%
         \else
           9%
         \fi\fi\fi\fi\ThPdNu@space
    \expandafter\ThPdNu@zero
    \expandafter0%
  \or
   \ThPbNu@StateN@SkipZeros%
  \or
    \ThPdNu@ReturnAndSkip{1}%
  \or
    \ThPbNu@StateN@Dot\ThPdNu@zero!!%
  \else
    \ThPdNu@ReturnError{0}%
  \fi
}
%    \end{macrocode}
%    \end{macro}
%
% \subsubsection{State definitions for decimal digits}
%
%    \begin{macro}{\ThPbNu@StateN@Dot}
%    \begin{macrocode}
\def\ThPbNu@StateN@Dot#1#2!#3!#4\fi#5{%
  \fi
  \ifcase\ifx\ThPdNu@FIN#5%
           0%
         \else\ifnum`#5>48 %
           \ifnum`#5<58 %
             1%
           \else
             9%
           \fi
         \else\ifx0#5%
           2%
         \else
           9%
         \fi\fi\fi\ThPdNu@space
    \ThPdNu@ReturnFracNumber#1#2!%
  \or
    \ThPbNu@StateN@Dot\ThPdNu@one#2#3#5!!%
  \or
    \ThPbNu@StateN@DotZero#1#2!#3#5!%
  \else
    \ThPdNu@ReturnFracNumberInvalid#1#2!%
  \fi
}
%    \end{macrocode}
%    \end{macro}
%
%    \begin{macro}{\ThPbNu@StateN@DotZero}
%    \begin{macrocode}
\def\ThPbNu@StateN@DotZero#1#2!#3!#4\fi#5{%
  \fi
  \ifcase\ifx\ThPdNu@FIN#5%
           0%
         \else\ifnum`#5>48 %
           \ifnum`#5<58 %
             1%
           \else
             9%
           \fi
         \else\ifx0#5%
           2%
         \else
           9%
         \fi\fi\fi\ThPdNu@space
    \ThPdNu@ReturnFracNumber#1#2!%
  \or
    \ThPbNu@StateN@Dot\ThPdNu@one#2#3#5!!%
  \or
    \ThPbNu@StateN@DotZero#1#2!#3#5!%
  \else
    \ThPdNu@ReturnFracNumberInvalid#1#2!%
  \fi
}
%    \end{macrocode}
%    \end{macro}
%
%    \begin{macro}{\ThPdNu@ReturnFracNumber}
%    \begin{macrocode}
\def\ThPdNu@ReturnFracNumber#1#2!#3\fi{%
  \fi
  \ifcase#1%
    \expandafter\ThPdNu@firstoftwo
  \else
    \expandafter\ThPdNu@secondoftwo
  \fi
  {\ThPdNu@zero 0}%
  {\ThPdNu@zero .#2}%
}
%    \end{macrocode}
%    \end{macro}
%
%    \begin{macro}{\ThPdNu@ReturnFracNumberInvalid}
%    \begin{macrocode}
\def\ThPdNu@ReturnFracNumberInvalid#1!#2\fi#3\ThPdNu@FIN{%
  \fi
  \iftrue\ThPdNu@ReturnFracNumber#1!\fi
  \thepdfnumber@ErrorInvalidToken
}
%    \end{macrocode}
%    \end{macro}
%
%    \begin{macrocode}
\ThPdNu@AtEnd%
%</package>
%    \end{macrocode}
%
% \section{Test}
%
% \subsection{Catcode checks for loading}
%
%    \begin{macrocode}
%<*test1>
%    \end{macrocode}
%    \begin{macrocode}
\catcode`\{=1 %
\catcode`\}=2 %
\catcode`\#=6 %
\catcode`\@=11 %
\expandafter\ifx\csname count@\endcsname\relax
  \countdef\count@=255 %
\fi
\expandafter\ifx\csname @gobble\endcsname\relax
  \long\def\@gobble#1{}%
\fi
\expandafter\ifx\csname @firstofone\endcsname\relax
  \long\def\@firstofone#1{#1}%
\fi
\expandafter\ifx\csname loop\endcsname\relax
  \expandafter\@firstofone
\else
  \expandafter\@gobble
\fi
{%
  \def\loop#1\repeat{%
    \def\body{#1}%
    \iterate
  }%
  \def\iterate{%
    \body
      \let\next\iterate
    \else
      \let\next\relax
    \fi
    \next
  }%
  \let\repeat=\fi
}%
\def\RestoreCatcodes{}
\count@=0 %
\loop
  \edef\RestoreCatcodes{%
    \RestoreCatcodes
    \catcode\the\count@=\the\catcode\count@\relax
  }%
\ifnum\count@<255 %
  \advance\count@ 1 %
\repeat

\def\RangeCatcodeInvalid#1#2{%
  \count@=#1\relax
  \loop
    \catcode\count@=15 %
  \ifnum\count@<#2\relax
    \advance\count@ 1 %
  \repeat
}
\def\RangeCatcodeCheck#1#2#3{%
  \count@=#1\relax
  \loop
    \ifnum#3=\catcode\count@
    \else
      \errmessage{%
        Character \the\count@\space
        with wrong catcode \the\catcode\count@\space
        instead of \number#3%
      }%
    \fi
  \ifnum\count@<#2\relax
    \advance\count@ 1 %
  \repeat
}
\def\space{ }
\expandafter\ifx\csname LoadCommand\endcsname\relax
  \def\LoadCommand{\input thepdfnumber.sty\relax}%
\fi
\def\Test{%
  \RangeCatcodeInvalid{0}{47}%
  \RangeCatcodeInvalid{58}{64}%
  \RangeCatcodeInvalid{91}{96}%
  \RangeCatcodeInvalid{123}{255}%
  \catcode`\@=12 %
  \catcode`\\=0 %
  \catcode`\%=14 %
  \LoadCommand
  \RangeCatcodeCheck{0}{36}{15}%
  \RangeCatcodeCheck{37}{37}{14}%
  \RangeCatcodeCheck{38}{47}{15}%
  \RangeCatcodeCheck{48}{57}{12}%
  \RangeCatcodeCheck{58}{63}{15}%
  \RangeCatcodeCheck{64}{64}{12}%
  \RangeCatcodeCheck{65}{90}{11}%
  \RangeCatcodeCheck{91}{91}{15}%
  \RangeCatcodeCheck{92}{92}{0}%
  \RangeCatcodeCheck{93}{96}{15}%
  \RangeCatcodeCheck{97}{122}{11}%
  \RangeCatcodeCheck{123}{255}{15}%
  \RestoreCatcodes
}
\Test
\csname @@end\endcsname
\end
%    \end{macrocode}
%    \begin{macrocode}
%</test1>
%    \end{macrocode}
%
% \subsection{Macro tests}
%
% \subsubsection{Preamble}
%
%    \begin{macrocode}
%<*test2>
\catcode`\{=1
\catcode`\}=2
\catcode`\#=6
\catcode`\@=11
\errorcontextlines=10000 %
\def\msg#{\immediate\write16}
\def\space{ }
\begingroup\expandafter\expandafter\expandafter\endgroup
\expandafter\ifx\csname RequirePackage\endcsname\relax
  \input thepdfnumber.sty\relax
\else
  \RequirePackage{thepdfnumber}[2016/05/16]%
\fi
%    \end{macrocode}
%
%    \begin{macro}{\Test}
%    \begin{macrocode}
\def\Test#1#2{%
  \def\TestExpected{#2}%
%<*active>
  \edef\TestExpected{\TestExpected}%
%</active>
  \expandafter\expandafter\expandafter\def
  \expandafter\expandafter\expandafter\TestResult
  \expandafter\expandafter\expandafter{%
    \TestCommand{#1}%
  }%
  \begingroup
    \toks0\expandafter{\TestResult}%
    \toks2\expandafter{\TestExpected}%
    \ifx\TestResult\TestExpected
      \msg{* OK: #1 => \the\toks0}%
    \else
      \errmessage{FAILED: #1 => \the\toks0\space(\the\toks2)}%
    \fi
  \endgroup
}
%    \end{macrocode}
%    \end{macro}
%
%    \begin{macrocode}
\let\thepdfnumber@ErrorUnexpectedEnd\relax
\let\thepdfnumber@ErrorInvalidToken\relax
%    \end{macrocode}
%
% \subsubsection{Test with active catcodes}
%
%    \begin{macrocode}
%<*active>
\catcode`\~=13 %
\def\ThPdNu@Temp#1{%
  \begingroup
    \lccode`\~=`#1 %
  \lowercase{\endgroup
    \edef~%
  }{\string#1}%
  \catcode`#1=13 %
}
\ThPdNu@Temp{.}
\ThPdNu@Temp{-}
\ThPdNu@Temp{+}
%</active>
%    \end{macrocode}
%
% \subsubsection{Test cases for \cs{thepdfnumber}}
%
%    \begin{macro}{\TestCommand}
%    \begin{macrocode}
\let\TestCommand\thepdfnumber
%    \end{macrocode}
%    \end{macro}
%
%    \begin{macrocode}
\Test{0}{0}
\Test{1}{1}
\Test{23}{23}
\Test{0045}{45}
\Test{4500}{4500}
\Test{-0}{0}
\Test{-1}{-1}
\Test{-00002}{-2}
\Test{-010203}{-10203}
\Test{+0}{0}
\Test{+123}{123}

\Test{+-4}{-4}
\Test{+-+-4}{4}
\Test{++--++--++4}{4}
\Test{+++---+++---+++---4}{-4}

\Test{.}{0}
\Test{0.}{0}
\Test{.0}{0}
\Test{0.}{0}
\Test{-.}{0}
\Test{-0.}{0}
\Test{-.0}{0}
\Test{-0.}{0}
\Test{010.020}{10.02}
\Test{123.456}{123.456}
\Test{12.}{12}
\Test{.123}{.123}
\Test{.0001}{.0001}
\Test{.00100}{.001}
\Test{.12003400560078009}{.12003400560078009}
\Test{-.12003400560078009}{-.12003400560078009}
\Test{04.0500000}{4.05}

\Test{}{0\thepdfnumber@ErrorUnexpectedEnd}
\Test{+}{0\thepdfnumber@ErrorUnexpectedEnd}
\Test{-}{0\thepdfnumber@ErrorUnexpectedEnd}
\Test{a}{0\thepdfnumber@ErrorInvalidToken}
\Test{0x1}{0\thepdfnumber@ErrorInvalidToken}
\Test{4x56}{4\thepdfnumber@ErrorInvalidToken}
\Test{012,34}{12\thepdfnumber@ErrorInvalidToken}
\Test{0.12x4}{.12\thepdfnumber@ErrorInvalidToken}
%    \end{macrocode}
%
% \subsubsection{Test cases for \cs{thepdfnumberNormZeroOne}}
%
%    \begin{macro}{\TestCommand}
%    \begin{macrocode}
\let\TestCommand\thepdfnumberNormZeroOne
%    \end{macrocode}
%    \end{macro}
%
%    \begin{macrocode}
\Test{0}{0}
\Test{1}{1}
\Test{23}{1}
\Test{12}{1}
\Test{10}{1}
\Test{0045}{1}
\Test{001}{1}
\Test{-0}{0}
\Test{-1}{0}
\Test{-01}{0}
\Test{-4}{0}

\Test{+-0}{0}
\Test{+-+-1}{1}
\Test{++--++--++1}{1}
\Test{+++---+++---+++---1}{0}

\Test{.}{0}
\Test{0.}{0}
\Test{.0}{0}
\Test{0.}{0}
\Test{-.}{0}
\Test{-0.}{0}
\Test{-.0}{0}
\Test{-0.}{0}
\Test{010.020}{1}
\Test{123.456}{1}
\Test{12.}{1}
\Test{.123}{.123}
\Test{.0001}{.0001}
\Test{.00100}{.001}
\Test{.12003400560078009}{.12003400560078009}
\Test{-.12003400560078009}{0}
\Test{04.0500000}{1}
\Test{0.1200340056}{.1200340056}
\Test{1.05}{1}

\Test{}{0\thepdfnumber@ErrorUnexpectedEnd}
\Test{+}{0\thepdfnumber@ErrorUnexpectedEnd}
\Test{-}{0\thepdfnumber@ErrorUnexpectedEnd}
\Test{a}{0\thepdfnumber@ErrorInvalidToken}
\Test{0x1}{0\thepdfnumber@ErrorInvalidToken}
\Test{4x56}{1}
\Test{012,34}{1}
\Test{.012x4}{.012\thepdfnumber@ErrorInvalidToken}
%    \end{macrocode}
%
%    \begin{macrocode}
\csname @@end\endcsname\end
%</test2>
%    \end{macrocode}
%
% \section{Installation}
%
% \subsection{Download}
%
% \paragraph{Package.} This package is available on
% CTAN\footnote{\url{http://ctan.org/pkg/thepdfnumber}}:
% \begin{description}
% \item[\CTAN{macros/latex/contrib/oberdiek/thepdfnumber.dtx}] The source file.
% \item[\CTAN{macros/latex/contrib/oberdiek/thepdfnumber.pdf}] Documentation.
% \end{description}
%
%
% \paragraph{Bundle.} All the packages of the bundle `oberdiek'
% are also available in a TDS compliant ZIP archive. There
% the packages are already unpacked and the documentation files
% are generated. The files and directories obey the TDS standard.
% \begin{description}
% \item[\CTAN{install/macros/latex/contrib/oberdiek.tds.zip}]
% \end{description}
% \emph{TDS} refers to the standard ``A Directory Structure
% for \TeX\ Files'' (\CTAN{tds/tds.pdf}). Directories
% with \xfile{texmf} in their name are usually organized this way.
%
% \subsection{Bundle installation}
%
% \paragraph{Unpacking.} Unpack the \xfile{oberdiek.tds.zip} in the
% TDS tree (also known as \xfile{texmf} tree) of your choice.
% Example (linux):
% \begin{quote}
%   |unzip oberdiek.tds.zip -d ~/texmf|
% \end{quote}
%
% \paragraph{Script installation.}
% Check the directory \xfile{TDS:scripts/oberdiek/} for
% scripts that need further installation steps.
% Package \xpackage{attachfile2} comes with the Perl script
% \xfile{pdfatfi.pl} that should be installed in such a way
% that it can be called as \texttt{pdfatfi}.
% Example (linux):
% \begin{quote}
%   |chmod +x scripts/oberdiek/pdfatfi.pl|\\
%   |cp scripts/oberdiek/pdfatfi.pl /usr/local/bin/|
% \end{quote}
%
% \subsection{Package installation}
%
% \paragraph{Unpacking.} The \xfile{.dtx} file is a self-extracting
% \docstrip\ archive. The files are extracted by running the
% \xfile{.dtx} through \plainTeX:
% \begin{quote}
%   \verb|tex thepdfnumber.dtx|
% \end{quote}
%
% \paragraph{TDS.} Now the different files must be moved into
% the different directories in your installation TDS tree
% (also known as \xfile{texmf} tree):
% \begin{quote}
% \def\t{^^A
% \begin{tabular}{@{}>{\ttfamily}l@{ $\rightarrow$ }>{\ttfamily}l@{}}
%   thepdfnumber.sty & tex/generic/oberdiek/thepdfnumber.sty\\
%   thepdfnumber.pdf & doc/latex/oberdiek/thepdfnumber.pdf\\
%   test/thepdfnumber-test1.tex & doc/latex/oberdiek/test/thepdfnumber-test1.tex\\
%   test/thepdfnumber-test2.tex & doc/latex/oberdiek/test/thepdfnumber-test2.tex\\
%   test/thepdfnumber-test3.tex & doc/latex/oberdiek/test/thepdfnumber-test3.tex\\
%   thepdfnumber.dtx & source/latex/oberdiek/thepdfnumber.dtx\\
% \end{tabular}^^A
% }^^A
% \sbox0{\t}^^A
% \ifdim\wd0>\linewidth
%   \begingroup
%     \advance\linewidth by\leftmargin
%     \advance\linewidth by\rightmargin
%   \edef\x{\endgroup
%     \def\noexpand\lw{\the\linewidth}^^A
%   }\x
%   \def\lwbox{^^A
%     \leavevmode
%     \hbox to \linewidth{^^A
%       \kern-\leftmargin\relax
%       \hss
%       \usebox0
%       \hss
%       \kern-\rightmargin\relax
%     }^^A
%   }^^A
%   \ifdim\wd0>\lw
%     \sbox0{\small\t}^^A
%     \ifdim\wd0>\linewidth
%       \ifdim\wd0>\lw
%         \sbox0{\footnotesize\t}^^A
%         \ifdim\wd0>\linewidth
%           \ifdim\wd0>\lw
%             \sbox0{\scriptsize\t}^^A
%             \ifdim\wd0>\linewidth
%               \ifdim\wd0>\lw
%                 \sbox0{\tiny\t}^^A
%                 \ifdim\wd0>\linewidth
%                   \lwbox
%                 \else
%                   \usebox0
%                 \fi
%               \else
%                 \lwbox
%               \fi
%             \else
%               \usebox0
%             \fi
%           \else
%             \lwbox
%           \fi
%         \else
%           \usebox0
%         \fi
%       \else
%         \lwbox
%       \fi
%     \else
%       \usebox0
%     \fi
%   \else
%     \lwbox
%   \fi
% \else
%   \usebox0
% \fi
% \end{quote}
% If you have a \xfile{docstrip.cfg} that configures and enables \docstrip's
% TDS installing feature, then some files can already be in the right
% place, see the documentation of \docstrip.
%
% \subsection{Refresh file name databases}
%
% If your \TeX~distribution
% (\teTeX, \mikTeX, \dots) relies on file name databases, you must refresh
% these. For example, \teTeX\ users run \verb|texhash| or
% \verb|mktexlsr|.
%
% \subsection{Some details for the interested}
%
% \paragraph{Attached source.}
%
% The PDF documentation on CTAN also includes the
% \xfile{.dtx} source file. It can be extracted by
% AcrobatReader 6 or higher. Another option is \textsf{pdftk},
% e.g. unpack the file into the current directory:
% \begin{quote}
%   \verb|pdftk thepdfnumber.pdf unpack_files output .|
% \end{quote}
%
% \paragraph{Unpacking with \LaTeX.}
% The \xfile{.dtx} chooses its action depending on the format:
% \begin{description}
% \item[\plainTeX:] Run \docstrip\ and extract the files.
% \item[\LaTeX:] Generate the documentation.
% \end{description}
% If you insist on using \LaTeX\ for \docstrip\ (really,
% \docstrip\ does not need \LaTeX), then inform the autodetect routine
% about your intention:
% \begin{quote}
%   \verb|latex \let\install=y\input{thepdfnumber.dtx}|
% \end{quote}
% Do not forget to quote the argument according to the demands
% of your shell.
%
% \paragraph{Generating the documentation.}
% You can use both the \xfile{.dtx} or the \xfile{.drv} to generate
% the documentation. The process can be configured by the
% configuration file \xfile{ltxdoc.cfg}. For instance, put this
% line into this file, if you want to have A4 as paper format:
% \begin{quote}
%   \verb|\PassOptionsToClass{a4paper}{article}|
% \end{quote}
% An example follows how to generate the
% documentation with pdf\LaTeX:
% \begin{quote}
%\begin{verbatim}
%pdflatex thepdfnumber.dtx
%makeindex -s gind.ist thepdfnumber.idx
%pdflatex thepdfnumber.dtx
%makeindex -s gind.ist thepdfnumber.idx
%pdflatex thepdfnumber.dtx
%\end{verbatim}
% \end{quote}
%
% \printbibliography[
%   heading=bibnumbered,
% ]
%
% \begin{History}
%   \begin{Version}{2011/11/24 v1.0}
%   \item
%     First version.
%   \end{Version}
%   \begin{Version}{2016/05/16 v1.1}
%   \item
%     Documentation updates.
%   \end{Version}
% \end{History}
%
% \PrintIndex
%
% \Finale
\endinput

%        (quote the arguments according to the demands of your shell)
%
% Documentation:
%    (a) If thepdfnumber.drv is present:
%           latex thepdfnumber.drv
%    (b) Without thepdfnumber.drv:
%           latex thepdfnumber.dtx; ...
%    The class ltxdoc loads the configuration file ltxdoc.cfg
%    if available. Here you can specify further options, e.g.
%    use A4 as paper format:
%       \PassOptionsToClass{a4paper}{article}
%
%    Programm calls to get the documentation (example):
%       pdflatex thepdfnumber.dtx
%       makeindex -s gind.ist thepdfnumber.idx
%       pdflatex thepdfnumber.dtx
%       makeindex -s gind.ist thepdfnumber.idx
%       pdflatex thepdfnumber.dtx
%
% Installation:
%    TDS:tex/generic/oberdiek/thepdfnumber.sty
%    TDS:doc/latex/oberdiek/thepdfnumber.pdf
%    TDS:doc/latex/oberdiek/test/thepdfnumber-test1.tex
%    TDS:doc/latex/oberdiek/test/thepdfnumber-test2.tex
%    TDS:doc/latex/oberdiek/test/thepdfnumber-test3.tex
%    TDS:source/latex/oberdiek/thepdfnumber.dtx
%
%<*ignore>
\begingroup
  \catcode123=1 %
  \catcode125=2 %
  \def\x{LaTeX2e}%
\expandafter\endgroup
\ifcase 0\ifx\install y1\fi\expandafter
         \ifx\csname processbatchFile\endcsname\relax\else1\fi
         \ifx\fmtname\x\else 1\fi\relax
\else\csname fi\endcsname
%</ignore>
%<*install>
\input docstrip.tex
\Msg{************************************************************************}
\Msg{* Installation}
\Msg{* Package: thepdfnumber 2016/05/16 v1.1 Print PDF numbers with minimal digits (HO)}
\Msg{************************************************************************}

\keepsilent
\askforoverwritefalse

\let\MetaPrefix\relax
\preamble

This is a generated file.

Project: thepdfnumber
Version: 2016/05/16 v1.1

Copyright (C) 2011 by
   Heiko Oberdiek <heiko.oberdiek at googlemail.com>

This work may be distributed and/or modified under the
conditions of the LaTeX Project Public License, either
version 1.3c of this license or (at your option) any later
version. This version of this license is in
   http://www.latex-project.org/lppl/lppl-1-3c.txt
and the latest version of this license is in
   http://www.latex-project.org/lppl.txt
and version 1.3 or later is part of all distributions of
LaTeX version 2005/12/01 or later.

This work has the LPPL maintenance status "maintained".

This Current Maintainer of this work is Heiko Oberdiek.

The Base Interpreter refers to any `TeX-Format',
because some files are installed in TDS:tex/generic//.

This work consists of the main source file thepdfnumber.dtx
and the derived files
   thepdfnumber.sty, thepdfnumber.pdf, thepdfnumber.ins, thepdfnumber.drv,
   thepdfnumber-test1.tex, thepdfnumber-test2.tex,
   thepdfnumber-test3.tex.

\endpreamble
\let\MetaPrefix\DoubleperCent

\generate{%
  \file{thepdfnumber.ins}{\from{thepdfnumber.dtx}{install}}%
  \file{thepdfnumber.drv}{\from{thepdfnumber.dtx}{driver}}%
  \usedir{tex/generic/oberdiek}%
  \file{thepdfnumber.sty}{\from{thepdfnumber.dtx}{package}}%
  \usedir{doc/latex/oberdiek/test}%
  \file{thepdfnumber-test1.tex}{\from{thepdfnumber.dtx}{test1}}%
  \file{thepdfnumber-test2.tex}{\from{thepdfnumber.dtx}{test2}}%
  \file{thepdfnumber-test3.tex}{\from{thepdfnumber.dtx}{test2,active}}%
}

\catcode32=13\relax% active space
\let =\space%
\Msg{************************************************************************}
\Msg{*}
\Msg{* To finish the installation you have to move the following}
\Msg{* file into a directory searched by TeX:}
\Msg{*}
\Msg{*     thepdfnumber.sty}
\Msg{*}
\Msg{* To produce the documentation run the file `thepdfnumber.drv'}
\Msg{* through LaTeX.}
\Msg{*}
\Msg{* Happy TeXing!}
\Msg{*}
\Msg{************************************************************************}

\endbatchfile
%</install>
%<*ignore>
\fi
%</ignore>
%<*driver>
\NeedsTeXFormat{LaTeX2e}
\ProvidesFile{thepdfnumber.drv}%
  [2016/05/16 v1.1 Print PDF numbers with minimal digits (HO)]%
\documentclass{ltxdoc}
\usepackage{holtxdoc}[2011/11/22]
\usepackage{csquotes}
\usepackage[
  backend=bibtex,
  bibencoding=ascii,
  alldates=iso8601,
]{biblatex}[2011/11/13]
\bibliography{oberdiek-source}
\begin{document}
  \DocInput{thepdfnumber.dtx}%
\end{document}
%</driver>
% \fi
%
%
% \CharacterTable
%  {Upper-case    \A\B\C\D\E\F\G\H\I\J\K\L\M\N\O\P\Q\R\S\T\U\V\W\X\Y\Z
%   Lower-case    \a\b\c\d\e\f\g\h\i\j\k\l\m\n\o\p\q\r\s\t\u\v\w\x\y\z
%   Digits        \0\1\2\3\4\5\6\7\8\9
%   Exclamation   \!     Double quote  \"     Hash (number) \#
%   Dollar        \$     Percent       \%     Ampersand     \&
%   Acute accent  \'     Left paren    \(     Right paren   \)
%   Asterisk      \*     Plus          \+     Comma         \,
%   Minus         \-     Point         \.     Solidus       \/
%   Colon         \:     Semicolon     \;     Less than     \<
%   Equals        \=     Greater than  \>     Question mark \?
%   Commercial at \@     Left bracket  \[     Backslash     \\
%   Right bracket \]     Circumflex    \^     Underscore    \_
%   Grave accent  \`     Left brace    \{     Vertical bar  \|
%   Right brace   \}     Tilde         \~}
%
% \GetFileInfo{thepdfnumber.drv}
%
% \title{The \xpackage{thepdfnumber} package}
% \date{2016/05/16 v1.1}
% \author{Heiko Oberdiek\thanks
% {Please report any issues at https://github.com/ho-tex/oberdiek/issues}\\
% \xemail{heiko.oberdiek at googlemail.com}}
%
% \maketitle
%
% \begin{abstract}
% The package converts real numbers to a minimal representation
% that is stripped from leading or trailing zeros,
% plus signs and decimal point if not necessary.
% \end{abstract}
%
% \tableofcontents
%
% \section{Documentation}
%
% \subsection{Introduction}
%
% Dealing with the PDF format, there is sometimes the need to write
% some low level PDF stuff. In case of numbers, the numbers can
% arise from user input (e.g. color or transparency specifications)
% or can be calculated. For example, \hologo{LaTeX}'s \cs{strip@pt}
% makes a good job to output a real number. It automatically suppresses
% the decimal part if the number is an integer. However it leaves
% a leading zero for numbers greater zero and smaller one.
% Thus the package provides macros that can be used with different
% formats, even with \hologo{iniTeX} and generates numbers that
% are valid numbers of the PDF format and whose length is minimal.
%
% \subsection{Usage}
%
% The package \xpackage{thepdfnumber} can be used with \hologo{LaTeX},
% \hologo{plainTeX} or even with \hologo{iniTeX}:
% \begin{quote}
%   |\RequirePackage{thepdfnumber}| \textit{\% \hologo{LaTeX}}\\
%   |\input thepdfnumber.sty| \textit{\% \hologo{plainTeX}/\hologo{iniTeX}}
% \end{quote}
% The package does not need and have package options.
%
% \subsection{User macros}
%
% All user macros are expandable in exact two expansion steps.
%
% \begin{declcs}{thepdfnumber} \M{number}
% \end{declcs}
% Macro \cs{thepdfnumber} takes a number
% as argument and expands to a minimal representation of that number.
% Some examples:
% \begin{quote}
% \begin{tabular}{r@{.: }>{\ttfamily}l@{ $\rightarrow$\ }>{\ttfamily}l@{}}
%   1& +123 & 123\\
%   2& --123 & 123\\
%   3& -01 & -1\\
%   4& 0045 & 45\\
%   5& 1.0 & 1\\
%   6& 1.20 & 1.2\\
%   7& 0.0 & 0\\
%   8& 0.78 & .78\\
%   9& +012.340 & 12.34\\
% \end{tabular}
% \end{quote}
% It reduces the length of the number representation:
% \begin{itemize}
% \item The signs are collapsed and only one minus sign
%   is output if the number is negative (see examples 1, 2, 3, 9).
% \item Leading zeros are removed (4, 8, 9) unless the number is zero (7).
% \item The decimal part is omitted, if the number is an integer (5, 7).
% \item Trailing zeros from the decimal part are stripped (5, 6, 7, 9).
% \end{itemize}
% The resulting number representation can be catched with one of the
% following Perl regular expressions:
% \begin{itemize}
% \item |^0$| (zero)
% \item |^-?[1-9][0-9]*$| (integer)
% \item |^-?[0-9]*\.[0-9]*[1-9]$| (real)
% \end{itemize}
% This is a valid numeric object of the PDF specification \cite[``7.3.3~Numeric Objects'']{pdfspec-iso32000-1}.
%
% \begin{declcs}{thepdfnumberNormZeroOne}
% \end{declcs}
% There are various places in the PDF specification where the
% number is in the domain 0.0 upto 1.0. Macro thepdfnumberNormZeroOne
% automatically adjusts the number to fit into that range. Negative
% numbers are mapped to |0| and numbers greater than one are
% replaced by |1|. Thus the result fits one of the following regular
% expressions:
% \begin{itemize}
% \item |^0$|
% \item |^\.[0-9]*[1-9]$|
% \item |^1$|
% \end{itemize}
% Examples:
% \begin{quote}
% \begin{tabular}{@{}>{\ttfamily}l@{ $\rightarrow$\ }>{\ttfamily}l@{}}
%   -456 & 0\\
%   -0.001 & 0\\
%   0.0 & 0\\
%   0.010 & .01\\
%   0.456 & .456\\
%   1.0 & 1\\
%   01.001 & 1\\
%   4 & 1\\
% \end{tabular}
% \end{quote}
%
% \subsection{Input number}
%
% The user macros expect a number as argument. The number can either
% be given explicitely or as macro that expands in one step
% to an explicite number, because the first token of the argument
% is expanded once.
%
% The explicite number consists of
% \begin{itemize}
% \item optional signs `\texttt{+}' and `\texttt{-}',
% \item digits `\texttt{0}' upto `\texttt{9}' and
% \item an optional dot `\texttt{.}'.
% \end{itemize}
% All tokens must have catcode 12 (other), the default catcodes
% for these characters in \hologo{LaTeX}, \hologo{plainTeX} or
% \hologo{iniTeX}. As Perl regular expression the number is expected
% in one of the following forms:
% \begin{itemize}
% \item |^[+-]*[0-9]+$|
% \item |^[+-]*[0-9]*\.[0-9]*$|
% \end{itemize}
% At least one digit or the dot must be present.
%
% \subsection{Error handling}
%
% The package is not intended for validating numbers or to
% decide if an argument is a number. Therefore it is an usage
% error to use the user macros with arguments that are not
% explicite numbers as specified in the previous sections.
% Nevertheless some error conditions are sometimes recognized.
% Errors are given in form of an undefined command sequence.
% It is the only way to notify \hologo{TeX} in expandable context.
% Expanding to some error text would invalidate the output.
% Currently the following errors are thrown:
% \begin{description}
% \item[\cs{thepdfnumber@ErrorEndMarker}:]
%   Internally the argument parsing uses an end marker that
%   is never called directly. If it is called with valid
%   user input, then this is a bug. Otherwise it means the user input
%   contains nasty stuff.
% \item[\cs{thepdfnumber@ErrorUnexpectedEnd}:] The macros expect at least one digit
%   or the dot, otherwise if the argument is empty or only contains
%   signs, then this error is called.
% \item[\cs{thepdfnumber@ErrorInvalidToken}:] It is called if the number
%   contains other tokens than signs, digits or the dot or the
%   token at the wrong place (e.g. a sign after a digit).
%   In case of \cs{thepdfnumberNormZeroOne} this error condition might
%   not always be detected, because the number parsing might stop
%   at an early point, when the result is already clear (e.g.
%   if the number is negative or will be greater than one).
% \item[Improper alphabetic constant:] This error might be
%   thrown by \hologo{TeX}, if the number contains
%   command tokens instead of characters.
% \end{description}
%
% \StopEventually{
% }
%
% \section{Implementation}
%
%    \begin{macrocode}
%<*package>
%    \end{macrocode}
%
% \subsection{Reload check and package identification}
%    Reload check, especially if the package is not used with \LaTeX.
%    \begin{macrocode}
\begingroup\catcode61\catcode48\catcode32=10\relax%
  \catcode13=5 % ^^M
  \endlinechar=13 %
  \catcode35=6 % #
  \catcode39=12 % '
  \catcode44=12 % ,
  \catcode45=12 % -
  \catcode46=12 % .
  \catcode58=12 % :
  \catcode64=11 % @
  \catcode123=1 % {
  \catcode125=2 % }
  \expandafter\let\expandafter\x\csname ver@thepdfnumber.sty\endcsname
  \ifx\x\relax % plain-TeX, first loading
  \else
    \def\empty{}%
    \ifx\x\empty % LaTeX, first loading,
      % variable is initialized, but \ProvidesPackage not yet seen
    \else
      \expandafter\ifx\csname PackageInfo\endcsname\relax
        \def\x#1#2{%
          \immediate\write-1{Package #1 Info: #2.}%
        }%
      \else
        \def\x#1#2{\PackageInfo{#1}{#2, stopped}}%
      \fi
      \x{thepdfnumber}{The package is already loaded}%
      \aftergroup\endinput
    \fi
  \fi
\endgroup%
%    \end{macrocode}
%    Package identification:
%    \begin{macrocode}
\begingroup\catcode61\catcode48\catcode32=10\relax%
  \catcode13=5 % ^^M
  \endlinechar=13 %
  \catcode35=6 % #
  \catcode39=12 % '
  \catcode40=12 % (
  \catcode41=12 % )
  \catcode44=12 % ,
  \catcode45=12 % -
  \catcode46=12 % .
  \catcode47=12 % /
  \catcode58=12 % :
  \catcode64=11 % @
  \catcode91=12 % [
  \catcode93=12 % ]
  \catcode123=1 % {
  \catcode125=2 % }
  \expandafter\ifx\csname ProvidesPackage\endcsname\relax
    \def\x#1#2#3[#4]{\endgroup
      \immediate\write-1{Package: #3 #4}%
      \xdef#1{#4}%
    }%
  \else
    \def\x#1#2[#3]{\endgroup
      #2[{#3}]%
      \ifx#1\@undefined
        \xdef#1{#3}%
      \fi
      \ifx#1\relax
        \xdef#1{#3}%
      \fi
    }%
  \fi
\expandafter\x\csname ver@thepdfnumber.sty\endcsname
\ProvidesPackage{thepdfnumber}%
  [2016/05/16 v1.1 Print PDF numbers with minimal digits (HO)]%
%    \end{macrocode}
%
% \subsection{Catcodes}
%
%    \begin{macrocode}
\begingroup\catcode61\catcode48\catcode32=10\relax%
  \catcode13=5 % ^^M
  \endlinechar=13 %
  \catcode123=1 % {
  \catcode125=2 % }
  \catcode64=11 % @
  \def\x{\endgroup
    \expandafter\edef\csname ThPdNu@AtEnd\endcsname{%
      \endlinechar=\the\endlinechar\relax
      \catcode13=\the\catcode13\relax
      \catcode32=\the\catcode32\relax
      \catcode35=\the\catcode35\relax
      \catcode61=\the\catcode61\relax
      \catcode64=\the\catcode64\relax
      \catcode123=\the\catcode123\relax
      \catcode125=\the\catcode125\relax
    }%
  }%
\x\catcode61\catcode48\catcode32=10\relax%
\catcode13=5 % ^^M
\endlinechar=13 %
\catcode35=6 % #
\catcode64=11 % @
\catcode123=1 % {
\catcode125=2 % }
\def\TMP@EnsureCode#1#2{%
  \edef\ThPdNu@AtEnd{%
    \ThPdNu@AtEnd
    \catcode#1=\the\catcode#1\relax
  }%
  \catcode#1=#2\relax
}
\TMP@EnsureCode{33}{12}% !
\TMP@EnsureCode{36}{3}% $
\TMP@EnsureCode{38}{4}% &
\TMP@EnsureCode{42}{12}% *
\TMP@EnsureCode{43}{12}% +
\TMP@EnsureCode{45}{12}% -
\TMP@EnsureCode{46}{12}% .
\TMP@EnsureCode{60}{12}% <
\TMP@EnsureCode{62}{12}% >
\TMP@EnsureCode{96}{12}% `
\edef\ThPdNu@AtEnd{\ThPdNu@AtEnd\noexpand\endinput}
%    \end{macrocode}
%
% \subsection{Helper macros}
%
%    \begin{macro}{\ThPdNu@FIN}
%    \begin{macrocode}
\def\ThPdNu@FIN{\thepdfnumber@ErrorEndMarker}
%    \end{macrocode}
%    \end{macro}
%
%    \begin{macro}{\ThPdNu@space}
%    \begin{macrocode}
\def\ThPdNu@space{ }
%    \end{macrocode}
%    \end{macro}
%
%    \begin{macro}{\ThPdNu@zero}
%    \begin{macrocode}
\chardef\ThPdNu@zero=0 %
%    \end{macrocode}
%    \end{macro}
%
%    \begin{macro}{\ThPdNu@one}
%    \begin{macrocode}
\chardef\ThPdNu@one=1 %
%    \end{macrocode}
%    \end{macro}
%
%    \begin{macro}{\ThPdNu@firstoftwo}
%    \begin{macrocode}
\long\def\ThPdNu@firstoftwo#1#2{#1}
%    \end{macrocode}
%    \end{macro}
%
%    \begin{macro}{\ThPdNu@secondoftwo}
%    \begin{macrocode}
\long\def\ThPdNu@secondoftwo#1#2{#2}
%    \end{macrocode}
%    \end{macro}
%
% \subsection{Detect \hologo{eTeX}}
%
%    \begin{macrocode}
\begingroup\expandafter\expandafter\expandafter\endgroup
\expandafter\ifx\csname detokenize\endcsname\relax
  \catcode`\&=14 %
  \catcode`\$=9 %
\else
  \catcode`\&=9 %
  \catcode`\$=14 %
\fi
%    \end{macrocode}
%
% \subsection{User macro \cs{thepdfnumber}}
%
%    \begin{macro}{\thepdfnumber}
%    \begin{macrocode}
\def\thepdfnumber#1{%
  \romannumeral
& \iftrue\expandafter\ThPdNu@State@Plus\expandafter\fi
& \detokenize\expandafter{#1}%
& \ThPdNu@FIN
$ \ifx\ThPdNu@FIN#1\ThPdNu@FIN
$   \expandafter\ThPdNu@firstoftwo
$ \else
$   \expandafter\ThPdNu@secondoftwo
$ \fi
$ {%
$   \ThPdNu@zero
$   0\thepdfnumber@ErrorUnexpectedEnd
$ }{%
$   \iftrue\expandafter\ThPdNu@State@Plus\expandafter\fi#1\ThPdNu@FIN
$ }%
}
%    \end{macrocode}
%    \end{macro}
%
% \subsubsection{State definitions for sign}
%
%    \begin{macro}{\ThPdNu@State@Plus}
%    \begin{macrocode}
\def\ThPdNu@State@Plus#1\fi#2{%
  \fi
  \ifcase\ifx\ThPdNu@FIN#2%
           0%
&        \else\ifx-#2%
$        \else\ifnum`#2=45 % -
           1%
         \else\ifx0#2%
           2%
         \else\ifnum`#2>48 %
           \ifnum`#2<58 %
             3%
           \else
             9%
           \fi
&        \else\ifx.#2%
$        \else\ifnum`#2=46 % .
           4%
&        \else\ifx+#2%
$        \else\ifnum`#2=43 % +
           5%
         \else
           9%
         \fi\fi\fi\fi\fi\fi\ThPdNu@space
    \expandafter\ThPdNu@zero
    \expandafter0%
    \expandafter\thepdfnumber@ErrorUnexpectedEnd
  \or
    \ThPdNu@State@Minus
  \or
    \ThPdNu@State@SkipZeros!%
  \or
    \ThPdNu@State@Int!#2!%
  \or
    \ThPdNu@State@Dot!\ThPdNu@zero*\ThPdNu@zero!!%
  \or
    \ThPdNu@State@Plus
  \else
    \ThPdNu@ReturnError{0}%
  \fi
}
%    \end{macrocode}
%    \end{macro}
%
%    \begin{macro}{\ThPdNu@State@Minus}
%    \begin{macrocode}
\def\ThPdNu@State@Minus#1\fi#2{%
  \fi
  \ifcase\ifx\ThPdNu@FIN#2%
           0%
         \else\ifx0#2%
           1%
         \else\ifnum`#2>48 %
           \ifnum`#2<58 %
             2%
           \else
             9%
           \fi
&        \else\ifx.#2%
$        \else\ifnum`#2=46 % .
           3%
&        \else\ifx-#2%
$        \else\ifnum`#2=45 % -
           4%
&        \else\ifx+#2%
$        \else\ifnum`#2=43 % +
           5%
         \else
           9%
         \fi\fi\fi\fi\fi\fi\ThPdNu@space
    \expandafter\ThPdNu@zero
    \expandafter0%
    \expandafter\thepdfnumber@ErrorUnexpectedEnd
  \or
    \ThPdNu@State@SkipZeros-!%
  \or
    \ThPdNu@State@Int-!#2!%
  \or
    \ThPdNu@State@Dot-!\ThPdNu@zero*\ThPdNu@zero!!%
  \or
    \ThPdNu@State@Plus
  \or
    \ThPdNu@State@Minus
  \else
    \ThPdNu@ReturnError{0}%
  \fi
}
%    \end{macrocode}
%    \end{macro}
%
%    \begin{macro}{\ThPdNu@ReturnError}
%    \begin{macrocode}
\def\ThPdNu@ReturnError#1#2\fi#3\ThPdNu@FIN{%
  \fi
  \ThPdNu@zero
  #1%
  \thepdfnumber@ErrorInvalidToken
}
%    \end{macrocode}
%    \end{macro}
%
% \subsubsection{State definitions for integer part}
%
%    \begin{macro}{\ThPdNu@State@SkipZeros}
%    \begin{macrocode}
\def\ThPdNu@State@SkipZeros#1!#2\fi#3{%
  \fi
  \ifcase\ifx\ThPdNu@FIN#3%
           0%
         \else\ifx0#3%
           1%
         \else\ifnum`#3>48 %
           \ifnum`#3<58 %
             2%
           \else
             9%
           \fi
&        \else\ifx.#3%
$        \else\ifnum`#3=46 % .
           3%
         \else
           9%
         \fi\fi\fi\fi\ThPdNu@space
    \expandafter\ThPdNu@zero
    \expandafter0%
  \or
   \ThPdNu@State@SkipZeros#1!%
  \or
    \ThPdNu@State@Int#1!#3!%
  \or
    \ThPdNu@State@Dot#1!\ThPdNu@zero*\ThPdNu@zero!!%
  \else
    \ThPdNu@ReturnError{0}%
  \fi
}
%    \end{macrocode}
%    \end{macro}
%
%    \begin{macro}{\ThPdNu@State@Int}
%    \begin{macrocode}
\def\ThPdNu@State@Int#1!#2!#3\fi#4{%
  \fi
  \ifcase\ifx\ThPdNu@FIN#4%
           0%
         \else\ifnum`#4>47 %
           \ifnum`#4<58 %
             1%
           \else
             9%
           \fi
&        \else\ifx.#4%
$        \else\ifnum`#4=46 % .
           2%
         \else
           9%
         \fi\fi\fi\ThPdNu@space
    \ThPdNu@ReturnInt{#1#2}%
  \or
    \ThPdNu@State@Int#1!#2#4!%
  \or
    \ThPdNu@State@Dot#1!\ThPdNu@one#2*\ThPdNu@zero!!%
  \else
    \ThPdNu@ReturnError{#1#2}%
  \fi
}
%    \end{macrocode}
%    \end{macro}
%
%    \begin{macro}{\ThPdNu@ReturnInt}
%    \begin{macrocode}
\def\ThPdNu@ReturnInt#1#2\fi{%
  \fi
  \ThPdNu@zero
  #1%
}
%    \end{macrocode}
%    \end{macro}
%
% \subsubsection{State definitions for decimal digits}
%
%    \begin{macro}{\ThPdNu@State@Dot}
%    \begin{macrocode}
\def\ThPdNu@State@Dot#1*#2#3!#4!#5\fi#6{%
  \fi
  \ifcase\ifx\ThPdNu@FIN#6%
           0%
         \else\ifnum`#6>48 %
           \ifnum`#6<58 %
             1%
           \else
             9%
           \fi
         \else\ifx0#6%
           2%
         \else
           9%
         \fi\fi\fi\ThPdNu@space
    \ThPdNu@ReturnNumber#1*#2#3!%
  \or
    \ThPdNu@State@Dot#1*\ThPdNu@one#3#4#6!!%
  \or
    \ThPdNu@State@DotZero#1*#2#3!#4#6!%
  \else
    \ThPdNu@ReturnNumberInvalid#1*#2#3!%
  \fi
}
%    \end{macrocode}
%    \end{macro}
%
%    \begin{macro}{\ThPdNu@State@DotZero}
%    \begin{macrocode}
\def\ThPdNu@State@DotZero#1*#2#3!#4!#5\fi#6{%
  \fi
  \ifcase\ifx\ThPdNu@FIN#6%
           0%
         \else\ifnum`#6>48 %
           \ifnum`#6<58 %
             1%
           \else
             9%
           \fi
         \else\ifx0#6%
           2%
         \else
           9%
         \fi\fi\fi\ThPdNu@space
    \ThPdNu@ReturnNumber#1*#2#3!%
  \or
    \ThPdNu@State@Dot#1*\ThPdNu@one#3#4#6!!%
  \or
    \ThPdNu@State@DotZero#1*#2#3!#4#6!%
  \else
    \ThPdNu@ReturnNumber#1*#2#3!%
  \fi
}
%    \end{macrocode}
%    \end{macro}
%
%    \begin{macro}{\ThPdNu@ReturnNumber}
%    \begin{macrocode}
\def\ThPdNu@ReturnNumber#1!#2#3*#4#5!#6\fi{%
  \fi
  \ifcase#2%
    \expandafter\ThPdNu@firstoftwo
  \else
    \expandafter\ThPdNu@secondoftwo
  \fi
  {%
    \ifcase#4%
      \expandafter\ThPdNu@firstoftwo
    \else
      \expandafter\ThPdNu@secondoftwo
    \fi
    {\ThPdNu@zero 0}%
    {\ThPdNu@zero #1.#5}%
  }{%
    \ifcase#4%
      \expandafter\ThPdNu@firstoftwo
    \else
      \expandafter\ThPdNu@secondoftwo
    \fi
    {\ThPdNu@zero #1#3}%
    {\ThPdNu@zero #1#3.#5}%
  }%
}
%    \end{macrocode}
%    \end{macro}
%
%    \begin{macro}{\ThPdNu@ReturnNumberInvalid}
%    \begin{macrocode}
\def\ThPdNu@ReturnNumberInvalid#1*#2!#3\fi#4\ThPdNu@FIN{%
  \fi
  \iftrue\ThPdNu@ReturnNumber#1*#2!\fi
  \thepdfnumber@ErrorInvalidToken
}
%    \end{macrocode}
%    \end{macro}
%
% \subsection{Norm macro}
%
%    \begin{macro}{\thepdfnumberNormZeroOne}
%    \begin{macrocode}
\def\thepdfnumberNormZeroOne#1{%
  \romannumeral
& \iftrue\expandafter\ThPbNu@StateN@Plus\expandafter\fi
& \detokenize\expandafter{#1}%
& \ThPdNu@FIN
$ \ifx\ThPdNu@FIN#1\ThPdNu@FIN
$   \expandafter\ThPdNu@firstoftwo
$ \else
$   \expandafter\ThPdNu@secondoftwo
$ \fi
$ {%
$   \ThPdNu@zero
$   0\thepdfnumber@ErrorUnexpectedEnd
$ }{%
$   \iftrue\expandafter\ThPbNu@StateN@Plus\expandafter\fi#1\ThPdNu@FIN
$ }%
}
%    \end{macrocode}
%    \end{macro}
%
% \subsubsection{State definitions for sign}
%
%    \begin{macro}{\ThPbNu@StateN@Plus}
%    \begin{macrocode}
\def\ThPbNu@StateN@Plus#1\fi#2{%
  \fi
  \ifcase\ifx\ThPdNu@FIN#2%
           0%
&        \else\ifx-#2%
$        \else\ifnum`#2=45 % -
           1%
         \else\ifx0#2%
           2%
         \else\ifnum`#2>48 %
           \ifnum`#2<58 %
             3%
           \else
             9%
           \fi
&        \else\ifx.#2%
$        \else\ifnum`#2=46 % .
           4%
&        \else\ifx+#2%
$        \else\ifnum`#2=43 % +
           5%
         \else
           9%
         \fi\fi\fi\fi\fi\fi\ThPdNu@space
    \expandafter\ThPdNu@zero
    \expandafter0%
    \expandafter\thepdfnumber@ErrorUnexpectedEnd
  \or
    \ThPbNu@StateN@Minus
  \or
    \ThPbNu@StateN@SkipZeros
  \or
    \ThPdNu@ReturnAndSkip{1}%
  \or
    \ThPbNu@StateN@Dot\ThPdNu@zero!!%
  \or
    \ThPbNu@StateN@Plus
  \else
    \ThPdNu@ReturnError{0}%
  \fi
}
%    \end{macrocode}
%    \end{macro}
%
%    \begin{macro}{\ThPbNu@StateN@Minus}
%    \begin{macrocode}
\def\ThPbNu@StateN@Minus#1\fi#2{%
  \fi
  \ifcase\ifx\ThPdNu@FIN#2%
           0%
         \else\ifnum`#2>47 %
           \ifnum`#2<58 %
             1%
           \else
             9%
           \fi
&        \else\ifx.#2%
$        \else\ifnum`#2=46 % .
           1%
&        \else\ifx-#2%
$        \else\ifnum`#2=45 % -
           2%
&        \else\ifx+#2%
$        \else\ifnum`#2=43 % +
           3%
         \else
           9%
         \fi\fi\fi\fi\fi\ThPdNu@space
    \expandafter\ThPdNu@zero
    \expandafter0%
    \expandafter\thepdfnumber@ErrorUnexpectedEnd
  \or
    \ThPdNu@ReturnAndSkip{0}%
  \or
    \ThPbNu@StateN@Plus
  \or
    \ThPbNu@StateN@Minus
  \else
    \ThPdNu@ReturnError{0}%
  \fi
}
%    \end{macrocode}
%    \end{macro}
%
%    \begin{macro}{\ThPdNu@ReturnAndSkip}
%    \begin{macrocode}
\def\ThPdNu@ReturnAndSkip#1#2\fi#3\ThPdNu@FIN{%
  \fi
  \ThPdNu@zero
  #1%
}
%    \end{macrocode}
%    \end{macro}
%
% \subsubsection{State definitions for integer part}
%
%    \begin{macro}{\ThPbNu@StateN@SkipZeros}
%    \begin{macrocode}
\def\ThPbNu@StateN@SkipZeros#1\fi#2{%
  \fi
  \ifcase\ifx\ThPdNu@FIN#2%
           0%
         \else\ifx0#2%
           1%
         \else\ifnum`#2>48 %
           \ifnum`#2<58 %
             2%
           \else
             9%
           \fi
&        \else\ifx.#2%
$        \else\ifnum`#2=46 % .
           3%
         \else
           9%
         \fi\fi\fi\fi\ThPdNu@space
    \expandafter\ThPdNu@zero
    \expandafter0%
  \or
   \ThPbNu@StateN@SkipZeros%
  \or
    \ThPdNu@ReturnAndSkip{1}%
  \or
    \ThPbNu@StateN@Dot\ThPdNu@zero!!%
  \else
    \ThPdNu@ReturnError{0}%
  \fi
}
%    \end{macrocode}
%    \end{macro}
%
% \subsubsection{State definitions for decimal digits}
%
%    \begin{macro}{\ThPbNu@StateN@Dot}
%    \begin{macrocode}
\def\ThPbNu@StateN@Dot#1#2!#3!#4\fi#5{%
  \fi
  \ifcase\ifx\ThPdNu@FIN#5%
           0%
         \else\ifnum`#5>48 %
           \ifnum`#5<58 %
             1%
           \else
             9%
           \fi
         \else\ifx0#5%
           2%
         \else
           9%
         \fi\fi\fi\ThPdNu@space
    \ThPdNu@ReturnFracNumber#1#2!%
  \or
    \ThPbNu@StateN@Dot\ThPdNu@one#2#3#5!!%
  \or
    \ThPbNu@StateN@DotZero#1#2!#3#5!%
  \else
    \ThPdNu@ReturnFracNumberInvalid#1#2!%
  \fi
}
%    \end{macrocode}
%    \end{macro}
%
%    \begin{macro}{\ThPbNu@StateN@DotZero}
%    \begin{macrocode}
\def\ThPbNu@StateN@DotZero#1#2!#3!#4\fi#5{%
  \fi
  \ifcase\ifx\ThPdNu@FIN#5%
           0%
         \else\ifnum`#5>48 %
           \ifnum`#5<58 %
             1%
           \else
             9%
           \fi
         \else\ifx0#5%
           2%
         \else
           9%
         \fi\fi\fi\ThPdNu@space
    \ThPdNu@ReturnFracNumber#1#2!%
  \or
    \ThPbNu@StateN@Dot\ThPdNu@one#2#3#5!!%
  \or
    \ThPbNu@StateN@DotZero#1#2!#3#5!%
  \else
    \ThPdNu@ReturnFracNumberInvalid#1#2!%
  \fi
}
%    \end{macrocode}
%    \end{macro}
%
%    \begin{macro}{\ThPdNu@ReturnFracNumber}
%    \begin{macrocode}
\def\ThPdNu@ReturnFracNumber#1#2!#3\fi{%
  \fi
  \ifcase#1%
    \expandafter\ThPdNu@firstoftwo
  \else
    \expandafter\ThPdNu@secondoftwo
  \fi
  {\ThPdNu@zero 0}%
  {\ThPdNu@zero .#2}%
}
%    \end{macrocode}
%    \end{macro}
%
%    \begin{macro}{\ThPdNu@ReturnFracNumberInvalid}
%    \begin{macrocode}
\def\ThPdNu@ReturnFracNumberInvalid#1!#2\fi#3\ThPdNu@FIN{%
  \fi
  \iftrue\ThPdNu@ReturnFracNumber#1!\fi
  \thepdfnumber@ErrorInvalidToken
}
%    \end{macrocode}
%    \end{macro}
%
%    \begin{macrocode}
\ThPdNu@AtEnd%
%</package>
%    \end{macrocode}
%
% \section{Test}
%
% \subsection{Catcode checks for loading}
%
%    \begin{macrocode}
%<*test1>
%    \end{macrocode}
%    \begin{macrocode}
\catcode`\{=1 %
\catcode`\}=2 %
\catcode`\#=6 %
\catcode`\@=11 %
\expandafter\ifx\csname count@\endcsname\relax
  \countdef\count@=255 %
\fi
\expandafter\ifx\csname @gobble\endcsname\relax
  \long\def\@gobble#1{}%
\fi
\expandafter\ifx\csname @firstofone\endcsname\relax
  \long\def\@firstofone#1{#1}%
\fi
\expandafter\ifx\csname loop\endcsname\relax
  \expandafter\@firstofone
\else
  \expandafter\@gobble
\fi
{%
  \def\loop#1\repeat{%
    \def\body{#1}%
    \iterate
  }%
  \def\iterate{%
    \body
      \let\next\iterate
    \else
      \let\next\relax
    \fi
    \next
  }%
  \let\repeat=\fi
}%
\def\RestoreCatcodes{}
\count@=0 %
\loop
  \edef\RestoreCatcodes{%
    \RestoreCatcodes
    \catcode\the\count@=\the\catcode\count@\relax
  }%
\ifnum\count@<255 %
  \advance\count@ 1 %
\repeat

\def\RangeCatcodeInvalid#1#2{%
  \count@=#1\relax
  \loop
    \catcode\count@=15 %
  \ifnum\count@<#2\relax
    \advance\count@ 1 %
  \repeat
}
\def\RangeCatcodeCheck#1#2#3{%
  \count@=#1\relax
  \loop
    \ifnum#3=\catcode\count@
    \else
      \errmessage{%
        Character \the\count@\space
        with wrong catcode \the\catcode\count@\space
        instead of \number#3%
      }%
    \fi
  \ifnum\count@<#2\relax
    \advance\count@ 1 %
  \repeat
}
\def\space{ }
\expandafter\ifx\csname LoadCommand\endcsname\relax
  \def\LoadCommand{\input thepdfnumber.sty\relax}%
\fi
\def\Test{%
  \RangeCatcodeInvalid{0}{47}%
  \RangeCatcodeInvalid{58}{64}%
  \RangeCatcodeInvalid{91}{96}%
  \RangeCatcodeInvalid{123}{255}%
  \catcode`\@=12 %
  \catcode`\\=0 %
  \catcode`\%=14 %
  \LoadCommand
  \RangeCatcodeCheck{0}{36}{15}%
  \RangeCatcodeCheck{37}{37}{14}%
  \RangeCatcodeCheck{38}{47}{15}%
  \RangeCatcodeCheck{48}{57}{12}%
  \RangeCatcodeCheck{58}{63}{15}%
  \RangeCatcodeCheck{64}{64}{12}%
  \RangeCatcodeCheck{65}{90}{11}%
  \RangeCatcodeCheck{91}{91}{15}%
  \RangeCatcodeCheck{92}{92}{0}%
  \RangeCatcodeCheck{93}{96}{15}%
  \RangeCatcodeCheck{97}{122}{11}%
  \RangeCatcodeCheck{123}{255}{15}%
  \RestoreCatcodes
}
\Test
\csname @@end\endcsname
\end
%    \end{macrocode}
%    \begin{macrocode}
%</test1>
%    \end{macrocode}
%
% \subsection{Macro tests}
%
% \subsubsection{Preamble}
%
%    \begin{macrocode}
%<*test2>
\catcode`\{=1
\catcode`\}=2
\catcode`\#=6
\catcode`\@=11
\errorcontextlines=10000 %
\def\msg#{\immediate\write16}
\def\space{ }
\begingroup\expandafter\expandafter\expandafter\endgroup
\expandafter\ifx\csname RequirePackage\endcsname\relax
  \input thepdfnumber.sty\relax
\else
  \RequirePackage{thepdfnumber}[2016/05/16]%
\fi
%    \end{macrocode}
%
%    \begin{macro}{\Test}
%    \begin{macrocode}
\def\Test#1#2{%
  \def\TestExpected{#2}%
%<*active>
  \edef\TestExpected{\TestExpected}%
%</active>
  \expandafter\expandafter\expandafter\def
  \expandafter\expandafter\expandafter\TestResult
  \expandafter\expandafter\expandafter{%
    \TestCommand{#1}%
  }%
  \begingroup
    \toks0\expandafter{\TestResult}%
    \toks2\expandafter{\TestExpected}%
    \ifx\TestResult\TestExpected
      \msg{* OK: #1 => \the\toks0}%
    \else
      \errmessage{FAILED: #1 => \the\toks0\space(\the\toks2)}%
    \fi
  \endgroup
}
%    \end{macrocode}
%    \end{macro}
%
%    \begin{macrocode}
\let\thepdfnumber@ErrorUnexpectedEnd\relax
\let\thepdfnumber@ErrorInvalidToken\relax
%    \end{macrocode}
%
% \subsubsection{Test with active catcodes}
%
%    \begin{macrocode}
%<*active>
\catcode`\~=13 %
\def\ThPdNu@Temp#1{%
  \begingroup
    \lccode`\~=`#1 %
  \lowercase{\endgroup
    \edef~%
  }{\string#1}%
  \catcode`#1=13 %
}
\ThPdNu@Temp{.}
\ThPdNu@Temp{-}
\ThPdNu@Temp{+}
%</active>
%    \end{macrocode}
%
% \subsubsection{Test cases for \cs{thepdfnumber}}
%
%    \begin{macro}{\TestCommand}
%    \begin{macrocode}
\let\TestCommand\thepdfnumber
%    \end{macrocode}
%    \end{macro}
%
%    \begin{macrocode}
\Test{0}{0}
\Test{1}{1}
\Test{23}{23}
\Test{0045}{45}
\Test{4500}{4500}
\Test{-0}{0}
\Test{-1}{-1}
\Test{-00002}{-2}
\Test{-010203}{-10203}
\Test{+0}{0}
\Test{+123}{123}

\Test{+-4}{-4}
\Test{+-+-4}{4}
\Test{++--++--++4}{4}
\Test{+++---+++---+++---4}{-4}

\Test{.}{0}
\Test{0.}{0}
\Test{.0}{0}
\Test{0.}{0}
\Test{-.}{0}
\Test{-0.}{0}
\Test{-.0}{0}
\Test{-0.}{0}
\Test{010.020}{10.02}
\Test{123.456}{123.456}
\Test{12.}{12}
\Test{.123}{.123}
\Test{.0001}{.0001}
\Test{.00100}{.001}
\Test{.12003400560078009}{.12003400560078009}
\Test{-.12003400560078009}{-.12003400560078009}
\Test{04.0500000}{4.05}

\Test{}{0\thepdfnumber@ErrorUnexpectedEnd}
\Test{+}{0\thepdfnumber@ErrorUnexpectedEnd}
\Test{-}{0\thepdfnumber@ErrorUnexpectedEnd}
\Test{a}{0\thepdfnumber@ErrorInvalidToken}
\Test{0x1}{0\thepdfnumber@ErrorInvalidToken}
\Test{4x56}{4\thepdfnumber@ErrorInvalidToken}
\Test{012,34}{12\thepdfnumber@ErrorInvalidToken}
\Test{0.12x4}{.12\thepdfnumber@ErrorInvalidToken}
%    \end{macrocode}
%
% \subsubsection{Test cases for \cs{thepdfnumberNormZeroOne}}
%
%    \begin{macro}{\TestCommand}
%    \begin{macrocode}
\let\TestCommand\thepdfnumberNormZeroOne
%    \end{macrocode}
%    \end{macro}
%
%    \begin{macrocode}
\Test{0}{0}
\Test{1}{1}
\Test{23}{1}
\Test{12}{1}
\Test{10}{1}
\Test{0045}{1}
\Test{001}{1}
\Test{-0}{0}
\Test{-1}{0}
\Test{-01}{0}
\Test{-4}{0}

\Test{+-0}{0}
\Test{+-+-1}{1}
\Test{++--++--++1}{1}
\Test{+++---+++---+++---1}{0}

\Test{.}{0}
\Test{0.}{0}
\Test{.0}{0}
\Test{0.}{0}
\Test{-.}{0}
\Test{-0.}{0}
\Test{-.0}{0}
\Test{-0.}{0}
\Test{010.020}{1}
\Test{123.456}{1}
\Test{12.}{1}
\Test{.123}{.123}
\Test{.0001}{.0001}
\Test{.00100}{.001}
\Test{.12003400560078009}{.12003400560078009}
\Test{-.12003400560078009}{0}
\Test{04.0500000}{1}
\Test{0.1200340056}{.1200340056}
\Test{1.05}{1}

\Test{}{0\thepdfnumber@ErrorUnexpectedEnd}
\Test{+}{0\thepdfnumber@ErrorUnexpectedEnd}
\Test{-}{0\thepdfnumber@ErrorUnexpectedEnd}
\Test{a}{0\thepdfnumber@ErrorInvalidToken}
\Test{0x1}{0\thepdfnumber@ErrorInvalidToken}
\Test{4x56}{1}
\Test{012,34}{1}
\Test{.012x4}{.012\thepdfnumber@ErrorInvalidToken}
%    \end{macrocode}
%
%    \begin{macrocode}
\csname @@end\endcsname\end
%</test2>
%    \end{macrocode}
%
% \section{Installation}
%
% \subsection{Download}
%
% \paragraph{Package.} This package is available on
% CTAN\footnote{\url{http://ctan.org/pkg/thepdfnumber}}:
% \begin{description}
% \item[\CTAN{macros/latex/contrib/oberdiek/thepdfnumber.dtx}] The source file.
% \item[\CTAN{macros/latex/contrib/oberdiek/thepdfnumber.pdf}] Documentation.
% \end{description}
%
%
% \paragraph{Bundle.} All the packages of the bundle `oberdiek'
% are also available in a TDS compliant ZIP archive. There
% the packages are already unpacked and the documentation files
% are generated. The files and directories obey the TDS standard.
% \begin{description}
% \item[\CTAN{install/macros/latex/contrib/oberdiek.tds.zip}]
% \end{description}
% \emph{TDS} refers to the standard ``A Directory Structure
% for \TeX\ Files'' (\CTAN{tds/tds.pdf}). Directories
% with \xfile{texmf} in their name are usually organized this way.
%
% \subsection{Bundle installation}
%
% \paragraph{Unpacking.} Unpack the \xfile{oberdiek.tds.zip} in the
% TDS tree (also known as \xfile{texmf} tree) of your choice.
% Example (linux):
% \begin{quote}
%   |unzip oberdiek.tds.zip -d ~/texmf|
% \end{quote}
%
% \paragraph{Script installation.}
% Check the directory \xfile{TDS:scripts/oberdiek/} for
% scripts that need further installation steps.
% Package \xpackage{attachfile2} comes with the Perl script
% \xfile{pdfatfi.pl} that should be installed in such a way
% that it can be called as \texttt{pdfatfi}.
% Example (linux):
% \begin{quote}
%   |chmod +x scripts/oberdiek/pdfatfi.pl|\\
%   |cp scripts/oberdiek/pdfatfi.pl /usr/local/bin/|
% \end{quote}
%
% \subsection{Package installation}
%
% \paragraph{Unpacking.} The \xfile{.dtx} file is a self-extracting
% \docstrip\ archive. The files are extracted by running the
% \xfile{.dtx} through \plainTeX:
% \begin{quote}
%   \verb|tex thepdfnumber.dtx|
% \end{quote}
%
% \paragraph{TDS.} Now the different files must be moved into
% the different directories in your installation TDS tree
% (also known as \xfile{texmf} tree):
% \begin{quote}
% \def\t{^^A
% \begin{tabular}{@{}>{\ttfamily}l@{ $\rightarrow$ }>{\ttfamily}l@{}}
%   thepdfnumber.sty & tex/generic/oberdiek/thepdfnumber.sty\\
%   thepdfnumber.pdf & doc/latex/oberdiek/thepdfnumber.pdf\\
%   test/thepdfnumber-test1.tex & doc/latex/oberdiek/test/thepdfnumber-test1.tex\\
%   test/thepdfnumber-test2.tex & doc/latex/oberdiek/test/thepdfnumber-test2.tex\\
%   test/thepdfnumber-test3.tex & doc/latex/oberdiek/test/thepdfnumber-test3.tex\\
%   thepdfnumber.dtx & source/latex/oberdiek/thepdfnumber.dtx\\
% \end{tabular}^^A
% }^^A
% \sbox0{\t}^^A
% \ifdim\wd0>\linewidth
%   \begingroup
%     \advance\linewidth by\leftmargin
%     \advance\linewidth by\rightmargin
%   \edef\x{\endgroup
%     \def\noexpand\lw{\the\linewidth}^^A
%   }\x
%   \def\lwbox{^^A
%     \leavevmode
%     \hbox to \linewidth{^^A
%       \kern-\leftmargin\relax
%       \hss
%       \usebox0
%       \hss
%       \kern-\rightmargin\relax
%     }^^A
%   }^^A
%   \ifdim\wd0>\lw
%     \sbox0{\small\t}^^A
%     \ifdim\wd0>\linewidth
%       \ifdim\wd0>\lw
%         \sbox0{\footnotesize\t}^^A
%         \ifdim\wd0>\linewidth
%           \ifdim\wd0>\lw
%             \sbox0{\scriptsize\t}^^A
%             \ifdim\wd0>\linewidth
%               \ifdim\wd0>\lw
%                 \sbox0{\tiny\t}^^A
%                 \ifdim\wd0>\linewidth
%                   \lwbox
%                 \else
%                   \usebox0
%                 \fi
%               \else
%                 \lwbox
%               \fi
%             \else
%               \usebox0
%             \fi
%           \else
%             \lwbox
%           \fi
%         \else
%           \usebox0
%         \fi
%       \else
%         \lwbox
%       \fi
%     \else
%       \usebox0
%     \fi
%   \else
%     \lwbox
%   \fi
% \else
%   \usebox0
% \fi
% \end{quote}
% If you have a \xfile{docstrip.cfg} that configures and enables \docstrip's
% TDS installing feature, then some files can already be in the right
% place, see the documentation of \docstrip.
%
% \subsection{Refresh file name databases}
%
% If your \TeX~distribution
% (\teTeX, \mikTeX, \dots) relies on file name databases, you must refresh
% these. For example, \teTeX\ users run \verb|texhash| or
% \verb|mktexlsr|.
%
% \subsection{Some details for the interested}
%
% \paragraph{Attached source.}
%
% The PDF documentation on CTAN also includes the
% \xfile{.dtx} source file. It can be extracted by
% AcrobatReader 6 or higher. Another option is \textsf{pdftk},
% e.g. unpack the file into the current directory:
% \begin{quote}
%   \verb|pdftk thepdfnumber.pdf unpack_files output .|
% \end{quote}
%
% \paragraph{Unpacking with \LaTeX.}
% The \xfile{.dtx} chooses its action depending on the format:
% \begin{description}
% \item[\plainTeX:] Run \docstrip\ and extract the files.
% \item[\LaTeX:] Generate the documentation.
% \end{description}
% If you insist on using \LaTeX\ for \docstrip\ (really,
% \docstrip\ does not need \LaTeX), then inform the autodetect routine
% about your intention:
% \begin{quote}
%   \verb|latex \let\install=y% \iffalse meta-comment
%
% File: thepdfnumber.dtx
% Version: 2016/05/16 v1.1
% Info: Print PDF numbers with minimal digits
%
% Copyright (C) 2011 by
%    Heiko Oberdiek <heiko.oberdiek at googlemail.com>
%    2016
%    https://github.com/ho-tex/oberdiek/issues
%
% This work may be distributed and/or modified under the
% conditions of the LaTeX Project Public License, either
% version 1.3c of this license or (at your option) any later
% version. This version of this license is in
%    http://www.latex-project.org/lppl/lppl-1-3c.txt
% and the latest version of this license is in
%    http://www.latex-project.org/lppl.txt
% and version 1.3 or later is part of all distributions of
% LaTeX version 2005/12/01 or later.
%
% This work has the LPPL maintenance status "maintained".
%
% This Current Maintainer of this work is Heiko Oberdiek.
%
% The Base Interpreter refers to any `TeX-Format',
% because some files are installed in TDS:tex/generic//.
%
% This work consists of the main source file thepdfnumber.dtx
% and the derived files
%    thepdfnumber.sty, thepdfnumber.pdf, thepdfnumber.ins, thepdfnumber.drv,
%    thepdfnumber-test1.tex, thepdfnumber-test2.tex,
%    thepdfnumber-test3.tex.
%
% Distribution:
%    CTAN:macros/latex/contrib/oberdiek/thepdfnumber.dtx
%    CTAN:macros/latex/contrib/oberdiek/thepdfnumber.pdf
%
% Unpacking:
%    (a) If thepdfnumber.ins is present:
%           tex thepdfnumber.ins
%    (b) Without thepdfnumber.ins:
%           tex thepdfnumber.dtx
%    (c) If you insist on using LaTeX
%           latex \let\install=y\input{thepdfnumber.dtx}
%        (quote the arguments according to the demands of your shell)
%
% Documentation:
%    (a) If thepdfnumber.drv is present:
%           latex thepdfnumber.drv
%    (b) Without thepdfnumber.drv:
%           latex thepdfnumber.dtx; ...
%    The class ltxdoc loads the configuration file ltxdoc.cfg
%    if available. Here you can specify further options, e.g.
%    use A4 as paper format:
%       \PassOptionsToClass{a4paper}{article}
%
%    Programm calls to get the documentation (example):
%       pdflatex thepdfnumber.dtx
%       makeindex -s gind.ist thepdfnumber.idx
%       pdflatex thepdfnumber.dtx
%       makeindex -s gind.ist thepdfnumber.idx
%       pdflatex thepdfnumber.dtx
%
% Installation:
%    TDS:tex/generic/oberdiek/thepdfnumber.sty
%    TDS:doc/latex/oberdiek/thepdfnumber.pdf
%    TDS:doc/latex/oberdiek/test/thepdfnumber-test1.tex
%    TDS:doc/latex/oberdiek/test/thepdfnumber-test2.tex
%    TDS:doc/latex/oberdiek/test/thepdfnumber-test3.tex
%    TDS:source/latex/oberdiek/thepdfnumber.dtx
%
%<*ignore>
\begingroup
  \catcode123=1 %
  \catcode125=2 %
  \def\x{LaTeX2e}%
\expandafter\endgroup
\ifcase 0\ifx\install y1\fi\expandafter
         \ifx\csname processbatchFile\endcsname\relax\else1\fi
         \ifx\fmtname\x\else 1\fi\relax
\else\csname fi\endcsname
%</ignore>
%<*install>
\input docstrip.tex
\Msg{************************************************************************}
\Msg{* Installation}
\Msg{* Package: thepdfnumber 2016/05/16 v1.1 Print PDF numbers with minimal digits (HO)}
\Msg{************************************************************************}

\keepsilent
\askforoverwritefalse

\let\MetaPrefix\relax
\preamble

This is a generated file.

Project: thepdfnumber
Version: 2016/05/16 v1.1

Copyright (C) 2011 by
   Heiko Oberdiek <heiko.oberdiek at googlemail.com>

This work may be distributed and/or modified under the
conditions of the LaTeX Project Public License, either
version 1.3c of this license or (at your option) any later
version. This version of this license is in
   http://www.latex-project.org/lppl/lppl-1-3c.txt
and the latest version of this license is in
   http://www.latex-project.org/lppl.txt
and version 1.3 or later is part of all distributions of
LaTeX version 2005/12/01 or later.

This work has the LPPL maintenance status "maintained".

This Current Maintainer of this work is Heiko Oberdiek.

The Base Interpreter refers to any `TeX-Format',
because some files are installed in TDS:tex/generic//.

This work consists of the main source file thepdfnumber.dtx
and the derived files
   thepdfnumber.sty, thepdfnumber.pdf, thepdfnumber.ins, thepdfnumber.drv,
   thepdfnumber-test1.tex, thepdfnumber-test2.tex,
   thepdfnumber-test3.tex.

\endpreamble
\let\MetaPrefix\DoubleperCent

\generate{%
  \file{thepdfnumber.ins}{\from{thepdfnumber.dtx}{install}}%
  \file{thepdfnumber.drv}{\from{thepdfnumber.dtx}{driver}}%
  \usedir{tex/generic/oberdiek}%
  \file{thepdfnumber.sty}{\from{thepdfnumber.dtx}{package}}%
  \usedir{doc/latex/oberdiek/test}%
  \file{thepdfnumber-test1.tex}{\from{thepdfnumber.dtx}{test1}}%
  \file{thepdfnumber-test2.tex}{\from{thepdfnumber.dtx}{test2}}%
  \file{thepdfnumber-test3.tex}{\from{thepdfnumber.dtx}{test2,active}}%
}

\catcode32=13\relax% active space
\let =\space%
\Msg{************************************************************************}
\Msg{*}
\Msg{* To finish the installation you have to move the following}
\Msg{* file into a directory searched by TeX:}
\Msg{*}
\Msg{*     thepdfnumber.sty}
\Msg{*}
\Msg{* To produce the documentation run the file `thepdfnumber.drv'}
\Msg{* through LaTeX.}
\Msg{*}
\Msg{* Happy TeXing!}
\Msg{*}
\Msg{************************************************************************}

\endbatchfile
%</install>
%<*ignore>
\fi
%</ignore>
%<*driver>
\NeedsTeXFormat{LaTeX2e}
\ProvidesFile{thepdfnumber.drv}%
  [2016/05/16 v1.1 Print PDF numbers with minimal digits (HO)]%
\documentclass{ltxdoc}
\usepackage{holtxdoc}[2011/11/22]
\usepackage{csquotes}
\usepackage[
  backend=bibtex,
  bibencoding=ascii,
  alldates=iso8601,
]{biblatex}[2011/11/13]
\bibliography{oberdiek-source}
\begin{document}
  \DocInput{thepdfnumber.dtx}%
\end{document}
%</driver>
% \fi
%
%
% \CharacterTable
%  {Upper-case    \A\B\C\D\E\F\G\H\I\J\K\L\M\N\O\P\Q\R\S\T\U\V\W\X\Y\Z
%   Lower-case    \a\b\c\d\e\f\g\h\i\j\k\l\m\n\o\p\q\r\s\t\u\v\w\x\y\z
%   Digits        \0\1\2\3\4\5\6\7\8\9
%   Exclamation   \!     Double quote  \"     Hash (number) \#
%   Dollar        \$     Percent       \%     Ampersand     \&
%   Acute accent  \'     Left paren    \(     Right paren   \)
%   Asterisk      \*     Plus          \+     Comma         \,
%   Minus         \-     Point         \.     Solidus       \/
%   Colon         \:     Semicolon     \;     Less than     \<
%   Equals        \=     Greater than  \>     Question mark \?
%   Commercial at \@     Left bracket  \[     Backslash     \\
%   Right bracket \]     Circumflex    \^     Underscore    \_
%   Grave accent  \`     Left brace    \{     Vertical bar  \|
%   Right brace   \}     Tilde         \~}
%
% \GetFileInfo{thepdfnumber.drv}
%
% \title{The \xpackage{thepdfnumber} package}
% \date{2016/05/16 v1.1}
% \author{Heiko Oberdiek\thanks
% {Please report any issues at https://github.com/ho-tex/oberdiek/issues}\\
% \xemail{heiko.oberdiek at googlemail.com}}
%
% \maketitle
%
% \begin{abstract}
% The package converts real numbers to a minimal representation
% that is stripped from leading or trailing zeros,
% plus signs and decimal point if not necessary.
% \end{abstract}
%
% \tableofcontents
%
% \section{Documentation}
%
% \subsection{Introduction}
%
% Dealing with the PDF format, there is sometimes the need to write
% some low level PDF stuff. In case of numbers, the numbers can
% arise from user input (e.g. color or transparency specifications)
% or can be calculated. For example, \hologo{LaTeX}'s \cs{strip@pt}
% makes a good job to output a real number. It automatically suppresses
% the decimal part if the number is an integer. However it leaves
% a leading zero for numbers greater zero and smaller one.
% Thus the package provides macros that can be used with different
% formats, even with \hologo{iniTeX} and generates numbers that
% are valid numbers of the PDF format and whose length is minimal.
%
% \subsection{Usage}
%
% The package \xpackage{thepdfnumber} can be used with \hologo{LaTeX},
% \hologo{plainTeX} or even with \hologo{iniTeX}:
% \begin{quote}
%   |\RequirePackage{thepdfnumber}| \textit{\% \hologo{LaTeX}}\\
%   |\input thepdfnumber.sty| \textit{\% \hologo{plainTeX}/\hologo{iniTeX}}
% \end{quote}
% The package does not need and have package options.
%
% \subsection{User macros}
%
% All user macros are expandable in exact two expansion steps.
%
% \begin{declcs}{thepdfnumber} \M{number}
% \end{declcs}
% Macro \cs{thepdfnumber} takes a number
% as argument and expands to a minimal representation of that number.
% Some examples:
% \begin{quote}
% \begin{tabular}{r@{.: }>{\ttfamily}l@{ $\rightarrow$\ }>{\ttfamily}l@{}}
%   1& +123 & 123\\
%   2& --123 & 123\\
%   3& -01 & -1\\
%   4& 0045 & 45\\
%   5& 1.0 & 1\\
%   6& 1.20 & 1.2\\
%   7& 0.0 & 0\\
%   8& 0.78 & .78\\
%   9& +012.340 & 12.34\\
% \end{tabular}
% \end{quote}
% It reduces the length of the number representation:
% \begin{itemize}
% \item The signs are collapsed and only one minus sign
%   is output if the number is negative (see examples 1, 2, 3, 9).
% \item Leading zeros are removed (4, 8, 9) unless the number is zero (7).
% \item The decimal part is omitted, if the number is an integer (5, 7).
% \item Trailing zeros from the decimal part are stripped (5, 6, 7, 9).
% \end{itemize}
% The resulting number representation can be catched with one of the
% following Perl regular expressions:
% \begin{itemize}
% \item |^0$| (zero)
% \item |^-?[1-9][0-9]*$| (integer)
% \item |^-?[0-9]*\.[0-9]*[1-9]$| (real)
% \end{itemize}
% This is a valid numeric object of the PDF specification \cite[``7.3.3~Numeric Objects'']{pdfspec-iso32000-1}.
%
% \begin{declcs}{thepdfnumberNormZeroOne}
% \end{declcs}
% There are various places in the PDF specification where the
% number is in the domain 0.0 upto 1.0. Macro thepdfnumberNormZeroOne
% automatically adjusts the number to fit into that range. Negative
% numbers are mapped to |0| and numbers greater than one are
% replaced by |1|. Thus the result fits one of the following regular
% expressions:
% \begin{itemize}
% \item |^0$|
% \item |^\.[0-9]*[1-9]$|
% \item |^1$|
% \end{itemize}
% Examples:
% \begin{quote}
% \begin{tabular}{@{}>{\ttfamily}l@{ $\rightarrow$\ }>{\ttfamily}l@{}}
%   -456 & 0\\
%   -0.001 & 0\\
%   0.0 & 0\\
%   0.010 & .01\\
%   0.456 & .456\\
%   1.0 & 1\\
%   01.001 & 1\\
%   4 & 1\\
% \end{tabular}
% \end{quote}
%
% \subsection{Input number}
%
% The user macros expect a number as argument. The number can either
% be given explicitely or as macro that expands in one step
% to an explicite number, because the first token of the argument
% is expanded once.
%
% The explicite number consists of
% \begin{itemize}
% \item optional signs `\texttt{+}' and `\texttt{-}',
% \item digits `\texttt{0}' upto `\texttt{9}' and
% \item an optional dot `\texttt{.}'.
% \end{itemize}
% All tokens must have catcode 12 (other), the default catcodes
% for these characters in \hologo{LaTeX}, \hologo{plainTeX} or
% \hologo{iniTeX}. As Perl regular expression the number is expected
% in one of the following forms:
% \begin{itemize}
% \item |^[+-]*[0-9]+$|
% \item |^[+-]*[0-9]*\.[0-9]*$|
% \end{itemize}
% At least one digit or the dot must be present.
%
% \subsection{Error handling}
%
% The package is not intended for validating numbers or to
% decide if an argument is a number. Therefore it is an usage
% error to use the user macros with arguments that are not
% explicite numbers as specified in the previous sections.
% Nevertheless some error conditions are sometimes recognized.
% Errors are given in form of an undefined command sequence.
% It is the only way to notify \hologo{TeX} in expandable context.
% Expanding to some error text would invalidate the output.
% Currently the following errors are thrown:
% \begin{description}
% \item[\cs{thepdfnumber@ErrorEndMarker}:]
%   Internally the argument parsing uses an end marker that
%   is never called directly. If it is called with valid
%   user input, then this is a bug. Otherwise it means the user input
%   contains nasty stuff.
% \item[\cs{thepdfnumber@ErrorUnexpectedEnd}:] The macros expect at least one digit
%   or the dot, otherwise if the argument is empty or only contains
%   signs, then this error is called.
% \item[\cs{thepdfnumber@ErrorInvalidToken}:] It is called if the number
%   contains other tokens than signs, digits or the dot or the
%   token at the wrong place (e.g. a sign after a digit).
%   In case of \cs{thepdfnumberNormZeroOne} this error condition might
%   not always be detected, because the number parsing might stop
%   at an early point, when the result is already clear (e.g.
%   if the number is negative or will be greater than one).
% \item[Improper alphabetic constant:] This error might be
%   thrown by \hologo{TeX}, if the number contains
%   command tokens instead of characters.
% \end{description}
%
% \StopEventually{
% }
%
% \section{Implementation}
%
%    \begin{macrocode}
%<*package>
%    \end{macrocode}
%
% \subsection{Reload check and package identification}
%    Reload check, especially if the package is not used with \LaTeX.
%    \begin{macrocode}
\begingroup\catcode61\catcode48\catcode32=10\relax%
  \catcode13=5 % ^^M
  \endlinechar=13 %
  \catcode35=6 % #
  \catcode39=12 % '
  \catcode44=12 % ,
  \catcode45=12 % -
  \catcode46=12 % .
  \catcode58=12 % :
  \catcode64=11 % @
  \catcode123=1 % {
  \catcode125=2 % }
  \expandafter\let\expandafter\x\csname ver@thepdfnumber.sty\endcsname
  \ifx\x\relax % plain-TeX, first loading
  \else
    \def\empty{}%
    \ifx\x\empty % LaTeX, first loading,
      % variable is initialized, but \ProvidesPackage not yet seen
    \else
      \expandafter\ifx\csname PackageInfo\endcsname\relax
        \def\x#1#2{%
          \immediate\write-1{Package #1 Info: #2.}%
        }%
      \else
        \def\x#1#2{\PackageInfo{#1}{#2, stopped}}%
      \fi
      \x{thepdfnumber}{The package is already loaded}%
      \aftergroup\endinput
    \fi
  \fi
\endgroup%
%    \end{macrocode}
%    Package identification:
%    \begin{macrocode}
\begingroup\catcode61\catcode48\catcode32=10\relax%
  \catcode13=5 % ^^M
  \endlinechar=13 %
  \catcode35=6 % #
  \catcode39=12 % '
  \catcode40=12 % (
  \catcode41=12 % )
  \catcode44=12 % ,
  \catcode45=12 % -
  \catcode46=12 % .
  \catcode47=12 % /
  \catcode58=12 % :
  \catcode64=11 % @
  \catcode91=12 % [
  \catcode93=12 % ]
  \catcode123=1 % {
  \catcode125=2 % }
  \expandafter\ifx\csname ProvidesPackage\endcsname\relax
    \def\x#1#2#3[#4]{\endgroup
      \immediate\write-1{Package: #3 #4}%
      \xdef#1{#4}%
    }%
  \else
    \def\x#1#2[#3]{\endgroup
      #2[{#3}]%
      \ifx#1\@undefined
        \xdef#1{#3}%
      \fi
      \ifx#1\relax
        \xdef#1{#3}%
      \fi
    }%
  \fi
\expandafter\x\csname ver@thepdfnumber.sty\endcsname
\ProvidesPackage{thepdfnumber}%
  [2016/05/16 v1.1 Print PDF numbers with minimal digits (HO)]%
%    \end{macrocode}
%
% \subsection{Catcodes}
%
%    \begin{macrocode}
\begingroup\catcode61\catcode48\catcode32=10\relax%
  \catcode13=5 % ^^M
  \endlinechar=13 %
  \catcode123=1 % {
  \catcode125=2 % }
  \catcode64=11 % @
  \def\x{\endgroup
    \expandafter\edef\csname ThPdNu@AtEnd\endcsname{%
      \endlinechar=\the\endlinechar\relax
      \catcode13=\the\catcode13\relax
      \catcode32=\the\catcode32\relax
      \catcode35=\the\catcode35\relax
      \catcode61=\the\catcode61\relax
      \catcode64=\the\catcode64\relax
      \catcode123=\the\catcode123\relax
      \catcode125=\the\catcode125\relax
    }%
  }%
\x\catcode61\catcode48\catcode32=10\relax%
\catcode13=5 % ^^M
\endlinechar=13 %
\catcode35=6 % #
\catcode64=11 % @
\catcode123=1 % {
\catcode125=2 % }
\def\TMP@EnsureCode#1#2{%
  \edef\ThPdNu@AtEnd{%
    \ThPdNu@AtEnd
    \catcode#1=\the\catcode#1\relax
  }%
  \catcode#1=#2\relax
}
\TMP@EnsureCode{33}{12}% !
\TMP@EnsureCode{36}{3}% $
\TMP@EnsureCode{38}{4}% &
\TMP@EnsureCode{42}{12}% *
\TMP@EnsureCode{43}{12}% +
\TMP@EnsureCode{45}{12}% -
\TMP@EnsureCode{46}{12}% .
\TMP@EnsureCode{60}{12}% <
\TMP@EnsureCode{62}{12}% >
\TMP@EnsureCode{96}{12}% `
\edef\ThPdNu@AtEnd{\ThPdNu@AtEnd\noexpand\endinput}
%    \end{macrocode}
%
% \subsection{Helper macros}
%
%    \begin{macro}{\ThPdNu@FIN}
%    \begin{macrocode}
\def\ThPdNu@FIN{\thepdfnumber@ErrorEndMarker}
%    \end{macrocode}
%    \end{macro}
%
%    \begin{macro}{\ThPdNu@space}
%    \begin{macrocode}
\def\ThPdNu@space{ }
%    \end{macrocode}
%    \end{macro}
%
%    \begin{macro}{\ThPdNu@zero}
%    \begin{macrocode}
\chardef\ThPdNu@zero=0 %
%    \end{macrocode}
%    \end{macro}
%
%    \begin{macro}{\ThPdNu@one}
%    \begin{macrocode}
\chardef\ThPdNu@one=1 %
%    \end{macrocode}
%    \end{macro}
%
%    \begin{macro}{\ThPdNu@firstoftwo}
%    \begin{macrocode}
\long\def\ThPdNu@firstoftwo#1#2{#1}
%    \end{macrocode}
%    \end{macro}
%
%    \begin{macro}{\ThPdNu@secondoftwo}
%    \begin{macrocode}
\long\def\ThPdNu@secondoftwo#1#2{#2}
%    \end{macrocode}
%    \end{macro}
%
% \subsection{Detect \hologo{eTeX}}
%
%    \begin{macrocode}
\begingroup\expandafter\expandafter\expandafter\endgroup
\expandafter\ifx\csname detokenize\endcsname\relax
  \catcode`\&=14 %
  \catcode`\$=9 %
\else
  \catcode`\&=9 %
  \catcode`\$=14 %
\fi
%    \end{macrocode}
%
% \subsection{User macro \cs{thepdfnumber}}
%
%    \begin{macro}{\thepdfnumber}
%    \begin{macrocode}
\def\thepdfnumber#1{%
  \romannumeral
& \iftrue\expandafter\ThPdNu@State@Plus\expandafter\fi
& \detokenize\expandafter{#1}%
& \ThPdNu@FIN
$ \ifx\ThPdNu@FIN#1\ThPdNu@FIN
$   \expandafter\ThPdNu@firstoftwo
$ \else
$   \expandafter\ThPdNu@secondoftwo
$ \fi
$ {%
$   \ThPdNu@zero
$   0\thepdfnumber@ErrorUnexpectedEnd
$ }{%
$   \iftrue\expandafter\ThPdNu@State@Plus\expandafter\fi#1\ThPdNu@FIN
$ }%
}
%    \end{macrocode}
%    \end{macro}
%
% \subsubsection{State definitions for sign}
%
%    \begin{macro}{\ThPdNu@State@Plus}
%    \begin{macrocode}
\def\ThPdNu@State@Plus#1\fi#2{%
  \fi
  \ifcase\ifx\ThPdNu@FIN#2%
           0%
&        \else\ifx-#2%
$        \else\ifnum`#2=45 % -
           1%
         \else\ifx0#2%
           2%
         \else\ifnum`#2>48 %
           \ifnum`#2<58 %
             3%
           \else
             9%
           \fi
&        \else\ifx.#2%
$        \else\ifnum`#2=46 % .
           4%
&        \else\ifx+#2%
$        \else\ifnum`#2=43 % +
           5%
         \else
           9%
         \fi\fi\fi\fi\fi\fi\ThPdNu@space
    \expandafter\ThPdNu@zero
    \expandafter0%
    \expandafter\thepdfnumber@ErrorUnexpectedEnd
  \or
    \ThPdNu@State@Minus
  \or
    \ThPdNu@State@SkipZeros!%
  \or
    \ThPdNu@State@Int!#2!%
  \or
    \ThPdNu@State@Dot!\ThPdNu@zero*\ThPdNu@zero!!%
  \or
    \ThPdNu@State@Plus
  \else
    \ThPdNu@ReturnError{0}%
  \fi
}
%    \end{macrocode}
%    \end{macro}
%
%    \begin{macro}{\ThPdNu@State@Minus}
%    \begin{macrocode}
\def\ThPdNu@State@Minus#1\fi#2{%
  \fi
  \ifcase\ifx\ThPdNu@FIN#2%
           0%
         \else\ifx0#2%
           1%
         \else\ifnum`#2>48 %
           \ifnum`#2<58 %
             2%
           \else
             9%
           \fi
&        \else\ifx.#2%
$        \else\ifnum`#2=46 % .
           3%
&        \else\ifx-#2%
$        \else\ifnum`#2=45 % -
           4%
&        \else\ifx+#2%
$        \else\ifnum`#2=43 % +
           5%
         \else
           9%
         \fi\fi\fi\fi\fi\fi\ThPdNu@space
    \expandafter\ThPdNu@zero
    \expandafter0%
    \expandafter\thepdfnumber@ErrorUnexpectedEnd
  \or
    \ThPdNu@State@SkipZeros-!%
  \or
    \ThPdNu@State@Int-!#2!%
  \or
    \ThPdNu@State@Dot-!\ThPdNu@zero*\ThPdNu@zero!!%
  \or
    \ThPdNu@State@Plus
  \or
    \ThPdNu@State@Minus
  \else
    \ThPdNu@ReturnError{0}%
  \fi
}
%    \end{macrocode}
%    \end{macro}
%
%    \begin{macro}{\ThPdNu@ReturnError}
%    \begin{macrocode}
\def\ThPdNu@ReturnError#1#2\fi#3\ThPdNu@FIN{%
  \fi
  \ThPdNu@zero
  #1%
  \thepdfnumber@ErrorInvalidToken
}
%    \end{macrocode}
%    \end{macro}
%
% \subsubsection{State definitions for integer part}
%
%    \begin{macro}{\ThPdNu@State@SkipZeros}
%    \begin{macrocode}
\def\ThPdNu@State@SkipZeros#1!#2\fi#3{%
  \fi
  \ifcase\ifx\ThPdNu@FIN#3%
           0%
         \else\ifx0#3%
           1%
         \else\ifnum`#3>48 %
           \ifnum`#3<58 %
             2%
           \else
             9%
           \fi
&        \else\ifx.#3%
$        \else\ifnum`#3=46 % .
           3%
         \else
           9%
         \fi\fi\fi\fi\ThPdNu@space
    \expandafter\ThPdNu@zero
    \expandafter0%
  \or
   \ThPdNu@State@SkipZeros#1!%
  \or
    \ThPdNu@State@Int#1!#3!%
  \or
    \ThPdNu@State@Dot#1!\ThPdNu@zero*\ThPdNu@zero!!%
  \else
    \ThPdNu@ReturnError{0}%
  \fi
}
%    \end{macrocode}
%    \end{macro}
%
%    \begin{macro}{\ThPdNu@State@Int}
%    \begin{macrocode}
\def\ThPdNu@State@Int#1!#2!#3\fi#4{%
  \fi
  \ifcase\ifx\ThPdNu@FIN#4%
           0%
         \else\ifnum`#4>47 %
           \ifnum`#4<58 %
             1%
           \else
             9%
           \fi
&        \else\ifx.#4%
$        \else\ifnum`#4=46 % .
           2%
         \else
           9%
         \fi\fi\fi\ThPdNu@space
    \ThPdNu@ReturnInt{#1#2}%
  \or
    \ThPdNu@State@Int#1!#2#4!%
  \or
    \ThPdNu@State@Dot#1!\ThPdNu@one#2*\ThPdNu@zero!!%
  \else
    \ThPdNu@ReturnError{#1#2}%
  \fi
}
%    \end{macrocode}
%    \end{macro}
%
%    \begin{macro}{\ThPdNu@ReturnInt}
%    \begin{macrocode}
\def\ThPdNu@ReturnInt#1#2\fi{%
  \fi
  \ThPdNu@zero
  #1%
}
%    \end{macrocode}
%    \end{macro}
%
% \subsubsection{State definitions for decimal digits}
%
%    \begin{macro}{\ThPdNu@State@Dot}
%    \begin{macrocode}
\def\ThPdNu@State@Dot#1*#2#3!#4!#5\fi#6{%
  \fi
  \ifcase\ifx\ThPdNu@FIN#6%
           0%
         \else\ifnum`#6>48 %
           \ifnum`#6<58 %
             1%
           \else
             9%
           \fi
         \else\ifx0#6%
           2%
         \else
           9%
         \fi\fi\fi\ThPdNu@space
    \ThPdNu@ReturnNumber#1*#2#3!%
  \or
    \ThPdNu@State@Dot#1*\ThPdNu@one#3#4#6!!%
  \or
    \ThPdNu@State@DotZero#1*#2#3!#4#6!%
  \else
    \ThPdNu@ReturnNumberInvalid#1*#2#3!%
  \fi
}
%    \end{macrocode}
%    \end{macro}
%
%    \begin{macro}{\ThPdNu@State@DotZero}
%    \begin{macrocode}
\def\ThPdNu@State@DotZero#1*#2#3!#4!#5\fi#6{%
  \fi
  \ifcase\ifx\ThPdNu@FIN#6%
           0%
         \else\ifnum`#6>48 %
           \ifnum`#6<58 %
             1%
           \else
             9%
           \fi
         \else\ifx0#6%
           2%
         \else
           9%
         \fi\fi\fi\ThPdNu@space
    \ThPdNu@ReturnNumber#1*#2#3!%
  \or
    \ThPdNu@State@Dot#1*\ThPdNu@one#3#4#6!!%
  \or
    \ThPdNu@State@DotZero#1*#2#3!#4#6!%
  \else
    \ThPdNu@ReturnNumber#1*#2#3!%
  \fi
}
%    \end{macrocode}
%    \end{macro}
%
%    \begin{macro}{\ThPdNu@ReturnNumber}
%    \begin{macrocode}
\def\ThPdNu@ReturnNumber#1!#2#3*#4#5!#6\fi{%
  \fi
  \ifcase#2%
    \expandafter\ThPdNu@firstoftwo
  \else
    \expandafter\ThPdNu@secondoftwo
  \fi
  {%
    \ifcase#4%
      \expandafter\ThPdNu@firstoftwo
    \else
      \expandafter\ThPdNu@secondoftwo
    \fi
    {\ThPdNu@zero 0}%
    {\ThPdNu@zero #1.#5}%
  }{%
    \ifcase#4%
      \expandafter\ThPdNu@firstoftwo
    \else
      \expandafter\ThPdNu@secondoftwo
    \fi
    {\ThPdNu@zero #1#3}%
    {\ThPdNu@zero #1#3.#5}%
  }%
}
%    \end{macrocode}
%    \end{macro}
%
%    \begin{macro}{\ThPdNu@ReturnNumberInvalid}
%    \begin{macrocode}
\def\ThPdNu@ReturnNumberInvalid#1*#2!#3\fi#4\ThPdNu@FIN{%
  \fi
  \iftrue\ThPdNu@ReturnNumber#1*#2!\fi
  \thepdfnumber@ErrorInvalidToken
}
%    \end{macrocode}
%    \end{macro}
%
% \subsection{Norm macro}
%
%    \begin{macro}{\thepdfnumberNormZeroOne}
%    \begin{macrocode}
\def\thepdfnumberNormZeroOne#1{%
  \romannumeral
& \iftrue\expandafter\ThPbNu@StateN@Plus\expandafter\fi
& \detokenize\expandafter{#1}%
& \ThPdNu@FIN
$ \ifx\ThPdNu@FIN#1\ThPdNu@FIN
$   \expandafter\ThPdNu@firstoftwo
$ \else
$   \expandafter\ThPdNu@secondoftwo
$ \fi
$ {%
$   \ThPdNu@zero
$   0\thepdfnumber@ErrorUnexpectedEnd
$ }{%
$   \iftrue\expandafter\ThPbNu@StateN@Plus\expandafter\fi#1\ThPdNu@FIN
$ }%
}
%    \end{macrocode}
%    \end{macro}
%
% \subsubsection{State definitions for sign}
%
%    \begin{macro}{\ThPbNu@StateN@Plus}
%    \begin{macrocode}
\def\ThPbNu@StateN@Plus#1\fi#2{%
  \fi
  \ifcase\ifx\ThPdNu@FIN#2%
           0%
&        \else\ifx-#2%
$        \else\ifnum`#2=45 % -
           1%
         \else\ifx0#2%
           2%
         \else\ifnum`#2>48 %
           \ifnum`#2<58 %
             3%
           \else
             9%
           \fi
&        \else\ifx.#2%
$        \else\ifnum`#2=46 % .
           4%
&        \else\ifx+#2%
$        \else\ifnum`#2=43 % +
           5%
         \else
           9%
         \fi\fi\fi\fi\fi\fi\ThPdNu@space
    \expandafter\ThPdNu@zero
    \expandafter0%
    \expandafter\thepdfnumber@ErrorUnexpectedEnd
  \or
    \ThPbNu@StateN@Minus
  \or
    \ThPbNu@StateN@SkipZeros
  \or
    \ThPdNu@ReturnAndSkip{1}%
  \or
    \ThPbNu@StateN@Dot\ThPdNu@zero!!%
  \or
    \ThPbNu@StateN@Plus
  \else
    \ThPdNu@ReturnError{0}%
  \fi
}
%    \end{macrocode}
%    \end{macro}
%
%    \begin{macro}{\ThPbNu@StateN@Minus}
%    \begin{macrocode}
\def\ThPbNu@StateN@Minus#1\fi#2{%
  \fi
  \ifcase\ifx\ThPdNu@FIN#2%
           0%
         \else\ifnum`#2>47 %
           \ifnum`#2<58 %
             1%
           \else
             9%
           \fi
&        \else\ifx.#2%
$        \else\ifnum`#2=46 % .
           1%
&        \else\ifx-#2%
$        \else\ifnum`#2=45 % -
           2%
&        \else\ifx+#2%
$        \else\ifnum`#2=43 % +
           3%
         \else
           9%
         \fi\fi\fi\fi\fi\ThPdNu@space
    \expandafter\ThPdNu@zero
    \expandafter0%
    \expandafter\thepdfnumber@ErrorUnexpectedEnd
  \or
    \ThPdNu@ReturnAndSkip{0}%
  \or
    \ThPbNu@StateN@Plus
  \or
    \ThPbNu@StateN@Minus
  \else
    \ThPdNu@ReturnError{0}%
  \fi
}
%    \end{macrocode}
%    \end{macro}
%
%    \begin{macro}{\ThPdNu@ReturnAndSkip}
%    \begin{macrocode}
\def\ThPdNu@ReturnAndSkip#1#2\fi#3\ThPdNu@FIN{%
  \fi
  \ThPdNu@zero
  #1%
}
%    \end{macrocode}
%    \end{macro}
%
% \subsubsection{State definitions for integer part}
%
%    \begin{macro}{\ThPbNu@StateN@SkipZeros}
%    \begin{macrocode}
\def\ThPbNu@StateN@SkipZeros#1\fi#2{%
  \fi
  \ifcase\ifx\ThPdNu@FIN#2%
           0%
         \else\ifx0#2%
           1%
         \else\ifnum`#2>48 %
           \ifnum`#2<58 %
             2%
           \else
             9%
           \fi
&        \else\ifx.#2%
$        \else\ifnum`#2=46 % .
           3%
         \else
           9%
         \fi\fi\fi\fi\ThPdNu@space
    \expandafter\ThPdNu@zero
    \expandafter0%
  \or
   \ThPbNu@StateN@SkipZeros%
  \or
    \ThPdNu@ReturnAndSkip{1}%
  \or
    \ThPbNu@StateN@Dot\ThPdNu@zero!!%
  \else
    \ThPdNu@ReturnError{0}%
  \fi
}
%    \end{macrocode}
%    \end{macro}
%
% \subsubsection{State definitions for decimal digits}
%
%    \begin{macro}{\ThPbNu@StateN@Dot}
%    \begin{macrocode}
\def\ThPbNu@StateN@Dot#1#2!#3!#4\fi#5{%
  \fi
  \ifcase\ifx\ThPdNu@FIN#5%
           0%
         \else\ifnum`#5>48 %
           \ifnum`#5<58 %
             1%
           \else
             9%
           \fi
         \else\ifx0#5%
           2%
         \else
           9%
         \fi\fi\fi\ThPdNu@space
    \ThPdNu@ReturnFracNumber#1#2!%
  \or
    \ThPbNu@StateN@Dot\ThPdNu@one#2#3#5!!%
  \or
    \ThPbNu@StateN@DotZero#1#2!#3#5!%
  \else
    \ThPdNu@ReturnFracNumberInvalid#1#2!%
  \fi
}
%    \end{macrocode}
%    \end{macro}
%
%    \begin{macro}{\ThPbNu@StateN@DotZero}
%    \begin{macrocode}
\def\ThPbNu@StateN@DotZero#1#2!#3!#4\fi#5{%
  \fi
  \ifcase\ifx\ThPdNu@FIN#5%
           0%
         \else\ifnum`#5>48 %
           \ifnum`#5<58 %
             1%
           \else
             9%
           \fi
         \else\ifx0#5%
           2%
         \else
           9%
         \fi\fi\fi\ThPdNu@space
    \ThPdNu@ReturnFracNumber#1#2!%
  \or
    \ThPbNu@StateN@Dot\ThPdNu@one#2#3#5!!%
  \or
    \ThPbNu@StateN@DotZero#1#2!#3#5!%
  \else
    \ThPdNu@ReturnFracNumberInvalid#1#2!%
  \fi
}
%    \end{macrocode}
%    \end{macro}
%
%    \begin{macro}{\ThPdNu@ReturnFracNumber}
%    \begin{macrocode}
\def\ThPdNu@ReturnFracNumber#1#2!#3\fi{%
  \fi
  \ifcase#1%
    \expandafter\ThPdNu@firstoftwo
  \else
    \expandafter\ThPdNu@secondoftwo
  \fi
  {\ThPdNu@zero 0}%
  {\ThPdNu@zero .#2}%
}
%    \end{macrocode}
%    \end{macro}
%
%    \begin{macro}{\ThPdNu@ReturnFracNumberInvalid}
%    \begin{macrocode}
\def\ThPdNu@ReturnFracNumberInvalid#1!#2\fi#3\ThPdNu@FIN{%
  \fi
  \iftrue\ThPdNu@ReturnFracNumber#1!\fi
  \thepdfnumber@ErrorInvalidToken
}
%    \end{macrocode}
%    \end{macro}
%
%    \begin{macrocode}
\ThPdNu@AtEnd%
%</package>
%    \end{macrocode}
%
% \section{Test}
%
% \subsection{Catcode checks for loading}
%
%    \begin{macrocode}
%<*test1>
%    \end{macrocode}
%    \begin{macrocode}
\catcode`\{=1 %
\catcode`\}=2 %
\catcode`\#=6 %
\catcode`\@=11 %
\expandafter\ifx\csname count@\endcsname\relax
  \countdef\count@=255 %
\fi
\expandafter\ifx\csname @gobble\endcsname\relax
  \long\def\@gobble#1{}%
\fi
\expandafter\ifx\csname @firstofone\endcsname\relax
  \long\def\@firstofone#1{#1}%
\fi
\expandafter\ifx\csname loop\endcsname\relax
  \expandafter\@firstofone
\else
  \expandafter\@gobble
\fi
{%
  \def\loop#1\repeat{%
    \def\body{#1}%
    \iterate
  }%
  \def\iterate{%
    \body
      \let\next\iterate
    \else
      \let\next\relax
    \fi
    \next
  }%
  \let\repeat=\fi
}%
\def\RestoreCatcodes{}
\count@=0 %
\loop
  \edef\RestoreCatcodes{%
    \RestoreCatcodes
    \catcode\the\count@=\the\catcode\count@\relax
  }%
\ifnum\count@<255 %
  \advance\count@ 1 %
\repeat

\def\RangeCatcodeInvalid#1#2{%
  \count@=#1\relax
  \loop
    \catcode\count@=15 %
  \ifnum\count@<#2\relax
    \advance\count@ 1 %
  \repeat
}
\def\RangeCatcodeCheck#1#2#3{%
  \count@=#1\relax
  \loop
    \ifnum#3=\catcode\count@
    \else
      \errmessage{%
        Character \the\count@\space
        with wrong catcode \the\catcode\count@\space
        instead of \number#3%
      }%
    \fi
  \ifnum\count@<#2\relax
    \advance\count@ 1 %
  \repeat
}
\def\space{ }
\expandafter\ifx\csname LoadCommand\endcsname\relax
  \def\LoadCommand{\input thepdfnumber.sty\relax}%
\fi
\def\Test{%
  \RangeCatcodeInvalid{0}{47}%
  \RangeCatcodeInvalid{58}{64}%
  \RangeCatcodeInvalid{91}{96}%
  \RangeCatcodeInvalid{123}{255}%
  \catcode`\@=12 %
  \catcode`\\=0 %
  \catcode`\%=14 %
  \LoadCommand
  \RangeCatcodeCheck{0}{36}{15}%
  \RangeCatcodeCheck{37}{37}{14}%
  \RangeCatcodeCheck{38}{47}{15}%
  \RangeCatcodeCheck{48}{57}{12}%
  \RangeCatcodeCheck{58}{63}{15}%
  \RangeCatcodeCheck{64}{64}{12}%
  \RangeCatcodeCheck{65}{90}{11}%
  \RangeCatcodeCheck{91}{91}{15}%
  \RangeCatcodeCheck{92}{92}{0}%
  \RangeCatcodeCheck{93}{96}{15}%
  \RangeCatcodeCheck{97}{122}{11}%
  \RangeCatcodeCheck{123}{255}{15}%
  \RestoreCatcodes
}
\Test
\csname @@end\endcsname
\end
%    \end{macrocode}
%    \begin{macrocode}
%</test1>
%    \end{macrocode}
%
% \subsection{Macro tests}
%
% \subsubsection{Preamble}
%
%    \begin{macrocode}
%<*test2>
\catcode`\{=1
\catcode`\}=2
\catcode`\#=6
\catcode`\@=11
\errorcontextlines=10000 %
\def\msg#{\immediate\write16}
\def\space{ }
\begingroup\expandafter\expandafter\expandafter\endgroup
\expandafter\ifx\csname RequirePackage\endcsname\relax
  \input thepdfnumber.sty\relax
\else
  \RequirePackage{thepdfnumber}[2016/05/16]%
\fi
%    \end{macrocode}
%
%    \begin{macro}{\Test}
%    \begin{macrocode}
\def\Test#1#2{%
  \def\TestExpected{#2}%
%<*active>
  \edef\TestExpected{\TestExpected}%
%</active>
  \expandafter\expandafter\expandafter\def
  \expandafter\expandafter\expandafter\TestResult
  \expandafter\expandafter\expandafter{%
    \TestCommand{#1}%
  }%
  \begingroup
    \toks0\expandafter{\TestResult}%
    \toks2\expandafter{\TestExpected}%
    \ifx\TestResult\TestExpected
      \msg{* OK: #1 => \the\toks0}%
    \else
      \errmessage{FAILED: #1 => \the\toks0\space(\the\toks2)}%
    \fi
  \endgroup
}
%    \end{macrocode}
%    \end{macro}
%
%    \begin{macrocode}
\let\thepdfnumber@ErrorUnexpectedEnd\relax
\let\thepdfnumber@ErrorInvalidToken\relax
%    \end{macrocode}
%
% \subsubsection{Test with active catcodes}
%
%    \begin{macrocode}
%<*active>
\catcode`\~=13 %
\def\ThPdNu@Temp#1{%
  \begingroup
    \lccode`\~=`#1 %
  \lowercase{\endgroup
    \edef~%
  }{\string#1}%
  \catcode`#1=13 %
}
\ThPdNu@Temp{.}
\ThPdNu@Temp{-}
\ThPdNu@Temp{+}
%</active>
%    \end{macrocode}
%
% \subsubsection{Test cases for \cs{thepdfnumber}}
%
%    \begin{macro}{\TestCommand}
%    \begin{macrocode}
\let\TestCommand\thepdfnumber
%    \end{macrocode}
%    \end{macro}
%
%    \begin{macrocode}
\Test{0}{0}
\Test{1}{1}
\Test{23}{23}
\Test{0045}{45}
\Test{4500}{4500}
\Test{-0}{0}
\Test{-1}{-1}
\Test{-00002}{-2}
\Test{-010203}{-10203}
\Test{+0}{0}
\Test{+123}{123}

\Test{+-4}{-4}
\Test{+-+-4}{4}
\Test{++--++--++4}{4}
\Test{+++---+++---+++---4}{-4}

\Test{.}{0}
\Test{0.}{0}
\Test{.0}{0}
\Test{0.}{0}
\Test{-.}{0}
\Test{-0.}{0}
\Test{-.0}{0}
\Test{-0.}{0}
\Test{010.020}{10.02}
\Test{123.456}{123.456}
\Test{12.}{12}
\Test{.123}{.123}
\Test{.0001}{.0001}
\Test{.00100}{.001}
\Test{.12003400560078009}{.12003400560078009}
\Test{-.12003400560078009}{-.12003400560078009}
\Test{04.0500000}{4.05}

\Test{}{0\thepdfnumber@ErrorUnexpectedEnd}
\Test{+}{0\thepdfnumber@ErrorUnexpectedEnd}
\Test{-}{0\thepdfnumber@ErrorUnexpectedEnd}
\Test{a}{0\thepdfnumber@ErrorInvalidToken}
\Test{0x1}{0\thepdfnumber@ErrorInvalidToken}
\Test{4x56}{4\thepdfnumber@ErrorInvalidToken}
\Test{012,34}{12\thepdfnumber@ErrorInvalidToken}
\Test{0.12x4}{.12\thepdfnumber@ErrorInvalidToken}
%    \end{macrocode}
%
% \subsubsection{Test cases for \cs{thepdfnumberNormZeroOne}}
%
%    \begin{macro}{\TestCommand}
%    \begin{macrocode}
\let\TestCommand\thepdfnumberNormZeroOne
%    \end{macrocode}
%    \end{macro}
%
%    \begin{macrocode}
\Test{0}{0}
\Test{1}{1}
\Test{23}{1}
\Test{12}{1}
\Test{10}{1}
\Test{0045}{1}
\Test{001}{1}
\Test{-0}{0}
\Test{-1}{0}
\Test{-01}{0}
\Test{-4}{0}

\Test{+-0}{0}
\Test{+-+-1}{1}
\Test{++--++--++1}{1}
\Test{+++---+++---+++---1}{0}

\Test{.}{0}
\Test{0.}{0}
\Test{.0}{0}
\Test{0.}{0}
\Test{-.}{0}
\Test{-0.}{0}
\Test{-.0}{0}
\Test{-0.}{0}
\Test{010.020}{1}
\Test{123.456}{1}
\Test{12.}{1}
\Test{.123}{.123}
\Test{.0001}{.0001}
\Test{.00100}{.001}
\Test{.12003400560078009}{.12003400560078009}
\Test{-.12003400560078009}{0}
\Test{04.0500000}{1}
\Test{0.1200340056}{.1200340056}
\Test{1.05}{1}

\Test{}{0\thepdfnumber@ErrorUnexpectedEnd}
\Test{+}{0\thepdfnumber@ErrorUnexpectedEnd}
\Test{-}{0\thepdfnumber@ErrorUnexpectedEnd}
\Test{a}{0\thepdfnumber@ErrorInvalidToken}
\Test{0x1}{0\thepdfnumber@ErrorInvalidToken}
\Test{4x56}{1}
\Test{012,34}{1}
\Test{.012x4}{.012\thepdfnumber@ErrorInvalidToken}
%    \end{macrocode}
%
%    \begin{macrocode}
\csname @@end\endcsname\end
%</test2>
%    \end{macrocode}
%
% \section{Installation}
%
% \subsection{Download}
%
% \paragraph{Package.} This package is available on
% CTAN\footnote{\url{http://ctan.org/pkg/thepdfnumber}}:
% \begin{description}
% \item[\CTAN{macros/latex/contrib/oberdiek/thepdfnumber.dtx}] The source file.
% \item[\CTAN{macros/latex/contrib/oberdiek/thepdfnumber.pdf}] Documentation.
% \end{description}
%
%
% \paragraph{Bundle.} All the packages of the bundle `oberdiek'
% are also available in a TDS compliant ZIP archive. There
% the packages are already unpacked and the documentation files
% are generated. The files and directories obey the TDS standard.
% \begin{description}
% \item[\CTAN{install/macros/latex/contrib/oberdiek.tds.zip}]
% \end{description}
% \emph{TDS} refers to the standard ``A Directory Structure
% for \TeX\ Files'' (\CTAN{tds/tds.pdf}). Directories
% with \xfile{texmf} in their name are usually organized this way.
%
% \subsection{Bundle installation}
%
% \paragraph{Unpacking.} Unpack the \xfile{oberdiek.tds.zip} in the
% TDS tree (also known as \xfile{texmf} tree) of your choice.
% Example (linux):
% \begin{quote}
%   |unzip oberdiek.tds.zip -d ~/texmf|
% \end{quote}
%
% \paragraph{Script installation.}
% Check the directory \xfile{TDS:scripts/oberdiek/} for
% scripts that need further installation steps.
% Package \xpackage{attachfile2} comes with the Perl script
% \xfile{pdfatfi.pl} that should be installed in such a way
% that it can be called as \texttt{pdfatfi}.
% Example (linux):
% \begin{quote}
%   |chmod +x scripts/oberdiek/pdfatfi.pl|\\
%   |cp scripts/oberdiek/pdfatfi.pl /usr/local/bin/|
% \end{quote}
%
% \subsection{Package installation}
%
% \paragraph{Unpacking.} The \xfile{.dtx} file is a self-extracting
% \docstrip\ archive. The files are extracted by running the
% \xfile{.dtx} through \plainTeX:
% \begin{quote}
%   \verb|tex thepdfnumber.dtx|
% \end{quote}
%
% \paragraph{TDS.} Now the different files must be moved into
% the different directories in your installation TDS tree
% (also known as \xfile{texmf} tree):
% \begin{quote}
% \def\t{^^A
% \begin{tabular}{@{}>{\ttfamily}l@{ $\rightarrow$ }>{\ttfamily}l@{}}
%   thepdfnumber.sty & tex/generic/oberdiek/thepdfnumber.sty\\
%   thepdfnumber.pdf & doc/latex/oberdiek/thepdfnumber.pdf\\
%   test/thepdfnumber-test1.tex & doc/latex/oberdiek/test/thepdfnumber-test1.tex\\
%   test/thepdfnumber-test2.tex & doc/latex/oberdiek/test/thepdfnumber-test2.tex\\
%   test/thepdfnumber-test3.tex & doc/latex/oberdiek/test/thepdfnumber-test3.tex\\
%   thepdfnumber.dtx & source/latex/oberdiek/thepdfnumber.dtx\\
% \end{tabular}^^A
% }^^A
% \sbox0{\t}^^A
% \ifdim\wd0>\linewidth
%   \begingroup
%     \advance\linewidth by\leftmargin
%     \advance\linewidth by\rightmargin
%   \edef\x{\endgroup
%     \def\noexpand\lw{\the\linewidth}^^A
%   }\x
%   \def\lwbox{^^A
%     \leavevmode
%     \hbox to \linewidth{^^A
%       \kern-\leftmargin\relax
%       \hss
%       \usebox0
%       \hss
%       \kern-\rightmargin\relax
%     }^^A
%   }^^A
%   \ifdim\wd0>\lw
%     \sbox0{\small\t}^^A
%     \ifdim\wd0>\linewidth
%       \ifdim\wd0>\lw
%         \sbox0{\footnotesize\t}^^A
%         \ifdim\wd0>\linewidth
%           \ifdim\wd0>\lw
%             \sbox0{\scriptsize\t}^^A
%             \ifdim\wd0>\linewidth
%               \ifdim\wd0>\lw
%                 \sbox0{\tiny\t}^^A
%                 \ifdim\wd0>\linewidth
%                   \lwbox
%                 \else
%                   \usebox0
%                 \fi
%               \else
%                 \lwbox
%               \fi
%             \else
%               \usebox0
%             \fi
%           \else
%             \lwbox
%           \fi
%         \else
%           \usebox0
%         \fi
%       \else
%         \lwbox
%       \fi
%     \else
%       \usebox0
%     \fi
%   \else
%     \lwbox
%   \fi
% \else
%   \usebox0
% \fi
% \end{quote}
% If you have a \xfile{docstrip.cfg} that configures and enables \docstrip's
% TDS installing feature, then some files can already be in the right
% place, see the documentation of \docstrip.
%
% \subsection{Refresh file name databases}
%
% If your \TeX~distribution
% (\teTeX, \mikTeX, \dots) relies on file name databases, you must refresh
% these. For example, \teTeX\ users run \verb|texhash| or
% \verb|mktexlsr|.
%
% \subsection{Some details for the interested}
%
% \paragraph{Attached source.}
%
% The PDF documentation on CTAN also includes the
% \xfile{.dtx} source file. It can be extracted by
% AcrobatReader 6 or higher. Another option is \textsf{pdftk},
% e.g. unpack the file into the current directory:
% \begin{quote}
%   \verb|pdftk thepdfnumber.pdf unpack_files output .|
% \end{quote}
%
% \paragraph{Unpacking with \LaTeX.}
% The \xfile{.dtx} chooses its action depending on the format:
% \begin{description}
% \item[\plainTeX:] Run \docstrip\ and extract the files.
% \item[\LaTeX:] Generate the documentation.
% \end{description}
% If you insist on using \LaTeX\ for \docstrip\ (really,
% \docstrip\ does not need \LaTeX), then inform the autodetect routine
% about your intention:
% \begin{quote}
%   \verb|latex \let\install=y\input{thepdfnumber.dtx}|
% \end{quote}
% Do not forget to quote the argument according to the demands
% of your shell.
%
% \paragraph{Generating the documentation.}
% You can use both the \xfile{.dtx} or the \xfile{.drv} to generate
% the documentation. The process can be configured by the
% configuration file \xfile{ltxdoc.cfg}. For instance, put this
% line into this file, if you want to have A4 as paper format:
% \begin{quote}
%   \verb|\PassOptionsToClass{a4paper}{article}|
% \end{quote}
% An example follows how to generate the
% documentation with pdf\LaTeX:
% \begin{quote}
%\begin{verbatim}
%pdflatex thepdfnumber.dtx
%makeindex -s gind.ist thepdfnumber.idx
%pdflatex thepdfnumber.dtx
%makeindex -s gind.ist thepdfnumber.idx
%pdflatex thepdfnumber.dtx
%\end{verbatim}
% \end{quote}
%
% \printbibliography[
%   heading=bibnumbered,
% ]
%
% \begin{History}
%   \begin{Version}{2011/11/24 v1.0}
%   \item
%     First version.
%   \end{Version}
%   \begin{Version}{2016/05/16 v1.1}
%   \item
%     Documentation updates.
%   \end{Version}
% \end{History}
%
% \PrintIndex
%
% \Finale
\endinput
|
% \end{quote}
% Do not forget to quote the argument according to the demands
% of your shell.
%
% \paragraph{Generating the documentation.}
% You can use both the \xfile{.dtx} or the \xfile{.drv} to generate
% the documentation. The process can be configured by the
% configuration file \xfile{ltxdoc.cfg}. For instance, put this
% line into this file, if you want to have A4 as paper format:
% \begin{quote}
%   \verb|\PassOptionsToClass{a4paper}{article}|
% \end{quote}
% An example follows how to generate the
% documentation with pdf\LaTeX:
% \begin{quote}
%\begin{verbatim}
%pdflatex thepdfnumber.dtx
%makeindex -s gind.ist thepdfnumber.idx
%pdflatex thepdfnumber.dtx
%makeindex -s gind.ist thepdfnumber.idx
%pdflatex thepdfnumber.dtx
%\end{verbatim}
% \end{quote}
%
% \printbibliography[
%   heading=bibnumbered,
% ]
%
% \begin{History}
%   \begin{Version}{2011/11/24 v1.0}
%   \item
%     First version.
%   \end{Version}
%   \begin{Version}{2016/05/16 v1.1}
%   \item
%     Documentation updates.
%   \end{Version}
% \end{History}
%
% \PrintIndex
%
% \Finale
\endinput

%        (quote the arguments according to the demands of your shell)
%
% Documentation:
%    (a) If thepdfnumber.drv is present:
%           latex thepdfnumber.drv
%    (b) Without thepdfnumber.drv:
%           latex thepdfnumber.dtx; ...
%    The class ltxdoc loads the configuration file ltxdoc.cfg
%    if available. Here you can specify further options, e.g.
%    use A4 as paper format:
%       \PassOptionsToClass{a4paper}{article}
%
%    Programm calls to get the documentation (example):
%       pdflatex thepdfnumber.dtx
%       makeindex -s gind.ist thepdfnumber.idx
%       pdflatex thepdfnumber.dtx
%       makeindex -s gind.ist thepdfnumber.idx
%       pdflatex thepdfnumber.dtx
%
% Installation:
%    TDS:tex/generic/oberdiek/thepdfnumber.sty
%    TDS:doc/latex/oberdiek/thepdfnumber.pdf
%    TDS:doc/latex/oberdiek/test/thepdfnumber-test1.tex
%    TDS:doc/latex/oberdiek/test/thepdfnumber-test2.tex
%    TDS:doc/latex/oberdiek/test/thepdfnumber-test3.tex
%    TDS:source/latex/oberdiek/thepdfnumber.dtx
%
%<*ignore>
\begingroup
  \catcode123=1 %
  \catcode125=2 %
  \def\x{LaTeX2e}%
\expandafter\endgroup
\ifcase 0\ifx\install y1\fi\expandafter
         \ifx\csname processbatchFile\endcsname\relax\else1\fi
         \ifx\fmtname\x\else 1\fi\relax
\else\csname fi\endcsname
%</ignore>
%<*install>
\input docstrip.tex
\Msg{************************************************************************}
\Msg{* Installation}
\Msg{* Package: thepdfnumber 2016/05/16 v1.1 Print PDF numbers with minimal digits (HO)}
\Msg{************************************************************************}

\keepsilent
\askforoverwritefalse

\let\MetaPrefix\relax
\preamble

This is a generated file.

Project: thepdfnumber
Version: 2016/05/16 v1.1

Copyright (C) 2011 by
   Heiko Oberdiek <heiko.oberdiek at googlemail.com>

This work may be distributed and/or modified under the
conditions of the LaTeX Project Public License, either
version 1.3c of this license or (at your option) any later
version. This version of this license is in
   http://www.latex-project.org/lppl/lppl-1-3c.txt
and the latest version of this license is in
   http://www.latex-project.org/lppl.txt
and version 1.3 or later is part of all distributions of
LaTeX version 2005/12/01 or later.

This work has the LPPL maintenance status "maintained".

This Current Maintainer of this work is Heiko Oberdiek.

The Base Interpreter refers to any `TeX-Format',
because some files are installed in TDS:tex/generic//.

This work consists of the main source file thepdfnumber.dtx
and the derived files
   thepdfnumber.sty, thepdfnumber.pdf, thepdfnumber.ins, thepdfnumber.drv,
   thepdfnumber-test1.tex, thepdfnumber-test2.tex,
   thepdfnumber-test3.tex.

\endpreamble
\let\MetaPrefix\DoubleperCent

\generate{%
  \file{thepdfnumber.ins}{\from{thepdfnumber.dtx}{install}}%
  \file{thepdfnumber.drv}{\from{thepdfnumber.dtx}{driver}}%
  \usedir{tex/generic/oberdiek}%
  \file{thepdfnumber.sty}{\from{thepdfnumber.dtx}{package}}%
  \usedir{doc/latex/oberdiek/test}%
  \file{thepdfnumber-test1.tex}{\from{thepdfnumber.dtx}{test1}}%
  \file{thepdfnumber-test2.tex}{\from{thepdfnumber.dtx}{test2}}%
  \file{thepdfnumber-test3.tex}{\from{thepdfnumber.dtx}{test2,active}}%
}

\catcode32=13\relax% active space
\let =\space%
\Msg{************************************************************************}
\Msg{*}
\Msg{* To finish the installation you have to move the following}
\Msg{* file into a directory searched by TeX:}
\Msg{*}
\Msg{*     thepdfnumber.sty}
\Msg{*}
\Msg{* To produce the documentation run the file `thepdfnumber.drv'}
\Msg{* through LaTeX.}
\Msg{*}
\Msg{* Happy TeXing!}
\Msg{*}
\Msg{************************************************************************}

\endbatchfile
%</install>
%<*ignore>
\fi
%</ignore>
%<*driver>
\NeedsTeXFormat{LaTeX2e}
\ProvidesFile{thepdfnumber.drv}%
  [2016/05/16 v1.1 Print PDF numbers with minimal digits (HO)]%
\documentclass{ltxdoc}
\usepackage{holtxdoc}[2011/11/22]
\usepackage{csquotes}
\usepackage[
  backend=bibtex,
  bibencoding=ascii,
  alldates=iso8601,
]{biblatex}[2011/11/13]
\bibliography{oberdiek-source}
\begin{document}
  \DocInput{thepdfnumber.dtx}%
\end{document}
%</driver>
% \fi
%
%
% \CharacterTable
%  {Upper-case    \A\B\C\D\E\F\G\H\I\J\K\L\M\N\O\P\Q\R\S\T\U\V\W\X\Y\Z
%   Lower-case    \a\b\c\d\e\f\g\h\i\j\k\l\m\n\o\p\q\r\s\t\u\v\w\x\y\z
%   Digits        \0\1\2\3\4\5\6\7\8\9
%   Exclamation   \!     Double quote  \"     Hash (number) \#
%   Dollar        \$     Percent       \%     Ampersand     \&
%   Acute accent  \'     Left paren    \(     Right paren   \)
%   Asterisk      \*     Plus          \+     Comma         \,
%   Minus         \-     Point         \.     Solidus       \/
%   Colon         \:     Semicolon     \;     Less than     \<
%   Equals        \=     Greater than  \>     Question mark \?
%   Commercial at \@     Left bracket  \[     Backslash     \\
%   Right bracket \]     Circumflex    \^     Underscore    \_
%   Grave accent  \`     Left brace    \{     Vertical bar  \|
%   Right brace   \}     Tilde         \~}
%
% \GetFileInfo{thepdfnumber.drv}
%
% \title{The \xpackage{thepdfnumber} package}
% \date{2016/05/16 v1.1}
% \author{Heiko Oberdiek\thanks
% {Please report any issues at https://github.com/ho-tex/oberdiek/issues}\\
% \xemail{heiko.oberdiek at googlemail.com}}
%
% \maketitle
%
% \begin{abstract}
% The package converts real numbers to a minimal representation
% that is stripped from leading or trailing zeros,
% plus signs and decimal point if not necessary.
% \end{abstract}
%
% \tableofcontents
%
% \section{Documentation}
%
% \subsection{Introduction}
%
% Dealing with the PDF format, there is sometimes the need to write
% some low level PDF stuff. In case of numbers, the numbers can
% arise from user input (e.g. color or transparency specifications)
% or can be calculated. For example, \hologo{LaTeX}'s \cs{strip@pt}
% makes a good job to output a real number. It automatically suppresses
% the decimal part if the number is an integer. However it leaves
% a leading zero for numbers greater zero and smaller one.
% Thus the package provides macros that can be used with different
% formats, even with \hologo{iniTeX} and generates numbers that
% are valid numbers of the PDF format and whose length is minimal.
%
% \subsection{Usage}
%
% The package \xpackage{thepdfnumber} can be used with \hologo{LaTeX},
% \hologo{plainTeX} or even with \hologo{iniTeX}:
% \begin{quote}
%   |\RequirePackage{thepdfnumber}| \textit{\% \hologo{LaTeX}}\\
%   |\input thepdfnumber.sty| \textit{\% \hologo{plainTeX}/\hologo{iniTeX}}
% \end{quote}
% The package does not need and have package options.
%
% \subsection{User macros}
%
% All user macros are expandable in exact two expansion steps.
%
% \begin{declcs}{thepdfnumber} \M{number}
% \end{declcs}
% Macro \cs{thepdfnumber} takes a number
% as argument and expands to a minimal representation of that number.
% Some examples:
% \begin{quote}
% \begin{tabular}{r@{.: }>{\ttfamily}l@{ $\rightarrow$\ }>{\ttfamily}l@{}}
%   1& +123 & 123\\
%   2& --123 & 123\\
%   3& -01 & -1\\
%   4& 0045 & 45\\
%   5& 1.0 & 1\\
%   6& 1.20 & 1.2\\
%   7& 0.0 & 0\\
%   8& 0.78 & .78\\
%   9& +012.340 & 12.34\\
% \end{tabular}
% \end{quote}
% It reduces the length of the number representation:
% \begin{itemize}
% \item The signs are collapsed and only one minus sign
%   is output if the number is negative (see examples 1, 2, 3, 9).
% \item Leading zeros are removed (4, 8, 9) unless the number is zero (7).
% \item The decimal part is omitted, if the number is an integer (5, 7).
% \item Trailing zeros from the decimal part are stripped (5, 6, 7, 9).
% \end{itemize}
% The resulting number representation can be catched with one of the
% following Perl regular expressions:
% \begin{itemize}
% \item |^0$| (zero)
% \item |^-?[1-9][0-9]*$| (integer)
% \item |^-?[0-9]*\.[0-9]*[1-9]$| (real)
% \end{itemize}
% This is a valid numeric object of the PDF specification \cite[``7.3.3~Numeric Objects'']{pdfspec-iso32000-1}.
%
% \begin{declcs}{thepdfnumberNormZeroOne}
% \end{declcs}
% There are various places in the PDF specification where the
% number is in the domain 0.0 upto 1.0. Macro thepdfnumberNormZeroOne
% automatically adjusts the number to fit into that range. Negative
% numbers are mapped to |0| and numbers greater than one are
% replaced by |1|. Thus the result fits one of the following regular
% expressions:
% \begin{itemize}
% \item |^0$|
% \item |^\.[0-9]*[1-9]$|
% \item |^1$|
% \end{itemize}
% Examples:
% \begin{quote}
% \begin{tabular}{@{}>{\ttfamily}l@{ $\rightarrow$\ }>{\ttfamily}l@{}}
%   -456 & 0\\
%   -0.001 & 0\\
%   0.0 & 0\\
%   0.010 & .01\\
%   0.456 & .456\\
%   1.0 & 1\\
%   01.001 & 1\\
%   4 & 1\\
% \end{tabular}
% \end{quote}
%
% \subsection{Input number}
%
% The user macros expect a number as argument. The number can either
% be given explicitely or as macro that expands in one step
% to an explicite number, because the first token of the argument
% is expanded once.
%
% The explicite number consists of
% \begin{itemize}
% \item optional signs `\texttt{+}' and `\texttt{-}',
% \item digits `\texttt{0}' upto `\texttt{9}' and
% \item an optional dot `\texttt{.}'.
% \end{itemize}
% All tokens must have catcode 12 (other), the default catcodes
% for these characters in \hologo{LaTeX}, \hologo{plainTeX} or
% \hologo{iniTeX}. As Perl regular expression the number is expected
% in one of the following forms:
% \begin{itemize}
% \item |^[+-]*[0-9]+$|
% \item |^[+-]*[0-9]*\.[0-9]*$|
% \end{itemize}
% At least one digit or the dot must be present.
%
% \subsection{Error handling}
%
% The package is not intended for validating numbers or to
% decide if an argument is a number. Therefore it is an usage
% error to use the user macros with arguments that are not
% explicite numbers as specified in the previous sections.
% Nevertheless some error conditions are sometimes recognized.
% Errors are given in form of an undefined command sequence.
% It is the only way to notify \hologo{TeX} in expandable context.
% Expanding to some error text would invalidate the output.
% Currently the following errors are thrown:
% \begin{description}
% \item[\cs{thepdfnumber@ErrorEndMarker}:]
%   Internally the argument parsing uses an end marker that
%   is never called directly. If it is called with valid
%   user input, then this is a bug. Otherwise it means the user input
%   contains nasty stuff.
% \item[\cs{thepdfnumber@ErrorUnexpectedEnd}:] The macros expect at least one digit
%   or the dot, otherwise if the argument is empty or only contains
%   signs, then this error is called.
% \item[\cs{thepdfnumber@ErrorInvalidToken}:] It is called if the number
%   contains other tokens than signs, digits or the dot or the
%   token at the wrong place (e.g. a sign after a digit).
%   In case of \cs{thepdfnumberNormZeroOne} this error condition might
%   not always be detected, because the number parsing might stop
%   at an early point, when the result is already clear (e.g.
%   if the number is negative or will be greater than one).
% \item[Improper alphabetic constant:] This error might be
%   thrown by \hologo{TeX}, if the number contains
%   command tokens instead of characters.
% \end{description}
%
% \StopEventually{
% }
%
% \section{Implementation}
%
%    \begin{macrocode}
%<*package>
%    \end{macrocode}
%
% \subsection{Reload check and package identification}
%    Reload check, especially if the package is not used with \LaTeX.
%    \begin{macrocode}
\begingroup\catcode61\catcode48\catcode32=10\relax%
  \catcode13=5 % ^^M
  \endlinechar=13 %
  \catcode35=6 % #
  \catcode39=12 % '
  \catcode44=12 % ,
  \catcode45=12 % -
  \catcode46=12 % .
  \catcode58=12 % :
  \catcode64=11 % @
  \catcode123=1 % {
  \catcode125=2 % }
  \expandafter\let\expandafter\x\csname ver@thepdfnumber.sty\endcsname
  \ifx\x\relax % plain-TeX, first loading
  \else
    \def\empty{}%
    \ifx\x\empty % LaTeX, first loading,
      % variable is initialized, but \ProvidesPackage not yet seen
    \else
      \expandafter\ifx\csname PackageInfo\endcsname\relax
        \def\x#1#2{%
          \immediate\write-1{Package #1 Info: #2.}%
        }%
      \else
        \def\x#1#2{\PackageInfo{#1}{#2, stopped}}%
      \fi
      \x{thepdfnumber}{The package is already loaded}%
      \aftergroup\endinput
    \fi
  \fi
\endgroup%
%    \end{macrocode}
%    Package identification:
%    \begin{macrocode}
\begingroup\catcode61\catcode48\catcode32=10\relax%
  \catcode13=5 % ^^M
  \endlinechar=13 %
  \catcode35=6 % #
  \catcode39=12 % '
  \catcode40=12 % (
  \catcode41=12 % )
  \catcode44=12 % ,
  \catcode45=12 % -
  \catcode46=12 % .
  \catcode47=12 % /
  \catcode58=12 % :
  \catcode64=11 % @
  \catcode91=12 % [
  \catcode93=12 % ]
  \catcode123=1 % {
  \catcode125=2 % }
  \expandafter\ifx\csname ProvidesPackage\endcsname\relax
    \def\x#1#2#3[#4]{\endgroup
      \immediate\write-1{Package: #3 #4}%
      \xdef#1{#4}%
    }%
  \else
    \def\x#1#2[#3]{\endgroup
      #2[{#3}]%
      \ifx#1\@undefined
        \xdef#1{#3}%
      \fi
      \ifx#1\relax
        \xdef#1{#3}%
      \fi
    }%
  \fi
\expandafter\x\csname ver@thepdfnumber.sty\endcsname
\ProvidesPackage{thepdfnumber}%
  [2016/05/16 v1.1 Print PDF numbers with minimal digits (HO)]%
%    \end{macrocode}
%
% \subsection{Catcodes}
%
%    \begin{macrocode}
\begingroup\catcode61\catcode48\catcode32=10\relax%
  \catcode13=5 % ^^M
  \endlinechar=13 %
  \catcode123=1 % {
  \catcode125=2 % }
  \catcode64=11 % @
  \def\x{\endgroup
    \expandafter\edef\csname ThPdNu@AtEnd\endcsname{%
      \endlinechar=\the\endlinechar\relax
      \catcode13=\the\catcode13\relax
      \catcode32=\the\catcode32\relax
      \catcode35=\the\catcode35\relax
      \catcode61=\the\catcode61\relax
      \catcode64=\the\catcode64\relax
      \catcode123=\the\catcode123\relax
      \catcode125=\the\catcode125\relax
    }%
  }%
\x\catcode61\catcode48\catcode32=10\relax%
\catcode13=5 % ^^M
\endlinechar=13 %
\catcode35=6 % #
\catcode64=11 % @
\catcode123=1 % {
\catcode125=2 % }
\def\TMP@EnsureCode#1#2{%
  \edef\ThPdNu@AtEnd{%
    \ThPdNu@AtEnd
    \catcode#1=\the\catcode#1\relax
  }%
  \catcode#1=#2\relax
}
\TMP@EnsureCode{33}{12}% !
\TMP@EnsureCode{36}{3}% $
\TMP@EnsureCode{38}{4}% &
\TMP@EnsureCode{42}{12}% *
\TMP@EnsureCode{43}{12}% +
\TMP@EnsureCode{45}{12}% -
\TMP@EnsureCode{46}{12}% .
\TMP@EnsureCode{60}{12}% <
\TMP@EnsureCode{62}{12}% >
\TMP@EnsureCode{96}{12}% `
\edef\ThPdNu@AtEnd{\ThPdNu@AtEnd\noexpand\endinput}
%    \end{macrocode}
%
% \subsection{Helper macros}
%
%    \begin{macro}{\ThPdNu@FIN}
%    \begin{macrocode}
\def\ThPdNu@FIN{\thepdfnumber@ErrorEndMarker}
%    \end{macrocode}
%    \end{macro}
%
%    \begin{macro}{\ThPdNu@space}
%    \begin{macrocode}
\def\ThPdNu@space{ }
%    \end{macrocode}
%    \end{macro}
%
%    \begin{macro}{\ThPdNu@zero}
%    \begin{macrocode}
\chardef\ThPdNu@zero=0 %
%    \end{macrocode}
%    \end{macro}
%
%    \begin{macro}{\ThPdNu@one}
%    \begin{macrocode}
\chardef\ThPdNu@one=1 %
%    \end{macrocode}
%    \end{macro}
%
%    \begin{macro}{\ThPdNu@firstoftwo}
%    \begin{macrocode}
\long\def\ThPdNu@firstoftwo#1#2{#1}
%    \end{macrocode}
%    \end{macro}
%
%    \begin{macro}{\ThPdNu@secondoftwo}
%    \begin{macrocode}
\long\def\ThPdNu@secondoftwo#1#2{#2}
%    \end{macrocode}
%    \end{macro}
%
% \subsection{Detect \hologo{eTeX}}
%
%    \begin{macrocode}
\begingroup\expandafter\expandafter\expandafter\endgroup
\expandafter\ifx\csname detokenize\endcsname\relax
  \catcode`\&=14 %
  \catcode`\$=9 %
\else
  \catcode`\&=9 %
  \catcode`\$=14 %
\fi
%    \end{macrocode}
%
% \subsection{User macro \cs{thepdfnumber}}
%
%    \begin{macro}{\thepdfnumber}
%    \begin{macrocode}
\def\thepdfnumber#1{%
  \romannumeral
& \iftrue\expandafter\ThPdNu@State@Plus\expandafter\fi
& \detokenize\expandafter{#1}%
& \ThPdNu@FIN
$ \ifx\ThPdNu@FIN#1\ThPdNu@FIN
$   \expandafter\ThPdNu@firstoftwo
$ \else
$   \expandafter\ThPdNu@secondoftwo
$ \fi
$ {%
$   \ThPdNu@zero
$   0\thepdfnumber@ErrorUnexpectedEnd
$ }{%
$   \iftrue\expandafter\ThPdNu@State@Plus\expandafter\fi#1\ThPdNu@FIN
$ }%
}
%    \end{macrocode}
%    \end{macro}
%
% \subsubsection{State definitions for sign}
%
%    \begin{macro}{\ThPdNu@State@Plus}
%    \begin{macrocode}
\def\ThPdNu@State@Plus#1\fi#2{%
  \fi
  \ifcase\ifx\ThPdNu@FIN#2%
           0%
&        \else\ifx-#2%
$        \else\ifnum`#2=45 % -
           1%
         \else\ifx0#2%
           2%
         \else\ifnum`#2>48 %
           \ifnum`#2<58 %
             3%
           \else
             9%
           \fi
&        \else\ifx.#2%
$        \else\ifnum`#2=46 % .
           4%
&        \else\ifx+#2%
$        \else\ifnum`#2=43 % +
           5%
         \else
           9%
         \fi\fi\fi\fi\fi\fi\ThPdNu@space
    \expandafter\ThPdNu@zero
    \expandafter0%
    \expandafter\thepdfnumber@ErrorUnexpectedEnd
  \or
    \ThPdNu@State@Minus
  \or
    \ThPdNu@State@SkipZeros!%
  \or
    \ThPdNu@State@Int!#2!%
  \or
    \ThPdNu@State@Dot!\ThPdNu@zero*\ThPdNu@zero!!%
  \or
    \ThPdNu@State@Plus
  \else
    \ThPdNu@ReturnError{0}%
  \fi
}
%    \end{macrocode}
%    \end{macro}
%
%    \begin{macro}{\ThPdNu@State@Minus}
%    \begin{macrocode}
\def\ThPdNu@State@Minus#1\fi#2{%
  \fi
  \ifcase\ifx\ThPdNu@FIN#2%
           0%
         \else\ifx0#2%
           1%
         \else\ifnum`#2>48 %
           \ifnum`#2<58 %
             2%
           \else
             9%
           \fi
&        \else\ifx.#2%
$        \else\ifnum`#2=46 % .
           3%
&        \else\ifx-#2%
$        \else\ifnum`#2=45 % -
           4%
&        \else\ifx+#2%
$        \else\ifnum`#2=43 % +
           5%
         \else
           9%
         \fi\fi\fi\fi\fi\fi\ThPdNu@space
    \expandafter\ThPdNu@zero
    \expandafter0%
    \expandafter\thepdfnumber@ErrorUnexpectedEnd
  \or
    \ThPdNu@State@SkipZeros-!%
  \or
    \ThPdNu@State@Int-!#2!%
  \or
    \ThPdNu@State@Dot-!\ThPdNu@zero*\ThPdNu@zero!!%
  \or
    \ThPdNu@State@Plus
  \or
    \ThPdNu@State@Minus
  \else
    \ThPdNu@ReturnError{0}%
  \fi
}
%    \end{macrocode}
%    \end{macro}
%
%    \begin{macro}{\ThPdNu@ReturnError}
%    \begin{macrocode}
\def\ThPdNu@ReturnError#1#2\fi#3\ThPdNu@FIN{%
  \fi
  \ThPdNu@zero
  #1%
  \thepdfnumber@ErrorInvalidToken
}
%    \end{macrocode}
%    \end{macro}
%
% \subsubsection{State definitions for integer part}
%
%    \begin{macro}{\ThPdNu@State@SkipZeros}
%    \begin{macrocode}
\def\ThPdNu@State@SkipZeros#1!#2\fi#3{%
  \fi
  \ifcase\ifx\ThPdNu@FIN#3%
           0%
         \else\ifx0#3%
           1%
         \else\ifnum`#3>48 %
           \ifnum`#3<58 %
             2%
           \else
             9%
           \fi
&        \else\ifx.#3%
$        \else\ifnum`#3=46 % .
           3%
         \else
           9%
         \fi\fi\fi\fi\ThPdNu@space
    \expandafter\ThPdNu@zero
    \expandafter0%
  \or
   \ThPdNu@State@SkipZeros#1!%
  \or
    \ThPdNu@State@Int#1!#3!%
  \or
    \ThPdNu@State@Dot#1!\ThPdNu@zero*\ThPdNu@zero!!%
  \else
    \ThPdNu@ReturnError{0}%
  \fi
}
%    \end{macrocode}
%    \end{macro}
%
%    \begin{macro}{\ThPdNu@State@Int}
%    \begin{macrocode}
\def\ThPdNu@State@Int#1!#2!#3\fi#4{%
  \fi
  \ifcase\ifx\ThPdNu@FIN#4%
           0%
         \else\ifnum`#4>47 %
           \ifnum`#4<58 %
             1%
           \else
             9%
           \fi
&        \else\ifx.#4%
$        \else\ifnum`#4=46 % .
           2%
         \else
           9%
         \fi\fi\fi\ThPdNu@space
    \ThPdNu@ReturnInt{#1#2}%
  \or
    \ThPdNu@State@Int#1!#2#4!%
  \or
    \ThPdNu@State@Dot#1!\ThPdNu@one#2*\ThPdNu@zero!!%
  \else
    \ThPdNu@ReturnError{#1#2}%
  \fi
}
%    \end{macrocode}
%    \end{macro}
%
%    \begin{macro}{\ThPdNu@ReturnInt}
%    \begin{macrocode}
\def\ThPdNu@ReturnInt#1#2\fi{%
  \fi
  \ThPdNu@zero
  #1%
}
%    \end{macrocode}
%    \end{macro}
%
% \subsubsection{State definitions for decimal digits}
%
%    \begin{macro}{\ThPdNu@State@Dot}
%    \begin{macrocode}
\def\ThPdNu@State@Dot#1*#2#3!#4!#5\fi#6{%
  \fi
  \ifcase\ifx\ThPdNu@FIN#6%
           0%
         \else\ifnum`#6>48 %
           \ifnum`#6<58 %
             1%
           \else
             9%
           \fi
         \else\ifx0#6%
           2%
         \else
           9%
         \fi\fi\fi\ThPdNu@space
    \ThPdNu@ReturnNumber#1*#2#3!%
  \or
    \ThPdNu@State@Dot#1*\ThPdNu@one#3#4#6!!%
  \or
    \ThPdNu@State@DotZero#1*#2#3!#4#6!%
  \else
    \ThPdNu@ReturnNumberInvalid#1*#2#3!%
  \fi
}
%    \end{macrocode}
%    \end{macro}
%
%    \begin{macro}{\ThPdNu@State@DotZero}
%    \begin{macrocode}
\def\ThPdNu@State@DotZero#1*#2#3!#4!#5\fi#6{%
  \fi
  \ifcase\ifx\ThPdNu@FIN#6%
           0%
         \else\ifnum`#6>48 %
           \ifnum`#6<58 %
             1%
           \else
             9%
           \fi
         \else\ifx0#6%
           2%
         \else
           9%
         \fi\fi\fi\ThPdNu@space
    \ThPdNu@ReturnNumber#1*#2#3!%
  \or
    \ThPdNu@State@Dot#1*\ThPdNu@one#3#4#6!!%
  \or
    \ThPdNu@State@DotZero#1*#2#3!#4#6!%
  \else
    \ThPdNu@ReturnNumber#1*#2#3!%
  \fi
}
%    \end{macrocode}
%    \end{macro}
%
%    \begin{macro}{\ThPdNu@ReturnNumber}
%    \begin{macrocode}
\def\ThPdNu@ReturnNumber#1!#2#3*#4#5!#6\fi{%
  \fi
  \ifcase#2%
    \expandafter\ThPdNu@firstoftwo
  \else
    \expandafter\ThPdNu@secondoftwo
  \fi
  {%
    \ifcase#4%
      \expandafter\ThPdNu@firstoftwo
    \else
      \expandafter\ThPdNu@secondoftwo
    \fi
    {\ThPdNu@zero 0}%
    {\ThPdNu@zero #1.#5}%
  }{%
    \ifcase#4%
      \expandafter\ThPdNu@firstoftwo
    \else
      \expandafter\ThPdNu@secondoftwo
    \fi
    {\ThPdNu@zero #1#3}%
    {\ThPdNu@zero #1#3.#5}%
  }%
}
%    \end{macrocode}
%    \end{macro}
%
%    \begin{macro}{\ThPdNu@ReturnNumberInvalid}
%    \begin{macrocode}
\def\ThPdNu@ReturnNumberInvalid#1*#2!#3\fi#4\ThPdNu@FIN{%
  \fi
  \iftrue\ThPdNu@ReturnNumber#1*#2!\fi
  \thepdfnumber@ErrorInvalidToken
}
%    \end{macrocode}
%    \end{macro}
%
% \subsection{Norm macro}
%
%    \begin{macro}{\thepdfnumberNormZeroOne}
%    \begin{macrocode}
\def\thepdfnumberNormZeroOne#1{%
  \romannumeral
& \iftrue\expandafter\ThPbNu@StateN@Plus\expandafter\fi
& \detokenize\expandafter{#1}%
& \ThPdNu@FIN
$ \ifx\ThPdNu@FIN#1\ThPdNu@FIN
$   \expandafter\ThPdNu@firstoftwo
$ \else
$   \expandafter\ThPdNu@secondoftwo
$ \fi
$ {%
$   \ThPdNu@zero
$   0\thepdfnumber@ErrorUnexpectedEnd
$ }{%
$   \iftrue\expandafter\ThPbNu@StateN@Plus\expandafter\fi#1\ThPdNu@FIN
$ }%
}
%    \end{macrocode}
%    \end{macro}
%
% \subsubsection{State definitions for sign}
%
%    \begin{macro}{\ThPbNu@StateN@Plus}
%    \begin{macrocode}
\def\ThPbNu@StateN@Plus#1\fi#2{%
  \fi
  \ifcase\ifx\ThPdNu@FIN#2%
           0%
&        \else\ifx-#2%
$        \else\ifnum`#2=45 % -
           1%
         \else\ifx0#2%
           2%
         \else\ifnum`#2>48 %
           \ifnum`#2<58 %
             3%
           \else
             9%
           \fi
&        \else\ifx.#2%
$        \else\ifnum`#2=46 % .
           4%
&        \else\ifx+#2%
$        \else\ifnum`#2=43 % +
           5%
         \else
           9%
         \fi\fi\fi\fi\fi\fi\ThPdNu@space
    \expandafter\ThPdNu@zero
    \expandafter0%
    \expandafter\thepdfnumber@ErrorUnexpectedEnd
  \or
    \ThPbNu@StateN@Minus
  \or
    \ThPbNu@StateN@SkipZeros
  \or
    \ThPdNu@ReturnAndSkip{1}%
  \or
    \ThPbNu@StateN@Dot\ThPdNu@zero!!%
  \or
    \ThPbNu@StateN@Plus
  \else
    \ThPdNu@ReturnError{0}%
  \fi
}
%    \end{macrocode}
%    \end{macro}
%
%    \begin{macro}{\ThPbNu@StateN@Minus}
%    \begin{macrocode}
\def\ThPbNu@StateN@Minus#1\fi#2{%
  \fi
  \ifcase\ifx\ThPdNu@FIN#2%
           0%
         \else\ifnum`#2>47 %
           \ifnum`#2<58 %
             1%
           \else
             9%
           \fi
&        \else\ifx.#2%
$        \else\ifnum`#2=46 % .
           1%
&        \else\ifx-#2%
$        \else\ifnum`#2=45 % -
           2%
&        \else\ifx+#2%
$        \else\ifnum`#2=43 % +
           3%
         \else
           9%
         \fi\fi\fi\fi\fi\ThPdNu@space
    \expandafter\ThPdNu@zero
    \expandafter0%
    \expandafter\thepdfnumber@ErrorUnexpectedEnd
  \or
    \ThPdNu@ReturnAndSkip{0}%
  \or
    \ThPbNu@StateN@Plus
  \or
    \ThPbNu@StateN@Minus
  \else
    \ThPdNu@ReturnError{0}%
  \fi
}
%    \end{macrocode}
%    \end{macro}
%
%    \begin{macro}{\ThPdNu@ReturnAndSkip}
%    \begin{macrocode}
\def\ThPdNu@ReturnAndSkip#1#2\fi#3\ThPdNu@FIN{%
  \fi
  \ThPdNu@zero
  #1%
}
%    \end{macrocode}
%    \end{macro}
%
% \subsubsection{State definitions for integer part}
%
%    \begin{macro}{\ThPbNu@StateN@SkipZeros}
%    \begin{macrocode}
\def\ThPbNu@StateN@SkipZeros#1\fi#2{%
  \fi
  \ifcase\ifx\ThPdNu@FIN#2%
           0%
         \else\ifx0#2%
           1%
         \else\ifnum`#2>48 %
           \ifnum`#2<58 %
             2%
           \else
             9%
           \fi
&        \else\ifx.#2%
$        \else\ifnum`#2=46 % .
           3%
         \else
           9%
         \fi\fi\fi\fi\ThPdNu@space
    \expandafter\ThPdNu@zero
    \expandafter0%
  \or
   \ThPbNu@StateN@SkipZeros%
  \or
    \ThPdNu@ReturnAndSkip{1}%
  \or
    \ThPbNu@StateN@Dot\ThPdNu@zero!!%
  \else
    \ThPdNu@ReturnError{0}%
  \fi
}
%    \end{macrocode}
%    \end{macro}
%
% \subsubsection{State definitions for decimal digits}
%
%    \begin{macro}{\ThPbNu@StateN@Dot}
%    \begin{macrocode}
\def\ThPbNu@StateN@Dot#1#2!#3!#4\fi#5{%
  \fi
  \ifcase\ifx\ThPdNu@FIN#5%
           0%
         \else\ifnum`#5>48 %
           \ifnum`#5<58 %
             1%
           \else
             9%
           \fi
         \else\ifx0#5%
           2%
         \else
           9%
         \fi\fi\fi\ThPdNu@space
    \ThPdNu@ReturnFracNumber#1#2!%
  \or
    \ThPbNu@StateN@Dot\ThPdNu@one#2#3#5!!%
  \or
    \ThPbNu@StateN@DotZero#1#2!#3#5!%
  \else
    \ThPdNu@ReturnFracNumberInvalid#1#2!%
  \fi
}
%    \end{macrocode}
%    \end{macro}
%
%    \begin{macro}{\ThPbNu@StateN@DotZero}
%    \begin{macrocode}
\def\ThPbNu@StateN@DotZero#1#2!#3!#4\fi#5{%
  \fi
  \ifcase\ifx\ThPdNu@FIN#5%
           0%
         \else\ifnum`#5>48 %
           \ifnum`#5<58 %
             1%
           \else
             9%
           \fi
         \else\ifx0#5%
           2%
         \else
           9%
         \fi\fi\fi\ThPdNu@space
    \ThPdNu@ReturnFracNumber#1#2!%
  \or
    \ThPbNu@StateN@Dot\ThPdNu@one#2#3#5!!%
  \or
    \ThPbNu@StateN@DotZero#1#2!#3#5!%
  \else
    \ThPdNu@ReturnFracNumberInvalid#1#2!%
  \fi
}
%    \end{macrocode}
%    \end{macro}
%
%    \begin{macro}{\ThPdNu@ReturnFracNumber}
%    \begin{macrocode}
\def\ThPdNu@ReturnFracNumber#1#2!#3\fi{%
  \fi
  \ifcase#1%
    \expandafter\ThPdNu@firstoftwo
  \else
    \expandafter\ThPdNu@secondoftwo
  \fi
  {\ThPdNu@zero 0}%
  {\ThPdNu@zero .#2}%
}
%    \end{macrocode}
%    \end{macro}
%
%    \begin{macro}{\ThPdNu@ReturnFracNumberInvalid}
%    \begin{macrocode}
\def\ThPdNu@ReturnFracNumberInvalid#1!#2\fi#3\ThPdNu@FIN{%
  \fi
  \iftrue\ThPdNu@ReturnFracNumber#1!\fi
  \thepdfnumber@ErrorInvalidToken
}
%    \end{macrocode}
%    \end{macro}
%
%    \begin{macrocode}
\ThPdNu@AtEnd%
%</package>
%    \end{macrocode}
%
% \section{Test}
%
% \subsection{Catcode checks for loading}
%
%    \begin{macrocode}
%<*test1>
%    \end{macrocode}
%    \begin{macrocode}
\catcode`\{=1 %
\catcode`\}=2 %
\catcode`\#=6 %
\catcode`\@=11 %
\expandafter\ifx\csname count@\endcsname\relax
  \countdef\count@=255 %
\fi
\expandafter\ifx\csname @gobble\endcsname\relax
  \long\def\@gobble#1{}%
\fi
\expandafter\ifx\csname @firstofone\endcsname\relax
  \long\def\@firstofone#1{#1}%
\fi
\expandafter\ifx\csname loop\endcsname\relax
  \expandafter\@firstofone
\else
  \expandafter\@gobble
\fi
{%
  \def\loop#1\repeat{%
    \def\body{#1}%
    \iterate
  }%
  \def\iterate{%
    \body
      \let\next\iterate
    \else
      \let\next\relax
    \fi
    \next
  }%
  \let\repeat=\fi
}%
\def\RestoreCatcodes{}
\count@=0 %
\loop
  \edef\RestoreCatcodes{%
    \RestoreCatcodes
    \catcode\the\count@=\the\catcode\count@\relax
  }%
\ifnum\count@<255 %
  \advance\count@ 1 %
\repeat

\def\RangeCatcodeInvalid#1#2{%
  \count@=#1\relax
  \loop
    \catcode\count@=15 %
  \ifnum\count@<#2\relax
    \advance\count@ 1 %
  \repeat
}
\def\RangeCatcodeCheck#1#2#3{%
  \count@=#1\relax
  \loop
    \ifnum#3=\catcode\count@
    \else
      \errmessage{%
        Character \the\count@\space
        with wrong catcode \the\catcode\count@\space
        instead of \number#3%
      }%
    \fi
  \ifnum\count@<#2\relax
    \advance\count@ 1 %
  \repeat
}
\def\space{ }
\expandafter\ifx\csname LoadCommand\endcsname\relax
  \def\LoadCommand{\input thepdfnumber.sty\relax}%
\fi
\def\Test{%
  \RangeCatcodeInvalid{0}{47}%
  \RangeCatcodeInvalid{58}{64}%
  \RangeCatcodeInvalid{91}{96}%
  \RangeCatcodeInvalid{123}{255}%
  \catcode`\@=12 %
  \catcode`\\=0 %
  \catcode`\%=14 %
  \LoadCommand
  \RangeCatcodeCheck{0}{36}{15}%
  \RangeCatcodeCheck{37}{37}{14}%
  \RangeCatcodeCheck{38}{47}{15}%
  \RangeCatcodeCheck{48}{57}{12}%
  \RangeCatcodeCheck{58}{63}{15}%
  \RangeCatcodeCheck{64}{64}{12}%
  \RangeCatcodeCheck{65}{90}{11}%
  \RangeCatcodeCheck{91}{91}{15}%
  \RangeCatcodeCheck{92}{92}{0}%
  \RangeCatcodeCheck{93}{96}{15}%
  \RangeCatcodeCheck{97}{122}{11}%
  \RangeCatcodeCheck{123}{255}{15}%
  \RestoreCatcodes
}
\Test
\csname @@end\endcsname
\end
%    \end{macrocode}
%    \begin{macrocode}
%</test1>
%    \end{macrocode}
%
% \subsection{Macro tests}
%
% \subsubsection{Preamble}
%
%    \begin{macrocode}
%<*test2>
\catcode`\{=1
\catcode`\}=2
\catcode`\#=6
\catcode`\@=11
\errorcontextlines=10000 %
\def\msg#{\immediate\write16}
\def\space{ }
\begingroup\expandafter\expandafter\expandafter\endgroup
\expandafter\ifx\csname RequirePackage\endcsname\relax
  \input thepdfnumber.sty\relax
\else
  \RequirePackage{thepdfnumber}[2016/05/16]%
\fi
%    \end{macrocode}
%
%    \begin{macro}{\Test}
%    \begin{macrocode}
\def\Test#1#2{%
  \def\TestExpected{#2}%
%<*active>
  \edef\TestExpected{\TestExpected}%
%</active>
  \expandafter\expandafter\expandafter\def
  \expandafter\expandafter\expandafter\TestResult
  \expandafter\expandafter\expandafter{%
    \TestCommand{#1}%
  }%
  \begingroup
    \toks0\expandafter{\TestResult}%
    \toks2\expandafter{\TestExpected}%
    \ifx\TestResult\TestExpected
      \msg{* OK: #1 => \the\toks0}%
    \else
      \errmessage{FAILED: #1 => \the\toks0\space(\the\toks2)}%
    \fi
  \endgroup
}
%    \end{macrocode}
%    \end{macro}
%
%    \begin{macrocode}
\let\thepdfnumber@ErrorUnexpectedEnd\relax
\let\thepdfnumber@ErrorInvalidToken\relax
%    \end{macrocode}
%
% \subsubsection{Test with active catcodes}
%
%    \begin{macrocode}
%<*active>
\catcode`\~=13 %
\def\ThPdNu@Temp#1{%
  \begingroup
    \lccode`\~=`#1 %
  \lowercase{\endgroup
    \edef~%
  }{\string#1}%
  \catcode`#1=13 %
}
\ThPdNu@Temp{.}
\ThPdNu@Temp{-}
\ThPdNu@Temp{+}
%</active>
%    \end{macrocode}
%
% \subsubsection{Test cases for \cs{thepdfnumber}}
%
%    \begin{macro}{\TestCommand}
%    \begin{macrocode}
\let\TestCommand\thepdfnumber
%    \end{macrocode}
%    \end{macro}
%
%    \begin{macrocode}
\Test{0}{0}
\Test{1}{1}
\Test{23}{23}
\Test{0045}{45}
\Test{4500}{4500}
\Test{-0}{0}
\Test{-1}{-1}
\Test{-00002}{-2}
\Test{-010203}{-10203}
\Test{+0}{0}
\Test{+123}{123}

\Test{+-4}{-4}
\Test{+-+-4}{4}
\Test{++--++--++4}{4}
\Test{+++---+++---+++---4}{-4}

\Test{.}{0}
\Test{0.}{0}
\Test{.0}{0}
\Test{0.}{0}
\Test{-.}{0}
\Test{-0.}{0}
\Test{-.0}{0}
\Test{-0.}{0}
\Test{010.020}{10.02}
\Test{123.456}{123.456}
\Test{12.}{12}
\Test{.123}{.123}
\Test{.0001}{.0001}
\Test{.00100}{.001}
\Test{.12003400560078009}{.12003400560078009}
\Test{-.12003400560078009}{-.12003400560078009}
\Test{04.0500000}{4.05}

\Test{}{0\thepdfnumber@ErrorUnexpectedEnd}
\Test{+}{0\thepdfnumber@ErrorUnexpectedEnd}
\Test{-}{0\thepdfnumber@ErrorUnexpectedEnd}
\Test{a}{0\thepdfnumber@ErrorInvalidToken}
\Test{0x1}{0\thepdfnumber@ErrorInvalidToken}
\Test{4x56}{4\thepdfnumber@ErrorInvalidToken}
\Test{012,34}{12\thepdfnumber@ErrorInvalidToken}
\Test{0.12x4}{.12\thepdfnumber@ErrorInvalidToken}
%    \end{macrocode}
%
% \subsubsection{Test cases for \cs{thepdfnumberNormZeroOne}}
%
%    \begin{macro}{\TestCommand}
%    \begin{macrocode}
\let\TestCommand\thepdfnumberNormZeroOne
%    \end{macrocode}
%    \end{macro}
%
%    \begin{macrocode}
\Test{0}{0}
\Test{1}{1}
\Test{23}{1}
\Test{12}{1}
\Test{10}{1}
\Test{0045}{1}
\Test{001}{1}
\Test{-0}{0}
\Test{-1}{0}
\Test{-01}{0}
\Test{-4}{0}

\Test{+-0}{0}
\Test{+-+-1}{1}
\Test{++--++--++1}{1}
\Test{+++---+++---+++---1}{0}

\Test{.}{0}
\Test{0.}{0}
\Test{.0}{0}
\Test{0.}{0}
\Test{-.}{0}
\Test{-0.}{0}
\Test{-.0}{0}
\Test{-0.}{0}
\Test{010.020}{1}
\Test{123.456}{1}
\Test{12.}{1}
\Test{.123}{.123}
\Test{.0001}{.0001}
\Test{.00100}{.001}
\Test{.12003400560078009}{.12003400560078009}
\Test{-.12003400560078009}{0}
\Test{04.0500000}{1}
\Test{0.1200340056}{.1200340056}
\Test{1.05}{1}

\Test{}{0\thepdfnumber@ErrorUnexpectedEnd}
\Test{+}{0\thepdfnumber@ErrorUnexpectedEnd}
\Test{-}{0\thepdfnumber@ErrorUnexpectedEnd}
\Test{a}{0\thepdfnumber@ErrorInvalidToken}
\Test{0x1}{0\thepdfnumber@ErrorInvalidToken}
\Test{4x56}{1}
\Test{012,34}{1}
\Test{.012x4}{.012\thepdfnumber@ErrorInvalidToken}
%    \end{macrocode}
%
%    \begin{macrocode}
\csname @@end\endcsname\end
%</test2>
%    \end{macrocode}
%
% \section{Installation}
%
% \subsection{Download}
%
% \paragraph{Package.} This package is available on
% CTAN\footnote{\url{http://ctan.org/pkg/thepdfnumber}}:
% \begin{description}
% \item[\CTAN{macros/latex/contrib/oberdiek/thepdfnumber.dtx}] The source file.
% \item[\CTAN{macros/latex/contrib/oberdiek/thepdfnumber.pdf}] Documentation.
% \end{description}
%
%
% \paragraph{Bundle.} All the packages of the bundle `oberdiek'
% are also available in a TDS compliant ZIP archive. There
% the packages are already unpacked and the documentation files
% are generated. The files and directories obey the TDS standard.
% \begin{description}
% \item[\CTAN{install/macros/latex/contrib/oberdiek.tds.zip}]
% \end{description}
% \emph{TDS} refers to the standard ``A Directory Structure
% for \TeX\ Files'' (\CTAN{tds/tds.pdf}). Directories
% with \xfile{texmf} in their name are usually organized this way.
%
% \subsection{Bundle installation}
%
% \paragraph{Unpacking.} Unpack the \xfile{oberdiek.tds.zip} in the
% TDS tree (also known as \xfile{texmf} tree) of your choice.
% Example (linux):
% \begin{quote}
%   |unzip oberdiek.tds.zip -d ~/texmf|
% \end{quote}
%
% \paragraph{Script installation.}
% Check the directory \xfile{TDS:scripts/oberdiek/} for
% scripts that need further installation steps.
% Package \xpackage{attachfile2} comes with the Perl script
% \xfile{pdfatfi.pl} that should be installed in such a way
% that it can be called as \texttt{pdfatfi}.
% Example (linux):
% \begin{quote}
%   |chmod +x scripts/oberdiek/pdfatfi.pl|\\
%   |cp scripts/oberdiek/pdfatfi.pl /usr/local/bin/|
% \end{quote}
%
% \subsection{Package installation}
%
% \paragraph{Unpacking.} The \xfile{.dtx} file is a self-extracting
% \docstrip\ archive. The files are extracted by running the
% \xfile{.dtx} through \plainTeX:
% \begin{quote}
%   \verb|tex thepdfnumber.dtx|
% \end{quote}
%
% \paragraph{TDS.} Now the different files must be moved into
% the different directories in your installation TDS tree
% (also known as \xfile{texmf} tree):
% \begin{quote}
% \def\t{^^A
% \begin{tabular}{@{}>{\ttfamily}l@{ $\rightarrow$ }>{\ttfamily}l@{}}
%   thepdfnumber.sty & tex/generic/oberdiek/thepdfnumber.sty\\
%   thepdfnumber.pdf & doc/latex/oberdiek/thepdfnumber.pdf\\
%   test/thepdfnumber-test1.tex & doc/latex/oberdiek/test/thepdfnumber-test1.tex\\
%   test/thepdfnumber-test2.tex & doc/latex/oberdiek/test/thepdfnumber-test2.tex\\
%   test/thepdfnumber-test3.tex & doc/latex/oberdiek/test/thepdfnumber-test3.tex\\
%   thepdfnumber.dtx & source/latex/oberdiek/thepdfnumber.dtx\\
% \end{tabular}^^A
% }^^A
% \sbox0{\t}^^A
% \ifdim\wd0>\linewidth
%   \begingroup
%     \advance\linewidth by\leftmargin
%     \advance\linewidth by\rightmargin
%   \edef\x{\endgroup
%     \def\noexpand\lw{\the\linewidth}^^A
%   }\x
%   \def\lwbox{^^A
%     \leavevmode
%     \hbox to \linewidth{^^A
%       \kern-\leftmargin\relax
%       \hss
%       \usebox0
%       \hss
%       \kern-\rightmargin\relax
%     }^^A
%   }^^A
%   \ifdim\wd0>\lw
%     \sbox0{\small\t}^^A
%     \ifdim\wd0>\linewidth
%       \ifdim\wd0>\lw
%         \sbox0{\footnotesize\t}^^A
%         \ifdim\wd0>\linewidth
%           \ifdim\wd0>\lw
%             \sbox0{\scriptsize\t}^^A
%             \ifdim\wd0>\linewidth
%               \ifdim\wd0>\lw
%                 \sbox0{\tiny\t}^^A
%                 \ifdim\wd0>\linewidth
%                   \lwbox
%                 \else
%                   \usebox0
%                 \fi
%               \else
%                 \lwbox
%               \fi
%             \else
%               \usebox0
%             \fi
%           \else
%             \lwbox
%           \fi
%         \else
%           \usebox0
%         \fi
%       \else
%         \lwbox
%       \fi
%     \else
%       \usebox0
%     \fi
%   \else
%     \lwbox
%   \fi
% \else
%   \usebox0
% \fi
% \end{quote}
% If you have a \xfile{docstrip.cfg} that configures and enables \docstrip's
% TDS installing feature, then some files can already be in the right
% place, see the documentation of \docstrip.
%
% \subsection{Refresh file name databases}
%
% If your \TeX~distribution
% (\teTeX, \mikTeX, \dots) relies on file name databases, you must refresh
% these. For example, \teTeX\ users run \verb|texhash| or
% \verb|mktexlsr|.
%
% \subsection{Some details for the interested}
%
% \paragraph{Attached source.}
%
% The PDF documentation on CTAN also includes the
% \xfile{.dtx} source file. It can be extracted by
% AcrobatReader 6 or higher. Another option is \textsf{pdftk},
% e.g. unpack the file into the current directory:
% \begin{quote}
%   \verb|pdftk thepdfnumber.pdf unpack_files output .|
% \end{quote}
%
% \paragraph{Unpacking with \LaTeX.}
% The \xfile{.dtx} chooses its action depending on the format:
% \begin{description}
% \item[\plainTeX:] Run \docstrip\ and extract the files.
% \item[\LaTeX:] Generate the documentation.
% \end{description}
% If you insist on using \LaTeX\ for \docstrip\ (really,
% \docstrip\ does not need \LaTeX), then inform the autodetect routine
% about your intention:
% \begin{quote}
%   \verb|latex \let\install=y% \iffalse meta-comment
%
% File: thepdfnumber.dtx
% Version: 2016/05/16 v1.1
% Info: Print PDF numbers with minimal digits
%
% Copyright (C) 2011 by
%    Heiko Oberdiek <heiko.oberdiek at googlemail.com>
%    2016
%    https://github.com/ho-tex/oberdiek/issues
%
% This work may be distributed and/or modified under the
% conditions of the LaTeX Project Public License, either
% version 1.3c of this license or (at your option) any later
% version. This version of this license is in
%    http://www.latex-project.org/lppl/lppl-1-3c.txt
% and the latest version of this license is in
%    http://www.latex-project.org/lppl.txt
% and version 1.3 or later is part of all distributions of
% LaTeX version 2005/12/01 or later.
%
% This work has the LPPL maintenance status "maintained".
%
% This Current Maintainer of this work is Heiko Oberdiek.
%
% The Base Interpreter refers to any `TeX-Format',
% because some files are installed in TDS:tex/generic//.
%
% This work consists of the main source file thepdfnumber.dtx
% and the derived files
%    thepdfnumber.sty, thepdfnumber.pdf, thepdfnumber.ins, thepdfnumber.drv,
%    thepdfnumber-test1.tex, thepdfnumber-test2.tex,
%    thepdfnumber-test3.tex.
%
% Distribution:
%    CTAN:macros/latex/contrib/oberdiek/thepdfnumber.dtx
%    CTAN:macros/latex/contrib/oberdiek/thepdfnumber.pdf
%
% Unpacking:
%    (a) If thepdfnumber.ins is present:
%           tex thepdfnumber.ins
%    (b) Without thepdfnumber.ins:
%           tex thepdfnumber.dtx
%    (c) If you insist on using LaTeX
%           latex \let\install=y% \iffalse meta-comment
%
% File: thepdfnumber.dtx
% Version: 2016/05/16 v1.1
% Info: Print PDF numbers with minimal digits
%
% Copyright (C) 2011 by
%    Heiko Oberdiek <heiko.oberdiek at googlemail.com>
%    2016
%    https://github.com/ho-tex/oberdiek/issues
%
% This work may be distributed and/or modified under the
% conditions of the LaTeX Project Public License, either
% version 1.3c of this license or (at your option) any later
% version. This version of this license is in
%    http://www.latex-project.org/lppl/lppl-1-3c.txt
% and the latest version of this license is in
%    http://www.latex-project.org/lppl.txt
% and version 1.3 or later is part of all distributions of
% LaTeX version 2005/12/01 or later.
%
% This work has the LPPL maintenance status "maintained".
%
% This Current Maintainer of this work is Heiko Oberdiek.
%
% The Base Interpreter refers to any `TeX-Format',
% because some files are installed in TDS:tex/generic//.
%
% This work consists of the main source file thepdfnumber.dtx
% and the derived files
%    thepdfnumber.sty, thepdfnumber.pdf, thepdfnumber.ins, thepdfnumber.drv,
%    thepdfnumber-test1.tex, thepdfnumber-test2.tex,
%    thepdfnumber-test3.tex.
%
% Distribution:
%    CTAN:macros/latex/contrib/oberdiek/thepdfnumber.dtx
%    CTAN:macros/latex/contrib/oberdiek/thepdfnumber.pdf
%
% Unpacking:
%    (a) If thepdfnumber.ins is present:
%           tex thepdfnumber.ins
%    (b) Without thepdfnumber.ins:
%           tex thepdfnumber.dtx
%    (c) If you insist on using LaTeX
%           latex \let\install=y\input{thepdfnumber.dtx}
%        (quote the arguments according to the demands of your shell)
%
% Documentation:
%    (a) If thepdfnumber.drv is present:
%           latex thepdfnumber.drv
%    (b) Without thepdfnumber.drv:
%           latex thepdfnumber.dtx; ...
%    The class ltxdoc loads the configuration file ltxdoc.cfg
%    if available. Here you can specify further options, e.g.
%    use A4 as paper format:
%       \PassOptionsToClass{a4paper}{article}
%
%    Programm calls to get the documentation (example):
%       pdflatex thepdfnumber.dtx
%       makeindex -s gind.ist thepdfnumber.idx
%       pdflatex thepdfnumber.dtx
%       makeindex -s gind.ist thepdfnumber.idx
%       pdflatex thepdfnumber.dtx
%
% Installation:
%    TDS:tex/generic/oberdiek/thepdfnumber.sty
%    TDS:doc/latex/oberdiek/thepdfnumber.pdf
%    TDS:doc/latex/oberdiek/test/thepdfnumber-test1.tex
%    TDS:doc/latex/oberdiek/test/thepdfnumber-test2.tex
%    TDS:doc/latex/oberdiek/test/thepdfnumber-test3.tex
%    TDS:source/latex/oberdiek/thepdfnumber.dtx
%
%<*ignore>
\begingroup
  \catcode123=1 %
  \catcode125=2 %
  \def\x{LaTeX2e}%
\expandafter\endgroup
\ifcase 0\ifx\install y1\fi\expandafter
         \ifx\csname processbatchFile\endcsname\relax\else1\fi
         \ifx\fmtname\x\else 1\fi\relax
\else\csname fi\endcsname
%</ignore>
%<*install>
\input docstrip.tex
\Msg{************************************************************************}
\Msg{* Installation}
\Msg{* Package: thepdfnumber 2016/05/16 v1.1 Print PDF numbers with minimal digits (HO)}
\Msg{************************************************************************}

\keepsilent
\askforoverwritefalse

\let\MetaPrefix\relax
\preamble

This is a generated file.

Project: thepdfnumber
Version: 2016/05/16 v1.1

Copyright (C) 2011 by
   Heiko Oberdiek <heiko.oberdiek at googlemail.com>

This work may be distributed and/or modified under the
conditions of the LaTeX Project Public License, either
version 1.3c of this license or (at your option) any later
version. This version of this license is in
   http://www.latex-project.org/lppl/lppl-1-3c.txt
and the latest version of this license is in
   http://www.latex-project.org/lppl.txt
and version 1.3 or later is part of all distributions of
LaTeX version 2005/12/01 or later.

This work has the LPPL maintenance status "maintained".

This Current Maintainer of this work is Heiko Oberdiek.

The Base Interpreter refers to any `TeX-Format',
because some files are installed in TDS:tex/generic//.

This work consists of the main source file thepdfnumber.dtx
and the derived files
   thepdfnumber.sty, thepdfnumber.pdf, thepdfnumber.ins, thepdfnumber.drv,
   thepdfnumber-test1.tex, thepdfnumber-test2.tex,
   thepdfnumber-test3.tex.

\endpreamble
\let\MetaPrefix\DoubleperCent

\generate{%
  \file{thepdfnumber.ins}{\from{thepdfnumber.dtx}{install}}%
  \file{thepdfnumber.drv}{\from{thepdfnumber.dtx}{driver}}%
  \usedir{tex/generic/oberdiek}%
  \file{thepdfnumber.sty}{\from{thepdfnumber.dtx}{package}}%
  \usedir{doc/latex/oberdiek/test}%
  \file{thepdfnumber-test1.tex}{\from{thepdfnumber.dtx}{test1}}%
  \file{thepdfnumber-test2.tex}{\from{thepdfnumber.dtx}{test2}}%
  \file{thepdfnumber-test3.tex}{\from{thepdfnumber.dtx}{test2,active}}%
}

\catcode32=13\relax% active space
\let =\space%
\Msg{************************************************************************}
\Msg{*}
\Msg{* To finish the installation you have to move the following}
\Msg{* file into a directory searched by TeX:}
\Msg{*}
\Msg{*     thepdfnumber.sty}
\Msg{*}
\Msg{* To produce the documentation run the file `thepdfnumber.drv'}
\Msg{* through LaTeX.}
\Msg{*}
\Msg{* Happy TeXing!}
\Msg{*}
\Msg{************************************************************************}

\endbatchfile
%</install>
%<*ignore>
\fi
%</ignore>
%<*driver>
\NeedsTeXFormat{LaTeX2e}
\ProvidesFile{thepdfnumber.drv}%
  [2016/05/16 v1.1 Print PDF numbers with minimal digits (HO)]%
\documentclass{ltxdoc}
\usepackage{holtxdoc}[2011/11/22]
\usepackage{csquotes}
\usepackage[
  backend=bibtex,
  bibencoding=ascii,
  alldates=iso8601,
]{biblatex}[2011/11/13]
\bibliography{oberdiek-source}
\begin{document}
  \DocInput{thepdfnumber.dtx}%
\end{document}
%</driver>
% \fi
%
%
% \CharacterTable
%  {Upper-case    \A\B\C\D\E\F\G\H\I\J\K\L\M\N\O\P\Q\R\S\T\U\V\W\X\Y\Z
%   Lower-case    \a\b\c\d\e\f\g\h\i\j\k\l\m\n\o\p\q\r\s\t\u\v\w\x\y\z
%   Digits        \0\1\2\3\4\5\6\7\8\9
%   Exclamation   \!     Double quote  \"     Hash (number) \#
%   Dollar        \$     Percent       \%     Ampersand     \&
%   Acute accent  \'     Left paren    \(     Right paren   \)
%   Asterisk      \*     Plus          \+     Comma         \,
%   Minus         \-     Point         \.     Solidus       \/
%   Colon         \:     Semicolon     \;     Less than     \<
%   Equals        \=     Greater than  \>     Question mark \?
%   Commercial at \@     Left bracket  \[     Backslash     \\
%   Right bracket \]     Circumflex    \^     Underscore    \_
%   Grave accent  \`     Left brace    \{     Vertical bar  \|
%   Right brace   \}     Tilde         \~}
%
% \GetFileInfo{thepdfnumber.drv}
%
% \title{The \xpackage{thepdfnumber} package}
% \date{2016/05/16 v1.1}
% \author{Heiko Oberdiek\thanks
% {Please report any issues at https://github.com/ho-tex/oberdiek/issues}\\
% \xemail{heiko.oberdiek at googlemail.com}}
%
% \maketitle
%
% \begin{abstract}
% The package converts real numbers to a minimal representation
% that is stripped from leading or trailing zeros,
% plus signs and decimal point if not necessary.
% \end{abstract}
%
% \tableofcontents
%
% \section{Documentation}
%
% \subsection{Introduction}
%
% Dealing with the PDF format, there is sometimes the need to write
% some low level PDF stuff. In case of numbers, the numbers can
% arise from user input (e.g. color or transparency specifications)
% or can be calculated. For example, \hologo{LaTeX}'s \cs{strip@pt}
% makes a good job to output a real number. It automatically suppresses
% the decimal part if the number is an integer. However it leaves
% a leading zero for numbers greater zero and smaller one.
% Thus the package provides macros that can be used with different
% formats, even with \hologo{iniTeX} and generates numbers that
% are valid numbers of the PDF format and whose length is minimal.
%
% \subsection{Usage}
%
% The package \xpackage{thepdfnumber} can be used with \hologo{LaTeX},
% \hologo{plainTeX} or even with \hologo{iniTeX}:
% \begin{quote}
%   |\RequirePackage{thepdfnumber}| \textit{\% \hologo{LaTeX}}\\
%   |\input thepdfnumber.sty| \textit{\% \hologo{plainTeX}/\hologo{iniTeX}}
% \end{quote}
% The package does not need and have package options.
%
% \subsection{User macros}
%
% All user macros are expandable in exact two expansion steps.
%
% \begin{declcs}{thepdfnumber} \M{number}
% \end{declcs}
% Macro \cs{thepdfnumber} takes a number
% as argument and expands to a minimal representation of that number.
% Some examples:
% \begin{quote}
% \begin{tabular}{r@{.: }>{\ttfamily}l@{ $\rightarrow$\ }>{\ttfamily}l@{}}
%   1& +123 & 123\\
%   2& --123 & 123\\
%   3& -01 & -1\\
%   4& 0045 & 45\\
%   5& 1.0 & 1\\
%   6& 1.20 & 1.2\\
%   7& 0.0 & 0\\
%   8& 0.78 & .78\\
%   9& +012.340 & 12.34\\
% \end{tabular}
% \end{quote}
% It reduces the length of the number representation:
% \begin{itemize}
% \item The signs are collapsed and only one minus sign
%   is output if the number is negative (see examples 1, 2, 3, 9).
% \item Leading zeros are removed (4, 8, 9) unless the number is zero (7).
% \item The decimal part is omitted, if the number is an integer (5, 7).
% \item Trailing zeros from the decimal part are stripped (5, 6, 7, 9).
% \end{itemize}
% The resulting number representation can be catched with one of the
% following Perl regular expressions:
% \begin{itemize}
% \item |^0$| (zero)
% \item |^-?[1-9][0-9]*$| (integer)
% \item |^-?[0-9]*\.[0-9]*[1-9]$| (real)
% \end{itemize}
% This is a valid numeric object of the PDF specification \cite[``7.3.3~Numeric Objects'']{pdfspec-iso32000-1}.
%
% \begin{declcs}{thepdfnumberNormZeroOne}
% \end{declcs}
% There are various places in the PDF specification where the
% number is in the domain 0.0 upto 1.0. Macro thepdfnumberNormZeroOne
% automatically adjusts the number to fit into that range. Negative
% numbers are mapped to |0| and numbers greater than one are
% replaced by |1|. Thus the result fits one of the following regular
% expressions:
% \begin{itemize}
% \item |^0$|
% \item |^\.[0-9]*[1-9]$|
% \item |^1$|
% \end{itemize}
% Examples:
% \begin{quote}
% \begin{tabular}{@{}>{\ttfamily}l@{ $\rightarrow$\ }>{\ttfamily}l@{}}
%   -456 & 0\\
%   -0.001 & 0\\
%   0.0 & 0\\
%   0.010 & .01\\
%   0.456 & .456\\
%   1.0 & 1\\
%   01.001 & 1\\
%   4 & 1\\
% \end{tabular}
% \end{quote}
%
% \subsection{Input number}
%
% The user macros expect a number as argument. The number can either
% be given explicitely or as macro that expands in one step
% to an explicite number, because the first token of the argument
% is expanded once.
%
% The explicite number consists of
% \begin{itemize}
% \item optional signs `\texttt{+}' and `\texttt{-}',
% \item digits `\texttt{0}' upto `\texttt{9}' and
% \item an optional dot `\texttt{.}'.
% \end{itemize}
% All tokens must have catcode 12 (other), the default catcodes
% for these characters in \hologo{LaTeX}, \hologo{plainTeX} or
% \hologo{iniTeX}. As Perl regular expression the number is expected
% in one of the following forms:
% \begin{itemize}
% \item |^[+-]*[0-9]+$|
% \item |^[+-]*[0-9]*\.[0-9]*$|
% \end{itemize}
% At least one digit or the dot must be present.
%
% \subsection{Error handling}
%
% The package is not intended for validating numbers or to
% decide if an argument is a number. Therefore it is an usage
% error to use the user macros with arguments that are not
% explicite numbers as specified in the previous sections.
% Nevertheless some error conditions are sometimes recognized.
% Errors are given in form of an undefined command sequence.
% It is the only way to notify \hologo{TeX} in expandable context.
% Expanding to some error text would invalidate the output.
% Currently the following errors are thrown:
% \begin{description}
% \item[\cs{thepdfnumber@ErrorEndMarker}:]
%   Internally the argument parsing uses an end marker that
%   is never called directly. If it is called with valid
%   user input, then this is a bug. Otherwise it means the user input
%   contains nasty stuff.
% \item[\cs{thepdfnumber@ErrorUnexpectedEnd}:] The macros expect at least one digit
%   or the dot, otherwise if the argument is empty or only contains
%   signs, then this error is called.
% \item[\cs{thepdfnumber@ErrorInvalidToken}:] It is called if the number
%   contains other tokens than signs, digits or the dot or the
%   token at the wrong place (e.g. a sign after a digit).
%   In case of \cs{thepdfnumberNormZeroOne} this error condition might
%   not always be detected, because the number parsing might stop
%   at an early point, when the result is already clear (e.g.
%   if the number is negative or will be greater than one).
% \item[Improper alphabetic constant:] This error might be
%   thrown by \hologo{TeX}, if the number contains
%   command tokens instead of characters.
% \end{description}
%
% \StopEventually{
% }
%
% \section{Implementation}
%
%    \begin{macrocode}
%<*package>
%    \end{macrocode}
%
% \subsection{Reload check and package identification}
%    Reload check, especially if the package is not used with \LaTeX.
%    \begin{macrocode}
\begingroup\catcode61\catcode48\catcode32=10\relax%
  \catcode13=5 % ^^M
  \endlinechar=13 %
  \catcode35=6 % #
  \catcode39=12 % '
  \catcode44=12 % ,
  \catcode45=12 % -
  \catcode46=12 % .
  \catcode58=12 % :
  \catcode64=11 % @
  \catcode123=1 % {
  \catcode125=2 % }
  \expandafter\let\expandafter\x\csname ver@thepdfnumber.sty\endcsname
  \ifx\x\relax % plain-TeX, first loading
  \else
    \def\empty{}%
    \ifx\x\empty % LaTeX, first loading,
      % variable is initialized, but \ProvidesPackage not yet seen
    \else
      \expandafter\ifx\csname PackageInfo\endcsname\relax
        \def\x#1#2{%
          \immediate\write-1{Package #1 Info: #2.}%
        }%
      \else
        \def\x#1#2{\PackageInfo{#1}{#2, stopped}}%
      \fi
      \x{thepdfnumber}{The package is already loaded}%
      \aftergroup\endinput
    \fi
  \fi
\endgroup%
%    \end{macrocode}
%    Package identification:
%    \begin{macrocode}
\begingroup\catcode61\catcode48\catcode32=10\relax%
  \catcode13=5 % ^^M
  \endlinechar=13 %
  \catcode35=6 % #
  \catcode39=12 % '
  \catcode40=12 % (
  \catcode41=12 % )
  \catcode44=12 % ,
  \catcode45=12 % -
  \catcode46=12 % .
  \catcode47=12 % /
  \catcode58=12 % :
  \catcode64=11 % @
  \catcode91=12 % [
  \catcode93=12 % ]
  \catcode123=1 % {
  \catcode125=2 % }
  \expandafter\ifx\csname ProvidesPackage\endcsname\relax
    \def\x#1#2#3[#4]{\endgroup
      \immediate\write-1{Package: #3 #4}%
      \xdef#1{#4}%
    }%
  \else
    \def\x#1#2[#3]{\endgroup
      #2[{#3}]%
      \ifx#1\@undefined
        \xdef#1{#3}%
      \fi
      \ifx#1\relax
        \xdef#1{#3}%
      \fi
    }%
  \fi
\expandafter\x\csname ver@thepdfnumber.sty\endcsname
\ProvidesPackage{thepdfnumber}%
  [2016/05/16 v1.1 Print PDF numbers with minimal digits (HO)]%
%    \end{macrocode}
%
% \subsection{Catcodes}
%
%    \begin{macrocode}
\begingroup\catcode61\catcode48\catcode32=10\relax%
  \catcode13=5 % ^^M
  \endlinechar=13 %
  \catcode123=1 % {
  \catcode125=2 % }
  \catcode64=11 % @
  \def\x{\endgroup
    \expandafter\edef\csname ThPdNu@AtEnd\endcsname{%
      \endlinechar=\the\endlinechar\relax
      \catcode13=\the\catcode13\relax
      \catcode32=\the\catcode32\relax
      \catcode35=\the\catcode35\relax
      \catcode61=\the\catcode61\relax
      \catcode64=\the\catcode64\relax
      \catcode123=\the\catcode123\relax
      \catcode125=\the\catcode125\relax
    }%
  }%
\x\catcode61\catcode48\catcode32=10\relax%
\catcode13=5 % ^^M
\endlinechar=13 %
\catcode35=6 % #
\catcode64=11 % @
\catcode123=1 % {
\catcode125=2 % }
\def\TMP@EnsureCode#1#2{%
  \edef\ThPdNu@AtEnd{%
    \ThPdNu@AtEnd
    \catcode#1=\the\catcode#1\relax
  }%
  \catcode#1=#2\relax
}
\TMP@EnsureCode{33}{12}% !
\TMP@EnsureCode{36}{3}% $
\TMP@EnsureCode{38}{4}% &
\TMP@EnsureCode{42}{12}% *
\TMP@EnsureCode{43}{12}% +
\TMP@EnsureCode{45}{12}% -
\TMP@EnsureCode{46}{12}% .
\TMP@EnsureCode{60}{12}% <
\TMP@EnsureCode{62}{12}% >
\TMP@EnsureCode{96}{12}% `
\edef\ThPdNu@AtEnd{\ThPdNu@AtEnd\noexpand\endinput}
%    \end{macrocode}
%
% \subsection{Helper macros}
%
%    \begin{macro}{\ThPdNu@FIN}
%    \begin{macrocode}
\def\ThPdNu@FIN{\thepdfnumber@ErrorEndMarker}
%    \end{macrocode}
%    \end{macro}
%
%    \begin{macro}{\ThPdNu@space}
%    \begin{macrocode}
\def\ThPdNu@space{ }
%    \end{macrocode}
%    \end{macro}
%
%    \begin{macro}{\ThPdNu@zero}
%    \begin{macrocode}
\chardef\ThPdNu@zero=0 %
%    \end{macrocode}
%    \end{macro}
%
%    \begin{macro}{\ThPdNu@one}
%    \begin{macrocode}
\chardef\ThPdNu@one=1 %
%    \end{macrocode}
%    \end{macro}
%
%    \begin{macro}{\ThPdNu@firstoftwo}
%    \begin{macrocode}
\long\def\ThPdNu@firstoftwo#1#2{#1}
%    \end{macrocode}
%    \end{macro}
%
%    \begin{macro}{\ThPdNu@secondoftwo}
%    \begin{macrocode}
\long\def\ThPdNu@secondoftwo#1#2{#2}
%    \end{macrocode}
%    \end{macro}
%
% \subsection{Detect \hologo{eTeX}}
%
%    \begin{macrocode}
\begingroup\expandafter\expandafter\expandafter\endgroup
\expandafter\ifx\csname detokenize\endcsname\relax
  \catcode`\&=14 %
  \catcode`\$=9 %
\else
  \catcode`\&=9 %
  \catcode`\$=14 %
\fi
%    \end{macrocode}
%
% \subsection{User macro \cs{thepdfnumber}}
%
%    \begin{macro}{\thepdfnumber}
%    \begin{macrocode}
\def\thepdfnumber#1{%
  \romannumeral
& \iftrue\expandafter\ThPdNu@State@Plus\expandafter\fi
& \detokenize\expandafter{#1}%
& \ThPdNu@FIN
$ \ifx\ThPdNu@FIN#1\ThPdNu@FIN
$   \expandafter\ThPdNu@firstoftwo
$ \else
$   \expandafter\ThPdNu@secondoftwo
$ \fi
$ {%
$   \ThPdNu@zero
$   0\thepdfnumber@ErrorUnexpectedEnd
$ }{%
$   \iftrue\expandafter\ThPdNu@State@Plus\expandafter\fi#1\ThPdNu@FIN
$ }%
}
%    \end{macrocode}
%    \end{macro}
%
% \subsubsection{State definitions for sign}
%
%    \begin{macro}{\ThPdNu@State@Plus}
%    \begin{macrocode}
\def\ThPdNu@State@Plus#1\fi#2{%
  \fi
  \ifcase\ifx\ThPdNu@FIN#2%
           0%
&        \else\ifx-#2%
$        \else\ifnum`#2=45 % -
           1%
         \else\ifx0#2%
           2%
         \else\ifnum`#2>48 %
           \ifnum`#2<58 %
             3%
           \else
             9%
           \fi
&        \else\ifx.#2%
$        \else\ifnum`#2=46 % .
           4%
&        \else\ifx+#2%
$        \else\ifnum`#2=43 % +
           5%
         \else
           9%
         \fi\fi\fi\fi\fi\fi\ThPdNu@space
    \expandafter\ThPdNu@zero
    \expandafter0%
    \expandafter\thepdfnumber@ErrorUnexpectedEnd
  \or
    \ThPdNu@State@Minus
  \or
    \ThPdNu@State@SkipZeros!%
  \or
    \ThPdNu@State@Int!#2!%
  \or
    \ThPdNu@State@Dot!\ThPdNu@zero*\ThPdNu@zero!!%
  \or
    \ThPdNu@State@Plus
  \else
    \ThPdNu@ReturnError{0}%
  \fi
}
%    \end{macrocode}
%    \end{macro}
%
%    \begin{macro}{\ThPdNu@State@Minus}
%    \begin{macrocode}
\def\ThPdNu@State@Minus#1\fi#2{%
  \fi
  \ifcase\ifx\ThPdNu@FIN#2%
           0%
         \else\ifx0#2%
           1%
         \else\ifnum`#2>48 %
           \ifnum`#2<58 %
             2%
           \else
             9%
           \fi
&        \else\ifx.#2%
$        \else\ifnum`#2=46 % .
           3%
&        \else\ifx-#2%
$        \else\ifnum`#2=45 % -
           4%
&        \else\ifx+#2%
$        \else\ifnum`#2=43 % +
           5%
         \else
           9%
         \fi\fi\fi\fi\fi\fi\ThPdNu@space
    \expandafter\ThPdNu@zero
    \expandafter0%
    \expandafter\thepdfnumber@ErrorUnexpectedEnd
  \or
    \ThPdNu@State@SkipZeros-!%
  \or
    \ThPdNu@State@Int-!#2!%
  \or
    \ThPdNu@State@Dot-!\ThPdNu@zero*\ThPdNu@zero!!%
  \or
    \ThPdNu@State@Plus
  \or
    \ThPdNu@State@Minus
  \else
    \ThPdNu@ReturnError{0}%
  \fi
}
%    \end{macrocode}
%    \end{macro}
%
%    \begin{macro}{\ThPdNu@ReturnError}
%    \begin{macrocode}
\def\ThPdNu@ReturnError#1#2\fi#3\ThPdNu@FIN{%
  \fi
  \ThPdNu@zero
  #1%
  \thepdfnumber@ErrorInvalidToken
}
%    \end{macrocode}
%    \end{macro}
%
% \subsubsection{State definitions for integer part}
%
%    \begin{macro}{\ThPdNu@State@SkipZeros}
%    \begin{macrocode}
\def\ThPdNu@State@SkipZeros#1!#2\fi#3{%
  \fi
  \ifcase\ifx\ThPdNu@FIN#3%
           0%
         \else\ifx0#3%
           1%
         \else\ifnum`#3>48 %
           \ifnum`#3<58 %
             2%
           \else
             9%
           \fi
&        \else\ifx.#3%
$        \else\ifnum`#3=46 % .
           3%
         \else
           9%
         \fi\fi\fi\fi\ThPdNu@space
    \expandafter\ThPdNu@zero
    \expandafter0%
  \or
   \ThPdNu@State@SkipZeros#1!%
  \or
    \ThPdNu@State@Int#1!#3!%
  \or
    \ThPdNu@State@Dot#1!\ThPdNu@zero*\ThPdNu@zero!!%
  \else
    \ThPdNu@ReturnError{0}%
  \fi
}
%    \end{macrocode}
%    \end{macro}
%
%    \begin{macro}{\ThPdNu@State@Int}
%    \begin{macrocode}
\def\ThPdNu@State@Int#1!#2!#3\fi#4{%
  \fi
  \ifcase\ifx\ThPdNu@FIN#4%
           0%
         \else\ifnum`#4>47 %
           \ifnum`#4<58 %
             1%
           \else
             9%
           \fi
&        \else\ifx.#4%
$        \else\ifnum`#4=46 % .
           2%
         \else
           9%
         \fi\fi\fi\ThPdNu@space
    \ThPdNu@ReturnInt{#1#2}%
  \or
    \ThPdNu@State@Int#1!#2#4!%
  \or
    \ThPdNu@State@Dot#1!\ThPdNu@one#2*\ThPdNu@zero!!%
  \else
    \ThPdNu@ReturnError{#1#2}%
  \fi
}
%    \end{macrocode}
%    \end{macro}
%
%    \begin{macro}{\ThPdNu@ReturnInt}
%    \begin{macrocode}
\def\ThPdNu@ReturnInt#1#2\fi{%
  \fi
  \ThPdNu@zero
  #1%
}
%    \end{macrocode}
%    \end{macro}
%
% \subsubsection{State definitions for decimal digits}
%
%    \begin{macro}{\ThPdNu@State@Dot}
%    \begin{macrocode}
\def\ThPdNu@State@Dot#1*#2#3!#4!#5\fi#6{%
  \fi
  \ifcase\ifx\ThPdNu@FIN#6%
           0%
         \else\ifnum`#6>48 %
           \ifnum`#6<58 %
             1%
           \else
             9%
           \fi
         \else\ifx0#6%
           2%
         \else
           9%
         \fi\fi\fi\ThPdNu@space
    \ThPdNu@ReturnNumber#1*#2#3!%
  \or
    \ThPdNu@State@Dot#1*\ThPdNu@one#3#4#6!!%
  \or
    \ThPdNu@State@DotZero#1*#2#3!#4#6!%
  \else
    \ThPdNu@ReturnNumberInvalid#1*#2#3!%
  \fi
}
%    \end{macrocode}
%    \end{macro}
%
%    \begin{macro}{\ThPdNu@State@DotZero}
%    \begin{macrocode}
\def\ThPdNu@State@DotZero#1*#2#3!#4!#5\fi#6{%
  \fi
  \ifcase\ifx\ThPdNu@FIN#6%
           0%
         \else\ifnum`#6>48 %
           \ifnum`#6<58 %
             1%
           \else
             9%
           \fi
         \else\ifx0#6%
           2%
         \else
           9%
         \fi\fi\fi\ThPdNu@space
    \ThPdNu@ReturnNumber#1*#2#3!%
  \or
    \ThPdNu@State@Dot#1*\ThPdNu@one#3#4#6!!%
  \or
    \ThPdNu@State@DotZero#1*#2#3!#4#6!%
  \else
    \ThPdNu@ReturnNumber#1*#2#3!%
  \fi
}
%    \end{macrocode}
%    \end{macro}
%
%    \begin{macro}{\ThPdNu@ReturnNumber}
%    \begin{macrocode}
\def\ThPdNu@ReturnNumber#1!#2#3*#4#5!#6\fi{%
  \fi
  \ifcase#2%
    \expandafter\ThPdNu@firstoftwo
  \else
    \expandafter\ThPdNu@secondoftwo
  \fi
  {%
    \ifcase#4%
      \expandafter\ThPdNu@firstoftwo
    \else
      \expandafter\ThPdNu@secondoftwo
    \fi
    {\ThPdNu@zero 0}%
    {\ThPdNu@zero #1.#5}%
  }{%
    \ifcase#4%
      \expandafter\ThPdNu@firstoftwo
    \else
      \expandafter\ThPdNu@secondoftwo
    \fi
    {\ThPdNu@zero #1#3}%
    {\ThPdNu@zero #1#3.#5}%
  }%
}
%    \end{macrocode}
%    \end{macro}
%
%    \begin{macro}{\ThPdNu@ReturnNumberInvalid}
%    \begin{macrocode}
\def\ThPdNu@ReturnNumberInvalid#1*#2!#3\fi#4\ThPdNu@FIN{%
  \fi
  \iftrue\ThPdNu@ReturnNumber#1*#2!\fi
  \thepdfnumber@ErrorInvalidToken
}
%    \end{macrocode}
%    \end{macro}
%
% \subsection{Norm macro}
%
%    \begin{macro}{\thepdfnumberNormZeroOne}
%    \begin{macrocode}
\def\thepdfnumberNormZeroOne#1{%
  \romannumeral
& \iftrue\expandafter\ThPbNu@StateN@Plus\expandafter\fi
& \detokenize\expandafter{#1}%
& \ThPdNu@FIN
$ \ifx\ThPdNu@FIN#1\ThPdNu@FIN
$   \expandafter\ThPdNu@firstoftwo
$ \else
$   \expandafter\ThPdNu@secondoftwo
$ \fi
$ {%
$   \ThPdNu@zero
$   0\thepdfnumber@ErrorUnexpectedEnd
$ }{%
$   \iftrue\expandafter\ThPbNu@StateN@Plus\expandafter\fi#1\ThPdNu@FIN
$ }%
}
%    \end{macrocode}
%    \end{macro}
%
% \subsubsection{State definitions for sign}
%
%    \begin{macro}{\ThPbNu@StateN@Plus}
%    \begin{macrocode}
\def\ThPbNu@StateN@Plus#1\fi#2{%
  \fi
  \ifcase\ifx\ThPdNu@FIN#2%
           0%
&        \else\ifx-#2%
$        \else\ifnum`#2=45 % -
           1%
         \else\ifx0#2%
           2%
         \else\ifnum`#2>48 %
           \ifnum`#2<58 %
             3%
           \else
             9%
           \fi
&        \else\ifx.#2%
$        \else\ifnum`#2=46 % .
           4%
&        \else\ifx+#2%
$        \else\ifnum`#2=43 % +
           5%
         \else
           9%
         \fi\fi\fi\fi\fi\fi\ThPdNu@space
    \expandafter\ThPdNu@zero
    \expandafter0%
    \expandafter\thepdfnumber@ErrorUnexpectedEnd
  \or
    \ThPbNu@StateN@Minus
  \or
    \ThPbNu@StateN@SkipZeros
  \or
    \ThPdNu@ReturnAndSkip{1}%
  \or
    \ThPbNu@StateN@Dot\ThPdNu@zero!!%
  \or
    \ThPbNu@StateN@Plus
  \else
    \ThPdNu@ReturnError{0}%
  \fi
}
%    \end{macrocode}
%    \end{macro}
%
%    \begin{macro}{\ThPbNu@StateN@Minus}
%    \begin{macrocode}
\def\ThPbNu@StateN@Minus#1\fi#2{%
  \fi
  \ifcase\ifx\ThPdNu@FIN#2%
           0%
         \else\ifnum`#2>47 %
           \ifnum`#2<58 %
             1%
           \else
             9%
           \fi
&        \else\ifx.#2%
$        \else\ifnum`#2=46 % .
           1%
&        \else\ifx-#2%
$        \else\ifnum`#2=45 % -
           2%
&        \else\ifx+#2%
$        \else\ifnum`#2=43 % +
           3%
         \else
           9%
         \fi\fi\fi\fi\fi\ThPdNu@space
    \expandafter\ThPdNu@zero
    \expandafter0%
    \expandafter\thepdfnumber@ErrorUnexpectedEnd
  \or
    \ThPdNu@ReturnAndSkip{0}%
  \or
    \ThPbNu@StateN@Plus
  \or
    \ThPbNu@StateN@Minus
  \else
    \ThPdNu@ReturnError{0}%
  \fi
}
%    \end{macrocode}
%    \end{macro}
%
%    \begin{macro}{\ThPdNu@ReturnAndSkip}
%    \begin{macrocode}
\def\ThPdNu@ReturnAndSkip#1#2\fi#3\ThPdNu@FIN{%
  \fi
  \ThPdNu@zero
  #1%
}
%    \end{macrocode}
%    \end{macro}
%
% \subsubsection{State definitions for integer part}
%
%    \begin{macro}{\ThPbNu@StateN@SkipZeros}
%    \begin{macrocode}
\def\ThPbNu@StateN@SkipZeros#1\fi#2{%
  \fi
  \ifcase\ifx\ThPdNu@FIN#2%
           0%
         \else\ifx0#2%
           1%
         \else\ifnum`#2>48 %
           \ifnum`#2<58 %
             2%
           \else
             9%
           \fi
&        \else\ifx.#2%
$        \else\ifnum`#2=46 % .
           3%
         \else
           9%
         \fi\fi\fi\fi\ThPdNu@space
    \expandafter\ThPdNu@zero
    \expandafter0%
  \or
   \ThPbNu@StateN@SkipZeros%
  \or
    \ThPdNu@ReturnAndSkip{1}%
  \or
    \ThPbNu@StateN@Dot\ThPdNu@zero!!%
  \else
    \ThPdNu@ReturnError{0}%
  \fi
}
%    \end{macrocode}
%    \end{macro}
%
% \subsubsection{State definitions for decimal digits}
%
%    \begin{macro}{\ThPbNu@StateN@Dot}
%    \begin{macrocode}
\def\ThPbNu@StateN@Dot#1#2!#3!#4\fi#5{%
  \fi
  \ifcase\ifx\ThPdNu@FIN#5%
           0%
         \else\ifnum`#5>48 %
           \ifnum`#5<58 %
             1%
           \else
             9%
           \fi
         \else\ifx0#5%
           2%
         \else
           9%
         \fi\fi\fi\ThPdNu@space
    \ThPdNu@ReturnFracNumber#1#2!%
  \or
    \ThPbNu@StateN@Dot\ThPdNu@one#2#3#5!!%
  \or
    \ThPbNu@StateN@DotZero#1#2!#3#5!%
  \else
    \ThPdNu@ReturnFracNumberInvalid#1#2!%
  \fi
}
%    \end{macrocode}
%    \end{macro}
%
%    \begin{macro}{\ThPbNu@StateN@DotZero}
%    \begin{macrocode}
\def\ThPbNu@StateN@DotZero#1#2!#3!#4\fi#5{%
  \fi
  \ifcase\ifx\ThPdNu@FIN#5%
           0%
         \else\ifnum`#5>48 %
           \ifnum`#5<58 %
             1%
           \else
             9%
           \fi
         \else\ifx0#5%
           2%
         \else
           9%
         \fi\fi\fi\ThPdNu@space
    \ThPdNu@ReturnFracNumber#1#2!%
  \or
    \ThPbNu@StateN@Dot\ThPdNu@one#2#3#5!!%
  \or
    \ThPbNu@StateN@DotZero#1#2!#3#5!%
  \else
    \ThPdNu@ReturnFracNumberInvalid#1#2!%
  \fi
}
%    \end{macrocode}
%    \end{macro}
%
%    \begin{macro}{\ThPdNu@ReturnFracNumber}
%    \begin{macrocode}
\def\ThPdNu@ReturnFracNumber#1#2!#3\fi{%
  \fi
  \ifcase#1%
    \expandafter\ThPdNu@firstoftwo
  \else
    \expandafter\ThPdNu@secondoftwo
  \fi
  {\ThPdNu@zero 0}%
  {\ThPdNu@zero .#2}%
}
%    \end{macrocode}
%    \end{macro}
%
%    \begin{macro}{\ThPdNu@ReturnFracNumberInvalid}
%    \begin{macrocode}
\def\ThPdNu@ReturnFracNumberInvalid#1!#2\fi#3\ThPdNu@FIN{%
  \fi
  \iftrue\ThPdNu@ReturnFracNumber#1!\fi
  \thepdfnumber@ErrorInvalidToken
}
%    \end{macrocode}
%    \end{macro}
%
%    \begin{macrocode}
\ThPdNu@AtEnd%
%</package>
%    \end{macrocode}
%
% \section{Test}
%
% \subsection{Catcode checks for loading}
%
%    \begin{macrocode}
%<*test1>
%    \end{macrocode}
%    \begin{macrocode}
\catcode`\{=1 %
\catcode`\}=2 %
\catcode`\#=6 %
\catcode`\@=11 %
\expandafter\ifx\csname count@\endcsname\relax
  \countdef\count@=255 %
\fi
\expandafter\ifx\csname @gobble\endcsname\relax
  \long\def\@gobble#1{}%
\fi
\expandafter\ifx\csname @firstofone\endcsname\relax
  \long\def\@firstofone#1{#1}%
\fi
\expandafter\ifx\csname loop\endcsname\relax
  \expandafter\@firstofone
\else
  \expandafter\@gobble
\fi
{%
  \def\loop#1\repeat{%
    \def\body{#1}%
    \iterate
  }%
  \def\iterate{%
    \body
      \let\next\iterate
    \else
      \let\next\relax
    \fi
    \next
  }%
  \let\repeat=\fi
}%
\def\RestoreCatcodes{}
\count@=0 %
\loop
  \edef\RestoreCatcodes{%
    \RestoreCatcodes
    \catcode\the\count@=\the\catcode\count@\relax
  }%
\ifnum\count@<255 %
  \advance\count@ 1 %
\repeat

\def\RangeCatcodeInvalid#1#2{%
  \count@=#1\relax
  \loop
    \catcode\count@=15 %
  \ifnum\count@<#2\relax
    \advance\count@ 1 %
  \repeat
}
\def\RangeCatcodeCheck#1#2#3{%
  \count@=#1\relax
  \loop
    \ifnum#3=\catcode\count@
    \else
      \errmessage{%
        Character \the\count@\space
        with wrong catcode \the\catcode\count@\space
        instead of \number#3%
      }%
    \fi
  \ifnum\count@<#2\relax
    \advance\count@ 1 %
  \repeat
}
\def\space{ }
\expandafter\ifx\csname LoadCommand\endcsname\relax
  \def\LoadCommand{\input thepdfnumber.sty\relax}%
\fi
\def\Test{%
  \RangeCatcodeInvalid{0}{47}%
  \RangeCatcodeInvalid{58}{64}%
  \RangeCatcodeInvalid{91}{96}%
  \RangeCatcodeInvalid{123}{255}%
  \catcode`\@=12 %
  \catcode`\\=0 %
  \catcode`\%=14 %
  \LoadCommand
  \RangeCatcodeCheck{0}{36}{15}%
  \RangeCatcodeCheck{37}{37}{14}%
  \RangeCatcodeCheck{38}{47}{15}%
  \RangeCatcodeCheck{48}{57}{12}%
  \RangeCatcodeCheck{58}{63}{15}%
  \RangeCatcodeCheck{64}{64}{12}%
  \RangeCatcodeCheck{65}{90}{11}%
  \RangeCatcodeCheck{91}{91}{15}%
  \RangeCatcodeCheck{92}{92}{0}%
  \RangeCatcodeCheck{93}{96}{15}%
  \RangeCatcodeCheck{97}{122}{11}%
  \RangeCatcodeCheck{123}{255}{15}%
  \RestoreCatcodes
}
\Test
\csname @@end\endcsname
\end
%    \end{macrocode}
%    \begin{macrocode}
%</test1>
%    \end{macrocode}
%
% \subsection{Macro tests}
%
% \subsubsection{Preamble}
%
%    \begin{macrocode}
%<*test2>
\catcode`\{=1
\catcode`\}=2
\catcode`\#=6
\catcode`\@=11
\errorcontextlines=10000 %
\def\msg#{\immediate\write16}
\def\space{ }
\begingroup\expandafter\expandafter\expandafter\endgroup
\expandafter\ifx\csname RequirePackage\endcsname\relax
  \input thepdfnumber.sty\relax
\else
  \RequirePackage{thepdfnumber}[2016/05/16]%
\fi
%    \end{macrocode}
%
%    \begin{macro}{\Test}
%    \begin{macrocode}
\def\Test#1#2{%
  \def\TestExpected{#2}%
%<*active>
  \edef\TestExpected{\TestExpected}%
%</active>
  \expandafter\expandafter\expandafter\def
  \expandafter\expandafter\expandafter\TestResult
  \expandafter\expandafter\expandafter{%
    \TestCommand{#1}%
  }%
  \begingroup
    \toks0\expandafter{\TestResult}%
    \toks2\expandafter{\TestExpected}%
    \ifx\TestResult\TestExpected
      \msg{* OK: #1 => \the\toks0}%
    \else
      \errmessage{FAILED: #1 => \the\toks0\space(\the\toks2)}%
    \fi
  \endgroup
}
%    \end{macrocode}
%    \end{macro}
%
%    \begin{macrocode}
\let\thepdfnumber@ErrorUnexpectedEnd\relax
\let\thepdfnumber@ErrorInvalidToken\relax
%    \end{macrocode}
%
% \subsubsection{Test with active catcodes}
%
%    \begin{macrocode}
%<*active>
\catcode`\~=13 %
\def\ThPdNu@Temp#1{%
  \begingroup
    \lccode`\~=`#1 %
  \lowercase{\endgroup
    \edef~%
  }{\string#1}%
  \catcode`#1=13 %
}
\ThPdNu@Temp{.}
\ThPdNu@Temp{-}
\ThPdNu@Temp{+}
%</active>
%    \end{macrocode}
%
% \subsubsection{Test cases for \cs{thepdfnumber}}
%
%    \begin{macro}{\TestCommand}
%    \begin{macrocode}
\let\TestCommand\thepdfnumber
%    \end{macrocode}
%    \end{macro}
%
%    \begin{macrocode}
\Test{0}{0}
\Test{1}{1}
\Test{23}{23}
\Test{0045}{45}
\Test{4500}{4500}
\Test{-0}{0}
\Test{-1}{-1}
\Test{-00002}{-2}
\Test{-010203}{-10203}
\Test{+0}{0}
\Test{+123}{123}

\Test{+-4}{-4}
\Test{+-+-4}{4}
\Test{++--++--++4}{4}
\Test{+++---+++---+++---4}{-4}

\Test{.}{0}
\Test{0.}{0}
\Test{.0}{0}
\Test{0.}{0}
\Test{-.}{0}
\Test{-0.}{0}
\Test{-.0}{0}
\Test{-0.}{0}
\Test{010.020}{10.02}
\Test{123.456}{123.456}
\Test{12.}{12}
\Test{.123}{.123}
\Test{.0001}{.0001}
\Test{.00100}{.001}
\Test{.12003400560078009}{.12003400560078009}
\Test{-.12003400560078009}{-.12003400560078009}
\Test{04.0500000}{4.05}

\Test{}{0\thepdfnumber@ErrorUnexpectedEnd}
\Test{+}{0\thepdfnumber@ErrorUnexpectedEnd}
\Test{-}{0\thepdfnumber@ErrorUnexpectedEnd}
\Test{a}{0\thepdfnumber@ErrorInvalidToken}
\Test{0x1}{0\thepdfnumber@ErrorInvalidToken}
\Test{4x56}{4\thepdfnumber@ErrorInvalidToken}
\Test{012,34}{12\thepdfnumber@ErrorInvalidToken}
\Test{0.12x4}{.12\thepdfnumber@ErrorInvalidToken}
%    \end{macrocode}
%
% \subsubsection{Test cases for \cs{thepdfnumberNormZeroOne}}
%
%    \begin{macro}{\TestCommand}
%    \begin{macrocode}
\let\TestCommand\thepdfnumberNormZeroOne
%    \end{macrocode}
%    \end{macro}
%
%    \begin{macrocode}
\Test{0}{0}
\Test{1}{1}
\Test{23}{1}
\Test{12}{1}
\Test{10}{1}
\Test{0045}{1}
\Test{001}{1}
\Test{-0}{0}
\Test{-1}{0}
\Test{-01}{0}
\Test{-4}{0}

\Test{+-0}{0}
\Test{+-+-1}{1}
\Test{++--++--++1}{1}
\Test{+++---+++---+++---1}{0}

\Test{.}{0}
\Test{0.}{0}
\Test{.0}{0}
\Test{0.}{0}
\Test{-.}{0}
\Test{-0.}{0}
\Test{-.0}{0}
\Test{-0.}{0}
\Test{010.020}{1}
\Test{123.456}{1}
\Test{12.}{1}
\Test{.123}{.123}
\Test{.0001}{.0001}
\Test{.00100}{.001}
\Test{.12003400560078009}{.12003400560078009}
\Test{-.12003400560078009}{0}
\Test{04.0500000}{1}
\Test{0.1200340056}{.1200340056}
\Test{1.05}{1}

\Test{}{0\thepdfnumber@ErrorUnexpectedEnd}
\Test{+}{0\thepdfnumber@ErrorUnexpectedEnd}
\Test{-}{0\thepdfnumber@ErrorUnexpectedEnd}
\Test{a}{0\thepdfnumber@ErrorInvalidToken}
\Test{0x1}{0\thepdfnumber@ErrorInvalidToken}
\Test{4x56}{1}
\Test{012,34}{1}
\Test{.012x4}{.012\thepdfnumber@ErrorInvalidToken}
%    \end{macrocode}
%
%    \begin{macrocode}
\csname @@end\endcsname\end
%</test2>
%    \end{macrocode}
%
% \section{Installation}
%
% \subsection{Download}
%
% \paragraph{Package.} This package is available on
% CTAN\footnote{\url{http://ctan.org/pkg/thepdfnumber}}:
% \begin{description}
% \item[\CTAN{macros/latex/contrib/oberdiek/thepdfnumber.dtx}] The source file.
% \item[\CTAN{macros/latex/contrib/oberdiek/thepdfnumber.pdf}] Documentation.
% \end{description}
%
%
% \paragraph{Bundle.} All the packages of the bundle `oberdiek'
% are also available in a TDS compliant ZIP archive. There
% the packages are already unpacked and the documentation files
% are generated. The files and directories obey the TDS standard.
% \begin{description}
% \item[\CTAN{install/macros/latex/contrib/oberdiek.tds.zip}]
% \end{description}
% \emph{TDS} refers to the standard ``A Directory Structure
% for \TeX\ Files'' (\CTAN{tds/tds.pdf}). Directories
% with \xfile{texmf} in their name are usually organized this way.
%
% \subsection{Bundle installation}
%
% \paragraph{Unpacking.} Unpack the \xfile{oberdiek.tds.zip} in the
% TDS tree (also known as \xfile{texmf} tree) of your choice.
% Example (linux):
% \begin{quote}
%   |unzip oberdiek.tds.zip -d ~/texmf|
% \end{quote}
%
% \paragraph{Script installation.}
% Check the directory \xfile{TDS:scripts/oberdiek/} for
% scripts that need further installation steps.
% Package \xpackage{attachfile2} comes with the Perl script
% \xfile{pdfatfi.pl} that should be installed in such a way
% that it can be called as \texttt{pdfatfi}.
% Example (linux):
% \begin{quote}
%   |chmod +x scripts/oberdiek/pdfatfi.pl|\\
%   |cp scripts/oberdiek/pdfatfi.pl /usr/local/bin/|
% \end{quote}
%
% \subsection{Package installation}
%
% \paragraph{Unpacking.} The \xfile{.dtx} file is a self-extracting
% \docstrip\ archive. The files are extracted by running the
% \xfile{.dtx} through \plainTeX:
% \begin{quote}
%   \verb|tex thepdfnumber.dtx|
% \end{quote}
%
% \paragraph{TDS.} Now the different files must be moved into
% the different directories in your installation TDS tree
% (also known as \xfile{texmf} tree):
% \begin{quote}
% \def\t{^^A
% \begin{tabular}{@{}>{\ttfamily}l@{ $\rightarrow$ }>{\ttfamily}l@{}}
%   thepdfnumber.sty & tex/generic/oberdiek/thepdfnumber.sty\\
%   thepdfnumber.pdf & doc/latex/oberdiek/thepdfnumber.pdf\\
%   test/thepdfnumber-test1.tex & doc/latex/oberdiek/test/thepdfnumber-test1.tex\\
%   test/thepdfnumber-test2.tex & doc/latex/oberdiek/test/thepdfnumber-test2.tex\\
%   test/thepdfnumber-test3.tex & doc/latex/oberdiek/test/thepdfnumber-test3.tex\\
%   thepdfnumber.dtx & source/latex/oberdiek/thepdfnumber.dtx\\
% \end{tabular}^^A
% }^^A
% \sbox0{\t}^^A
% \ifdim\wd0>\linewidth
%   \begingroup
%     \advance\linewidth by\leftmargin
%     \advance\linewidth by\rightmargin
%   \edef\x{\endgroup
%     \def\noexpand\lw{\the\linewidth}^^A
%   }\x
%   \def\lwbox{^^A
%     \leavevmode
%     \hbox to \linewidth{^^A
%       \kern-\leftmargin\relax
%       \hss
%       \usebox0
%       \hss
%       \kern-\rightmargin\relax
%     }^^A
%   }^^A
%   \ifdim\wd0>\lw
%     \sbox0{\small\t}^^A
%     \ifdim\wd0>\linewidth
%       \ifdim\wd0>\lw
%         \sbox0{\footnotesize\t}^^A
%         \ifdim\wd0>\linewidth
%           \ifdim\wd0>\lw
%             \sbox0{\scriptsize\t}^^A
%             \ifdim\wd0>\linewidth
%               \ifdim\wd0>\lw
%                 \sbox0{\tiny\t}^^A
%                 \ifdim\wd0>\linewidth
%                   \lwbox
%                 \else
%                   \usebox0
%                 \fi
%               \else
%                 \lwbox
%               \fi
%             \else
%               \usebox0
%             \fi
%           \else
%             \lwbox
%           \fi
%         \else
%           \usebox0
%         \fi
%       \else
%         \lwbox
%       \fi
%     \else
%       \usebox0
%     \fi
%   \else
%     \lwbox
%   \fi
% \else
%   \usebox0
% \fi
% \end{quote}
% If you have a \xfile{docstrip.cfg} that configures and enables \docstrip's
% TDS installing feature, then some files can already be in the right
% place, see the documentation of \docstrip.
%
% \subsection{Refresh file name databases}
%
% If your \TeX~distribution
% (\teTeX, \mikTeX, \dots) relies on file name databases, you must refresh
% these. For example, \teTeX\ users run \verb|texhash| or
% \verb|mktexlsr|.
%
% \subsection{Some details for the interested}
%
% \paragraph{Attached source.}
%
% The PDF documentation on CTAN also includes the
% \xfile{.dtx} source file. It can be extracted by
% AcrobatReader 6 or higher. Another option is \textsf{pdftk},
% e.g. unpack the file into the current directory:
% \begin{quote}
%   \verb|pdftk thepdfnumber.pdf unpack_files output .|
% \end{quote}
%
% \paragraph{Unpacking with \LaTeX.}
% The \xfile{.dtx} chooses its action depending on the format:
% \begin{description}
% \item[\plainTeX:] Run \docstrip\ and extract the files.
% \item[\LaTeX:] Generate the documentation.
% \end{description}
% If you insist on using \LaTeX\ for \docstrip\ (really,
% \docstrip\ does not need \LaTeX), then inform the autodetect routine
% about your intention:
% \begin{quote}
%   \verb|latex \let\install=y\input{thepdfnumber.dtx}|
% \end{quote}
% Do not forget to quote the argument according to the demands
% of your shell.
%
% \paragraph{Generating the documentation.}
% You can use both the \xfile{.dtx} or the \xfile{.drv} to generate
% the documentation. The process can be configured by the
% configuration file \xfile{ltxdoc.cfg}. For instance, put this
% line into this file, if you want to have A4 as paper format:
% \begin{quote}
%   \verb|\PassOptionsToClass{a4paper}{article}|
% \end{quote}
% An example follows how to generate the
% documentation with pdf\LaTeX:
% \begin{quote}
%\begin{verbatim}
%pdflatex thepdfnumber.dtx
%makeindex -s gind.ist thepdfnumber.idx
%pdflatex thepdfnumber.dtx
%makeindex -s gind.ist thepdfnumber.idx
%pdflatex thepdfnumber.dtx
%\end{verbatim}
% \end{quote}
%
% \printbibliography[
%   heading=bibnumbered,
% ]
%
% \begin{History}
%   \begin{Version}{2011/11/24 v1.0}
%   \item
%     First version.
%   \end{Version}
%   \begin{Version}{2016/05/16 v1.1}
%   \item
%     Documentation updates.
%   \end{Version}
% \end{History}
%
% \PrintIndex
%
% \Finale
\endinput

%        (quote the arguments according to the demands of your shell)
%
% Documentation:
%    (a) If thepdfnumber.drv is present:
%           latex thepdfnumber.drv
%    (b) Without thepdfnumber.drv:
%           latex thepdfnumber.dtx; ...
%    The class ltxdoc loads the configuration file ltxdoc.cfg
%    if available. Here you can specify further options, e.g.
%    use A4 as paper format:
%       \PassOptionsToClass{a4paper}{article}
%
%    Programm calls to get the documentation (example):
%       pdflatex thepdfnumber.dtx
%       makeindex -s gind.ist thepdfnumber.idx
%       pdflatex thepdfnumber.dtx
%       makeindex -s gind.ist thepdfnumber.idx
%       pdflatex thepdfnumber.dtx
%
% Installation:
%    TDS:tex/generic/oberdiek/thepdfnumber.sty
%    TDS:doc/latex/oberdiek/thepdfnumber.pdf
%    TDS:doc/latex/oberdiek/test/thepdfnumber-test1.tex
%    TDS:doc/latex/oberdiek/test/thepdfnumber-test2.tex
%    TDS:doc/latex/oberdiek/test/thepdfnumber-test3.tex
%    TDS:source/latex/oberdiek/thepdfnumber.dtx
%
%<*ignore>
\begingroup
  \catcode123=1 %
  \catcode125=2 %
  \def\x{LaTeX2e}%
\expandafter\endgroup
\ifcase 0\ifx\install y1\fi\expandafter
         \ifx\csname processbatchFile\endcsname\relax\else1\fi
         \ifx\fmtname\x\else 1\fi\relax
\else\csname fi\endcsname
%</ignore>
%<*install>
\input docstrip.tex
\Msg{************************************************************************}
\Msg{* Installation}
\Msg{* Package: thepdfnumber 2016/05/16 v1.1 Print PDF numbers with minimal digits (HO)}
\Msg{************************************************************************}

\keepsilent
\askforoverwritefalse

\let\MetaPrefix\relax
\preamble

This is a generated file.

Project: thepdfnumber
Version: 2016/05/16 v1.1

Copyright (C) 2011 by
   Heiko Oberdiek <heiko.oberdiek at googlemail.com>

This work may be distributed and/or modified under the
conditions of the LaTeX Project Public License, either
version 1.3c of this license or (at your option) any later
version. This version of this license is in
   http://www.latex-project.org/lppl/lppl-1-3c.txt
and the latest version of this license is in
   http://www.latex-project.org/lppl.txt
and version 1.3 or later is part of all distributions of
LaTeX version 2005/12/01 or later.

This work has the LPPL maintenance status "maintained".

This Current Maintainer of this work is Heiko Oberdiek.

The Base Interpreter refers to any `TeX-Format',
because some files are installed in TDS:tex/generic//.

This work consists of the main source file thepdfnumber.dtx
and the derived files
   thepdfnumber.sty, thepdfnumber.pdf, thepdfnumber.ins, thepdfnumber.drv,
   thepdfnumber-test1.tex, thepdfnumber-test2.tex,
   thepdfnumber-test3.tex.

\endpreamble
\let\MetaPrefix\DoubleperCent

\generate{%
  \file{thepdfnumber.ins}{\from{thepdfnumber.dtx}{install}}%
  \file{thepdfnumber.drv}{\from{thepdfnumber.dtx}{driver}}%
  \usedir{tex/generic/oberdiek}%
  \file{thepdfnumber.sty}{\from{thepdfnumber.dtx}{package}}%
  \usedir{doc/latex/oberdiek/test}%
  \file{thepdfnumber-test1.tex}{\from{thepdfnumber.dtx}{test1}}%
  \file{thepdfnumber-test2.tex}{\from{thepdfnumber.dtx}{test2}}%
  \file{thepdfnumber-test3.tex}{\from{thepdfnumber.dtx}{test2,active}}%
}

\catcode32=13\relax% active space
\let =\space%
\Msg{************************************************************************}
\Msg{*}
\Msg{* To finish the installation you have to move the following}
\Msg{* file into a directory searched by TeX:}
\Msg{*}
\Msg{*     thepdfnumber.sty}
\Msg{*}
\Msg{* To produce the documentation run the file `thepdfnumber.drv'}
\Msg{* through LaTeX.}
\Msg{*}
\Msg{* Happy TeXing!}
\Msg{*}
\Msg{************************************************************************}

\endbatchfile
%</install>
%<*ignore>
\fi
%</ignore>
%<*driver>
\NeedsTeXFormat{LaTeX2e}
\ProvidesFile{thepdfnumber.drv}%
  [2016/05/16 v1.1 Print PDF numbers with minimal digits (HO)]%
\documentclass{ltxdoc}
\usepackage{holtxdoc}[2011/11/22]
\usepackage{csquotes}
\usepackage[
  backend=bibtex,
  bibencoding=ascii,
  alldates=iso8601,
]{biblatex}[2011/11/13]
\bibliography{oberdiek-source}
\begin{document}
  \DocInput{thepdfnumber.dtx}%
\end{document}
%</driver>
% \fi
%
%
% \CharacterTable
%  {Upper-case    \A\B\C\D\E\F\G\H\I\J\K\L\M\N\O\P\Q\R\S\T\U\V\W\X\Y\Z
%   Lower-case    \a\b\c\d\e\f\g\h\i\j\k\l\m\n\o\p\q\r\s\t\u\v\w\x\y\z
%   Digits        \0\1\2\3\4\5\6\7\8\9
%   Exclamation   \!     Double quote  \"     Hash (number) \#
%   Dollar        \$     Percent       \%     Ampersand     \&
%   Acute accent  \'     Left paren    \(     Right paren   \)
%   Asterisk      \*     Plus          \+     Comma         \,
%   Minus         \-     Point         \.     Solidus       \/
%   Colon         \:     Semicolon     \;     Less than     \<
%   Equals        \=     Greater than  \>     Question mark \?
%   Commercial at \@     Left bracket  \[     Backslash     \\
%   Right bracket \]     Circumflex    \^     Underscore    \_
%   Grave accent  \`     Left brace    \{     Vertical bar  \|
%   Right brace   \}     Tilde         \~}
%
% \GetFileInfo{thepdfnumber.drv}
%
% \title{The \xpackage{thepdfnumber} package}
% \date{2016/05/16 v1.1}
% \author{Heiko Oberdiek\thanks
% {Please report any issues at https://github.com/ho-tex/oberdiek/issues}\\
% \xemail{heiko.oberdiek at googlemail.com}}
%
% \maketitle
%
% \begin{abstract}
% The package converts real numbers to a minimal representation
% that is stripped from leading or trailing zeros,
% plus signs and decimal point if not necessary.
% \end{abstract}
%
% \tableofcontents
%
% \section{Documentation}
%
% \subsection{Introduction}
%
% Dealing with the PDF format, there is sometimes the need to write
% some low level PDF stuff. In case of numbers, the numbers can
% arise from user input (e.g. color or transparency specifications)
% or can be calculated. For example, \hologo{LaTeX}'s \cs{strip@pt}
% makes a good job to output a real number. It automatically suppresses
% the decimal part if the number is an integer. However it leaves
% a leading zero for numbers greater zero and smaller one.
% Thus the package provides macros that can be used with different
% formats, even with \hologo{iniTeX} and generates numbers that
% are valid numbers of the PDF format and whose length is minimal.
%
% \subsection{Usage}
%
% The package \xpackage{thepdfnumber} can be used with \hologo{LaTeX},
% \hologo{plainTeX} or even with \hologo{iniTeX}:
% \begin{quote}
%   |\RequirePackage{thepdfnumber}| \textit{\% \hologo{LaTeX}}\\
%   |\input thepdfnumber.sty| \textit{\% \hologo{plainTeX}/\hologo{iniTeX}}
% \end{quote}
% The package does not need and have package options.
%
% \subsection{User macros}
%
% All user macros are expandable in exact two expansion steps.
%
% \begin{declcs}{thepdfnumber} \M{number}
% \end{declcs}
% Macro \cs{thepdfnumber} takes a number
% as argument and expands to a minimal representation of that number.
% Some examples:
% \begin{quote}
% \begin{tabular}{r@{.: }>{\ttfamily}l@{ $\rightarrow$\ }>{\ttfamily}l@{}}
%   1& +123 & 123\\
%   2& --123 & 123\\
%   3& -01 & -1\\
%   4& 0045 & 45\\
%   5& 1.0 & 1\\
%   6& 1.20 & 1.2\\
%   7& 0.0 & 0\\
%   8& 0.78 & .78\\
%   9& +012.340 & 12.34\\
% \end{tabular}
% \end{quote}
% It reduces the length of the number representation:
% \begin{itemize}
% \item The signs are collapsed and only one minus sign
%   is output if the number is negative (see examples 1, 2, 3, 9).
% \item Leading zeros are removed (4, 8, 9) unless the number is zero (7).
% \item The decimal part is omitted, if the number is an integer (5, 7).
% \item Trailing zeros from the decimal part are stripped (5, 6, 7, 9).
% \end{itemize}
% The resulting number representation can be catched with one of the
% following Perl regular expressions:
% \begin{itemize}
% \item |^0$| (zero)
% \item |^-?[1-9][0-9]*$| (integer)
% \item |^-?[0-9]*\.[0-9]*[1-9]$| (real)
% \end{itemize}
% This is a valid numeric object of the PDF specification \cite[``7.3.3~Numeric Objects'']{pdfspec-iso32000-1}.
%
% \begin{declcs}{thepdfnumberNormZeroOne}
% \end{declcs}
% There are various places in the PDF specification where the
% number is in the domain 0.0 upto 1.0. Macro thepdfnumberNormZeroOne
% automatically adjusts the number to fit into that range. Negative
% numbers are mapped to |0| and numbers greater than one are
% replaced by |1|. Thus the result fits one of the following regular
% expressions:
% \begin{itemize}
% \item |^0$|
% \item |^\.[0-9]*[1-9]$|
% \item |^1$|
% \end{itemize}
% Examples:
% \begin{quote}
% \begin{tabular}{@{}>{\ttfamily}l@{ $\rightarrow$\ }>{\ttfamily}l@{}}
%   -456 & 0\\
%   -0.001 & 0\\
%   0.0 & 0\\
%   0.010 & .01\\
%   0.456 & .456\\
%   1.0 & 1\\
%   01.001 & 1\\
%   4 & 1\\
% \end{tabular}
% \end{quote}
%
% \subsection{Input number}
%
% The user macros expect a number as argument. The number can either
% be given explicitely or as macro that expands in one step
% to an explicite number, because the first token of the argument
% is expanded once.
%
% The explicite number consists of
% \begin{itemize}
% \item optional signs `\texttt{+}' and `\texttt{-}',
% \item digits `\texttt{0}' upto `\texttt{9}' and
% \item an optional dot `\texttt{.}'.
% \end{itemize}
% All tokens must have catcode 12 (other), the default catcodes
% for these characters in \hologo{LaTeX}, \hologo{plainTeX} or
% \hologo{iniTeX}. As Perl regular expression the number is expected
% in one of the following forms:
% \begin{itemize}
% \item |^[+-]*[0-9]+$|
% \item |^[+-]*[0-9]*\.[0-9]*$|
% \end{itemize}
% At least one digit or the dot must be present.
%
% \subsection{Error handling}
%
% The package is not intended for validating numbers or to
% decide if an argument is a number. Therefore it is an usage
% error to use the user macros with arguments that are not
% explicite numbers as specified in the previous sections.
% Nevertheless some error conditions are sometimes recognized.
% Errors are given in form of an undefined command sequence.
% It is the only way to notify \hologo{TeX} in expandable context.
% Expanding to some error text would invalidate the output.
% Currently the following errors are thrown:
% \begin{description}
% \item[\cs{thepdfnumber@ErrorEndMarker}:]
%   Internally the argument parsing uses an end marker that
%   is never called directly. If it is called with valid
%   user input, then this is a bug. Otherwise it means the user input
%   contains nasty stuff.
% \item[\cs{thepdfnumber@ErrorUnexpectedEnd}:] The macros expect at least one digit
%   or the dot, otherwise if the argument is empty or only contains
%   signs, then this error is called.
% \item[\cs{thepdfnumber@ErrorInvalidToken}:] It is called if the number
%   contains other tokens than signs, digits or the dot or the
%   token at the wrong place (e.g. a sign after a digit).
%   In case of \cs{thepdfnumberNormZeroOne} this error condition might
%   not always be detected, because the number parsing might stop
%   at an early point, when the result is already clear (e.g.
%   if the number is negative or will be greater than one).
% \item[Improper alphabetic constant:] This error might be
%   thrown by \hologo{TeX}, if the number contains
%   command tokens instead of characters.
% \end{description}
%
% \StopEventually{
% }
%
% \section{Implementation}
%
%    \begin{macrocode}
%<*package>
%    \end{macrocode}
%
% \subsection{Reload check and package identification}
%    Reload check, especially if the package is not used with \LaTeX.
%    \begin{macrocode}
\begingroup\catcode61\catcode48\catcode32=10\relax%
  \catcode13=5 % ^^M
  \endlinechar=13 %
  \catcode35=6 % #
  \catcode39=12 % '
  \catcode44=12 % ,
  \catcode45=12 % -
  \catcode46=12 % .
  \catcode58=12 % :
  \catcode64=11 % @
  \catcode123=1 % {
  \catcode125=2 % }
  \expandafter\let\expandafter\x\csname ver@thepdfnumber.sty\endcsname
  \ifx\x\relax % plain-TeX, first loading
  \else
    \def\empty{}%
    \ifx\x\empty % LaTeX, first loading,
      % variable is initialized, but \ProvidesPackage not yet seen
    \else
      \expandafter\ifx\csname PackageInfo\endcsname\relax
        \def\x#1#2{%
          \immediate\write-1{Package #1 Info: #2.}%
        }%
      \else
        \def\x#1#2{\PackageInfo{#1}{#2, stopped}}%
      \fi
      \x{thepdfnumber}{The package is already loaded}%
      \aftergroup\endinput
    \fi
  \fi
\endgroup%
%    \end{macrocode}
%    Package identification:
%    \begin{macrocode}
\begingroup\catcode61\catcode48\catcode32=10\relax%
  \catcode13=5 % ^^M
  \endlinechar=13 %
  \catcode35=6 % #
  \catcode39=12 % '
  \catcode40=12 % (
  \catcode41=12 % )
  \catcode44=12 % ,
  \catcode45=12 % -
  \catcode46=12 % .
  \catcode47=12 % /
  \catcode58=12 % :
  \catcode64=11 % @
  \catcode91=12 % [
  \catcode93=12 % ]
  \catcode123=1 % {
  \catcode125=2 % }
  \expandafter\ifx\csname ProvidesPackage\endcsname\relax
    \def\x#1#2#3[#4]{\endgroup
      \immediate\write-1{Package: #3 #4}%
      \xdef#1{#4}%
    }%
  \else
    \def\x#1#2[#3]{\endgroup
      #2[{#3}]%
      \ifx#1\@undefined
        \xdef#1{#3}%
      \fi
      \ifx#1\relax
        \xdef#1{#3}%
      \fi
    }%
  \fi
\expandafter\x\csname ver@thepdfnumber.sty\endcsname
\ProvidesPackage{thepdfnumber}%
  [2016/05/16 v1.1 Print PDF numbers with minimal digits (HO)]%
%    \end{macrocode}
%
% \subsection{Catcodes}
%
%    \begin{macrocode}
\begingroup\catcode61\catcode48\catcode32=10\relax%
  \catcode13=5 % ^^M
  \endlinechar=13 %
  \catcode123=1 % {
  \catcode125=2 % }
  \catcode64=11 % @
  \def\x{\endgroup
    \expandafter\edef\csname ThPdNu@AtEnd\endcsname{%
      \endlinechar=\the\endlinechar\relax
      \catcode13=\the\catcode13\relax
      \catcode32=\the\catcode32\relax
      \catcode35=\the\catcode35\relax
      \catcode61=\the\catcode61\relax
      \catcode64=\the\catcode64\relax
      \catcode123=\the\catcode123\relax
      \catcode125=\the\catcode125\relax
    }%
  }%
\x\catcode61\catcode48\catcode32=10\relax%
\catcode13=5 % ^^M
\endlinechar=13 %
\catcode35=6 % #
\catcode64=11 % @
\catcode123=1 % {
\catcode125=2 % }
\def\TMP@EnsureCode#1#2{%
  \edef\ThPdNu@AtEnd{%
    \ThPdNu@AtEnd
    \catcode#1=\the\catcode#1\relax
  }%
  \catcode#1=#2\relax
}
\TMP@EnsureCode{33}{12}% !
\TMP@EnsureCode{36}{3}% $
\TMP@EnsureCode{38}{4}% &
\TMP@EnsureCode{42}{12}% *
\TMP@EnsureCode{43}{12}% +
\TMP@EnsureCode{45}{12}% -
\TMP@EnsureCode{46}{12}% .
\TMP@EnsureCode{60}{12}% <
\TMP@EnsureCode{62}{12}% >
\TMP@EnsureCode{96}{12}% `
\edef\ThPdNu@AtEnd{\ThPdNu@AtEnd\noexpand\endinput}
%    \end{macrocode}
%
% \subsection{Helper macros}
%
%    \begin{macro}{\ThPdNu@FIN}
%    \begin{macrocode}
\def\ThPdNu@FIN{\thepdfnumber@ErrorEndMarker}
%    \end{macrocode}
%    \end{macro}
%
%    \begin{macro}{\ThPdNu@space}
%    \begin{macrocode}
\def\ThPdNu@space{ }
%    \end{macrocode}
%    \end{macro}
%
%    \begin{macro}{\ThPdNu@zero}
%    \begin{macrocode}
\chardef\ThPdNu@zero=0 %
%    \end{macrocode}
%    \end{macro}
%
%    \begin{macro}{\ThPdNu@one}
%    \begin{macrocode}
\chardef\ThPdNu@one=1 %
%    \end{macrocode}
%    \end{macro}
%
%    \begin{macro}{\ThPdNu@firstoftwo}
%    \begin{macrocode}
\long\def\ThPdNu@firstoftwo#1#2{#1}
%    \end{macrocode}
%    \end{macro}
%
%    \begin{macro}{\ThPdNu@secondoftwo}
%    \begin{macrocode}
\long\def\ThPdNu@secondoftwo#1#2{#2}
%    \end{macrocode}
%    \end{macro}
%
% \subsection{Detect \hologo{eTeX}}
%
%    \begin{macrocode}
\begingroup\expandafter\expandafter\expandafter\endgroup
\expandafter\ifx\csname detokenize\endcsname\relax
  \catcode`\&=14 %
  \catcode`\$=9 %
\else
  \catcode`\&=9 %
  \catcode`\$=14 %
\fi
%    \end{macrocode}
%
% \subsection{User macro \cs{thepdfnumber}}
%
%    \begin{macro}{\thepdfnumber}
%    \begin{macrocode}
\def\thepdfnumber#1{%
  \romannumeral
& \iftrue\expandafter\ThPdNu@State@Plus\expandafter\fi
& \detokenize\expandafter{#1}%
& \ThPdNu@FIN
$ \ifx\ThPdNu@FIN#1\ThPdNu@FIN
$   \expandafter\ThPdNu@firstoftwo
$ \else
$   \expandafter\ThPdNu@secondoftwo
$ \fi
$ {%
$   \ThPdNu@zero
$   0\thepdfnumber@ErrorUnexpectedEnd
$ }{%
$   \iftrue\expandafter\ThPdNu@State@Plus\expandafter\fi#1\ThPdNu@FIN
$ }%
}
%    \end{macrocode}
%    \end{macro}
%
% \subsubsection{State definitions for sign}
%
%    \begin{macro}{\ThPdNu@State@Plus}
%    \begin{macrocode}
\def\ThPdNu@State@Plus#1\fi#2{%
  \fi
  \ifcase\ifx\ThPdNu@FIN#2%
           0%
&        \else\ifx-#2%
$        \else\ifnum`#2=45 % -
           1%
         \else\ifx0#2%
           2%
         \else\ifnum`#2>48 %
           \ifnum`#2<58 %
             3%
           \else
             9%
           \fi
&        \else\ifx.#2%
$        \else\ifnum`#2=46 % .
           4%
&        \else\ifx+#2%
$        \else\ifnum`#2=43 % +
           5%
         \else
           9%
         \fi\fi\fi\fi\fi\fi\ThPdNu@space
    \expandafter\ThPdNu@zero
    \expandafter0%
    \expandafter\thepdfnumber@ErrorUnexpectedEnd
  \or
    \ThPdNu@State@Minus
  \or
    \ThPdNu@State@SkipZeros!%
  \or
    \ThPdNu@State@Int!#2!%
  \or
    \ThPdNu@State@Dot!\ThPdNu@zero*\ThPdNu@zero!!%
  \or
    \ThPdNu@State@Plus
  \else
    \ThPdNu@ReturnError{0}%
  \fi
}
%    \end{macrocode}
%    \end{macro}
%
%    \begin{macro}{\ThPdNu@State@Minus}
%    \begin{macrocode}
\def\ThPdNu@State@Minus#1\fi#2{%
  \fi
  \ifcase\ifx\ThPdNu@FIN#2%
           0%
         \else\ifx0#2%
           1%
         \else\ifnum`#2>48 %
           \ifnum`#2<58 %
             2%
           \else
             9%
           \fi
&        \else\ifx.#2%
$        \else\ifnum`#2=46 % .
           3%
&        \else\ifx-#2%
$        \else\ifnum`#2=45 % -
           4%
&        \else\ifx+#2%
$        \else\ifnum`#2=43 % +
           5%
         \else
           9%
         \fi\fi\fi\fi\fi\fi\ThPdNu@space
    \expandafter\ThPdNu@zero
    \expandafter0%
    \expandafter\thepdfnumber@ErrorUnexpectedEnd
  \or
    \ThPdNu@State@SkipZeros-!%
  \or
    \ThPdNu@State@Int-!#2!%
  \or
    \ThPdNu@State@Dot-!\ThPdNu@zero*\ThPdNu@zero!!%
  \or
    \ThPdNu@State@Plus
  \or
    \ThPdNu@State@Minus
  \else
    \ThPdNu@ReturnError{0}%
  \fi
}
%    \end{macrocode}
%    \end{macro}
%
%    \begin{macro}{\ThPdNu@ReturnError}
%    \begin{macrocode}
\def\ThPdNu@ReturnError#1#2\fi#3\ThPdNu@FIN{%
  \fi
  \ThPdNu@zero
  #1%
  \thepdfnumber@ErrorInvalidToken
}
%    \end{macrocode}
%    \end{macro}
%
% \subsubsection{State definitions for integer part}
%
%    \begin{macro}{\ThPdNu@State@SkipZeros}
%    \begin{macrocode}
\def\ThPdNu@State@SkipZeros#1!#2\fi#3{%
  \fi
  \ifcase\ifx\ThPdNu@FIN#3%
           0%
         \else\ifx0#3%
           1%
         \else\ifnum`#3>48 %
           \ifnum`#3<58 %
             2%
           \else
             9%
           \fi
&        \else\ifx.#3%
$        \else\ifnum`#3=46 % .
           3%
         \else
           9%
         \fi\fi\fi\fi\ThPdNu@space
    \expandafter\ThPdNu@zero
    \expandafter0%
  \or
   \ThPdNu@State@SkipZeros#1!%
  \or
    \ThPdNu@State@Int#1!#3!%
  \or
    \ThPdNu@State@Dot#1!\ThPdNu@zero*\ThPdNu@zero!!%
  \else
    \ThPdNu@ReturnError{0}%
  \fi
}
%    \end{macrocode}
%    \end{macro}
%
%    \begin{macro}{\ThPdNu@State@Int}
%    \begin{macrocode}
\def\ThPdNu@State@Int#1!#2!#3\fi#4{%
  \fi
  \ifcase\ifx\ThPdNu@FIN#4%
           0%
         \else\ifnum`#4>47 %
           \ifnum`#4<58 %
             1%
           \else
             9%
           \fi
&        \else\ifx.#4%
$        \else\ifnum`#4=46 % .
           2%
         \else
           9%
         \fi\fi\fi\ThPdNu@space
    \ThPdNu@ReturnInt{#1#2}%
  \or
    \ThPdNu@State@Int#1!#2#4!%
  \or
    \ThPdNu@State@Dot#1!\ThPdNu@one#2*\ThPdNu@zero!!%
  \else
    \ThPdNu@ReturnError{#1#2}%
  \fi
}
%    \end{macrocode}
%    \end{macro}
%
%    \begin{macro}{\ThPdNu@ReturnInt}
%    \begin{macrocode}
\def\ThPdNu@ReturnInt#1#2\fi{%
  \fi
  \ThPdNu@zero
  #1%
}
%    \end{macrocode}
%    \end{macro}
%
% \subsubsection{State definitions for decimal digits}
%
%    \begin{macro}{\ThPdNu@State@Dot}
%    \begin{macrocode}
\def\ThPdNu@State@Dot#1*#2#3!#4!#5\fi#6{%
  \fi
  \ifcase\ifx\ThPdNu@FIN#6%
           0%
         \else\ifnum`#6>48 %
           \ifnum`#6<58 %
             1%
           \else
             9%
           \fi
         \else\ifx0#6%
           2%
         \else
           9%
         \fi\fi\fi\ThPdNu@space
    \ThPdNu@ReturnNumber#1*#2#3!%
  \or
    \ThPdNu@State@Dot#1*\ThPdNu@one#3#4#6!!%
  \or
    \ThPdNu@State@DotZero#1*#2#3!#4#6!%
  \else
    \ThPdNu@ReturnNumberInvalid#1*#2#3!%
  \fi
}
%    \end{macrocode}
%    \end{macro}
%
%    \begin{macro}{\ThPdNu@State@DotZero}
%    \begin{macrocode}
\def\ThPdNu@State@DotZero#1*#2#3!#4!#5\fi#6{%
  \fi
  \ifcase\ifx\ThPdNu@FIN#6%
           0%
         \else\ifnum`#6>48 %
           \ifnum`#6<58 %
             1%
           \else
             9%
           \fi
         \else\ifx0#6%
           2%
         \else
           9%
         \fi\fi\fi\ThPdNu@space
    \ThPdNu@ReturnNumber#1*#2#3!%
  \or
    \ThPdNu@State@Dot#1*\ThPdNu@one#3#4#6!!%
  \or
    \ThPdNu@State@DotZero#1*#2#3!#4#6!%
  \else
    \ThPdNu@ReturnNumber#1*#2#3!%
  \fi
}
%    \end{macrocode}
%    \end{macro}
%
%    \begin{macro}{\ThPdNu@ReturnNumber}
%    \begin{macrocode}
\def\ThPdNu@ReturnNumber#1!#2#3*#4#5!#6\fi{%
  \fi
  \ifcase#2%
    \expandafter\ThPdNu@firstoftwo
  \else
    \expandafter\ThPdNu@secondoftwo
  \fi
  {%
    \ifcase#4%
      \expandafter\ThPdNu@firstoftwo
    \else
      \expandafter\ThPdNu@secondoftwo
    \fi
    {\ThPdNu@zero 0}%
    {\ThPdNu@zero #1.#5}%
  }{%
    \ifcase#4%
      \expandafter\ThPdNu@firstoftwo
    \else
      \expandafter\ThPdNu@secondoftwo
    \fi
    {\ThPdNu@zero #1#3}%
    {\ThPdNu@zero #1#3.#5}%
  }%
}
%    \end{macrocode}
%    \end{macro}
%
%    \begin{macro}{\ThPdNu@ReturnNumberInvalid}
%    \begin{macrocode}
\def\ThPdNu@ReturnNumberInvalid#1*#2!#3\fi#4\ThPdNu@FIN{%
  \fi
  \iftrue\ThPdNu@ReturnNumber#1*#2!\fi
  \thepdfnumber@ErrorInvalidToken
}
%    \end{macrocode}
%    \end{macro}
%
% \subsection{Norm macro}
%
%    \begin{macro}{\thepdfnumberNormZeroOne}
%    \begin{macrocode}
\def\thepdfnumberNormZeroOne#1{%
  \romannumeral
& \iftrue\expandafter\ThPbNu@StateN@Plus\expandafter\fi
& \detokenize\expandafter{#1}%
& \ThPdNu@FIN
$ \ifx\ThPdNu@FIN#1\ThPdNu@FIN
$   \expandafter\ThPdNu@firstoftwo
$ \else
$   \expandafter\ThPdNu@secondoftwo
$ \fi
$ {%
$   \ThPdNu@zero
$   0\thepdfnumber@ErrorUnexpectedEnd
$ }{%
$   \iftrue\expandafter\ThPbNu@StateN@Plus\expandafter\fi#1\ThPdNu@FIN
$ }%
}
%    \end{macrocode}
%    \end{macro}
%
% \subsubsection{State definitions for sign}
%
%    \begin{macro}{\ThPbNu@StateN@Plus}
%    \begin{macrocode}
\def\ThPbNu@StateN@Plus#1\fi#2{%
  \fi
  \ifcase\ifx\ThPdNu@FIN#2%
           0%
&        \else\ifx-#2%
$        \else\ifnum`#2=45 % -
           1%
         \else\ifx0#2%
           2%
         \else\ifnum`#2>48 %
           \ifnum`#2<58 %
             3%
           \else
             9%
           \fi
&        \else\ifx.#2%
$        \else\ifnum`#2=46 % .
           4%
&        \else\ifx+#2%
$        \else\ifnum`#2=43 % +
           5%
         \else
           9%
         \fi\fi\fi\fi\fi\fi\ThPdNu@space
    \expandafter\ThPdNu@zero
    \expandafter0%
    \expandafter\thepdfnumber@ErrorUnexpectedEnd
  \or
    \ThPbNu@StateN@Minus
  \or
    \ThPbNu@StateN@SkipZeros
  \or
    \ThPdNu@ReturnAndSkip{1}%
  \or
    \ThPbNu@StateN@Dot\ThPdNu@zero!!%
  \or
    \ThPbNu@StateN@Plus
  \else
    \ThPdNu@ReturnError{0}%
  \fi
}
%    \end{macrocode}
%    \end{macro}
%
%    \begin{macro}{\ThPbNu@StateN@Minus}
%    \begin{macrocode}
\def\ThPbNu@StateN@Minus#1\fi#2{%
  \fi
  \ifcase\ifx\ThPdNu@FIN#2%
           0%
         \else\ifnum`#2>47 %
           \ifnum`#2<58 %
             1%
           \else
             9%
           \fi
&        \else\ifx.#2%
$        \else\ifnum`#2=46 % .
           1%
&        \else\ifx-#2%
$        \else\ifnum`#2=45 % -
           2%
&        \else\ifx+#2%
$        \else\ifnum`#2=43 % +
           3%
         \else
           9%
         \fi\fi\fi\fi\fi\ThPdNu@space
    \expandafter\ThPdNu@zero
    \expandafter0%
    \expandafter\thepdfnumber@ErrorUnexpectedEnd
  \or
    \ThPdNu@ReturnAndSkip{0}%
  \or
    \ThPbNu@StateN@Plus
  \or
    \ThPbNu@StateN@Minus
  \else
    \ThPdNu@ReturnError{0}%
  \fi
}
%    \end{macrocode}
%    \end{macro}
%
%    \begin{macro}{\ThPdNu@ReturnAndSkip}
%    \begin{macrocode}
\def\ThPdNu@ReturnAndSkip#1#2\fi#3\ThPdNu@FIN{%
  \fi
  \ThPdNu@zero
  #1%
}
%    \end{macrocode}
%    \end{macro}
%
% \subsubsection{State definitions for integer part}
%
%    \begin{macro}{\ThPbNu@StateN@SkipZeros}
%    \begin{macrocode}
\def\ThPbNu@StateN@SkipZeros#1\fi#2{%
  \fi
  \ifcase\ifx\ThPdNu@FIN#2%
           0%
         \else\ifx0#2%
           1%
         \else\ifnum`#2>48 %
           \ifnum`#2<58 %
             2%
           \else
             9%
           \fi
&        \else\ifx.#2%
$        \else\ifnum`#2=46 % .
           3%
         \else
           9%
         \fi\fi\fi\fi\ThPdNu@space
    \expandafter\ThPdNu@zero
    \expandafter0%
  \or
   \ThPbNu@StateN@SkipZeros%
  \or
    \ThPdNu@ReturnAndSkip{1}%
  \or
    \ThPbNu@StateN@Dot\ThPdNu@zero!!%
  \else
    \ThPdNu@ReturnError{0}%
  \fi
}
%    \end{macrocode}
%    \end{macro}
%
% \subsubsection{State definitions for decimal digits}
%
%    \begin{macro}{\ThPbNu@StateN@Dot}
%    \begin{macrocode}
\def\ThPbNu@StateN@Dot#1#2!#3!#4\fi#5{%
  \fi
  \ifcase\ifx\ThPdNu@FIN#5%
           0%
         \else\ifnum`#5>48 %
           \ifnum`#5<58 %
             1%
           \else
             9%
           \fi
         \else\ifx0#5%
           2%
         \else
           9%
         \fi\fi\fi\ThPdNu@space
    \ThPdNu@ReturnFracNumber#1#2!%
  \or
    \ThPbNu@StateN@Dot\ThPdNu@one#2#3#5!!%
  \or
    \ThPbNu@StateN@DotZero#1#2!#3#5!%
  \else
    \ThPdNu@ReturnFracNumberInvalid#1#2!%
  \fi
}
%    \end{macrocode}
%    \end{macro}
%
%    \begin{macro}{\ThPbNu@StateN@DotZero}
%    \begin{macrocode}
\def\ThPbNu@StateN@DotZero#1#2!#3!#4\fi#5{%
  \fi
  \ifcase\ifx\ThPdNu@FIN#5%
           0%
         \else\ifnum`#5>48 %
           \ifnum`#5<58 %
             1%
           \else
             9%
           \fi
         \else\ifx0#5%
           2%
         \else
           9%
         \fi\fi\fi\ThPdNu@space
    \ThPdNu@ReturnFracNumber#1#2!%
  \or
    \ThPbNu@StateN@Dot\ThPdNu@one#2#3#5!!%
  \or
    \ThPbNu@StateN@DotZero#1#2!#3#5!%
  \else
    \ThPdNu@ReturnFracNumberInvalid#1#2!%
  \fi
}
%    \end{macrocode}
%    \end{macro}
%
%    \begin{macro}{\ThPdNu@ReturnFracNumber}
%    \begin{macrocode}
\def\ThPdNu@ReturnFracNumber#1#2!#3\fi{%
  \fi
  \ifcase#1%
    \expandafter\ThPdNu@firstoftwo
  \else
    \expandafter\ThPdNu@secondoftwo
  \fi
  {\ThPdNu@zero 0}%
  {\ThPdNu@zero .#2}%
}
%    \end{macrocode}
%    \end{macro}
%
%    \begin{macro}{\ThPdNu@ReturnFracNumberInvalid}
%    \begin{macrocode}
\def\ThPdNu@ReturnFracNumberInvalid#1!#2\fi#3\ThPdNu@FIN{%
  \fi
  \iftrue\ThPdNu@ReturnFracNumber#1!\fi
  \thepdfnumber@ErrorInvalidToken
}
%    \end{macrocode}
%    \end{macro}
%
%    \begin{macrocode}
\ThPdNu@AtEnd%
%</package>
%    \end{macrocode}
%
% \section{Test}
%
% \subsection{Catcode checks for loading}
%
%    \begin{macrocode}
%<*test1>
%    \end{macrocode}
%    \begin{macrocode}
\catcode`\{=1 %
\catcode`\}=2 %
\catcode`\#=6 %
\catcode`\@=11 %
\expandafter\ifx\csname count@\endcsname\relax
  \countdef\count@=255 %
\fi
\expandafter\ifx\csname @gobble\endcsname\relax
  \long\def\@gobble#1{}%
\fi
\expandafter\ifx\csname @firstofone\endcsname\relax
  \long\def\@firstofone#1{#1}%
\fi
\expandafter\ifx\csname loop\endcsname\relax
  \expandafter\@firstofone
\else
  \expandafter\@gobble
\fi
{%
  \def\loop#1\repeat{%
    \def\body{#1}%
    \iterate
  }%
  \def\iterate{%
    \body
      \let\next\iterate
    \else
      \let\next\relax
    \fi
    \next
  }%
  \let\repeat=\fi
}%
\def\RestoreCatcodes{}
\count@=0 %
\loop
  \edef\RestoreCatcodes{%
    \RestoreCatcodes
    \catcode\the\count@=\the\catcode\count@\relax
  }%
\ifnum\count@<255 %
  \advance\count@ 1 %
\repeat

\def\RangeCatcodeInvalid#1#2{%
  \count@=#1\relax
  \loop
    \catcode\count@=15 %
  \ifnum\count@<#2\relax
    \advance\count@ 1 %
  \repeat
}
\def\RangeCatcodeCheck#1#2#3{%
  \count@=#1\relax
  \loop
    \ifnum#3=\catcode\count@
    \else
      \errmessage{%
        Character \the\count@\space
        with wrong catcode \the\catcode\count@\space
        instead of \number#3%
      }%
    \fi
  \ifnum\count@<#2\relax
    \advance\count@ 1 %
  \repeat
}
\def\space{ }
\expandafter\ifx\csname LoadCommand\endcsname\relax
  \def\LoadCommand{\input thepdfnumber.sty\relax}%
\fi
\def\Test{%
  \RangeCatcodeInvalid{0}{47}%
  \RangeCatcodeInvalid{58}{64}%
  \RangeCatcodeInvalid{91}{96}%
  \RangeCatcodeInvalid{123}{255}%
  \catcode`\@=12 %
  \catcode`\\=0 %
  \catcode`\%=14 %
  \LoadCommand
  \RangeCatcodeCheck{0}{36}{15}%
  \RangeCatcodeCheck{37}{37}{14}%
  \RangeCatcodeCheck{38}{47}{15}%
  \RangeCatcodeCheck{48}{57}{12}%
  \RangeCatcodeCheck{58}{63}{15}%
  \RangeCatcodeCheck{64}{64}{12}%
  \RangeCatcodeCheck{65}{90}{11}%
  \RangeCatcodeCheck{91}{91}{15}%
  \RangeCatcodeCheck{92}{92}{0}%
  \RangeCatcodeCheck{93}{96}{15}%
  \RangeCatcodeCheck{97}{122}{11}%
  \RangeCatcodeCheck{123}{255}{15}%
  \RestoreCatcodes
}
\Test
\csname @@end\endcsname
\end
%    \end{macrocode}
%    \begin{macrocode}
%</test1>
%    \end{macrocode}
%
% \subsection{Macro tests}
%
% \subsubsection{Preamble}
%
%    \begin{macrocode}
%<*test2>
\catcode`\{=1
\catcode`\}=2
\catcode`\#=6
\catcode`\@=11
\errorcontextlines=10000 %
\def\msg#{\immediate\write16}
\def\space{ }
\begingroup\expandafter\expandafter\expandafter\endgroup
\expandafter\ifx\csname RequirePackage\endcsname\relax
  \input thepdfnumber.sty\relax
\else
  \RequirePackage{thepdfnumber}[2016/05/16]%
\fi
%    \end{macrocode}
%
%    \begin{macro}{\Test}
%    \begin{macrocode}
\def\Test#1#2{%
  \def\TestExpected{#2}%
%<*active>
  \edef\TestExpected{\TestExpected}%
%</active>
  \expandafter\expandafter\expandafter\def
  \expandafter\expandafter\expandafter\TestResult
  \expandafter\expandafter\expandafter{%
    \TestCommand{#1}%
  }%
  \begingroup
    \toks0\expandafter{\TestResult}%
    \toks2\expandafter{\TestExpected}%
    \ifx\TestResult\TestExpected
      \msg{* OK: #1 => \the\toks0}%
    \else
      \errmessage{FAILED: #1 => \the\toks0\space(\the\toks2)}%
    \fi
  \endgroup
}
%    \end{macrocode}
%    \end{macro}
%
%    \begin{macrocode}
\let\thepdfnumber@ErrorUnexpectedEnd\relax
\let\thepdfnumber@ErrorInvalidToken\relax
%    \end{macrocode}
%
% \subsubsection{Test with active catcodes}
%
%    \begin{macrocode}
%<*active>
\catcode`\~=13 %
\def\ThPdNu@Temp#1{%
  \begingroup
    \lccode`\~=`#1 %
  \lowercase{\endgroup
    \edef~%
  }{\string#1}%
  \catcode`#1=13 %
}
\ThPdNu@Temp{.}
\ThPdNu@Temp{-}
\ThPdNu@Temp{+}
%</active>
%    \end{macrocode}
%
% \subsubsection{Test cases for \cs{thepdfnumber}}
%
%    \begin{macro}{\TestCommand}
%    \begin{macrocode}
\let\TestCommand\thepdfnumber
%    \end{macrocode}
%    \end{macro}
%
%    \begin{macrocode}
\Test{0}{0}
\Test{1}{1}
\Test{23}{23}
\Test{0045}{45}
\Test{4500}{4500}
\Test{-0}{0}
\Test{-1}{-1}
\Test{-00002}{-2}
\Test{-010203}{-10203}
\Test{+0}{0}
\Test{+123}{123}

\Test{+-4}{-4}
\Test{+-+-4}{4}
\Test{++--++--++4}{4}
\Test{+++---+++---+++---4}{-4}

\Test{.}{0}
\Test{0.}{0}
\Test{.0}{0}
\Test{0.}{0}
\Test{-.}{0}
\Test{-0.}{0}
\Test{-.0}{0}
\Test{-0.}{0}
\Test{010.020}{10.02}
\Test{123.456}{123.456}
\Test{12.}{12}
\Test{.123}{.123}
\Test{.0001}{.0001}
\Test{.00100}{.001}
\Test{.12003400560078009}{.12003400560078009}
\Test{-.12003400560078009}{-.12003400560078009}
\Test{04.0500000}{4.05}

\Test{}{0\thepdfnumber@ErrorUnexpectedEnd}
\Test{+}{0\thepdfnumber@ErrorUnexpectedEnd}
\Test{-}{0\thepdfnumber@ErrorUnexpectedEnd}
\Test{a}{0\thepdfnumber@ErrorInvalidToken}
\Test{0x1}{0\thepdfnumber@ErrorInvalidToken}
\Test{4x56}{4\thepdfnumber@ErrorInvalidToken}
\Test{012,34}{12\thepdfnumber@ErrorInvalidToken}
\Test{0.12x4}{.12\thepdfnumber@ErrorInvalidToken}
%    \end{macrocode}
%
% \subsubsection{Test cases for \cs{thepdfnumberNormZeroOne}}
%
%    \begin{macro}{\TestCommand}
%    \begin{macrocode}
\let\TestCommand\thepdfnumberNormZeroOne
%    \end{macrocode}
%    \end{macro}
%
%    \begin{macrocode}
\Test{0}{0}
\Test{1}{1}
\Test{23}{1}
\Test{12}{1}
\Test{10}{1}
\Test{0045}{1}
\Test{001}{1}
\Test{-0}{0}
\Test{-1}{0}
\Test{-01}{0}
\Test{-4}{0}

\Test{+-0}{0}
\Test{+-+-1}{1}
\Test{++--++--++1}{1}
\Test{+++---+++---+++---1}{0}

\Test{.}{0}
\Test{0.}{0}
\Test{.0}{0}
\Test{0.}{0}
\Test{-.}{0}
\Test{-0.}{0}
\Test{-.0}{0}
\Test{-0.}{0}
\Test{010.020}{1}
\Test{123.456}{1}
\Test{12.}{1}
\Test{.123}{.123}
\Test{.0001}{.0001}
\Test{.00100}{.001}
\Test{.12003400560078009}{.12003400560078009}
\Test{-.12003400560078009}{0}
\Test{04.0500000}{1}
\Test{0.1200340056}{.1200340056}
\Test{1.05}{1}

\Test{}{0\thepdfnumber@ErrorUnexpectedEnd}
\Test{+}{0\thepdfnumber@ErrorUnexpectedEnd}
\Test{-}{0\thepdfnumber@ErrorUnexpectedEnd}
\Test{a}{0\thepdfnumber@ErrorInvalidToken}
\Test{0x1}{0\thepdfnumber@ErrorInvalidToken}
\Test{4x56}{1}
\Test{012,34}{1}
\Test{.012x4}{.012\thepdfnumber@ErrorInvalidToken}
%    \end{macrocode}
%
%    \begin{macrocode}
\csname @@end\endcsname\end
%</test2>
%    \end{macrocode}
%
% \section{Installation}
%
% \subsection{Download}
%
% \paragraph{Package.} This package is available on
% CTAN\footnote{\url{http://ctan.org/pkg/thepdfnumber}}:
% \begin{description}
% \item[\CTAN{macros/latex/contrib/oberdiek/thepdfnumber.dtx}] The source file.
% \item[\CTAN{macros/latex/contrib/oberdiek/thepdfnumber.pdf}] Documentation.
% \end{description}
%
%
% \paragraph{Bundle.} All the packages of the bundle `oberdiek'
% are also available in a TDS compliant ZIP archive. There
% the packages are already unpacked and the documentation files
% are generated. The files and directories obey the TDS standard.
% \begin{description}
% \item[\CTAN{install/macros/latex/contrib/oberdiek.tds.zip}]
% \end{description}
% \emph{TDS} refers to the standard ``A Directory Structure
% for \TeX\ Files'' (\CTAN{tds/tds.pdf}). Directories
% with \xfile{texmf} in their name are usually organized this way.
%
% \subsection{Bundle installation}
%
% \paragraph{Unpacking.} Unpack the \xfile{oberdiek.tds.zip} in the
% TDS tree (also known as \xfile{texmf} tree) of your choice.
% Example (linux):
% \begin{quote}
%   |unzip oberdiek.tds.zip -d ~/texmf|
% \end{quote}
%
% \paragraph{Script installation.}
% Check the directory \xfile{TDS:scripts/oberdiek/} for
% scripts that need further installation steps.
% Package \xpackage{attachfile2} comes with the Perl script
% \xfile{pdfatfi.pl} that should be installed in such a way
% that it can be called as \texttt{pdfatfi}.
% Example (linux):
% \begin{quote}
%   |chmod +x scripts/oberdiek/pdfatfi.pl|\\
%   |cp scripts/oberdiek/pdfatfi.pl /usr/local/bin/|
% \end{quote}
%
% \subsection{Package installation}
%
% \paragraph{Unpacking.} The \xfile{.dtx} file is a self-extracting
% \docstrip\ archive. The files are extracted by running the
% \xfile{.dtx} through \plainTeX:
% \begin{quote}
%   \verb|tex thepdfnumber.dtx|
% \end{quote}
%
% \paragraph{TDS.} Now the different files must be moved into
% the different directories in your installation TDS tree
% (also known as \xfile{texmf} tree):
% \begin{quote}
% \def\t{^^A
% \begin{tabular}{@{}>{\ttfamily}l@{ $\rightarrow$ }>{\ttfamily}l@{}}
%   thepdfnumber.sty & tex/generic/oberdiek/thepdfnumber.sty\\
%   thepdfnumber.pdf & doc/latex/oberdiek/thepdfnumber.pdf\\
%   test/thepdfnumber-test1.tex & doc/latex/oberdiek/test/thepdfnumber-test1.tex\\
%   test/thepdfnumber-test2.tex & doc/latex/oberdiek/test/thepdfnumber-test2.tex\\
%   test/thepdfnumber-test3.tex & doc/latex/oberdiek/test/thepdfnumber-test3.tex\\
%   thepdfnumber.dtx & source/latex/oberdiek/thepdfnumber.dtx\\
% \end{tabular}^^A
% }^^A
% \sbox0{\t}^^A
% \ifdim\wd0>\linewidth
%   \begingroup
%     \advance\linewidth by\leftmargin
%     \advance\linewidth by\rightmargin
%   \edef\x{\endgroup
%     \def\noexpand\lw{\the\linewidth}^^A
%   }\x
%   \def\lwbox{^^A
%     \leavevmode
%     \hbox to \linewidth{^^A
%       \kern-\leftmargin\relax
%       \hss
%       \usebox0
%       \hss
%       \kern-\rightmargin\relax
%     }^^A
%   }^^A
%   \ifdim\wd0>\lw
%     \sbox0{\small\t}^^A
%     \ifdim\wd0>\linewidth
%       \ifdim\wd0>\lw
%         \sbox0{\footnotesize\t}^^A
%         \ifdim\wd0>\linewidth
%           \ifdim\wd0>\lw
%             \sbox0{\scriptsize\t}^^A
%             \ifdim\wd0>\linewidth
%               \ifdim\wd0>\lw
%                 \sbox0{\tiny\t}^^A
%                 \ifdim\wd0>\linewidth
%                   \lwbox
%                 \else
%                   \usebox0
%                 \fi
%               \else
%                 \lwbox
%               \fi
%             \else
%               \usebox0
%             \fi
%           \else
%             \lwbox
%           \fi
%         \else
%           \usebox0
%         \fi
%       \else
%         \lwbox
%       \fi
%     \else
%       \usebox0
%     \fi
%   \else
%     \lwbox
%   \fi
% \else
%   \usebox0
% \fi
% \end{quote}
% If you have a \xfile{docstrip.cfg} that configures and enables \docstrip's
% TDS installing feature, then some files can already be in the right
% place, see the documentation of \docstrip.
%
% \subsection{Refresh file name databases}
%
% If your \TeX~distribution
% (\teTeX, \mikTeX, \dots) relies on file name databases, you must refresh
% these. For example, \teTeX\ users run \verb|texhash| or
% \verb|mktexlsr|.
%
% \subsection{Some details for the interested}
%
% \paragraph{Attached source.}
%
% The PDF documentation on CTAN also includes the
% \xfile{.dtx} source file. It can be extracted by
% AcrobatReader 6 or higher. Another option is \textsf{pdftk},
% e.g. unpack the file into the current directory:
% \begin{quote}
%   \verb|pdftk thepdfnumber.pdf unpack_files output .|
% \end{quote}
%
% \paragraph{Unpacking with \LaTeX.}
% The \xfile{.dtx} chooses its action depending on the format:
% \begin{description}
% \item[\plainTeX:] Run \docstrip\ and extract the files.
% \item[\LaTeX:] Generate the documentation.
% \end{description}
% If you insist on using \LaTeX\ for \docstrip\ (really,
% \docstrip\ does not need \LaTeX), then inform the autodetect routine
% about your intention:
% \begin{quote}
%   \verb|latex \let\install=y% \iffalse meta-comment
%
% File: thepdfnumber.dtx
% Version: 2016/05/16 v1.1
% Info: Print PDF numbers with minimal digits
%
% Copyright (C) 2011 by
%    Heiko Oberdiek <heiko.oberdiek at googlemail.com>
%    2016
%    https://github.com/ho-tex/oberdiek/issues
%
% This work may be distributed and/or modified under the
% conditions of the LaTeX Project Public License, either
% version 1.3c of this license or (at your option) any later
% version. This version of this license is in
%    http://www.latex-project.org/lppl/lppl-1-3c.txt
% and the latest version of this license is in
%    http://www.latex-project.org/lppl.txt
% and version 1.3 or later is part of all distributions of
% LaTeX version 2005/12/01 or later.
%
% This work has the LPPL maintenance status "maintained".
%
% This Current Maintainer of this work is Heiko Oberdiek.
%
% The Base Interpreter refers to any `TeX-Format',
% because some files are installed in TDS:tex/generic//.
%
% This work consists of the main source file thepdfnumber.dtx
% and the derived files
%    thepdfnumber.sty, thepdfnumber.pdf, thepdfnumber.ins, thepdfnumber.drv,
%    thepdfnumber-test1.tex, thepdfnumber-test2.tex,
%    thepdfnumber-test3.tex.
%
% Distribution:
%    CTAN:macros/latex/contrib/oberdiek/thepdfnumber.dtx
%    CTAN:macros/latex/contrib/oberdiek/thepdfnumber.pdf
%
% Unpacking:
%    (a) If thepdfnumber.ins is present:
%           tex thepdfnumber.ins
%    (b) Without thepdfnumber.ins:
%           tex thepdfnumber.dtx
%    (c) If you insist on using LaTeX
%           latex \let\install=y\input{thepdfnumber.dtx}
%        (quote the arguments according to the demands of your shell)
%
% Documentation:
%    (a) If thepdfnumber.drv is present:
%           latex thepdfnumber.drv
%    (b) Without thepdfnumber.drv:
%           latex thepdfnumber.dtx; ...
%    The class ltxdoc loads the configuration file ltxdoc.cfg
%    if available. Here you can specify further options, e.g.
%    use A4 as paper format:
%       \PassOptionsToClass{a4paper}{article}
%
%    Programm calls to get the documentation (example):
%       pdflatex thepdfnumber.dtx
%       makeindex -s gind.ist thepdfnumber.idx
%       pdflatex thepdfnumber.dtx
%       makeindex -s gind.ist thepdfnumber.idx
%       pdflatex thepdfnumber.dtx
%
% Installation:
%    TDS:tex/generic/oberdiek/thepdfnumber.sty
%    TDS:doc/latex/oberdiek/thepdfnumber.pdf
%    TDS:doc/latex/oberdiek/test/thepdfnumber-test1.tex
%    TDS:doc/latex/oberdiek/test/thepdfnumber-test2.tex
%    TDS:doc/latex/oberdiek/test/thepdfnumber-test3.tex
%    TDS:source/latex/oberdiek/thepdfnumber.dtx
%
%<*ignore>
\begingroup
  \catcode123=1 %
  \catcode125=2 %
  \def\x{LaTeX2e}%
\expandafter\endgroup
\ifcase 0\ifx\install y1\fi\expandafter
         \ifx\csname processbatchFile\endcsname\relax\else1\fi
         \ifx\fmtname\x\else 1\fi\relax
\else\csname fi\endcsname
%</ignore>
%<*install>
\input docstrip.tex
\Msg{************************************************************************}
\Msg{* Installation}
\Msg{* Package: thepdfnumber 2016/05/16 v1.1 Print PDF numbers with minimal digits (HO)}
\Msg{************************************************************************}

\keepsilent
\askforoverwritefalse

\let\MetaPrefix\relax
\preamble

This is a generated file.

Project: thepdfnumber
Version: 2016/05/16 v1.1

Copyright (C) 2011 by
   Heiko Oberdiek <heiko.oberdiek at googlemail.com>

This work may be distributed and/or modified under the
conditions of the LaTeX Project Public License, either
version 1.3c of this license or (at your option) any later
version. This version of this license is in
   http://www.latex-project.org/lppl/lppl-1-3c.txt
and the latest version of this license is in
   http://www.latex-project.org/lppl.txt
and version 1.3 or later is part of all distributions of
LaTeX version 2005/12/01 or later.

This work has the LPPL maintenance status "maintained".

This Current Maintainer of this work is Heiko Oberdiek.

The Base Interpreter refers to any `TeX-Format',
because some files are installed in TDS:tex/generic//.

This work consists of the main source file thepdfnumber.dtx
and the derived files
   thepdfnumber.sty, thepdfnumber.pdf, thepdfnumber.ins, thepdfnumber.drv,
   thepdfnumber-test1.tex, thepdfnumber-test2.tex,
   thepdfnumber-test3.tex.

\endpreamble
\let\MetaPrefix\DoubleperCent

\generate{%
  \file{thepdfnumber.ins}{\from{thepdfnumber.dtx}{install}}%
  \file{thepdfnumber.drv}{\from{thepdfnumber.dtx}{driver}}%
  \usedir{tex/generic/oberdiek}%
  \file{thepdfnumber.sty}{\from{thepdfnumber.dtx}{package}}%
  \usedir{doc/latex/oberdiek/test}%
  \file{thepdfnumber-test1.tex}{\from{thepdfnumber.dtx}{test1}}%
  \file{thepdfnumber-test2.tex}{\from{thepdfnumber.dtx}{test2}}%
  \file{thepdfnumber-test3.tex}{\from{thepdfnumber.dtx}{test2,active}}%
}

\catcode32=13\relax% active space
\let =\space%
\Msg{************************************************************************}
\Msg{*}
\Msg{* To finish the installation you have to move the following}
\Msg{* file into a directory searched by TeX:}
\Msg{*}
\Msg{*     thepdfnumber.sty}
\Msg{*}
\Msg{* To produce the documentation run the file `thepdfnumber.drv'}
\Msg{* through LaTeX.}
\Msg{*}
\Msg{* Happy TeXing!}
\Msg{*}
\Msg{************************************************************************}

\endbatchfile
%</install>
%<*ignore>
\fi
%</ignore>
%<*driver>
\NeedsTeXFormat{LaTeX2e}
\ProvidesFile{thepdfnumber.drv}%
  [2016/05/16 v1.1 Print PDF numbers with minimal digits (HO)]%
\documentclass{ltxdoc}
\usepackage{holtxdoc}[2011/11/22]
\usepackage{csquotes}
\usepackage[
  backend=bibtex,
  bibencoding=ascii,
  alldates=iso8601,
]{biblatex}[2011/11/13]
\bibliography{oberdiek-source}
\begin{document}
  \DocInput{thepdfnumber.dtx}%
\end{document}
%</driver>
% \fi
%
%
% \CharacterTable
%  {Upper-case    \A\B\C\D\E\F\G\H\I\J\K\L\M\N\O\P\Q\R\S\T\U\V\W\X\Y\Z
%   Lower-case    \a\b\c\d\e\f\g\h\i\j\k\l\m\n\o\p\q\r\s\t\u\v\w\x\y\z
%   Digits        \0\1\2\3\4\5\6\7\8\9
%   Exclamation   \!     Double quote  \"     Hash (number) \#
%   Dollar        \$     Percent       \%     Ampersand     \&
%   Acute accent  \'     Left paren    \(     Right paren   \)
%   Asterisk      \*     Plus          \+     Comma         \,
%   Minus         \-     Point         \.     Solidus       \/
%   Colon         \:     Semicolon     \;     Less than     \<
%   Equals        \=     Greater than  \>     Question mark \?
%   Commercial at \@     Left bracket  \[     Backslash     \\
%   Right bracket \]     Circumflex    \^     Underscore    \_
%   Grave accent  \`     Left brace    \{     Vertical bar  \|
%   Right brace   \}     Tilde         \~}
%
% \GetFileInfo{thepdfnumber.drv}
%
% \title{The \xpackage{thepdfnumber} package}
% \date{2016/05/16 v1.1}
% \author{Heiko Oberdiek\thanks
% {Please report any issues at https://github.com/ho-tex/oberdiek/issues}\\
% \xemail{heiko.oberdiek at googlemail.com}}
%
% \maketitle
%
% \begin{abstract}
% The package converts real numbers to a minimal representation
% that is stripped from leading or trailing zeros,
% plus signs and decimal point if not necessary.
% \end{abstract}
%
% \tableofcontents
%
% \section{Documentation}
%
% \subsection{Introduction}
%
% Dealing with the PDF format, there is sometimes the need to write
% some low level PDF stuff. In case of numbers, the numbers can
% arise from user input (e.g. color or transparency specifications)
% or can be calculated. For example, \hologo{LaTeX}'s \cs{strip@pt}
% makes a good job to output a real number. It automatically suppresses
% the decimal part if the number is an integer. However it leaves
% a leading zero for numbers greater zero and smaller one.
% Thus the package provides macros that can be used with different
% formats, even with \hologo{iniTeX} and generates numbers that
% are valid numbers of the PDF format and whose length is minimal.
%
% \subsection{Usage}
%
% The package \xpackage{thepdfnumber} can be used with \hologo{LaTeX},
% \hologo{plainTeX} or even with \hologo{iniTeX}:
% \begin{quote}
%   |\RequirePackage{thepdfnumber}| \textit{\% \hologo{LaTeX}}\\
%   |\input thepdfnumber.sty| \textit{\% \hologo{plainTeX}/\hologo{iniTeX}}
% \end{quote}
% The package does not need and have package options.
%
% \subsection{User macros}
%
% All user macros are expandable in exact two expansion steps.
%
% \begin{declcs}{thepdfnumber} \M{number}
% \end{declcs}
% Macro \cs{thepdfnumber} takes a number
% as argument and expands to a minimal representation of that number.
% Some examples:
% \begin{quote}
% \begin{tabular}{r@{.: }>{\ttfamily}l@{ $\rightarrow$\ }>{\ttfamily}l@{}}
%   1& +123 & 123\\
%   2& --123 & 123\\
%   3& -01 & -1\\
%   4& 0045 & 45\\
%   5& 1.0 & 1\\
%   6& 1.20 & 1.2\\
%   7& 0.0 & 0\\
%   8& 0.78 & .78\\
%   9& +012.340 & 12.34\\
% \end{tabular}
% \end{quote}
% It reduces the length of the number representation:
% \begin{itemize}
% \item The signs are collapsed and only one minus sign
%   is output if the number is negative (see examples 1, 2, 3, 9).
% \item Leading zeros are removed (4, 8, 9) unless the number is zero (7).
% \item The decimal part is omitted, if the number is an integer (5, 7).
% \item Trailing zeros from the decimal part are stripped (5, 6, 7, 9).
% \end{itemize}
% The resulting number representation can be catched with one of the
% following Perl regular expressions:
% \begin{itemize}
% \item |^0$| (zero)
% \item |^-?[1-9][0-9]*$| (integer)
% \item |^-?[0-9]*\.[0-9]*[1-9]$| (real)
% \end{itemize}
% This is a valid numeric object of the PDF specification \cite[``7.3.3~Numeric Objects'']{pdfspec-iso32000-1}.
%
% \begin{declcs}{thepdfnumberNormZeroOne}
% \end{declcs}
% There are various places in the PDF specification where the
% number is in the domain 0.0 upto 1.0. Macro thepdfnumberNormZeroOne
% automatically adjusts the number to fit into that range. Negative
% numbers are mapped to |0| and numbers greater than one are
% replaced by |1|. Thus the result fits one of the following regular
% expressions:
% \begin{itemize}
% \item |^0$|
% \item |^\.[0-9]*[1-9]$|
% \item |^1$|
% \end{itemize}
% Examples:
% \begin{quote}
% \begin{tabular}{@{}>{\ttfamily}l@{ $\rightarrow$\ }>{\ttfamily}l@{}}
%   -456 & 0\\
%   -0.001 & 0\\
%   0.0 & 0\\
%   0.010 & .01\\
%   0.456 & .456\\
%   1.0 & 1\\
%   01.001 & 1\\
%   4 & 1\\
% \end{tabular}
% \end{quote}
%
% \subsection{Input number}
%
% The user macros expect a number as argument. The number can either
% be given explicitely or as macro that expands in one step
% to an explicite number, because the first token of the argument
% is expanded once.
%
% The explicite number consists of
% \begin{itemize}
% \item optional signs `\texttt{+}' and `\texttt{-}',
% \item digits `\texttt{0}' upto `\texttt{9}' and
% \item an optional dot `\texttt{.}'.
% \end{itemize}
% All tokens must have catcode 12 (other), the default catcodes
% for these characters in \hologo{LaTeX}, \hologo{plainTeX} or
% \hologo{iniTeX}. As Perl regular expression the number is expected
% in one of the following forms:
% \begin{itemize}
% \item |^[+-]*[0-9]+$|
% \item |^[+-]*[0-9]*\.[0-9]*$|
% \end{itemize}
% At least one digit or the dot must be present.
%
% \subsection{Error handling}
%
% The package is not intended for validating numbers or to
% decide if an argument is a number. Therefore it is an usage
% error to use the user macros with arguments that are not
% explicite numbers as specified in the previous sections.
% Nevertheless some error conditions are sometimes recognized.
% Errors are given in form of an undefined command sequence.
% It is the only way to notify \hologo{TeX} in expandable context.
% Expanding to some error text would invalidate the output.
% Currently the following errors are thrown:
% \begin{description}
% \item[\cs{thepdfnumber@ErrorEndMarker}:]
%   Internally the argument parsing uses an end marker that
%   is never called directly. If it is called with valid
%   user input, then this is a bug. Otherwise it means the user input
%   contains nasty stuff.
% \item[\cs{thepdfnumber@ErrorUnexpectedEnd}:] The macros expect at least one digit
%   or the dot, otherwise if the argument is empty or only contains
%   signs, then this error is called.
% \item[\cs{thepdfnumber@ErrorInvalidToken}:] It is called if the number
%   contains other tokens than signs, digits or the dot or the
%   token at the wrong place (e.g. a sign after a digit).
%   In case of \cs{thepdfnumberNormZeroOne} this error condition might
%   not always be detected, because the number parsing might stop
%   at an early point, when the result is already clear (e.g.
%   if the number is negative or will be greater than one).
% \item[Improper alphabetic constant:] This error might be
%   thrown by \hologo{TeX}, if the number contains
%   command tokens instead of characters.
% \end{description}
%
% \StopEventually{
% }
%
% \section{Implementation}
%
%    \begin{macrocode}
%<*package>
%    \end{macrocode}
%
% \subsection{Reload check and package identification}
%    Reload check, especially if the package is not used with \LaTeX.
%    \begin{macrocode}
\begingroup\catcode61\catcode48\catcode32=10\relax%
  \catcode13=5 % ^^M
  \endlinechar=13 %
  \catcode35=6 % #
  \catcode39=12 % '
  \catcode44=12 % ,
  \catcode45=12 % -
  \catcode46=12 % .
  \catcode58=12 % :
  \catcode64=11 % @
  \catcode123=1 % {
  \catcode125=2 % }
  \expandafter\let\expandafter\x\csname ver@thepdfnumber.sty\endcsname
  \ifx\x\relax % plain-TeX, first loading
  \else
    \def\empty{}%
    \ifx\x\empty % LaTeX, first loading,
      % variable is initialized, but \ProvidesPackage not yet seen
    \else
      \expandafter\ifx\csname PackageInfo\endcsname\relax
        \def\x#1#2{%
          \immediate\write-1{Package #1 Info: #2.}%
        }%
      \else
        \def\x#1#2{\PackageInfo{#1}{#2, stopped}}%
      \fi
      \x{thepdfnumber}{The package is already loaded}%
      \aftergroup\endinput
    \fi
  \fi
\endgroup%
%    \end{macrocode}
%    Package identification:
%    \begin{macrocode}
\begingroup\catcode61\catcode48\catcode32=10\relax%
  \catcode13=5 % ^^M
  \endlinechar=13 %
  \catcode35=6 % #
  \catcode39=12 % '
  \catcode40=12 % (
  \catcode41=12 % )
  \catcode44=12 % ,
  \catcode45=12 % -
  \catcode46=12 % .
  \catcode47=12 % /
  \catcode58=12 % :
  \catcode64=11 % @
  \catcode91=12 % [
  \catcode93=12 % ]
  \catcode123=1 % {
  \catcode125=2 % }
  \expandafter\ifx\csname ProvidesPackage\endcsname\relax
    \def\x#1#2#3[#4]{\endgroup
      \immediate\write-1{Package: #3 #4}%
      \xdef#1{#4}%
    }%
  \else
    \def\x#1#2[#3]{\endgroup
      #2[{#3}]%
      \ifx#1\@undefined
        \xdef#1{#3}%
      \fi
      \ifx#1\relax
        \xdef#1{#3}%
      \fi
    }%
  \fi
\expandafter\x\csname ver@thepdfnumber.sty\endcsname
\ProvidesPackage{thepdfnumber}%
  [2016/05/16 v1.1 Print PDF numbers with minimal digits (HO)]%
%    \end{macrocode}
%
% \subsection{Catcodes}
%
%    \begin{macrocode}
\begingroup\catcode61\catcode48\catcode32=10\relax%
  \catcode13=5 % ^^M
  \endlinechar=13 %
  \catcode123=1 % {
  \catcode125=2 % }
  \catcode64=11 % @
  \def\x{\endgroup
    \expandafter\edef\csname ThPdNu@AtEnd\endcsname{%
      \endlinechar=\the\endlinechar\relax
      \catcode13=\the\catcode13\relax
      \catcode32=\the\catcode32\relax
      \catcode35=\the\catcode35\relax
      \catcode61=\the\catcode61\relax
      \catcode64=\the\catcode64\relax
      \catcode123=\the\catcode123\relax
      \catcode125=\the\catcode125\relax
    }%
  }%
\x\catcode61\catcode48\catcode32=10\relax%
\catcode13=5 % ^^M
\endlinechar=13 %
\catcode35=6 % #
\catcode64=11 % @
\catcode123=1 % {
\catcode125=2 % }
\def\TMP@EnsureCode#1#2{%
  \edef\ThPdNu@AtEnd{%
    \ThPdNu@AtEnd
    \catcode#1=\the\catcode#1\relax
  }%
  \catcode#1=#2\relax
}
\TMP@EnsureCode{33}{12}% !
\TMP@EnsureCode{36}{3}% $
\TMP@EnsureCode{38}{4}% &
\TMP@EnsureCode{42}{12}% *
\TMP@EnsureCode{43}{12}% +
\TMP@EnsureCode{45}{12}% -
\TMP@EnsureCode{46}{12}% .
\TMP@EnsureCode{60}{12}% <
\TMP@EnsureCode{62}{12}% >
\TMP@EnsureCode{96}{12}% `
\edef\ThPdNu@AtEnd{\ThPdNu@AtEnd\noexpand\endinput}
%    \end{macrocode}
%
% \subsection{Helper macros}
%
%    \begin{macro}{\ThPdNu@FIN}
%    \begin{macrocode}
\def\ThPdNu@FIN{\thepdfnumber@ErrorEndMarker}
%    \end{macrocode}
%    \end{macro}
%
%    \begin{macro}{\ThPdNu@space}
%    \begin{macrocode}
\def\ThPdNu@space{ }
%    \end{macrocode}
%    \end{macro}
%
%    \begin{macro}{\ThPdNu@zero}
%    \begin{macrocode}
\chardef\ThPdNu@zero=0 %
%    \end{macrocode}
%    \end{macro}
%
%    \begin{macro}{\ThPdNu@one}
%    \begin{macrocode}
\chardef\ThPdNu@one=1 %
%    \end{macrocode}
%    \end{macro}
%
%    \begin{macro}{\ThPdNu@firstoftwo}
%    \begin{macrocode}
\long\def\ThPdNu@firstoftwo#1#2{#1}
%    \end{macrocode}
%    \end{macro}
%
%    \begin{macro}{\ThPdNu@secondoftwo}
%    \begin{macrocode}
\long\def\ThPdNu@secondoftwo#1#2{#2}
%    \end{macrocode}
%    \end{macro}
%
% \subsection{Detect \hologo{eTeX}}
%
%    \begin{macrocode}
\begingroup\expandafter\expandafter\expandafter\endgroup
\expandafter\ifx\csname detokenize\endcsname\relax
  \catcode`\&=14 %
  \catcode`\$=9 %
\else
  \catcode`\&=9 %
  \catcode`\$=14 %
\fi
%    \end{macrocode}
%
% \subsection{User macro \cs{thepdfnumber}}
%
%    \begin{macro}{\thepdfnumber}
%    \begin{macrocode}
\def\thepdfnumber#1{%
  \romannumeral
& \iftrue\expandafter\ThPdNu@State@Plus\expandafter\fi
& \detokenize\expandafter{#1}%
& \ThPdNu@FIN
$ \ifx\ThPdNu@FIN#1\ThPdNu@FIN
$   \expandafter\ThPdNu@firstoftwo
$ \else
$   \expandafter\ThPdNu@secondoftwo
$ \fi
$ {%
$   \ThPdNu@zero
$   0\thepdfnumber@ErrorUnexpectedEnd
$ }{%
$   \iftrue\expandafter\ThPdNu@State@Plus\expandafter\fi#1\ThPdNu@FIN
$ }%
}
%    \end{macrocode}
%    \end{macro}
%
% \subsubsection{State definitions for sign}
%
%    \begin{macro}{\ThPdNu@State@Plus}
%    \begin{macrocode}
\def\ThPdNu@State@Plus#1\fi#2{%
  \fi
  \ifcase\ifx\ThPdNu@FIN#2%
           0%
&        \else\ifx-#2%
$        \else\ifnum`#2=45 % -
           1%
         \else\ifx0#2%
           2%
         \else\ifnum`#2>48 %
           \ifnum`#2<58 %
             3%
           \else
             9%
           \fi
&        \else\ifx.#2%
$        \else\ifnum`#2=46 % .
           4%
&        \else\ifx+#2%
$        \else\ifnum`#2=43 % +
           5%
         \else
           9%
         \fi\fi\fi\fi\fi\fi\ThPdNu@space
    \expandafter\ThPdNu@zero
    \expandafter0%
    \expandafter\thepdfnumber@ErrorUnexpectedEnd
  \or
    \ThPdNu@State@Minus
  \or
    \ThPdNu@State@SkipZeros!%
  \or
    \ThPdNu@State@Int!#2!%
  \or
    \ThPdNu@State@Dot!\ThPdNu@zero*\ThPdNu@zero!!%
  \or
    \ThPdNu@State@Plus
  \else
    \ThPdNu@ReturnError{0}%
  \fi
}
%    \end{macrocode}
%    \end{macro}
%
%    \begin{macro}{\ThPdNu@State@Minus}
%    \begin{macrocode}
\def\ThPdNu@State@Minus#1\fi#2{%
  \fi
  \ifcase\ifx\ThPdNu@FIN#2%
           0%
         \else\ifx0#2%
           1%
         \else\ifnum`#2>48 %
           \ifnum`#2<58 %
             2%
           \else
             9%
           \fi
&        \else\ifx.#2%
$        \else\ifnum`#2=46 % .
           3%
&        \else\ifx-#2%
$        \else\ifnum`#2=45 % -
           4%
&        \else\ifx+#2%
$        \else\ifnum`#2=43 % +
           5%
         \else
           9%
         \fi\fi\fi\fi\fi\fi\ThPdNu@space
    \expandafter\ThPdNu@zero
    \expandafter0%
    \expandafter\thepdfnumber@ErrorUnexpectedEnd
  \or
    \ThPdNu@State@SkipZeros-!%
  \or
    \ThPdNu@State@Int-!#2!%
  \or
    \ThPdNu@State@Dot-!\ThPdNu@zero*\ThPdNu@zero!!%
  \or
    \ThPdNu@State@Plus
  \or
    \ThPdNu@State@Minus
  \else
    \ThPdNu@ReturnError{0}%
  \fi
}
%    \end{macrocode}
%    \end{macro}
%
%    \begin{macro}{\ThPdNu@ReturnError}
%    \begin{macrocode}
\def\ThPdNu@ReturnError#1#2\fi#3\ThPdNu@FIN{%
  \fi
  \ThPdNu@zero
  #1%
  \thepdfnumber@ErrorInvalidToken
}
%    \end{macrocode}
%    \end{macro}
%
% \subsubsection{State definitions for integer part}
%
%    \begin{macro}{\ThPdNu@State@SkipZeros}
%    \begin{macrocode}
\def\ThPdNu@State@SkipZeros#1!#2\fi#3{%
  \fi
  \ifcase\ifx\ThPdNu@FIN#3%
           0%
         \else\ifx0#3%
           1%
         \else\ifnum`#3>48 %
           \ifnum`#3<58 %
             2%
           \else
             9%
           \fi
&        \else\ifx.#3%
$        \else\ifnum`#3=46 % .
           3%
         \else
           9%
         \fi\fi\fi\fi\ThPdNu@space
    \expandafter\ThPdNu@zero
    \expandafter0%
  \or
   \ThPdNu@State@SkipZeros#1!%
  \or
    \ThPdNu@State@Int#1!#3!%
  \or
    \ThPdNu@State@Dot#1!\ThPdNu@zero*\ThPdNu@zero!!%
  \else
    \ThPdNu@ReturnError{0}%
  \fi
}
%    \end{macrocode}
%    \end{macro}
%
%    \begin{macro}{\ThPdNu@State@Int}
%    \begin{macrocode}
\def\ThPdNu@State@Int#1!#2!#3\fi#4{%
  \fi
  \ifcase\ifx\ThPdNu@FIN#4%
           0%
         \else\ifnum`#4>47 %
           \ifnum`#4<58 %
             1%
           \else
             9%
           \fi
&        \else\ifx.#4%
$        \else\ifnum`#4=46 % .
           2%
         \else
           9%
         \fi\fi\fi\ThPdNu@space
    \ThPdNu@ReturnInt{#1#2}%
  \or
    \ThPdNu@State@Int#1!#2#4!%
  \or
    \ThPdNu@State@Dot#1!\ThPdNu@one#2*\ThPdNu@zero!!%
  \else
    \ThPdNu@ReturnError{#1#2}%
  \fi
}
%    \end{macrocode}
%    \end{macro}
%
%    \begin{macro}{\ThPdNu@ReturnInt}
%    \begin{macrocode}
\def\ThPdNu@ReturnInt#1#2\fi{%
  \fi
  \ThPdNu@zero
  #1%
}
%    \end{macrocode}
%    \end{macro}
%
% \subsubsection{State definitions for decimal digits}
%
%    \begin{macro}{\ThPdNu@State@Dot}
%    \begin{macrocode}
\def\ThPdNu@State@Dot#1*#2#3!#4!#5\fi#6{%
  \fi
  \ifcase\ifx\ThPdNu@FIN#6%
           0%
         \else\ifnum`#6>48 %
           \ifnum`#6<58 %
             1%
           \else
             9%
           \fi
         \else\ifx0#6%
           2%
         \else
           9%
         \fi\fi\fi\ThPdNu@space
    \ThPdNu@ReturnNumber#1*#2#3!%
  \or
    \ThPdNu@State@Dot#1*\ThPdNu@one#3#4#6!!%
  \or
    \ThPdNu@State@DotZero#1*#2#3!#4#6!%
  \else
    \ThPdNu@ReturnNumberInvalid#1*#2#3!%
  \fi
}
%    \end{macrocode}
%    \end{macro}
%
%    \begin{macro}{\ThPdNu@State@DotZero}
%    \begin{macrocode}
\def\ThPdNu@State@DotZero#1*#2#3!#4!#5\fi#6{%
  \fi
  \ifcase\ifx\ThPdNu@FIN#6%
           0%
         \else\ifnum`#6>48 %
           \ifnum`#6<58 %
             1%
           \else
             9%
           \fi
         \else\ifx0#6%
           2%
         \else
           9%
         \fi\fi\fi\ThPdNu@space
    \ThPdNu@ReturnNumber#1*#2#3!%
  \or
    \ThPdNu@State@Dot#1*\ThPdNu@one#3#4#6!!%
  \or
    \ThPdNu@State@DotZero#1*#2#3!#4#6!%
  \else
    \ThPdNu@ReturnNumber#1*#2#3!%
  \fi
}
%    \end{macrocode}
%    \end{macro}
%
%    \begin{macro}{\ThPdNu@ReturnNumber}
%    \begin{macrocode}
\def\ThPdNu@ReturnNumber#1!#2#3*#4#5!#6\fi{%
  \fi
  \ifcase#2%
    \expandafter\ThPdNu@firstoftwo
  \else
    \expandafter\ThPdNu@secondoftwo
  \fi
  {%
    \ifcase#4%
      \expandafter\ThPdNu@firstoftwo
    \else
      \expandafter\ThPdNu@secondoftwo
    \fi
    {\ThPdNu@zero 0}%
    {\ThPdNu@zero #1.#5}%
  }{%
    \ifcase#4%
      \expandafter\ThPdNu@firstoftwo
    \else
      \expandafter\ThPdNu@secondoftwo
    \fi
    {\ThPdNu@zero #1#3}%
    {\ThPdNu@zero #1#3.#5}%
  }%
}
%    \end{macrocode}
%    \end{macro}
%
%    \begin{macro}{\ThPdNu@ReturnNumberInvalid}
%    \begin{macrocode}
\def\ThPdNu@ReturnNumberInvalid#1*#2!#3\fi#4\ThPdNu@FIN{%
  \fi
  \iftrue\ThPdNu@ReturnNumber#1*#2!\fi
  \thepdfnumber@ErrorInvalidToken
}
%    \end{macrocode}
%    \end{macro}
%
% \subsection{Norm macro}
%
%    \begin{macro}{\thepdfnumberNormZeroOne}
%    \begin{macrocode}
\def\thepdfnumberNormZeroOne#1{%
  \romannumeral
& \iftrue\expandafter\ThPbNu@StateN@Plus\expandafter\fi
& \detokenize\expandafter{#1}%
& \ThPdNu@FIN
$ \ifx\ThPdNu@FIN#1\ThPdNu@FIN
$   \expandafter\ThPdNu@firstoftwo
$ \else
$   \expandafter\ThPdNu@secondoftwo
$ \fi
$ {%
$   \ThPdNu@zero
$   0\thepdfnumber@ErrorUnexpectedEnd
$ }{%
$   \iftrue\expandafter\ThPbNu@StateN@Plus\expandafter\fi#1\ThPdNu@FIN
$ }%
}
%    \end{macrocode}
%    \end{macro}
%
% \subsubsection{State definitions for sign}
%
%    \begin{macro}{\ThPbNu@StateN@Plus}
%    \begin{macrocode}
\def\ThPbNu@StateN@Plus#1\fi#2{%
  \fi
  \ifcase\ifx\ThPdNu@FIN#2%
           0%
&        \else\ifx-#2%
$        \else\ifnum`#2=45 % -
           1%
         \else\ifx0#2%
           2%
         \else\ifnum`#2>48 %
           \ifnum`#2<58 %
             3%
           \else
             9%
           \fi
&        \else\ifx.#2%
$        \else\ifnum`#2=46 % .
           4%
&        \else\ifx+#2%
$        \else\ifnum`#2=43 % +
           5%
         \else
           9%
         \fi\fi\fi\fi\fi\fi\ThPdNu@space
    \expandafter\ThPdNu@zero
    \expandafter0%
    \expandafter\thepdfnumber@ErrorUnexpectedEnd
  \or
    \ThPbNu@StateN@Minus
  \or
    \ThPbNu@StateN@SkipZeros
  \or
    \ThPdNu@ReturnAndSkip{1}%
  \or
    \ThPbNu@StateN@Dot\ThPdNu@zero!!%
  \or
    \ThPbNu@StateN@Plus
  \else
    \ThPdNu@ReturnError{0}%
  \fi
}
%    \end{macrocode}
%    \end{macro}
%
%    \begin{macro}{\ThPbNu@StateN@Minus}
%    \begin{macrocode}
\def\ThPbNu@StateN@Minus#1\fi#2{%
  \fi
  \ifcase\ifx\ThPdNu@FIN#2%
           0%
         \else\ifnum`#2>47 %
           \ifnum`#2<58 %
             1%
           \else
             9%
           \fi
&        \else\ifx.#2%
$        \else\ifnum`#2=46 % .
           1%
&        \else\ifx-#2%
$        \else\ifnum`#2=45 % -
           2%
&        \else\ifx+#2%
$        \else\ifnum`#2=43 % +
           3%
         \else
           9%
         \fi\fi\fi\fi\fi\ThPdNu@space
    \expandafter\ThPdNu@zero
    \expandafter0%
    \expandafter\thepdfnumber@ErrorUnexpectedEnd
  \or
    \ThPdNu@ReturnAndSkip{0}%
  \or
    \ThPbNu@StateN@Plus
  \or
    \ThPbNu@StateN@Minus
  \else
    \ThPdNu@ReturnError{0}%
  \fi
}
%    \end{macrocode}
%    \end{macro}
%
%    \begin{macro}{\ThPdNu@ReturnAndSkip}
%    \begin{macrocode}
\def\ThPdNu@ReturnAndSkip#1#2\fi#3\ThPdNu@FIN{%
  \fi
  \ThPdNu@zero
  #1%
}
%    \end{macrocode}
%    \end{macro}
%
% \subsubsection{State definitions for integer part}
%
%    \begin{macro}{\ThPbNu@StateN@SkipZeros}
%    \begin{macrocode}
\def\ThPbNu@StateN@SkipZeros#1\fi#2{%
  \fi
  \ifcase\ifx\ThPdNu@FIN#2%
           0%
         \else\ifx0#2%
           1%
         \else\ifnum`#2>48 %
           \ifnum`#2<58 %
             2%
           \else
             9%
           \fi
&        \else\ifx.#2%
$        \else\ifnum`#2=46 % .
           3%
         \else
           9%
         \fi\fi\fi\fi\ThPdNu@space
    \expandafter\ThPdNu@zero
    \expandafter0%
  \or
   \ThPbNu@StateN@SkipZeros%
  \or
    \ThPdNu@ReturnAndSkip{1}%
  \or
    \ThPbNu@StateN@Dot\ThPdNu@zero!!%
  \else
    \ThPdNu@ReturnError{0}%
  \fi
}
%    \end{macrocode}
%    \end{macro}
%
% \subsubsection{State definitions for decimal digits}
%
%    \begin{macro}{\ThPbNu@StateN@Dot}
%    \begin{macrocode}
\def\ThPbNu@StateN@Dot#1#2!#3!#4\fi#5{%
  \fi
  \ifcase\ifx\ThPdNu@FIN#5%
           0%
         \else\ifnum`#5>48 %
           \ifnum`#5<58 %
             1%
           \else
             9%
           \fi
         \else\ifx0#5%
           2%
         \else
           9%
         \fi\fi\fi\ThPdNu@space
    \ThPdNu@ReturnFracNumber#1#2!%
  \or
    \ThPbNu@StateN@Dot\ThPdNu@one#2#3#5!!%
  \or
    \ThPbNu@StateN@DotZero#1#2!#3#5!%
  \else
    \ThPdNu@ReturnFracNumberInvalid#1#2!%
  \fi
}
%    \end{macrocode}
%    \end{macro}
%
%    \begin{macro}{\ThPbNu@StateN@DotZero}
%    \begin{macrocode}
\def\ThPbNu@StateN@DotZero#1#2!#3!#4\fi#5{%
  \fi
  \ifcase\ifx\ThPdNu@FIN#5%
           0%
         \else\ifnum`#5>48 %
           \ifnum`#5<58 %
             1%
           \else
             9%
           \fi
         \else\ifx0#5%
           2%
         \else
           9%
         \fi\fi\fi\ThPdNu@space
    \ThPdNu@ReturnFracNumber#1#2!%
  \or
    \ThPbNu@StateN@Dot\ThPdNu@one#2#3#5!!%
  \or
    \ThPbNu@StateN@DotZero#1#2!#3#5!%
  \else
    \ThPdNu@ReturnFracNumberInvalid#1#2!%
  \fi
}
%    \end{macrocode}
%    \end{macro}
%
%    \begin{macro}{\ThPdNu@ReturnFracNumber}
%    \begin{macrocode}
\def\ThPdNu@ReturnFracNumber#1#2!#3\fi{%
  \fi
  \ifcase#1%
    \expandafter\ThPdNu@firstoftwo
  \else
    \expandafter\ThPdNu@secondoftwo
  \fi
  {\ThPdNu@zero 0}%
  {\ThPdNu@zero .#2}%
}
%    \end{macrocode}
%    \end{macro}
%
%    \begin{macro}{\ThPdNu@ReturnFracNumberInvalid}
%    \begin{macrocode}
\def\ThPdNu@ReturnFracNumberInvalid#1!#2\fi#3\ThPdNu@FIN{%
  \fi
  \iftrue\ThPdNu@ReturnFracNumber#1!\fi
  \thepdfnumber@ErrorInvalidToken
}
%    \end{macrocode}
%    \end{macro}
%
%    \begin{macrocode}
\ThPdNu@AtEnd%
%</package>
%    \end{macrocode}
%
% \section{Test}
%
% \subsection{Catcode checks for loading}
%
%    \begin{macrocode}
%<*test1>
%    \end{macrocode}
%    \begin{macrocode}
\catcode`\{=1 %
\catcode`\}=2 %
\catcode`\#=6 %
\catcode`\@=11 %
\expandafter\ifx\csname count@\endcsname\relax
  \countdef\count@=255 %
\fi
\expandafter\ifx\csname @gobble\endcsname\relax
  \long\def\@gobble#1{}%
\fi
\expandafter\ifx\csname @firstofone\endcsname\relax
  \long\def\@firstofone#1{#1}%
\fi
\expandafter\ifx\csname loop\endcsname\relax
  \expandafter\@firstofone
\else
  \expandafter\@gobble
\fi
{%
  \def\loop#1\repeat{%
    \def\body{#1}%
    \iterate
  }%
  \def\iterate{%
    \body
      \let\next\iterate
    \else
      \let\next\relax
    \fi
    \next
  }%
  \let\repeat=\fi
}%
\def\RestoreCatcodes{}
\count@=0 %
\loop
  \edef\RestoreCatcodes{%
    \RestoreCatcodes
    \catcode\the\count@=\the\catcode\count@\relax
  }%
\ifnum\count@<255 %
  \advance\count@ 1 %
\repeat

\def\RangeCatcodeInvalid#1#2{%
  \count@=#1\relax
  \loop
    \catcode\count@=15 %
  \ifnum\count@<#2\relax
    \advance\count@ 1 %
  \repeat
}
\def\RangeCatcodeCheck#1#2#3{%
  \count@=#1\relax
  \loop
    \ifnum#3=\catcode\count@
    \else
      \errmessage{%
        Character \the\count@\space
        with wrong catcode \the\catcode\count@\space
        instead of \number#3%
      }%
    \fi
  \ifnum\count@<#2\relax
    \advance\count@ 1 %
  \repeat
}
\def\space{ }
\expandafter\ifx\csname LoadCommand\endcsname\relax
  \def\LoadCommand{\input thepdfnumber.sty\relax}%
\fi
\def\Test{%
  \RangeCatcodeInvalid{0}{47}%
  \RangeCatcodeInvalid{58}{64}%
  \RangeCatcodeInvalid{91}{96}%
  \RangeCatcodeInvalid{123}{255}%
  \catcode`\@=12 %
  \catcode`\\=0 %
  \catcode`\%=14 %
  \LoadCommand
  \RangeCatcodeCheck{0}{36}{15}%
  \RangeCatcodeCheck{37}{37}{14}%
  \RangeCatcodeCheck{38}{47}{15}%
  \RangeCatcodeCheck{48}{57}{12}%
  \RangeCatcodeCheck{58}{63}{15}%
  \RangeCatcodeCheck{64}{64}{12}%
  \RangeCatcodeCheck{65}{90}{11}%
  \RangeCatcodeCheck{91}{91}{15}%
  \RangeCatcodeCheck{92}{92}{0}%
  \RangeCatcodeCheck{93}{96}{15}%
  \RangeCatcodeCheck{97}{122}{11}%
  \RangeCatcodeCheck{123}{255}{15}%
  \RestoreCatcodes
}
\Test
\csname @@end\endcsname
\end
%    \end{macrocode}
%    \begin{macrocode}
%</test1>
%    \end{macrocode}
%
% \subsection{Macro tests}
%
% \subsubsection{Preamble}
%
%    \begin{macrocode}
%<*test2>
\catcode`\{=1
\catcode`\}=2
\catcode`\#=6
\catcode`\@=11
\errorcontextlines=10000 %
\def\msg#{\immediate\write16}
\def\space{ }
\begingroup\expandafter\expandafter\expandafter\endgroup
\expandafter\ifx\csname RequirePackage\endcsname\relax
  \input thepdfnumber.sty\relax
\else
  \RequirePackage{thepdfnumber}[2016/05/16]%
\fi
%    \end{macrocode}
%
%    \begin{macro}{\Test}
%    \begin{macrocode}
\def\Test#1#2{%
  \def\TestExpected{#2}%
%<*active>
  \edef\TestExpected{\TestExpected}%
%</active>
  \expandafter\expandafter\expandafter\def
  \expandafter\expandafter\expandafter\TestResult
  \expandafter\expandafter\expandafter{%
    \TestCommand{#1}%
  }%
  \begingroup
    \toks0\expandafter{\TestResult}%
    \toks2\expandafter{\TestExpected}%
    \ifx\TestResult\TestExpected
      \msg{* OK: #1 => \the\toks0}%
    \else
      \errmessage{FAILED: #1 => \the\toks0\space(\the\toks2)}%
    \fi
  \endgroup
}
%    \end{macrocode}
%    \end{macro}
%
%    \begin{macrocode}
\let\thepdfnumber@ErrorUnexpectedEnd\relax
\let\thepdfnumber@ErrorInvalidToken\relax
%    \end{macrocode}
%
% \subsubsection{Test with active catcodes}
%
%    \begin{macrocode}
%<*active>
\catcode`\~=13 %
\def\ThPdNu@Temp#1{%
  \begingroup
    \lccode`\~=`#1 %
  \lowercase{\endgroup
    \edef~%
  }{\string#1}%
  \catcode`#1=13 %
}
\ThPdNu@Temp{.}
\ThPdNu@Temp{-}
\ThPdNu@Temp{+}
%</active>
%    \end{macrocode}
%
% \subsubsection{Test cases for \cs{thepdfnumber}}
%
%    \begin{macro}{\TestCommand}
%    \begin{macrocode}
\let\TestCommand\thepdfnumber
%    \end{macrocode}
%    \end{macro}
%
%    \begin{macrocode}
\Test{0}{0}
\Test{1}{1}
\Test{23}{23}
\Test{0045}{45}
\Test{4500}{4500}
\Test{-0}{0}
\Test{-1}{-1}
\Test{-00002}{-2}
\Test{-010203}{-10203}
\Test{+0}{0}
\Test{+123}{123}

\Test{+-4}{-4}
\Test{+-+-4}{4}
\Test{++--++--++4}{4}
\Test{+++---+++---+++---4}{-4}

\Test{.}{0}
\Test{0.}{0}
\Test{.0}{0}
\Test{0.}{0}
\Test{-.}{0}
\Test{-0.}{0}
\Test{-.0}{0}
\Test{-0.}{0}
\Test{010.020}{10.02}
\Test{123.456}{123.456}
\Test{12.}{12}
\Test{.123}{.123}
\Test{.0001}{.0001}
\Test{.00100}{.001}
\Test{.12003400560078009}{.12003400560078009}
\Test{-.12003400560078009}{-.12003400560078009}
\Test{04.0500000}{4.05}

\Test{}{0\thepdfnumber@ErrorUnexpectedEnd}
\Test{+}{0\thepdfnumber@ErrorUnexpectedEnd}
\Test{-}{0\thepdfnumber@ErrorUnexpectedEnd}
\Test{a}{0\thepdfnumber@ErrorInvalidToken}
\Test{0x1}{0\thepdfnumber@ErrorInvalidToken}
\Test{4x56}{4\thepdfnumber@ErrorInvalidToken}
\Test{012,34}{12\thepdfnumber@ErrorInvalidToken}
\Test{0.12x4}{.12\thepdfnumber@ErrorInvalidToken}
%    \end{macrocode}
%
% \subsubsection{Test cases for \cs{thepdfnumberNormZeroOne}}
%
%    \begin{macro}{\TestCommand}
%    \begin{macrocode}
\let\TestCommand\thepdfnumberNormZeroOne
%    \end{macrocode}
%    \end{macro}
%
%    \begin{macrocode}
\Test{0}{0}
\Test{1}{1}
\Test{23}{1}
\Test{12}{1}
\Test{10}{1}
\Test{0045}{1}
\Test{001}{1}
\Test{-0}{0}
\Test{-1}{0}
\Test{-01}{0}
\Test{-4}{0}

\Test{+-0}{0}
\Test{+-+-1}{1}
\Test{++--++--++1}{1}
\Test{+++---+++---+++---1}{0}

\Test{.}{0}
\Test{0.}{0}
\Test{.0}{0}
\Test{0.}{0}
\Test{-.}{0}
\Test{-0.}{0}
\Test{-.0}{0}
\Test{-0.}{0}
\Test{010.020}{1}
\Test{123.456}{1}
\Test{12.}{1}
\Test{.123}{.123}
\Test{.0001}{.0001}
\Test{.00100}{.001}
\Test{.12003400560078009}{.12003400560078009}
\Test{-.12003400560078009}{0}
\Test{04.0500000}{1}
\Test{0.1200340056}{.1200340056}
\Test{1.05}{1}

\Test{}{0\thepdfnumber@ErrorUnexpectedEnd}
\Test{+}{0\thepdfnumber@ErrorUnexpectedEnd}
\Test{-}{0\thepdfnumber@ErrorUnexpectedEnd}
\Test{a}{0\thepdfnumber@ErrorInvalidToken}
\Test{0x1}{0\thepdfnumber@ErrorInvalidToken}
\Test{4x56}{1}
\Test{012,34}{1}
\Test{.012x4}{.012\thepdfnumber@ErrorInvalidToken}
%    \end{macrocode}
%
%    \begin{macrocode}
\csname @@end\endcsname\end
%</test2>
%    \end{macrocode}
%
% \section{Installation}
%
% \subsection{Download}
%
% \paragraph{Package.} This package is available on
% CTAN\footnote{\url{http://ctan.org/pkg/thepdfnumber}}:
% \begin{description}
% \item[\CTAN{macros/latex/contrib/oberdiek/thepdfnumber.dtx}] The source file.
% \item[\CTAN{macros/latex/contrib/oberdiek/thepdfnumber.pdf}] Documentation.
% \end{description}
%
%
% \paragraph{Bundle.} All the packages of the bundle `oberdiek'
% are also available in a TDS compliant ZIP archive. There
% the packages are already unpacked and the documentation files
% are generated. The files and directories obey the TDS standard.
% \begin{description}
% \item[\CTAN{install/macros/latex/contrib/oberdiek.tds.zip}]
% \end{description}
% \emph{TDS} refers to the standard ``A Directory Structure
% for \TeX\ Files'' (\CTAN{tds/tds.pdf}). Directories
% with \xfile{texmf} in their name are usually organized this way.
%
% \subsection{Bundle installation}
%
% \paragraph{Unpacking.} Unpack the \xfile{oberdiek.tds.zip} in the
% TDS tree (also known as \xfile{texmf} tree) of your choice.
% Example (linux):
% \begin{quote}
%   |unzip oberdiek.tds.zip -d ~/texmf|
% \end{quote}
%
% \paragraph{Script installation.}
% Check the directory \xfile{TDS:scripts/oberdiek/} for
% scripts that need further installation steps.
% Package \xpackage{attachfile2} comes with the Perl script
% \xfile{pdfatfi.pl} that should be installed in such a way
% that it can be called as \texttt{pdfatfi}.
% Example (linux):
% \begin{quote}
%   |chmod +x scripts/oberdiek/pdfatfi.pl|\\
%   |cp scripts/oberdiek/pdfatfi.pl /usr/local/bin/|
% \end{quote}
%
% \subsection{Package installation}
%
% \paragraph{Unpacking.} The \xfile{.dtx} file is a self-extracting
% \docstrip\ archive. The files are extracted by running the
% \xfile{.dtx} through \plainTeX:
% \begin{quote}
%   \verb|tex thepdfnumber.dtx|
% \end{quote}
%
% \paragraph{TDS.} Now the different files must be moved into
% the different directories in your installation TDS tree
% (also known as \xfile{texmf} tree):
% \begin{quote}
% \def\t{^^A
% \begin{tabular}{@{}>{\ttfamily}l@{ $\rightarrow$ }>{\ttfamily}l@{}}
%   thepdfnumber.sty & tex/generic/oberdiek/thepdfnumber.sty\\
%   thepdfnumber.pdf & doc/latex/oberdiek/thepdfnumber.pdf\\
%   test/thepdfnumber-test1.tex & doc/latex/oberdiek/test/thepdfnumber-test1.tex\\
%   test/thepdfnumber-test2.tex & doc/latex/oberdiek/test/thepdfnumber-test2.tex\\
%   test/thepdfnumber-test3.tex & doc/latex/oberdiek/test/thepdfnumber-test3.tex\\
%   thepdfnumber.dtx & source/latex/oberdiek/thepdfnumber.dtx\\
% \end{tabular}^^A
% }^^A
% \sbox0{\t}^^A
% \ifdim\wd0>\linewidth
%   \begingroup
%     \advance\linewidth by\leftmargin
%     \advance\linewidth by\rightmargin
%   \edef\x{\endgroup
%     \def\noexpand\lw{\the\linewidth}^^A
%   }\x
%   \def\lwbox{^^A
%     \leavevmode
%     \hbox to \linewidth{^^A
%       \kern-\leftmargin\relax
%       \hss
%       \usebox0
%       \hss
%       \kern-\rightmargin\relax
%     }^^A
%   }^^A
%   \ifdim\wd0>\lw
%     \sbox0{\small\t}^^A
%     \ifdim\wd0>\linewidth
%       \ifdim\wd0>\lw
%         \sbox0{\footnotesize\t}^^A
%         \ifdim\wd0>\linewidth
%           \ifdim\wd0>\lw
%             \sbox0{\scriptsize\t}^^A
%             \ifdim\wd0>\linewidth
%               \ifdim\wd0>\lw
%                 \sbox0{\tiny\t}^^A
%                 \ifdim\wd0>\linewidth
%                   \lwbox
%                 \else
%                   \usebox0
%                 \fi
%               \else
%                 \lwbox
%               \fi
%             \else
%               \usebox0
%             \fi
%           \else
%             \lwbox
%           \fi
%         \else
%           \usebox0
%         \fi
%       \else
%         \lwbox
%       \fi
%     \else
%       \usebox0
%     \fi
%   \else
%     \lwbox
%   \fi
% \else
%   \usebox0
% \fi
% \end{quote}
% If you have a \xfile{docstrip.cfg} that configures and enables \docstrip's
% TDS installing feature, then some files can already be in the right
% place, see the documentation of \docstrip.
%
% \subsection{Refresh file name databases}
%
% If your \TeX~distribution
% (\teTeX, \mikTeX, \dots) relies on file name databases, you must refresh
% these. For example, \teTeX\ users run \verb|texhash| or
% \verb|mktexlsr|.
%
% \subsection{Some details for the interested}
%
% \paragraph{Attached source.}
%
% The PDF documentation on CTAN also includes the
% \xfile{.dtx} source file. It can be extracted by
% AcrobatReader 6 or higher. Another option is \textsf{pdftk},
% e.g. unpack the file into the current directory:
% \begin{quote}
%   \verb|pdftk thepdfnumber.pdf unpack_files output .|
% \end{quote}
%
% \paragraph{Unpacking with \LaTeX.}
% The \xfile{.dtx} chooses its action depending on the format:
% \begin{description}
% \item[\plainTeX:] Run \docstrip\ and extract the files.
% \item[\LaTeX:] Generate the documentation.
% \end{description}
% If you insist on using \LaTeX\ for \docstrip\ (really,
% \docstrip\ does not need \LaTeX), then inform the autodetect routine
% about your intention:
% \begin{quote}
%   \verb|latex \let\install=y\input{thepdfnumber.dtx}|
% \end{quote}
% Do not forget to quote the argument according to the demands
% of your shell.
%
% \paragraph{Generating the documentation.}
% You can use both the \xfile{.dtx} or the \xfile{.drv} to generate
% the documentation. The process can be configured by the
% configuration file \xfile{ltxdoc.cfg}. For instance, put this
% line into this file, if you want to have A4 as paper format:
% \begin{quote}
%   \verb|\PassOptionsToClass{a4paper}{article}|
% \end{quote}
% An example follows how to generate the
% documentation with pdf\LaTeX:
% \begin{quote}
%\begin{verbatim}
%pdflatex thepdfnumber.dtx
%makeindex -s gind.ist thepdfnumber.idx
%pdflatex thepdfnumber.dtx
%makeindex -s gind.ist thepdfnumber.idx
%pdflatex thepdfnumber.dtx
%\end{verbatim}
% \end{quote}
%
% \printbibliography[
%   heading=bibnumbered,
% ]
%
% \begin{History}
%   \begin{Version}{2011/11/24 v1.0}
%   \item
%     First version.
%   \end{Version}
%   \begin{Version}{2016/05/16 v1.1}
%   \item
%     Documentation updates.
%   \end{Version}
% \end{History}
%
% \PrintIndex
%
% \Finale
\endinput
|
% \end{quote}
% Do not forget to quote the argument according to the demands
% of your shell.
%
% \paragraph{Generating the documentation.}
% You can use both the \xfile{.dtx} or the \xfile{.drv} to generate
% the documentation. The process can be configured by the
% configuration file \xfile{ltxdoc.cfg}. For instance, put this
% line into this file, if you want to have A4 as paper format:
% \begin{quote}
%   \verb|\PassOptionsToClass{a4paper}{article}|
% \end{quote}
% An example follows how to generate the
% documentation with pdf\LaTeX:
% \begin{quote}
%\begin{verbatim}
%pdflatex thepdfnumber.dtx
%makeindex -s gind.ist thepdfnumber.idx
%pdflatex thepdfnumber.dtx
%makeindex -s gind.ist thepdfnumber.idx
%pdflatex thepdfnumber.dtx
%\end{verbatim}
% \end{quote}
%
% \printbibliography[
%   heading=bibnumbered,
% ]
%
% \begin{History}
%   \begin{Version}{2011/11/24 v1.0}
%   \item
%     First version.
%   \end{Version}
%   \begin{Version}{2016/05/16 v1.1}
%   \item
%     Documentation updates.
%   \end{Version}
% \end{History}
%
% \PrintIndex
%
% \Finale
\endinput
|
% \end{quote}
% Do not forget to quote the argument according to the demands
% of your shell.
%
% \paragraph{Generating the documentation.}
% You can use both the \xfile{.dtx} or the \xfile{.drv} to generate
% the documentation. The process can be configured by the
% configuration file \xfile{ltxdoc.cfg}. For instance, put this
% line into this file, if you want to have A4 as paper format:
% \begin{quote}
%   \verb|\PassOptionsToClass{a4paper}{article}|
% \end{quote}
% An example follows how to generate the
% documentation with pdf\LaTeX:
% \begin{quote}
%\begin{verbatim}
%pdflatex thepdfnumber.dtx
%makeindex -s gind.ist thepdfnumber.idx
%pdflatex thepdfnumber.dtx
%makeindex -s gind.ist thepdfnumber.idx
%pdflatex thepdfnumber.dtx
%\end{verbatim}
% \end{quote}
%
% \printbibliography[
%   heading=bibnumbered,
% ]
%
% \begin{History}
%   \begin{Version}{2011/11/24 v1.0}
%   \item
%     First version.
%   \end{Version}
%   \begin{Version}{2016/05/16 v1.1}
%   \item
%     Documentation updates.
%   \end{Version}
% \end{History}
%
% \PrintIndex
%
% \Finale
\endinput

%        (quote the arguments according to the demands of your shell)
%
% Documentation:
%    (a) If thepdfnumber.drv is present:
%           latex thepdfnumber.drv
%    (b) Without thepdfnumber.drv:
%           latex thepdfnumber.dtx; ...
%    The class ltxdoc loads the configuration file ltxdoc.cfg
%    if available. Here you can specify further options, e.g.
%    use A4 as paper format:
%       \PassOptionsToClass{a4paper}{article}
%
%    Programm calls to get the documentation (example):
%       pdflatex thepdfnumber.dtx
%       makeindex -s gind.ist thepdfnumber.idx
%       pdflatex thepdfnumber.dtx
%       makeindex -s gind.ist thepdfnumber.idx
%       pdflatex thepdfnumber.dtx
%
% Installation:
%    TDS:tex/generic/oberdiek/thepdfnumber.sty
%    TDS:doc/latex/oberdiek/thepdfnumber.pdf
%    TDS:doc/latex/oberdiek/test/thepdfnumber-test1.tex
%    TDS:doc/latex/oberdiek/test/thepdfnumber-test2.tex
%    TDS:doc/latex/oberdiek/test/thepdfnumber-test3.tex
%    TDS:source/latex/oberdiek/thepdfnumber.dtx
%
%<*ignore>
\begingroup
  \catcode123=1 %
  \catcode125=2 %
  \def\x{LaTeX2e}%
\expandafter\endgroup
\ifcase 0\ifx\install y1\fi\expandafter
         \ifx\csname processbatchFile\endcsname\relax\else1\fi
         \ifx\fmtname\x\else 1\fi\relax
\else\csname fi\endcsname
%</ignore>
%<*install>
\input docstrip.tex
\Msg{************************************************************************}
\Msg{* Installation}
\Msg{* Package: thepdfnumber 2016/05/16 v1.1 Print PDF numbers with minimal digits (HO)}
\Msg{************************************************************************}

\keepsilent
\askforoverwritefalse

\let\MetaPrefix\relax
\preamble

This is a generated file.

Project: thepdfnumber
Version: 2016/05/16 v1.1

Copyright (C) 2011 by
   Heiko Oberdiek <heiko.oberdiek at googlemail.com>

This work may be distributed and/or modified under the
conditions of the LaTeX Project Public License, either
version 1.3c of this license or (at your option) any later
version. This version of this license is in
   http://www.latex-project.org/lppl/lppl-1-3c.txt
and the latest version of this license is in
   http://www.latex-project.org/lppl.txt
and version 1.3 or later is part of all distributions of
LaTeX version 2005/12/01 or later.

This work has the LPPL maintenance status "maintained".

This Current Maintainer of this work is Heiko Oberdiek.

The Base Interpreter refers to any `TeX-Format',
because some files are installed in TDS:tex/generic//.

This work consists of the main source file thepdfnumber.dtx
and the derived files
   thepdfnumber.sty, thepdfnumber.pdf, thepdfnumber.ins, thepdfnumber.drv,
   thepdfnumber-test1.tex, thepdfnumber-test2.tex,
   thepdfnumber-test3.tex.

\endpreamble
\let\MetaPrefix\DoubleperCent

\generate{%
  \file{thepdfnumber.ins}{\from{thepdfnumber.dtx}{install}}%
  \file{thepdfnumber.drv}{\from{thepdfnumber.dtx}{driver}}%
  \usedir{tex/generic/oberdiek}%
  \file{thepdfnumber.sty}{\from{thepdfnumber.dtx}{package}}%
  \usedir{doc/latex/oberdiek/test}%
  \file{thepdfnumber-test1.tex}{\from{thepdfnumber.dtx}{test1}}%
  \file{thepdfnumber-test2.tex}{\from{thepdfnumber.dtx}{test2}}%
  \file{thepdfnumber-test3.tex}{\from{thepdfnumber.dtx}{test2,active}}%
}

\catcode32=13\relax% active space
\let =\space%
\Msg{************************************************************************}
\Msg{*}
\Msg{* To finish the installation you have to move the following}
\Msg{* file into a directory searched by TeX:}
\Msg{*}
\Msg{*     thepdfnumber.sty}
\Msg{*}
\Msg{* To produce the documentation run the file `thepdfnumber.drv'}
\Msg{* through LaTeX.}
\Msg{*}
\Msg{* Happy TeXing!}
\Msg{*}
\Msg{************************************************************************}

\endbatchfile
%</install>
%<*ignore>
\fi
%</ignore>
%<*driver>
\NeedsTeXFormat{LaTeX2e}
\ProvidesFile{thepdfnumber.drv}%
  [2016/05/16 v1.1 Print PDF numbers with minimal digits (HO)]%
\documentclass{ltxdoc}
\usepackage{holtxdoc}[2011/11/22]
\usepackage{csquotes}
\usepackage[
  backend=bibtex,
  bibencoding=ascii,
  alldates=iso8601,
]{biblatex}[2011/11/13]
\bibliography{oberdiek-source}
\begin{document}
  \DocInput{thepdfnumber.dtx}%
\end{document}
%</driver>
% \fi
%
%
% \CharacterTable
%  {Upper-case    \A\B\C\D\E\F\G\H\I\J\K\L\M\N\O\P\Q\R\S\T\U\V\W\X\Y\Z
%   Lower-case    \a\b\c\d\e\f\g\h\i\j\k\l\m\n\o\p\q\r\s\t\u\v\w\x\y\z
%   Digits        \0\1\2\3\4\5\6\7\8\9
%   Exclamation   \!     Double quote  \"     Hash (number) \#
%   Dollar        \$     Percent       \%     Ampersand     \&
%   Acute accent  \'     Left paren    \(     Right paren   \)
%   Asterisk      \*     Plus          \+     Comma         \,
%   Minus         \-     Point         \.     Solidus       \/
%   Colon         \:     Semicolon     \;     Less than     \<
%   Equals        \=     Greater than  \>     Question mark \?
%   Commercial at \@     Left bracket  \[     Backslash     \\
%   Right bracket \]     Circumflex    \^     Underscore    \_
%   Grave accent  \`     Left brace    \{     Vertical bar  \|
%   Right brace   \}     Tilde         \~}
%
% \GetFileInfo{thepdfnumber.drv}
%
% \title{The \xpackage{thepdfnumber} package}
% \date{2016/05/16 v1.1}
% \author{Heiko Oberdiek\thanks
% {Please report any issues at https://github.com/ho-tex/oberdiek/issues}\\
% \xemail{heiko.oberdiek at googlemail.com}}
%
% \maketitle
%
% \begin{abstract}
% The package converts real numbers to a minimal representation
% that is stripped from leading or trailing zeros,
% plus signs and decimal point if not necessary.
% \end{abstract}
%
% \tableofcontents
%
% \section{Documentation}
%
% \subsection{Introduction}
%
% Dealing with the PDF format, there is sometimes the need to write
% some low level PDF stuff. In case of numbers, the numbers can
% arise from user input (e.g. color or transparency specifications)
% or can be calculated. For example, \hologo{LaTeX}'s \cs{strip@pt}
% makes a good job to output a real number. It automatically suppresses
% the decimal part if the number is an integer. However it leaves
% a leading zero for numbers greater zero and smaller one.
% Thus the package provides macros that can be used with different
% formats, even with \hologo{iniTeX} and generates numbers that
% are valid numbers of the PDF format and whose length is minimal.
%
% \subsection{Usage}
%
% The package \xpackage{thepdfnumber} can be used with \hologo{LaTeX},
% \hologo{plainTeX} or even with \hologo{iniTeX}:
% \begin{quote}
%   |\RequirePackage{thepdfnumber}| \textit{\% \hologo{LaTeX}}\\
%   |\input thepdfnumber.sty| \textit{\% \hologo{plainTeX}/\hologo{iniTeX}}
% \end{quote}
% The package does not need and have package options.
%
% \subsection{User macros}
%
% All user macros are expandable in exact two expansion steps.
%
% \begin{declcs}{thepdfnumber} \M{number}
% \end{declcs}
% Macro \cs{thepdfnumber} takes a number
% as argument and expands to a minimal representation of that number.
% Some examples:
% \begin{quote}
% \begin{tabular}{r@{.: }>{\ttfamily}l@{ $\rightarrow$\ }>{\ttfamily}l@{}}
%   1& +123 & 123\\
%   2& --123 & 123\\
%   3& -01 & -1\\
%   4& 0045 & 45\\
%   5& 1.0 & 1\\
%   6& 1.20 & 1.2\\
%   7& 0.0 & 0\\
%   8& 0.78 & .78\\
%   9& +012.340 & 12.34\\
% \end{tabular}
% \end{quote}
% It reduces the length of the number representation:
% \begin{itemize}
% \item The signs are collapsed and only one minus sign
%   is output if the number is negative (see examples 1, 2, 3, 9).
% \item Leading zeros are removed (4, 8, 9) unless the number is zero (7).
% \item The decimal part is omitted, if the number is an integer (5, 7).
% \item Trailing zeros from the decimal part are stripped (5, 6, 7, 9).
% \end{itemize}
% The resulting number representation can be catched with one of the
% following Perl regular expressions:
% \begin{itemize}
% \item |^0$| (zero)
% \item |^-?[1-9][0-9]*$| (integer)
% \item |^-?[0-9]*\.[0-9]*[1-9]$| (real)
% \end{itemize}
% This is a valid numeric object of the PDF specification \cite[``7.3.3~Numeric Objects'']{pdfspec-iso32000-1}.
%
% \begin{declcs}{thepdfnumberNormZeroOne}
% \end{declcs}
% There are various places in the PDF specification where the
% number is in the domain 0.0 upto 1.0. Macro thepdfnumberNormZeroOne
% automatically adjusts the number to fit into that range. Negative
% numbers are mapped to |0| and numbers greater than one are
% replaced by |1|. Thus the result fits one of the following regular
% expressions:
% \begin{itemize}
% \item |^0$|
% \item |^\.[0-9]*[1-9]$|
% \item |^1$|
% \end{itemize}
% Examples:
% \begin{quote}
% \begin{tabular}{@{}>{\ttfamily}l@{ $\rightarrow$\ }>{\ttfamily}l@{}}
%   -456 & 0\\
%   -0.001 & 0\\
%   0.0 & 0\\
%   0.010 & .01\\
%   0.456 & .456\\
%   1.0 & 1\\
%   01.001 & 1\\
%   4 & 1\\
% \end{tabular}
% \end{quote}
%
% \subsection{Input number}
%
% The user macros expect a number as argument. The number can either
% be given explicitely or as macro that expands in one step
% to an explicite number, because the first token of the argument
% is expanded once.
%
% The explicite number consists of
% \begin{itemize}
% \item optional signs `\texttt{+}' and `\texttt{-}',
% \item digits `\texttt{0}' upto `\texttt{9}' and
% \item an optional dot `\texttt{.}'.
% \end{itemize}
% All tokens must have catcode 12 (other), the default catcodes
% for these characters in \hologo{LaTeX}, \hologo{plainTeX} or
% \hologo{iniTeX}. As Perl regular expression the number is expected
% in one of the following forms:
% \begin{itemize}
% \item |^[+-]*[0-9]+$|
% \item |^[+-]*[0-9]*\.[0-9]*$|
% \end{itemize}
% At least one digit or the dot must be present.
%
% \subsection{Error handling}
%
% The package is not intended for validating numbers or to
% decide if an argument is a number. Therefore it is an usage
% error to use the user macros with arguments that are not
% explicite numbers as specified in the previous sections.
% Nevertheless some error conditions are sometimes recognized.
% Errors are given in form of an undefined command sequence.
% It is the only way to notify \hologo{TeX} in expandable context.
% Expanding to some error text would invalidate the output.
% Currently the following errors are thrown:
% \begin{description}
% \item[\cs{thepdfnumber@ErrorEndMarker}:]
%   Internally the argument parsing uses an end marker that
%   is never called directly. If it is called with valid
%   user input, then this is a bug. Otherwise it means the user input
%   contains nasty stuff.
% \item[\cs{thepdfnumber@ErrorUnexpectedEnd}:] The macros expect at least one digit
%   or the dot, otherwise if the argument is empty or only contains
%   signs, then this error is called.
% \item[\cs{thepdfnumber@ErrorInvalidToken}:] It is called if the number
%   contains other tokens than signs, digits or the dot or the
%   token at the wrong place (e.g. a sign after a digit).
%   In case of \cs{thepdfnumberNormZeroOne} this error condition might
%   not always be detected, because the number parsing might stop
%   at an early point, when the result is already clear (e.g.
%   if the number is negative or will be greater than one).
% \item[Improper alphabetic constant:] This error might be
%   thrown by \hologo{TeX}, if the number contains
%   command tokens instead of characters.
% \end{description}
%
% \StopEventually{
% }
%
% \section{Implementation}
%
%    \begin{macrocode}
%<*package>
%    \end{macrocode}
%
% \subsection{Reload check and package identification}
%    Reload check, especially if the package is not used with \LaTeX.
%    \begin{macrocode}
\begingroup\catcode61\catcode48\catcode32=10\relax%
  \catcode13=5 % ^^M
  \endlinechar=13 %
  \catcode35=6 % #
  \catcode39=12 % '
  \catcode44=12 % ,
  \catcode45=12 % -
  \catcode46=12 % .
  \catcode58=12 % :
  \catcode64=11 % @
  \catcode123=1 % {
  \catcode125=2 % }
  \expandafter\let\expandafter\x\csname ver@thepdfnumber.sty\endcsname
  \ifx\x\relax % plain-TeX, first loading
  \else
    \def\empty{}%
    \ifx\x\empty % LaTeX, first loading,
      % variable is initialized, but \ProvidesPackage not yet seen
    \else
      \expandafter\ifx\csname PackageInfo\endcsname\relax
        \def\x#1#2{%
          \immediate\write-1{Package #1 Info: #2.}%
        }%
      \else
        \def\x#1#2{\PackageInfo{#1}{#2, stopped}}%
      \fi
      \x{thepdfnumber}{The package is already loaded}%
      \aftergroup\endinput
    \fi
  \fi
\endgroup%
%    \end{macrocode}
%    Package identification:
%    \begin{macrocode}
\begingroup\catcode61\catcode48\catcode32=10\relax%
  \catcode13=5 % ^^M
  \endlinechar=13 %
  \catcode35=6 % #
  \catcode39=12 % '
  \catcode40=12 % (
  \catcode41=12 % )
  \catcode44=12 % ,
  \catcode45=12 % -
  \catcode46=12 % .
  \catcode47=12 % /
  \catcode58=12 % :
  \catcode64=11 % @
  \catcode91=12 % [
  \catcode93=12 % ]
  \catcode123=1 % {
  \catcode125=2 % }
  \expandafter\ifx\csname ProvidesPackage\endcsname\relax
    \def\x#1#2#3[#4]{\endgroup
      \immediate\write-1{Package: #3 #4}%
      \xdef#1{#4}%
    }%
  \else
    \def\x#1#2[#3]{\endgroup
      #2[{#3}]%
      \ifx#1\@undefined
        \xdef#1{#3}%
      \fi
      \ifx#1\relax
        \xdef#1{#3}%
      \fi
    }%
  \fi
\expandafter\x\csname ver@thepdfnumber.sty\endcsname
\ProvidesPackage{thepdfnumber}%
  [2016/05/16 v1.1 Print PDF numbers with minimal digits (HO)]%
%    \end{macrocode}
%
% \subsection{Catcodes}
%
%    \begin{macrocode}
\begingroup\catcode61\catcode48\catcode32=10\relax%
  \catcode13=5 % ^^M
  \endlinechar=13 %
  \catcode123=1 % {
  \catcode125=2 % }
  \catcode64=11 % @
  \def\x{\endgroup
    \expandafter\edef\csname ThPdNu@AtEnd\endcsname{%
      \endlinechar=\the\endlinechar\relax
      \catcode13=\the\catcode13\relax
      \catcode32=\the\catcode32\relax
      \catcode35=\the\catcode35\relax
      \catcode61=\the\catcode61\relax
      \catcode64=\the\catcode64\relax
      \catcode123=\the\catcode123\relax
      \catcode125=\the\catcode125\relax
    }%
  }%
\x\catcode61\catcode48\catcode32=10\relax%
\catcode13=5 % ^^M
\endlinechar=13 %
\catcode35=6 % #
\catcode64=11 % @
\catcode123=1 % {
\catcode125=2 % }
\def\TMP@EnsureCode#1#2{%
  \edef\ThPdNu@AtEnd{%
    \ThPdNu@AtEnd
    \catcode#1=\the\catcode#1\relax
  }%
  \catcode#1=#2\relax
}
\TMP@EnsureCode{33}{12}% !
\TMP@EnsureCode{36}{3}% $
\TMP@EnsureCode{38}{4}% &
\TMP@EnsureCode{42}{12}% *
\TMP@EnsureCode{43}{12}% +
\TMP@EnsureCode{45}{12}% -
\TMP@EnsureCode{46}{12}% .
\TMP@EnsureCode{60}{12}% <
\TMP@EnsureCode{62}{12}% >
\TMP@EnsureCode{96}{12}% `
\edef\ThPdNu@AtEnd{\ThPdNu@AtEnd\noexpand\endinput}
%    \end{macrocode}
%
% \subsection{Helper macros}
%
%    \begin{macro}{\ThPdNu@FIN}
%    \begin{macrocode}
\def\ThPdNu@FIN{\thepdfnumber@ErrorEndMarker}
%    \end{macrocode}
%    \end{macro}
%
%    \begin{macro}{\ThPdNu@space}
%    \begin{macrocode}
\def\ThPdNu@space{ }
%    \end{macrocode}
%    \end{macro}
%
%    \begin{macro}{\ThPdNu@zero}
%    \begin{macrocode}
\chardef\ThPdNu@zero=0 %
%    \end{macrocode}
%    \end{macro}
%
%    \begin{macro}{\ThPdNu@one}
%    \begin{macrocode}
\chardef\ThPdNu@one=1 %
%    \end{macrocode}
%    \end{macro}
%
%    \begin{macro}{\ThPdNu@firstoftwo}
%    \begin{macrocode}
\long\def\ThPdNu@firstoftwo#1#2{#1}
%    \end{macrocode}
%    \end{macro}
%
%    \begin{macro}{\ThPdNu@secondoftwo}
%    \begin{macrocode}
\long\def\ThPdNu@secondoftwo#1#2{#2}
%    \end{macrocode}
%    \end{macro}
%
% \subsection{Detect \hologo{eTeX}}
%
%    \begin{macrocode}
\begingroup\expandafter\expandafter\expandafter\endgroup
\expandafter\ifx\csname detokenize\endcsname\relax
  \catcode`\&=14 %
  \catcode`\$=9 %
\else
  \catcode`\&=9 %
  \catcode`\$=14 %
\fi
%    \end{macrocode}
%
% \subsection{User macro \cs{thepdfnumber}}
%
%    \begin{macro}{\thepdfnumber}
%    \begin{macrocode}
\def\thepdfnumber#1{%
  \romannumeral
& \iftrue\expandafter\ThPdNu@State@Plus\expandafter\fi
& \detokenize\expandafter{#1}%
& \ThPdNu@FIN
$ \ifx\ThPdNu@FIN#1\ThPdNu@FIN
$   \expandafter\ThPdNu@firstoftwo
$ \else
$   \expandafter\ThPdNu@secondoftwo
$ \fi
$ {%
$   \ThPdNu@zero
$   0\thepdfnumber@ErrorUnexpectedEnd
$ }{%
$   \iftrue\expandafter\ThPdNu@State@Plus\expandafter\fi#1\ThPdNu@FIN
$ }%
}
%    \end{macrocode}
%    \end{macro}
%
% \subsubsection{State definitions for sign}
%
%    \begin{macro}{\ThPdNu@State@Plus}
%    \begin{macrocode}
\def\ThPdNu@State@Plus#1\fi#2{%
  \fi
  \ifcase\ifx\ThPdNu@FIN#2%
           0%
&        \else\ifx-#2%
$        \else\ifnum`#2=45 % -
           1%
         \else\ifx0#2%
           2%
         \else\ifnum`#2>48 %
           \ifnum`#2<58 %
             3%
           \else
             9%
           \fi
&        \else\ifx.#2%
$        \else\ifnum`#2=46 % .
           4%
&        \else\ifx+#2%
$        \else\ifnum`#2=43 % +
           5%
         \else
           9%
         \fi\fi\fi\fi\fi\fi\ThPdNu@space
    \expandafter\ThPdNu@zero
    \expandafter0%
    \expandafter\thepdfnumber@ErrorUnexpectedEnd
  \or
    \ThPdNu@State@Minus
  \or
    \ThPdNu@State@SkipZeros!%
  \or
    \ThPdNu@State@Int!#2!%
  \or
    \ThPdNu@State@Dot!\ThPdNu@zero*\ThPdNu@zero!!%
  \or
    \ThPdNu@State@Plus
  \else
    \ThPdNu@ReturnError{0}%
  \fi
}
%    \end{macrocode}
%    \end{macro}
%
%    \begin{macro}{\ThPdNu@State@Minus}
%    \begin{macrocode}
\def\ThPdNu@State@Minus#1\fi#2{%
  \fi
  \ifcase\ifx\ThPdNu@FIN#2%
           0%
         \else\ifx0#2%
           1%
         \else\ifnum`#2>48 %
           \ifnum`#2<58 %
             2%
           \else
             9%
           \fi
&        \else\ifx.#2%
$        \else\ifnum`#2=46 % .
           3%
&        \else\ifx-#2%
$        \else\ifnum`#2=45 % -
           4%
&        \else\ifx+#2%
$        \else\ifnum`#2=43 % +
           5%
         \else
           9%
         \fi\fi\fi\fi\fi\fi\ThPdNu@space
    \expandafter\ThPdNu@zero
    \expandafter0%
    \expandafter\thepdfnumber@ErrorUnexpectedEnd
  \or
    \ThPdNu@State@SkipZeros-!%
  \or
    \ThPdNu@State@Int-!#2!%
  \or
    \ThPdNu@State@Dot-!\ThPdNu@zero*\ThPdNu@zero!!%
  \or
    \ThPdNu@State@Plus
  \or
    \ThPdNu@State@Minus
  \else
    \ThPdNu@ReturnError{0}%
  \fi
}
%    \end{macrocode}
%    \end{macro}
%
%    \begin{macro}{\ThPdNu@ReturnError}
%    \begin{macrocode}
\def\ThPdNu@ReturnError#1#2\fi#3\ThPdNu@FIN{%
  \fi
  \ThPdNu@zero
  #1%
  \thepdfnumber@ErrorInvalidToken
}
%    \end{macrocode}
%    \end{macro}
%
% \subsubsection{State definitions for integer part}
%
%    \begin{macro}{\ThPdNu@State@SkipZeros}
%    \begin{macrocode}
\def\ThPdNu@State@SkipZeros#1!#2\fi#3{%
  \fi
  \ifcase\ifx\ThPdNu@FIN#3%
           0%
         \else\ifx0#3%
           1%
         \else\ifnum`#3>48 %
           \ifnum`#3<58 %
             2%
           \else
             9%
           \fi
&        \else\ifx.#3%
$        \else\ifnum`#3=46 % .
           3%
         \else
           9%
         \fi\fi\fi\fi\ThPdNu@space
    \expandafter\ThPdNu@zero
    \expandafter0%
  \or
   \ThPdNu@State@SkipZeros#1!%
  \or
    \ThPdNu@State@Int#1!#3!%
  \or
    \ThPdNu@State@Dot#1!\ThPdNu@zero*\ThPdNu@zero!!%
  \else
    \ThPdNu@ReturnError{0}%
  \fi
}
%    \end{macrocode}
%    \end{macro}
%
%    \begin{macro}{\ThPdNu@State@Int}
%    \begin{macrocode}
\def\ThPdNu@State@Int#1!#2!#3\fi#4{%
  \fi
  \ifcase\ifx\ThPdNu@FIN#4%
           0%
         \else\ifnum`#4>47 %
           \ifnum`#4<58 %
             1%
           \else
             9%
           \fi
&        \else\ifx.#4%
$        \else\ifnum`#4=46 % .
           2%
         \else
           9%
         \fi\fi\fi\ThPdNu@space
    \ThPdNu@ReturnInt{#1#2}%
  \or
    \ThPdNu@State@Int#1!#2#4!%
  \or
    \ThPdNu@State@Dot#1!\ThPdNu@one#2*\ThPdNu@zero!!%
  \else
    \ThPdNu@ReturnError{#1#2}%
  \fi
}
%    \end{macrocode}
%    \end{macro}
%
%    \begin{macro}{\ThPdNu@ReturnInt}
%    \begin{macrocode}
\def\ThPdNu@ReturnInt#1#2\fi{%
  \fi
  \ThPdNu@zero
  #1%
}
%    \end{macrocode}
%    \end{macro}
%
% \subsubsection{State definitions for decimal digits}
%
%    \begin{macro}{\ThPdNu@State@Dot}
%    \begin{macrocode}
\def\ThPdNu@State@Dot#1*#2#3!#4!#5\fi#6{%
  \fi
  \ifcase\ifx\ThPdNu@FIN#6%
           0%
         \else\ifnum`#6>48 %
           \ifnum`#6<58 %
             1%
           \else
             9%
           \fi
         \else\ifx0#6%
           2%
         \else
           9%
         \fi\fi\fi\ThPdNu@space
    \ThPdNu@ReturnNumber#1*#2#3!%
  \or
    \ThPdNu@State@Dot#1*\ThPdNu@one#3#4#6!!%
  \or
    \ThPdNu@State@DotZero#1*#2#3!#4#6!%
  \else
    \ThPdNu@ReturnNumberInvalid#1*#2#3!%
  \fi
}
%    \end{macrocode}
%    \end{macro}
%
%    \begin{macro}{\ThPdNu@State@DotZero}
%    \begin{macrocode}
\def\ThPdNu@State@DotZero#1*#2#3!#4!#5\fi#6{%
  \fi
  \ifcase\ifx\ThPdNu@FIN#6%
           0%
         \else\ifnum`#6>48 %
           \ifnum`#6<58 %
             1%
           \else
             9%
           \fi
         \else\ifx0#6%
           2%
         \else
           9%
         \fi\fi\fi\ThPdNu@space
    \ThPdNu@ReturnNumber#1*#2#3!%
  \or
    \ThPdNu@State@Dot#1*\ThPdNu@one#3#4#6!!%
  \or
    \ThPdNu@State@DotZero#1*#2#3!#4#6!%
  \else
    \ThPdNu@ReturnNumber#1*#2#3!%
  \fi
}
%    \end{macrocode}
%    \end{macro}
%
%    \begin{macro}{\ThPdNu@ReturnNumber}
%    \begin{macrocode}
\def\ThPdNu@ReturnNumber#1!#2#3*#4#5!#6\fi{%
  \fi
  \ifcase#2%
    \expandafter\ThPdNu@firstoftwo
  \else
    \expandafter\ThPdNu@secondoftwo
  \fi
  {%
    \ifcase#4%
      \expandafter\ThPdNu@firstoftwo
    \else
      \expandafter\ThPdNu@secondoftwo
    \fi
    {\ThPdNu@zero 0}%
    {\ThPdNu@zero #1.#5}%
  }{%
    \ifcase#4%
      \expandafter\ThPdNu@firstoftwo
    \else
      \expandafter\ThPdNu@secondoftwo
    \fi
    {\ThPdNu@zero #1#3}%
    {\ThPdNu@zero #1#3.#5}%
  }%
}
%    \end{macrocode}
%    \end{macro}
%
%    \begin{macro}{\ThPdNu@ReturnNumberInvalid}
%    \begin{macrocode}
\def\ThPdNu@ReturnNumberInvalid#1*#2!#3\fi#4\ThPdNu@FIN{%
  \fi
  \iftrue\ThPdNu@ReturnNumber#1*#2!\fi
  \thepdfnumber@ErrorInvalidToken
}
%    \end{macrocode}
%    \end{macro}
%
% \subsection{Norm macro}
%
%    \begin{macro}{\thepdfnumberNormZeroOne}
%    \begin{macrocode}
\def\thepdfnumberNormZeroOne#1{%
  \romannumeral
& \iftrue\expandafter\ThPbNu@StateN@Plus\expandafter\fi
& \detokenize\expandafter{#1}%
& \ThPdNu@FIN
$ \ifx\ThPdNu@FIN#1\ThPdNu@FIN
$   \expandafter\ThPdNu@firstoftwo
$ \else
$   \expandafter\ThPdNu@secondoftwo
$ \fi
$ {%
$   \ThPdNu@zero
$   0\thepdfnumber@ErrorUnexpectedEnd
$ }{%
$   \iftrue\expandafter\ThPbNu@StateN@Plus\expandafter\fi#1\ThPdNu@FIN
$ }%
}
%    \end{macrocode}
%    \end{macro}
%
% \subsubsection{State definitions for sign}
%
%    \begin{macro}{\ThPbNu@StateN@Plus}
%    \begin{macrocode}
\def\ThPbNu@StateN@Plus#1\fi#2{%
  \fi
  \ifcase\ifx\ThPdNu@FIN#2%
           0%
&        \else\ifx-#2%
$        \else\ifnum`#2=45 % -
           1%
         \else\ifx0#2%
           2%
         \else\ifnum`#2>48 %
           \ifnum`#2<58 %
             3%
           \else
             9%
           \fi
&        \else\ifx.#2%
$        \else\ifnum`#2=46 % .
           4%
&        \else\ifx+#2%
$        \else\ifnum`#2=43 % +
           5%
         \else
           9%
         \fi\fi\fi\fi\fi\fi\ThPdNu@space
    \expandafter\ThPdNu@zero
    \expandafter0%
    \expandafter\thepdfnumber@ErrorUnexpectedEnd
  \or
    \ThPbNu@StateN@Minus
  \or
    \ThPbNu@StateN@SkipZeros
  \or
    \ThPdNu@ReturnAndSkip{1}%
  \or
    \ThPbNu@StateN@Dot\ThPdNu@zero!!%
  \or
    \ThPbNu@StateN@Plus
  \else
    \ThPdNu@ReturnError{0}%
  \fi
}
%    \end{macrocode}
%    \end{macro}
%
%    \begin{macro}{\ThPbNu@StateN@Minus}
%    \begin{macrocode}
\def\ThPbNu@StateN@Minus#1\fi#2{%
  \fi
  \ifcase\ifx\ThPdNu@FIN#2%
           0%
         \else\ifnum`#2>47 %
           \ifnum`#2<58 %
             1%
           \else
             9%
           \fi
&        \else\ifx.#2%
$        \else\ifnum`#2=46 % .
           1%
&        \else\ifx-#2%
$        \else\ifnum`#2=45 % -
           2%
&        \else\ifx+#2%
$        \else\ifnum`#2=43 % +
           3%
         \else
           9%
         \fi\fi\fi\fi\fi\ThPdNu@space
    \expandafter\ThPdNu@zero
    \expandafter0%
    \expandafter\thepdfnumber@ErrorUnexpectedEnd
  \or
    \ThPdNu@ReturnAndSkip{0}%
  \or
    \ThPbNu@StateN@Plus
  \or
    \ThPbNu@StateN@Minus
  \else
    \ThPdNu@ReturnError{0}%
  \fi
}
%    \end{macrocode}
%    \end{macro}
%
%    \begin{macro}{\ThPdNu@ReturnAndSkip}
%    \begin{macrocode}
\def\ThPdNu@ReturnAndSkip#1#2\fi#3\ThPdNu@FIN{%
  \fi
  \ThPdNu@zero
  #1%
}
%    \end{macrocode}
%    \end{macro}
%
% \subsubsection{State definitions for integer part}
%
%    \begin{macro}{\ThPbNu@StateN@SkipZeros}
%    \begin{macrocode}
\def\ThPbNu@StateN@SkipZeros#1\fi#2{%
  \fi
  \ifcase\ifx\ThPdNu@FIN#2%
           0%
         \else\ifx0#2%
           1%
         \else\ifnum`#2>48 %
           \ifnum`#2<58 %
             2%
           \else
             9%
           \fi
&        \else\ifx.#2%
$        \else\ifnum`#2=46 % .
           3%
         \else
           9%
         \fi\fi\fi\fi\ThPdNu@space
    \expandafter\ThPdNu@zero
    \expandafter0%
  \or
   \ThPbNu@StateN@SkipZeros%
  \or
    \ThPdNu@ReturnAndSkip{1}%
  \or
    \ThPbNu@StateN@Dot\ThPdNu@zero!!%
  \else
    \ThPdNu@ReturnError{0}%
  \fi
}
%    \end{macrocode}
%    \end{macro}
%
% \subsubsection{State definitions for decimal digits}
%
%    \begin{macro}{\ThPbNu@StateN@Dot}
%    \begin{macrocode}
\def\ThPbNu@StateN@Dot#1#2!#3!#4\fi#5{%
  \fi
  \ifcase\ifx\ThPdNu@FIN#5%
           0%
         \else\ifnum`#5>48 %
           \ifnum`#5<58 %
             1%
           \else
             9%
           \fi
         \else\ifx0#5%
           2%
         \else
           9%
         \fi\fi\fi\ThPdNu@space
    \ThPdNu@ReturnFracNumber#1#2!%
  \or
    \ThPbNu@StateN@Dot\ThPdNu@one#2#3#5!!%
  \or
    \ThPbNu@StateN@DotZero#1#2!#3#5!%
  \else
    \ThPdNu@ReturnFracNumberInvalid#1#2!%
  \fi
}
%    \end{macrocode}
%    \end{macro}
%
%    \begin{macro}{\ThPbNu@StateN@DotZero}
%    \begin{macrocode}
\def\ThPbNu@StateN@DotZero#1#2!#3!#4\fi#5{%
  \fi
  \ifcase\ifx\ThPdNu@FIN#5%
           0%
         \else\ifnum`#5>48 %
           \ifnum`#5<58 %
             1%
           \else
             9%
           \fi
         \else\ifx0#5%
           2%
         \else
           9%
         \fi\fi\fi\ThPdNu@space
    \ThPdNu@ReturnFracNumber#1#2!%
  \or
    \ThPbNu@StateN@Dot\ThPdNu@one#2#3#5!!%
  \or
    \ThPbNu@StateN@DotZero#1#2!#3#5!%
  \else
    \ThPdNu@ReturnFracNumberInvalid#1#2!%
  \fi
}
%    \end{macrocode}
%    \end{macro}
%
%    \begin{macro}{\ThPdNu@ReturnFracNumber}
%    \begin{macrocode}
\def\ThPdNu@ReturnFracNumber#1#2!#3\fi{%
  \fi
  \ifcase#1%
    \expandafter\ThPdNu@firstoftwo
  \else
    \expandafter\ThPdNu@secondoftwo
  \fi
  {\ThPdNu@zero 0}%
  {\ThPdNu@zero .#2}%
}
%    \end{macrocode}
%    \end{macro}
%
%    \begin{macro}{\ThPdNu@ReturnFracNumberInvalid}
%    \begin{macrocode}
\def\ThPdNu@ReturnFracNumberInvalid#1!#2\fi#3\ThPdNu@FIN{%
  \fi
  \iftrue\ThPdNu@ReturnFracNumber#1!\fi
  \thepdfnumber@ErrorInvalidToken
}
%    \end{macrocode}
%    \end{macro}
%
%    \begin{macrocode}
\ThPdNu@AtEnd%
%</package>
%    \end{macrocode}
%
% \section{Test}
%
% \subsection{Catcode checks for loading}
%
%    \begin{macrocode}
%<*test1>
%    \end{macrocode}
%    \begin{macrocode}
\catcode`\{=1 %
\catcode`\}=2 %
\catcode`\#=6 %
\catcode`\@=11 %
\expandafter\ifx\csname count@\endcsname\relax
  \countdef\count@=255 %
\fi
\expandafter\ifx\csname @gobble\endcsname\relax
  \long\def\@gobble#1{}%
\fi
\expandafter\ifx\csname @firstofone\endcsname\relax
  \long\def\@firstofone#1{#1}%
\fi
\expandafter\ifx\csname loop\endcsname\relax
  \expandafter\@firstofone
\else
  \expandafter\@gobble
\fi
{%
  \def\loop#1\repeat{%
    \def\body{#1}%
    \iterate
  }%
  \def\iterate{%
    \body
      \let\next\iterate
    \else
      \let\next\relax
    \fi
    \next
  }%
  \let\repeat=\fi
}%
\def\RestoreCatcodes{}
\count@=0 %
\loop
  \edef\RestoreCatcodes{%
    \RestoreCatcodes
    \catcode\the\count@=\the\catcode\count@\relax
  }%
\ifnum\count@<255 %
  \advance\count@ 1 %
\repeat

\def\RangeCatcodeInvalid#1#2{%
  \count@=#1\relax
  \loop
    \catcode\count@=15 %
  \ifnum\count@<#2\relax
    \advance\count@ 1 %
  \repeat
}
\def\RangeCatcodeCheck#1#2#3{%
  \count@=#1\relax
  \loop
    \ifnum#3=\catcode\count@
    \else
      \errmessage{%
        Character \the\count@\space
        with wrong catcode \the\catcode\count@\space
        instead of \number#3%
      }%
    \fi
  \ifnum\count@<#2\relax
    \advance\count@ 1 %
  \repeat
}
\def\space{ }
\expandafter\ifx\csname LoadCommand\endcsname\relax
  \def\LoadCommand{\input thepdfnumber.sty\relax}%
\fi
\def\Test{%
  \RangeCatcodeInvalid{0}{47}%
  \RangeCatcodeInvalid{58}{64}%
  \RangeCatcodeInvalid{91}{96}%
  \RangeCatcodeInvalid{123}{255}%
  \catcode`\@=12 %
  \catcode`\\=0 %
  \catcode`\%=14 %
  \LoadCommand
  \RangeCatcodeCheck{0}{36}{15}%
  \RangeCatcodeCheck{37}{37}{14}%
  \RangeCatcodeCheck{38}{47}{15}%
  \RangeCatcodeCheck{48}{57}{12}%
  \RangeCatcodeCheck{58}{63}{15}%
  \RangeCatcodeCheck{64}{64}{12}%
  \RangeCatcodeCheck{65}{90}{11}%
  \RangeCatcodeCheck{91}{91}{15}%
  \RangeCatcodeCheck{92}{92}{0}%
  \RangeCatcodeCheck{93}{96}{15}%
  \RangeCatcodeCheck{97}{122}{11}%
  \RangeCatcodeCheck{123}{255}{15}%
  \RestoreCatcodes
}
\Test
\csname @@end\endcsname
\end
%    \end{macrocode}
%    \begin{macrocode}
%</test1>
%    \end{macrocode}
%
% \subsection{Macro tests}
%
% \subsubsection{Preamble}
%
%    \begin{macrocode}
%<*test2>
\catcode`\{=1
\catcode`\}=2
\catcode`\#=6
\catcode`\@=11
\errorcontextlines=10000 %
\def\msg#{\immediate\write16}
\def\space{ }
\begingroup\expandafter\expandafter\expandafter\endgroup
\expandafter\ifx\csname RequirePackage\endcsname\relax
  \input thepdfnumber.sty\relax
\else
  \RequirePackage{thepdfnumber}[2016/05/16]%
\fi
%    \end{macrocode}
%
%    \begin{macro}{\Test}
%    \begin{macrocode}
\def\Test#1#2{%
  \def\TestExpected{#2}%
%<*active>
  \edef\TestExpected{\TestExpected}%
%</active>
  \expandafter\expandafter\expandafter\def
  \expandafter\expandafter\expandafter\TestResult
  \expandafter\expandafter\expandafter{%
    \TestCommand{#1}%
  }%
  \begingroup
    \toks0\expandafter{\TestResult}%
    \toks2\expandafter{\TestExpected}%
    \ifx\TestResult\TestExpected
      \msg{* OK: #1 => \the\toks0}%
    \else
      \errmessage{FAILED: #1 => \the\toks0\space(\the\toks2)}%
    \fi
  \endgroup
}
%    \end{macrocode}
%    \end{macro}
%
%    \begin{macrocode}
\let\thepdfnumber@ErrorUnexpectedEnd\relax
\let\thepdfnumber@ErrorInvalidToken\relax
%    \end{macrocode}
%
% \subsubsection{Test with active catcodes}
%
%    \begin{macrocode}
%<*active>
\catcode`\~=13 %
\def\ThPdNu@Temp#1{%
  \begingroup
    \lccode`\~=`#1 %
  \lowercase{\endgroup
    \edef~%
  }{\string#1}%
  \catcode`#1=13 %
}
\ThPdNu@Temp{.}
\ThPdNu@Temp{-}
\ThPdNu@Temp{+}
%</active>
%    \end{macrocode}
%
% \subsubsection{Test cases for \cs{thepdfnumber}}
%
%    \begin{macro}{\TestCommand}
%    \begin{macrocode}
\let\TestCommand\thepdfnumber
%    \end{macrocode}
%    \end{macro}
%
%    \begin{macrocode}
\Test{0}{0}
\Test{1}{1}
\Test{23}{23}
\Test{0045}{45}
\Test{4500}{4500}
\Test{-0}{0}
\Test{-1}{-1}
\Test{-00002}{-2}
\Test{-010203}{-10203}
\Test{+0}{0}
\Test{+123}{123}

\Test{+-4}{-4}
\Test{+-+-4}{4}
\Test{++--++--++4}{4}
\Test{+++---+++---+++---4}{-4}

\Test{.}{0}
\Test{0.}{0}
\Test{.0}{0}
\Test{0.}{0}
\Test{-.}{0}
\Test{-0.}{0}
\Test{-.0}{0}
\Test{-0.}{0}
\Test{010.020}{10.02}
\Test{123.456}{123.456}
\Test{12.}{12}
\Test{.123}{.123}
\Test{.0001}{.0001}
\Test{.00100}{.001}
\Test{.12003400560078009}{.12003400560078009}
\Test{-.12003400560078009}{-.12003400560078009}
\Test{04.0500000}{4.05}

\Test{}{0\thepdfnumber@ErrorUnexpectedEnd}
\Test{+}{0\thepdfnumber@ErrorUnexpectedEnd}
\Test{-}{0\thepdfnumber@ErrorUnexpectedEnd}
\Test{a}{0\thepdfnumber@ErrorInvalidToken}
\Test{0x1}{0\thepdfnumber@ErrorInvalidToken}
\Test{4x56}{4\thepdfnumber@ErrorInvalidToken}
\Test{012,34}{12\thepdfnumber@ErrorInvalidToken}
\Test{0.12x4}{.12\thepdfnumber@ErrorInvalidToken}
%    \end{macrocode}
%
% \subsubsection{Test cases for \cs{thepdfnumberNormZeroOne}}
%
%    \begin{macro}{\TestCommand}
%    \begin{macrocode}
\let\TestCommand\thepdfnumberNormZeroOne
%    \end{macrocode}
%    \end{macro}
%
%    \begin{macrocode}
\Test{0}{0}
\Test{1}{1}
\Test{23}{1}
\Test{12}{1}
\Test{10}{1}
\Test{0045}{1}
\Test{001}{1}
\Test{-0}{0}
\Test{-1}{0}
\Test{-01}{0}
\Test{-4}{0}

\Test{+-0}{0}
\Test{+-+-1}{1}
\Test{++--++--++1}{1}
\Test{+++---+++---+++---1}{0}

\Test{.}{0}
\Test{0.}{0}
\Test{.0}{0}
\Test{0.}{0}
\Test{-.}{0}
\Test{-0.}{0}
\Test{-.0}{0}
\Test{-0.}{0}
\Test{010.020}{1}
\Test{123.456}{1}
\Test{12.}{1}
\Test{.123}{.123}
\Test{.0001}{.0001}
\Test{.00100}{.001}
\Test{.12003400560078009}{.12003400560078009}
\Test{-.12003400560078009}{0}
\Test{04.0500000}{1}
\Test{0.1200340056}{.1200340056}
\Test{1.05}{1}

\Test{}{0\thepdfnumber@ErrorUnexpectedEnd}
\Test{+}{0\thepdfnumber@ErrorUnexpectedEnd}
\Test{-}{0\thepdfnumber@ErrorUnexpectedEnd}
\Test{a}{0\thepdfnumber@ErrorInvalidToken}
\Test{0x1}{0\thepdfnumber@ErrorInvalidToken}
\Test{4x56}{1}
\Test{012,34}{1}
\Test{.012x4}{.012\thepdfnumber@ErrorInvalidToken}
%    \end{macrocode}
%
%    \begin{macrocode}
\csname @@end\endcsname\end
%</test2>
%    \end{macrocode}
%
% \section{Installation}
%
% \subsection{Download}
%
% \paragraph{Package.} This package is available on
% CTAN\footnote{\url{http://ctan.org/pkg/thepdfnumber}}:
% \begin{description}
% \item[\CTAN{macros/latex/contrib/oberdiek/thepdfnumber.dtx}] The source file.
% \item[\CTAN{macros/latex/contrib/oberdiek/thepdfnumber.pdf}] Documentation.
% \end{description}
%
%
% \paragraph{Bundle.} All the packages of the bundle `oberdiek'
% are also available in a TDS compliant ZIP archive. There
% the packages are already unpacked and the documentation files
% are generated. The files and directories obey the TDS standard.
% \begin{description}
% \item[\CTAN{install/macros/latex/contrib/oberdiek.tds.zip}]
% \end{description}
% \emph{TDS} refers to the standard ``A Directory Structure
% for \TeX\ Files'' (\CTAN{tds/tds.pdf}). Directories
% with \xfile{texmf} in their name are usually organized this way.
%
% \subsection{Bundle installation}
%
% \paragraph{Unpacking.} Unpack the \xfile{oberdiek.tds.zip} in the
% TDS tree (also known as \xfile{texmf} tree) of your choice.
% Example (linux):
% \begin{quote}
%   |unzip oberdiek.tds.zip -d ~/texmf|
% \end{quote}
%
% \paragraph{Script installation.}
% Check the directory \xfile{TDS:scripts/oberdiek/} for
% scripts that need further installation steps.
% Package \xpackage{attachfile2} comes with the Perl script
% \xfile{pdfatfi.pl} that should be installed in such a way
% that it can be called as \texttt{pdfatfi}.
% Example (linux):
% \begin{quote}
%   |chmod +x scripts/oberdiek/pdfatfi.pl|\\
%   |cp scripts/oberdiek/pdfatfi.pl /usr/local/bin/|
% \end{quote}
%
% \subsection{Package installation}
%
% \paragraph{Unpacking.} The \xfile{.dtx} file is a self-extracting
% \docstrip\ archive. The files are extracted by running the
% \xfile{.dtx} through \plainTeX:
% \begin{quote}
%   \verb|tex thepdfnumber.dtx|
% \end{quote}
%
% \paragraph{TDS.} Now the different files must be moved into
% the different directories in your installation TDS tree
% (also known as \xfile{texmf} tree):
% \begin{quote}
% \def\t{^^A
% \begin{tabular}{@{}>{\ttfamily}l@{ $\rightarrow$ }>{\ttfamily}l@{}}
%   thepdfnumber.sty & tex/generic/oberdiek/thepdfnumber.sty\\
%   thepdfnumber.pdf & doc/latex/oberdiek/thepdfnumber.pdf\\
%   test/thepdfnumber-test1.tex & doc/latex/oberdiek/test/thepdfnumber-test1.tex\\
%   test/thepdfnumber-test2.tex & doc/latex/oberdiek/test/thepdfnumber-test2.tex\\
%   test/thepdfnumber-test3.tex & doc/latex/oberdiek/test/thepdfnumber-test3.tex\\
%   thepdfnumber.dtx & source/latex/oberdiek/thepdfnumber.dtx\\
% \end{tabular}^^A
% }^^A
% \sbox0{\t}^^A
% \ifdim\wd0>\linewidth
%   \begingroup
%     \advance\linewidth by\leftmargin
%     \advance\linewidth by\rightmargin
%   \edef\x{\endgroup
%     \def\noexpand\lw{\the\linewidth}^^A
%   }\x
%   \def\lwbox{^^A
%     \leavevmode
%     \hbox to \linewidth{^^A
%       \kern-\leftmargin\relax
%       \hss
%       \usebox0
%       \hss
%       \kern-\rightmargin\relax
%     }^^A
%   }^^A
%   \ifdim\wd0>\lw
%     \sbox0{\small\t}^^A
%     \ifdim\wd0>\linewidth
%       \ifdim\wd0>\lw
%         \sbox0{\footnotesize\t}^^A
%         \ifdim\wd0>\linewidth
%           \ifdim\wd0>\lw
%             \sbox0{\scriptsize\t}^^A
%             \ifdim\wd0>\linewidth
%               \ifdim\wd0>\lw
%                 \sbox0{\tiny\t}^^A
%                 \ifdim\wd0>\linewidth
%                   \lwbox
%                 \else
%                   \usebox0
%                 \fi
%               \else
%                 \lwbox
%               \fi
%             \else
%               \usebox0
%             \fi
%           \else
%             \lwbox
%           \fi
%         \else
%           \usebox0
%         \fi
%       \else
%         \lwbox
%       \fi
%     \else
%       \usebox0
%     \fi
%   \else
%     \lwbox
%   \fi
% \else
%   \usebox0
% \fi
% \end{quote}
% If you have a \xfile{docstrip.cfg} that configures and enables \docstrip's
% TDS installing feature, then some files can already be in the right
% place, see the documentation of \docstrip.
%
% \subsection{Refresh file name databases}
%
% If your \TeX~distribution
% (\teTeX, \mikTeX, \dots) relies on file name databases, you must refresh
% these. For example, \teTeX\ users run \verb|texhash| or
% \verb|mktexlsr|.
%
% \subsection{Some details for the interested}
%
% \paragraph{Attached source.}
%
% The PDF documentation on CTAN also includes the
% \xfile{.dtx} source file. It can be extracted by
% AcrobatReader 6 or higher. Another option is \textsf{pdftk},
% e.g. unpack the file into the current directory:
% \begin{quote}
%   \verb|pdftk thepdfnumber.pdf unpack_files output .|
% \end{quote}
%
% \paragraph{Unpacking with \LaTeX.}
% The \xfile{.dtx} chooses its action depending on the format:
% \begin{description}
% \item[\plainTeX:] Run \docstrip\ and extract the files.
% \item[\LaTeX:] Generate the documentation.
% \end{description}
% If you insist on using \LaTeX\ for \docstrip\ (really,
% \docstrip\ does not need \LaTeX), then inform the autodetect routine
% about your intention:
% \begin{quote}
%   \verb|latex \let\install=y% \iffalse meta-comment
%
% File: thepdfnumber.dtx
% Version: 2016/05/16 v1.1
% Info: Print PDF numbers with minimal digits
%
% Copyright (C) 2011 by
%    Heiko Oberdiek <heiko.oberdiek at googlemail.com>
%    2016
%    https://github.com/ho-tex/oberdiek/issues
%
% This work may be distributed and/or modified under the
% conditions of the LaTeX Project Public License, either
% version 1.3c of this license or (at your option) any later
% version. This version of this license is in
%    http://www.latex-project.org/lppl/lppl-1-3c.txt
% and the latest version of this license is in
%    http://www.latex-project.org/lppl.txt
% and version 1.3 or later is part of all distributions of
% LaTeX version 2005/12/01 or later.
%
% This work has the LPPL maintenance status "maintained".
%
% This Current Maintainer of this work is Heiko Oberdiek.
%
% The Base Interpreter refers to any `TeX-Format',
% because some files are installed in TDS:tex/generic//.
%
% This work consists of the main source file thepdfnumber.dtx
% and the derived files
%    thepdfnumber.sty, thepdfnumber.pdf, thepdfnumber.ins, thepdfnumber.drv,
%    thepdfnumber-test1.tex, thepdfnumber-test2.tex,
%    thepdfnumber-test3.tex.
%
% Distribution:
%    CTAN:macros/latex/contrib/oberdiek/thepdfnumber.dtx
%    CTAN:macros/latex/contrib/oberdiek/thepdfnumber.pdf
%
% Unpacking:
%    (a) If thepdfnumber.ins is present:
%           tex thepdfnumber.ins
%    (b) Without thepdfnumber.ins:
%           tex thepdfnumber.dtx
%    (c) If you insist on using LaTeX
%           latex \let\install=y% \iffalse meta-comment
%
% File: thepdfnumber.dtx
% Version: 2016/05/16 v1.1
% Info: Print PDF numbers with minimal digits
%
% Copyright (C) 2011 by
%    Heiko Oberdiek <heiko.oberdiek at googlemail.com>
%    2016
%    https://github.com/ho-tex/oberdiek/issues
%
% This work may be distributed and/or modified under the
% conditions of the LaTeX Project Public License, either
% version 1.3c of this license or (at your option) any later
% version. This version of this license is in
%    http://www.latex-project.org/lppl/lppl-1-3c.txt
% and the latest version of this license is in
%    http://www.latex-project.org/lppl.txt
% and version 1.3 or later is part of all distributions of
% LaTeX version 2005/12/01 or later.
%
% This work has the LPPL maintenance status "maintained".
%
% This Current Maintainer of this work is Heiko Oberdiek.
%
% The Base Interpreter refers to any `TeX-Format',
% because some files are installed in TDS:tex/generic//.
%
% This work consists of the main source file thepdfnumber.dtx
% and the derived files
%    thepdfnumber.sty, thepdfnumber.pdf, thepdfnumber.ins, thepdfnumber.drv,
%    thepdfnumber-test1.tex, thepdfnumber-test2.tex,
%    thepdfnumber-test3.tex.
%
% Distribution:
%    CTAN:macros/latex/contrib/oberdiek/thepdfnumber.dtx
%    CTAN:macros/latex/contrib/oberdiek/thepdfnumber.pdf
%
% Unpacking:
%    (a) If thepdfnumber.ins is present:
%           tex thepdfnumber.ins
%    (b) Without thepdfnumber.ins:
%           tex thepdfnumber.dtx
%    (c) If you insist on using LaTeX
%           latex \let\install=y% \iffalse meta-comment
%
% File: thepdfnumber.dtx
% Version: 2016/05/16 v1.1
% Info: Print PDF numbers with minimal digits
%
% Copyright (C) 2011 by
%    Heiko Oberdiek <heiko.oberdiek at googlemail.com>
%    2016
%    https://github.com/ho-tex/oberdiek/issues
%
% This work may be distributed and/or modified under the
% conditions of the LaTeX Project Public License, either
% version 1.3c of this license or (at your option) any later
% version. This version of this license is in
%    http://www.latex-project.org/lppl/lppl-1-3c.txt
% and the latest version of this license is in
%    http://www.latex-project.org/lppl.txt
% and version 1.3 or later is part of all distributions of
% LaTeX version 2005/12/01 or later.
%
% This work has the LPPL maintenance status "maintained".
%
% This Current Maintainer of this work is Heiko Oberdiek.
%
% The Base Interpreter refers to any `TeX-Format',
% because some files are installed in TDS:tex/generic//.
%
% This work consists of the main source file thepdfnumber.dtx
% and the derived files
%    thepdfnumber.sty, thepdfnumber.pdf, thepdfnumber.ins, thepdfnumber.drv,
%    thepdfnumber-test1.tex, thepdfnumber-test2.tex,
%    thepdfnumber-test3.tex.
%
% Distribution:
%    CTAN:macros/latex/contrib/oberdiek/thepdfnumber.dtx
%    CTAN:macros/latex/contrib/oberdiek/thepdfnumber.pdf
%
% Unpacking:
%    (a) If thepdfnumber.ins is present:
%           tex thepdfnumber.ins
%    (b) Without thepdfnumber.ins:
%           tex thepdfnumber.dtx
%    (c) If you insist on using LaTeX
%           latex \let\install=y\input{thepdfnumber.dtx}
%        (quote the arguments according to the demands of your shell)
%
% Documentation:
%    (a) If thepdfnumber.drv is present:
%           latex thepdfnumber.drv
%    (b) Without thepdfnumber.drv:
%           latex thepdfnumber.dtx; ...
%    The class ltxdoc loads the configuration file ltxdoc.cfg
%    if available. Here you can specify further options, e.g.
%    use A4 as paper format:
%       \PassOptionsToClass{a4paper}{article}
%
%    Programm calls to get the documentation (example):
%       pdflatex thepdfnumber.dtx
%       makeindex -s gind.ist thepdfnumber.idx
%       pdflatex thepdfnumber.dtx
%       makeindex -s gind.ist thepdfnumber.idx
%       pdflatex thepdfnumber.dtx
%
% Installation:
%    TDS:tex/generic/oberdiek/thepdfnumber.sty
%    TDS:doc/latex/oberdiek/thepdfnumber.pdf
%    TDS:doc/latex/oberdiek/test/thepdfnumber-test1.tex
%    TDS:doc/latex/oberdiek/test/thepdfnumber-test2.tex
%    TDS:doc/latex/oberdiek/test/thepdfnumber-test3.tex
%    TDS:source/latex/oberdiek/thepdfnumber.dtx
%
%<*ignore>
\begingroup
  \catcode123=1 %
  \catcode125=2 %
  \def\x{LaTeX2e}%
\expandafter\endgroup
\ifcase 0\ifx\install y1\fi\expandafter
         \ifx\csname processbatchFile\endcsname\relax\else1\fi
         \ifx\fmtname\x\else 1\fi\relax
\else\csname fi\endcsname
%</ignore>
%<*install>
\input docstrip.tex
\Msg{************************************************************************}
\Msg{* Installation}
\Msg{* Package: thepdfnumber 2016/05/16 v1.1 Print PDF numbers with minimal digits (HO)}
\Msg{************************************************************************}

\keepsilent
\askforoverwritefalse

\let\MetaPrefix\relax
\preamble

This is a generated file.

Project: thepdfnumber
Version: 2016/05/16 v1.1

Copyright (C) 2011 by
   Heiko Oberdiek <heiko.oberdiek at googlemail.com>

This work may be distributed and/or modified under the
conditions of the LaTeX Project Public License, either
version 1.3c of this license or (at your option) any later
version. This version of this license is in
   http://www.latex-project.org/lppl/lppl-1-3c.txt
and the latest version of this license is in
   http://www.latex-project.org/lppl.txt
and version 1.3 or later is part of all distributions of
LaTeX version 2005/12/01 or later.

This work has the LPPL maintenance status "maintained".

This Current Maintainer of this work is Heiko Oberdiek.

The Base Interpreter refers to any `TeX-Format',
because some files are installed in TDS:tex/generic//.

This work consists of the main source file thepdfnumber.dtx
and the derived files
   thepdfnumber.sty, thepdfnumber.pdf, thepdfnumber.ins, thepdfnumber.drv,
   thepdfnumber-test1.tex, thepdfnumber-test2.tex,
   thepdfnumber-test3.tex.

\endpreamble
\let\MetaPrefix\DoubleperCent

\generate{%
  \file{thepdfnumber.ins}{\from{thepdfnumber.dtx}{install}}%
  \file{thepdfnumber.drv}{\from{thepdfnumber.dtx}{driver}}%
  \usedir{tex/generic/oberdiek}%
  \file{thepdfnumber.sty}{\from{thepdfnumber.dtx}{package}}%
  \usedir{doc/latex/oberdiek/test}%
  \file{thepdfnumber-test1.tex}{\from{thepdfnumber.dtx}{test1}}%
  \file{thepdfnumber-test2.tex}{\from{thepdfnumber.dtx}{test2}}%
  \file{thepdfnumber-test3.tex}{\from{thepdfnumber.dtx}{test2,active}}%
}

\catcode32=13\relax% active space
\let =\space%
\Msg{************************************************************************}
\Msg{*}
\Msg{* To finish the installation you have to move the following}
\Msg{* file into a directory searched by TeX:}
\Msg{*}
\Msg{*     thepdfnumber.sty}
\Msg{*}
\Msg{* To produce the documentation run the file `thepdfnumber.drv'}
\Msg{* through LaTeX.}
\Msg{*}
\Msg{* Happy TeXing!}
\Msg{*}
\Msg{************************************************************************}

\endbatchfile
%</install>
%<*ignore>
\fi
%</ignore>
%<*driver>
\NeedsTeXFormat{LaTeX2e}
\ProvidesFile{thepdfnumber.drv}%
  [2016/05/16 v1.1 Print PDF numbers with minimal digits (HO)]%
\documentclass{ltxdoc}
\usepackage{holtxdoc}[2011/11/22]
\usepackage{csquotes}
\usepackage[
  backend=bibtex,
  bibencoding=ascii,
  alldates=iso8601,
]{biblatex}[2011/11/13]
\bibliography{oberdiek-source}
\begin{document}
  \DocInput{thepdfnumber.dtx}%
\end{document}
%</driver>
% \fi
%
%
% \CharacterTable
%  {Upper-case    \A\B\C\D\E\F\G\H\I\J\K\L\M\N\O\P\Q\R\S\T\U\V\W\X\Y\Z
%   Lower-case    \a\b\c\d\e\f\g\h\i\j\k\l\m\n\o\p\q\r\s\t\u\v\w\x\y\z
%   Digits        \0\1\2\3\4\5\6\7\8\9
%   Exclamation   \!     Double quote  \"     Hash (number) \#
%   Dollar        \$     Percent       \%     Ampersand     \&
%   Acute accent  \'     Left paren    \(     Right paren   \)
%   Asterisk      \*     Plus          \+     Comma         \,
%   Minus         \-     Point         \.     Solidus       \/
%   Colon         \:     Semicolon     \;     Less than     \<
%   Equals        \=     Greater than  \>     Question mark \?
%   Commercial at \@     Left bracket  \[     Backslash     \\
%   Right bracket \]     Circumflex    \^     Underscore    \_
%   Grave accent  \`     Left brace    \{     Vertical bar  \|
%   Right brace   \}     Tilde         \~}
%
% \GetFileInfo{thepdfnumber.drv}
%
% \title{The \xpackage{thepdfnumber} package}
% \date{2016/05/16 v1.1}
% \author{Heiko Oberdiek\thanks
% {Please report any issues at https://github.com/ho-tex/oberdiek/issues}\\
% \xemail{heiko.oberdiek at googlemail.com}}
%
% \maketitle
%
% \begin{abstract}
% The package converts real numbers to a minimal representation
% that is stripped from leading or trailing zeros,
% plus signs and decimal point if not necessary.
% \end{abstract}
%
% \tableofcontents
%
% \section{Documentation}
%
% \subsection{Introduction}
%
% Dealing with the PDF format, there is sometimes the need to write
% some low level PDF stuff. In case of numbers, the numbers can
% arise from user input (e.g. color or transparency specifications)
% or can be calculated. For example, \hologo{LaTeX}'s \cs{strip@pt}
% makes a good job to output a real number. It automatically suppresses
% the decimal part if the number is an integer. However it leaves
% a leading zero for numbers greater zero and smaller one.
% Thus the package provides macros that can be used with different
% formats, even with \hologo{iniTeX} and generates numbers that
% are valid numbers of the PDF format and whose length is minimal.
%
% \subsection{Usage}
%
% The package \xpackage{thepdfnumber} can be used with \hologo{LaTeX},
% \hologo{plainTeX} or even with \hologo{iniTeX}:
% \begin{quote}
%   |\RequirePackage{thepdfnumber}| \textit{\% \hologo{LaTeX}}\\
%   |\input thepdfnumber.sty| \textit{\% \hologo{plainTeX}/\hologo{iniTeX}}
% \end{quote}
% The package does not need and have package options.
%
% \subsection{User macros}
%
% All user macros are expandable in exact two expansion steps.
%
% \begin{declcs}{thepdfnumber} \M{number}
% \end{declcs}
% Macro \cs{thepdfnumber} takes a number
% as argument and expands to a minimal representation of that number.
% Some examples:
% \begin{quote}
% \begin{tabular}{r@{.: }>{\ttfamily}l@{ $\rightarrow$\ }>{\ttfamily}l@{}}
%   1& +123 & 123\\
%   2& --123 & 123\\
%   3& -01 & -1\\
%   4& 0045 & 45\\
%   5& 1.0 & 1\\
%   6& 1.20 & 1.2\\
%   7& 0.0 & 0\\
%   8& 0.78 & .78\\
%   9& +012.340 & 12.34\\
% \end{tabular}
% \end{quote}
% It reduces the length of the number representation:
% \begin{itemize}
% \item The signs are collapsed and only one minus sign
%   is output if the number is negative (see examples 1, 2, 3, 9).
% \item Leading zeros are removed (4, 8, 9) unless the number is zero (7).
% \item The decimal part is omitted, if the number is an integer (5, 7).
% \item Trailing zeros from the decimal part are stripped (5, 6, 7, 9).
% \end{itemize}
% The resulting number representation can be catched with one of the
% following Perl regular expressions:
% \begin{itemize}
% \item |^0$| (zero)
% \item |^-?[1-9][0-9]*$| (integer)
% \item |^-?[0-9]*\.[0-9]*[1-9]$| (real)
% \end{itemize}
% This is a valid numeric object of the PDF specification \cite[``7.3.3~Numeric Objects'']{pdfspec-iso32000-1}.
%
% \begin{declcs}{thepdfnumberNormZeroOne}
% \end{declcs}
% There are various places in the PDF specification where the
% number is in the domain 0.0 upto 1.0. Macro thepdfnumberNormZeroOne
% automatically adjusts the number to fit into that range. Negative
% numbers are mapped to |0| and numbers greater than one are
% replaced by |1|. Thus the result fits one of the following regular
% expressions:
% \begin{itemize}
% \item |^0$|
% \item |^\.[0-9]*[1-9]$|
% \item |^1$|
% \end{itemize}
% Examples:
% \begin{quote}
% \begin{tabular}{@{}>{\ttfamily}l@{ $\rightarrow$\ }>{\ttfamily}l@{}}
%   -456 & 0\\
%   -0.001 & 0\\
%   0.0 & 0\\
%   0.010 & .01\\
%   0.456 & .456\\
%   1.0 & 1\\
%   01.001 & 1\\
%   4 & 1\\
% \end{tabular}
% \end{quote}
%
% \subsection{Input number}
%
% The user macros expect a number as argument. The number can either
% be given explicitely or as macro that expands in one step
% to an explicite number, because the first token of the argument
% is expanded once.
%
% The explicite number consists of
% \begin{itemize}
% \item optional signs `\texttt{+}' and `\texttt{-}',
% \item digits `\texttt{0}' upto `\texttt{9}' and
% \item an optional dot `\texttt{.}'.
% \end{itemize}
% All tokens must have catcode 12 (other), the default catcodes
% for these characters in \hologo{LaTeX}, \hologo{plainTeX} or
% \hologo{iniTeX}. As Perl regular expression the number is expected
% in one of the following forms:
% \begin{itemize}
% \item |^[+-]*[0-9]+$|
% \item |^[+-]*[0-9]*\.[0-9]*$|
% \end{itemize}
% At least one digit or the dot must be present.
%
% \subsection{Error handling}
%
% The package is not intended for validating numbers or to
% decide if an argument is a number. Therefore it is an usage
% error to use the user macros with arguments that are not
% explicite numbers as specified in the previous sections.
% Nevertheless some error conditions are sometimes recognized.
% Errors are given in form of an undefined command sequence.
% It is the only way to notify \hologo{TeX} in expandable context.
% Expanding to some error text would invalidate the output.
% Currently the following errors are thrown:
% \begin{description}
% \item[\cs{thepdfnumber@ErrorEndMarker}:]
%   Internally the argument parsing uses an end marker that
%   is never called directly. If it is called with valid
%   user input, then this is a bug. Otherwise it means the user input
%   contains nasty stuff.
% \item[\cs{thepdfnumber@ErrorUnexpectedEnd}:] The macros expect at least one digit
%   or the dot, otherwise if the argument is empty or only contains
%   signs, then this error is called.
% \item[\cs{thepdfnumber@ErrorInvalidToken}:] It is called if the number
%   contains other tokens than signs, digits or the dot or the
%   token at the wrong place (e.g. a sign after a digit).
%   In case of \cs{thepdfnumberNormZeroOne} this error condition might
%   not always be detected, because the number parsing might stop
%   at an early point, when the result is already clear (e.g.
%   if the number is negative or will be greater than one).
% \item[Improper alphabetic constant:] This error might be
%   thrown by \hologo{TeX}, if the number contains
%   command tokens instead of characters.
% \end{description}
%
% \StopEventually{
% }
%
% \section{Implementation}
%
%    \begin{macrocode}
%<*package>
%    \end{macrocode}
%
% \subsection{Reload check and package identification}
%    Reload check, especially if the package is not used with \LaTeX.
%    \begin{macrocode}
\begingroup\catcode61\catcode48\catcode32=10\relax%
  \catcode13=5 % ^^M
  \endlinechar=13 %
  \catcode35=6 % #
  \catcode39=12 % '
  \catcode44=12 % ,
  \catcode45=12 % -
  \catcode46=12 % .
  \catcode58=12 % :
  \catcode64=11 % @
  \catcode123=1 % {
  \catcode125=2 % }
  \expandafter\let\expandafter\x\csname ver@thepdfnumber.sty\endcsname
  \ifx\x\relax % plain-TeX, first loading
  \else
    \def\empty{}%
    \ifx\x\empty % LaTeX, first loading,
      % variable is initialized, but \ProvidesPackage not yet seen
    \else
      \expandafter\ifx\csname PackageInfo\endcsname\relax
        \def\x#1#2{%
          \immediate\write-1{Package #1 Info: #2.}%
        }%
      \else
        \def\x#1#2{\PackageInfo{#1}{#2, stopped}}%
      \fi
      \x{thepdfnumber}{The package is already loaded}%
      \aftergroup\endinput
    \fi
  \fi
\endgroup%
%    \end{macrocode}
%    Package identification:
%    \begin{macrocode}
\begingroup\catcode61\catcode48\catcode32=10\relax%
  \catcode13=5 % ^^M
  \endlinechar=13 %
  \catcode35=6 % #
  \catcode39=12 % '
  \catcode40=12 % (
  \catcode41=12 % )
  \catcode44=12 % ,
  \catcode45=12 % -
  \catcode46=12 % .
  \catcode47=12 % /
  \catcode58=12 % :
  \catcode64=11 % @
  \catcode91=12 % [
  \catcode93=12 % ]
  \catcode123=1 % {
  \catcode125=2 % }
  \expandafter\ifx\csname ProvidesPackage\endcsname\relax
    \def\x#1#2#3[#4]{\endgroup
      \immediate\write-1{Package: #3 #4}%
      \xdef#1{#4}%
    }%
  \else
    \def\x#1#2[#3]{\endgroup
      #2[{#3}]%
      \ifx#1\@undefined
        \xdef#1{#3}%
      \fi
      \ifx#1\relax
        \xdef#1{#3}%
      \fi
    }%
  \fi
\expandafter\x\csname ver@thepdfnumber.sty\endcsname
\ProvidesPackage{thepdfnumber}%
  [2016/05/16 v1.1 Print PDF numbers with minimal digits (HO)]%
%    \end{macrocode}
%
% \subsection{Catcodes}
%
%    \begin{macrocode}
\begingroup\catcode61\catcode48\catcode32=10\relax%
  \catcode13=5 % ^^M
  \endlinechar=13 %
  \catcode123=1 % {
  \catcode125=2 % }
  \catcode64=11 % @
  \def\x{\endgroup
    \expandafter\edef\csname ThPdNu@AtEnd\endcsname{%
      \endlinechar=\the\endlinechar\relax
      \catcode13=\the\catcode13\relax
      \catcode32=\the\catcode32\relax
      \catcode35=\the\catcode35\relax
      \catcode61=\the\catcode61\relax
      \catcode64=\the\catcode64\relax
      \catcode123=\the\catcode123\relax
      \catcode125=\the\catcode125\relax
    }%
  }%
\x\catcode61\catcode48\catcode32=10\relax%
\catcode13=5 % ^^M
\endlinechar=13 %
\catcode35=6 % #
\catcode64=11 % @
\catcode123=1 % {
\catcode125=2 % }
\def\TMP@EnsureCode#1#2{%
  \edef\ThPdNu@AtEnd{%
    \ThPdNu@AtEnd
    \catcode#1=\the\catcode#1\relax
  }%
  \catcode#1=#2\relax
}
\TMP@EnsureCode{33}{12}% !
\TMP@EnsureCode{36}{3}% $
\TMP@EnsureCode{38}{4}% &
\TMP@EnsureCode{42}{12}% *
\TMP@EnsureCode{43}{12}% +
\TMP@EnsureCode{45}{12}% -
\TMP@EnsureCode{46}{12}% .
\TMP@EnsureCode{60}{12}% <
\TMP@EnsureCode{62}{12}% >
\TMP@EnsureCode{96}{12}% `
\edef\ThPdNu@AtEnd{\ThPdNu@AtEnd\noexpand\endinput}
%    \end{macrocode}
%
% \subsection{Helper macros}
%
%    \begin{macro}{\ThPdNu@FIN}
%    \begin{macrocode}
\def\ThPdNu@FIN{\thepdfnumber@ErrorEndMarker}
%    \end{macrocode}
%    \end{macro}
%
%    \begin{macro}{\ThPdNu@space}
%    \begin{macrocode}
\def\ThPdNu@space{ }
%    \end{macrocode}
%    \end{macro}
%
%    \begin{macro}{\ThPdNu@zero}
%    \begin{macrocode}
\chardef\ThPdNu@zero=0 %
%    \end{macrocode}
%    \end{macro}
%
%    \begin{macro}{\ThPdNu@one}
%    \begin{macrocode}
\chardef\ThPdNu@one=1 %
%    \end{macrocode}
%    \end{macro}
%
%    \begin{macro}{\ThPdNu@firstoftwo}
%    \begin{macrocode}
\long\def\ThPdNu@firstoftwo#1#2{#1}
%    \end{macrocode}
%    \end{macro}
%
%    \begin{macro}{\ThPdNu@secondoftwo}
%    \begin{macrocode}
\long\def\ThPdNu@secondoftwo#1#2{#2}
%    \end{macrocode}
%    \end{macro}
%
% \subsection{Detect \hologo{eTeX}}
%
%    \begin{macrocode}
\begingroup\expandafter\expandafter\expandafter\endgroup
\expandafter\ifx\csname detokenize\endcsname\relax
  \catcode`\&=14 %
  \catcode`\$=9 %
\else
  \catcode`\&=9 %
  \catcode`\$=14 %
\fi
%    \end{macrocode}
%
% \subsection{User macro \cs{thepdfnumber}}
%
%    \begin{macro}{\thepdfnumber}
%    \begin{macrocode}
\def\thepdfnumber#1{%
  \romannumeral
& \iftrue\expandafter\ThPdNu@State@Plus\expandafter\fi
& \detokenize\expandafter{#1}%
& \ThPdNu@FIN
$ \ifx\ThPdNu@FIN#1\ThPdNu@FIN
$   \expandafter\ThPdNu@firstoftwo
$ \else
$   \expandafter\ThPdNu@secondoftwo
$ \fi
$ {%
$   \ThPdNu@zero
$   0\thepdfnumber@ErrorUnexpectedEnd
$ }{%
$   \iftrue\expandafter\ThPdNu@State@Plus\expandafter\fi#1\ThPdNu@FIN
$ }%
}
%    \end{macrocode}
%    \end{macro}
%
% \subsubsection{State definitions for sign}
%
%    \begin{macro}{\ThPdNu@State@Plus}
%    \begin{macrocode}
\def\ThPdNu@State@Plus#1\fi#2{%
  \fi
  \ifcase\ifx\ThPdNu@FIN#2%
           0%
&        \else\ifx-#2%
$        \else\ifnum`#2=45 % -
           1%
         \else\ifx0#2%
           2%
         \else\ifnum`#2>48 %
           \ifnum`#2<58 %
             3%
           \else
             9%
           \fi
&        \else\ifx.#2%
$        \else\ifnum`#2=46 % .
           4%
&        \else\ifx+#2%
$        \else\ifnum`#2=43 % +
           5%
         \else
           9%
         \fi\fi\fi\fi\fi\fi\ThPdNu@space
    \expandafter\ThPdNu@zero
    \expandafter0%
    \expandafter\thepdfnumber@ErrorUnexpectedEnd
  \or
    \ThPdNu@State@Minus
  \or
    \ThPdNu@State@SkipZeros!%
  \or
    \ThPdNu@State@Int!#2!%
  \or
    \ThPdNu@State@Dot!\ThPdNu@zero*\ThPdNu@zero!!%
  \or
    \ThPdNu@State@Plus
  \else
    \ThPdNu@ReturnError{0}%
  \fi
}
%    \end{macrocode}
%    \end{macro}
%
%    \begin{macro}{\ThPdNu@State@Minus}
%    \begin{macrocode}
\def\ThPdNu@State@Minus#1\fi#2{%
  \fi
  \ifcase\ifx\ThPdNu@FIN#2%
           0%
         \else\ifx0#2%
           1%
         \else\ifnum`#2>48 %
           \ifnum`#2<58 %
             2%
           \else
             9%
           \fi
&        \else\ifx.#2%
$        \else\ifnum`#2=46 % .
           3%
&        \else\ifx-#2%
$        \else\ifnum`#2=45 % -
           4%
&        \else\ifx+#2%
$        \else\ifnum`#2=43 % +
           5%
         \else
           9%
         \fi\fi\fi\fi\fi\fi\ThPdNu@space
    \expandafter\ThPdNu@zero
    \expandafter0%
    \expandafter\thepdfnumber@ErrorUnexpectedEnd
  \or
    \ThPdNu@State@SkipZeros-!%
  \or
    \ThPdNu@State@Int-!#2!%
  \or
    \ThPdNu@State@Dot-!\ThPdNu@zero*\ThPdNu@zero!!%
  \or
    \ThPdNu@State@Plus
  \or
    \ThPdNu@State@Minus
  \else
    \ThPdNu@ReturnError{0}%
  \fi
}
%    \end{macrocode}
%    \end{macro}
%
%    \begin{macro}{\ThPdNu@ReturnError}
%    \begin{macrocode}
\def\ThPdNu@ReturnError#1#2\fi#3\ThPdNu@FIN{%
  \fi
  \ThPdNu@zero
  #1%
  \thepdfnumber@ErrorInvalidToken
}
%    \end{macrocode}
%    \end{macro}
%
% \subsubsection{State definitions for integer part}
%
%    \begin{macro}{\ThPdNu@State@SkipZeros}
%    \begin{macrocode}
\def\ThPdNu@State@SkipZeros#1!#2\fi#3{%
  \fi
  \ifcase\ifx\ThPdNu@FIN#3%
           0%
         \else\ifx0#3%
           1%
         \else\ifnum`#3>48 %
           \ifnum`#3<58 %
             2%
           \else
             9%
           \fi
&        \else\ifx.#3%
$        \else\ifnum`#3=46 % .
           3%
         \else
           9%
         \fi\fi\fi\fi\ThPdNu@space
    \expandafter\ThPdNu@zero
    \expandafter0%
  \or
   \ThPdNu@State@SkipZeros#1!%
  \or
    \ThPdNu@State@Int#1!#3!%
  \or
    \ThPdNu@State@Dot#1!\ThPdNu@zero*\ThPdNu@zero!!%
  \else
    \ThPdNu@ReturnError{0}%
  \fi
}
%    \end{macrocode}
%    \end{macro}
%
%    \begin{macro}{\ThPdNu@State@Int}
%    \begin{macrocode}
\def\ThPdNu@State@Int#1!#2!#3\fi#4{%
  \fi
  \ifcase\ifx\ThPdNu@FIN#4%
           0%
         \else\ifnum`#4>47 %
           \ifnum`#4<58 %
             1%
           \else
             9%
           \fi
&        \else\ifx.#4%
$        \else\ifnum`#4=46 % .
           2%
         \else
           9%
         \fi\fi\fi\ThPdNu@space
    \ThPdNu@ReturnInt{#1#2}%
  \or
    \ThPdNu@State@Int#1!#2#4!%
  \or
    \ThPdNu@State@Dot#1!\ThPdNu@one#2*\ThPdNu@zero!!%
  \else
    \ThPdNu@ReturnError{#1#2}%
  \fi
}
%    \end{macrocode}
%    \end{macro}
%
%    \begin{macro}{\ThPdNu@ReturnInt}
%    \begin{macrocode}
\def\ThPdNu@ReturnInt#1#2\fi{%
  \fi
  \ThPdNu@zero
  #1%
}
%    \end{macrocode}
%    \end{macro}
%
% \subsubsection{State definitions for decimal digits}
%
%    \begin{macro}{\ThPdNu@State@Dot}
%    \begin{macrocode}
\def\ThPdNu@State@Dot#1*#2#3!#4!#5\fi#6{%
  \fi
  \ifcase\ifx\ThPdNu@FIN#6%
           0%
         \else\ifnum`#6>48 %
           \ifnum`#6<58 %
             1%
           \else
             9%
           \fi
         \else\ifx0#6%
           2%
         \else
           9%
         \fi\fi\fi\ThPdNu@space
    \ThPdNu@ReturnNumber#1*#2#3!%
  \or
    \ThPdNu@State@Dot#1*\ThPdNu@one#3#4#6!!%
  \or
    \ThPdNu@State@DotZero#1*#2#3!#4#6!%
  \else
    \ThPdNu@ReturnNumberInvalid#1*#2#3!%
  \fi
}
%    \end{macrocode}
%    \end{macro}
%
%    \begin{macro}{\ThPdNu@State@DotZero}
%    \begin{macrocode}
\def\ThPdNu@State@DotZero#1*#2#3!#4!#5\fi#6{%
  \fi
  \ifcase\ifx\ThPdNu@FIN#6%
           0%
         \else\ifnum`#6>48 %
           \ifnum`#6<58 %
             1%
           \else
             9%
           \fi
         \else\ifx0#6%
           2%
         \else
           9%
         \fi\fi\fi\ThPdNu@space
    \ThPdNu@ReturnNumber#1*#2#3!%
  \or
    \ThPdNu@State@Dot#1*\ThPdNu@one#3#4#6!!%
  \or
    \ThPdNu@State@DotZero#1*#2#3!#4#6!%
  \else
    \ThPdNu@ReturnNumber#1*#2#3!%
  \fi
}
%    \end{macrocode}
%    \end{macro}
%
%    \begin{macro}{\ThPdNu@ReturnNumber}
%    \begin{macrocode}
\def\ThPdNu@ReturnNumber#1!#2#3*#4#5!#6\fi{%
  \fi
  \ifcase#2%
    \expandafter\ThPdNu@firstoftwo
  \else
    \expandafter\ThPdNu@secondoftwo
  \fi
  {%
    \ifcase#4%
      \expandafter\ThPdNu@firstoftwo
    \else
      \expandafter\ThPdNu@secondoftwo
    \fi
    {\ThPdNu@zero 0}%
    {\ThPdNu@zero #1.#5}%
  }{%
    \ifcase#4%
      \expandafter\ThPdNu@firstoftwo
    \else
      \expandafter\ThPdNu@secondoftwo
    \fi
    {\ThPdNu@zero #1#3}%
    {\ThPdNu@zero #1#3.#5}%
  }%
}
%    \end{macrocode}
%    \end{macro}
%
%    \begin{macro}{\ThPdNu@ReturnNumberInvalid}
%    \begin{macrocode}
\def\ThPdNu@ReturnNumberInvalid#1*#2!#3\fi#4\ThPdNu@FIN{%
  \fi
  \iftrue\ThPdNu@ReturnNumber#1*#2!\fi
  \thepdfnumber@ErrorInvalidToken
}
%    \end{macrocode}
%    \end{macro}
%
% \subsection{Norm macro}
%
%    \begin{macro}{\thepdfnumberNormZeroOne}
%    \begin{macrocode}
\def\thepdfnumberNormZeroOne#1{%
  \romannumeral
& \iftrue\expandafter\ThPbNu@StateN@Plus\expandafter\fi
& \detokenize\expandafter{#1}%
& \ThPdNu@FIN
$ \ifx\ThPdNu@FIN#1\ThPdNu@FIN
$   \expandafter\ThPdNu@firstoftwo
$ \else
$   \expandafter\ThPdNu@secondoftwo
$ \fi
$ {%
$   \ThPdNu@zero
$   0\thepdfnumber@ErrorUnexpectedEnd
$ }{%
$   \iftrue\expandafter\ThPbNu@StateN@Plus\expandafter\fi#1\ThPdNu@FIN
$ }%
}
%    \end{macrocode}
%    \end{macro}
%
% \subsubsection{State definitions for sign}
%
%    \begin{macro}{\ThPbNu@StateN@Plus}
%    \begin{macrocode}
\def\ThPbNu@StateN@Plus#1\fi#2{%
  \fi
  \ifcase\ifx\ThPdNu@FIN#2%
           0%
&        \else\ifx-#2%
$        \else\ifnum`#2=45 % -
           1%
         \else\ifx0#2%
           2%
         \else\ifnum`#2>48 %
           \ifnum`#2<58 %
             3%
           \else
             9%
           \fi
&        \else\ifx.#2%
$        \else\ifnum`#2=46 % .
           4%
&        \else\ifx+#2%
$        \else\ifnum`#2=43 % +
           5%
         \else
           9%
         \fi\fi\fi\fi\fi\fi\ThPdNu@space
    \expandafter\ThPdNu@zero
    \expandafter0%
    \expandafter\thepdfnumber@ErrorUnexpectedEnd
  \or
    \ThPbNu@StateN@Minus
  \or
    \ThPbNu@StateN@SkipZeros
  \or
    \ThPdNu@ReturnAndSkip{1}%
  \or
    \ThPbNu@StateN@Dot\ThPdNu@zero!!%
  \or
    \ThPbNu@StateN@Plus
  \else
    \ThPdNu@ReturnError{0}%
  \fi
}
%    \end{macrocode}
%    \end{macro}
%
%    \begin{macro}{\ThPbNu@StateN@Minus}
%    \begin{macrocode}
\def\ThPbNu@StateN@Minus#1\fi#2{%
  \fi
  \ifcase\ifx\ThPdNu@FIN#2%
           0%
         \else\ifnum`#2>47 %
           \ifnum`#2<58 %
             1%
           \else
             9%
           \fi
&        \else\ifx.#2%
$        \else\ifnum`#2=46 % .
           1%
&        \else\ifx-#2%
$        \else\ifnum`#2=45 % -
           2%
&        \else\ifx+#2%
$        \else\ifnum`#2=43 % +
           3%
         \else
           9%
         \fi\fi\fi\fi\fi\ThPdNu@space
    \expandafter\ThPdNu@zero
    \expandafter0%
    \expandafter\thepdfnumber@ErrorUnexpectedEnd
  \or
    \ThPdNu@ReturnAndSkip{0}%
  \or
    \ThPbNu@StateN@Plus
  \or
    \ThPbNu@StateN@Minus
  \else
    \ThPdNu@ReturnError{0}%
  \fi
}
%    \end{macrocode}
%    \end{macro}
%
%    \begin{macro}{\ThPdNu@ReturnAndSkip}
%    \begin{macrocode}
\def\ThPdNu@ReturnAndSkip#1#2\fi#3\ThPdNu@FIN{%
  \fi
  \ThPdNu@zero
  #1%
}
%    \end{macrocode}
%    \end{macro}
%
% \subsubsection{State definitions for integer part}
%
%    \begin{macro}{\ThPbNu@StateN@SkipZeros}
%    \begin{macrocode}
\def\ThPbNu@StateN@SkipZeros#1\fi#2{%
  \fi
  \ifcase\ifx\ThPdNu@FIN#2%
           0%
         \else\ifx0#2%
           1%
         \else\ifnum`#2>48 %
           \ifnum`#2<58 %
             2%
           \else
             9%
           \fi
&        \else\ifx.#2%
$        \else\ifnum`#2=46 % .
           3%
         \else
           9%
         \fi\fi\fi\fi\ThPdNu@space
    \expandafter\ThPdNu@zero
    \expandafter0%
  \or
   \ThPbNu@StateN@SkipZeros%
  \or
    \ThPdNu@ReturnAndSkip{1}%
  \or
    \ThPbNu@StateN@Dot\ThPdNu@zero!!%
  \else
    \ThPdNu@ReturnError{0}%
  \fi
}
%    \end{macrocode}
%    \end{macro}
%
% \subsubsection{State definitions for decimal digits}
%
%    \begin{macro}{\ThPbNu@StateN@Dot}
%    \begin{macrocode}
\def\ThPbNu@StateN@Dot#1#2!#3!#4\fi#5{%
  \fi
  \ifcase\ifx\ThPdNu@FIN#5%
           0%
         \else\ifnum`#5>48 %
           \ifnum`#5<58 %
             1%
           \else
             9%
           \fi
         \else\ifx0#5%
           2%
         \else
           9%
         \fi\fi\fi\ThPdNu@space
    \ThPdNu@ReturnFracNumber#1#2!%
  \or
    \ThPbNu@StateN@Dot\ThPdNu@one#2#3#5!!%
  \or
    \ThPbNu@StateN@DotZero#1#2!#3#5!%
  \else
    \ThPdNu@ReturnFracNumberInvalid#1#2!%
  \fi
}
%    \end{macrocode}
%    \end{macro}
%
%    \begin{macro}{\ThPbNu@StateN@DotZero}
%    \begin{macrocode}
\def\ThPbNu@StateN@DotZero#1#2!#3!#4\fi#5{%
  \fi
  \ifcase\ifx\ThPdNu@FIN#5%
           0%
         \else\ifnum`#5>48 %
           \ifnum`#5<58 %
             1%
           \else
             9%
           \fi
         \else\ifx0#5%
           2%
         \else
           9%
         \fi\fi\fi\ThPdNu@space
    \ThPdNu@ReturnFracNumber#1#2!%
  \or
    \ThPbNu@StateN@Dot\ThPdNu@one#2#3#5!!%
  \or
    \ThPbNu@StateN@DotZero#1#2!#3#5!%
  \else
    \ThPdNu@ReturnFracNumberInvalid#1#2!%
  \fi
}
%    \end{macrocode}
%    \end{macro}
%
%    \begin{macro}{\ThPdNu@ReturnFracNumber}
%    \begin{macrocode}
\def\ThPdNu@ReturnFracNumber#1#2!#3\fi{%
  \fi
  \ifcase#1%
    \expandafter\ThPdNu@firstoftwo
  \else
    \expandafter\ThPdNu@secondoftwo
  \fi
  {\ThPdNu@zero 0}%
  {\ThPdNu@zero .#2}%
}
%    \end{macrocode}
%    \end{macro}
%
%    \begin{macro}{\ThPdNu@ReturnFracNumberInvalid}
%    \begin{macrocode}
\def\ThPdNu@ReturnFracNumberInvalid#1!#2\fi#3\ThPdNu@FIN{%
  \fi
  \iftrue\ThPdNu@ReturnFracNumber#1!\fi
  \thepdfnumber@ErrorInvalidToken
}
%    \end{macrocode}
%    \end{macro}
%
%    \begin{macrocode}
\ThPdNu@AtEnd%
%</package>
%    \end{macrocode}
%
% \section{Test}
%
% \subsection{Catcode checks for loading}
%
%    \begin{macrocode}
%<*test1>
%    \end{macrocode}
%    \begin{macrocode}
\catcode`\{=1 %
\catcode`\}=2 %
\catcode`\#=6 %
\catcode`\@=11 %
\expandafter\ifx\csname count@\endcsname\relax
  \countdef\count@=255 %
\fi
\expandafter\ifx\csname @gobble\endcsname\relax
  \long\def\@gobble#1{}%
\fi
\expandafter\ifx\csname @firstofone\endcsname\relax
  \long\def\@firstofone#1{#1}%
\fi
\expandafter\ifx\csname loop\endcsname\relax
  \expandafter\@firstofone
\else
  \expandafter\@gobble
\fi
{%
  \def\loop#1\repeat{%
    \def\body{#1}%
    \iterate
  }%
  \def\iterate{%
    \body
      \let\next\iterate
    \else
      \let\next\relax
    \fi
    \next
  }%
  \let\repeat=\fi
}%
\def\RestoreCatcodes{}
\count@=0 %
\loop
  \edef\RestoreCatcodes{%
    \RestoreCatcodes
    \catcode\the\count@=\the\catcode\count@\relax
  }%
\ifnum\count@<255 %
  \advance\count@ 1 %
\repeat

\def\RangeCatcodeInvalid#1#2{%
  \count@=#1\relax
  \loop
    \catcode\count@=15 %
  \ifnum\count@<#2\relax
    \advance\count@ 1 %
  \repeat
}
\def\RangeCatcodeCheck#1#2#3{%
  \count@=#1\relax
  \loop
    \ifnum#3=\catcode\count@
    \else
      \errmessage{%
        Character \the\count@\space
        with wrong catcode \the\catcode\count@\space
        instead of \number#3%
      }%
    \fi
  \ifnum\count@<#2\relax
    \advance\count@ 1 %
  \repeat
}
\def\space{ }
\expandafter\ifx\csname LoadCommand\endcsname\relax
  \def\LoadCommand{\input thepdfnumber.sty\relax}%
\fi
\def\Test{%
  \RangeCatcodeInvalid{0}{47}%
  \RangeCatcodeInvalid{58}{64}%
  \RangeCatcodeInvalid{91}{96}%
  \RangeCatcodeInvalid{123}{255}%
  \catcode`\@=12 %
  \catcode`\\=0 %
  \catcode`\%=14 %
  \LoadCommand
  \RangeCatcodeCheck{0}{36}{15}%
  \RangeCatcodeCheck{37}{37}{14}%
  \RangeCatcodeCheck{38}{47}{15}%
  \RangeCatcodeCheck{48}{57}{12}%
  \RangeCatcodeCheck{58}{63}{15}%
  \RangeCatcodeCheck{64}{64}{12}%
  \RangeCatcodeCheck{65}{90}{11}%
  \RangeCatcodeCheck{91}{91}{15}%
  \RangeCatcodeCheck{92}{92}{0}%
  \RangeCatcodeCheck{93}{96}{15}%
  \RangeCatcodeCheck{97}{122}{11}%
  \RangeCatcodeCheck{123}{255}{15}%
  \RestoreCatcodes
}
\Test
\csname @@end\endcsname
\end
%    \end{macrocode}
%    \begin{macrocode}
%</test1>
%    \end{macrocode}
%
% \subsection{Macro tests}
%
% \subsubsection{Preamble}
%
%    \begin{macrocode}
%<*test2>
\catcode`\{=1
\catcode`\}=2
\catcode`\#=6
\catcode`\@=11
\errorcontextlines=10000 %
\def\msg#{\immediate\write16}
\def\space{ }
\begingroup\expandafter\expandafter\expandafter\endgroup
\expandafter\ifx\csname RequirePackage\endcsname\relax
  \input thepdfnumber.sty\relax
\else
  \RequirePackage{thepdfnumber}[2016/05/16]%
\fi
%    \end{macrocode}
%
%    \begin{macro}{\Test}
%    \begin{macrocode}
\def\Test#1#2{%
  \def\TestExpected{#2}%
%<*active>
  \edef\TestExpected{\TestExpected}%
%</active>
  \expandafter\expandafter\expandafter\def
  \expandafter\expandafter\expandafter\TestResult
  \expandafter\expandafter\expandafter{%
    \TestCommand{#1}%
  }%
  \begingroup
    \toks0\expandafter{\TestResult}%
    \toks2\expandafter{\TestExpected}%
    \ifx\TestResult\TestExpected
      \msg{* OK: #1 => \the\toks0}%
    \else
      \errmessage{FAILED: #1 => \the\toks0\space(\the\toks2)}%
    \fi
  \endgroup
}
%    \end{macrocode}
%    \end{macro}
%
%    \begin{macrocode}
\let\thepdfnumber@ErrorUnexpectedEnd\relax
\let\thepdfnumber@ErrorInvalidToken\relax
%    \end{macrocode}
%
% \subsubsection{Test with active catcodes}
%
%    \begin{macrocode}
%<*active>
\catcode`\~=13 %
\def\ThPdNu@Temp#1{%
  \begingroup
    \lccode`\~=`#1 %
  \lowercase{\endgroup
    \edef~%
  }{\string#1}%
  \catcode`#1=13 %
}
\ThPdNu@Temp{.}
\ThPdNu@Temp{-}
\ThPdNu@Temp{+}
%</active>
%    \end{macrocode}
%
% \subsubsection{Test cases for \cs{thepdfnumber}}
%
%    \begin{macro}{\TestCommand}
%    \begin{macrocode}
\let\TestCommand\thepdfnumber
%    \end{macrocode}
%    \end{macro}
%
%    \begin{macrocode}
\Test{0}{0}
\Test{1}{1}
\Test{23}{23}
\Test{0045}{45}
\Test{4500}{4500}
\Test{-0}{0}
\Test{-1}{-1}
\Test{-00002}{-2}
\Test{-010203}{-10203}
\Test{+0}{0}
\Test{+123}{123}

\Test{+-4}{-4}
\Test{+-+-4}{4}
\Test{++--++--++4}{4}
\Test{+++---+++---+++---4}{-4}

\Test{.}{0}
\Test{0.}{0}
\Test{.0}{0}
\Test{0.}{0}
\Test{-.}{0}
\Test{-0.}{0}
\Test{-.0}{0}
\Test{-0.}{0}
\Test{010.020}{10.02}
\Test{123.456}{123.456}
\Test{12.}{12}
\Test{.123}{.123}
\Test{.0001}{.0001}
\Test{.00100}{.001}
\Test{.12003400560078009}{.12003400560078009}
\Test{-.12003400560078009}{-.12003400560078009}
\Test{04.0500000}{4.05}

\Test{}{0\thepdfnumber@ErrorUnexpectedEnd}
\Test{+}{0\thepdfnumber@ErrorUnexpectedEnd}
\Test{-}{0\thepdfnumber@ErrorUnexpectedEnd}
\Test{a}{0\thepdfnumber@ErrorInvalidToken}
\Test{0x1}{0\thepdfnumber@ErrorInvalidToken}
\Test{4x56}{4\thepdfnumber@ErrorInvalidToken}
\Test{012,34}{12\thepdfnumber@ErrorInvalidToken}
\Test{0.12x4}{.12\thepdfnumber@ErrorInvalidToken}
%    \end{macrocode}
%
% \subsubsection{Test cases for \cs{thepdfnumberNormZeroOne}}
%
%    \begin{macro}{\TestCommand}
%    \begin{macrocode}
\let\TestCommand\thepdfnumberNormZeroOne
%    \end{macrocode}
%    \end{macro}
%
%    \begin{macrocode}
\Test{0}{0}
\Test{1}{1}
\Test{23}{1}
\Test{12}{1}
\Test{10}{1}
\Test{0045}{1}
\Test{001}{1}
\Test{-0}{0}
\Test{-1}{0}
\Test{-01}{0}
\Test{-4}{0}

\Test{+-0}{0}
\Test{+-+-1}{1}
\Test{++--++--++1}{1}
\Test{+++---+++---+++---1}{0}

\Test{.}{0}
\Test{0.}{0}
\Test{.0}{0}
\Test{0.}{0}
\Test{-.}{0}
\Test{-0.}{0}
\Test{-.0}{0}
\Test{-0.}{0}
\Test{010.020}{1}
\Test{123.456}{1}
\Test{12.}{1}
\Test{.123}{.123}
\Test{.0001}{.0001}
\Test{.00100}{.001}
\Test{.12003400560078009}{.12003400560078009}
\Test{-.12003400560078009}{0}
\Test{04.0500000}{1}
\Test{0.1200340056}{.1200340056}
\Test{1.05}{1}

\Test{}{0\thepdfnumber@ErrorUnexpectedEnd}
\Test{+}{0\thepdfnumber@ErrorUnexpectedEnd}
\Test{-}{0\thepdfnumber@ErrorUnexpectedEnd}
\Test{a}{0\thepdfnumber@ErrorInvalidToken}
\Test{0x1}{0\thepdfnumber@ErrorInvalidToken}
\Test{4x56}{1}
\Test{012,34}{1}
\Test{.012x4}{.012\thepdfnumber@ErrorInvalidToken}
%    \end{macrocode}
%
%    \begin{macrocode}
\csname @@end\endcsname\end
%</test2>
%    \end{macrocode}
%
% \section{Installation}
%
% \subsection{Download}
%
% \paragraph{Package.} This package is available on
% CTAN\footnote{\url{http://ctan.org/pkg/thepdfnumber}}:
% \begin{description}
% \item[\CTAN{macros/latex/contrib/oberdiek/thepdfnumber.dtx}] The source file.
% \item[\CTAN{macros/latex/contrib/oberdiek/thepdfnumber.pdf}] Documentation.
% \end{description}
%
%
% \paragraph{Bundle.} All the packages of the bundle `oberdiek'
% are also available in a TDS compliant ZIP archive. There
% the packages are already unpacked and the documentation files
% are generated. The files and directories obey the TDS standard.
% \begin{description}
% \item[\CTAN{install/macros/latex/contrib/oberdiek.tds.zip}]
% \end{description}
% \emph{TDS} refers to the standard ``A Directory Structure
% for \TeX\ Files'' (\CTAN{tds/tds.pdf}). Directories
% with \xfile{texmf} in their name are usually organized this way.
%
% \subsection{Bundle installation}
%
% \paragraph{Unpacking.} Unpack the \xfile{oberdiek.tds.zip} in the
% TDS tree (also known as \xfile{texmf} tree) of your choice.
% Example (linux):
% \begin{quote}
%   |unzip oberdiek.tds.zip -d ~/texmf|
% \end{quote}
%
% \paragraph{Script installation.}
% Check the directory \xfile{TDS:scripts/oberdiek/} for
% scripts that need further installation steps.
% Package \xpackage{attachfile2} comes with the Perl script
% \xfile{pdfatfi.pl} that should be installed in such a way
% that it can be called as \texttt{pdfatfi}.
% Example (linux):
% \begin{quote}
%   |chmod +x scripts/oberdiek/pdfatfi.pl|\\
%   |cp scripts/oberdiek/pdfatfi.pl /usr/local/bin/|
% \end{quote}
%
% \subsection{Package installation}
%
% \paragraph{Unpacking.} The \xfile{.dtx} file is a self-extracting
% \docstrip\ archive. The files are extracted by running the
% \xfile{.dtx} through \plainTeX:
% \begin{quote}
%   \verb|tex thepdfnumber.dtx|
% \end{quote}
%
% \paragraph{TDS.} Now the different files must be moved into
% the different directories in your installation TDS tree
% (also known as \xfile{texmf} tree):
% \begin{quote}
% \def\t{^^A
% \begin{tabular}{@{}>{\ttfamily}l@{ $\rightarrow$ }>{\ttfamily}l@{}}
%   thepdfnumber.sty & tex/generic/oberdiek/thepdfnumber.sty\\
%   thepdfnumber.pdf & doc/latex/oberdiek/thepdfnumber.pdf\\
%   test/thepdfnumber-test1.tex & doc/latex/oberdiek/test/thepdfnumber-test1.tex\\
%   test/thepdfnumber-test2.tex & doc/latex/oberdiek/test/thepdfnumber-test2.tex\\
%   test/thepdfnumber-test3.tex & doc/latex/oberdiek/test/thepdfnumber-test3.tex\\
%   thepdfnumber.dtx & source/latex/oberdiek/thepdfnumber.dtx\\
% \end{tabular}^^A
% }^^A
% \sbox0{\t}^^A
% \ifdim\wd0>\linewidth
%   \begingroup
%     \advance\linewidth by\leftmargin
%     \advance\linewidth by\rightmargin
%   \edef\x{\endgroup
%     \def\noexpand\lw{\the\linewidth}^^A
%   }\x
%   \def\lwbox{^^A
%     \leavevmode
%     \hbox to \linewidth{^^A
%       \kern-\leftmargin\relax
%       \hss
%       \usebox0
%       \hss
%       \kern-\rightmargin\relax
%     }^^A
%   }^^A
%   \ifdim\wd0>\lw
%     \sbox0{\small\t}^^A
%     \ifdim\wd0>\linewidth
%       \ifdim\wd0>\lw
%         \sbox0{\footnotesize\t}^^A
%         \ifdim\wd0>\linewidth
%           \ifdim\wd0>\lw
%             \sbox0{\scriptsize\t}^^A
%             \ifdim\wd0>\linewidth
%               \ifdim\wd0>\lw
%                 \sbox0{\tiny\t}^^A
%                 \ifdim\wd0>\linewidth
%                   \lwbox
%                 \else
%                   \usebox0
%                 \fi
%               \else
%                 \lwbox
%               \fi
%             \else
%               \usebox0
%             \fi
%           \else
%             \lwbox
%           \fi
%         \else
%           \usebox0
%         \fi
%       \else
%         \lwbox
%       \fi
%     \else
%       \usebox0
%     \fi
%   \else
%     \lwbox
%   \fi
% \else
%   \usebox0
% \fi
% \end{quote}
% If you have a \xfile{docstrip.cfg} that configures and enables \docstrip's
% TDS installing feature, then some files can already be in the right
% place, see the documentation of \docstrip.
%
% \subsection{Refresh file name databases}
%
% If your \TeX~distribution
% (\teTeX, \mikTeX, \dots) relies on file name databases, you must refresh
% these. For example, \teTeX\ users run \verb|texhash| or
% \verb|mktexlsr|.
%
% \subsection{Some details for the interested}
%
% \paragraph{Attached source.}
%
% The PDF documentation on CTAN also includes the
% \xfile{.dtx} source file. It can be extracted by
% AcrobatReader 6 or higher. Another option is \textsf{pdftk},
% e.g. unpack the file into the current directory:
% \begin{quote}
%   \verb|pdftk thepdfnumber.pdf unpack_files output .|
% \end{quote}
%
% \paragraph{Unpacking with \LaTeX.}
% The \xfile{.dtx} chooses its action depending on the format:
% \begin{description}
% \item[\plainTeX:] Run \docstrip\ and extract the files.
% \item[\LaTeX:] Generate the documentation.
% \end{description}
% If you insist on using \LaTeX\ for \docstrip\ (really,
% \docstrip\ does not need \LaTeX), then inform the autodetect routine
% about your intention:
% \begin{quote}
%   \verb|latex \let\install=y\input{thepdfnumber.dtx}|
% \end{quote}
% Do not forget to quote the argument according to the demands
% of your shell.
%
% \paragraph{Generating the documentation.}
% You can use both the \xfile{.dtx} or the \xfile{.drv} to generate
% the documentation. The process can be configured by the
% configuration file \xfile{ltxdoc.cfg}. For instance, put this
% line into this file, if you want to have A4 as paper format:
% \begin{quote}
%   \verb|\PassOptionsToClass{a4paper}{article}|
% \end{quote}
% An example follows how to generate the
% documentation with pdf\LaTeX:
% \begin{quote}
%\begin{verbatim}
%pdflatex thepdfnumber.dtx
%makeindex -s gind.ist thepdfnumber.idx
%pdflatex thepdfnumber.dtx
%makeindex -s gind.ist thepdfnumber.idx
%pdflatex thepdfnumber.dtx
%\end{verbatim}
% \end{quote}
%
% \printbibliography[
%   heading=bibnumbered,
% ]
%
% \begin{History}
%   \begin{Version}{2011/11/24 v1.0}
%   \item
%     First version.
%   \end{Version}
%   \begin{Version}{2016/05/16 v1.1}
%   \item
%     Documentation updates.
%   \end{Version}
% \end{History}
%
% \PrintIndex
%
% \Finale
\endinput

%        (quote the arguments according to the demands of your shell)
%
% Documentation:
%    (a) If thepdfnumber.drv is present:
%           latex thepdfnumber.drv
%    (b) Without thepdfnumber.drv:
%           latex thepdfnumber.dtx; ...
%    The class ltxdoc loads the configuration file ltxdoc.cfg
%    if available. Here you can specify further options, e.g.
%    use A4 as paper format:
%       \PassOptionsToClass{a4paper}{article}
%
%    Programm calls to get the documentation (example):
%       pdflatex thepdfnumber.dtx
%       makeindex -s gind.ist thepdfnumber.idx
%       pdflatex thepdfnumber.dtx
%       makeindex -s gind.ist thepdfnumber.idx
%       pdflatex thepdfnumber.dtx
%
% Installation:
%    TDS:tex/generic/oberdiek/thepdfnumber.sty
%    TDS:doc/latex/oberdiek/thepdfnumber.pdf
%    TDS:doc/latex/oberdiek/test/thepdfnumber-test1.tex
%    TDS:doc/latex/oberdiek/test/thepdfnumber-test2.tex
%    TDS:doc/latex/oberdiek/test/thepdfnumber-test3.tex
%    TDS:source/latex/oberdiek/thepdfnumber.dtx
%
%<*ignore>
\begingroup
  \catcode123=1 %
  \catcode125=2 %
  \def\x{LaTeX2e}%
\expandafter\endgroup
\ifcase 0\ifx\install y1\fi\expandafter
         \ifx\csname processbatchFile\endcsname\relax\else1\fi
         \ifx\fmtname\x\else 1\fi\relax
\else\csname fi\endcsname
%</ignore>
%<*install>
\input docstrip.tex
\Msg{************************************************************************}
\Msg{* Installation}
\Msg{* Package: thepdfnumber 2016/05/16 v1.1 Print PDF numbers with minimal digits (HO)}
\Msg{************************************************************************}

\keepsilent
\askforoverwritefalse

\let\MetaPrefix\relax
\preamble

This is a generated file.

Project: thepdfnumber
Version: 2016/05/16 v1.1

Copyright (C) 2011 by
   Heiko Oberdiek <heiko.oberdiek at googlemail.com>

This work may be distributed and/or modified under the
conditions of the LaTeX Project Public License, either
version 1.3c of this license or (at your option) any later
version. This version of this license is in
   http://www.latex-project.org/lppl/lppl-1-3c.txt
and the latest version of this license is in
   http://www.latex-project.org/lppl.txt
and version 1.3 or later is part of all distributions of
LaTeX version 2005/12/01 or later.

This work has the LPPL maintenance status "maintained".

This Current Maintainer of this work is Heiko Oberdiek.

The Base Interpreter refers to any `TeX-Format',
because some files are installed in TDS:tex/generic//.

This work consists of the main source file thepdfnumber.dtx
and the derived files
   thepdfnumber.sty, thepdfnumber.pdf, thepdfnumber.ins, thepdfnumber.drv,
   thepdfnumber-test1.tex, thepdfnumber-test2.tex,
   thepdfnumber-test3.tex.

\endpreamble
\let\MetaPrefix\DoubleperCent

\generate{%
  \file{thepdfnumber.ins}{\from{thepdfnumber.dtx}{install}}%
  \file{thepdfnumber.drv}{\from{thepdfnumber.dtx}{driver}}%
  \usedir{tex/generic/oberdiek}%
  \file{thepdfnumber.sty}{\from{thepdfnumber.dtx}{package}}%
  \usedir{doc/latex/oberdiek/test}%
  \file{thepdfnumber-test1.tex}{\from{thepdfnumber.dtx}{test1}}%
  \file{thepdfnumber-test2.tex}{\from{thepdfnumber.dtx}{test2}}%
  \file{thepdfnumber-test3.tex}{\from{thepdfnumber.dtx}{test2,active}}%
}

\catcode32=13\relax% active space
\let =\space%
\Msg{************************************************************************}
\Msg{*}
\Msg{* To finish the installation you have to move the following}
\Msg{* file into a directory searched by TeX:}
\Msg{*}
\Msg{*     thepdfnumber.sty}
\Msg{*}
\Msg{* To produce the documentation run the file `thepdfnumber.drv'}
\Msg{* through LaTeX.}
\Msg{*}
\Msg{* Happy TeXing!}
\Msg{*}
\Msg{************************************************************************}

\endbatchfile
%</install>
%<*ignore>
\fi
%</ignore>
%<*driver>
\NeedsTeXFormat{LaTeX2e}
\ProvidesFile{thepdfnumber.drv}%
  [2016/05/16 v1.1 Print PDF numbers with minimal digits (HO)]%
\documentclass{ltxdoc}
\usepackage{holtxdoc}[2011/11/22]
\usepackage{csquotes}
\usepackage[
  backend=bibtex,
  bibencoding=ascii,
  alldates=iso8601,
]{biblatex}[2011/11/13]
\bibliography{oberdiek-source}
\begin{document}
  \DocInput{thepdfnumber.dtx}%
\end{document}
%</driver>
% \fi
%
%
% \CharacterTable
%  {Upper-case    \A\B\C\D\E\F\G\H\I\J\K\L\M\N\O\P\Q\R\S\T\U\V\W\X\Y\Z
%   Lower-case    \a\b\c\d\e\f\g\h\i\j\k\l\m\n\o\p\q\r\s\t\u\v\w\x\y\z
%   Digits        \0\1\2\3\4\5\6\7\8\9
%   Exclamation   \!     Double quote  \"     Hash (number) \#
%   Dollar        \$     Percent       \%     Ampersand     \&
%   Acute accent  \'     Left paren    \(     Right paren   \)
%   Asterisk      \*     Plus          \+     Comma         \,
%   Minus         \-     Point         \.     Solidus       \/
%   Colon         \:     Semicolon     \;     Less than     \<
%   Equals        \=     Greater than  \>     Question mark \?
%   Commercial at \@     Left bracket  \[     Backslash     \\
%   Right bracket \]     Circumflex    \^     Underscore    \_
%   Grave accent  \`     Left brace    \{     Vertical bar  \|
%   Right brace   \}     Tilde         \~}
%
% \GetFileInfo{thepdfnumber.drv}
%
% \title{The \xpackage{thepdfnumber} package}
% \date{2016/05/16 v1.1}
% \author{Heiko Oberdiek\thanks
% {Please report any issues at https://github.com/ho-tex/oberdiek/issues}\\
% \xemail{heiko.oberdiek at googlemail.com}}
%
% \maketitle
%
% \begin{abstract}
% The package converts real numbers to a minimal representation
% that is stripped from leading or trailing zeros,
% plus signs and decimal point if not necessary.
% \end{abstract}
%
% \tableofcontents
%
% \section{Documentation}
%
% \subsection{Introduction}
%
% Dealing with the PDF format, there is sometimes the need to write
% some low level PDF stuff. In case of numbers, the numbers can
% arise from user input (e.g. color or transparency specifications)
% or can be calculated. For example, \hologo{LaTeX}'s \cs{strip@pt}
% makes a good job to output a real number. It automatically suppresses
% the decimal part if the number is an integer. However it leaves
% a leading zero for numbers greater zero and smaller one.
% Thus the package provides macros that can be used with different
% formats, even with \hologo{iniTeX} and generates numbers that
% are valid numbers of the PDF format and whose length is minimal.
%
% \subsection{Usage}
%
% The package \xpackage{thepdfnumber} can be used with \hologo{LaTeX},
% \hologo{plainTeX} or even with \hologo{iniTeX}:
% \begin{quote}
%   |\RequirePackage{thepdfnumber}| \textit{\% \hologo{LaTeX}}\\
%   |\input thepdfnumber.sty| \textit{\% \hologo{plainTeX}/\hologo{iniTeX}}
% \end{quote}
% The package does not need and have package options.
%
% \subsection{User macros}
%
% All user macros are expandable in exact two expansion steps.
%
% \begin{declcs}{thepdfnumber} \M{number}
% \end{declcs}
% Macro \cs{thepdfnumber} takes a number
% as argument and expands to a minimal representation of that number.
% Some examples:
% \begin{quote}
% \begin{tabular}{r@{.: }>{\ttfamily}l@{ $\rightarrow$\ }>{\ttfamily}l@{}}
%   1& +123 & 123\\
%   2& --123 & 123\\
%   3& -01 & -1\\
%   4& 0045 & 45\\
%   5& 1.0 & 1\\
%   6& 1.20 & 1.2\\
%   7& 0.0 & 0\\
%   8& 0.78 & .78\\
%   9& +012.340 & 12.34\\
% \end{tabular}
% \end{quote}
% It reduces the length of the number representation:
% \begin{itemize}
% \item The signs are collapsed and only one minus sign
%   is output if the number is negative (see examples 1, 2, 3, 9).
% \item Leading zeros are removed (4, 8, 9) unless the number is zero (7).
% \item The decimal part is omitted, if the number is an integer (5, 7).
% \item Trailing zeros from the decimal part are stripped (5, 6, 7, 9).
% \end{itemize}
% The resulting number representation can be catched with one of the
% following Perl regular expressions:
% \begin{itemize}
% \item |^0$| (zero)
% \item |^-?[1-9][0-9]*$| (integer)
% \item |^-?[0-9]*\.[0-9]*[1-9]$| (real)
% \end{itemize}
% This is a valid numeric object of the PDF specification \cite[``7.3.3~Numeric Objects'']{pdfspec-iso32000-1}.
%
% \begin{declcs}{thepdfnumberNormZeroOne}
% \end{declcs}
% There are various places in the PDF specification where the
% number is in the domain 0.0 upto 1.0. Macro thepdfnumberNormZeroOne
% automatically adjusts the number to fit into that range. Negative
% numbers are mapped to |0| and numbers greater than one are
% replaced by |1|. Thus the result fits one of the following regular
% expressions:
% \begin{itemize}
% \item |^0$|
% \item |^\.[0-9]*[1-9]$|
% \item |^1$|
% \end{itemize}
% Examples:
% \begin{quote}
% \begin{tabular}{@{}>{\ttfamily}l@{ $\rightarrow$\ }>{\ttfamily}l@{}}
%   -456 & 0\\
%   -0.001 & 0\\
%   0.0 & 0\\
%   0.010 & .01\\
%   0.456 & .456\\
%   1.0 & 1\\
%   01.001 & 1\\
%   4 & 1\\
% \end{tabular}
% \end{quote}
%
% \subsection{Input number}
%
% The user macros expect a number as argument. The number can either
% be given explicitely or as macro that expands in one step
% to an explicite number, because the first token of the argument
% is expanded once.
%
% The explicite number consists of
% \begin{itemize}
% \item optional signs `\texttt{+}' and `\texttt{-}',
% \item digits `\texttt{0}' upto `\texttt{9}' and
% \item an optional dot `\texttt{.}'.
% \end{itemize}
% All tokens must have catcode 12 (other), the default catcodes
% for these characters in \hologo{LaTeX}, \hologo{plainTeX} or
% \hologo{iniTeX}. As Perl regular expression the number is expected
% in one of the following forms:
% \begin{itemize}
% \item |^[+-]*[0-9]+$|
% \item |^[+-]*[0-9]*\.[0-9]*$|
% \end{itemize}
% At least one digit or the dot must be present.
%
% \subsection{Error handling}
%
% The package is not intended for validating numbers or to
% decide if an argument is a number. Therefore it is an usage
% error to use the user macros with arguments that are not
% explicite numbers as specified in the previous sections.
% Nevertheless some error conditions are sometimes recognized.
% Errors are given in form of an undefined command sequence.
% It is the only way to notify \hologo{TeX} in expandable context.
% Expanding to some error text would invalidate the output.
% Currently the following errors are thrown:
% \begin{description}
% \item[\cs{thepdfnumber@ErrorEndMarker}:]
%   Internally the argument parsing uses an end marker that
%   is never called directly. If it is called with valid
%   user input, then this is a bug. Otherwise it means the user input
%   contains nasty stuff.
% \item[\cs{thepdfnumber@ErrorUnexpectedEnd}:] The macros expect at least one digit
%   or the dot, otherwise if the argument is empty or only contains
%   signs, then this error is called.
% \item[\cs{thepdfnumber@ErrorInvalidToken}:] It is called if the number
%   contains other tokens than signs, digits or the dot or the
%   token at the wrong place (e.g. a sign after a digit).
%   In case of \cs{thepdfnumberNormZeroOne} this error condition might
%   not always be detected, because the number parsing might stop
%   at an early point, when the result is already clear (e.g.
%   if the number is negative or will be greater than one).
% \item[Improper alphabetic constant:] This error might be
%   thrown by \hologo{TeX}, if the number contains
%   command tokens instead of characters.
% \end{description}
%
% \StopEventually{
% }
%
% \section{Implementation}
%
%    \begin{macrocode}
%<*package>
%    \end{macrocode}
%
% \subsection{Reload check and package identification}
%    Reload check, especially if the package is not used with \LaTeX.
%    \begin{macrocode}
\begingroup\catcode61\catcode48\catcode32=10\relax%
  \catcode13=5 % ^^M
  \endlinechar=13 %
  \catcode35=6 % #
  \catcode39=12 % '
  \catcode44=12 % ,
  \catcode45=12 % -
  \catcode46=12 % .
  \catcode58=12 % :
  \catcode64=11 % @
  \catcode123=1 % {
  \catcode125=2 % }
  \expandafter\let\expandafter\x\csname ver@thepdfnumber.sty\endcsname
  \ifx\x\relax % plain-TeX, first loading
  \else
    \def\empty{}%
    \ifx\x\empty % LaTeX, first loading,
      % variable is initialized, but \ProvidesPackage not yet seen
    \else
      \expandafter\ifx\csname PackageInfo\endcsname\relax
        \def\x#1#2{%
          \immediate\write-1{Package #1 Info: #2.}%
        }%
      \else
        \def\x#1#2{\PackageInfo{#1}{#2, stopped}}%
      \fi
      \x{thepdfnumber}{The package is already loaded}%
      \aftergroup\endinput
    \fi
  \fi
\endgroup%
%    \end{macrocode}
%    Package identification:
%    \begin{macrocode}
\begingroup\catcode61\catcode48\catcode32=10\relax%
  \catcode13=5 % ^^M
  \endlinechar=13 %
  \catcode35=6 % #
  \catcode39=12 % '
  \catcode40=12 % (
  \catcode41=12 % )
  \catcode44=12 % ,
  \catcode45=12 % -
  \catcode46=12 % .
  \catcode47=12 % /
  \catcode58=12 % :
  \catcode64=11 % @
  \catcode91=12 % [
  \catcode93=12 % ]
  \catcode123=1 % {
  \catcode125=2 % }
  \expandafter\ifx\csname ProvidesPackage\endcsname\relax
    \def\x#1#2#3[#4]{\endgroup
      \immediate\write-1{Package: #3 #4}%
      \xdef#1{#4}%
    }%
  \else
    \def\x#1#2[#3]{\endgroup
      #2[{#3}]%
      \ifx#1\@undefined
        \xdef#1{#3}%
      \fi
      \ifx#1\relax
        \xdef#1{#3}%
      \fi
    }%
  \fi
\expandafter\x\csname ver@thepdfnumber.sty\endcsname
\ProvidesPackage{thepdfnumber}%
  [2016/05/16 v1.1 Print PDF numbers with minimal digits (HO)]%
%    \end{macrocode}
%
% \subsection{Catcodes}
%
%    \begin{macrocode}
\begingroup\catcode61\catcode48\catcode32=10\relax%
  \catcode13=5 % ^^M
  \endlinechar=13 %
  \catcode123=1 % {
  \catcode125=2 % }
  \catcode64=11 % @
  \def\x{\endgroup
    \expandafter\edef\csname ThPdNu@AtEnd\endcsname{%
      \endlinechar=\the\endlinechar\relax
      \catcode13=\the\catcode13\relax
      \catcode32=\the\catcode32\relax
      \catcode35=\the\catcode35\relax
      \catcode61=\the\catcode61\relax
      \catcode64=\the\catcode64\relax
      \catcode123=\the\catcode123\relax
      \catcode125=\the\catcode125\relax
    }%
  }%
\x\catcode61\catcode48\catcode32=10\relax%
\catcode13=5 % ^^M
\endlinechar=13 %
\catcode35=6 % #
\catcode64=11 % @
\catcode123=1 % {
\catcode125=2 % }
\def\TMP@EnsureCode#1#2{%
  \edef\ThPdNu@AtEnd{%
    \ThPdNu@AtEnd
    \catcode#1=\the\catcode#1\relax
  }%
  \catcode#1=#2\relax
}
\TMP@EnsureCode{33}{12}% !
\TMP@EnsureCode{36}{3}% $
\TMP@EnsureCode{38}{4}% &
\TMP@EnsureCode{42}{12}% *
\TMP@EnsureCode{43}{12}% +
\TMP@EnsureCode{45}{12}% -
\TMP@EnsureCode{46}{12}% .
\TMP@EnsureCode{60}{12}% <
\TMP@EnsureCode{62}{12}% >
\TMP@EnsureCode{96}{12}% `
\edef\ThPdNu@AtEnd{\ThPdNu@AtEnd\noexpand\endinput}
%    \end{macrocode}
%
% \subsection{Helper macros}
%
%    \begin{macro}{\ThPdNu@FIN}
%    \begin{macrocode}
\def\ThPdNu@FIN{\thepdfnumber@ErrorEndMarker}
%    \end{macrocode}
%    \end{macro}
%
%    \begin{macro}{\ThPdNu@space}
%    \begin{macrocode}
\def\ThPdNu@space{ }
%    \end{macrocode}
%    \end{macro}
%
%    \begin{macro}{\ThPdNu@zero}
%    \begin{macrocode}
\chardef\ThPdNu@zero=0 %
%    \end{macrocode}
%    \end{macro}
%
%    \begin{macro}{\ThPdNu@one}
%    \begin{macrocode}
\chardef\ThPdNu@one=1 %
%    \end{macrocode}
%    \end{macro}
%
%    \begin{macro}{\ThPdNu@firstoftwo}
%    \begin{macrocode}
\long\def\ThPdNu@firstoftwo#1#2{#1}
%    \end{macrocode}
%    \end{macro}
%
%    \begin{macro}{\ThPdNu@secondoftwo}
%    \begin{macrocode}
\long\def\ThPdNu@secondoftwo#1#2{#2}
%    \end{macrocode}
%    \end{macro}
%
% \subsection{Detect \hologo{eTeX}}
%
%    \begin{macrocode}
\begingroup\expandafter\expandafter\expandafter\endgroup
\expandafter\ifx\csname detokenize\endcsname\relax
  \catcode`\&=14 %
  \catcode`\$=9 %
\else
  \catcode`\&=9 %
  \catcode`\$=14 %
\fi
%    \end{macrocode}
%
% \subsection{User macro \cs{thepdfnumber}}
%
%    \begin{macro}{\thepdfnumber}
%    \begin{macrocode}
\def\thepdfnumber#1{%
  \romannumeral
& \iftrue\expandafter\ThPdNu@State@Plus\expandafter\fi
& \detokenize\expandafter{#1}%
& \ThPdNu@FIN
$ \ifx\ThPdNu@FIN#1\ThPdNu@FIN
$   \expandafter\ThPdNu@firstoftwo
$ \else
$   \expandafter\ThPdNu@secondoftwo
$ \fi
$ {%
$   \ThPdNu@zero
$   0\thepdfnumber@ErrorUnexpectedEnd
$ }{%
$   \iftrue\expandafter\ThPdNu@State@Plus\expandafter\fi#1\ThPdNu@FIN
$ }%
}
%    \end{macrocode}
%    \end{macro}
%
% \subsubsection{State definitions for sign}
%
%    \begin{macro}{\ThPdNu@State@Plus}
%    \begin{macrocode}
\def\ThPdNu@State@Plus#1\fi#2{%
  \fi
  \ifcase\ifx\ThPdNu@FIN#2%
           0%
&        \else\ifx-#2%
$        \else\ifnum`#2=45 % -
           1%
         \else\ifx0#2%
           2%
         \else\ifnum`#2>48 %
           \ifnum`#2<58 %
             3%
           \else
             9%
           \fi
&        \else\ifx.#2%
$        \else\ifnum`#2=46 % .
           4%
&        \else\ifx+#2%
$        \else\ifnum`#2=43 % +
           5%
         \else
           9%
         \fi\fi\fi\fi\fi\fi\ThPdNu@space
    \expandafter\ThPdNu@zero
    \expandafter0%
    \expandafter\thepdfnumber@ErrorUnexpectedEnd
  \or
    \ThPdNu@State@Minus
  \or
    \ThPdNu@State@SkipZeros!%
  \or
    \ThPdNu@State@Int!#2!%
  \or
    \ThPdNu@State@Dot!\ThPdNu@zero*\ThPdNu@zero!!%
  \or
    \ThPdNu@State@Plus
  \else
    \ThPdNu@ReturnError{0}%
  \fi
}
%    \end{macrocode}
%    \end{macro}
%
%    \begin{macro}{\ThPdNu@State@Minus}
%    \begin{macrocode}
\def\ThPdNu@State@Minus#1\fi#2{%
  \fi
  \ifcase\ifx\ThPdNu@FIN#2%
           0%
         \else\ifx0#2%
           1%
         \else\ifnum`#2>48 %
           \ifnum`#2<58 %
             2%
           \else
             9%
           \fi
&        \else\ifx.#2%
$        \else\ifnum`#2=46 % .
           3%
&        \else\ifx-#2%
$        \else\ifnum`#2=45 % -
           4%
&        \else\ifx+#2%
$        \else\ifnum`#2=43 % +
           5%
         \else
           9%
         \fi\fi\fi\fi\fi\fi\ThPdNu@space
    \expandafter\ThPdNu@zero
    \expandafter0%
    \expandafter\thepdfnumber@ErrorUnexpectedEnd
  \or
    \ThPdNu@State@SkipZeros-!%
  \or
    \ThPdNu@State@Int-!#2!%
  \or
    \ThPdNu@State@Dot-!\ThPdNu@zero*\ThPdNu@zero!!%
  \or
    \ThPdNu@State@Plus
  \or
    \ThPdNu@State@Minus
  \else
    \ThPdNu@ReturnError{0}%
  \fi
}
%    \end{macrocode}
%    \end{macro}
%
%    \begin{macro}{\ThPdNu@ReturnError}
%    \begin{macrocode}
\def\ThPdNu@ReturnError#1#2\fi#3\ThPdNu@FIN{%
  \fi
  \ThPdNu@zero
  #1%
  \thepdfnumber@ErrorInvalidToken
}
%    \end{macrocode}
%    \end{macro}
%
% \subsubsection{State definitions for integer part}
%
%    \begin{macro}{\ThPdNu@State@SkipZeros}
%    \begin{macrocode}
\def\ThPdNu@State@SkipZeros#1!#2\fi#3{%
  \fi
  \ifcase\ifx\ThPdNu@FIN#3%
           0%
         \else\ifx0#3%
           1%
         \else\ifnum`#3>48 %
           \ifnum`#3<58 %
             2%
           \else
             9%
           \fi
&        \else\ifx.#3%
$        \else\ifnum`#3=46 % .
           3%
         \else
           9%
         \fi\fi\fi\fi\ThPdNu@space
    \expandafter\ThPdNu@zero
    \expandafter0%
  \or
   \ThPdNu@State@SkipZeros#1!%
  \or
    \ThPdNu@State@Int#1!#3!%
  \or
    \ThPdNu@State@Dot#1!\ThPdNu@zero*\ThPdNu@zero!!%
  \else
    \ThPdNu@ReturnError{0}%
  \fi
}
%    \end{macrocode}
%    \end{macro}
%
%    \begin{macro}{\ThPdNu@State@Int}
%    \begin{macrocode}
\def\ThPdNu@State@Int#1!#2!#3\fi#4{%
  \fi
  \ifcase\ifx\ThPdNu@FIN#4%
           0%
         \else\ifnum`#4>47 %
           \ifnum`#4<58 %
             1%
           \else
             9%
           \fi
&        \else\ifx.#4%
$        \else\ifnum`#4=46 % .
           2%
         \else
           9%
         \fi\fi\fi\ThPdNu@space
    \ThPdNu@ReturnInt{#1#2}%
  \or
    \ThPdNu@State@Int#1!#2#4!%
  \or
    \ThPdNu@State@Dot#1!\ThPdNu@one#2*\ThPdNu@zero!!%
  \else
    \ThPdNu@ReturnError{#1#2}%
  \fi
}
%    \end{macrocode}
%    \end{macro}
%
%    \begin{macro}{\ThPdNu@ReturnInt}
%    \begin{macrocode}
\def\ThPdNu@ReturnInt#1#2\fi{%
  \fi
  \ThPdNu@zero
  #1%
}
%    \end{macrocode}
%    \end{macro}
%
% \subsubsection{State definitions for decimal digits}
%
%    \begin{macro}{\ThPdNu@State@Dot}
%    \begin{macrocode}
\def\ThPdNu@State@Dot#1*#2#3!#4!#5\fi#6{%
  \fi
  \ifcase\ifx\ThPdNu@FIN#6%
           0%
         \else\ifnum`#6>48 %
           \ifnum`#6<58 %
             1%
           \else
             9%
           \fi
         \else\ifx0#6%
           2%
         \else
           9%
         \fi\fi\fi\ThPdNu@space
    \ThPdNu@ReturnNumber#1*#2#3!%
  \or
    \ThPdNu@State@Dot#1*\ThPdNu@one#3#4#6!!%
  \or
    \ThPdNu@State@DotZero#1*#2#3!#4#6!%
  \else
    \ThPdNu@ReturnNumberInvalid#1*#2#3!%
  \fi
}
%    \end{macrocode}
%    \end{macro}
%
%    \begin{macro}{\ThPdNu@State@DotZero}
%    \begin{macrocode}
\def\ThPdNu@State@DotZero#1*#2#3!#4!#5\fi#6{%
  \fi
  \ifcase\ifx\ThPdNu@FIN#6%
           0%
         \else\ifnum`#6>48 %
           \ifnum`#6<58 %
             1%
           \else
             9%
           \fi
         \else\ifx0#6%
           2%
         \else
           9%
         \fi\fi\fi\ThPdNu@space
    \ThPdNu@ReturnNumber#1*#2#3!%
  \or
    \ThPdNu@State@Dot#1*\ThPdNu@one#3#4#6!!%
  \or
    \ThPdNu@State@DotZero#1*#2#3!#4#6!%
  \else
    \ThPdNu@ReturnNumber#1*#2#3!%
  \fi
}
%    \end{macrocode}
%    \end{macro}
%
%    \begin{macro}{\ThPdNu@ReturnNumber}
%    \begin{macrocode}
\def\ThPdNu@ReturnNumber#1!#2#3*#4#5!#6\fi{%
  \fi
  \ifcase#2%
    \expandafter\ThPdNu@firstoftwo
  \else
    \expandafter\ThPdNu@secondoftwo
  \fi
  {%
    \ifcase#4%
      \expandafter\ThPdNu@firstoftwo
    \else
      \expandafter\ThPdNu@secondoftwo
    \fi
    {\ThPdNu@zero 0}%
    {\ThPdNu@zero #1.#5}%
  }{%
    \ifcase#4%
      \expandafter\ThPdNu@firstoftwo
    \else
      \expandafter\ThPdNu@secondoftwo
    \fi
    {\ThPdNu@zero #1#3}%
    {\ThPdNu@zero #1#3.#5}%
  }%
}
%    \end{macrocode}
%    \end{macro}
%
%    \begin{macro}{\ThPdNu@ReturnNumberInvalid}
%    \begin{macrocode}
\def\ThPdNu@ReturnNumberInvalid#1*#2!#3\fi#4\ThPdNu@FIN{%
  \fi
  \iftrue\ThPdNu@ReturnNumber#1*#2!\fi
  \thepdfnumber@ErrorInvalidToken
}
%    \end{macrocode}
%    \end{macro}
%
% \subsection{Norm macro}
%
%    \begin{macro}{\thepdfnumberNormZeroOne}
%    \begin{macrocode}
\def\thepdfnumberNormZeroOne#1{%
  \romannumeral
& \iftrue\expandafter\ThPbNu@StateN@Plus\expandafter\fi
& \detokenize\expandafter{#1}%
& \ThPdNu@FIN
$ \ifx\ThPdNu@FIN#1\ThPdNu@FIN
$   \expandafter\ThPdNu@firstoftwo
$ \else
$   \expandafter\ThPdNu@secondoftwo
$ \fi
$ {%
$   \ThPdNu@zero
$   0\thepdfnumber@ErrorUnexpectedEnd
$ }{%
$   \iftrue\expandafter\ThPbNu@StateN@Plus\expandafter\fi#1\ThPdNu@FIN
$ }%
}
%    \end{macrocode}
%    \end{macro}
%
% \subsubsection{State definitions for sign}
%
%    \begin{macro}{\ThPbNu@StateN@Plus}
%    \begin{macrocode}
\def\ThPbNu@StateN@Plus#1\fi#2{%
  \fi
  \ifcase\ifx\ThPdNu@FIN#2%
           0%
&        \else\ifx-#2%
$        \else\ifnum`#2=45 % -
           1%
         \else\ifx0#2%
           2%
         \else\ifnum`#2>48 %
           \ifnum`#2<58 %
             3%
           \else
             9%
           \fi
&        \else\ifx.#2%
$        \else\ifnum`#2=46 % .
           4%
&        \else\ifx+#2%
$        \else\ifnum`#2=43 % +
           5%
         \else
           9%
         \fi\fi\fi\fi\fi\fi\ThPdNu@space
    \expandafter\ThPdNu@zero
    \expandafter0%
    \expandafter\thepdfnumber@ErrorUnexpectedEnd
  \or
    \ThPbNu@StateN@Minus
  \or
    \ThPbNu@StateN@SkipZeros
  \or
    \ThPdNu@ReturnAndSkip{1}%
  \or
    \ThPbNu@StateN@Dot\ThPdNu@zero!!%
  \or
    \ThPbNu@StateN@Plus
  \else
    \ThPdNu@ReturnError{0}%
  \fi
}
%    \end{macrocode}
%    \end{macro}
%
%    \begin{macro}{\ThPbNu@StateN@Minus}
%    \begin{macrocode}
\def\ThPbNu@StateN@Minus#1\fi#2{%
  \fi
  \ifcase\ifx\ThPdNu@FIN#2%
           0%
         \else\ifnum`#2>47 %
           \ifnum`#2<58 %
             1%
           \else
             9%
           \fi
&        \else\ifx.#2%
$        \else\ifnum`#2=46 % .
           1%
&        \else\ifx-#2%
$        \else\ifnum`#2=45 % -
           2%
&        \else\ifx+#2%
$        \else\ifnum`#2=43 % +
           3%
         \else
           9%
         \fi\fi\fi\fi\fi\ThPdNu@space
    \expandafter\ThPdNu@zero
    \expandafter0%
    \expandafter\thepdfnumber@ErrorUnexpectedEnd
  \or
    \ThPdNu@ReturnAndSkip{0}%
  \or
    \ThPbNu@StateN@Plus
  \or
    \ThPbNu@StateN@Minus
  \else
    \ThPdNu@ReturnError{0}%
  \fi
}
%    \end{macrocode}
%    \end{macro}
%
%    \begin{macro}{\ThPdNu@ReturnAndSkip}
%    \begin{macrocode}
\def\ThPdNu@ReturnAndSkip#1#2\fi#3\ThPdNu@FIN{%
  \fi
  \ThPdNu@zero
  #1%
}
%    \end{macrocode}
%    \end{macro}
%
% \subsubsection{State definitions for integer part}
%
%    \begin{macro}{\ThPbNu@StateN@SkipZeros}
%    \begin{macrocode}
\def\ThPbNu@StateN@SkipZeros#1\fi#2{%
  \fi
  \ifcase\ifx\ThPdNu@FIN#2%
           0%
         \else\ifx0#2%
           1%
         \else\ifnum`#2>48 %
           \ifnum`#2<58 %
             2%
           \else
             9%
           \fi
&        \else\ifx.#2%
$        \else\ifnum`#2=46 % .
           3%
         \else
           9%
         \fi\fi\fi\fi\ThPdNu@space
    \expandafter\ThPdNu@zero
    \expandafter0%
  \or
   \ThPbNu@StateN@SkipZeros%
  \or
    \ThPdNu@ReturnAndSkip{1}%
  \or
    \ThPbNu@StateN@Dot\ThPdNu@zero!!%
  \else
    \ThPdNu@ReturnError{0}%
  \fi
}
%    \end{macrocode}
%    \end{macro}
%
% \subsubsection{State definitions for decimal digits}
%
%    \begin{macro}{\ThPbNu@StateN@Dot}
%    \begin{macrocode}
\def\ThPbNu@StateN@Dot#1#2!#3!#4\fi#5{%
  \fi
  \ifcase\ifx\ThPdNu@FIN#5%
           0%
         \else\ifnum`#5>48 %
           \ifnum`#5<58 %
             1%
           \else
             9%
           \fi
         \else\ifx0#5%
           2%
         \else
           9%
         \fi\fi\fi\ThPdNu@space
    \ThPdNu@ReturnFracNumber#1#2!%
  \or
    \ThPbNu@StateN@Dot\ThPdNu@one#2#3#5!!%
  \or
    \ThPbNu@StateN@DotZero#1#2!#3#5!%
  \else
    \ThPdNu@ReturnFracNumberInvalid#1#2!%
  \fi
}
%    \end{macrocode}
%    \end{macro}
%
%    \begin{macro}{\ThPbNu@StateN@DotZero}
%    \begin{macrocode}
\def\ThPbNu@StateN@DotZero#1#2!#3!#4\fi#5{%
  \fi
  \ifcase\ifx\ThPdNu@FIN#5%
           0%
         \else\ifnum`#5>48 %
           \ifnum`#5<58 %
             1%
           \else
             9%
           \fi
         \else\ifx0#5%
           2%
         \else
           9%
         \fi\fi\fi\ThPdNu@space
    \ThPdNu@ReturnFracNumber#1#2!%
  \or
    \ThPbNu@StateN@Dot\ThPdNu@one#2#3#5!!%
  \or
    \ThPbNu@StateN@DotZero#1#2!#3#5!%
  \else
    \ThPdNu@ReturnFracNumberInvalid#1#2!%
  \fi
}
%    \end{macrocode}
%    \end{macro}
%
%    \begin{macro}{\ThPdNu@ReturnFracNumber}
%    \begin{macrocode}
\def\ThPdNu@ReturnFracNumber#1#2!#3\fi{%
  \fi
  \ifcase#1%
    \expandafter\ThPdNu@firstoftwo
  \else
    \expandafter\ThPdNu@secondoftwo
  \fi
  {\ThPdNu@zero 0}%
  {\ThPdNu@zero .#2}%
}
%    \end{macrocode}
%    \end{macro}
%
%    \begin{macro}{\ThPdNu@ReturnFracNumberInvalid}
%    \begin{macrocode}
\def\ThPdNu@ReturnFracNumberInvalid#1!#2\fi#3\ThPdNu@FIN{%
  \fi
  \iftrue\ThPdNu@ReturnFracNumber#1!\fi
  \thepdfnumber@ErrorInvalidToken
}
%    \end{macrocode}
%    \end{macro}
%
%    \begin{macrocode}
\ThPdNu@AtEnd%
%</package>
%    \end{macrocode}
%
% \section{Test}
%
% \subsection{Catcode checks for loading}
%
%    \begin{macrocode}
%<*test1>
%    \end{macrocode}
%    \begin{macrocode}
\catcode`\{=1 %
\catcode`\}=2 %
\catcode`\#=6 %
\catcode`\@=11 %
\expandafter\ifx\csname count@\endcsname\relax
  \countdef\count@=255 %
\fi
\expandafter\ifx\csname @gobble\endcsname\relax
  \long\def\@gobble#1{}%
\fi
\expandafter\ifx\csname @firstofone\endcsname\relax
  \long\def\@firstofone#1{#1}%
\fi
\expandafter\ifx\csname loop\endcsname\relax
  \expandafter\@firstofone
\else
  \expandafter\@gobble
\fi
{%
  \def\loop#1\repeat{%
    \def\body{#1}%
    \iterate
  }%
  \def\iterate{%
    \body
      \let\next\iterate
    \else
      \let\next\relax
    \fi
    \next
  }%
  \let\repeat=\fi
}%
\def\RestoreCatcodes{}
\count@=0 %
\loop
  \edef\RestoreCatcodes{%
    \RestoreCatcodes
    \catcode\the\count@=\the\catcode\count@\relax
  }%
\ifnum\count@<255 %
  \advance\count@ 1 %
\repeat

\def\RangeCatcodeInvalid#1#2{%
  \count@=#1\relax
  \loop
    \catcode\count@=15 %
  \ifnum\count@<#2\relax
    \advance\count@ 1 %
  \repeat
}
\def\RangeCatcodeCheck#1#2#3{%
  \count@=#1\relax
  \loop
    \ifnum#3=\catcode\count@
    \else
      \errmessage{%
        Character \the\count@\space
        with wrong catcode \the\catcode\count@\space
        instead of \number#3%
      }%
    \fi
  \ifnum\count@<#2\relax
    \advance\count@ 1 %
  \repeat
}
\def\space{ }
\expandafter\ifx\csname LoadCommand\endcsname\relax
  \def\LoadCommand{\input thepdfnumber.sty\relax}%
\fi
\def\Test{%
  \RangeCatcodeInvalid{0}{47}%
  \RangeCatcodeInvalid{58}{64}%
  \RangeCatcodeInvalid{91}{96}%
  \RangeCatcodeInvalid{123}{255}%
  \catcode`\@=12 %
  \catcode`\\=0 %
  \catcode`\%=14 %
  \LoadCommand
  \RangeCatcodeCheck{0}{36}{15}%
  \RangeCatcodeCheck{37}{37}{14}%
  \RangeCatcodeCheck{38}{47}{15}%
  \RangeCatcodeCheck{48}{57}{12}%
  \RangeCatcodeCheck{58}{63}{15}%
  \RangeCatcodeCheck{64}{64}{12}%
  \RangeCatcodeCheck{65}{90}{11}%
  \RangeCatcodeCheck{91}{91}{15}%
  \RangeCatcodeCheck{92}{92}{0}%
  \RangeCatcodeCheck{93}{96}{15}%
  \RangeCatcodeCheck{97}{122}{11}%
  \RangeCatcodeCheck{123}{255}{15}%
  \RestoreCatcodes
}
\Test
\csname @@end\endcsname
\end
%    \end{macrocode}
%    \begin{macrocode}
%</test1>
%    \end{macrocode}
%
% \subsection{Macro tests}
%
% \subsubsection{Preamble}
%
%    \begin{macrocode}
%<*test2>
\catcode`\{=1
\catcode`\}=2
\catcode`\#=6
\catcode`\@=11
\errorcontextlines=10000 %
\def\msg#{\immediate\write16}
\def\space{ }
\begingroup\expandafter\expandafter\expandafter\endgroup
\expandafter\ifx\csname RequirePackage\endcsname\relax
  \input thepdfnumber.sty\relax
\else
  \RequirePackage{thepdfnumber}[2016/05/16]%
\fi
%    \end{macrocode}
%
%    \begin{macro}{\Test}
%    \begin{macrocode}
\def\Test#1#2{%
  \def\TestExpected{#2}%
%<*active>
  \edef\TestExpected{\TestExpected}%
%</active>
  \expandafter\expandafter\expandafter\def
  \expandafter\expandafter\expandafter\TestResult
  \expandafter\expandafter\expandafter{%
    \TestCommand{#1}%
  }%
  \begingroup
    \toks0\expandafter{\TestResult}%
    \toks2\expandafter{\TestExpected}%
    \ifx\TestResult\TestExpected
      \msg{* OK: #1 => \the\toks0}%
    \else
      \errmessage{FAILED: #1 => \the\toks0\space(\the\toks2)}%
    \fi
  \endgroup
}
%    \end{macrocode}
%    \end{macro}
%
%    \begin{macrocode}
\let\thepdfnumber@ErrorUnexpectedEnd\relax
\let\thepdfnumber@ErrorInvalidToken\relax
%    \end{macrocode}
%
% \subsubsection{Test with active catcodes}
%
%    \begin{macrocode}
%<*active>
\catcode`\~=13 %
\def\ThPdNu@Temp#1{%
  \begingroup
    \lccode`\~=`#1 %
  \lowercase{\endgroup
    \edef~%
  }{\string#1}%
  \catcode`#1=13 %
}
\ThPdNu@Temp{.}
\ThPdNu@Temp{-}
\ThPdNu@Temp{+}
%</active>
%    \end{macrocode}
%
% \subsubsection{Test cases for \cs{thepdfnumber}}
%
%    \begin{macro}{\TestCommand}
%    \begin{macrocode}
\let\TestCommand\thepdfnumber
%    \end{macrocode}
%    \end{macro}
%
%    \begin{macrocode}
\Test{0}{0}
\Test{1}{1}
\Test{23}{23}
\Test{0045}{45}
\Test{4500}{4500}
\Test{-0}{0}
\Test{-1}{-1}
\Test{-00002}{-2}
\Test{-010203}{-10203}
\Test{+0}{0}
\Test{+123}{123}

\Test{+-4}{-4}
\Test{+-+-4}{4}
\Test{++--++--++4}{4}
\Test{+++---+++---+++---4}{-4}

\Test{.}{0}
\Test{0.}{0}
\Test{.0}{0}
\Test{0.}{0}
\Test{-.}{0}
\Test{-0.}{0}
\Test{-.0}{0}
\Test{-0.}{0}
\Test{010.020}{10.02}
\Test{123.456}{123.456}
\Test{12.}{12}
\Test{.123}{.123}
\Test{.0001}{.0001}
\Test{.00100}{.001}
\Test{.12003400560078009}{.12003400560078009}
\Test{-.12003400560078009}{-.12003400560078009}
\Test{04.0500000}{4.05}

\Test{}{0\thepdfnumber@ErrorUnexpectedEnd}
\Test{+}{0\thepdfnumber@ErrorUnexpectedEnd}
\Test{-}{0\thepdfnumber@ErrorUnexpectedEnd}
\Test{a}{0\thepdfnumber@ErrorInvalidToken}
\Test{0x1}{0\thepdfnumber@ErrorInvalidToken}
\Test{4x56}{4\thepdfnumber@ErrorInvalidToken}
\Test{012,34}{12\thepdfnumber@ErrorInvalidToken}
\Test{0.12x4}{.12\thepdfnumber@ErrorInvalidToken}
%    \end{macrocode}
%
% \subsubsection{Test cases for \cs{thepdfnumberNormZeroOne}}
%
%    \begin{macro}{\TestCommand}
%    \begin{macrocode}
\let\TestCommand\thepdfnumberNormZeroOne
%    \end{macrocode}
%    \end{macro}
%
%    \begin{macrocode}
\Test{0}{0}
\Test{1}{1}
\Test{23}{1}
\Test{12}{1}
\Test{10}{1}
\Test{0045}{1}
\Test{001}{1}
\Test{-0}{0}
\Test{-1}{0}
\Test{-01}{0}
\Test{-4}{0}

\Test{+-0}{0}
\Test{+-+-1}{1}
\Test{++--++--++1}{1}
\Test{+++---+++---+++---1}{0}

\Test{.}{0}
\Test{0.}{0}
\Test{.0}{0}
\Test{0.}{0}
\Test{-.}{0}
\Test{-0.}{0}
\Test{-.0}{0}
\Test{-0.}{0}
\Test{010.020}{1}
\Test{123.456}{1}
\Test{12.}{1}
\Test{.123}{.123}
\Test{.0001}{.0001}
\Test{.00100}{.001}
\Test{.12003400560078009}{.12003400560078009}
\Test{-.12003400560078009}{0}
\Test{04.0500000}{1}
\Test{0.1200340056}{.1200340056}
\Test{1.05}{1}

\Test{}{0\thepdfnumber@ErrorUnexpectedEnd}
\Test{+}{0\thepdfnumber@ErrorUnexpectedEnd}
\Test{-}{0\thepdfnumber@ErrorUnexpectedEnd}
\Test{a}{0\thepdfnumber@ErrorInvalidToken}
\Test{0x1}{0\thepdfnumber@ErrorInvalidToken}
\Test{4x56}{1}
\Test{012,34}{1}
\Test{.012x4}{.012\thepdfnumber@ErrorInvalidToken}
%    \end{macrocode}
%
%    \begin{macrocode}
\csname @@end\endcsname\end
%</test2>
%    \end{macrocode}
%
% \section{Installation}
%
% \subsection{Download}
%
% \paragraph{Package.} This package is available on
% CTAN\footnote{\url{http://ctan.org/pkg/thepdfnumber}}:
% \begin{description}
% \item[\CTAN{macros/latex/contrib/oberdiek/thepdfnumber.dtx}] The source file.
% \item[\CTAN{macros/latex/contrib/oberdiek/thepdfnumber.pdf}] Documentation.
% \end{description}
%
%
% \paragraph{Bundle.} All the packages of the bundle `oberdiek'
% are also available in a TDS compliant ZIP archive. There
% the packages are already unpacked and the documentation files
% are generated. The files and directories obey the TDS standard.
% \begin{description}
% \item[\CTAN{install/macros/latex/contrib/oberdiek.tds.zip}]
% \end{description}
% \emph{TDS} refers to the standard ``A Directory Structure
% for \TeX\ Files'' (\CTAN{tds/tds.pdf}). Directories
% with \xfile{texmf} in their name are usually organized this way.
%
% \subsection{Bundle installation}
%
% \paragraph{Unpacking.} Unpack the \xfile{oberdiek.tds.zip} in the
% TDS tree (also known as \xfile{texmf} tree) of your choice.
% Example (linux):
% \begin{quote}
%   |unzip oberdiek.tds.zip -d ~/texmf|
% \end{quote}
%
% \paragraph{Script installation.}
% Check the directory \xfile{TDS:scripts/oberdiek/} for
% scripts that need further installation steps.
% Package \xpackage{attachfile2} comes with the Perl script
% \xfile{pdfatfi.pl} that should be installed in such a way
% that it can be called as \texttt{pdfatfi}.
% Example (linux):
% \begin{quote}
%   |chmod +x scripts/oberdiek/pdfatfi.pl|\\
%   |cp scripts/oberdiek/pdfatfi.pl /usr/local/bin/|
% \end{quote}
%
% \subsection{Package installation}
%
% \paragraph{Unpacking.} The \xfile{.dtx} file is a self-extracting
% \docstrip\ archive. The files are extracted by running the
% \xfile{.dtx} through \plainTeX:
% \begin{quote}
%   \verb|tex thepdfnumber.dtx|
% \end{quote}
%
% \paragraph{TDS.} Now the different files must be moved into
% the different directories in your installation TDS tree
% (also known as \xfile{texmf} tree):
% \begin{quote}
% \def\t{^^A
% \begin{tabular}{@{}>{\ttfamily}l@{ $\rightarrow$ }>{\ttfamily}l@{}}
%   thepdfnumber.sty & tex/generic/oberdiek/thepdfnumber.sty\\
%   thepdfnumber.pdf & doc/latex/oberdiek/thepdfnumber.pdf\\
%   test/thepdfnumber-test1.tex & doc/latex/oberdiek/test/thepdfnumber-test1.tex\\
%   test/thepdfnumber-test2.tex & doc/latex/oberdiek/test/thepdfnumber-test2.tex\\
%   test/thepdfnumber-test3.tex & doc/latex/oberdiek/test/thepdfnumber-test3.tex\\
%   thepdfnumber.dtx & source/latex/oberdiek/thepdfnumber.dtx\\
% \end{tabular}^^A
% }^^A
% \sbox0{\t}^^A
% \ifdim\wd0>\linewidth
%   \begingroup
%     \advance\linewidth by\leftmargin
%     \advance\linewidth by\rightmargin
%   \edef\x{\endgroup
%     \def\noexpand\lw{\the\linewidth}^^A
%   }\x
%   \def\lwbox{^^A
%     \leavevmode
%     \hbox to \linewidth{^^A
%       \kern-\leftmargin\relax
%       \hss
%       \usebox0
%       \hss
%       \kern-\rightmargin\relax
%     }^^A
%   }^^A
%   \ifdim\wd0>\lw
%     \sbox0{\small\t}^^A
%     \ifdim\wd0>\linewidth
%       \ifdim\wd0>\lw
%         \sbox0{\footnotesize\t}^^A
%         \ifdim\wd0>\linewidth
%           \ifdim\wd0>\lw
%             \sbox0{\scriptsize\t}^^A
%             \ifdim\wd0>\linewidth
%               \ifdim\wd0>\lw
%                 \sbox0{\tiny\t}^^A
%                 \ifdim\wd0>\linewidth
%                   \lwbox
%                 \else
%                   \usebox0
%                 \fi
%               \else
%                 \lwbox
%               \fi
%             \else
%               \usebox0
%             \fi
%           \else
%             \lwbox
%           \fi
%         \else
%           \usebox0
%         \fi
%       \else
%         \lwbox
%       \fi
%     \else
%       \usebox0
%     \fi
%   \else
%     \lwbox
%   \fi
% \else
%   \usebox0
% \fi
% \end{quote}
% If you have a \xfile{docstrip.cfg} that configures and enables \docstrip's
% TDS installing feature, then some files can already be in the right
% place, see the documentation of \docstrip.
%
% \subsection{Refresh file name databases}
%
% If your \TeX~distribution
% (\teTeX, \mikTeX, \dots) relies on file name databases, you must refresh
% these. For example, \teTeX\ users run \verb|texhash| or
% \verb|mktexlsr|.
%
% \subsection{Some details for the interested}
%
% \paragraph{Attached source.}
%
% The PDF documentation on CTAN also includes the
% \xfile{.dtx} source file. It can be extracted by
% AcrobatReader 6 or higher. Another option is \textsf{pdftk},
% e.g. unpack the file into the current directory:
% \begin{quote}
%   \verb|pdftk thepdfnumber.pdf unpack_files output .|
% \end{quote}
%
% \paragraph{Unpacking with \LaTeX.}
% The \xfile{.dtx} chooses its action depending on the format:
% \begin{description}
% \item[\plainTeX:] Run \docstrip\ and extract the files.
% \item[\LaTeX:] Generate the documentation.
% \end{description}
% If you insist on using \LaTeX\ for \docstrip\ (really,
% \docstrip\ does not need \LaTeX), then inform the autodetect routine
% about your intention:
% \begin{quote}
%   \verb|latex \let\install=y% \iffalse meta-comment
%
% File: thepdfnumber.dtx
% Version: 2016/05/16 v1.1
% Info: Print PDF numbers with minimal digits
%
% Copyright (C) 2011 by
%    Heiko Oberdiek <heiko.oberdiek at googlemail.com>
%    2016
%    https://github.com/ho-tex/oberdiek/issues
%
% This work may be distributed and/or modified under the
% conditions of the LaTeX Project Public License, either
% version 1.3c of this license or (at your option) any later
% version. This version of this license is in
%    http://www.latex-project.org/lppl/lppl-1-3c.txt
% and the latest version of this license is in
%    http://www.latex-project.org/lppl.txt
% and version 1.3 or later is part of all distributions of
% LaTeX version 2005/12/01 or later.
%
% This work has the LPPL maintenance status "maintained".
%
% This Current Maintainer of this work is Heiko Oberdiek.
%
% The Base Interpreter refers to any `TeX-Format',
% because some files are installed in TDS:tex/generic//.
%
% This work consists of the main source file thepdfnumber.dtx
% and the derived files
%    thepdfnumber.sty, thepdfnumber.pdf, thepdfnumber.ins, thepdfnumber.drv,
%    thepdfnumber-test1.tex, thepdfnumber-test2.tex,
%    thepdfnumber-test3.tex.
%
% Distribution:
%    CTAN:macros/latex/contrib/oberdiek/thepdfnumber.dtx
%    CTAN:macros/latex/contrib/oberdiek/thepdfnumber.pdf
%
% Unpacking:
%    (a) If thepdfnumber.ins is present:
%           tex thepdfnumber.ins
%    (b) Without thepdfnumber.ins:
%           tex thepdfnumber.dtx
%    (c) If you insist on using LaTeX
%           latex \let\install=y\input{thepdfnumber.dtx}
%        (quote the arguments according to the demands of your shell)
%
% Documentation:
%    (a) If thepdfnumber.drv is present:
%           latex thepdfnumber.drv
%    (b) Without thepdfnumber.drv:
%           latex thepdfnumber.dtx; ...
%    The class ltxdoc loads the configuration file ltxdoc.cfg
%    if available. Here you can specify further options, e.g.
%    use A4 as paper format:
%       \PassOptionsToClass{a4paper}{article}
%
%    Programm calls to get the documentation (example):
%       pdflatex thepdfnumber.dtx
%       makeindex -s gind.ist thepdfnumber.idx
%       pdflatex thepdfnumber.dtx
%       makeindex -s gind.ist thepdfnumber.idx
%       pdflatex thepdfnumber.dtx
%
% Installation:
%    TDS:tex/generic/oberdiek/thepdfnumber.sty
%    TDS:doc/latex/oberdiek/thepdfnumber.pdf
%    TDS:doc/latex/oberdiek/test/thepdfnumber-test1.tex
%    TDS:doc/latex/oberdiek/test/thepdfnumber-test2.tex
%    TDS:doc/latex/oberdiek/test/thepdfnumber-test3.tex
%    TDS:source/latex/oberdiek/thepdfnumber.dtx
%
%<*ignore>
\begingroup
  \catcode123=1 %
  \catcode125=2 %
  \def\x{LaTeX2e}%
\expandafter\endgroup
\ifcase 0\ifx\install y1\fi\expandafter
         \ifx\csname processbatchFile\endcsname\relax\else1\fi
         \ifx\fmtname\x\else 1\fi\relax
\else\csname fi\endcsname
%</ignore>
%<*install>
\input docstrip.tex
\Msg{************************************************************************}
\Msg{* Installation}
\Msg{* Package: thepdfnumber 2016/05/16 v1.1 Print PDF numbers with minimal digits (HO)}
\Msg{************************************************************************}

\keepsilent
\askforoverwritefalse

\let\MetaPrefix\relax
\preamble

This is a generated file.

Project: thepdfnumber
Version: 2016/05/16 v1.1

Copyright (C) 2011 by
   Heiko Oberdiek <heiko.oberdiek at googlemail.com>

This work may be distributed and/or modified under the
conditions of the LaTeX Project Public License, either
version 1.3c of this license or (at your option) any later
version. This version of this license is in
   http://www.latex-project.org/lppl/lppl-1-3c.txt
and the latest version of this license is in
   http://www.latex-project.org/lppl.txt
and version 1.3 or later is part of all distributions of
LaTeX version 2005/12/01 or later.

This work has the LPPL maintenance status "maintained".

This Current Maintainer of this work is Heiko Oberdiek.

The Base Interpreter refers to any `TeX-Format',
because some files are installed in TDS:tex/generic//.

This work consists of the main source file thepdfnumber.dtx
and the derived files
   thepdfnumber.sty, thepdfnumber.pdf, thepdfnumber.ins, thepdfnumber.drv,
   thepdfnumber-test1.tex, thepdfnumber-test2.tex,
   thepdfnumber-test3.tex.

\endpreamble
\let\MetaPrefix\DoubleperCent

\generate{%
  \file{thepdfnumber.ins}{\from{thepdfnumber.dtx}{install}}%
  \file{thepdfnumber.drv}{\from{thepdfnumber.dtx}{driver}}%
  \usedir{tex/generic/oberdiek}%
  \file{thepdfnumber.sty}{\from{thepdfnumber.dtx}{package}}%
  \usedir{doc/latex/oberdiek/test}%
  \file{thepdfnumber-test1.tex}{\from{thepdfnumber.dtx}{test1}}%
  \file{thepdfnumber-test2.tex}{\from{thepdfnumber.dtx}{test2}}%
  \file{thepdfnumber-test3.tex}{\from{thepdfnumber.dtx}{test2,active}}%
}

\catcode32=13\relax% active space
\let =\space%
\Msg{************************************************************************}
\Msg{*}
\Msg{* To finish the installation you have to move the following}
\Msg{* file into a directory searched by TeX:}
\Msg{*}
\Msg{*     thepdfnumber.sty}
\Msg{*}
\Msg{* To produce the documentation run the file `thepdfnumber.drv'}
\Msg{* through LaTeX.}
\Msg{*}
\Msg{* Happy TeXing!}
\Msg{*}
\Msg{************************************************************************}

\endbatchfile
%</install>
%<*ignore>
\fi
%</ignore>
%<*driver>
\NeedsTeXFormat{LaTeX2e}
\ProvidesFile{thepdfnumber.drv}%
  [2016/05/16 v1.1 Print PDF numbers with minimal digits (HO)]%
\documentclass{ltxdoc}
\usepackage{holtxdoc}[2011/11/22]
\usepackage{csquotes}
\usepackage[
  backend=bibtex,
  bibencoding=ascii,
  alldates=iso8601,
]{biblatex}[2011/11/13]
\bibliography{oberdiek-source}
\begin{document}
  \DocInput{thepdfnumber.dtx}%
\end{document}
%</driver>
% \fi
%
%
% \CharacterTable
%  {Upper-case    \A\B\C\D\E\F\G\H\I\J\K\L\M\N\O\P\Q\R\S\T\U\V\W\X\Y\Z
%   Lower-case    \a\b\c\d\e\f\g\h\i\j\k\l\m\n\o\p\q\r\s\t\u\v\w\x\y\z
%   Digits        \0\1\2\3\4\5\6\7\8\9
%   Exclamation   \!     Double quote  \"     Hash (number) \#
%   Dollar        \$     Percent       \%     Ampersand     \&
%   Acute accent  \'     Left paren    \(     Right paren   \)
%   Asterisk      \*     Plus          \+     Comma         \,
%   Minus         \-     Point         \.     Solidus       \/
%   Colon         \:     Semicolon     \;     Less than     \<
%   Equals        \=     Greater than  \>     Question mark \?
%   Commercial at \@     Left bracket  \[     Backslash     \\
%   Right bracket \]     Circumflex    \^     Underscore    \_
%   Grave accent  \`     Left brace    \{     Vertical bar  \|
%   Right brace   \}     Tilde         \~}
%
% \GetFileInfo{thepdfnumber.drv}
%
% \title{The \xpackage{thepdfnumber} package}
% \date{2016/05/16 v1.1}
% \author{Heiko Oberdiek\thanks
% {Please report any issues at https://github.com/ho-tex/oberdiek/issues}\\
% \xemail{heiko.oberdiek at googlemail.com}}
%
% \maketitle
%
% \begin{abstract}
% The package converts real numbers to a minimal representation
% that is stripped from leading or trailing zeros,
% plus signs and decimal point if not necessary.
% \end{abstract}
%
% \tableofcontents
%
% \section{Documentation}
%
% \subsection{Introduction}
%
% Dealing with the PDF format, there is sometimes the need to write
% some low level PDF stuff. In case of numbers, the numbers can
% arise from user input (e.g. color or transparency specifications)
% or can be calculated. For example, \hologo{LaTeX}'s \cs{strip@pt}
% makes a good job to output a real number. It automatically suppresses
% the decimal part if the number is an integer. However it leaves
% a leading zero for numbers greater zero and smaller one.
% Thus the package provides macros that can be used with different
% formats, even with \hologo{iniTeX} and generates numbers that
% are valid numbers of the PDF format and whose length is minimal.
%
% \subsection{Usage}
%
% The package \xpackage{thepdfnumber} can be used with \hologo{LaTeX},
% \hologo{plainTeX} or even with \hologo{iniTeX}:
% \begin{quote}
%   |\RequirePackage{thepdfnumber}| \textit{\% \hologo{LaTeX}}\\
%   |\input thepdfnumber.sty| \textit{\% \hologo{plainTeX}/\hologo{iniTeX}}
% \end{quote}
% The package does not need and have package options.
%
% \subsection{User macros}
%
% All user macros are expandable in exact two expansion steps.
%
% \begin{declcs}{thepdfnumber} \M{number}
% \end{declcs}
% Macro \cs{thepdfnumber} takes a number
% as argument and expands to a minimal representation of that number.
% Some examples:
% \begin{quote}
% \begin{tabular}{r@{.: }>{\ttfamily}l@{ $\rightarrow$\ }>{\ttfamily}l@{}}
%   1& +123 & 123\\
%   2& --123 & 123\\
%   3& -01 & -1\\
%   4& 0045 & 45\\
%   5& 1.0 & 1\\
%   6& 1.20 & 1.2\\
%   7& 0.0 & 0\\
%   8& 0.78 & .78\\
%   9& +012.340 & 12.34\\
% \end{tabular}
% \end{quote}
% It reduces the length of the number representation:
% \begin{itemize}
% \item The signs are collapsed and only one minus sign
%   is output if the number is negative (see examples 1, 2, 3, 9).
% \item Leading zeros are removed (4, 8, 9) unless the number is zero (7).
% \item The decimal part is omitted, if the number is an integer (5, 7).
% \item Trailing zeros from the decimal part are stripped (5, 6, 7, 9).
% \end{itemize}
% The resulting number representation can be catched with one of the
% following Perl regular expressions:
% \begin{itemize}
% \item |^0$| (zero)
% \item |^-?[1-9][0-9]*$| (integer)
% \item |^-?[0-9]*\.[0-9]*[1-9]$| (real)
% \end{itemize}
% This is a valid numeric object of the PDF specification \cite[``7.3.3~Numeric Objects'']{pdfspec-iso32000-1}.
%
% \begin{declcs}{thepdfnumberNormZeroOne}
% \end{declcs}
% There are various places in the PDF specification where the
% number is in the domain 0.0 upto 1.0. Macro thepdfnumberNormZeroOne
% automatically adjusts the number to fit into that range. Negative
% numbers are mapped to |0| and numbers greater than one are
% replaced by |1|. Thus the result fits one of the following regular
% expressions:
% \begin{itemize}
% \item |^0$|
% \item |^\.[0-9]*[1-9]$|
% \item |^1$|
% \end{itemize}
% Examples:
% \begin{quote}
% \begin{tabular}{@{}>{\ttfamily}l@{ $\rightarrow$\ }>{\ttfamily}l@{}}
%   -456 & 0\\
%   -0.001 & 0\\
%   0.0 & 0\\
%   0.010 & .01\\
%   0.456 & .456\\
%   1.0 & 1\\
%   01.001 & 1\\
%   4 & 1\\
% \end{tabular}
% \end{quote}
%
% \subsection{Input number}
%
% The user macros expect a number as argument. The number can either
% be given explicitely or as macro that expands in one step
% to an explicite number, because the first token of the argument
% is expanded once.
%
% The explicite number consists of
% \begin{itemize}
% \item optional signs `\texttt{+}' and `\texttt{-}',
% \item digits `\texttt{0}' upto `\texttt{9}' and
% \item an optional dot `\texttt{.}'.
% \end{itemize}
% All tokens must have catcode 12 (other), the default catcodes
% for these characters in \hologo{LaTeX}, \hologo{plainTeX} or
% \hologo{iniTeX}. As Perl regular expression the number is expected
% in one of the following forms:
% \begin{itemize}
% \item |^[+-]*[0-9]+$|
% \item |^[+-]*[0-9]*\.[0-9]*$|
% \end{itemize}
% At least one digit or the dot must be present.
%
% \subsection{Error handling}
%
% The package is not intended for validating numbers or to
% decide if an argument is a number. Therefore it is an usage
% error to use the user macros with arguments that are not
% explicite numbers as specified in the previous sections.
% Nevertheless some error conditions are sometimes recognized.
% Errors are given in form of an undefined command sequence.
% It is the only way to notify \hologo{TeX} in expandable context.
% Expanding to some error text would invalidate the output.
% Currently the following errors are thrown:
% \begin{description}
% \item[\cs{thepdfnumber@ErrorEndMarker}:]
%   Internally the argument parsing uses an end marker that
%   is never called directly. If it is called with valid
%   user input, then this is a bug. Otherwise it means the user input
%   contains nasty stuff.
% \item[\cs{thepdfnumber@ErrorUnexpectedEnd}:] The macros expect at least one digit
%   or the dot, otherwise if the argument is empty or only contains
%   signs, then this error is called.
% \item[\cs{thepdfnumber@ErrorInvalidToken}:] It is called if the number
%   contains other tokens than signs, digits or the dot or the
%   token at the wrong place (e.g. a sign after a digit).
%   In case of \cs{thepdfnumberNormZeroOne} this error condition might
%   not always be detected, because the number parsing might stop
%   at an early point, when the result is already clear (e.g.
%   if the number is negative or will be greater than one).
% \item[Improper alphabetic constant:] This error might be
%   thrown by \hologo{TeX}, if the number contains
%   command tokens instead of characters.
% \end{description}
%
% \StopEventually{
% }
%
% \section{Implementation}
%
%    \begin{macrocode}
%<*package>
%    \end{macrocode}
%
% \subsection{Reload check and package identification}
%    Reload check, especially if the package is not used with \LaTeX.
%    \begin{macrocode}
\begingroup\catcode61\catcode48\catcode32=10\relax%
  \catcode13=5 % ^^M
  \endlinechar=13 %
  \catcode35=6 % #
  \catcode39=12 % '
  \catcode44=12 % ,
  \catcode45=12 % -
  \catcode46=12 % .
  \catcode58=12 % :
  \catcode64=11 % @
  \catcode123=1 % {
  \catcode125=2 % }
  \expandafter\let\expandafter\x\csname ver@thepdfnumber.sty\endcsname
  \ifx\x\relax % plain-TeX, first loading
  \else
    \def\empty{}%
    \ifx\x\empty % LaTeX, first loading,
      % variable is initialized, but \ProvidesPackage not yet seen
    \else
      \expandafter\ifx\csname PackageInfo\endcsname\relax
        \def\x#1#2{%
          \immediate\write-1{Package #1 Info: #2.}%
        }%
      \else
        \def\x#1#2{\PackageInfo{#1}{#2, stopped}}%
      \fi
      \x{thepdfnumber}{The package is already loaded}%
      \aftergroup\endinput
    \fi
  \fi
\endgroup%
%    \end{macrocode}
%    Package identification:
%    \begin{macrocode}
\begingroup\catcode61\catcode48\catcode32=10\relax%
  \catcode13=5 % ^^M
  \endlinechar=13 %
  \catcode35=6 % #
  \catcode39=12 % '
  \catcode40=12 % (
  \catcode41=12 % )
  \catcode44=12 % ,
  \catcode45=12 % -
  \catcode46=12 % .
  \catcode47=12 % /
  \catcode58=12 % :
  \catcode64=11 % @
  \catcode91=12 % [
  \catcode93=12 % ]
  \catcode123=1 % {
  \catcode125=2 % }
  \expandafter\ifx\csname ProvidesPackage\endcsname\relax
    \def\x#1#2#3[#4]{\endgroup
      \immediate\write-1{Package: #3 #4}%
      \xdef#1{#4}%
    }%
  \else
    \def\x#1#2[#3]{\endgroup
      #2[{#3}]%
      \ifx#1\@undefined
        \xdef#1{#3}%
      \fi
      \ifx#1\relax
        \xdef#1{#3}%
      \fi
    }%
  \fi
\expandafter\x\csname ver@thepdfnumber.sty\endcsname
\ProvidesPackage{thepdfnumber}%
  [2016/05/16 v1.1 Print PDF numbers with minimal digits (HO)]%
%    \end{macrocode}
%
% \subsection{Catcodes}
%
%    \begin{macrocode}
\begingroup\catcode61\catcode48\catcode32=10\relax%
  \catcode13=5 % ^^M
  \endlinechar=13 %
  \catcode123=1 % {
  \catcode125=2 % }
  \catcode64=11 % @
  \def\x{\endgroup
    \expandafter\edef\csname ThPdNu@AtEnd\endcsname{%
      \endlinechar=\the\endlinechar\relax
      \catcode13=\the\catcode13\relax
      \catcode32=\the\catcode32\relax
      \catcode35=\the\catcode35\relax
      \catcode61=\the\catcode61\relax
      \catcode64=\the\catcode64\relax
      \catcode123=\the\catcode123\relax
      \catcode125=\the\catcode125\relax
    }%
  }%
\x\catcode61\catcode48\catcode32=10\relax%
\catcode13=5 % ^^M
\endlinechar=13 %
\catcode35=6 % #
\catcode64=11 % @
\catcode123=1 % {
\catcode125=2 % }
\def\TMP@EnsureCode#1#2{%
  \edef\ThPdNu@AtEnd{%
    \ThPdNu@AtEnd
    \catcode#1=\the\catcode#1\relax
  }%
  \catcode#1=#2\relax
}
\TMP@EnsureCode{33}{12}% !
\TMP@EnsureCode{36}{3}% $
\TMP@EnsureCode{38}{4}% &
\TMP@EnsureCode{42}{12}% *
\TMP@EnsureCode{43}{12}% +
\TMP@EnsureCode{45}{12}% -
\TMP@EnsureCode{46}{12}% .
\TMP@EnsureCode{60}{12}% <
\TMP@EnsureCode{62}{12}% >
\TMP@EnsureCode{96}{12}% `
\edef\ThPdNu@AtEnd{\ThPdNu@AtEnd\noexpand\endinput}
%    \end{macrocode}
%
% \subsection{Helper macros}
%
%    \begin{macro}{\ThPdNu@FIN}
%    \begin{macrocode}
\def\ThPdNu@FIN{\thepdfnumber@ErrorEndMarker}
%    \end{macrocode}
%    \end{macro}
%
%    \begin{macro}{\ThPdNu@space}
%    \begin{macrocode}
\def\ThPdNu@space{ }
%    \end{macrocode}
%    \end{macro}
%
%    \begin{macro}{\ThPdNu@zero}
%    \begin{macrocode}
\chardef\ThPdNu@zero=0 %
%    \end{macrocode}
%    \end{macro}
%
%    \begin{macro}{\ThPdNu@one}
%    \begin{macrocode}
\chardef\ThPdNu@one=1 %
%    \end{macrocode}
%    \end{macro}
%
%    \begin{macro}{\ThPdNu@firstoftwo}
%    \begin{macrocode}
\long\def\ThPdNu@firstoftwo#1#2{#1}
%    \end{macrocode}
%    \end{macro}
%
%    \begin{macro}{\ThPdNu@secondoftwo}
%    \begin{macrocode}
\long\def\ThPdNu@secondoftwo#1#2{#2}
%    \end{macrocode}
%    \end{macro}
%
% \subsection{Detect \hologo{eTeX}}
%
%    \begin{macrocode}
\begingroup\expandafter\expandafter\expandafter\endgroup
\expandafter\ifx\csname detokenize\endcsname\relax
  \catcode`\&=14 %
  \catcode`\$=9 %
\else
  \catcode`\&=9 %
  \catcode`\$=14 %
\fi
%    \end{macrocode}
%
% \subsection{User macro \cs{thepdfnumber}}
%
%    \begin{macro}{\thepdfnumber}
%    \begin{macrocode}
\def\thepdfnumber#1{%
  \romannumeral
& \iftrue\expandafter\ThPdNu@State@Plus\expandafter\fi
& \detokenize\expandafter{#1}%
& \ThPdNu@FIN
$ \ifx\ThPdNu@FIN#1\ThPdNu@FIN
$   \expandafter\ThPdNu@firstoftwo
$ \else
$   \expandafter\ThPdNu@secondoftwo
$ \fi
$ {%
$   \ThPdNu@zero
$   0\thepdfnumber@ErrorUnexpectedEnd
$ }{%
$   \iftrue\expandafter\ThPdNu@State@Plus\expandafter\fi#1\ThPdNu@FIN
$ }%
}
%    \end{macrocode}
%    \end{macro}
%
% \subsubsection{State definitions for sign}
%
%    \begin{macro}{\ThPdNu@State@Plus}
%    \begin{macrocode}
\def\ThPdNu@State@Plus#1\fi#2{%
  \fi
  \ifcase\ifx\ThPdNu@FIN#2%
           0%
&        \else\ifx-#2%
$        \else\ifnum`#2=45 % -
           1%
         \else\ifx0#2%
           2%
         \else\ifnum`#2>48 %
           \ifnum`#2<58 %
             3%
           \else
             9%
           \fi
&        \else\ifx.#2%
$        \else\ifnum`#2=46 % .
           4%
&        \else\ifx+#2%
$        \else\ifnum`#2=43 % +
           5%
         \else
           9%
         \fi\fi\fi\fi\fi\fi\ThPdNu@space
    \expandafter\ThPdNu@zero
    \expandafter0%
    \expandafter\thepdfnumber@ErrorUnexpectedEnd
  \or
    \ThPdNu@State@Minus
  \or
    \ThPdNu@State@SkipZeros!%
  \or
    \ThPdNu@State@Int!#2!%
  \or
    \ThPdNu@State@Dot!\ThPdNu@zero*\ThPdNu@zero!!%
  \or
    \ThPdNu@State@Plus
  \else
    \ThPdNu@ReturnError{0}%
  \fi
}
%    \end{macrocode}
%    \end{macro}
%
%    \begin{macro}{\ThPdNu@State@Minus}
%    \begin{macrocode}
\def\ThPdNu@State@Minus#1\fi#2{%
  \fi
  \ifcase\ifx\ThPdNu@FIN#2%
           0%
         \else\ifx0#2%
           1%
         \else\ifnum`#2>48 %
           \ifnum`#2<58 %
             2%
           \else
             9%
           \fi
&        \else\ifx.#2%
$        \else\ifnum`#2=46 % .
           3%
&        \else\ifx-#2%
$        \else\ifnum`#2=45 % -
           4%
&        \else\ifx+#2%
$        \else\ifnum`#2=43 % +
           5%
         \else
           9%
         \fi\fi\fi\fi\fi\fi\ThPdNu@space
    \expandafter\ThPdNu@zero
    \expandafter0%
    \expandafter\thepdfnumber@ErrorUnexpectedEnd
  \or
    \ThPdNu@State@SkipZeros-!%
  \or
    \ThPdNu@State@Int-!#2!%
  \or
    \ThPdNu@State@Dot-!\ThPdNu@zero*\ThPdNu@zero!!%
  \or
    \ThPdNu@State@Plus
  \or
    \ThPdNu@State@Minus
  \else
    \ThPdNu@ReturnError{0}%
  \fi
}
%    \end{macrocode}
%    \end{macro}
%
%    \begin{macro}{\ThPdNu@ReturnError}
%    \begin{macrocode}
\def\ThPdNu@ReturnError#1#2\fi#3\ThPdNu@FIN{%
  \fi
  \ThPdNu@zero
  #1%
  \thepdfnumber@ErrorInvalidToken
}
%    \end{macrocode}
%    \end{macro}
%
% \subsubsection{State definitions for integer part}
%
%    \begin{macro}{\ThPdNu@State@SkipZeros}
%    \begin{macrocode}
\def\ThPdNu@State@SkipZeros#1!#2\fi#3{%
  \fi
  \ifcase\ifx\ThPdNu@FIN#3%
           0%
         \else\ifx0#3%
           1%
         \else\ifnum`#3>48 %
           \ifnum`#3<58 %
             2%
           \else
             9%
           \fi
&        \else\ifx.#3%
$        \else\ifnum`#3=46 % .
           3%
         \else
           9%
         \fi\fi\fi\fi\ThPdNu@space
    \expandafter\ThPdNu@zero
    \expandafter0%
  \or
   \ThPdNu@State@SkipZeros#1!%
  \or
    \ThPdNu@State@Int#1!#3!%
  \or
    \ThPdNu@State@Dot#1!\ThPdNu@zero*\ThPdNu@zero!!%
  \else
    \ThPdNu@ReturnError{0}%
  \fi
}
%    \end{macrocode}
%    \end{macro}
%
%    \begin{macro}{\ThPdNu@State@Int}
%    \begin{macrocode}
\def\ThPdNu@State@Int#1!#2!#3\fi#4{%
  \fi
  \ifcase\ifx\ThPdNu@FIN#4%
           0%
         \else\ifnum`#4>47 %
           \ifnum`#4<58 %
             1%
           \else
             9%
           \fi
&        \else\ifx.#4%
$        \else\ifnum`#4=46 % .
           2%
         \else
           9%
         \fi\fi\fi\ThPdNu@space
    \ThPdNu@ReturnInt{#1#2}%
  \or
    \ThPdNu@State@Int#1!#2#4!%
  \or
    \ThPdNu@State@Dot#1!\ThPdNu@one#2*\ThPdNu@zero!!%
  \else
    \ThPdNu@ReturnError{#1#2}%
  \fi
}
%    \end{macrocode}
%    \end{macro}
%
%    \begin{macro}{\ThPdNu@ReturnInt}
%    \begin{macrocode}
\def\ThPdNu@ReturnInt#1#2\fi{%
  \fi
  \ThPdNu@zero
  #1%
}
%    \end{macrocode}
%    \end{macro}
%
% \subsubsection{State definitions for decimal digits}
%
%    \begin{macro}{\ThPdNu@State@Dot}
%    \begin{macrocode}
\def\ThPdNu@State@Dot#1*#2#3!#4!#5\fi#6{%
  \fi
  \ifcase\ifx\ThPdNu@FIN#6%
           0%
         \else\ifnum`#6>48 %
           \ifnum`#6<58 %
             1%
           \else
             9%
           \fi
         \else\ifx0#6%
           2%
         \else
           9%
         \fi\fi\fi\ThPdNu@space
    \ThPdNu@ReturnNumber#1*#2#3!%
  \or
    \ThPdNu@State@Dot#1*\ThPdNu@one#3#4#6!!%
  \or
    \ThPdNu@State@DotZero#1*#2#3!#4#6!%
  \else
    \ThPdNu@ReturnNumberInvalid#1*#2#3!%
  \fi
}
%    \end{macrocode}
%    \end{macro}
%
%    \begin{macro}{\ThPdNu@State@DotZero}
%    \begin{macrocode}
\def\ThPdNu@State@DotZero#1*#2#3!#4!#5\fi#6{%
  \fi
  \ifcase\ifx\ThPdNu@FIN#6%
           0%
         \else\ifnum`#6>48 %
           \ifnum`#6<58 %
             1%
           \else
             9%
           \fi
         \else\ifx0#6%
           2%
         \else
           9%
         \fi\fi\fi\ThPdNu@space
    \ThPdNu@ReturnNumber#1*#2#3!%
  \or
    \ThPdNu@State@Dot#1*\ThPdNu@one#3#4#6!!%
  \or
    \ThPdNu@State@DotZero#1*#2#3!#4#6!%
  \else
    \ThPdNu@ReturnNumber#1*#2#3!%
  \fi
}
%    \end{macrocode}
%    \end{macro}
%
%    \begin{macro}{\ThPdNu@ReturnNumber}
%    \begin{macrocode}
\def\ThPdNu@ReturnNumber#1!#2#3*#4#5!#6\fi{%
  \fi
  \ifcase#2%
    \expandafter\ThPdNu@firstoftwo
  \else
    \expandafter\ThPdNu@secondoftwo
  \fi
  {%
    \ifcase#4%
      \expandafter\ThPdNu@firstoftwo
    \else
      \expandafter\ThPdNu@secondoftwo
    \fi
    {\ThPdNu@zero 0}%
    {\ThPdNu@zero #1.#5}%
  }{%
    \ifcase#4%
      \expandafter\ThPdNu@firstoftwo
    \else
      \expandafter\ThPdNu@secondoftwo
    \fi
    {\ThPdNu@zero #1#3}%
    {\ThPdNu@zero #1#3.#5}%
  }%
}
%    \end{macrocode}
%    \end{macro}
%
%    \begin{macro}{\ThPdNu@ReturnNumberInvalid}
%    \begin{macrocode}
\def\ThPdNu@ReturnNumberInvalid#1*#2!#3\fi#4\ThPdNu@FIN{%
  \fi
  \iftrue\ThPdNu@ReturnNumber#1*#2!\fi
  \thepdfnumber@ErrorInvalidToken
}
%    \end{macrocode}
%    \end{macro}
%
% \subsection{Norm macro}
%
%    \begin{macro}{\thepdfnumberNormZeroOne}
%    \begin{macrocode}
\def\thepdfnumberNormZeroOne#1{%
  \romannumeral
& \iftrue\expandafter\ThPbNu@StateN@Plus\expandafter\fi
& \detokenize\expandafter{#1}%
& \ThPdNu@FIN
$ \ifx\ThPdNu@FIN#1\ThPdNu@FIN
$   \expandafter\ThPdNu@firstoftwo
$ \else
$   \expandafter\ThPdNu@secondoftwo
$ \fi
$ {%
$   \ThPdNu@zero
$   0\thepdfnumber@ErrorUnexpectedEnd
$ }{%
$   \iftrue\expandafter\ThPbNu@StateN@Plus\expandafter\fi#1\ThPdNu@FIN
$ }%
}
%    \end{macrocode}
%    \end{macro}
%
% \subsubsection{State definitions for sign}
%
%    \begin{macro}{\ThPbNu@StateN@Plus}
%    \begin{macrocode}
\def\ThPbNu@StateN@Plus#1\fi#2{%
  \fi
  \ifcase\ifx\ThPdNu@FIN#2%
           0%
&        \else\ifx-#2%
$        \else\ifnum`#2=45 % -
           1%
         \else\ifx0#2%
           2%
         \else\ifnum`#2>48 %
           \ifnum`#2<58 %
             3%
           \else
             9%
           \fi
&        \else\ifx.#2%
$        \else\ifnum`#2=46 % .
           4%
&        \else\ifx+#2%
$        \else\ifnum`#2=43 % +
           5%
         \else
           9%
         \fi\fi\fi\fi\fi\fi\ThPdNu@space
    \expandafter\ThPdNu@zero
    \expandafter0%
    \expandafter\thepdfnumber@ErrorUnexpectedEnd
  \or
    \ThPbNu@StateN@Minus
  \or
    \ThPbNu@StateN@SkipZeros
  \or
    \ThPdNu@ReturnAndSkip{1}%
  \or
    \ThPbNu@StateN@Dot\ThPdNu@zero!!%
  \or
    \ThPbNu@StateN@Plus
  \else
    \ThPdNu@ReturnError{0}%
  \fi
}
%    \end{macrocode}
%    \end{macro}
%
%    \begin{macro}{\ThPbNu@StateN@Minus}
%    \begin{macrocode}
\def\ThPbNu@StateN@Minus#1\fi#2{%
  \fi
  \ifcase\ifx\ThPdNu@FIN#2%
           0%
         \else\ifnum`#2>47 %
           \ifnum`#2<58 %
             1%
           \else
             9%
           \fi
&        \else\ifx.#2%
$        \else\ifnum`#2=46 % .
           1%
&        \else\ifx-#2%
$        \else\ifnum`#2=45 % -
           2%
&        \else\ifx+#2%
$        \else\ifnum`#2=43 % +
           3%
         \else
           9%
         \fi\fi\fi\fi\fi\ThPdNu@space
    \expandafter\ThPdNu@zero
    \expandafter0%
    \expandafter\thepdfnumber@ErrorUnexpectedEnd
  \or
    \ThPdNu@ReturnAndSkip{0}%
  \or
    \ThPbNu@StateN@Plus
  \or
    \ThPbNu@StateN@Minus
  \else
    \ThPdNu@ReturnError{0}%
  \fi
}
%    \end{macrocode}
%    \end{macro}
%
%    \begin{macro}{\ThPdNu@ReturnAndSkip}
%    \begin{macrocode}
\def\ThPdNu@ReturnAndSkip#1#2\fi#3\ThPdNu@FIN{%
  \fi
  \ThPdNu@zero
  #1%
}
%    \end{macrocode}
%    \end{macro}
%
% \subsubsection{State definitions for integer part}
%
%    \begin{macro}{\ThPbNu@StateN@SkipZeros}
%    \begin{macrocode}
\def\ThPbNu@StateN@SkipZeros#1\fi#2{%
  \fi
  \ifcase\ifx\ThPdNu@FIN#2%
           0%
         \else\ifx0#2%
           1%
         \else\ifnum`#2>48 %
           \ifnum`#2<58 %
             2%
           \else
             9%
           \fi
&        \else\ifx.#2%
$        \else\ifnum`#2=46 % .
           3%
         \else
           9%
         \fi\fi\fi\fi\ThPdNu@space
    \expandafter\ThPdNu@zero
    \expandafter0%
  \or
   \ThPbNu@StateN@SkipZeros%
  \or
    \ThPdNu@ReturnAndSkip{1}%
  \or
    \ThPbNu@StateN@Dot\ThPdNu@zero!!%
  \else
    \ThPdNu@ReturnError{0}%
  \fi
}
%    \end{macrocode}
%    \end{macro}
%
% \subsubsection{State definitions for decimal digits}
%
%    \begin{macro}{\ThPbNu@StateN@Dot}
%    \begin{macrocode}
\def\ThPbNu@StateN@Dot#1#2!#3!#4\fi#5{%
  \fi
  \ifcase\ifx\ThPdNu@FIN#5%
           0%
         \else\ifnum`#5>48 %
           \ifnum`#5<58 %
             1%
           \else
             9%
           \fi
         \else\ifx0#5%
           2%
         \else
           9%
         \fi\fi\fi\ThPdNu@space
    \ThPdNu@ReturnFracNumber#1#2!%
  \or
    \ThPbNu@StateN@Dot\ThPdNu@one#2#3#5!!%
  \or
    \ThPbNu@StateN@DotZero#1#2!#3#5!%
  \else
    \ThPdNu@ReturnFracNumberInvalid#1#2!%
  \fi
}
%    \end{macrocode}
%    \end{macro}
%
%    \begin{macro}{\ThPbNu@StateN@DotZero}
%    \begin{macrocode}
\def\ThPbNu@StateN@DotZero#1#2!#3!#4\fi#5{%
  \fi
  \ifcase\ifx\ThPdNu@FIN#5%
           0%
         \else\ifnum`#5>48 %
           \ifnum`#5<58 %
             1%
           \else
             9%
           \fi
         \else\ifx0#5%
           2%
         \else
           9%
         \fi\fi\fi\ThPdNu@space
    \ThPdNu@ReturnFracNumber#1#2!%
  \or
    \ThPbNu@StateN@Dot\ThPdNu@one#2#3#5!!%
  \or
    \ThPbNu@StateN@DotZero#1#2!#3#5!%
  \else
    \ThPdNu@ReturnFracNumberInvalid#1#2!%
  \fi
}
%    \end{macrocode}
%    \end{macro}
%
%    \begin{macro}{\ThPdNu@ReturnFracNumber}
%    \begin{macrocode}
\def\ThPdNu@ReturnFracNumber#1#2!#3\fi{%
  \fi
  \ifcase#1%
    \expandafter\ThPdNu@firstoftwo
  \else
    \expandafter\ThPdNu@secondoftwo
  \fi
  {\ThPdNu@zero 0}%
  {\ThPdNu@zero .#2}%
}
%    \end{macrocode}
%    \end{macro}
%
%    \begin{macro}{\ThPdNu@ReturnFracNumberInvalid}
%    \begin{macrocode}
\def\ThPdNu@ReturnFracNumberInvalid#1!#2\fi#3\ThPdNu@FIN{%
  \fi
  \iftrue\ThPdNu@ReturnFracNumber#1!\fi
  \thepdfnumber@ErrorInvalidToken
}
%    \end{macrocode}
%    \end{macro}
%
%    \begin{macrocode}
\ThPdNu@AtEnd%
%</package>
%    \end{macrocode}
%
% \section{Test}
%
% \subsection{Catcode checks for loading}
%
%    \begin{macrocode}
%<*test1>
%    \end{macrocode}
%    \begin{macrocode}
\catcode`\{=1 %
\catcode`\}=2 %
\catcode`\#=6 %
\catcode`\@=11 %
\expandafter\ifx\csname count@\endcsname\relax
  \countdef\count@=255 %
\fi
\expandafter\ifx\csname @gobble\endcsname\relax
  \long\def\@gobble#1{}%
\fi
\expandafter\ifx\csname @firstofone\endcsname\relax
  \long\def\@firstofone#1{#1}%
\fi
\expandafter\ifx\csname loop\endcsname\relax
  \expandafter\@firstofone
\else
  \expandafter\@gobble
\fi
{%
  \def\loop#1\repeat{%
    \def\body{#1}%
    \iterate
  }%
  \def\iterate{%
    \body
      \let\next\iterate
    \else
      \let\next\relax
    \fi
    \next
  }%
  \let\repeat=\fi
}%
\def\RestoreCatcodes{}
\count@=0 %
\loop
  \edef\RestoreCatcodes{%
    \RestoreCatcodes
    \catcode\the\count@=\the\catcode\count@\relax
  }%
\ifnum\count@<255 %
  \advance\count@ 1 %
\repeat

\def\RangeCatcodeInvalid#1#2{%
  \count@=#1\relax
  \loop
    \catcode\count@=15 %
  \ifnum\count@<#2\relax
    \advance\count@ 1 %
  \repeat
}
\def\RangeCatcodeCheck#1#2#3{%
  \count@=#1\relax
  \loop
    \ifnum#3=\catcode\count@
    \else
      \errmessage{%
        Character \the\count@\space
        with wrong catcode \the\catcode\count@\space
        instead of \number#3%
      }%
    \fi
  \ifnum\count@<#2\relax
    \advance\count@ 1 %
  \repeat
}
\def\space{ }
\expandafter\ifx\csname LoadCommand\endcsname\relax
  \def\LoadCommand{\input thepdfnumber.sty\relax}%
\fi
\def\Test{%
  \RangeCatcodeInvalid{0}{47}%
  \RangeCatcodeInvalid{58}{64}%
  \RangeCatcodeInvalid{91}{96}%
  \RangeCatcodeInvalid{123}{255}%
  \catcode`\@=12 %
  \catcode`\\=0 %
  \catcode`\%=14 %
  \LoadCommand
  \RangeCatcodeCheck{0}{36}{15}%
  \RangeCatcodeCheck{37}{37}{14}%
  \RangeCatcodeCheck{38}{47}{15}%
  \RangeCatcodeCheck{48}{57}{12}%
  \RangeCatcodeCheck{58}{63}{15}%
  \RangeCatcodeCheck{64}{64}{12}%
  \RangeCatcodeCheck{65}{90}{11}%
  \RangeCatcodeCheck{91}{91}{15}%
  \RangeCatcodeCheck{92}{92}{0}%
  \RangeCatcodeCheck{93}{96}{15}%
  \RangeCatcodeCheck{97}{122}{11}%
  \RangeCatcodeCheck{123}{255}{15}%
  \RestoreCatcodes
}
\Test
\csname @@end\endcsname
\end
%    \end{macrocode}
%    \begin{macrocode}
%</test1>
%    \end{macrocode}
%
% \subsection{Macro tests}
%
% \subsubsection{Preamble}
%
%    \begin{macrocode}
%<*test2>
\catcode`\{=1
\catcode`\}=2
\catcode`\#=6
\catcode`\@=11
\errorcontextlines=10000 %
\def\msg#{\immediate\write16}
\def\space{ }
\begingroup\expandafter\expandafter\expandafter\endgroup
\expandafter\ifx\csname RequirePackage\endcsname\relax
  \input thepdfnumber.sty\relax
\else
  \RequirePackage{thepdfnumber}[2016/05/16]%
\fi
%    \end{macrocode}
%
%    \begin{macro}{\Test}
%    \begin{macrocode}
\def\Test#1#2{%
  \def\TestExpected{#2}%
%<*active>
  \edef\TestExpected{\TestExpected}%
%</active>
  \expandafter\expandafter\expandafter\def
  \expandafter\expandafter\expandafter\TestResult
  \expandafter\expandafter\expandafter{%
    \TestCommand{#1}%
  }%
  \begingroup
    \toks0\expandafter{\TestResult}%
    \toks2\expandafter{\TestExpected}%
    \ifx\TestResult\TestExpected
      \msg{* OK: #1 => \the\toks0}%
    \else
      \errmessage{FAILED: #1 => \the\toks0\space(\the\toks2)}%
    \fi
  \endgroup
}
%    \end{macrocode}
%    \end{macro}
%
%    \begin{macrocode}
\let\thepdfnumber@ErrorUnexpectedEnd\relax
\let\thepdfnumber@ErrorInvalidToken\relax
%    \end{macrocode}
%
% \subsubsection{Test with active catcodes}
%
%    \begin{macrocode}
%<*active>
\catcode`\~=13 %
\def\ThPdNu@Temp#1{%
  \begingroup
    \lccode`\~=`#1 %
  \lowercase{\endgroup
    \edef~%
  }{\string#1}%
  \catcode`#1=13 %
}
\ThPdNu@Temp{.}
\ThPdNu@Temp{-}
\ThPdNu@Temp{+}
%</active>
%    \end{macrocode}
%
% \subsubsection{Test cases for \cs{thepdfnumber}}
%
%    \begin{macro}{\TestCommand}
%    \begin{macrocode}
\let\TestCommand\thepdfnumber
%    \end{macrocode}
%    \end{macro}
%
%    \begin{macrocode}
\Test{0}{0}
\Test{1}{1}
\Test{23}{23}
\Test{0045}{45}
\Test{4500}{4500}
\Test{-0}{0}
\Test{-1}{-1}
\Test{-00002}{-2}
\Test{-010203}{-10203}
\Test{+0}{0}
\Test{+123}{123}

\Test{+-4}{-4}
\Test{+-+-4}{4}
\Test{++--++--++4}{4}
\Test{+++---+++---+++---4}{-4}

\Test{.}{0}
\Test{0.}{0}
\Test{.0}{0}
\Test{0.}{0}
\Test{-.}{0}
\Test{-0.}{0}
\Test{-.0}{0}
\Test{-0.}{0}
\Test{010.020}{10.02}
\Test{123.456}{123.456}
\Test{12.}{12}
\Test{.123}{.123}
\Test{.0001}{.0001}
\Test{.00100}{.001}
\Test{.12003400560078009}{.12003400560078009}
\Test{-.12003400560078009}{-.12003400560078009}
\Test{04.0500000}{4.05}

\Test{}{0\thepdfnumber@ErrorUnexpectedEnd}
\Test{+}{0\thepdfnumber@ErrorUnexpectedEnd}
\Test{-}{0\thepdfnumber@ErrorUnexpectedEnd}
\Test{a}{0\thepdfnumber@ErrorInvalidToken}
\Test{0x1}{0\thepdfnumber@ErrorInvalidToken}
\Test{4x56}{4\thepdfnumber@ErrorInvalidToken}
\Test{012,34}{12\thepdfnumber@ErrorInvalidToken}
\Test{0.12x4}{.12\thepdfnumber@ErrorInvalidToken}
%    \end{macrocode}
%
% \subsubsection{Test cases for \cs{thepdfnumberNormZeroOne}}
%
%    \begin{macro}{\TestCommand}
%    \begin{macrocode}
\let\TestCommand\thepdfnumberNormZeroOne
%    \end{macrocode}
%    \end{macro}
%
%    \begin{macrocode}
\Test{0}{0}
\Test{1}{1}
\Test{23}{1}
\Test{12}{1}
\Test{10}{1}
\Test{0045}{1}
\Test{001}{1}
\Test{-0}{0}
\Test{-1}{0}
\Test{-01}{0}
\Test{-4}{0}

\Test{+-0}{0}
\Test{+-+-1}{1}
\Test{++--++--++1}{1}
\Test{+++---+++---+++---1}{0}

\Test{.}{0}
\Test{0.}{0}
\Test{.0}{0}
\Test{0.}{0}
\Test{-.}{0}
\Test{-0.}{0}
\Test{-.0}{0}
\Test{-0.}{0}
\Test{010.020}{1}
\Test{123.456}{1}
\Test{12.}{1}
\Test{.123}{.123}
\Test{.0001}{.0001}
\Test{.00100}{.001}
\Test{.12003400560078009}{.12003400560078009}
\Test{-.12003400560078009}{0}
\Test{04.0500000}{1}
\Test{0.1200340056}{.1200340056}
\Test{1.05}{1}

\Test{}{0\thepdfnumber@ErrorUnexpectedEnd}
\Test{+}{0\thepdfnumber@ErrorUnexpectedEnd}
\Test{-}{0\thepdfnumber@ErrorUnexpectedEnd}
\Test{a}{0\thepdfnumber@ErrorInvalidToken}
\Test{0x1}{0\thepdfnumber@ErrorInvalidToken}
\Test{4x56}{1}
\Test{012,34}{1}
\Test{.012x4}{.012\thepdfnumber@ErrorInvalidToken}
%    \end{macrocode}
%
%    \begin{macrocode}
\csname @@end\endcsname\end
%</test2>
%    \end{macrocode}
%
% \section{Installation}
%
% \subsection{Download}
%
% \paragraph{Package.} This package is available on
% CTAN\footnote{\url{http://ctan.org/pkg/thepdfnumber}}:
% \begin{description}
% \item[\CTAN{macros/latex/contrib/oberdiek/thepdfnumber.dtx}] The source file.
% \item[\CTAN{macros/latex/contrib/oberdiek/thepdfnumber.pdf}] Documentation.
% \end{description}
%
%
% \paragraph{Bundle.} All the packages of the bundle `oberdiek'
% are also available in a TDS compliant ZIP archive. There
% the packages are already unpacked and the documentation files
% are generated. The files and directories obey the TDS standard.
% \begin{description}
% \item[\CTAN{install/macros/latex/contrib/oberdiek.tds.zip}]
% \end{description}
% \emph{TDS} refers to the standard ``A Directory Structure
% for \TeX\ Files'' (\CTAN{tds/tds.pdf}). Directories
% with \xfile{texmf} in their name are usually organized this way.
%
% \subsection{Bundle installation}
%
% \paragraph{Unpacking.} Unpack the \xfile{oberdiek.tds.zip} in the
% TDS tree (also known as \xfile{texmf} tree) of your choice.
% Example (linux):
% \begin{quote}
%   |unzip oberdiek.tds.zip -d ~/texmf|
% \end{quote}
%
% \paragraph{Script installation.}
% Check the directory \xfile{TDS:scripts/oberdiek/} for
% scripts that need further installation steps.
% Package \xpackage{attachfile2} comes with the Perl script
% \xfile{pdfatfi.pl} that should be installed in such a way
% that it can be called as \texttt{pdfatfi}.
% Example (linux):
% \begin{quote}
%   |chmod +x scripts/oberdiek/pdfatfi.pl|\\
%   |cp scripts/oberdiek/pdfatfi.pl /usr/local/bin/|
% \end{quote}
%
% \subsection{Package installation}
%
% \paragraph{Unpacking.} The \xfile{.dtx} file is a self-extracting
% \docstrip\ archive. The files are extracted by running the
% \xfile{.dtx} through \plainTeX:
% \begin{quote}
%   \verb|tex thepdfnumber.dtx|
% \end{quote}
%
% \paragraph{TDS.} Now the different files must be moved into
% the different directories in your installation TDS tree
% (also known as \xfile{texmf} tree):
% \begin{quote}
% \def\t{^^A
% \begin{tabular}{@{}>{\ttfamily}l@{ $\rightarrow$ }>{\ttfamily}l@{}}
%   thepdfnumber.sty & tex/generic/oberdiek/thepdfnumber.sty\\
%   thepdfnumber.pdf & doc/latex/oberdiek/thepdfnumber.pdf\\
%   test/thepdfnumber-test1.tex & doc/latex/oberdiek/test/thepdfnumber-test1.tex\\
%   test/thepdfnumber-test2.tex & doc/latex/oberdiek/test/thepdfnumber-test2.tex\\
%   test/thepdfnumber-test3.tex & doc/latex/oberdiek/test/thepdfnumber-test3.tex\\
%   thepdfnumber.dtx & source/latex/oberdiek/thepdfnumber.dtx\\
% \end{tabular}^^A
% }^^A
% \sbox0{\t}^^A
% \ifdim\wd0>\linewidth
%   \begingroup
%     \advance\linewidth by\leftmargin
%     \advance\linewidth by\rightmargin
%   \edef\x{\endgroup
%     \def\noexpand\lw{\the\linewidth}^^A
%   }\x
%   \def\lwbox{^^A
%     \leavevmode
%     \hbox to \linewidth{^^A
%       \kern-\leftmargin\relax
%       \hss
%       \usebox0
%       \hss
%       \kern-\rightmargin\relax
%     }^^A
%   }^^A
%   \ifdim\wd0>\lw
%     \sbox0{\small\t}^^A
%     \ifdim\wd0>\linewidth
%       \ifdim\wd0>\lw
%         \sbox0{\footnotesize\t}^^A
%         \ifdim\wd0>\linewidth
%           \ifdim\wd0>\lw
%             \sbox0{\scriptsize\t}^^A
%             \ifdim\wd0>\linewidth
%               \ifdim\wd0>\lw
%                 \sbox0{\tiny\t}^^A
%                 \ifdim\wd0>\linewidth
%                   \lwbox
%                 \else
%                   \usebox0
%                 \fi
%               \else
%                 \lwbox
%               \fi
%             \else
%               \usebox0
%             \fi
%           \else
%             \lwbox
%           \fi
%         \else
%           \usebox0
%         \fi
%       \else
%         \lwbox
%       \fi
%     \else
%       \usebox0
%     \fi
%   \else
%     \lwbox
%   \fi
% \else
%   \usebox0
% \fi
% \end{quote}
% If you have a \xfile{docstrip.cfg} that configures and enables \docstrip's
% TDS installing feature, then some files can already be in the right
% place, see the documentation of \docstrip.
%
% \subsection{Refresh file name databases}
%
% If your \TeX~distribution
% (\teTeX, \mikTeX, \dots) relies on file name databases, you must refresh
% these. For example, \teTeX\ users run \verb|texhash| or
% \verb|mktexlsr|.
%
% \subsection{Some details for the interested}
%
% \paragraph{Attached source.}
%
% The PDF documentation on CTAN also includes the
% \xfile{.dtx} source file. It can be extracted by
% AcrobatReader 6 or higher. Another option is \textsf{pdftk},
% e.g. unpack the file into the current directory:
% \begin{quote}
%   \verb|pdftk thepdfnumber.pdf unpack_files output .|
% \end{quote}
%
% \paragraph{Unpacking with \LaTeX.}
% The \xfile{.dtx} chooses its action depending on the format:
% \begin{description}
% \item[\plainTeX:] Run \docstrip\ and extract the files.
% \item[\LaTeX:] Generate the documentation.
% \end{description}
% If you insist on using \LaTeX\ for \docstrip\ (really,
% \docstrip\ does not need \LaTeX), then inform the autodetect routine
% about your intention:
% \begin{quote}
%   \verb|latex \let\install=y\input{thepdfnumber.dtx}|
% \end{quote}
% Do not forget to quote the argument according to the demands
% of your shell.
%
% \paragraph{Generating the documentation.}
% You can use both the \xfile{.dtx} or the \xfile{.drv} to generate
% the documentation. The process can be configured by the
% configuration file \xfile{ltxdoc.cfg}. For instance, put this
% line into this file, if you want to have A4 as paper format:
% \begin{quote}
%   \verb|\PassOptionsToClass{a4paper}{article}|
% \end{quote}
% An example follows how to generate the
% documentation with pdf\LaTeX:
% \begin{quote}
%\begin{verbatim}
%pdflatex thepdfnumber.dtx
%makeindex -s gind.ist thepdfnumber.idx
%pdflatex thepdfnumber.dtx
%makeindex -s gind.ist thepdfnumber.idx
%pdflatex thepdfnumber.dtx
%\end{verbatim}
% \end{quote}
%
% \printbibliography[
%   heading=bibnumbered,
% ]
%
% \begin{History}
%   \begin{Version}{2011/11/24 v1.0}
%   \item
%     First version.
%   \end{Version}
%   \begin{Version}{2016/05/16 v1.1}
%   \item
%     Documentation updates.
%   \end{Version}
% \end{History}
%
% \PrintIndex
%
% \Finale
\endinput
|
% \end{quote}
% Do not forget to quote the argument according to the demands
% of your shell.
%
% \paragraph{Generating the documentation.}
% You can use both the \xfile{.dtx} or the \xfile{.drv} to generate
% the documentation. The process can be configured by the
% configuration file \xfile{ltxdoc.cfg}. For instance, put this
% line into this file, if you want to have A4 as paper format:
% \begin{quote}
%   \verb|\PassOptionsToClass{a4paper}{article}|
% \end{quote}
% An example follows how to generate the
% documentation with pdf\LaTeX:
% \begin{quote}
%\begin{verbatim}
%pdflatex thepdfnumber.dtx
%makeindex -s gind.ist thepdfnumber.idx
%pdflatex thepdfnumber.dtx
%makeindex -s gind.ist thepdfnumber.idx
%pdflatex thepdfnumber.dtx
%\end{verbatim}
% \end{quote}
%
% \printbibliography[
%   heading=bibnumbered,
% ]
%
% \begin{History}
%   \begin{Version}{2011/11/24 v1.0}
%   \item
%     First version.
%   \end{Version}
%   \begin{Version}{2016/05/16 v1.1}
%   \item
%     Documentation updates.
%   \end{Version}
% \end{History}
%
% \PrintIndex
%
% \Finale
\endinput

%        (quote the arguments according to the demands of your shell)
%
% Documentation:
%    (a) If thepdfnumber.drv is present:
%           latex thepdfnumber.drv
%    (b) Without thepdfnumber.drv:
%           latex thepdfnumber.dtx; ...
%    The class ltxdoc loads the configuration file ltxdoc.cfg
%    if available. Here you can specify further options, e.g.
%    use A4 as paper format:
%       \PassOptionsToClass{a4paper}{article}
%
%    Programm calls to get the documentation (example):
%       pdflatex thepdfnumber.dtx
%       makeindex -s gind.ist thepdfnumber.idx
%       pdflatex thepdfnumber.dtx
%       makeindex -s gind.ist thepdfnumber.idx
%       pdflatex thepdfnumber.dtx
%
% Installation:
%    TDS:tex/generic/oberdiek/thepdfnumber.sty
%    TDS:doc/latex/oberdiek/thepdfnumber.pdf
%    TDS:doc/latex/oberdiek/test/thepdfnumber-test1.tex
%    TDS:doc/latex/oberdiek/test/thepdfnumber-test2.tex
%    TDS:doc/latex/oberdiek/test/thepdfnumber-test3.tex
%    TDS:source/latex/oberdiek/thepdfnumber.dtx
%
%<*ignore>
\begingroup
  \catcode123=1 %
  \catcode125=2 %
  \def\x{LaTeX2e}%
\expandafter\endgroup
\ifcase 0\ifx\install y1\fi\expandafter
         \ifx\csname processbatchFile\endcsname\relax\else1\fi
         \ifx\fmtname\x\else 1\fi\relax
\else\csname fi\endcsname
%</ignore>
%<*install>
\input docstrip.tex
\Msg{************************************************************************}
\Msg{* Installation}
\Msg{* Package: thepdfnumber 2016/05/16 v1.1 Print PDF numbers with minimal digits (HO)}
\Msg{************************************************************************}

\keepsilent
\askforoverwritefalse

\let\MetaPrefix\relax
\preamble

This is a generated file.

Project: thepdfnumber
Version: 2016/05/16 v1.1

Copyright (C) 2011 by
   Heiko Oberdiek <heiko.oberdiek at googlemail.com>

This work may be distributed and/or modified under the
conditions of the LaTeX Project Public License, either
version 1.3c of this license or (at your option) any later
version. This version of this license is in
   http://www.latex-project.org/lppl/lppl-1-3c.txt
and the latest version of this license is in
   http://www.latex-project.org/lppl.txt
and version 1.3 or later is part of all distributions of
LaTeX version 2005/12/01 or later.

This work has the LPPL maintenance status "maintained".

This Current Maintainer of this work is Heiko Oberdiek.

The Base Interpreter refers to any `TeX-Format',
because some files are installed in TDS:tex/generic//.

This work consists of the main source file thepdfnumber.dtx
and the derived files
   thepdfnumber.sty, thepdfnumber.pdf, thepdfnumber.ins, thepdfnumber.drv,
   thepdfnumber-test1.tex, thepdfnumber-test2.tex,
   thepdfnumber-test3.tex.

\endpreamble
\let\MetaPrefix\DoubleperCent

\generate{%
  \file{thepdfnumber.ins}{\from{thepdfnumber.dtx}{install}}%
  \file{thepdfnumber.drv}{\from{thepdfnumber.dtx}{driver}}%
  \usedir{tex/generic/oberdiek}%
  \file{thepdfnumber.sty}{\from{thepdfnumber.dtx}{package}}%
  \usedir{doc/latex/oberdiek/test}%
  \file{thepdfnumber-test1.tex}{\from{thepdfnumber.dtx}{test1}}%
  \file{thepdfnumber-test2.tex}{\from{thepdfnumber.dtx}{test2}}%
  \file{thepdfnumber-test3.tex}{\from{thepdfnumber.dtx}{test2,active}}%
}

\catcode32=13\relax% active space
\let =\space%
\Msg{************************************************************************}
\Msg{*}
\Msg{* To finish the installation you have to move the following}
\Msg{* file into a directory searched by TeX:}
\Msg{*}
\Msg{*     thepdfnumber.sty}
\Msg{*}
\Msg{* To produce the documentation run the file `thepdfnumber.drv'}
\Msg{* through LaTeX.}
\Msg{*}
\Msg{* Happy TeXing!}
\Msg{*}
\Msg{************************************************************************}

\endbatchfile
%</install>
%<*ignore>
\fi
%</ignore>
%<*driver>
\NeedsTeXFormat{LaTeX2e}
\ProvidesFile{thepdfnumber.drv}%
  [2016/05/16 v1.1 Print PDF numbers with minimal digits (HO)]%
\documentclass{ltxdoc}
\usepackage{holtxdoc}[2011/11/22]
\usepackage{csquotes}
\usepackage[
  backend=bibtex,
  bibencoding=ascii,
  alldates=iso8601,
]{biblatex}[2011/11/13]
\bibliography{oberdiek-source}
\begin{document}
  \DocInput{thepdfnumber.dtx}%
\end{document}
%</driver>
% \fi
%
%
% \CharacterTable
%  {Upper-case    \A\B\C\D\E\F\G\H\I\J\K\L\M\N\O\P\Q\R\S\T\U\V\W\X\Y\Z
%   Lower-case    \a\b\c\d\e\f\g\h\i\j\k\l\m\n\o\p\q\r\s\t\u\v\w\x\y\z
%   Digits        \0\1\2\3\4\5\6\7\8\9
%   Exclamation   \!     Double quote  \"     Hash (number) \#
%   Dollar        \$     Percent       \%     Ampersand     \&
%   Acute accent  \'     Left paren    \(     Right paren   \)
%   Asterisk      \*     Plus          \+     Comma         \,
%   Minus         \-     Point         \.     Solidus       \/
%   Colon         \:     Semicolon     \;     Less than     \<
%   Equals        \=     Greater than  \>     Question mark \?
%   Commercial at \@     Left bracket  \[     Backslash     \\
%   Right bracket \]     Circumflex    \^     Underscore    \_
%   Grave accent  \`     Left brace    \{     Vertical bar  \|
%   Right brace   \}     Tilde         \~}
%
% \GetFileInfo{thepdfnumber.drv}
%
% \title{The \xpackage{thepdfnumber} package}
% \date{2016/05/16 v1.1}
% \author{Heiko Oberdiek\thanks
% {Please report any issues at https://github.com/ho-tex/oberdiek/issues}\\
% \xemail{heiko.oberdiek at googlemail.com}}
%
% \maketitle
%
% \begin{abstract}
% The package converts real numbers to a minimal representation
% that is stripped from leading or trailing zeros,
% plus signs and decimal point if not necessary.
% \end{abstract}
%
% \tableofcontents
%
% \section{Documentation}
%
% \subsection{Introduction}
%
% Dealing with the PDF format, there is sometimes the need to write
% some low level PDF stuff. In case of numbers, the numbers can
% arise from user input (e.g. color or transparency specifications)
% or can be calculated. For example, \hologo{LaTeX}'s \cs{strip@pt}
% makes a good job to output a real number. It automatically suppresses
% the decimal part if the number is an integer. However it leaves
% a leading zero for numbers greater zero and smaller one.
% Thus the package provides macros that can be used with different
% formats, even with \hologo{iniTeX} and generates numbers that
% are valid numbers of the PDF format and whose length is minimal.
%
% \subsection{Usage}
%
% The package \xpackage{thepdfnumber} can be used with \hologo{LaTeX},
% \hologo{plainTeX} or even with \hologo{iniTeX}:
% \begin{quote}
%   |\RequirePackage{thepdfnumber}| \textit{\% \hologo{LaTeX}}\\
%   |\input thepdfnumber.sty| \textit{\% \hologo{plainTeX}/\hologo{iniTeX}}
% \end{quote}
% The package does not need and have package options.
%
% \subsection{User macros}
%
% All user macros are expandable in exact two expansion steps.
%
% \begin{declcs}{thepdfnumber} \M{number}
% \end{declcs}
% Macro \cs{thepdfnumber} takes a number
% as argument and expands to a minimal representation of that number.
% Some examples:
% \begin{quote}
% \begin{tabular}{r@{.: }>{\ttfamily}l@{ $\rightarrow$\ }>{\ttfamily}l@{}}
%   1& +123 & 123\\
%   2& --123 & 123\\
%   3& -01 & -1\\
%   4& 0045 & 45\\
%   5& 1.0 & 1\\
%   6& 1.20 & 1.2\\
%   7& 0.0 & 0\\
%   8& 0.78 & .78\\
%   9& +012.340 & 12.34\\
% \end{tabular}
% \end{quote}
% It reduces the length of the number representation:
% \begin{itemize}
% \item The signs are collapsed and only one minus sign
%   is output if the number is negative (see examples 1, 2, 3, 9).
% \item Leading zeros are removed (4, 8, 9) unless the number is zero (7).
% \item The decimal part is omitted, if the number is an integer (5, 7).
% \item Trailing zeros from the decimal part are stripped (5, 6, 7, 9).
% \end{itemize}
% The resulting number representation can be catched with one of the
% following Perl regular expressions:
% \begin{itemize}
% \item |^0$| (zero)
% \item |^-?[1-9][0-9]*$| (integer)
% \item |^-?[0-9]*\.[0-9]*[1-9]$| (real)
% \end{itemize}
% This is a valid numeric object of the PDF specification \cite[``7.3.3~Numeric Objects'']{pdfspec-iso32000-1}.
%
% \begin{declcs}{thepdfnumberNormZeroOne}
% \end{declcs}
% There are various places in the PDF specification where the
% number is in the domain 0.0 upto 1.0. Macro thepdfnumberNormZeroOne
% automatically adjusts the number to fit into that range. Negative
% numbers are mapped to |0| and numbers greater than one are
% replaced by |1|. Thus the result fits one of the following regular
% expressions:
% \begin{itemize}
% \item |^0$|
% \item |^\.[0-9]*[1-9]$|
% \item |^1$|
% \end{itemize}
% Examples:
% \begin{quote}
% \begin{tabular}{@{}>{\ttfamily}l@{ $\rightarrow$\ }>{\ttfamily}l@{}}
%   -456 & 0\\
%   -0.001 & 0\\
%   0.0 & 0\\
%   0.010 & .01\\
%   0.456 & .456\\
%   1.0 & 1\\
%   01.001 & 1\\
%   4 & 1\\
% \end{tabular}
% \end{quote}
%
% \subsection{Input number}
%
% The user macros expect a number as argument. The number can either
% be given explicitely or as macro that expands in one step
% to an explicite number, because the first token of the argument
% is expanded once.
%
% The explicite number consists of
% \begin{itemize}
% \item optional signs `\texttt{+}' and `\texttt{-}',
% \item digits `\texttt{0}' upto `\texttt{9}' and
% \item an optional dot `\texttt{.}'.
% \end{itemize}
% All tokens must have catcode 12 (other), the default catcodes
% for these characters in \hologo{LaTeX}, \hologo{plainTeX} or
% \hologo{iniTeX}. As Perl regular expression the number is expected
% in one of the following forms:
% \begin{itemize}
% \item |^[+-]*[0-9]+$|
% \item |^[+-]*[0-9]*\.[0-9]*$|
% \end{itemize}
% At least one digit or the dot must be present.
%
% \subsection{Error handling}
%
% The package is not intended for validating numbers or to
% decide if an argument is a number. Therefore it is an usage
% error to use the user macros with arguments that are not
% explicite numbers as specified in the previous sections.
% Nevertheless some error conditions are sometimes recognized.
% Errors are given in form of an undefined command sequence.
% It is the only way to notify \hologo{TeX} in expandable context.
% Expanding to some error text would invalidate the output.
% Currently the following errors are thrown:
% \begin{description}
% \item[\cs{thepdfnumber@ErrorEndMarker}:]
%   Internally the argument parsing uses an end marker that
%   is never called directly. If it is called with valid
%   user input, then this is a bug. Otherwise it means the user input
%   contains nasty stuff.
% \item[\cs{thepdfnumber@ErrorUnexpectedEnd}:] The macros expect at least one digit
%   or the dot, otherwise if the argument is empty or only contains
%   signs, then this error is called.
% \item[\cs{thepdfnumber@ErrorInvalidToken}:] It is called if the number
%   contains other tokens than signs, digits or the dot or the
%   token at the wrong place (e.g. a sign after a digit).
%   In case of \cs{thepdfnumberNormZeroOne} this error condition might
%   not always be detected, because the number parsing might stop
%   at an early point, when the result is already clear (e.g.
%   if the number is negative or will be greater than one).
% \item[Improper alphabetic constant:] This error might be
%   thrown by \hologo{TeX}, if the number contains
%   command tokens instead of characters.
% \end{description}
%
% \StopEventually{
% }
%
% \section{Implementation}
%
%    \begin{macrocode}
%<*package>
%    \end{macrocode}
%
% \subsection{Reload check and package identification}
%    Reload check, especially if the package is not used with \LaTeX.
%    \begin{macrocode}
\begingroup\catcode61\catcode48\catcode32=10\relax%
  \catcode13=5 % ^^M
  \endlinechar=13 %
  \catcode35=6 % #
  \catcode39=12 % '
  \catcode44=12 % ,
  \catcode45=12 % -
  \catcode46=12 % .
  \catcode58=12 % :
  \catcode64=11 % @
  \catcode123=1 % {
  \catcode125=2 % }
  \expandafter\let\expandafter\x\csname ver@thepdfnumber.sty\endcsname
  \ifx\x\relax % plain-TeX, first loading
  \else
    \def\empty{}%
    \ifx\x\empty % LaTeX, first loading,
      % variable is initialized, but \ProvidesPackage not yet seen
    \else
      \expandafter\ifx\csname PackageInfo\endcsname\relax
        \def\x#1#2{%
          \immediate\write-1{Package #1 Info: #2.}%
        }%
      \else
        \def\x#1#2{\PackageInfo{#1}{#2, stopped}}%
      \fi
      \x{thepdfnumber}{The package is already loaded}%
      \aftergroup\endinput
    \fi
  \fi
\endgroup%
%    \end{macrocode}
%    Package identification:
%    \begin{macrocode}
\begingroup\catcode61\catcode48\catcode32=10\relax%
  \catcode13=5 % ^^M
  \endlinechar=13 %
  \catcode35=6 % #
  \catcode39=12 % '
  \catcode40=12 % (
  \catcode41=12 % )
  \catcode44=12 % ,
  \catcode45=12 % -
  \catcode46=12 % .
  \catcode47=12 % /
  \catcode58=12 % :
  \catcode64=11 % @
  \catcode91=12 % [
  \catcode93=12 % ]
  \catcode123=1 % {
  \catcode125=2 % }
  \expandafter\ifx\csname ProvidesPackage\endcsname\relax
    \def\x#1#2#3[#4]{\endgroup
      \immediate\write-1{Package: #3 #4}%
      \xdef#1{#4}%
    }%
  \else
    \def\x#1#2[#3]{\endgroup
      #2[{#3}]%
      \ifx#1\@undefined
        \xdef#1{#3}%
      \fi
      \ifx#1\relax
        \xdef#1{#3}%
      \fi
    }%
  \fi
\expandafter\x\csname ver@thepdfnumber.sty\endcsname
\ProvidesPackage{thepdfnumber}%
  [2016/05/16 v1.1 Print PDF numbers with minimal digits (HO)]%
%    \end{macrocode}
%
% \subsection{Catcodes}
%
%    \begin{macrocode}
\begingroup\catcode61\catcode48\catcode32=10\relax%
  \catcode13=5 % ^^M
  \endlinechar=13 %
  \catcode123=1 % {
  \catcode125=2 % }
  \catcode64=11 % @
  \def\x{\endgroup
    \expandafter\edef\csname ThPdNu@AtEnd\endcsname{%
      \endlinechar=\the\endlinechar\relax
      \catcode13=\the\catcode13\relax
      \catcode32=\the\catcode32\relax
      \catcode35=\the\catcode35\relax
      \catcode61=\the\catcode61\relax
      \catcode64=\the\catcode64\relax
      \catcode123=\the\catcode123\relax
      \catcode125=\the\catcode125\relax
    }%
  }%
\x\catcode61\catcode48\catcode32=10\relax%
\catcode13=5 % ^^M
\endlinechar=13 %
\catcode35=6 % #
\catcode64=11 % @
\catcode123=1 % {
\catcode125=2 % }
\def\TMP@EnsureCode#1#2{%
  \edef\ThPdNu@AtEnd{%
    \ThPdNu@AtEnd
    \catcode#1=\the\catcode#1\relax
  }%
  \catcode#1=#2\relax
}
\TMP@EnsureCode{33}{12}% !
\TMP@EnsureCode{36}{3}% $
\TMP@EnsureCode{38}{4}% &
\TMP@EnsureCode{42}{12}% *
\TMP@EnsureCode{43}{12}% +
\TMP@EnsureCode{45}{12}% -
\TMP@EnsureCode{46}{12}% .
\TMP@EnsureCode{60}{12}% <
\TMP@EnsureCode{62}{12}% >
\TMP@EnsureCode{96}{12}% `
\edef\ThPdNu@AtEnd{\ThPdNu@AtEnd\noexpand\endinput}
%    \end{macrocode}
%
% \subsection{Helper macros}
%
%    \begin{macro}{\ThPdNu@FIN}
%    \begin{macrocode}
\def\ThPdNu@FIN{\thepdfnumber@ErrorEndMarker}
%    \end{macrocode}
%    \end{macro}
%
%    \begin{macro}{\ThPdNu@space}
%    \begin{macrocode}
\def\ThPdNu@space{ }
%    \end{macrocode}
%    \end{macro}
%
%    \begin{macro}{\ThPdNu@zero}
%    \begin{macrocode}
\chardef\ThPdNu@zero=0 %
%    \end{macrocode}
%    \end{macro}
%
%    \begin{macro}{\ThPdNu@one}
%    \begin{macrocode}
\chardef\ThPdNu@one=1 %
%    \end{macrocode}
%    \end{macro}
%
%    \begin{macro}{\ThPdNu@firstoftwo}
%    \begin{macrocode}
\long\def\ThPdNu@firstoftwo#1#2{#1}
%    \end{macrocode}
%    \end{macro}
%
%    \begin{macro}{\ThPdNu@secondoftwo}
%    \begin{macrocode}
\long\def\ThPdNu@secondoftwo#1#2{#2}
%    \end{macrocode}
%    \end{macro}
%
% \subsection{Detect \hologo{eTeX}}
%
%    \begin{macrocode}
\begingroup\expandafter\expandafter\expandafter\endgroup
\expandafter\ifx\csname detokenize\endcsname\relax
  \catcode`\&=14 %
  \catcode`\$=9 %
\else
  \catcode`\&=9 %
  \catcode`\$=14 %
\fi
%    \end{macrocode}
%
% \subsection{User macro \cs{thepdfnumber}}
%
%    \begin{macro}{\thepdfnumber}
%    \begin{macrocode}
\def\thepdfnumber#1{%
  \romannumeral
& \iftrue\expandafter\ThPdNu@State@Plus\expandafter\fi
& \detokenize\expandafter{#1}%
& \ThPdNu@FIN
$ \ifx\ThPdNu@FIN#1\ThPdNu@FIN
$   \expandafter\ThPdNu@firstoftwo
$ \else
$   \expandafter\ThPdNu@secondoftwo
$ \fi
$ {%
$   \ThPdNu@zero
$   0\thepdfnumber@ErrorUnexpectedEnd
$ }{%
$   \iftrue\expandafter\ThPdNu@State@Plus\expandafter\fi#1\ThPdNu@FIN
$ }%
}
%    \end{macrocode}
%    \end{macro}
%
% \subsubsection{State definitions for sign}
%
%    \begin{macro}{\ThPdNu@State@Plus}
%    \begin{macrocode}
\def\ThPdNu@State@Plus#1\fi#2{%
  \fi
  \ifcase\ifx\ThPdNu@FIN#2%
           0%
&        \else\ifx-#2%
$        \else\ifnum`#2=45 % -
           1%
         \else\ifx0#2%
           2%
         \else\ifnum`#2>48 %
           \ifnum`#2<58 %
             3%
           \else
             9%
           \fi
&        \else\ifx.#2%
$        \else\ifnum`#2=46 % .
           4%
&        \else\ifx+#2%
$        \else\ifnum`#2=43 % +
           5%
         \else
           9%
         \fi\fi\fi\fi\fi\fi\ThPdNu@space
    \expandafter\ThPdNu@zero
    \expandafter0%
    \expandafter\thepdfnumber@ErrorUnexpectedEnd
  \or
    \ThPdNu@State@Minus
  \or
    \ThPdNu@State@SkipZeros!%
  \or
    \ThPdNu@State@Int!#2!%
  \or
    \ThPdNu@State@Dot!\ThPdNu@zero*\ThPdNu@zero!!%
  \or
    \ThPdNu@State@Plus
  \else
    \ThPdNu@ReturnError{0}%
  \fi
}
%    \end{macrocode}
%    \end{macro}
%
%    \begin{macro}{\ThPdNu@State@Minus}
%    \begin{macrocode}
\def\ThPdNu@State@Minus#1\fi#2{%
  \fi
  \ifcase\ifx\ThPdNu@FIN#2%
           0%
         \else\ifx0#2%
           1%
         \else\ifnum`#2>48 %
           \ifnum`#2<58 %
             2%
           \else
             9%
           \fi
&        \else\ifx.#2%
$        \else\ifnum`#2=46 % .
           3%
&        \else\ifx-#2%
$        \else\ifnum`#2=45 % -
           4%
&        \else\ifx+#2%
$        \else\ifnum`#2=43 % +
           5%
         \else
           9%
         \fi\fi\fi\fi\fi\fi\ThPdNu@space
    \expandafter\ThPdNu@zero
    \expandafter0%
    \expandafter\thepdfnumber@ErrorUnexpectedEnd
  \or
    \ThPdNu@State@SkipZeros-!%
  \or
    \ThPdNu@State@Int-!#2!%
  \or
    \ThPdNu@State@Dot-!\ThPdNu@zero*\ThPdNu@zero!!%
  \or
    \ThPdNu@State@Plus
  \or
    \ThPdNu@State@Minus
  \else
    \ThPdNu@ReturnError{0}%
  \fi
}
%    \end{macrocode}
%    \end{macro}
%
%    \begin{macro}{\ThPdNu@ReturnError}
%    \begin{macrocode}
\def\ThPdNu@ReturnError#1#2\fi#3\ThPdNu@FIN{%
  \fi
  \ThPdNu@zero
  #1%
  \thepdfnumber@ErrorInvalidToken
}
%    \end{macrocode}
%    \end{macro}
%
% \subsubsection{State definitions for integer part}
%
%    \begin{macro}{\ThPdNu@State@SkipZeros}
%    \begin{macrocode}
\def\ThPdNu@State@SkipZeros#1!#2\fi#3{%
  \fi
  \ifcase\ifx\ThPdNu@FIN#3%
           0%
         \else\ifx0#3%
           1%
         \else\ifnum`#3>48 %
           \ifnum`#3<58 %
             2%
           \else
             9%
           \fi
&        \else\ifx.#3%
$        \else\ifnum`#3=46 % .
           3%
         \else
           9%
         \fi\fi\fi\fi\ThPdNu@space
    \expandafter\ThPdNu@zero
    \expandafter0%
  \or
   \ThPdNu@State@SkipZeros#1!%
  \or
    \ThPdNu@State@Int#1!#3!%
  \or
    \ThPdNu@State@Dot#1!\ThPdNu@zero*\ThPdNu@zero!!%
  \else
    \ThPdNu@ReturnError{0}%
  \fi
}
%    \end{macrocode}
%    \end{macro}
%
%    \begin{macro}{\ThPdNu@State@Int}
%    \begin{macrocode}
\def\ThPdNu@State@Int#1!#2!#3\fi#4{%
  \fi
  \ifcase\ifx\ThPdNu@FIN#4%
           0%
         \else\ifnum`#4>47 %
           \ifnum`#4<58 %
             1%
           \else
             9%
           \fi
&        \else\ifx.#4%
$        \else\ifnum`#4=46 % .
           2%
         \else
           9%
         \fi\fi\fi\ThPdNu@space
    \ThPdNu@ReturnInt{#1#2}%
  \or
    \ThPdNu@State@Int#1!#2#4!%
  \or
    \ThPdNu@State@Dot#1!\ThPdNu@one#2*\ThPdNu@zero!!%
  \else
    \ThPdNu@ReturnError{#1#2}%
  \fi
}
%    \end{macrocode}
%    \end{macro}
%
%    \begin{macro}{\ThPdNu@ReturnInt}
%    \begin{macrocode}
\def\ThPdNu@ReturnInt#1#2\fi{%
  \fi
  \ThPdNu@zero
  #1%
}
%    \end{macrocode}
%    \end{macro}
%
% \subsubsection{State definitions for decimal digits}
%
%    \begin{macro}{\ThPdNu@State@Dot}
%    \begin{macrocode}
\def\ThPdNu@State@Dot#1*#2#3!#4!#5\fi#6{%
  \fi
  \ifcase\ifx\ThPdNu@FIN#6%
           0%
         \else\ifnum`#6>48 %
           \ifnum`#6<58 %
             1%
           \else
             9%
           \fi
         \else\ifx0#6%
           2%
         \else
           9%
         \fi\fi\fi\ThPdNu@space
    \ThPdNu@ReturnNumber#1*#2#3!%
  \or
    \ThPdNu@State@Dot#1*\ThPdNu@one#3#4#6!!%
  \or
    \ThPdNu@State@DotZero#1*#2#3!#4#6!%
  \else
    \ThPdNu@ReturnNumberInvalid#1*#2#3!%
  \fi
}
%    \end{macrocode}
%    \end{macro}
%
%    \begin{macro}{\ThPdNu@State@DotZero}
%    \begin{macrocode}
\def\ThPdNu@State@DotZero#1*#2#3!#4!#5\fi#6{%
  \fi
  \ifcase\ifx\ThPdNu@FIN#6%
           0%
         \else\ifnum`#6>48 %
           \ifnum`#6<58 %
             1%
           \else
             9%
           \fi
         \else\ifx0#6%
           2%
         \else
           9%
         \fi\fi\fi\ThPdNu@space
    \ThPdNu@ReturnNumber#1*#2#3!%
  \or
    \ThPdNu@State@Dot#1*\ThPdNu@one#3#4#6!!%
  \or
    \ThPdNu@State@DotZero#1*#2#3!#4#6!%
  \else
    \ThPdNu@ReturnNumber#1*#2#3!%
  \fi
}
%    \end{macrocode}
%    \end{macro}
%
%    \begin{macro}{\ThPdNu@ReturnNumber}
%    \begin{macrocode}
\def\ThPdNu@ReturnNumber#1!#2#3*#4#5!#6\fi{%
  \fi
  \ifcase#2%
    \expandafter\ThPdNu@firstoftwo
  \else
    \expandafter\ThPdNu@secondoftwo
  \fi
  {%
    \ifcase#4%
      \expandafter\ThPdNu@firstoftwo
    \else
      \expandafter\ThPdNu@secondoftwo
    \fi
    {\ThPdNu@zero 0}%
    {\ThPdNu@zero #1.#5}%
  }{%
    \ifcase#4%
      \expandafter\ThPdNu@firstoftwo
    \else
      \expandafter\ThPdNu@secondoftwo
    \fi
    {\ThPdNu@zero #1#3}%
    {\ThPdNu@zero #1#3.#5}%
  }%
}
%    \end{macrocode}
%    \end{macro}
%
%    \begin{macro}{\ThPdNu@ReturnNumberInvalid}
%    \begin{macrocode}
\def\ThPdNu@ReturnNumberInvalid#1*#2!#3\fi#4\ThPdNu@FIN{%
  \fi
  \iftrue\ThPdNu@ReturnNumber#1*#2!\fi
  \thepdfnumber@ErrorInvalidToken
}
%    \end{macrocode}
%    \end{macro}
%
% \subsection{Norm macro}
%
%    \begin{macro}{\thepdfnumberNormZeroOne}
%    \begin{macrocode}
\def\thepdfnumberNormZeroOne#1{%
  \romannumeral
& \iftrue\expandafter\ThPbNu@StateN@Plus\expandafter\fi
& \detokenize\expandafter{#1}%
& \ThPdNu@FIN
$ \ifx\ThPdNu@FIN#1\ThPdNu@FIN
$   \expandafter\ThPdNu@firstoftwo
$ \else
$   \expandafter\ThPdNu@secondoftwo
$ \fi
$ {%
$   \ThPdNu@zero
$   0\thepdfnumber@ErrorUnexpectedEnd
$ }{%
$   \iftrue\expandafter\ThPbNu@StateN@Plus\expandafter\fi#1\ThPdNu@FIN
$ }%
}
%    \end{macrocode}
%    \end{macro}
%
% \subsubsection{State definitions for sign}
%
%    \begin{macro}{\ThPbNu@StateN@Plus}
%    \begin{macrocode}
\def\ThPbNu@StateN@Plus#1\fi#2{%
  \fi
  \ifcase\ifx\ThPdNu@FIN#2%
           0%
&        \else\ifx-#2%
$        \else\ifnum`#2=45 % -
           1%
         \else\ifx0#2%
           2%
         \else\ifnum`#2>48 %
           \ifnum`#2<58 %
             3%
           \else
             9%
           \fi
&        \else\ifx.#2%
$        \else\ifnum`#2=46 % .
           4%
&        \else\ifx+#2%
$        \else\ifnum`#2=43 % +
           5%
         \else
           9%
         \fi\fi\fi\fi\fi\fi\ThPdNu@space
    \expandafter\ThPdNu@zero
    \expandafter0%
    \expandafter\thepdfnumber@ErrorUnexpectedEnd
  \or
    \ThPbNu@StateN@Minus
  \or
    \ThPbNu@StateN@SkipZeros
  \or
    \ThPdNu@ReturnAndSkip{1}%
  \or
    \ThPbNu@StateN@Dot\ThPdNu@zero!!%
  \or
    \ThPbNu@StateN@Plus
  \else
    \ThPdNu@ReturnError{0}%
  \fi
}
%    \end{macrocode}
%    \end{macro}
%
%    \begin{macro}{\ThPbNu@StateN@Minus}
%    \begin{macrocode}
\def\ThPbNu@StateN@Minus#1\fi#2{%
  \fi
  \ifcase\ifx\ThPdNu@FIN#2%
           0%
         \else\ifnum`#2>47 %
           \ifnum`#2<58 %
             1%
           \else
             9%
           \fi
&        \else\ifx.#2%
$        \else\ifnum`#2=46 % .
           1%
&        \else\ifx-#2%
$        \else\ifnum`#2=45 % -
           2%
&        \else\ifx+#2%
$        \else\ifnum`#2=43 % +
           3%
         \else
           9%
         \fi\fi\fi\fi\fi\ThPdNu@space
    \expandafter\ThPdNu@zero
    \expandafter0%
    \expandafter\thepdfnumber@ErrorUnexpectedEnd
  \or
    \ThPdNu@ReturnAndSkip{0}%
  \or
    \ThPbNu@StateN@Plus
  \or
    \ThPbNu@StateN@Minus
  \else
    \ThPdNu@ReturnError{0}%
  \fi
}
%    \end{macrocode}
%    \end{macro}
%
%    \begin{macro}{\ThPdNu@ReturnAndSkip}
%    \begin{macrocode}
\def\ThPdNu@ReturnAndSkip#1#2\fi#3\ThPdNu@FIN{%
  \fi
  \ThPdNu@zero
  #1%
}
%    \end{macrocode}
%    \end{macro}
%
% \subsubsection{State definitions for integer part}
%
%    \begin{macro}{\ThPbNu@StateN@SkipZeros}
%    \begin{macrocode}
\def\ThPbNu@StateN@SkipZeros#1\fi#2{%
  \fi
  \ifcase\ifx\ThPdNu@FIN#2%
           0%
         \else\ifx0#2%
           1%
         \else\ifnum`#2>48 %
           \ifnum`#2<58 %
             2%
           \else
             9%
           \fi
&        \else\ifx.#2%
$        \else\ifnum`#2=46 % .
           3%
         \else
           9%
         \fi\fi\fi\fi\ThPdNu@space
    \expandafter\ThPdNu@zero
    \expandafter0%
  \or
   \ThPbNu@StateN@SkipZeros%
  \or
    \ThPdNu@ReturnAndSkip{1}%
  \or
    \ThPbNu@StateN@Dot\ThPdNu@zero!!%
  \else
    \ThPdNu@ReturnError{0}%
  \fi
}
%    \end{macrocode}
%    \end{macro}
%
% \subsubsection{State definitions for decimal digits}
%
%    \begin{macro}{\ThPbNu@StateN@Dot}
%    \begin{macrocode}
\def\ThPbNu@StateN@Dot#1#2!#3!#4\fi#5{%
  \fi
  \ifcase\ifx\ThPdNu@FIN#5%
           0%
         \else\ifnum`#5>48 %
           \ifnum`#5<58 %
             1%
           \else
             9%
           \fi
         \else\ifx0#5%
           2%
         \else
           9%
         \fi\fi\fi\ThPdNu@space
    \ThPdNu@ReturnFracNumber#1#2!%
  \or
    \ThPbNu@StateN@Dot\ThPdNu@one#2#3#5!!%
  \or
    \ThPbNu@StateN@DotZero#1#2!#3#5!%
  \else
    \ThPdNu@ReturnFracNumberInvalid#1#2!%
  \fi
}
%    \end{macrocode}
%    \end{macro}
%
%    \begin{macro}{\ThPbNu@StateN@DotZero}
%    \begin{macrocode}
\def\ThPbNu@StateN@DotZero#1#2!#3!#4\fi#5{%
  \fi
  \ifcase\ifx\ThPdNu@FIN#5%
           0%
         \else\ifnum`#5>48 %
           \ifnum`#5<58 %
             1%
           \else
             9%
           \fi
         \else\ifx0#5%
           2%
         \else
           9%
         \fi\fi\fi\ThPdNu@space
    \ThPdNu@ReturnFracNumber#1#2!%
  \or
    \ThPbNu@StateN@Dot\ThPdNu@one#2#3#5!!%
  \or
    \ThPbNu@StateN@DotZero#1#2!#3#5!%
  \else
    \ThPdNu@ReturnFracNumberInvalid#1#2!%
  \fi
}
%    \end{macrocode}
%    \end{macro}
%
%    \begin{macro}{\ThPdNu@ReturnFracNumber}
%    \begin{macrocode}
\def\ThPdNu@ReturnFracNumber#1#2!#3\fi{%
  \fi
  \ifcase#1%
    \expandafter\ThPdNu@firstoftwo
  \else
    \expandafter\ThPdNu@secondoftwo
  \fi
  {\ThPdNu@zero 0}%
  {\ThPdNu@zero .#2}%
}
%    \end{macrocode}
%    \end{macro}
%
%    \begin{macro}{\ThPdNu@ReturnFracNumberInvalid}
%    \begin{macrocode}
\def\ThPdNu@ReturnFracNumberInvalid#1!#2\fi#3\ThPdNu@FIN{%
  \fi
  \iftrue\ThPdNu@ReturnFracNumber#1!\fi
  \thepdfnumber@ErrorInvalidToken
}
%    \end{macrocode}
%    \end{macro}
%
%    \begin{macrocode}
\ThPdNu@AtEnd%
%</package>
%    \end{macrocode}
%
% \section{Test}
%
% \subsection{Catcode checks for loading}
%
%    \begin{macrocode}
%<*test1>
%    \end{macrocode}
%    \begin{macrocode}
\catcode`\{=1 %
\catcode`\}=2 %
\catcode`\#=6 %
\catcode`\@=11 %
\expandafter\ifx\csname count@\endcsname\relax
  \countdef\count@=255 %
\fi
\expandafter\ifx\csname @gobble\endcsname\relax
  \long\def\@gobble#1{}%
\fi
\expandafter\ifx\csname @firstofone\endcsname\relax
  \long\def\@firstofone#1{#1}%
\fi
\expandafter\ifx\csname loop\endcsname\relax
  \expandafter\@firstofone
\else
  \expandafter\@gobble
\fi
{%
  \def\loop#1\repeat{%
    \def\body{#1}%
    \iterate
  }%
  \def\iterate{%
    \body
      \let\next\iterate
    \else
      \let\next\relax
    \fi
    \next
  }%
  \let\repeat=\fi
}%
\def\RestoreCatcodes{}
\count@=0 %
\loop
  \edef\RestoreCatcodes{%
    \RestoreCatcodes
    \catcode\the\count@=\the\catcode\count@\relax
  }%
\ifnum\count@<255 %
  \advance\count@ 1 %
\repeat

\def\RangeCatcodeInvalid#1#2{%
  \count@=#1\relax
  \loop
    \catcode\count@=15 %
  \ifnum\count@<#2\relax
    \advance\count@ 1 %
  \repeat
}
\def\RangeCatcodeCheck#1#2#3{%
  \count@=#1\relax
  \loop
    \ifnum#3=\catcode\count@
    \else
      \errmessage{%
        Character \the\count@\space
        with wrong catcode \the\catcode\count@\space
        instead of \number#3%
      }%
    \fi
  \ifnum\count@<#2\relax
    \advance\count@ 1 %
  \repeat
}
\def\space{ }
\expandafter\ifx\csname LoadCommand\endcsname\relax
  \def\LoadCommand{\input thepdfnumber.sty\relax}%
\fi
\def\Test{%
  \RangeCatcodeInvalid{0}{47}%
  \RangeCatcodeInvalid{58}{64}%
  \RangeCatcodeInvalid{91}{96}%
  \RangeCatcodeInvalid{123}{255}%
  \catcode`\@=12 %
  \catcode`\\=0 %
  \catcode`\%=14 %
  \LoadCommand
  \RangeCatcodeCheck{0}{36}{15}%
  \RangeCatcodeCheck{37}{37}{14}%
  \RangeCatcodeCheck{38}{47}{15}%
  \RangeCatcodeCheck{48}{57}{12}%
  \RangeCatcodeCheck{58}{63}{15}%
  \RangeCatcodeCheck{64}{64}{12}%
  \RangeCatcodeCheck{65}{90}{11}%
  \RangeCatcodeCheck{91}{91}{15}%
  \RangeCatcodeCheck{92}{92}{0}%
  \RangeCatcodeCheck{93}{96}{15}%
  \RangeCatcodeCheck{97}{122}{11}%
  \RangeCatcodeCheck{123}{255}{15}%
  \RestoreCatcodes
}
\Test
\csname @@end\endcsname
\end
%    \end{macrocode}
%    \begin{macrocode}
%</test1>
%    \end{macrocode}
%
% \subsection{Macro tests}
%
% \subsubsection{Preamble}
%
%    \begin{macrocode}
%<*test2>
\catcode`\{=1
\catcode`\}=2
\catcode`\#=6
\catcode`\@=11
\errorcontextlines=10000 %
\def\msg#{\immediate\write16}
\def\space{ }
\begingroup\expandafter\expandafter\expandafter\endgroup
\expandafter\ifx\csname RequirePackage\endcsname\relax
  \input thepdfnumber.sty\relax
\else
  \RequirePackage{thepdfnumber}[2016/05/16]%
\fi
%    \end{macrocode}
%
%    \begin{macro}{\Test}
%    \begin{macrocode}
\def\Test#1#2{%
  \def\TestExpected{#2}%
%<*active>
  \edef\TestExpected{\TestExpected}%
%</active>
  \expandafter\expandafter\expandafter\def
  \expandafter\expandafter\expandafter\TestResult
  \expandafter\expandafter\expandafter{%
    \TestCommand{#1}%
  }%
  \begingroup
    \toks0\expandafter{\TestResult}%
    \toks2\expandafter{\TestExpected}%
    \ifx\TestResult\TestExpected
      \msg{* OK: #1 => \the\toks0}%
    \else
      \errmessage{FAILED: #1 => \the\toks0\space(\the\toks2)}%
    \fi
  \endgroup
}
%    \end{macrocode}
%    \end{macro}
%
%    \begin{macrocode}
\let\thepdfnumber@ErrorUnexpectedEnd\relax
\let\thepdfnumber@ErrorInvalidToken\relax
%    \end{macrocode}
%
% \subsubsection{Test with active catcodes}
%
%    \begin{macrocode}
%<*active>
\catcode`\~=13 %
\def\ThPdNu@Temp#1{%
  \begingroup
    \lccode`\~=`#1 %
  \lowercase{\endgroup
    \edef~%
  }{\string#1}%
  \catcode`#1=13 %
}
\ThPdNu@Temp{.}
\ThPdNu@Temp{-}
\ThPdNu@Temp{+}
%</active>
%    \end{macrocode}
%
% \subsubsection{Test cases for \cs{thepdfnumber}}
%
%    \begin{macro}{\TestCommand}
%    \begin{macrocode}
\let\TestCommand\thepdfnumber
%    \end{macrocode}
%    \end{macro}
%
%    \begin{macrocode}
\Test{0}{0}
\Test{1}{1}
\Test{23}{23}
\Test{0045}{45}
\Test{4500}{4500}
\Test{-0}{0}
\Test{-1}{-1}
\Test{-00002}{-2}
\Test{-010203}{-10203}
\Test{+0}{0}
\Test{+123}{123}

\Test{+-4}{-4}
\Test{+-+-4}{4}
\Test{++--++--++4}{4}
\Test{+++---+++---+++---4}{-4}

\Test{.}{0}
\Test{0.}{0}
\Test{.0}{0}
\Test{0.}{0}
\Test{-.}{0}
\Test{-0.}{0}
\Test{-.0}{0}
\Test{-0.}{0}
\Test{010.020}{10.02}
\Test{123.456}{123.456}
\Test{12.}{12}
\Test{.123}{.123}
\Test{.0001}{.0001}
\Test{.00100}{.001}
\Test{.12003400560078009}{.12003400560078009}
\Test{-.12003400560078009}{-.12003400560078009}
\Test{04.0500000}{4.05}

\Test{}{0\thepdfnumber@ErrorUnexpectedEnd}
\Test{+}{0\thepdfnumber@ErrorUnexpectedEnd}
\Test{-}{0\thepdfnumber@ErrorUnexpectedEnd}
\Test{a}{0\thepdfnumber@ErrorInvalidToken}
\Test{0x1}{0\thepdfnumber@ErrorInvalidToken}
\Test{4x56}{4\thepdfnumber@ErrorInvalidToken}
\Test{012,34}{12\thepdfnumber@ErrorInvalidToken}
\Test{0.12x4}{.12\thepdfnumber@ErrorInvalidToken}
%    \end{macrocode}
%
% \subsubsection{Test cases for \cs{thepdfnumberNormZeroOne}}
%
%    \begin{macro}{\TestCommand}
%    \begin{macrocode}
\let\TestCommand\thepdfnumberNormZeroOne
%    \end{macrocode}
%    \end{macro}
%
%    \begin{macrocode}
\Test{0}{0}
\Test{1}{1}
\Test{23}{1}
\Test{12}{1}
\Test{10}{1}
\Test{0045}{1}
\Test{001}{1}
\Test{-0}{0}
\Test{-1}{0}
\Test{-01}{0}
\Test{-4}{0}

\Test{+-0}{0}
\Test{+-+-1}{1}
\Test{++--++--++1}{1}
\Test{+++---+++---+++---1}{0}

\Test{.}{0}
\Test{0.}{0}
\Test{.0}{0}
\Test{0.}{0}
\Test{-.}{0}
\Test{-0.}{0}
\Test{-.0}{0}
\Test{-0.}{0}
\Test{010.020}{1}
\Test{123.456}{1}
\Test{12.}{1}
\Test{.123}{.123}
\Test{.0001}{.0001}
\Test{.00100}{.001}
\Test{.12003400560078009}{.12003400560078009}
\Test{-.12003400560078009}{0}
\Test{04.0500000}{1}
\Test{0.1200340056}{.1200340056}
\Test{1.05}{1}

\Test{}{0\thepdfnumber@ErrorUnexpectedEnd}
\Test{+}{0\thepdfnumber@ErrorUnexpectedEnd}
\Test{-}{0\thepdfnumber@ErrorUnexpectedEnd}
\Test{a}{0\thepdfnumber@ErrorInvalidToken}
\Test{0x1}{0\thepdfnumber@ErrorInvalidToken}
\Test{4x56}{1}
\Test{012,34}{1}
\Test{.012x4}{.012\thepdfnumber@ErrorInvalidToken}
%    \end{macrocode}
%
%    \begin{macrocode}
\csname @@end\endcsname\end
%</test2>
%    \end{macrocode}
%
% \section{Installation}
%
% \subsection{Download}
%
% \paragraph{Package.} This package is available on
% CTAN\footnote{\url{http://ctan.org/pkg/thepdfnumber}}:
% \begin{description}
% \item[\CTAN{macros/latex/contrib/oberdiek/thepdfnumber.dtx}] The source file.
% \item[\CTAN{macros/latex/contrib/oberdiek/thepdfnumber.pdf}] Documentation.
% \end{description}
%
%
% \paragraph{Bundle.} All the packages of the bundle `oberdiek'
% are also available in a TDS compliant ZIP archive. There
% the packages are already unpacked and the documentation files
% are generated. The files and directories obey the TDS standard.
% \begin{description}
% \item[\CTAN{install/macros/latex/contrib/oberdiek.tds.zip}]
% \end{description}
% \emph{TDS} refers to the standard ``A Directory Structure
% for \TeX\ Files'' (\CTAN{tds/tds.pdf}). Directories
% with \xfile{texmf} in their name are usually organized this way.
%
% \subsection{Bundle installation}
%
% \paragraph{Unpacking.} Unpack the \xfile{oberdiek.tds.zip} in the
% TDS tree (also known as \xfile{texmf} tree) of your choice.
% Example (linux):
% \begin{quote}
%   |unzip oberdiek.tds.zip -d ~/texmf|
% \end{quote}
%
% \paragraph{Script installation.}
% Check the directory \xfile{TDS:scripts/oberdiek/} for
% scripts that need further installation steps.
% Package \xpackage{attachfile2} comes with the Perl script
% \xfile{pdfatfi.pl} that should be installed in such a way
% that it can be called as \texttt{pdfatfi}.
% Example (linux):
% \begin{quote}
%   |chmod +x scripts/oberdiek/pdfatfi.pl|\\
%   |cp scripts/oberdiek/pdfatfi.pl /usr/local/bin/|
% \end{quote}
%
% \subsection{Package installation}
%
% \paragraph{Unpacking.} The \xfile{.dtx} file is a self-extracting
% \docstrip\ archive. The files are extracted by running the
% \xfile{.dtx} through \plainTeX:
% \begin{quote}
%   \verb|tex thepdfnumber.dtx|
% \end{quote}
%
% \paragraph{TDS.} Now the different files must be moved into
% the different directories in your installation TDS tree
% (also known as \xfile{texmf} tree):
% \begin{quote}
% \def\t{^^A
% \begin{tabular}{@{}>{\ttfamily}l@{ $\rightarrow$ }>{\ttfamily}l@{}}
%   thepdfnumber.sty & tex/generic/oberdiek/thepdfnumber.sty\\
%   thepdfnumber.pdf & doc/latex/oberdiek/thepdfnumber.pdf\\
%   test/thepdfnumber-test1.tex & doc/latex/oberdiek/test/thepdfnumber-test1.tex\\
%   test/thepdfnumber-test2.tex & doc/latex/oberdiek/test/thepdfnumber-test2.tex\\
%   test/thepdfnumber-test3.tex & doc/latex/oberdiek/test/thepdfnumber-test3.tex\\
%   thepdfnumber.dtx & source/latex/oberdiek/thepdfnumber.dtx\\
% \end{tabular}^^A
% }^^A
% \sbox0{\t}^^A
% \ifdim\wd0>\linewidth
%   \begingroup
%     \advance\linewidth by\leftmargin
%     \advance\linewidth by\rightmargin
%   \edef\x{\endgroup
%     \def\noexpand\lw{\the\linewidth}^^A
%   }\x
%   \def\lwbox{^^A
%     \leavevmode
%     \hbox to \linewidth{^^A
%       \kern-\leftmargin\relax
%       \hss
%       \usebox0
%       \hss
%       \kern-\rightmargin\relax
%     }^^A
%   }^^A
%   \ifdim\wd0>\lw
%     \sbox0{\small\t}^^A
%     \ifdim\wd0>\linewidth
%       \ifdim\wd0>\lw
%         \sbox0{\footnotesize\t}^^A
%         \ifdim\wd0>\linewidth
%           \ifdim\wd0>\lw
%             \sbox0{\scriptsize\t}^^A
%             \ifdim\wd0>\linewidth
%               \ifdim\wd0>\lw
%                 \sbox0{\tiny\t}^^A
%                 \ifdim\wd0>\linewidth
%                   \lwbox
%                 \else
%                   \usebox0
%                 \fi
%               \else
%                 \lwbox
%               \fi
%             \else
%               \usebox0
%             \fi
%           \else
%             \lwbox
%           \fi
%         \else
%           \usebox0
%         \fi
%       \else
%         \lwbox
%       \fi
%     \else
%       \usebox0
%     \fi
%   \else
%     \lwbox
%   \fi
% \else
%   \usebox0
% \fi
% \end{quote}
% If you have a \xfile{docstrip.cfg} that configures and enables \docstrip's
% TDS installing feature, then some files can already be in the right
% place, see the documentation of \docstrip.
%
% \subsection{Refresh file name databases}
%
% If your \TeX~distribution
% (\teTeX, \mikTeX, \dots) relies on file name databases, you must refresh
% these. For example, \teTeX\ users run \verb|texhash| or
% \verb|mktexlsr|.
%
% \subsection{Some details for the interested}
%
% \paragraph{Attached source.}
%
% The PDF documentation on CTAN also includes the
% \xfile{.dtx} source file. It can be extracted by
% AcrobatReader 6 or higher. Another option is \textsf{pdftk},
% e.g. unpack the file into the current directory:
% \begin{quote}
%   \verb|pdftk thepdfnumber.pdf unpack_files output .|
% \end{quote}
%
% \paragraph{Unpacking with \LaTeX.}
% The \xfile{.dtx} chooses its action depending on the format:
% \begin{description}
% \item[\plainTeX:] Run \docstrip\ and extract the files.
% \item[\LaTeX:] Generate the documentation.
% \end{description}
% If you insist on using \LaTeX\ for \docstrip\ (really,
% \docstrip\ does not need \LaTeX), then inform the autodetect routine
% about your intention:
% \begin{quote}
%   \verb|latex \let\install=y% \iffalse meta-comment
%
% File: thepdfnumber.dtx
% Version: 2016/05/16 v1.1
% Info: Print PDF numbers with minimal digits
%
% Copyright (C) 2011 by
%    Heiko Oberdiek <heiko.oberdiek at googlemail.com>
%    2016
%    https://github.com/ho-tex/oberdiek/issues
%
% This work may be distributed and/or modified under the
% conditions of the LaTeX Project Public License, either
% version 1.3c of this license or (at your option) any later
% version. This version of this license is in
%    http://www.latex-project.org/lppl/lppl-1-3c.txt
% and the latest version of this license is in
%    http://www.latex-project.org/lppl.txt
% and version 1.3 or later is part of all distributions of
% LaTeX version 2005/12/01 or later.
%
% This work has the LPPL maintenance status "maintained".
%
% This Current Maintainer of this work is Heiko Oberdiek.
%
% The Base Interpreter refers to any `TeX-Format',
% because some files are installed in TDS:tex/generic//.
%
% This work consists of the main source file thepdfnumber.dtx
% and the derived files
%    thepdfnumber.sty, thepdfnumber.pdf, thepdfnumber.ins, thepdfnumber.drv,
%    thepdfnumber-test1.tex, thepdfnumber-test2.tex,
%    thepdfnumber-test3.tex.
%
% Distribution:
%    CTAN:macros/latex/contrib/oberdiek/thepdfnumber.dtx
%    CTAN:macros/latex/contrib/oberdiek/thepdfnumber.pdf
%
% Unpacking:
%    (a) If thepdfnumber.ins is present:
%           tex thepdfnumber.ins
%    (b) Without thepdfnumber.ins:
%           tex thepdfnumber.dtx
%    (c) If you insist on using LaTeX
%           latex \let\install=y% \iffalse meta-comment
%
% File: thepdfnumber.dtx
% Version: 2016/05/16 v1.1
% Info: Print PDF numbers with minimal digits
%
% Copyright (C) 2011 by
%    Heiko Oberdiek <heiko.oberdiek at googlemail.com>
%    2016
%    https://github.com/ho-tex/oberdiek/issues
%
% This work may be distributed and/or modified under the
% conditions of the LaTeX Project Public License, either
% version 1.3c of this license or (at your option) any later
% version. This version of this license is in
%    http://www.latex-project.org/lppl/lppl-1-3c.txt
% and the latest version of this license is in
%    http://www.latex-project.org/lppl.txt
% and version 1.3 or later is part of all distributions of
% LaTeX version 2005/12/01 or later.
%
% This work has the LPPL maintenance status "maintained".
%
% This Current Maintainer of this work is Heiko Oberdiek.
%
% The Base Interpreter refers to any `TeX-Format',
% because some files are installed in TDS:tex/generic//.
%
% This work consists of the main source file thepdfnumber.dtx
% and the derived files
%    thepdfnumber.sty, thepdfnumber.pdf, thepdfnumber.ins, thepdfnumber.drv,
%    thepdfnumber-test1.tex, thepdfnumber-test2.tex,
%    thepdfnumber-test3.tex.
%
% Distribution:
%    CTAN:macros/latex/contrib/oberdiek/thepdfnumber.dtx
%    CTAN:macros/latex/contrib/oberdiek/thepdfnumber.pdf
%
% Unpacking:
%    (a) If thepdfnumber.ins is present:
%           tex thepdfnumber.ins
%    (b) Without thepdfnumber.ins:
%           tex thepdfnumber.dtx
%    (c) If you insist on using LaTeX
%           latex \let\install=y\input{thepdfnumber.dtx}
%        (quote the arguments according to the demands of your shell)
%
% Documentation:
%    (a) If thepdfnumber.drv is present:
%           latex thepdfnumber.drv
%    (b) Without thepdfnumber.drv:
%           latex thepdfnumber.dtx; ...
%    The class ltxdoc loads the configuration file ltxdoc.cfg
%    if available. Here you can specify further options, e.g.
%    use A4 as paper format:
%       \PassOptionsToClass{a4paper}{article}
%
%    Programm calls to get the documentation (example):
%       pdflatex thepdfnumber.dtx
%       makeindex -s gind.ist thepdfnumber.idx
%       pdflatex thepdfnumber.dtx
%       makeindex -s gind.ist thepdfnumber.idx
%       pdflatex thepdfnumber.dtx
%
% Installation:
%    TDS:tex/generic/oberdiek/thepdfnumber.sty
%    TDS:doc/latex/oberdiek/thepdfnumber.pdf
%    TDS:doc/latex/oberdiek/test/thepdfnumber-test1.tex
%    TDS:doc/latex/oberdiek/test/thepdfnumber-test2.tex
%    TDS:doc/latex/oberdiek/test/thepdfnumber-test3.tex
%    TDS:source/latex/oberdiek/thepdfnumber.dtx
%
%<*ignore>
\begingroup
  \catcode123=1 %
  \catcode125=2 %
  \def\x{LaTeX2e}%
\expandafter\endgroup
\ifcase 0\ifx\install y1\fi\expandafter
         \ifx\csname processbatchFile\endcsname\relax\else1\fi
         \ifx\fmtname\x\else 1\fi\relax
\else\csname fi\endcsname
%</ignore>
%<*install>
\input docstrip.tex
\Msg{************************************************************************}
\Msg{* Installation}
\Msg{* Package: thepdfnumber 2016/05/16 v1.1 Print PDF numbers with minimal digits (HO)}
\Msg{************************************************************************}

\keepsilent
\askforoverwritefalse

\let\MetaPrefix\relax
\preamble

This is a generated file.

Project: thepdfnumber
Version: 2016/05/16 v1.1

Copyright (C) 2011 by
   Heiko Oberdiek <heiko.oberdiek at googlemail.com>

This work may be distributed and/or modified under the
conditions of the LaTeX Project Public License, either
version 1.3c of this license or (at your option) any later
version. This version of this license is in
   http://www.latex-project.org/lppl/lppl-1-3c.txt
and the latest version of this license is in
   http://www.latex-project.org/lppl.txt
and version 1.3 or later is part of all distributions of
LaTeX version 2005/12/01 or later.

This work has the LPPL maintenance status "maintained".

This Current Maintainer of this work is Heiko Oberdiek.

The Base Interpreter refers to any `TeX-Format',
because some files are installed in TDS:tex/generic//.

This work consists of the main source file thepdfnumber.dtx
and the derived files
   thepdfnumber.sty, thepdfnumber.pdf, thepdfnumber.ins, thepdfnumber.drv,
   thepdfnumber-test1.tex, thepdfnumber-test2.tex,
   thepdfnumber-test3.tex.

\endpreamble
\let\MetaPrefix\DoubleperCent

\generate{%
  \file{thepdfnumber.ins}{\from{thepdfnumber.dtx}{install}}%
  \file{thepdfnumber.drv}{\from{thepdfnumber.dtx}{driver}}%
  \usedir{tex/generic/oberdiek}%
  \file{thepdfnumber.sty}{\from{thepdfnumber.dtx}{package}}%
  \usedir{doc/latex/oberdiek/test}%
  \file{thepdfnumber-test1.tex}{\from{thepdfnumber.dtx}{test1}}%
  \file{thepdfnumber-test2.tex}{\from{thepdfnumber.dtx}{test2}}%
  \file{thepdfnumber-test3.tex}{\from{thepdfnumber.dtx}{test2,active}}%
}

\catcode32=13\relax% active space
\let =\space%
\Msg{************************************************************************}
\Msg{*}
\Msg{* To finish the installation you have to move the following}
\Msg{* file into a directory searched by TeX:}
\Msg{*}
\Msg{*     thepdfnumber.sty}
\Msg{*}
\Msg{* To produce the documentation run the file `thepdfnumber.drv'}
\Msg{* through LaTeX.}
\Msg{*}
\Msg{* Happy TeXing!}
\Msg{*}
\Msg{************************************************************************}

\endbatchfile
%</install>
%<*ignore>
\fi
%</ignore>
%<*driver>
\NeedsTeXFormat{LaTeX2e}
\ProvidesFile{thepdfnumber.drv}%
  [2016/05/16 v1.1 Print PDF numbers with minimal digits (HO)]%
\documentclass{ltxdoc}
\usepackage{holtxdoc}[2011/11/22]
\usepackage{csquotes}
\usepackage[
  backend=bibtex,
  bibencoding=ascii,
  alldates=iso8601,
]{biblatex}[2011/11/13]
\bibliography{oberdiek-source}
\begin{document}
  \DocInput{thepdfnumber.dtx}%
\end{document}
%</driver>
% \fi
%
%
% \CharacterTable
%  {Upper-case    \A\B\C\D\E\F\G\H\I\J\K\L\M\N\O\P\Q\R\S\T\U\V\W\X\Y\Z
%   Lower-case    \a\b\c\d\e\f\g\h\i\j\k\l\m\n\o\p\q\r\s\t\u\v\w\x\y\z
%   Digits        \0\1\2\3\4\5\6\7\8\9
%   Exclamation   \!     Double quote  \"     Hash (number) \#
%   Dollar        \$     Percent       \%     Ampersand     \&
%   Acute accent  \'     Left paren    \(     Right paren   \)
%   Asterisk      \*     Plus          \+     Comma         \,
%   Minus         \-     Point         \.     Solidus       \/
%   Colon         \:     Semicolon     \;     Less than     \<
%   Equals        \=     Greater than  \>     Question mark \?
%   Commercial at \@     Left bracket  \[     Backslash     \\
%   Right bracket \]     Circumflex    \^     Underscore    \_
%   Grave accent  \`     Left brace    \{     Vertical bar  \|
%   Right brace   \}     Tilde         \~}
%
% \GetFileInfo{thepdfnumber.drv}
%
% \title{The \xpackage{thepdfnumber} package}
% \date{2016/05/16 v1.1}
% \author{Heiko Oberdiek\thanks
% {Please report any issues at https://github.com/ho-tex/oberdiek/issues}\\
% \xemail{heiko.oberdiek at googlemail.com}}
%
% \maketitle
%
% \begin{abstract}
% The package converts real numbers to a minimal representation
% that is stripped from leading or trailing zeros,
% plus signs and decimal point if not necessary.
% \end{abstract}
%
% \tableofcontents
%
% \section{Documentation}
%
% \subsection{Introduction}
%
% Dealing with the PDF format, there is sometimes the need to write
% some low level PDF stuff. In case of numbers, the numbers can
% arise from user input (e.g. color or transparency specifications)
% or can be calculated. For example, \hologo{LaTeX}'s \cs{strip@pt}
% makes a good job to output a real number. It automatically suppresses
% the decimal part if the number is an integer. However it leaves
% a leading zero for numbers greater zero and smaller one.
% Thus the package provides macros that can be used with different
% formats, even with \hologo{iniTeX} and generates numbers that
% are valid numbers of the PDF format and whose length is minimal.
%
% \subsection{Usage}
%
% The package \xpackage{thepdfnumber} can be used with \hologo{LaTeX},
% \hologo{plainTeX} or even with \hologo{iniTeX}:
% \begin{quote}
%   |\RequirePackage{thepdfnumber}| \textit{\% \hologo{LaTeX}}\\
%   |\input thepdfnumber.sty| \textit{\% \hologo{plainTeX}/\hologo{iniTeX}}
% \end{quote}
% The package does not need and have package options.
%
% \subsection{User macros}
%
% All user macros are expandable in exact two expansion steps.
%
% \begin{declcs}{thepdfnumber} \M{number}
% \end{declcs}
% Macro \cs{thepdfnumber} takes a number
% as argument and expands to a minimal representation of that number.
% Some examples:
% \begin{quote}
% \begin{tabular}{r@{.: }>{\ttfamily}l@{ $\rightarrow$\ }>{\ttfamily}l@{}}
%   1& +123 & 123\\
%   2& --123 & 123\\
%   3& -01 & -1\\
%   4& 0045 & 45\\
%   5& 1.0 & 1\\
%   6& 1.20 & 1.2\\
%   7& 0.0 & 0\\
%   8& 0.78 & .78\\
%   9& +012.340 & 12.34\\
% \end{tabular}
% \end{quote}
% It reduces the length of the number representation:
% \begin{itemize}
% \item The signs are collapsed and only one minus sign
%   is output if the number is negative (see examples 1, 2, 3, 9).
% \item Leading zeros are removed (4, 8, 9) unless the number is zero (7).
% \item The decimal part is omitted, if the number is an integer (5, 7).
% \item Trailing zeros from the decimal part are stripped (5, 6, 7, 9).
% \end{itemize}
% The resulting number representation can be catched with one of the
% following Perl regular expressions:
% \begin{itemize}
% \item |^0$| (zero)
% \item |^-?[1-9][0-9]*$| (integer)
% \item |^-?[0-9]*\.[0-9]*[1-9]$| (real)
% \end{itemize}
% This is a valid numeric object of the PDF specification \cite[``7.3.3~Numeric Objects'']{pdfspec-iso32000-1}.
%
% \begin{declcs}{thepdfnumberNormZeroOne}
% \end{declcs}
% There are various places in the PDF specification where the
% number is in the domain 0.0 upto 1.0. Macro thepdfnumberNormZeroOne
% automatically adjusts the number to fit into that range. Negative
% numbers are mapped to |0| and numbers greater than one are
% replaced by |1|. Thus the result fits one of the following regular
% expressions:
% \begin{itemize}
% \item |^0$|
% \item |^\.[0-9]*[1-9]$|
% \item |^1$|
% \end{itemize}
% Examples:
% \begin{quote}
% \begin{tabular}{@{}>{\ttfamily}l@{ $\rightarrow$\ }>{\ttfamily}l@{}}
%   -456 & 0\\
%   -0.001 & 0\\
%   0.0 & 0\\
%   0.010 & .01\\
%   0.456 & .456\\
%   1.0 & 1\\
%   01.001 & 1\\
%   4 & 1\\
% \end{tabular}
% \end{quote}
%
% \subsection{Input number}
%
% The user macros expect a number as argument. The number can either
% be given explicitely or as macro that expands in one step
% to an explicite number, because the first token of the argument
% is expanded once.
%
% The explicite number consists of
% \begin{itemize}
% \item optional signs `\texttt{+}' and `\texttt{-}',
% \item digits `\texttt{0}' upto `\texttt{9}' and
% \item an optional dot `\texttt{.}'.
% \end{itemize}
% All tokens must have catcode 12 (other), the default catcodes
% for these characters in \hologo{LaTeX}, \hologo{plainTeX} or
% \hologo{iniTeX}. As Perl regular expression the number is expected
% in one of the following forms:
% \begin{itemize}
% \item |^[+-]*[0-9]+$|
% \item |^[+-]*[0-9]*\.[0-9]*$|
% \end{itemize}
% At least one digit or the dot must be present.
%
% \subsection{Error handling}
%
% The package is not intended for validating numbers or to
% decide if an argument is a number. Therefore it is an usage
% error to use the user macros with arguments that are not
% explicite numbers as specified in the previous sections.
% Nevertheless some error conditions are sometimes recognized.
% Errors are given in form of an undefined command sequence.
% It is the only way to notify \hologo{TeX} in expandable context.
% Expanding to some error text would invalidate the output.
% Currently the following errors are thrown:
% \begin{description}
% \item[\cs{thepdfnumber@ErrorEndMarker}:]
%   Internally the argument parsing uses an end marker that
%   is never called directly. If it is called with valid
%   user input, then this is a bug. Otherwise it means the user input
%   contains nasty stuff.
% \item[\cs{thepdfnumber@ErrorUnexpectedEnd}:] The macros expect at least one digit
%   or the dot, otherwise if the argument is empty or only contains
%   signs, then this error is called.
% \item[\cs{thepdfnumber@ErrorInvalidToken}:] It is called if the number
%   contains other tokens than signs, digits or the dot or the
%   token at the wrong place (e.g. a sign after a digit).
%   In case of \cs{thepdfnumberNormZeroOne} this error condition might
%   not always be detected, because the number parsing might stop
%   at an early point, when the result is already clear (e.g.
%   if the number is negative or will be greater than one).
% \item[Improper alphabetic constant:] This error might be
%   thrown by \hologo{TeX}, if the number contains
%   command tokens instead of characters.
% \end{description}
%
% \StopEventually{
% }
%
% \section{Implementation}
%
%    \begin{macrocode}
%<*package>
%    \end{macrocode}
%
% \subsection{Reload check and package identification}
%    Reload check, especially if the package is not used with \LaTeX.
%    \begin{macrocode}
\begingroup\catcode61\catcode48\catcode32=10\relax%
  \catcode13=5 % ^^M
  \endlinechar=13 %
  \catcode35=6 % #
  \catcode39=12 % '
  \catcode44=12 % ,
  \catcode45=12 % -
  \catcode46=12 % .
  \catcode58=12 % :
  \catcode64=11 % @
  \catcode123=1 % {
  \catcode125=2 % }
  \expandafter\let\expandafter\x\csname ver@thepdfnumber.sty\endcsname
  \ifx\x\relax % plain-TeX, first loading
  \else
    \def\empty{}%
    \ifx\x\empty % LaTeX, first loading,
      % variable is initialized, but \ProvidesPackage not yet seen
    \else
      \expandafter\ifx\csname PackageInfo\endcsname\relax
        \def\x#1#2{%
          \immediate\write-1{Package #1 Info: #2.}%
        }%
      \else
        \def\x#1#2{\PackageInfo{#1}{#2, stopped}}%
      \fi
      \x{thepdfnumber}{The package is already loaded}%
      \aftergroup\endinput
    \fi
  \fi
\endgroup%
%    \end{macrocode}
%    Package identification:
%    \begin{macrocode}
\begingroup\catcode61\catcode48\catcode32=10\relax%
  \catcode13=5 % ^^M
  \endlinechar=13 %
  \catcode35=6 % #
  \catcode39=12 % '
  \catcode40=12 % (
  \catcode41=12 % )
  \catcode44=12 % ,
  \catcode45=12 % -
  \catcode46=12 % .
  \catcode47=12 % /
  \catcode58=12 % :
  \catcode64=11 % @
  \catcode91=12 % [
  \catcode93=12 % ]
  \catcode123=1 % {
  \catcode125=2 % }
  \expandafter\ifx\csname ProvidesPackage\endcsname\relax
    \def\x#1#2#3[#4]{\endgroup
      \immediate\write-1{Package: #3 #4}%
      \xdef#1{#4}%
    }%
  \else
    \def\x#1#2[#3]{\endgroup
      #2[{#3}]%
      \ifx#1\@undefined
        \xdef#1{#3}%
      \fi
      \ifx#1\relax
        \xdef#1{#3}%
      \fi
    }%
  \fi
\expandafter\x\csname ver@thepdfnumber.sty\endcsname
\ProvidesPackage{thepdfnumber}%
  [2016/05/16 v1.1 Print PDF numbers with minimal digits (HO)]%
%    \end{macrocode}
%
% \subsection{Catcodes}
%
%    \begin{macrocode}
\begingroup\catcode61\catcode48\catcode32=10\relax%
  \catcode13=5 % ^^M
  \endlinechar=13 %
  \catcode123=1 % {
  \catcode125=2 % }
  \catcode64=11 % @
  \def\x{\endgroup
    \expandafter\edef\csname ThPdNu@AtEnd\endcsname{%
      \endlinechar=\the\endlinechar\relax
      \catcode13=\the\catcode13\relax
      \catcode32=\the\catcode32\relax
      \catcode35=\the\catcode35\relax
      \catcode61=\the\catcode61\relax
      \catcode64=\the\catcode64\relax
      \catcode123=\the\catcode123\relax
      \catcode125=\the\catcode125\relax
    }%
  }%
\x\catcode61\catcode48\catcode32=10\relax%
\catcode13=5 % ^^M
\endlinechar=13 %
\catcode35=6 % #
\catcode64=11 % @
\catcode123=1 % {
\catcode125=2 % }
\def\TMP@EnsureCode#1#2{%
  \edef\ThPdNu@AtEnd{%
    \ThPdNu@AtEnd
    \catcode#1=\the\catcode#1\relax
  }%
  \catcode#1=#2\relax
}
\TMP@EnsureCode{33}{12}% !
\TMP@EnsureCode{36}{3}% $
\TMP@EnsureCode{38}{4}% &
\TMP@EnsureCode{42}{12}% *
\TMP@EnsureCode{43}{12}% +
\TMP@EnsureCode{45}{12}% -
\TMP@EnsureCode{46}{12}% .
\TMP@EnsureCode{60}{12}% <
\TMP@EnsureCode{62}{12}% >
\TMP@EnsureCode{96}{12}% `
\edef\ThPdNu@AtEnd{\ThPdNu@AtEnd\noexpand\endinput}
%    \end{macrocode}
%
% \subsection{Helper macros}
%
%    \begin{macro}{\ThPdNu@FIN}
%    \begin{macrocode}
\def\ThPdNu@FIN{\thepdfnumber@ErrorEndMarker}
%    \end{macrocode}
%    \end{macro}
%
%    \begin{macro}{\ThPdNu@space}
%    \begin{macrocode}
\def\ThPdNu@space{ }
%    \end{macrocode}
%    \end{macro}
%
%    \begin{macro}{\ThPdNu@zero}
%    \begin{macrocode}
\chardef\ThPdNu@zero=0 %
%    \end{macrocode}
%    \end{macro}
%
%    \begin{macro}{\ThPdNu@one}
%    \begin{macrocode}
\chardef\ThPdNu@one=1 %
%    \end{macrocode}
%    \end{macro}
%
%    \begin{macro}{\ThPdNu@firstoftwo}
%    \begin{macrocode}
\long\def\ThPdNu@firstoftwo#1#2{#1}
%    \end{macrocode}
%    \end{macro}
%
%    \begin{macro}{\ThPdNu@secondoftwo}
%    \begin{macrocode}
\long\def\ThPdNu@secondoftwo#1#2{#2}
%    \end{macrocode}
%    \end{macro}
%
% \subsection{Detect \hologo{eTeX}}
%
%    \begin{macrocode}
\begingroup\expandafter\expandafter\expandafter\endgroup
\expandafter\ifx\csname detokenize\endcsname\relax
  \catcode`\&=14 %
  \catcode`\$=9 %
\else
  \catcode`\&=9 %
  \catcode`\$=14 %
\fi
%    \end{macrocode}
%
% \subsection{User macro \cs{thepdfnumber}}
%
%    \begin{macro}{\thepdfnumber}
%    \begin{macrocode}
\def\thepdfnumber#1{%
  \romannumeral
& \iftrue\expandafter\ThPdNu@State@Plus\expandafter\fi
& \detokenize\expandafter{#1}%
& \ThPdNu@FIN
$ \ifx\ThPdNu@FIN#1\ThPdNu@FIN
$   \expandafter\ThPdNu@firstoftwo
$ \else
$   \expandafter\ThPdNu@secondoftwo
$ \fi
$ {%
$   \ThPdNu@zero
$   0\thepdfnumber@ErrorUnexpectedEnd
$ }{%
$   \iftrue\expandafter\ThPdNu@State@Plus\expandafter\fi#1\ThPdNu@FIN
$ }%
}
%    \end{macrocode}
%    \end{macro}
%
% \subsubsection{State definitions for sign}
%
%    \begin{macro}{\ThPdNu@State@Plus}
%    \begin{macrocode}
\def\ThPdNu@State@Plus#1\fi#2{%
  \fi
  \ifcase\ifx\ThPdNu@FIN#2%
           0%
&        \else\ifx-#2%
$        \else\ifnum`#2=45 % -
           1%
         \else\ifx0#2%
           2%
         \else\ifnum`#2>48 %
           \ifnum`#2<58 %
             3%
           \else
             9%
           \fi
&        \else\ifx.#2%
$        \else\ifnum`#2=46 % .
           4%
&        \else\ifx+#2%
$        \else\ifnum`#2=43 % +
           5%
         \else
           9%
         \fi\fi\fi\fi\fi\fi\ThPdNu@space
    \expandafter\ThPdNu@zero
    \expandafter0%
    \expandafter\thepdfnumber@ErrorUnexpectedEnd
  \or
    \ThPdNu@State@Minus
  \or
    \ThPdNu@State@SkipZeros!%
  \or
    \ThPdNu@State@Int!#2!%
  \or
    \ThPdNu@State@Dot!\ThPdNu@zero*\ThPdNu@zero!!%
  \or
    \ThPdNu@State@Plus
  \else
    \ThPdNu@ReturnError{0}%
  \fi
}
%    \end{macrocode}
%    \end{macro}
%
%    \begin{macro}{\ThPdNu@State@Minus}
%    \begin{macrocode}
\def\ThPdNu@State@Minus#1\fi#2{%
  \fi
  \ifcase\ifx\ThPdNu@FIN#2%
           0%
         \else\ifx0#2%
           1%
         \else\ifnum`#2>48 %
           \ifnum`#2<58 %
             2%
           \else
             9%
           \fi
&        \else\ifx.#2%
$        \else\ifnum`#2=46 % .
           3%
&        \else\ifx-#2%
$        \else\ifnum`#2=45 % -
           4%
&        \else\ifx+#2%
$        \else\ifnum`#2=43 % +
           5%
         \else
           9%
         \fi\fi\fi\fi\fi\fi\ThPdNu@space
    \expandafter\ThPdNu@zero
    \expandafter0%
    \expandafter\thepdfnumber@ErrorUnexpectedEnd
  \or
    \ThPdNu@State@SkipZeros-!%
  \or
    \ThPdNu@State@Int-!#2!%
  \or
    \ThPdNu@State@Dot-!\ThPdNu@zero*\ThPdNu@zero!!%
  \or
    \ThPdNu@State@Plus
  \or
    \ThPdNu@State@Minus
  \else
    \ThPdNu@ReturnError{0}%
  \fi
}
%    \end{macrocode}
%    \end{macro}
%
%    \begin{macro}{\ThPdNu@ReturnError}
%    \begin{macrocode}
\def\ThPdNu@ReturnError#1#2\fi#3\ThPdNu@FIN{%
  \fi
  \ThPdNu@zero
  #1%
  \thepdfnumber@ErrorInvalidToken
}
%    \end{macrocode}
%    \end{macro}
%
% \subsubsection{State definitions for integer part}
%
%    \begin{macro}{\ThPdNu@State@SkipZeros}
%    \begin{macrocode}
\def\ThPdNu@State@SkipZeros#1!#2\fi#3{%
  \fi
  \ifcase\ifx\ThPdNu@FIN#3%
           0%
         \else\ifx0#3%
           1%
         \else\ifnum`#3>48 %
           \ifnum`#3<58 %
             2%
           \else
             9%
           \fi
&        \else\ifx.#3%
$        \else\ifnum`#3=46 % .
           3%
         \else
           9%
         \fi\fi\fi\fi\ThPdNu@space
    \expandafter\ThPdNu@zero
    \expandafter0%
  \or
   \ThPdNu@State@SkipZeros#1!%
  \or
    \ThPdNu@State@Int#1!#3!%
  \or
    \ThPdNu@State@Dot#1!\ThPdNu@zero*\ThPdNu@zero!!%
  \else
    \ThPdNu@ReturnError{0}%
  \fi
}
%    \end{macrocode}
%    \end{macro}
%
%    \begin{macro}{\ThPdNu@State@Int}
%    \begin{macrocode}
\def\ThPdNu@State@Int#1!#2!#3\fi#4{%
  \fi
  \ifcase\ifx\ThPdNu@FIN#4%
           0%
         \else\ifnum`#4>47 %
           \ifnum`#4<58 %
             1%
           \else
             9%
           \fi
&        \else\ifx.#4%
$        \else\ifnum`#4=46 % .
           2%
         \else
           9%
         \fi\fi\fi\ThPdNu@space
    \ThPdNu@ReturnInt{#1#2}%
  \or
    \ThPdNu@State@Int#1!#2#4!%
  \or
    \ThPdNu@State@Dot#1!\ThPdNu@one#2*\ThPdNu@zero!!%
  \else
    \ThPdNu@ReturnError{#1#2}%
  \fi
}
%    \end{macrocode}
%    \end{macro}
%
%    \begin{macro}{\ThPdNu@ReturnInt}
%    \begin{macrocode}
\def\ThPdNu@ReturnInt#1#2\fi{%
  \fi
  \ThPdNu@zero
  #1%
}
%    \end{macrocode}
%    \end{macro}
%
% \subsubsection{State definitions for decimal digits}
%
%    \begin{macro}{\ThPdNu@State@Dot}
%    \begin{macrocode}
\def\ThPdNu@State@Dot#1*#2#3!#4!#5\fi#6{%
  \fi
  \ifcase\ifx\ThPdNu@FIN#6%
           0%
         \else\ifnum`#6>48 %
           \ifnum`#6<58 %
             1%
           \else
             9%
           \fi
         \else\ifx0#6%
           2%
         \else
           9%
         \fi\fi\fi\ThPdNu@space
    \ThPdNu@ReturnNumber#1*#2#3!%
  \or
    \ThPdNu@State@Dot#1*\ThPdNu@one#3#4#6!!%
  \or
    \ThPdNu@State@DotZero#1*#2#3!#4#6!%
  \else
    \ThPdNu@ReturnNumberInvalid#1*#2#3!%
  \fi
}
%    \end{macrocode}
%    \end{macro}
%
%    \begin{macro}{\ThPdNu@State@DotZero}
%    \begin{macrocode}
\def\ThPdNu@State@DotZero#1*#2#3!#4!#5\fi#6{%
  \fi
  \ifcase\ifx\ThPdNu@FIN#6%
           0%
         \else\ifnum`#6>48 %
           \ifnum`#6<58 %
             1%
           \else
             9%
           \fi
         \else\ifx0#6%
           2%
         \else
           9%
         \fi\fi\fi\ThPdNu@space
    \ThPdNu@ReturnNumber#1*#2#3!%
  \or
    \ThPdNu@State@Dot#1*\ThPdNu@one#3#4#6!!%
  \or
    \ThPdNu@State@DotZero#1*#2#3!#4#6!%
  \else
    \ThPdNu@ReturnNumber#1*#2#3!%
  \fi
}
%    \end{macrocode}
%    \end{macro}
%
%    \begin{macro}{\ThPdNu@ReturnNumber}
%    \begin{macrocode}
\def\ThPdNu@ReturnNumber#1!#2#3*#4#5!#6\fi{%
  \fi
  \ifcase#2%
    \expandafter\ThPdNu@firstoftwo
  \else
    \expandafter\ThPdNu@secondoftwo
  \fi
  {%
    \ifcase#4%
      \expandafter\ThPdNu@firstoftwo
    \else
      \expandafter\ThPdNu@secondoftwo
    \fi
    {\ThPdNu@zero 0}%
    {\ThPdNu@zero #1.#5}%
  }{%
    \ifcase#4%
      \expandafter\ThPdNu@firstoftwo
    \else
      \expandafter\ThPdNu@secondoftwo
    \fi
    {\ThPdNu@zero #1#3}%
    {\ThPdNu@zero #1#3.#5}%
  }%
}
%    \end{macrocode}
%    \end{macro}
%
%    \begin{macro}{\ThPdNu@ReturnNumberInvalid}
%    \begin{macrocode}
\def\ThPdNu@ReturnNumberInvalid#1*#2!#3\fi#4\ThPdNu@FIN{%
  \fi
  \iftrue\ThPdNu@ReturnNumber#1*#2!\fi
  \thepdfnumber@ErrorInvalidToken
}
%    \end{macrocode}
%    \end{macro}
%
% \subsection{Norm macro}
%
%    \begin{macro}{\thepdfnumberNormZeroOne}
%    \begin{macrocode}
\def\thepdfnumberNormZeroOne#1{%
  \romannumeral
& \iftrue\expandafter\ThPbNu@StateN@Plus\expandafter\fi
& \detokenize\expandafter{#1}%
& \ThPdNu@FIN
$ \ifx\ThPdNu@FIN#1\ThPdNu@FIN
$   \expandafter\ThPdNu@firstoftwo
$ \else
$   \expandafter\ThPdNu@secondoftwo
$ \fi
$ {%
$   \ThPdNu@zero
$   0\thepdfnumber@ErrorUnexpectedEnd
$ }{%
$   \iftrue\expandafter\ThPbNu@StateN@Plus\expandafter\fi#1\ThPdNu@FIN
$ }%
}
%    \end{macrocode}
%    \end{macro}
%
% \subsubsection{State definitions for sign}
%
%    \begin{macro}{\ThPbNu@StateN@Plus}
%    \begin{macrocode}
\def\ThPbNu@StateN@Plus#1\fi#2{%
  \fi
  \ifcase\ifx\ThPdNu@FIN#2%
           0%
&        \else\ifx-#2%
$        \else\ifnum`#2=45 % -
           1%
         \else\ifx0#2%
           2%
         \else\ifnum`#2>48 %
           \ifnum`#2<58 %
             3%
           \else
             9%
           \fi
&        \else\ifx.#2%
$        \else\ifnum`#2=46 % .
           4%
&        \else\ifx+#2%
$        \else\ifnum`#2=43 % +
           5%
         \else
           9%
         \fi\fi\fi\fi\fi\fi\ThPdNu@space
    \expandafter\ThPdNu@zero
    \expandafter0%
    \expandafter\thepdfnumber@ErrorUnexpectedEnd
  \or
    \ThPbNu@StateN@Minus
  \or
    \ThPbNu@StateN@SkipZeros
  \or
    \ThPdNu@ReturnAndSkip{1}%
  \or
    \ThPbNu@StateN@Dot\ThPdNu@zero!!%
  \or
    \ThPbNu@StateN@Plus
  \else
    \ThPdNu@ReturnError{0}%
  \fi
}
%    \end{macrocode}
%    \end{macro}
%
%    \begin{macro}{\ThPbNu@StateN@Minus}
%    \begin{macrocode}
\def\ThPbNu@StateN@Minus#1\fi#2{%
  \fi
  \ifcase\ifx\ThPdNu@FIN#2%
           0%
         \else\ifnum`#2>47 %
           \ifnum`#2<58 %
             1%
           \else
             9%
           \fi
&        \else\ifx.#2%
$        \else\ifnum`#2=46 % .
           1%
&        \else\ifx-#2%
$        \else\ifnum`#2=45 % -
           2%
&        \else\ifx+#2%
$        \else\ifnum`#2=43 % +
           3%
         \else
           9%
         \fi\fi\fi\fi\fi\ThPdNu@space
    \expandafter\ThPdNu@zero
    \expandafter0%
    \expandafter\thepdfnumber@ErrorUnexpectedEnd
  \or
    \ThPdNu@ReturnAndSkip{0}%
  \or
    \ThPbNu@StateN@Plus
  \or
    \ThPbNu@StateN@Minus
  \else
    \ThPdNu@ReturnError{0}%
  \fi
}
%    \end{macrocode}
%    \end{macro}
%
%    \begin{macro}{\ThPdNu@ReturnAndSkip}
%    \begin{macrocode}
\def\ThPdNu@ReturnAndSkip#1#2\fi#3\ThPdNu@FIN{%
  \fi
  \ThPdNu@zero
  #1%
}
%    \end{macrocode}
%    \end{macro}
%
% \subsubsection{State definitions for integer part}
%
%    \begin{macro}{\ThPbNu@StateN@SkipZeros}
%    \begin{macrocode}
\def\ThPbNu@StateN@SkipZeros#1\fi#2{%
  \fi
  \ifcase\ifx\ThPdNu@FIN#2%
           0%
         \else\ifx0#2%
           1%
         \else\ifnum`#2>48 %
           \ifnum`#2<58 %
             2%
           \else
             9%
           \fi
&        \else\ifx.#2%
$        \else\ifnum`#2=46 % .
           3%
         \else
           9%
         \fi\fi\fi\fi\ThPdNu@space
    \expandafter\ThPdNu@zero
    \expandafter0%
  \or
   \ThPbNu@StateN@SkipZeros%
  \or
    \ThPdNu@ReturnAndSkip{1}%
  \or
    \ThPbNu@StateN@Dot\ThPdNu@zero!!%
  \else
    \ThPdNu@ReturnError{0}%
  \fi
}
%    \end{macrocode}
%    \end{macro}
%
% \subsubsection{State definitions for decimal digits}
%
%    \begin{macro}{\ThPbNu@StateN@Dot}
%    \begin{macrocode}
\def\ThPbNu@StateN@Dot#1#2!#3!#4\fi#5{%
  \fi
  \ifcase\ifx\ThPdNu@FIN#5%
           0%
         \else\ifnum`#5>48 %
           \ifnum`#5<58 %
             1%
           \else
             9%
           \fi
         \else\ifx0#5%
           2%
         \else
           9%
         \fi\fi\fi\ThPdNu@space
    \ThPdNu@ReturnFracNumber#1#2!%
  \or
    \ThPbNu@StateN@Dot\ThPdNu@one#2#3#5!!%
  \or
    \ThPbNu@StateN@DotZero#1#2!#3#5!%
  \else
    \ThPdNu@ReturnFracNumberInvalid#1#2!%
  \fi
}
%    \end{macrocode}
%    \end{macro}
%
%    \begin{macro}{\ThPbNu@StateN@DotZero}
%    \begin{macrocode}
\def\ThPbNu@StateN@DotZero#1#2!#3!#4\fi#5{%
  \fi
  \ifcase\ifx\ThPdNu@FIN#5%
           0%
         \else\ifnum`#5>48 %
           \ifnum`#5<58 %
             1%
           \else
             9%
           \fi
         \else\ifx0#5%
           2%
         \else
           9%
         \fi\fi\fi\ThPdNu@space
    \ThPdNu@ReturnFracNumber#1#2!%
  \or
    \ThPbNu@StateN@Dot\ThPdNu@one#2#3#5!!%
  \or
    \ThPbNu@StateN@DotZero#1#2!#3#5!%
  \else
    \ThPdNu@ReturnFracNumberInvalid#1#2!%
  \fi
}
%    \end{macrocode}
%    \end{macro}
%
%    \begin{macro}{\ThPdNu@ReturnFracNumber}
%    \begin{macrocode}
\def\ThPdNu@ReturnFracNumber#1#2!#3\fi{%
  \fi
  \ifcase#1%
    \expandafter\ThPdNu@firstoftwo
  \else
    \expandafter\ThPdNu@secondoftwo
  \fi
  {\ThPdNu@zero 0}%
  {\ThPdNu@zero .#2}%
}
%    \end{macrocode}
%    \end{macro}
%
%    \begin{macro}{\ThPdNu@ReturnFracNumberInvalid}
%    \begin{macrocode}
\def\ThPdNu@ReturnFracNumberInvalid#1!#2\fi#3\ThPdNu@FIN{%
  \fi
  \iftrue\ThPdNu@ReturnFracNumber#1!\fi
  \thepdfnumber@ErrorInvalidToken
}
%    \end{macrocode}
%    \end{macro}
%
%    \begin{macrocode}
\ThPdNu@AtEnd%
%</package>
%    \end{macrocode}
%
% \section{Test}
%
% \subsection{Catcode checks for loading}
%
%    \begin{macrocode}
%<*test1>
%    \end{macrocode}
%    \begin{macrocode}
\catcode`\{=1 %
\catcode`\}=2 %
\catcode`\#=6 %
\catcode`\@=11 %
\expandafter\ifx\csname count@\endcsname\relax
  \countdef\count@=255 %
\fi
\expandafter\ifx\csname @gobble\endcsname\relax
  \long\def\@gobble#1{}%
\fi
\expandafter\ifx\csname @firstofone\endcsname\relax
  \long\def\@firstofone#1{#1}%
\fi
\expandafter\ifx\csname loop\endcsname\relax
  \expandafter\@firstofone
\else
  \expandafter\@gobble
\fi
{%
  \def\loop#1\repeat{%
    \def\body{#1}%
    \iterate
  }%
  \def\iterate{%
    \body
      \let\next\iterate
    \else
      \let\next\relax
    \fi
    \next
  }%
  \let\repeat=\fi
}%
\def\RestoreCatcodes{}
\count@=0 %
\loop
  \edef\RestoreCatcodes{%
    \RestoreCatcodes
    \catcode\the\count@=\the\catcode\count@\relax
  }%
\ifnum\count@<255 %
  \advance\count@ 1 %
\repeat

\def\RangeCatcodeInvalid#1#2{%
  \count@=#1\relax
  \loop
    \catcode\count@=15 %
  \ifnum\count@<#2\relax
    \advance\count@ 1 %
  \repeat
}
\def\RangeCatcodeCheck#1#2#3{%
  \count@=#1\relax
  \loop
    \ifnum#3=\catcode\count@
    \else
      \errmessage{%
        Character \the\count@\space
        with wrong catcode \the\catcode\count@\space
        instead of \number#3%
      }%
    \fi
  \ifnum\count@<#2\relax
    \advance\count@ 1 %
  \repeat
}
\def\space{ }
\expandafter\ifx\csname LoadCommand\endcsname\relax
  \def\LoadCommand{\input thepdfnumber.sty\relax}%
\fi
\def\Test{%
  \RangeCatcodeInvalid{0}{47}%
  \RangeCatcodeInvalid{58}{64}%
  \RangeCatcodeInvalid{91}{96}%
  \RangeCatcodeInvalid{123}{255}%
  \catcode`\@=12 %
  \catcode`\\=0 %
  \catcode`\%=14 %
  \LoadCommand
  \RangeCatcodeCheck{0}{36}{15}%
  \RangeCatcodeCheck{37}{37}{14}%
  \RangeCatcodeCheck{38}{47}{15}%
  \RangeCatcodeCheck{48}{57}{12}%
  \RangeCatcodeCheck{58}{63}{15}%
  \RangeCatcodeCheck{64}{64}{12}%
  \RangeCatcodeCheck{65}{90}{11}%
  \RangeCatcodeCheck{91}{91}{15}%
  \RangeCatcodeCheck{92}{92}{0}%
  \RangeCatcodeCheck{93}{96}{15}%
  \RangeCatcodeCheck{97}{122}{11}%
  \RangeCatcodeCheck{123}{255}{15}%
  \RestoreCatcodes
}
\Test
\csname @@end\endcsname
\end
%    \end{macrocode}
%    \begin{macrocode}
%</test1>
%    \end{macrocode}
%
% \subsection{Macro tests}
%
% \subsubsection{Preamble}
%
%    \begin{macrocode}
%<*test2>
\catcode`\{=1
\catcode`\}=2
\catcode`\#=6
\catcode`\@=11
\errorcontextlines=10000 %
\def\msg#{\immediate\write16}
\def\space{ }
\begingroup\expandafter\expandafter\expandafter\endgroup
\expandafter\ifx\csname RequirePackage\endcsname\relax
  \input thepdfnumber.sty\relax
\else
  \RequirePackage{thepdfnumber}[2016/05/16]%
\fi
%    \end{macrocode}
%
%    \begin{macro}{\Test}
%    \begin{macrocode}
\def\Test#1#2{%
  \def\TestExpected{#2}%
%<*active>
  \edef\TestExpected{\TestExpected}%
%</active>
  \expandafter\expandafter\expandafter\def
  \expandafter\expandafter\expandafter\TestResult
  \expandafter\expandafter\expandafter{%
    \TestCommand{#1}%
  }%
  \begingroup
    \toks0\expandafter{\TestResult}%
    \toks2\expandafter{\TestExpected}%
    \ifx\TestResult\TestExpected
      \msg{* OK: #1 => \the\toks0}%
    \else
      \errmessage{FAILED: #1 => \the\toks0\space(\the\toks2)}%
    \fi
  \endgroup
}
%    \end{macrocode}
%    \end{macro}
%
%    \begin{macrocode}
\let\thepdfnumber@ErrorUnexpectedEnd\relax
\let\thepdfnumber@ErrorInvalidToken\relax
%    \end{macrocode}
%
% \subsubsection{Test with active catcodes}
%
%    \begin{macrocode}
%<*active>
\catcode`\~=13 %
\def\ThPdNu@Temp#1{%
  \begingroup
    \lccode`\~=`#1 %
  \lowercase{\endgroup
    \edef~%
  }{\string#1}%
  \catcode`#1=13 %
}
\ThPdNu@Temp{.}
\ThPdNu@Temp{-}
\ThPdNu@Temp{+}
%</active>
%    \end{macrocode}
%
% \subsubsection{Test cases for \cs{thepdfnumber}}
%
%    \begin{macro}{\TestCommand}
%    \begin{macrocode}
\let\TestCommand\thepdfnumber
%    \end{macrocode}
%    \end{macro}
%
%    \begin{macrocode}
\Test{0}{0}
\Test{1}{1}
\Test{23}{23}
\Test{0045}{45}
\Test{4500}{4500}
\Test{-0}{0}
\Test{-1}{-1}
\Test{-00002}{-2}
\Test{-010203}{-10203}
\Test{+0}{0}
\Test{+123}{123}

\Test{+-4}{-4}
\Test{+-+-4}{4}
\Test{++--++--++4}{4}
\Test{+++---+++---+++---4}{-4}

\Test{.}{0}
\Test{0.}{0}
\Test{.0}{0}
\Test{0.}{0}
\Test{-.}{0}
\Test{-0.}{0}
\Test{-.0}{0}
\Test{-0.}{0}
\Test{010.020}{10.02}
\Test{123.456}{123.456}
\Test{12.}{12}
\Test{.123}{.123}
\Test{.0001}{.0001}
\Test{.00100}{.001}
\Test{.12003400560078009}{.12003400560078009}
\Test{-.12003400560078009}{-.12003400560078009}
\Test{04.0500000}{4.05}

\Test{}{0\thepdfnumber@ErrorUnexpectedEnd}
\Test{+}{0\thepdfnumber@ErrorUnexpectedEnd}
\Test{-}{0\thepdfnumber@ErrorUnexpectedEnd}
\Test{a}{0\thepdfnumber@ErrorInvalidToken}
\Test{0x1}{0\thepdfnumber@ErrorInvalidToken}
\Test{4x56}{4\thepdfnumber@ErrorInvalidToken}
\Test{012,34}{12\thepdfnumber@ErrorInvalidToken}
\Test{0.12x4}{.12\thepdfnumber@ErrorInvalidToken}
%    \end{macrocode}
%
% \subsubsection{Test cases for \cs{thepdfnumberNormZeroOne}}
%
%    \begin{macro}{\TestCommand}
%    \begin{macrocode}
\let\TestCommand\thepdfnumberNormZeroOne
%    \end{macrocode}
%    \end{macro}
%
%    \begin{macrocode}
\Test{0}{0}
\Test{1}{1}
\Test{23}{1}
\Test{12}{1}
\Test{10}{1}
\Test{0045}{1}
\Test{001}{1}
\Test{-0}{0}
\Test{-1}{0}
\Test{-01}{0}
\Test{-4}{0}

\Test{+-0}{0}
\Test{+-+-1}{1}
\Test{++--++--++1}{1}
\Test{+++---+++---+++---1}{0}

\Test{.}{0}
\Test{0.}{0}
\Test{.0}{0}
\Test{0.}{0}
\Test{-.}{0}
\Test{-0.}{0}
\Test{-.0}{0}
\Test{-0.}{0}
\Test{010.020}{1}
\Test{123.456}{1}
\Test{12.}{1}
\Test{.123}{.123}
\Test{.0001}{.0001}
\Test{.00100}{.001}
\Test{.12003400560078009}{.12003400560078009}
\Test{-.12003400560078009}{0}
\Test{04.0500000}{1}
\Test{0.1200340056}{.1200340056}
\Test{1.05}{1}

\Test{}{0\thepdfnumber@ErrorUnexpectedEnd}
\Test{+}{0\thepdfnumber@ErrorUnexpectedEnd}
\Test{-}{0\thepdfnumber@ErrorUnexpectedEnd}
\Test{a}{0\thepdfnumber@ErrorInvalidToken}
\Test{0x1}{0\thepdfnumber@ErrorInvalidToken}
\Test{4x56}{1}
\Test{012,34}{1}
\Test{.012x4}{.012\thepdfnumber@ErrorInvalidToken}
%    \end{macrocode}
%
%    \begin{macrocode}
\csname @@end\endcsname\end
%</test2>
%    \end{macrocode}
%
% \section{Installation}
%
% \subsection{Download}
%
% \paragraph{Package.} This package is available on
% CTAN\footnote{\url{http://ctan.org/pkg/thepdfnumber}}:
% \begin{description}
% \item[\CTAN{macros/latex/contrib/oberdiek/thepdfnumber.dtx}] The source file.
% \item[\CTAN{macros/latex/contrib/oberdiek/thepdfnumber.pdf}] Documentation.
% \end{description}
%
%
% \paragraph{Bundle.} All the packages of the bundle `oberdiek'
% are also available in a TDS compliant ZIP archive. There
% the packages are already unpacked and the documentation files
% are generated. The files and directories obey the TDS standard.
% \begin{description}
% \item[\CTAN{install/macros/latex/contrib/oberdiek.tds.zip}]
% \end{description}
% \emph{TDS} refers to the standard ``A Directory Structure
% for \TeX\ Files'' (\CTAN{tds/tds.pdf}). Directories
% with \xfile{texmf} in their name are usually organized this way.
%
% \subsection{Bundle installation}
%
% \paragraph{Unpacking.} Unpack the \xfile{oberdiek.tds.zip} in the
% TDS tree (also known as \xfile{texmf} tree) of your choice.
% Example (linux):
% \begin{quote}
%   |unzip oberdiek.tds.zip -d ~/texmf|
% \end{quote}
%
% \paragraph{Script installation.}
% Check the directory \xfile{TDS:scripts/oberdiek/} for
% scripts that need further installation steps.
% Package \xpackage{attachfile2} comes with the Perl script
% \xfile{pdfatfi.pl} that should be installed in such a way
% that it can be called as \texttt{pdfatfi}.
% Example (linux):
% \begin{quote}
%   |chmod +x scripts/oberdiek/pdfatfi.pl|\\
%   |cp scripts/oberdiek/pdfatfi.pl /usr/local/bin/|
% \end{quote}
%
% \subsection{Package installation}
%
% \paragraph{Unpacking.} The \xfile{.dtx} file is a self-extracting
% \docstrip\ archive. The files are extracted by running the
% \xfile{.dtx} through \plainTeX:
% \begin{quote}
%   \verb|tex thepdfnumber.dtx|
% \end{quote}
%
% \paragraph{TDS.} Now the different files must be moved into
% the different directories in your installation TDS tree
% (also known as \xfile{texmf} tree):
% \begin{quote}
% \def\t{^^A
% \begin{tabular}{@{}>{\ttfamily}l@{ $\rightarrow$ }>{\ttfamily}l@{}}
%   thepdfnumber.sty & tex/generic/oberdiek/thepdfnumber.sty\\
%   thepdfnumber.pdf & doc/latex/oberdiek/thepdfnumber.pdf\\
%   test/thepdfnumber-test1.tex & doc/latex/oberdiek/test/thepdfnumber-test1.tex\\
%   test/thepdfnumber-test2.tex & doc/latex/oberdiek/test/thepdfnumber-test2.tex\\
%   test/thepdfnumber-test3.tex & doc/latex/oberdiek/test/thepdfnumber-test3.tex\\
%   thepdfnumber.dtx & source/latex/oberdiek/thepdfnumber.dtx\\
% \end{tabular}^^A
% }^^A
% \sbox0{\t}^^A
% \ifdim\wd0>\linewidth
%   \begingroup
%     \advance\linewidth by\leftmargin
%     \advance\linewidth by\rightmargin
%   \edef\x{\endgroup
%     \def\noexpand\lw{\the\linewidth}^^A
%   }\x
%   \def\lwbox{^^A
%     \leavevmode
%     \hbox to \linewidth{^^A
%       \kern-\leftmargin\relax
%       \hss
%       \usebox0
%       \hss
%       \kern-\rightmargin\relax
%     }^^A
%   }^^A
%   \ifdim\wd0>\lw
%     \sbox0{\small\t}^^A
%     \ifdim\wd0>\linewidth
%       \ifdim\wd0>\lw
%         \sbox0{\footnotesize\t}^^A
%         \ifdim\wd0>\linewidth
%           \ifdim\wd0>\lw
%             \sbox0{\scriptsize\t}^^A
%             \ifdim\wd0>\linewidth
%               \ifdim\wd0>\lw
%                 \sbox0{\tiny\t}^^A
%                 \ifdim\wd0>\linewidth
%                   \lwbox
%                 \else
%                   \usebox0
%                 \fi
%               \else
%                 \lwbox
%               \fi
%             \else
%               \usebox0
%             \fi
%           \else
%             \lwbox
%           \fi
%         \else
%           \usebox0
%         \fi
%       \else
%         \lwbox
%       \fi
%     \else
%       \usebox0
%     \fi
%   \else
%     \lwbox
%   \fi
% \else
%   \usebox0
% \fi
% \end{quote}
% If you have a \xfile{docstrip.cfg} that configures and enables \docstrip's
% TDS installing feature, then some files can already be in the right
% place, see the documentation of \docstrip.
%
% \subsection{Refresh file name databases}
%
% If your \TeX~distribution
% (\teTeX, \mikTeX, \dots) relies on file name databases, you must refresh
% these. For example, \teTeX\ users run \verb|texhash| or
% \verb|mktexlsr|.
%
% \subsection{Some details for the interested}
%
% \paragraph{Attached source.}
%
% The PDF documentation on CTAN also includes the
% \xfile{.dtx} source file. It can be extracted by
% AcrobatReader 6 or higher. Another option is \textsf{pdftk},
% e.g. unpack the file into the current directory:
% \begin{quote}
%   \verb|pdftk thepdfnumber.pdf unpack_files output .|
% \end{quote}
%
% \paragraph{Unpacking with \LaTeX.}
% The \xfile{.dtx} chooses its action depending on the format:
% \begin{description}
% \item[\plainTeX:] Run \docstrip\ and extract the files.
% \item[\LaTeX:] Generate the documentation.
% \end{description}
% If you insist on using \LaTeX\ for \docstrip\ (really,
% \docstrip\ does not need \LaTeX), then inform the autodetect routine
% about your intention:
% \begin{quote}
%   \verb|latex \let\install=y\input{thepdfnumber.dtx}|
% \end{quote}
% Do not forget to quote the argument according to the demands
% of your shell.
%
% \paragraph{Generating the documentation.}
% You can use both the \xfile{.dtx} or the \xfile{.drv} to generate
% the documentation. The process can be configured by the
% configuration file \xfile{ltxdoc.cfg}. For instance, put this
% line into this file, if you want to have A4 as paper format:
% \begin{quote}
%   \verb|\PassOptionsToClass{a4paper}{article}|
% \end{quote}
% An example follows how to generate the
% documentation with pdf\LaTeX:
% \begin{quote}
%\begin{verbatim}
%pdflatex thepdfnumber.dtx
%makeindex -s gind.ist thepdfnumber.idx
%pdflatex thepdfnumber.dtx
%makeindex -s gind.ist thepdfnumber.idx
%pdflatex thepdfnumber.dtx
%\end{verbatim}
% \end{quote}
%
% \printbibliography[
%   heading=bibnumbered,
% ]
%
% \begin{History}
%   \begin{Version}{2011/11/24 v1.0}
%   \item
%     First version.
%   \end{Version}
%   \begin{Version}{2016/05/16 v1.1}
%   \item
%     Documentation updates.
%   \end{Version}
% \end{History}
%
% \PrintIndex
%
% \Finale
\endinput

%        (quote the arguments according to the demands of your shell)
%
% Documentation:
%    (a) If thepdfnumber.drv is present:
%           latex thepdfnumber.drv
%    (b) Without thepdfnumber.drv:
%           latex thepdfnumber.dtx; ...
%    The class ltxdoc loads the configuration file ltxdoc.cfg
%    if available. Here you can specify further options, e.g.
%    use A4 as paper format:
%       \PassOptionsToClass{a4paper}{article}
%
%    Programm calls to get the documentation (example):
%       pdflatex thepdfnumber.dtx
%       makeindex -s gind.ist thepdfnumber.idx
%       pdflatex thepdfnumber.dtx
%       makeindex -s gind.ist thepdfnumber.idx
%       pdflatex thepdfnumber.dtx
%
% Installation:
%    TDS:tex/generic/oberdiek/thepdfnumber.sty
%    TDS:doc/latex/oberdiek/thepdfnumber.pdf
%    TDS:doc/latex/oberdiek/test/thepdfnumber-test1.tex
%    TDS:doc/latex/oberdiek/test/thepdfnumber-test2.tex
%    TDS:doc/latex/oberdiek/test/thepdfnumber-test3.tex
%    TDS:source/latex/oberdiek/thepdfnumber.dtx
%
%<*ignore>
\begingroup
  \catcode123=1 %
  \catcode125=2 %
  \def\x{LaTeX2e}%
\expandafter\endgroup
\ifcase 0\ifx\install y1\fi\expandafter
         \ifx\csname processbatchFile\endcsname\relax\else1\fi
         \ifx\fmtname\x\else 1\fi\relax
\else\csname fi\endcsname
%</ignore>
%<*install>
\input docstrip.tex
\Msg{************************************************************************}
\Msg{* Installation}
\Msg{* Package: thepdfnumber 2016/05/16 v1.1 Print PDF numbers with minimal digits (HO)}
\Msg{************************************************************************}

\keepsilent
\askforoverwritefalse

\let\MetaPrefix\relax
\preamble

This is a generated file.

Project: thepdfnumber
Version: 2016/05/16 v1.1

Copyright (C) 2011 by
   Heiko Oberdiek <heiko.oberdiek at googlemail.com>

This work may be distributed and/or modified under the
conditions of the LaTeX Project Public License, either
version 1.3c of this license or (at your option) any later
version. This version of this license is in
   http://www.latex-project.org/lppl/lppl-1-3c.txt
and the latest version of this license is in
   http://www.latex-project.org/lppl.txt
and version 1.3 or later is part of all distributions of
LaTeX version 2005/12/01 or later.

This work has the LPPL maintenance status "maintained".

This Current Maintainer of this work is Heiko Oberdiek.

The Base Interpreter refers to any `TeX-Format',
because some files are installed in TDS:tex/generic//.

This work consists of the main source file thepdfnumber.dtx
and the derived files
   thepdfnumber.sty, thepdfnumber.pdf, thepdfnumber.ins, thepdfnumber.drv,
   thepdfnumber-test1.tex, thepdfnumber-test2.tex,
   thepdfnumber-test3.tex.

\endpreamble
\let\MetaPrefix\DoubleperCent

\generate{%
  \file{thepdfnumber.ins}{\from{thepdfnumber.dtx}{install}}%
  \file{thepdfnumber.drv}{\from{thepdfnumber.dtx}{driver}}%
  \usedir{tex/generic/oberdiek}%
  \file{thepdfnumber.sty}{\from{thepdfnumber.dtx}{package}}%
  \usedir{doc/latex/oberdiek/test}%
  \file{thepdfnumber-test1.tex}{\from{thepdfnumber.dtx}{test1}}%
  \file{thepdfnumber-test2.tex}{\from{thepdfnumber.dtx}{test2}}%
  \file{thepdfnumber-test3.tex}{\from{thepdfnumber.dtx}{test2,active}}%
}

\catcode32=13\relax% active space
\let =\space%
\Msg{************************************************************************}
\Msg{*}
\Msg{* To finish the installation you have to move the following}
\Msg{* file into a directory searched by TeX:}
\Msg{*}
\Msg{*     thepdfnumber.sty}
\Msg{*}
\Msg{* To produce the documentation run the file `thepdfnumber.drv'}
\Msg{* through LaTeX.}
\Msg{*}
\Msg{* Happy TeXing!}
\Msg{*}
\Msg{************************************************************************}

\endbatchfile
%</install>
%<*ignore>
\fi
%</ignore>
%<*driver>
\NeedsTeXFormat{LaTeX2e}
\ProvidesFile{thepdfnumber.drv}%
  [2016/05/16 v1.1 Print PDF numbers with minimal digits (HO)]%
\documentclass{ltxdoc}
\usepackage{holtxdoc}[2011/11/22]
\usepackage{csquotes}
\usepackage[
  backend=bibtex,
  bibencoding=ascii,
  alldates=iso8601,
]{biblatex}[2011/11/13]
\bibliography{oberdiek-source}
\begin{document}
  \DocInput{thepdfnumber.dtx}%
\end{document}
%</driver>
% \fi
%
%
% \CharacterTable
%  {Upper-case    \A\B\C\D\E\F\G\H\I\J\K\L\M\N\O\P\Q\R\S\T\U\V\W\X\Y\Z
%   Lower-case    \a\b\c\d\e\f\g\h\i\j\k\l\m\n\o\p\q\r\s\t\u\v\w\x\y\z
%   Digits        \0\1\2\3\4\5\6\7\8\9
%   Exclamation   \!     Double quote  \"     Hash (number) \#
%   Dollar        \$     Percent       \%     Ampersand     \&
%   Acute accent  \'     Left paren    \(     Right paren   \)
%   Asterisk      \*     Plus          \+     Comma         \,
%   Minus         \-     Point         \.     Solidus       \/
%   Colon         \:     Semicolon     \;     Less than     \<
%   Equals        \=     Greater than  \>     Question mark \?
%   Commercial at \@     Left bracket  \[     Backslash     \\
%   Right bracket \]     Circumflex    \^     Underscore    \_
%   Grave accent  \`     Left brace    \{     Vertical bar  \|
%   Right brace   \}     Tilde         \~}
%
% \GetFileInfo{thepdfnumber.drv}
%
% \title{The \xpackage{thepdfnumber} package}
% \date{2016/05/16 v1.1}
% \author{Heiko Oberdiek\thanks
% {Please report any issues at https://github.com/ho-tex/oberdiek/issues}\\
% \xemail{heiko.oberdiek at googlemail.com}}
%
% \maketitle
%
% \begin{abstract}
% The package converts real numbers to a minimal representation
% that is stripped from leading or trailing zeros,
% plus signs and decimal point if not necessary.
% \end{abstract}
%
% \tableofcontents
%
% \section{Documentation}
%
% \subsection{Introduction}
%
% Dealing with the PDF format, there is sometimes the need to write
% some low level PDF stuff. In case of numbers, the numbers can
% arise from user input (e.g. color or transparency specifications)
% or can be calculated. For example, \hologo{LaTeX}'s \cs{strip@pt}
% makes a good job to output a real number. It automatically suppresses
% the decimal part if the number is an integer. However it leaves
% a leading zero for numbers greater zero and smaller one.
% Thus the package provides macros that can be used with different
% formats, even with \hologo{iniTeX} and generates numbers that
% are valid numbers of the PDF format and whose length is minimal.
%
% \subsection{Usage}
%
% The package \xpackage{thepdfnumber} can be used with \hologo{LaTeX},
% \hologo{plainTeX} or even with \hologo{iniTeX}:
% \begin{quote}
%   |\RequirePackage{thepdfnumber}| \textit{\% \hologo{LaTeX}}\\
%   |\input thepdfnumber.sty| \textit{\% \hologo{plainTeX}/\hologo{iniTeX}}
% \end{quote}
% The package does not need and have package options.
%
% \subsection{User macros}
%
% All user macros are expandable in exact two expansion steps.
%
% \begin{declcs}{thepdfnumber} \M{number}
% \end{declcs}
% Macro \cs{thepdfnumber} takes a number
% as argument and expands to a minimal representation of that number.
% Some examples:
% \begin{quote}
% \begin{tabular}{r@{.: }>{\ttfamily}l@{ $\rightarrow$\ }>{\ttfamily}l@{}}
%   1& +123 & 123\\
%   2& --123 & 123\\
%   3& -01 & -1\\
%   4& 0045 & 45\\
%   5& 1.0 & 1\\
%   6& 1.20 & 1.2\\
%   7& 0.0 & 0\\
%   8& 0.78 & .78\\
%   9& +012.340 & 12.34\\
% \end{tabular}
% \end{quote}
% It reduces the length of the number representation:
% \begin{itemize}
% \item The signs are collapsed and only one minus sign
%   is output if the number is negative (see examples 1, 2, 3, 9).
% \item Leading zeros are removed (4, 8, 9) unless the number is zero (7).
% \item The decimal part is omitted, if the number is an integer (5, 7).
% \item Trailing zeros from the decimal part are stripped (5, 6, 7, 9).
% \end{itemize}
% The resulting number representation can be catched with one of the
% following Perl regular expressions:
% \begin{itemize}
% \item |^0$| (zero)
% \item |^-?[1-9][0-9]*$| (integer)
% \item |^-?[0-9]*\.[0-9]*[1-9]$| (real)
% \end{itemize}
% This is a valid numeric object of the PDF specification \cite[``7.3.3~Numeric Objects'']{pdfspec-iso32000-1}.
%
% \begin{declcs}{thepdfnumberNormZeroOne}
% \end{declcs}
% There are various places in the PDF specification where the
% number is in the domain 0.0 upto 1.0. Macro thepdfnumberNormZeroOne
% automatically adjusts the number to fit into that range. Negative
% numbers are mapped to |0| and numbers greater than one are
% replaced by |1|. Thus the result fits one of the following regular
% expressions:
% \begin{itemize}
% \item |^0$|
% \item |^\.[0-9]*[1-9]$|
% \item |^1$|
% \end{itemize}
% Examples:
% \begin{quote}
% \begin{tabular}{@{}>{\ttfamily}l@{ $\rightarrow$\ }>{\ttfamily}l@{}}
%   -456 & 0\\
%   -0.001 & 0\\
%   0.0 & 0\\
%   0.010 & .01\\
%   0.456 & .456\\
%   1.0 & 1\\
%   01.001 & 1\\
%   4 & 1\\
% \end{tabular}
% \end{quote}
%
% \subsection{Input number}
%
% The user macros expect a number as argument. The number can either
% be given explicitely or as macro that expands in one step
% to an explicite number, because the first token of the argument
% is expanded once.
%
% The explicite number consists of
% \begin{itemize}
% \item optional signs `\texttt{+}' and `\texttt{-}',
% \item digits `\texttt{0}' upto `\texttt{9}' and
% \item an optional dot `\texttt{.}'.
% \end{itemize}
% All tokens must have catcode 12 (other), the default catcodes
% for these characters in \hologo{LaTeX}, \hologo{plainTeX} or
% \hologo{iniTeX}. As Perl regular expression the number is expected
% in one of the following forms:
% \begin{itemize}
% \item |^[+-]*[0-9]+$|
% \item |^[+-]*[0-9]*\.[0-9]*$|
% \end{itemize}
% At least one digit or the dot must be present.
%
% \subsection{Error handling}
%
% The package is not intended for validating numbers or to
% decide if an argument is a number. Therefore it is an usage
% error to use the user macros with arguments that are not
% explicite numbers as specified in the previous sections.
% Nevertheless some error conditions are sometimes recognized.
% Errors are given in form of an undefined command sequence.
% It is the only way to notify \hologo{TeX} in expandable context.
% Expanding to some error text would invalidate the output.
% Currently the following errors are thrown:
% \begin{description}
% \item[\cs{thepdfnumber@ErrorEndMarker}:]
%   Internally the argument parsing uses an end marker that
%   is never called directly. If it is called with valid
%   user input, then this is a bug. Otherwise it means the user input
%   contains nasty stuff.
% \item[\cs{thepdfnumber@ErrorUnexpectedEnd}:] The macros expect at least one digit
%   or the dot, otherwise if the argument is empty or only contains
%   signs, then this error is called.
% \item[\cs{thepdfnumber@ErrorInvalidToken}:] It is called if the number
%   contains other tokens than signs, digits or the dot or the
%   token at the wrong place (e.g. a sign after a digit).
%   In case of \cs{thepdfnumberNormZeroOne} this error condition might
%   not always be detected, because the number parsing might stop
%   at an early point, when the result is already clear (e.g.
%   if the number is negative or will be greater than one).
% \item[Improper alphabetic constant:] This error might be
%   thrown by \hologo{TeX}, if the number contains
%   command tokens instead of characters.
% \end{description}
%
% \StopEventually{
% }
%
% \section{Implementation}
%
%    \begin{macrocode}
%<*package>
%    \end{macrocode}
%
% \subsection{Reload check and package identification}
%    Reload check, especially if the package is not used with \LaTeX.
%    \begin{macrocode}
\begingroup\catcode61\catcode48\catcode32=10\relax%
  \catcode13=5 % ^^M
  \endlinechar=13 %
  \catcode35=6 % #
  \catcode39=12 % '
  \catcode44=12 % ,
  \catcode45=12 % -
  \catcode46=12 % .
  \catcode58=12 % :
  \catcode64=11 % @
  \catcode123=1 % {
  \catcode125=2 % }
  \expandafter\let\expandafter\x\csname ver@thepdfnumber.sty\endcsname
  \ifx\x\relax % plain-TeX, first loading
  \else
    \def\empty{}%
    \ifx\x\empty % LaTeX, first loading,
      % variable is initialized, but \ProvidesPackage not yet seen
    \else
      \expandafter\ifx\csname PackageInfo\endcsname\relax
        \def\x#1#2{%
          \immediate\write-1{Package #1 Info: #2.}%
        }%
      \else
        \def\x#1#2{\PackageInfo{#1}{#2, stopped}}%
      \fi
      \x{thepdfnumber}{The package is already loaded}%
      \aftergroup\endinput
    \fi
  \fi
\endgroup%
%    \end{macrocode}
%    Package identification:
%    \begin{macrocode}
\begingroup\catcode61\catcode48\catcode32=10\relax%
  \catcode13=5 % ^^M
  \endlinechar=13 %
  \catcode35=6 % #
  \catcode39=12 % '
  \catcode40=12 % (
  \catcode41=12 % )
  \catcode44=12 % ,
  \catcode45=12 % -
  \catcode46=12 % .
  \catcode47=12 % /
  \catcode58=12 % :
  \catcode64=11 % @
  \catcode91=12 % [
  \catcode93=12 % ]
  \catcode123=1 % {
  \catcode125=2 % }
  \expandafter\ifx\csname ProvidesPackage\endcsname\relax
    \def\x#1#2#3[#4]{\endgroup
      \immediate\write-1{Package: #3 #4}%
      \xdef#1{#4}%
    }%
  \else
    \def\x#1#2[#3]{\endgroup
      #2[{#3}]%
      \ifx#1\@undefined
        \xdef#1{#3}%
      \fi
      \ifx#1\relax
        \xdef#1{#3}%
      \fi
    }%
  \fi
\expandafter\x\csname ver@thepdfnumber.sty\endcsname
\ProvidesPackage{thepdfnumber}%
  [2016/05/16 v1.1 Print PDF numbers with minimal digits (HO)]%
%    \end{macrocode}
%
% \subsection{Catcodes}
%
%    \begin{macrocode}
\begingroup\catcode61\catcode48\catcode32=10\relax%
  \catcode13=5 % ^^M
  \endlinechar=13 %
  \catcode123=1 % {
  \catcode125=2 % }
  \catcode64=11 % @
  \def\x{\endgroup
    \expandafter\edef\csname ThPdNu@AtEnd\endcsname{%
      \endlinechar=\the\endlinechar\relax
      \catcode13=\the\catcode13\relax
      \catcode32=\the\catcode32\relax
      \catcode35=\the\catcode35\relax
      \catcode61=\the\catcode61\relax
      \catcode64=\the\catcode64\relax
      \catcode123=\the\catcode123\relax
      \catcode125=\the\catcode125\relax
    }%
  }%
\x\catcode61\catcode48\catcode32=10\relax%
\catcode13=5 % ^^M
\endlinechar=13 %
\catcode35=6 % #
\catcode64=11 % @
\catcode123=1 % {
\catcode125=2 % }
\def\TMP@EnsureCode#1#2{%
  \edef\ThPdNu@AtEnd{%
    \ThPdNu@AtEnd
    \catcode#1=\the\catcode#1\relax
  }%
  \catcode#1=#2\relax
}
\TMP@EnsureCode{33}{12}% !
\TMP@EnsureCode{36}{3}% $
\TMP@EnsureCode{38}{4}% &
\TMP@EnsureCode{42}{12}% *
\TMP@EnsureCode{43}{12}% +
\TMP@EnsureCode{45}{12}% -
\TMP@EnsureCode{46}{12}% .
\TMP@EnsureCode{60}{12}% <
\TMP@EnsureCode{62}{12}% >
\TMP@EnsureCode{96}{12}% `
\edef\ThPdNu@AtEnd{\ThPdNu@AtEnd\noexpand\endinput}
%    \end{macrocode}
%
% \subsection{Helper macros}
%
%    \begin{macro}{\ThPdNu@FIN}
%    \begin{macrocode}
\def\ThPdNu@FIN{\thepdfnumber@ErrorEndMarker}
%    \end{macrocode}
%    \end{macro}
%
%    \begin{macro}{\ThPdNu@space}
%    \begin{macrocode}
\def\ThPdNu@space{ }
%    \end{macrocode}
%    \end{macro}
%
%    \begin{macro}{\ThPdNu@zero}
%    \begin{macrocode}
\chardef\ThPdNu@zero=0 %
%    \end{macrocode}
%    \end{macro}
%
%    \begin{macro}{\ThPdNu@one}
%    \begin{macrocode}
\chardef\ThPdNu@one=1 %
%    \end{macrocode}
%    \end{macro}
%
%    \begin{macro}{\ThPdNu@firstoftwo}
%    \begin{macrocode}
\long\def\ThPdNu@firstoftwo#1#2{#1}
%    \end{macrocode}
%    \end{macro}
%
%    \begin{macro}{\ThPdNu@secondoftwo}
%    \begin{macrocode}
\long\def\ThPdNu@secondoftwo#1#2{#2}
%    \end{macrocode}
%    \end{macro}
%
% \subsection{Detect \hologo{eTeX}}
%
%    \begin{macrocode}
\begingroup\expandafter\expandafter\expandafter\endgroup
\expandafter\ifx\csname detokenize\endcsname\relax
  \catcode`\&=14 %
  \catcode`\$=9 %
\else
  \catcode`\&=9 %
  \catcode`\$=14 %
\fi
%    \end{macrocode}
%
% \subsection{User macro \cs{thepdfnumber}}
%
%    \begin{macro}{\thepdfnumber}
%    \begin{macrocode}
\def\thepdfnumber#1{%
  \romannumeral
& \iftrue\expandafter\ThPdNu@State@Plus\expandafter\fi
& \detokenize\expandafter{#1}%
& \ThPdNu@FIN
$ \ifx\ThPdNu@FIN#1\ThPdNu@FIN
$   \expandafter\ThPdNu@firstoftwo
$ \else
$   \expandafter\ThPdNu@secondoftwo
$ \fi
$ {%
$   \ThPdNu@zero
$   0\thepdfnumber@ErrorUnexpectedEnd
$ }{%
$   \iftrue\expandafter\ThPdNu@State@Plus\expandafter\fi#1\ThPdNu@FIN
$ }%
}
%    \end{macrocode}
%    \end{macro}
%
% \subsubsection{State definitions for sign}
%
%    \begin{macro}{\ThPdNu@State@Plus}
%    \begin{macrocode}
\def\ThPdNu@State@Plus#1\fi#2{%
  \fi
  \ifcase\ifx\ThPdNu@FIN#2%
           0%
&        \else\ifx-#2%
$        \else\ifnum`#2=45 % -
           1%
         \else\ifx0#2%
           2%
         \else\ifnum`#2>48 %
           \ifnum`#2<58 %
             3%
           \else
             9%
           \fi
&        \else\ifx.#2%
$        \else\ifnum`#2=46 % .
           4%
&        \else\ifx+#2%
$        \else\ifnum`#2=43 % +
           5%
         \else
           9%
         \fi\fi\fi\fi\fi\fi\ThPdNu@space
    \expandafter\ThPdNu@zero
    \expandafter0%
    \expandafter\thepdfnumber@ErrorUnexpectedEnd
  \or
    \ThPdNu@State@Minus
  \or
    \ThPdNu@State@SkipZeros!%
  \or
    \ThPdNu@State@Int!#2!%
  \or
    \ThPdNu@State@Dot!\ThPdNu@zero*\ThPdNu@zero!!%
  \or
    \ThPdNu@State@Plus
  \else
    \ThPdNu@ReturnError{0}%
  \fi
}
%    \end{macrocode}
%    \end{macro}
%
%    \begin{macro}{\ThPdNu@State@Minus}
%    \begin{macrocode}
\def\ThPdNu@State@Minus#1\fi#2{%
  \fi
  \ifcase\ifx\ThPdNu@FIN#2%
           0%
         \else\ifx0#2%
           1%
         \else\ifnum`#2>48 %
           \ifnum`#2<58 %
             2%
           \else
             9%
           \fi
&        \else\ifx.#2%
$        \else\ifnum`#2=46 % .
           3%
&        \else\ifx-#2%
$        \else\ifnum`#2=45 % -
           4%
&        \else\ifx+#2%
$        \else\ifnum`#2=43 % +
           5%
         \else
           9%
         \fi\fi\fi\fi\fi\fi\ThPdNu@space
    \expandafter\ThPdNu@zero
    \expandafter0%
    \expandafter\thepdfnumber@ErrorUnexpectedEnd
  \or
    \ThPdNu@State@SkipZeros-!%
  \or
    \ThPdNu@State@Int-!#2!%
  \or
    \ThPdNu@State@Dot-!\ThPdNu@zero*\ThPdNu@zero!!%
  \or
    \ThPdNu@State@Plus
  \or
    \ThPdNu@State@Minus
  \else
    \ThPdNu@ReturnError{0}%
  \fi
}
%    \end{macrocode}
%    \end{macro}
%
%    \begin{macro}{\ThPdNu@ReturnError}
%    \begin{macrocode}
\def\ThPdNu@ReturnError#1#2\fi#3\ThPdNu@FIN{%
  \fi
  \ThPdNu@zero
  #1%
  \thepdfnumber@ErrorInvalidToken
}
%    \end{macrocode}
%    \end{macro}
%
% \subsubsection{State definitions for integer part}
%
%    \begin{macro}{\ThPdNu@State@SkipZeros}
%    \begin{macrocode}
\def\ThPdNu@State@SkipZeros#1!#2\fi#3{%
  \fi
  \ifcase\ifx\ThPdNu@FIN#3%
           0%
         \else\ifx0#3%
           1%
         \else\ifnum`#3>48 %
           \ifnum`#3<58 %
             2%
           \else
             9%
           \fi
&        \else\ifx.#3%
$        \else\ifnum`#3=46 % .
           3%
         \else
           9%
         \fi\fi\fi\fi\ThPdNu@space
    \expandafter\ThPdNu@zero
    \expandafter0%
  \or
   \ThPdNu@State@SkipZeros#1!%
  \or
    \ThPdNu@State@Int#1!#3!%
  \or
    \ThPdNu@State@Dot#1!\ThPdNu@zero*\ThPdNu@zero!!%
  \else
    \ThPdNu@ReturnError{0}%
  \fi
}
%    \end{macrocode}
%    \end{macro}
%
%    \begin{macro}{\ThPdNu@State@Int}
%    \begin{macrocode}
\def\ThPdNu@State@Int#1!#2!#3\fi#4{%
  \fi
  \ifcase\ifx\ThPdNu@FIN#4%
           0%
         \else\ifnum`#4>47 %
           \ifnum`#4<58 %
             1%
           \else
             9%
           \fi
&        \else\ifx.#4%
$        \else\ifnum`#4=46 % .
           2%
         \else
           9%
         \fi\fi\fi\ThPdNu@space
    \ThPdNu@ReturnInt{#1#2}%
  \or
    \ThPdNu@State@Int#1!#2#4!%
  \or
    \ThPdNu@State@Dot#1!\ThPdNu@one#2*\ThPdNu@zero!!%
  \else
    \ThPdNu@ReturnError{#1#2}%
  \fi
}
%    \end{macrocode}
%    \end{macro}
%
%    \begin{macro}{\ThPdNu@ReturnInt}
%    \begin{macrocode}
\def\ThPdNu@ReturnInt#1#2\fi{%
  \fi
  \ThPdNu@zero
  #1%
}
%    \end{macrocode}
%    \end{macro}
%
% \subsubsection{State definitions for decimal digits}
%
%    \begin{macro}{\ThPdNu@State@Dot}
%    \begin{macrocode}
\def\ThPdNu@State@Dot#1*#2#3!#4!#5\fi#6{%
  \fi
  \ifcase\ifx\ThPdNu@FIN#6%
           0%
         \else\ifnum`#6>48 %
           \ifnum`#6<58 %
             1%
           \else
             9%
           \fi
         \else\ifx0#6%
           2%
         \else
           9%
         \fi\fi\fi\ThPdNu@space
    \ThPdNu@ReturnNumber#1*#2#3!%
  \or
    \ThPdNu@State@Dot#1*\ThPdNu@one#3#4#6!!%
  \or
    \ThPdNu@State@DotZero#1*#2#3!#4#6!%
  \else
    \ThPdNu@ReturnNumberInvalid#1*#2#3!%
  \fi
}
%    \end{macrocode}
%    \end{macro}
%
%    \begin{macro}{\ThPdNu@State@DotZero}
%    \begin{macrocode}
\def\ThPdNu@State@DotZero#1*#2#3!#4!#5\fi#6{%
  \fi
  \ifcase\ifx\ThPdNu@FIN#6%
           0%
         \else\ifnum`#6>48 %
           \ifnum`#6<58 %
             1%
           \else
             9%
           \fi
         \else\ifx0#6%
           2%
         \else
           9%
         \fi\fi\fi\ThPdNu@space
    \ThPdNu@ReturnNumber#1*#2#3!%
  \or
    \ThPdNu@State@Dot#1*\ThPdNu@one#3#4#6!!%
  \or
    \ThPdNu@State@DotZero#1*#2#3!#4#6!%
  \else
    \ThPdNu@ReturnNumber#1*#2#3!%
  \fi
}
%    \end{macrocode}
%    \end{macro}
%
%    \begin{macro}{\ThPdNu@ReturnNumber}
%    \begin{macrocode}
\def\ThPdNu@ReturnNumber#1!#2#3*#4#5!#6\fi{%
  \fi
  \ifcase#2%
    \expandafter\ThPdNu@firstoftwo
  \else
    \expandafter\ThPdNu@secondoftwo
  \fi
  {%
    \ifcase#4%
      \expandafter\ThPdNu@firstoftwo
    \else
      \expandafter\ThPdNu@secondoftwo
    \fi
    {\ThPdNu@zero 0}%
    {\ThPdNu@zero #1.#5}%
  }{%
    \ifcase#4%
      \expandafter\ThPdNu@firstoftwo
    \else
      \expandafter\ThPdNu@secondoftwo
    \fi
    {\ThPdNu@zero #1#3}%
    {\ThPdNu@zero #1#3.#5}%
  }%
}
%    \end{macrocode}
%    \end{macro}
%
%    \begin{macro}{\ThPdNu@ReturnNumberInvalid}
%    \begin{macrocode}
\def\ThPdNu@ReturnNumberInvalid#1*#2!#3\fi#4\ThPdNu@FIN{%
  \fi
  \iftrue\ThPdNu@ReturnNumber#1*#2!\fi
  \thepdfnumber@ErrorInvalidToken
}
%    \end{macrocode}
%    \end{macro}
%
% \subsection{Norm macro}
%
%    \begin{macro}{\thepdfnumberNormZeroOne}
%    \begin{macrocode}
\def\thepdfnumberNormZeroOne#1{%
  \romannumeral
& \iftrue\expandafter\ThPbNu@StateN@Plus\expandafter\fi
& \detokenize\expandafter{#1}%
& \ThPdNu@FIN
$ \ifx\ThPdNu@FIN#1\ThPdNu@FIN
$   \expandafter\ThPdNu@firstoftwo
$ \else
$   \expandafter\ThPdNu@secondoftwo
$ \fi
$ {%
$   \ThPdNu@zero
$   0\thepdfnumber@ErrorUnexpectedEnd
$ }{%
$   \iftrue\expandafter\ThPbNu@StateN@Plus\expandafter\fi#1\ThPdNu@FIN
$ }%
}
%    \end{macrocode}
%    \end{macro}
%
% \subsubsection{State definitions for sign}
%
%    \begin{macro}{\ThPbNu@StateN@Plus}
%    \begin{macrocode}
\def\ThPbNu@StateN@Plus#1\fi#2{%
  \fi
  \ifcase\ifx\ThPdNu@FIN#2%
           0%
&        \else\ifx-#2%
$        \else\ifnum`#2=45 % -
           1%
         \else\ifx0#2%
           2%
         \else\ifnum`#2>48 %
           \ifnum`#2<58 %
             3%
           \else
             9%
           \fi
&        \else\ifx.#2%
$        \else\ifnum`#2=46 % .
           4%
&        \else\ifx+#2%
$        \else\ifnum`#2=43 % +
           5%
         \else
           9%
         \fi\fi\fi\fi\fi\fi\ThPdNu@space
    \expandafter\ThPdNu@zero
    \expandafter0%
    \expandafter\thepdfnumber@ErrorUnexpectedEnd
  \or
    \ThPbNu@StateN@Minus
  \or
    \ThPbNu@StateN@SkipZeros
  \or
    \ThPdNu@ReturnAndSkip{1}%
  \or
    \ThPbNu@StateN@Dot\ThPdNu@zero!!%
  \or
    \ThPbNu@StateN@Plus
  \else
    \ThPdNu@ReturnError{0}%
  \fi
}
%    \end{macrocode}
%    \end{macro}
%
%    \begin{macro}{\ThPbNu@StateN@Minus}
%    \begin{macrocode}
\def\ThPbNu@StateN@Minus#1\fi#2{%
  \fi
  \ifcase\ifx\ThPdNu@FIN#2%
           0%
         \else\ifnum`#2>47 %
           \ifnum`#2<58 %
             1%
           \else
             9%
           \fi
&        \else\ifx.#2%
$        \else\ifnum`#2=46 % .
           1%
&        \else\ifx-#2%
$        \else\ifnum`#2=45 % -
           2%
&        \else\ifx+#2%
$        \else\ifnum`#2=43 % +
           3%
         \else
           9%
         \fi\fi\fi\fi\fi\ThPdNu@space
    \expandafter\ThPdNu@zero
    \expandafter0%
    \expandafter\thepdfnumber@ErrorUnexpectedEnd
  \or
    \ThPdNu@ReturnAndSkip{0}%
  \or
    \ThPbNu@StateN@Plus
  \or
    \ThPbNu@StateN@Minus
  \else
    \ThPdNu@ReturnError{0}%
  \fi
}
%    \end{macrocode}
%    \end{macro}
%
%    \begin{macro}{\ThPdNu@ReturnAndSkip}
%    \begin{macrocode}
\def\ThPdNu@ReturnAndSkip#1#2\fi#3\ThPdNu@FIN{%
  \fi
  \ThPdNu@zero
  #1%
}
%    \end{macrocode}
%    \end{macro}
%
% \subsubsection{State definitions for integer part}
%
%    \begin{macro}{\ThPbNu@StateN@SkipZeros}
%    \begin{macrocode}
\def\ThPbNu@StateN@SkipZeros#1\fi#2{%
  \fi
  \ifcase\ifx\ThPdNu@FIN#2%
           0%
         \else\ifx0#2%
           1%
         \else\ifnum`#2>48 %
           \ifnum`#2<58 %
             2%
           \else
             9%
           \fi
&        \else\ifx.#2%
$        \else\ifnum`#2=46 % .
           3%
         \else
           9%
         \fi\fi\fi\fi\ThPdNu@space
    \expandafter\ThPdNu@zero
    \expandafter0%
  \or
   \ThPbNu@StateN@SkipZeros%
  \or
    \ThPdNu@ReturnAndSkip{1}%
  \or
    \ThPbNu@StateN@Dot\ThPdNu@zero!!%
  \else
    \ThPdNu@ReturnError{0}%
  \fi
}
%    \end{macrocode}
%    \end{macro}
%
% \subsubsection{State definitions for decimal digits}
%
%    \begin{macro}{\ThPbNu@StateN@Dot}
%    \begin{macrocode}
\def\ThPbNu@StateN@Dot#1#2!#3!#4\fi#5{%
  \fi
  \ifcase\ifx\ThPdNu@FIN#5%
           0%
         \else\ifnum`#5>48 %
           \ifnum`#5<58 %
             1%
           \else
             9%
           \fi
         \else\ifx0#5%
           2%
         \else
           9%
         \fi\fi\fi\ThPdNu@space
    \ThPdNu@ReturnFracNumber#1#2!%
  \or
    \ThPbNu@StateN@Dot\ThPdNu@one#2#3#5!!%
  \or
    \ThPbNu@StateN@DotZero#1#2!#3#5!%
  \else
    \ThPdNu@ReturnFracNumberInvalid#1#2!%
  \fi
}
%    \end{macrocode}
%    \end{macro}
%
%    \begin{macro}{\ThPbNu@StateN@DotZero}
%    \begin{macrocode}
\def\ThPbNu@StateN@DotZero#1#2!#3!#4\fi#5{%
  \fi
  \ifcase\ifx\ThPdNu@FIN#5%
           0%
         \else\ifnum`#5>48 %
           \ifnum`#5<58 %
             1%
           \else
             9%
           \fi
         \else\ifx0#5%
           2%
         \else
           9%
         \fi\fi\fi\ThPdNu@space
    \ThPdNu@ReturnFracNumber#1#2!%
  \or
    \ThPbNu@StateN@Dot\ThPdNu@one#2#3#5!!%
  \or
    \ThPbNu@StateN@DotZero#1#2!#3#5!%
  \else
    \ThPdNu@ReturnFracNumberInvalid#1#2!%
  \fi
}
%    \end{macrocode}
%    \end{macro}
%
%    \begin{macro}{\ThPdNu@ReturnFracNumber}
%    \begin{macrocode}
\def\ThPdNu@ReturnFracNumber#1#2!#3\fi{%
  \fi
  \ifcase#1%
    \expandafter\ThPdNu@firstoftwo
  \else
    \expandafter\ThPdNu@secondoftwo
  \fi
  {\ThPdNu@zero 0}%
  {\ThPdNu@zero .#2}%
}
%    \end{macrocode}
%    \end{macro}
%
%    \begin{macro}{\ThPdNu@ReturnFracNumberInvalid}
%    \begin{macrocode}
\def\ThPdNu@ReturnFracNumberInvalid#1!#2\fi#3\ThPdNu@FIN{%
  \fi
  \iftrue\ThPdNu@ReturnFracNumber#1!\fi
  \thepdfnumber@ErrorInvalidToken
}
%    \end{macrocode}
%    \end{macro}
%
%    \begin{macrocode}
\ThPdNu@AtEnd%
%</package>
%    \end{macrocode}
%
% \section{Test}
%
% \subsection{Catcode checks for loading}
%
%    \begin{macrocode}
%<*test1>
%    \end{macrocode}
%    \begin{macrocode}
\catcode`\{=1 %
\catcode`\}=2 %
\catcode`\#=6 %
\catcode`\@=11 %
\expandafter\ifx\csname count@\endcsname\relax
  \countdef\count@=255 %
\fi
\expandafter\ifx\csname @gobble\endcsname\relax
  \long\def\@gobble#1{}%
\fi
\expandafter\ifx\csname @firstofone\endcsname\relax
  \long\def\@firstofone#1{#1}%
\fi
\expandafter\ifx\csname loop\endcsname\relax
  \expandafter\@firstofone
\else
  \expandafter\@gobble
\fi
{%
  \def\loop#1\repeat{%
    \def\body{#1}%
    \iterate
  }%
  \def\iterate{%
    \body
      \let\next\iterate
    \else
      \let\next\relax
    \fi
    \next
  }%
  \let\repeat=\fi
}%
\def\RestoreCatcodes{}
\count@=0 %
\loop
  \edef\RestoreCatcodes{%
    \RestoreCatcodes
    \catcode\the\count@=\the\catcode\count@\relax
  }%
\ifnum\count@<255 %
  \advance\count@ 1 %
\repeat

\def\RangeCatcodeInvalid#1#2{%
  \count@=#1\relax
  \loop
    \catcode\count@=15 %
  \ifnum\count@<#2\relax
    \advance\count@ 1 %
  \repeat
}
\def\RangeCatcodeCheck#1#2#3{%
  \count@=#1\relax
  \loop
    \ifnum#3=\catcode\count@
    \else
      \errmessage{%
        Character \the\count@\space
        with wrong catcode \the\catcode\count@\space
        instead of \number#3%
      }%
    \fi
  \ifnum\count@<#2\relax
    \advance\count@ 1 %
  \repeat
}
\def\space{ }
\expandafter\ifx\csname LoadCommand\endcsname\relax
  \def\LoadCommand{\input thepdfnumber.sty\relax}%
\fi
\def\Test{%
  \RangeCatcodeInvalid{0}{47}%
  \RangeCatcodeInvalid{58}{64}%
  \RangeCatcodeInvalid{91}{96}%
  \RangeCatcodeInvalid{123}{255}%
  \catcode`\@=12 %
  \catcode`\\=0 %
  \catcode`\%=14 %
  \LoadCommand
  \RangeCatcodeCheck{0}{36}{15}%
  \RangeCatcodeCheck{37}{37}{14}%
  \RangeCatcodeCheck{38}{47}{15}%
  \RangeCatcodeCheck{48}{57}{12}%
  \RangeCatcodeCheck{58}{63}{15}%
  \RangeCatcodeCheck{64}{64}{12}%
  \RangeCatcodeCheck{65}{90}{11}%
  \RangeCatcodeCheck{91}{91}{15}%
  \RangeCatcodeCheck{92}{92}{0}%
  \RangeCatcodeCheck{93}{96}{15}%
  \RangeCatcodeCheck{97}{122}{11}%
  \RangeCatcodeCheck{123}{255}{15}%
  \RestoreCatcodes
}
\Test
\csname @@end\endcsname
\end
%    \end{macrocode}
%    \begin{macrocode}
%</test1>
%    \end{macrocode}
%
% \subsection{Macro tests}
%
% \subsubsection{Preamble}
%
%    \begin{macrocode}
%<*test2>
\catcode`\{=1
\catcode`\}=2
\catcode`\#=6
\catcode`\@=11
\errorcontextlines=10000 %
\def\msg#{\immediate\write16}
\def\space{ }
\begingroup\expandafter\expandafter\expandafter\endgroup
\expandafter\ifx\csname RequirePackage\endcsname\relax
  \input thepdfnumber.sty\relax
\else
  \RequirePackage{thepdfnumber}[2016/05/16]%
\fi
%    \end{macrocode}
%
%    \begin{macro}{\Test}
%    \begin{macrocode}
\def\Test#1#2{%
  \def\TestExpected{#2}%
%<*active>
  \edef\TestExpected{\TestExpected}%
%</active>
  \expandafter\expandafter\expandafter\def
  \expandafter\expandafter\expandafter\TestResult
  \expandafter\expandafter\expandafter{%
    \TestCommand{#1}%
  }%
  \begingroup
    \toks0\expandafter{\TestResult}%
    \toks2\expandafter{\TestExpected}%
    \ifx\TestResult\TestExpected
      \msg{* OK: #1 => \the\toks0}%
    \else
      \errmessage{FAILED: #1 => \the\toks0\space(\the\toks2)}%
    \fi
  \endgroup
}
%    \end{macrocode}
%    \end{macro}
%
%    \begin{macrocode}
\let\thepdfnumber@ErrorUnexpectedEnd\relax
\let\thepdfnumber@ErrorInvalidToken\relax
%    \end{macrocode}
%
% \subsubsection{Test with active catcodes}
%
%    \begin{macrocode}
%<*active>
\catcode`\~=13 %
\def\ThPdNu@Temp#1{%
  \begingroup
    \lccode`\~=`#1 %
  \lowercase{\endgroup
    \edef~%
  }{\string#1}%
  \catcode`#1=13 %
}
\ThPdNu@Temp{.}
\ThPdNu@Temp{-}
\ThPdNu@Temp{+}
%</active>
%    \end{macrocode}
%
% \subsubsection{Test cases for \cs{thepdfnumber}}
%
%    \begin{macro}{\TestCommand}
%    \begin{macrocode}
\let\TestCommand\thepdfnumber
%    \end{macrocode}
%    \end{macro}
%
%    \begin{macrocode}
\Test{0}{0}
\Test{1}{1}
\Test{23}{23}
\Test{0045}{45}
\Test{4500}{4500}
\Test{-0}{0}
\Test{-1}{-1}
\Test{-00002}{-2}
\Test{-010203}{-10203}
\Test{+0}{0}
\Test{+123}{123}

\Test{+-4}{-4}
\Test{+-+-4}{4}
\Test{++--++--++4}{4}
\Test{+++---+++---+++---4}{-4}

\Test{.}{0}
\Test{0.}{0}
\Test{.0}{0}
\Test{0.}{0}
\Test{-.}{0}
\Test{-0.}{0}
\Test{-.0}{0}
\Test{-0.}{0}
\Test{010.020}{10.02}
\Test{123.456}{123.456}
\Test{12.}{12}
\Test{.123}{.123}
\Test{.0001}{.0001}
\Test{.00100}{.001}
\Test{.12003400560078009}{.12003400560078009}
\Test{-.12003400560078009}{-.12003400560078009}
\Test{04.0500000}{4.05}

\Test{}{0\thepdfnumber@ErrorUnexpectedEnd}
\Test{+}{0\thepdfnumber@ErrorUnexpectedEnd}
\Test{-}{0\thepdfnumber@ErrorUnexpectedEnd}
\Test{a}{0\thepdfnumber@ErrorInvalidToken}
\Test{0x1}{0\thepdfnumber@ErrorInvalidToken}
\Test{4x56}{4\thepdfnumber@ErrorInvalidToken}
\Test{012,34}{12\thepdfnumber@ErrorInvalidToken}
\Test{0.12x4}{.12\thepdfnumber@ErrorInvalidToken}
%    \end{macrocode}
%
% \subsubsection{Test cases for \cs{thepdfnumberNormZeroOne}}
%
%    \begin{macro}{\TestCommand}
%    \begin{macrocode}
\let\TestCommand\thepdfnumberNormZeroOne
%    \end{macrocode}
%    \end{macro}
%
%    \begin{macrocode}
\Test{0}{0}
\Test{1}{1}
\Test{23}{1}
\Test{12}{1}
\Test{10}{1}
\Test{0045}{1}
\Test{001}{1}
\Test{-0}{0}
\Test{-1}{0}
\Test{-01}{0}
\Test{-4}{0}

\Test{+-0}{0}
\Test{+-+-1}{1}
\Test{++--++--++1}{1}
\Test{+++---+++---+++---1}{0}

\Test{.}{0}
\Test{0.}{0}
\Test{.0}{0}
\Test{0.}{0}
\Test{-.}{0}
\Test{-0.}{0}
\Test{-.0}{0}
\Test{-0.}{0}
\Test{010.020}{1}
\Test{123.456}{1}
\Test{12.}{1}
\Test{.123}{.123}
\Test{.0001}{.0001}
\Test{.00100}{.001}
\Test{.12003400560078009}{.12003400560078009}
\Test{-.12003400560078009}{0}
\Test{04.0500000}{1}
\Test{0.1200340056}{.1200340056}
\Test{1.05}{1}

\Test{}{0\thepdfnumber@ErrorUnexpectedEnd}
\Test{+}{0\thepdfnumber@ErrorUnexpectedEnd}
\Test{-}{0\thepdfnumber@ErrorUnexpectedEnd}
\Test{a}{0\thepdfnumber@ErrorInvalidToken}
\Test{0x1}{0\thepdfnumber@ErrorInvalidToken}
\Test{4x56}{1}
\Test{012,34}{1}
\Test{.012x4}{.012\thepdfnumber@ErrorInvalidToken}
%    \end{macrocode}
%
%    \begin{macrocode}
\csname @@end\endcsname\end
%</test2>
%    \end{macrocode}
%
% \section{Installation}
%
% \subsection{Download}
%
% \paragraph{Package.} This package is available on
% CTAN\footnote{\url{http://ctan.org/pkg/thepdfnumber}}:
% \begin{description}
% \item[\CTAN{macros/latex/contrib/oberdiek/thepdfnumber.dtx}] The source file.
% \item[\CTAN{macros/latex/contrib/oberdiek/thepdfnumber.pdf}] Documentation.
% \end{description}
%
%
% \paragraph{Bundle.} All the packages of the bundle `oberdiek'
% are also available in a TDS compliant ZIP archive. There
% the packages are already unpacked and the documentation files
% are generated. The files and directories obey the TDS standard.
% \begin{description}
% \item[\CTAN{install/macros/latex/contrib/oberdiek.tds.zip}]
% \end{description}
% \emph{TDS} refers to the standard ``A Directory Structure
% for \TeX\ Files'' (\CTAN{tds/tds.pdf}). Directories
% with \xfile{texmf} in their name are usually organized this way.
%
% \subsection{Bundle installation}
%
% \paragraph{Unpacking.} Unpack the \xfile{oberdiek.tds.zip} in the
% TDS tree (also known as \xfile{texmf} tree) of your choice.
% Example (linux):
% \begin{quote}
%   |unzip oberdiek.tds.zip -d ~/texmf|
% \end{quote}
%
% \paragraph{Script installation.}
% Check the directory \xfile{TDS:scripts/oberdiek/} for
% scripts that need further installation steps.
% Package \xpackage{attachfile2} comes with the Perl script
% \xfile{pdfatfi.pl} that should be installed in such a way
% that it can be called as \texttt{pdfatfi}.
% Example (linux):
% \begin{quote}
%   |chmod +x scripts/oberdiek/pdfatfi.pl|\\
%   |cp scripts/oberdiek/pdfatfi.pl /usr/local/bin/|
% \end{quote}
%
% \subsection{Package installation}
%
% \paragraph{Unpacking.} The \xfile{.dtx} file is a self-extracting
% \docstrip\ archive. The files are extracted by running the
% \xfile{.dtx} through \plainTeX:
% \begin{quote}
%   \verb|tex thepdfnumber.dtx|
% \end{quote}
%
% \paragraph{TDS.} Now the different files must be moved into
% the different directories in your installation TDS tree
% (also known as \xfile{texmf} tree):
% \begin{quote}
% \def\t{^^A
% \begin{tabular}{@{}>{\ttfamily}l@{ $\rightarrow$ }>{\ttfamily}l@{}}
%   thepdfnumber.sty & tex/generic/oberdiek/thepdfnumber.sty\\
%   thepdfnumber.pdf & doc/latex/oberdiek/thepdfnumber.pdf\\
%   test/thepdfnumber-test1.tex & doc/latex/oberdiek/test/thepdfnumber-test1.tex\\
%   test/thepdfnumber-test2.tex & doc/latex/oberdiek/test/thepdfnumber-test2.tex\\
%   test/thepdfnumber-test3.tex & doc/latex/oberdiek/test/thepdfnumber-test3.tex\\
%   thepdfnumber.dtx & source/latex/oberdiek/thepdfnumber.dtx\\
% \end{tabular}^^A
% }^^A
% \sbox0{\t}^^A
% \ifdim\wd0>\linewidth
%   \begingroup
%     \advance\linewidth by\leftmargin
%     \advance\linewidth by\rightmargin
%   \edef\x{\endgroup
%     \def\noexpand\lw{\the\linewidth}^^A
%   }\x
%   \def\lwbox{^^A
%     \leavevmode
%     \hbox to \linewidth{^^A
%       \kern-\leftmargin\relax
%       \hss
%       \usebox0
%       \hss
%       \kern-\rightmargin\relax
%     }^^A
%   }^^A
%   \ifdim\wd0>\lw
%     \sbox0{\small\t}^^A
%     \ifdim\wd0>\linewidth
%       \ifdim\wd0>\lw
%         \sbox0{\footnotesize\t}^^A
%         \ifdim\wd0>\linewidth
%           \ifdim\wd0>\lw
%             \sbox0{\scriptsize\t}^^A
%             \ifdim\wd0>\linewidth
%               \ifdim\wd0>\lw
%                 \sbox0{\tiny\t}^^A
%                 \ifdim\wd0>\linewidth
%                   \lwbox
%                 \else
%                   \usebox0
%                 \fi
%               \else
%                 \lwbox
%               \fi
%             \else
%               \usebox0
%             \fi
%           \else
%             \lwbox
%           \fi
%         \else
%           \usebox0
%         \fi
%       \else
%         \lwbox
%       \fi
%     \else
%       \usebox0
%     \fi
%   \else
%     \lwbox
%   \fi
% \else
%   \usebox0
% \fi
% \end{quote}
% If you have a \xfile{docstrip.cfg} that configures and enables \docstrip's
% TDS installing feature, then some files can already be in the right
% place, see the documentation of \docstrip.
%
% \subsection{Refresh file name databases}
%
% If your \TeX~distribution
% (\teTeX, \mikTeX, \dots) relies on file name databases, you must refresh
% these. For example, \teTeX\ users run \verb|texhash| or
% \verb|mktexlsr|.
%
% \subsection{Some details for the interested}
%
% \paragraph{Attached source.}
%
% The PDF documentation on CTAN also includes the
% \xfile{.dtx} source file. It can be extracted by
% AcrobatReader 6 or higher. Another option is \textsf{pdftk},
% e.g. unpack the file into the current directory:
% \begin{quote}
%   \verb|pdftk thepdfnumber.pdf unpack_files output .|
% \end{quote}
%
% \paragraph{Unpacking with \LaTeX.}
% The \xfile{.dtx} chooses its action depending on the format:
% \begin{description}
% \item[\plainTeX:] Run \docstrip\ and extract the files.
% \item[\LaTeX:] Generate the documentation.
% \end{description}
% If you insist on using \LaTeX\ for \docstrip\ (really,
% \docstrip\ does not need \LaTeX), then inform the autodetect routine
% about your intention:
% \begin{quote}
%   \verb|latex \let\install=y% \iffalse meta-comment
%
% File: thepdfnumber.dtx
% Version: 2016/05/16 v1.1
% Info: Print PDF numbers with minimal digits
%
% Copyright (C) 2011 by
%    Heiko Oberdiek <heiko.oberdiek at googlemail.com>
%    2016
%    https://github.com/ho-tex/oberdiek/issues
%
% This work may be distributed and/or modified under the
% conditions of the LaTeX Project Public License, either
% version 1.3c of this license or (at your option) any later
% version. This version of this license is in
%    http://www.latex-project.org/lppl/lppl-1-3c.txt
% and the latest version of this license is in
%    http://www.latex-project.org/lppl.txt
% and version 1.3 or later is part of all distributions of
% LaTeX version 2005/12/01 or later.
%
% This work has the LPPL maintenance status "maintained".
%
% This Current Maintainer of this work is Heiko Oberdiek.
%
% The Base Interpreter refers to any `TeX-Format',
% because some files are installed in TDS:tex/generic//.
%
% This work consists of the main source file thepdfnumber.dtx
% and the derived files
%    thepdfnumber.sty, thepdfnumber.pdf, thepdfnumber.ins, thepdfnumber.drv,
%    thepdfnumber-test1.tex, thepdfnumber-test2.tex,
%    thepdfnumber-test3.tex.
%
% Distribution:
%    CTAN:macros/latex/contrib/oberdiek/thepdfnumber.dtx
%    CTAN:macros/latex/contrib/oberdiek/thepdfnumber.pdf
%
% Unpacking:
%    (a) If thepdfnumber.ins is present:
%           tex thepdfnumber.ins
%    (b) Without thepdfnumber.ins:
%           tex thepdfnumber.dtx
%    (c) If you insist on using LaTeX
%           latex \let\install=y\input{thepdfnumber.dtx}
%        (quote the arguments according to the demands of your shell)
%
% Documentation:
%    (a) If thepdfnumber.drv is present:
%           latex thepdfnumber.drv
%    (b) Without thepdfnumber.drv:
%           latex thepdfnumber.dtx; ...
%    The class ltxdoc loads the configuration file ltxdoc.cfg
%    if available. Here you can specify further options, e.g.
%    use A4 as paper format:
%       \PassOptionsToClass{a4paper}{article}
%
%    Programm calls to get the documentation (example):
%       pdflatex thepdfnumber.dtx
%       makeindex -s gind.ist thepdfnumber.idx
%       pdflatex thepdfnumber.dtx
%       makeindex -s gind.ist thepdfnumber.idx
%       pdflatex thepdfnumber.dtx
%
% Installation:
%    TDS:tex/generic/oberdiek/thepdfnumber.sty
%    TDS:doc/latex/oberdiek/thepdfnumber.pdf
%    TDS:doc/latex/oberdiek/test/thepdfnumber-test1.tex
%    TDS:doc/latex/oberdiek/test/thepdfnumber-test2.tex
%    TDS:doc/latex/oberdiek/test/thepdfnumber-test3.tex
%    TDS:source/latex/oberdiek/thepdfnumber.dtx
%
%<*ignore>
\begingroup
  \catcode123=1 %
  \catcode125=2 %
  \def\x{LaTeX2e}%
\expandafter\endgroup
\ifcase 0\ifx\install y1\fi\expandafter
         \ifx\csname processbatchFile\endcsname\relax\else1\fi
         \ifx\fmtname\x\else 1\fi\relax
\else\csname fi\endcsname
%</ignore>
%<*install>
\input docstrip.tex
\Msg{************************************************************************}
\Msg{* Installation}
\Msg{* Package: thepdfnumber 2016/05/16 v1.1 Print PDF numbers with minimal digits (HO)}
\Msg{************************************************************************}

\keepsilent
\askforoverwritefalse

\let\MetaPrefix\relax
\preamble

This is a generated file.

Project: thepdfnumber
Version: 2016/05/16 v1.1

Copyright (C) 2011 by
   Heiko Oberdiek <heiko.oberdiek at googlemail.com>

This work may be distributed and/or modified under the
conditions of the LaTeX Project Public License, either
version 1.3c of this license or (at your option) any later
version. This version of this license is in
   http://www.latex-project.org/lppl/lppl-1-3c.txt
and the latest version of this license is in
   http://www.latex-project.org/lppl.txt
and version 1.3 or later is part of all distributions of
LaTeX version 2005/12/01 or later.

This work has the LPPL maintenance status "maintained".

This Current Maintainer of this work is Heiko Oberdiek.

The Base Interpreter refers to any `TeX-Format',
because some files are installed in TDS:tex/generic//.

This work consists of the main source file thepdfnumber.dtx
and the derived files
   thepdfnumber.sty, thepdfnumber.pdf, thepdfnumber.ins, thepdfnumber.drv,
   thepdfnumber-test1.tex, thepdfnumber-test2.tex,
   thepdfnumber-test3.tex.

\endpreamble
\let\MetaPrefix\DoubleperCent

\generate{%
  \file{thepdfnumber.ins}{\from{thepdfnumber.dtx}{install}}%
  \file{thepdfnumber.drv}{\from{thepdfnumber.dtx}{driver}}%
  \usedir{tex/generic/oberdiek}%
  \file{thepdfnumber.sty}{\from{thepdfnumber.dtx}{package}}%
  \usedir{doc/latex/oberdiek/test}%
  \file{thepdfnumber-test1.tex}{\from{thepdfnumber.dtx}{test1}}%
  \file{thepdfnumber-test2.tex}{\from{thepdfnumber.dtx}{test2}}%
  \file{thepdfnumber-test3.tex}{\from{thepdfnumber.dtx}{test2,active}}%
}

\catcode32=13\relax% active space
\let =\space%
\Msg{************************************************************************}
\Msg{*}
\Msg{* To finish the installation you have to move the following}
\Msg{* file into a directory searched by TeX:}
\Msg{*}
\Msg{*     thepdfnumber.sty}
\Msg{*}
\Msg{* To produce the documentation run the file `thepdfnumber.drv'}
\Msg{* through LaTeX.}
\Msg{*}
\Msg{* Happy TeXing!}
\Msg{*}
\Msg{************************************************************************}

\endbatchfile
%</install>
%<*ignore>
\fi
%</ignore>
%<*driver>
\NeedsTeXFormat{LaTeX2e}
\ProvidesFile{thepdfnumber.drv}%
  [2016/05/16 v1.1 Print PDF numbers with minimal digits (HO)]%
\documentclass{ltxdoc}
\usepackage{holtxdoc}[2011/11/22]
\usepackage{csquotes}
\usepackage[
  backend=bibtex,
  bibencoding=ascii,
  alldates=iso8601,
]{biblatex}[2011/11/13]
\bibliography{oberdiek-source}
\begin{document}
  \DocInput{thepdfnumber.dtx}%
\end{document}
%</driver>
% \fi
%
%
% \CharacterTable
%  {Upper-case    \A\B\C\D\E\F\G\H\I\J\K\L\M\N\O\P\Q\R\S\T\U\V\W\X\Y\Z
%   Lower-case    \a\b\c\d\e\f\g\h\i\j\k\l\m\n\o\p\q\r\s\t\u\v\w\x\y\z
%   Digits        \0\1\2\3\4\5\6\7\8\9
%   Exclamation   \!     Double quote  \"     Hash (number) \#
%   Dollar        \$     Percent       \%     Ampersand     \&
%   Acute accent  \'     Left paren    \(     Right paren   \)
%   Asterisk      \*     Plus          \+     Comma         \,
%   Minus         \-     Point         \.     Solidus       \/
%   Colon         \:     Semicolon     \;     Less than     \<
%   Equals        \=     Greater than  \>     Question mark \?
%   Commercial at \@     Left bracket  \[     Backslash     \\
%   Right bracket \]     Circumflex    \^     Underscore    \_
%   Grave accent  \`     Left brace    \{     Vertical bar  \|
%   Right brace   \}     Tilde         \~}
%
% \GetFileInfo{thepdfnumber.drv}
%
% \title{The \xpackage{thepdfnumber} package}
% \date{2016/05/16 v1.1}
% \author{Heiko Oberdiek\thanks
% {Please report any issues at https://github.com/ho-tex/oberdiek/issues}\\
% \xemail{heiko.oberdiek at googlemail.com}}
%
% \maketitle
%
% \begin{abstract}
% The package converts real numbers to a minimal representation
% that is stripped from leading or trailing zeros,
% plus signs and decimal point if not necessary.
% \end{abstract}
%
% \tableofcontents
%
% \section{Documentation}
%
% \subsection{Introduction}
%
% Dealing with the PDF format, there is sometimes the need to write
% some low level PDF stuff. In case of numbers, the numbers can
% arise from user input (e.g. color or transparency specifications)
% or can be calculated. For example, \hologo{LaTeX}'s \cs{strip@pt}
% makes a good job to output a real number. It automatically suppresses
% the decimal part if the number is an integer. However it leaves
% a leading zero for numbers greater zero and smaller one.
% Thus the package provides macros that can be used with different
% formats, even with \hologo{iniTeX} and generates numbers that
% are valid numbers of the PDF format and whose length is minimal.
%
% \subsection{Usage}
%
% The package \xpackage{thepdfnumber} can be used with \hologo{LaTeX},
% \hologo{plainTeX} or even with \hologo{iniTeX}:
% \begin{quote}
%   |\RequirePackage{thepdfnumber}| \textit{\% \hologo{LaTeX}}\\
%   |\input thepdfnumber.sty| \textit{\% \hologo{plainTeX}/\hologo{iniTeX}}
% \end{quote}
% The package does not need and have package options.
%
% \subsection{User macros}
%
% All user macros are expandable in exact two expansion steps.
%
% \begin{declcs}{thepdfnumber} \M{number}
% \end{declcs}
% Macro \cs{thepdfnumber} takes a number
% as argument and expands to a minimal representation of that number.
% Some examples:
% \begin{quote}
% \begin{tabular}{r@{.: }>{\ttfamily}l@{ $\rightarrow$\ }>{\ttfamily}l@{}}
%   1& +123 & 123\\
%   2& --123 & 123\\
%   3& -01 & -1\\
%   4& 0045 & 45\\
%   5& 1.0 & 1\\
%   6& 1.20 & 1.2\\
%   7& 0.0 & 0\\
%   8& 0.78 & .78\\
%   9& +012.340 & 12.34\\
% \end{tabular}
% \end{quote}
% It reduces the length of the number representation:
% \begin{itemize}
% \item The signs are collapsed and only one minus sign
%   is output if the number is negative (see examples 1, 2, 3, 9).
% \item Leading zeros are removed (4, 8, 9) unless the number is zero (7).
% \item The decimal part is omitted, if the number is an integer (5, 7).
% \item Trailing zeros from the decimal part are stripped (5, 6, 7, 9).
% \end{itemize}
% The resulting number representation can be catched with one of the
% following Perl regular expressions:
% \begin{itemize}
% \item |^0$| (zero)
% \item |^-?[1-9][0-9]*$| (integer)
% \item |^-?[0-9]*\.[0-9]*[1-9]$| (real)
% \end{itemize}
% This is a valid numeric object of the PDF specification \cite[``7.3.3~Numeric Objects'']{pdfspec-iso32000-1}.
%
% \begin{declcs}{thepdfnumberNormZeroOne}
% \end{declcs}
% There are various places in the PDF specification where the
% number is in the domain 0.0 upto 1.0. Macro thepdfnumberNormZeroOne
% automatically adjusts the number to fit into that range. Negative
% numbers are mapped to |0| and numbers greater than one are
% replaced by |1|. Thus the result fits one of the following regular
% expressions:
% \begin{itemize}
% \item |^0$|
% \item |^\.[0-9]*[1-9]$|
% \item |^1$|
% \end{itemize}
% Examples:
% \begin{quote}
% \begin{tabular}{@{}>{\ttfamily}l@{ $\rightarrow$\ }>{\ttfamily}l@{}}
%   -456 & 0\\
%   -0.001 & 0\\
%   0.0 & 0\\
%   0.010 & .01\\
%   0.456 & .456\\
%   1.0 & 1\\
%   01.001 & 1\\
%   4 & 1\\
% \end{tabular}
% \end{quote}
%
% \subsection{Input number}
%
% The user macros expect a number as argument. The number can either
% be given explicitely or as macro that expands in one step
% to an explicite number, because the first token of the argument
% is expanded once.
%
% The explicite number consists of
% \begin{itemize}
% \item optional signs `\texttt{+}' and `\texttt{-}',
% \item digits `\texttt{0}' upto `\texttt{9}' and
% \item an optional dot `\texttt{.}'.
% \end{itemize}
% All tokens must have catcode 12 (other), the default catcodes
% for these characters in \hologo{LaTeX}, \hologo{plainTeX} or
% \hologo{iniTeX}. As Perl regular expression the number is expected
% in one of the following forms:
% \begin{itemize}
% \item |^[+-]*[0-9]+$|
% \item |^[+-]*[0-9]*\.[0-9]*$|
% \end{itemize}
% At least one digit or the dot must be present.
%
% \subsection{Error handling}
%
% The package is not intended for validating numbers or to
% decide if an argument is a number. Therefore it is an usage
% error to use the user macros with arguments that are not
% explicite numbers as specified in the previous sections.
% Nevertheless some error conditions are sometimes recognized.
% Errors are given in form of an undefined command sequence.
% It is the only way to notify \hologo{TeX} in expandable context.
% Expanding to some error text would invalidate the output.
% Currently the following errors are thrown:
% \begin{description}
% \item[\cs{thepdfnumber@ErrorEndMarker}:]
%   Internally the argument parsing uses an end marker that
%   is never called directly. If it is called with valid
%   user input, then this is a bug. Otherwise it means the user input
%   contains nasty stuff.
% \item[\cs{thepdfnumber@ErrorUnexpectedEnd}:] The macros expect at least one digit
%   or the dot, otherwise if the argument is empty or only contains
%   signs, then this error is called.
% \item[\cs{thepdfnumber@ErrorInvalidToken}:] It is called if the number
%   contains other tokens than signs, digits or the dot or the
%   token at the wrong place (e.g. a sign after a digit).
%   In case of \cs{thepdfnumberNormZeroOne} this error condition might
%   not always be detected, because the number parsing might stop
%   at an early point, when the result is already clear (e.g.
%   if the number is negative or will be greater than one).
% \item[Improper alphabetic constant:] This error might be
%   thrown by \hologo{TeX}, if the number contains
%   command tokens instead of characters.
% \end{description}
%
% \StopEventually{
% }
%
% \section{Implementation}
%
%    \begin{macrocode}
%<*package>
%    \end{macrocode}
%
% \subsection{Reload check and package identification}
%    Reload check, especially if the package is not used with \LaTeX.
%    \begin{macrocode}
\begingroup\catcode61\catcode48\catcode32=10\relax%
  \catcode13=5 % ^^M
  \endlinechar=13 %
  \catcode35=6 % #
  \catcode39=12 % '
  \catcode44=12 % ,
  \catcode45=12 % -
  \catcode46=12 % .
  \catcode58=12 % :
  \catcode64=11 % @
  \catcode123=1 % {
  \catcode125=2 % }
  \expandafter\let\expandafter\x\csname ver@thepdfnumber.sty\endcsname
  \ifx\x\relax % plain-TeX, first loading
  \else
    \def\empty{}%
    \ifx\x\empty % LaTeX, first loading,
      % variable is initialized, but \ProvidesPackage not yet seen
    \else
      \expandafter\ifx\csname PackageInfo\endcsname\relax
        \def\x#1#2{%
          \immediate\write-1{Package #1 Info: #2.}%
        }%
      \else
        \def\x#1#2{\PackageInfo{#1}{#2, stopped}}%
      \fi
      \x{thepdfnumber}{The package is already loaded}%
      \aftergroup\endinput
    \fi
  \fi
\endgroup%
%    \end{macrocode}
%    Package identification:
%    \begin{macrocode}
\begingroup\catcode61\catcode48\catcode32=10\relax%
  \catcode13=5 % ^^M
  \endlinechar=13 %
  \catcode35=6 % #
  \catcode39=12 % '
  \catcode40=12 % (
  \catcode41=12 % )
  \catcode44=12 % ,
  \catcode45=12 % -
  \catcode46=12 % .
  \catcode47=12 % /
  \catcode58=12 % :
  \catcode64=11 % @
  \catcode91=12 % [
  \catcode93=12 % ]
  \catcode123=1 % {
  \catcode125=2 % }
  \expandafter\ifx\csname ProvidesPackage\endcsname\relax
    \def\x#1#2#3[#4]{\endgroup
      \immediate\write-1{Package: #3 #4}%
      \xdef#1{#4}%
    }%
  \else
    \def\x#1#2[#3]{\endgroup
      #2[{#3}]%
      \ifx#1\@undefined
        \xdef#1{#3}%
      \fi
      \ifx#1\relax
        \xdef#1{#3}%
      \fi
    }%
  \fi
\expandafter\x\csname ver@thepdfnumber.sty\endcsname
\ProvidesPackage{thepdfnumber}%
  [2016/05/16 v1.1 Print PDF numbers with minimal digits (HO)]%
%    \end{macrocode}
%
% \subsection{Catcodes}
%
%    \begin{macrocode}
\begingroup\catcode61\catcode48\catcode32=10\relax%
  \catcode13=5 % ^^M
  \endlinechar=13 %
  \catcode123=1 % {
  \catcode125=2 % }
  \catcode64=11 % @
  \def\x{\endgroup
    \expandafter\edef\csname ThPdNu@AtEnd\endcsname{%
      \endlinechar=\the\endlinechar\relax
      \catcode13=\the\catcode13\relax
      \catcode32=\the\catcode32\relax
      \catcode35=\the\catcode35\relax
      \catcode61=\the\catcode61\relax
      \catcode64=\the\catcode64\relax
      \catcode123=\the\catcode123\relax
      \catcode125=\the\catcode125\relax
    }%
  }%
\x\catcode61\catcode48\catcode32=10\relax%
\catcode13=5 % ^^M
\endlinechar=13 %
\catcode35=6 % #
\catcode64=11 % @
\catcode123=1 % {
\catcode125=2 % }
\def\TMP@EnsureCode#1#2{%
  \edef\ThPdNu@AtEnd{%
    \ThPdNu@AtEnd
    \catcode#1=\the\catcode#1\relax
  }%
  \catcode#1=#2\relax
}
\TMP@EnsureCode{33}{12}% !
\TMP@EnsureCode{36}{3}% $
\TMP@EnsureCode{38}{4}% &
\TMP@EnsureCode{42}{12}% *
\TMP@EnsureCode{43}{12}% +
\TMP@EnsureCode{45}{12}% -
\TMP@EnsureCode{46}{12}% .
\TMP@EnsureCode{60}{12}% <
\TMP@EnsureCode{62}{12}% >
\TMP@EnsureCode{96}{12}% `
\edef\ThPdNu@AtEnd{\ThPdNu@AtEnd\noexpand\endinput}
%    \end{macrocode}
%
% \subsection{Helper macros}
%
%    \begin{macro}{\ThPdNu@FIN}
%    \begin{macrocode}
\def\ThPdNu@FIN{\thepdfnumber@ErrorEndMarker}
%    \end{macrocode}
%    \end{macro}
%
%    \begin{macro}{\ThPdNu@space}
%    \begin{macrocode}
\def\ThPdNu@space{ }
%    \end{macrocode}
%    \end{macro}
%
%    \begin{macro}{\ThPdNu@zero}
%    \begin{macrocode}
\chardef\ThPdNu@zero=0 %
%    \end{macrocode}
%    \end{macro}
%
%    \begin{macro}{\ThPdNu@one}
%    \begin{macrocode}
\chardef\ThPdNu@one=1 %
%    \end{macrocode}
%    \end{macro}
%
%    \begin{macro}{\ThPdNu@firstoftwo}
%    \begin{macrocode}
\long\def\ThPdNu@firstoftwo#1#2{#1}
%    \end{macrocode}
%    \end{macro}
%
%    \begin{macro}{\ThPdNu@secondoftwo}
%    \begin{macrocode}
\long\def\ThPdNu@secondoftwo#1#2{#2}
%    \end{macrocode}
%    \end{macro}
%
% \subsection{Detect \hologo{eTeX}}
%
%    \begin{macrocode}
\begingroup\expandafter\expandafter\expandafter\endgroup
\expandafter\ifx\csname detokenize\endcsname\relax
  \catcode`\&=14 %
  \catcode`\$=9 %
\else
  \catcode`\&=9 %
  \catcode`\$=14 %
\fi
%    \end{macrocode}
%
% \subsection{User macro \cs{thepdfnumber}}
%
%    \begin{macro}{\thepdfnumber}
%    \begin{macrocode}
\def\thepdfnumber#1{%
  \romannumeral
& \iftrue\expandafter\ThPdNu@State@Plus\expandafter\fi
& \detokenize\expandafter{#1}%
& \ThPdNu@FIN
$ \ifx\ThPdNu@FIN#1\ThPdNu@FIN
$   \expandafter\ThPdNu@firstoftwo
$ \else
$   \expandafter\ThPdNu@secondoftwo
$ \fi
$ {%
$   \ThPdNu@zero
$   0\thepdfnumber@ErrorUnexpectedEnd
$ }{%
$   \iftrue\expandafter\ThPdNu@State@Plus\expandafter\fi#1\ThPdNu@FIN
$ }%
}
%    \end{macrocode}
%    \end{macro}
%
% \subsubsection{State definitions for sign}
%
%    \begin{macro}{\ThPdNu@State@Plus}
%    \begin{macrocode}
\def\ThPdNu@State@Plus#1\fi#2{%
  \fi
  \ifcase\ifx\ThPdNu@FIN#2%
           0%
&        \else\ifx-#2%
$        \else\ifnum`#2=45 % -
           1%
         \else\ifx0#2%
           2%
         \else\ifnum`#2>48 %
           \ifnum`#2<58 %
             3%
           \else
             9%
           \fi
&        \else\ifx.#2%
$        \else\ifnum`#2=46 % .
           4%
&        \else\ifx+#2%
$        \else\ifnum`#2=43 % +
           5%
         \else
           9%
         \fi\fi\fi\fi\fi\fi\ThPdNu@space
    \expandafter\ThPdNu@zero
    \expandafter0%
    \expandafter\thepdfnumber@ErrorUnexpectedEnd
  \or
    \ThPdNu@State@Minus
  \or
    \ThPdNu@State@SkipZeros!%
  \or
    \ThPdNu@State@Int!#2!%
  \or
    \ThPdNu@State@Dot!\ThPdNu@zero*\ThPdNu@zero!!%
  \or
    \ThPdNu@State@Plus
  \else
    \ThPdNu@ReturnError{0}%
  \fi
}
%    \end{macrocode}
%    \end{macro}
%
%    \begin{macro}{\ThPdNu@State@Minus}
%    \begin{macrocode}
\def\ThPdNu@State@Minus#1\fi#2{%
  \fi
  \ifcase\ifx\ThPdNu@FIN#2%
           0%
         \else\ifx0#2%
           1%
         \else\ifnum`#2>48 %
           \ifnum`#2<58 %
             2%
           \else
             9%
           \fi
&        \else\ifx.#2%
$        \else\ifnum`#2=46 % .
           3%
&        \else\ifx-#2%
$        \else\ifnum`#2=45 % -
           4%
&        \else\ifx+#2%
$        \else\ifnum`#2=43 % +
           5%
         \else
           9%
         \fi\fi\fi\fi\fi\fi\ThPdNu@space
    \expandafter\ThPdNu@zero
    \expandafter0%
    \expandafter\thepdfnumber@ErrorUnexpectedEnd
  \or
    \ThPdNu@State@SkipZeros-!%
  \or
    \ThPdNu@State@Int-!#2!%
  \or
    \ThPdNu@State@Dot-!\ThPdNu@zero*\ThPdNu@zero!!%
  \or
    \ThPdNu@State@Plus
  \or
    \ThPdNu@State@Minus
  \else
    \ThPdNu@ReturnError{0}%
  \fi
}
%    \end{macrocode}
%    \end{macro}
%
%    \begin{macro}{\ThPdNu@ReturnError}
%    \begin{macrocode}
\def\ThPdNu@ReturnError#1#2\fi#3\ThPdNu@FIN{%
  \fi
  \ThPdNu@zero
  #1%
  \thepdfnumber@ErrorInvalidToken
}
%    \end{macrocode}
%    \end{macro}
%
% \subsubsection{State definitions for integer part}
%
%    \begin{macro}{\ThPdNu@State@SkipZeros}
%    \begin{macrocode}
\def\ThPdNu@State@SkipZeros#1!#2\fi#3{%
  \fi
  \ifcase\ifx\ThPdNu@FIN#3%
           0%
         \else\ifx0#3%
           1%
         \else\ifnum`#3>48 %
           \ifnum`#3<58 %
             2%
           \else
             9%
           \fi
&        \else\ifx.#3%
$        \else\ifnum`#3=46 % .
           3%
         \else
           9%
         \fi\fi\fi\fi\ThPdNu@space
    \expandafter\ThPdNu@zero
    \expandafter0%
  \or
   \ThPdNu@State@SkipZeros#1!%
  \or
    \ThPdNu@State@Int#1!#3!%
  \or
    \ThPdNu@State@Dot#1!\ThPdNu@zero*\ThPdNu@zero!!%
  \else
    \ThPdNu@ReturnError{0}%
  \fi
}
%    \end{macrocode}
%    \end{macro}
%
%    \begin{macro}{\ThPdNu@State@Int}
%    \begin{macrocode}
\def\ThPdNu@State@Int#1!#2!#3\fi#4{%
  \fi
  \ifcase\ifx\ThPdNu@FIN#4%
           0%
         \else\ifnum`#4>47 %
           \ifnum`#4<58 %
             1%
           \else
             9%
           \fi
&        \else\ifx.#4%
$        \else\ifnum`#4=46 % .
           2%
         \else
           9%
         \fi\fi\fi\ThPdNu@space
    \ThPdNu@ReturnInt{#1#2}%
  \or
    \ThPdNu@State@Int#1!#2#4!%
  \or
    \ThPdNu@State@Dot#1!\ThPdNu@one#2*\ThPdNu@zero!!%
  \else
    \ThPdNu@ReturnError{#1#2}%
  \fi
}
%    \end{macrocode}
%    \end{macro}
%
%    \begin{macro}{\ThPdNu@ReturnInt}
%    \begin{macrocode}
\def\ThPdNu@ReturnInt#1#2\fi{%
  \fi
  \ThPdNu@zero
  #1%
}
%    \end{macrocode}
%    \end{macro}
%
% \subsubsection{State definitions for decimal digits}
%
%    \begin{macro}{\ThPdNu@State@Dot}
%    \begin{macrocode}
\def\ThPdNu@State@Dot#1*#2#3!#4!#5\fi#6{%
  \fi
  \ifcase\ifx\ThPdNu@FIN#6%
           0%
         \else\ifnum`#6>48 %
           \ifnum`#6<58 %
             1%
           \else
             9%
           \fi
         \else\ifx0#6%
           2%
         \else
           9%
         \fi\fi\fi\ThPdNu@space
    \ThPdNu@ReturnNumber#1*#2#3!%
  \or
    \ThPdNu@State@Dot#1*\ThPdNu@one#3#4#6!!%
  \or
    \ThPdNu@State@DotZero#1*#2#3!#4#6!%
  \else
    \ThPdNu@ReturnNumberInvalid#1*#2#3!%
  \fi
}
%    \end{macrocode}
%    \end{macro}
%
%    \begin{macro}{\ThPdNu@State@DotZero}
%    \begin{macrocode}
\def\ThPdNu@State@DotZero#1*#2#3!#4!#5\fi#6{%
  \fi
  \ifcase\ifx\ThPdNu@FIN#6%
           0%
         \else\ifnum`#6>48 %
           \ifnum`#6<58 %
             1%
           \else
             9%
           \fi
         \else\ifx0#6%
           2%
         \else
           9%
         \fi\fi\fi\ThPdNu@space
    \ThPdNu@ReturnNumber#1*#2#3!%
  \or
    \ThPdNu@State@Dot#1*\ThPdNu@one#3#4#6!!%
  \or
    \ThPdNu@State@DotZero#1*#2#3!#4#6!%
  \else
    \ThPdNu@ReturnNumber#1*#2#3!%
  \fi
}
%    \end{macrocode}
%    \end{macro}
%
%    \begin{macro}{\ThPdNu@ReturnNumber}
%    \begin{macrocode}
\def\ThPdNu@ReturnNumber#1!#2#3*#4#5!#6\fi{%
  \fi
  \ifcase#2%
    \expandafter\ThPdNu@firstoftwo
  \else
    \expandafter\ThPdNu@secondoftwo
  \fi
  {%
    \ifcase#4%
      \expandafter\ThPdNu@firstoftwo
    \else
      \expandafter\ThPdNu@secondoftwo
    \fi
    {\ThPdNu@zero 0}%
    {\ThPdNu@zero #1.#5}%
  }{%
    \ifcase#4%
      \expandafter\ThPdNu@firstoftwo
    \else
      \expandafter\ThPdNu@secondoftwo
    \fi
    {\ThPdNu@zero #1#3}%
    {\ThPdNu@zero #1#3.#5}%
  }%
}
%    \end{macrocode}
%    \end{macro}
%
%    \begin{macro}{\ThPdNu@ReturnNumberInvalid}
%    \begin{macrocode}
\def\ThPdNu@ReturnNumberInvalid#1*#2!#3\fi#4\ThPdNu@FIN{%
  \fi
  \iftrue\ThPdNu@ReturnNumber#1*#2!\fi
  \thepdfnumber@ErrorInvalidToken
}
%    \end{macrocode}
%    \end{macro}
%
% \subsection{Norm macro}
%
%    \begin{macro}{\thepdfnumberNormZeroOne}
%    \begin{macrocode}
\def\thepdfnumberNormZeroOne#1{%
  \romannumeral
& \iftrue\expandafter\ThPbNu@StateN@Plus\expandafter\fi
& \detokenize\expandafter{#1}%
& \ThPdNu@FIN
$ \ifx\ThPdNu@FIN#1\ThPdNu@FIN
$   \expandafter\ThPdNu@firstoftwo
$ \else
$   \expandafter\ThPdNu@secondoftwo
$ \fi
$ {%
$   \ThPdNu@zero
$   0\thepdfnumber@ErrorUnexpectedEnd
$ }{%
$   \iftrue\expandafter\ThPbNu@StateN@Plus\expandafter\fi#1\ThPdNu@FIN
$ }%
}
%    \end{macrocode}
%    \end{macro}
%
% \subsubsection{State definitions for sign}
%
%    \begin{macro}{\ThPbNu@StateN@Plus}
%    \begin{macrocode}
\def\ThPbNu@StateN@Plus#1\fi#2{%
  \fi
  \ifcase\ifx\ThPdNu@FIN#2%
           0%
&        \else\ifx-#2%
$        \else\ifnum`#2=45 % -
           1%
         \else\ifx0#2%
           2%
         \else\ifnum`#2>48 %
           \ifnum`#2<58 %
             3%
           \else
             9%
           \fi
&        \else\ifx.#2%
$        \else\ifnum`#2=46 % .
           4%
&        \else\ifx+#2%
$        \else\ifnum`#2=43 % +
           5%
         \else
           9%
         \fi\fi\fi\fi\fi\fi\ThPdNu@space
    \expandafter\ThPdNu@zero
    \expandafter0%
    \expandafter\thepdfnumber@ErrorUnexpectedEnd
  \or
    \ThPbNu@StateN@Minus
  \or
    \ThPbNu@StateN@SkipZeros
  \or
    \ThPdNu@ReturnAndSkip{1}%
  \or
    \ThPbNu@StateN@Dot\ThPdNu@zero!!%
  \or
    \ThPbNu@StateN@Plus
  \else
    \ThPdNu@ReturnError{0}%
  \fi
}
%    \end{macrocode}
%    \end{macro}
%
%    \begin{macro}{\ThPbNu@StateN@Minus}
%    \begin{macrocode}
\def\ThPbNu@StateN@Minus#1\fi#2{%
  \fi
  \ifcase\ifx\ThPdNu@FIN#2%
           0%
         \else\ifnum`#2>47 %
           \ifnum`#2<58 %
             1%
           \else
             9%
           \fi
&        \else\ifx.#2%
$        \else\ifnum`#2=46 % .
           1%
&        \else\ifx-#2%
$        \else\ifnum`#2=45 % -
           2%
&        \else\ifx+#2%
$        \else\ifnum`#2=43 % +
           3%
         \else
           9%
         \fi\fi\fi\fi\fi\ThPdNu@space
    \expandafter\ThPdNu@zero
    \expandafter0%
    \expandafter\thepdfnumber@ErrorUnexpectedEnd
  \or
    \ThPdNu@ReturnAndSkip{0}%
  \or
    \ThPbNu@StateN@Plus
  \or
    \ThPbNu@StateN@Minus
  \else
    \ThPdNu@ReturnError{0}%
  \fi
}
%    \end{macrocode}
%    \end{macro}
%
%    \begin{macro}{\ThPdNu@ReturnAndSkip}
%    \begin{macrocode}
\def\ThPdNu@ReturnAndSkip#1#2\fi#3\ThPdNu@FIN{%
  \fi
  \ThPdNu@zero
  #1%
}
%    \end{macrocode}
%    \end{macro}
%
% \subsubsection{State definitions for integer part}
%
%    \begin{macro}{\ThPbNu@StateN@SkipZeros}
%    \begin{macrocode}
\def\ThPbNu@StateN@SkipZeros#1\fi#2{%
  \fi
  \ifcase\ifx\ThPdNu@FIN#2%
           0%
         \else\ifx0#2%
           1%
         \else\ifnum`#2>48 %
           \ifnum`#2<58 %
             2%
           \else
             9%
           \fi
&        \else\ifx.#2%
$        \else\ifnum`#2=46 % .
           3%
         \else
           9%
         \fi\fi\fi\fi\ThPdNu@space
    \expandafter\ThPdNu@zero
    \expandafter0%
  \or
   \ThPbNu@StateN@SkipZeros%
  \or
    \ThPdNu@ReturnAndSkip{1}%
  \or
    \ThPbNu@StateN@Dot\ThPdNu@zero!!%
  \else
    \ThPdNu@ReturnError{0}%
  \fi
}
%    \end{macrocode}
%    \end{macro}
%
% \subsubsection{State definitions for decimal digits}
%
%    \begin{macro}{\ThPbNu@StateN@Dot}
%    \begin{macrocode}
\def\ThPbNu@StateN@Dot#1#2!#3!#4\fi#5{%
  \fi
  \ifcase\ifx\ThPdNu@FIN#5%
           0%
         \else\ifnum`#5>48 %
           \ifnum`#5<58 %
             1%
           \else
             9%
           \fi
         \else\ifx0#5%
           2%
         \else
           9%
         \fi\fi\fi\ThPdNu@space
    \ThPdNu@ReturnFracNumber#1#2!%
  \or
    \ThPbNu@StateN@Dot\ThPdNu@one#2#3#5!!%
  \or
    \ThPbNu@StateN@DotZero#1#2!#3#5!%
  \else
    \ThPdNu@ReturnFracNumberInvalid#1#2!%
  \fi
}
%    \end{macrocode}
%    \end{macro}
%
%    \begin{macro}{\ThPbNu@StateN@DotZero}
%    \begin{macrocode}
\def\ThPbNu@StateN@DotZero#1#2!#3!#4\fi#5{%
  \fi
  \ifcase\ifx\ThPdNu@FIN#5%
           0%
         \else\ifnum`#5>48 %
           \ifnum`#5<58 %
             1%
           \else
             9%
           \fi
         \else\ifx0#5%
           2%
         \else
           9%
         \fi\fi\fi\ThPdNu@space
    \ThPdNu@ReturnFracNumber#1#2!%
  \or
    \ThPbNu@StateN@Dot\ThPdNu@one#2#3#5!!%
  \or
    \ThPbNu@StateN@DotZero#1#2!#3#5!%
  \else
    \ThPdNu@ReturnFracNumberInvalid#1#2!%
  \fi
}
%    \end{macrocode}
%    \end{macro}
%
%    \begin{macro}{\ThPdNu@ReturnFracNumber}
%    \begin{macrocode}
\def\ThPdNu@ReturnFracNumber#1#2!#3\fi{%
  \fi
  \ifcase#1%
    \expandafter\ThPdNu@firstoftwo
  \else
    \expandafter\ThPdNu@secondoftwo
  \fi
  {\ThPdNu@zero 0}%
  {\ThPdNu@zero .#2}%
}
%    \end{macrocode}
%    \end{macro}
%
%    \begin{macro}{\ThPdNu@ReturnFracNumberInvalid}
%    \begin{macrocode}
\def\ThPdNu@ReturnFracNumberInvalid#1!#2\fi#3\ThPdNu@FIN{%
  \fi
  \iftrue\ThPdNu@ReturnFracNumber#1!\fi
  \thepdfnumber@ErrorInvalidToken
}
%    \end{macrocode}
%    \end{macro}
%
%    \begin{macrocode}
\ThPdNu@AtEnd%
%</package>
%    \end{macrocode}
%
% \section{Test}
%
% \subsection{Catcode checks for loading}
%
%    \begin{macrocode}
%<*test1>
%    \end{macrocode}
%    \begin{macrocode}
\catcode`\{=1 %
\catcode`\}=2 %
\catcode`\#=6 %
\catcode`\@=11 %
\expandafter\ifx\csname count@\endcsname\relax
  \countdef\count@=255 %
\fi
\expandafter\ifx\csname @gobble\endcsname\relax
  \long\def\@gobble#1{}%
\fi
\expandafter\ifx\csname @firstofone\endcsname\relax
  \long\def\@firstofone#1{#1}%
\fi
\expandafter\ifx\csname loop\endcsname\relax
  \expandafter\@firstofone
\else
  \expandafter\@gobble
\fi
{%
  \def\loop#1\repeat{%
    \def\body{#1}%
    \iterate
  }%
  \def\iterate{%
    \body
      \let\next\iterate
    \else
      \let\next\relax
    \fi
    \next
  }%
  \let\repeat=\fi
}%
\def\RestoreCatcodes{}
\count@=0 %
\loop
  \edef\RestoreCatcodes{%
    \RestoreCatcodes
    \catcode\the\count@=\the\catcode\count@\relax
  }%
\ifnum\count@<255 %
  \advance\count@ 1 %
\repeat

\def\RangeCatcodeInvalid#1#2{%
  \count@=#1\relax
  \loop
    \catcode\count@=15 %
  \ifnum\count@<#2\relax
    \advance\count@ 1 %
  \repeat
}
\def\RangeCatcodeCheck#1#2#3{%
  \count@=#1\relax
  \loop
    \ifnum#3=\catcode\count@
    \else
      \errmessage{%
        Character \the\count@\space
        with wrong catcode \the\catcode\count@\space
        instead of \number#3%
      }%
    \fi
  \ifnum\count@<#2\relax
    \advance\count@ 1 %
  \repeat
}
\def\space{ }
\expandafter\ifx\csname LoadCommand\endcsname\relax
  \def\LoadCommand{\input thepdfnumber.sty\relax}%
\fi
\def\Test{%
  \RangeCatcodeInvalid{0}{47}%
  \RangeCatcodeInvalid{58}{64}%
  \RangeCatcodeInvalid{91}{96}%
  \RangeCatcodeInvalid{123}{255}%
  \catcode`\@=12 %
  \catcode`\\=0 %
  \catcode`\%=14 %
  \LoadCommand
  \RangeCatcodeCheck{0}{36}{15}%
  \RangeCatcodeCheck{37}{37}{14}%
  \RangeCatcodeCheck{38}{47}{15}%
  \RangeCatcodeCheck{48}{57}{12}%
  \RangeCatcodeCheck{58}{63}{15}%
  \RangeCatcodeCheck{64}{64}{12}%
  \RangeCatcodeCheck{65}{90}{11}%
  \RangeCatcodeCheck{91}{91}{15}%
  \RangeCatcodeCheck{92}{92}{0}%
  \RangeCatcodeCheck{93}{96}{15}%
  \RangeCatcodeCheck{97}{122}{11}%
  \RangeCatcodeCheck{123}{255}{15}%
  \RestoreCatcodes
}
\Test
\csname @@end\endcsname
\end
%    \end{macrocode}
%    \begin{macrocode}
%</test1>
%    \end{macrocode}
%
% \subsection{Macro tests}
%
% \subsubsection{Preamble}
%
%    \begin{macrocode}
%<*test2>
\catcode`\{=1
\catcode`\}=2
\catcode`\#=6
\catcode`\@=11
\errorcontextlines=10000 %
\def\msg#{\immediate\write16}
\def\space{ }
\begingroup\expandafter\expandafter\expandafter\endgroup
\expandafter\ifx\csname RequirePackage\endcsname\relax
  \input thepdfnumber.sty\relax
\else
  \RequirePackage{thepdfnumber}[2016/05/16]%
\fi
%    \end{macrocode}
%
%    \begin{macro}{\Test}
%    \begin{macrocode}
\def\Test#1#2{%
  \def\TestExpected{#2}%
%<*active>
  \edef\TestExpected{\TestExpected}%
%</active>
  \expandafter\expandafter\expandafter\def
  \expandafter\expandafter\expandafter\TestResult
  \expandafter\expandafter\expandafter{%
    \TestCommand{#1}%
  }%
  \begingroup
    \toks0\expandafter{\TestResult}%
    \toks2\expandafter{\TestExpected}%
    \ifx\TestResult\TestExpected
      \msg{* OK: #1 => \the\toks0}%
    \else
      \errmessage{FAILED: #1 => \the\toks0\space(\the\toks2)}%
    \fi
  \endgroup
}
%    \end{macrocode}
%    \end{macro}
%
%    \begin{macrocode}
\let\thepdfnumber@ErrorUnexpectedEnd\relax
\let\thepdfnumber@ErrorInvalidToken\relax
%    \end{macrocode}
%
% \subsubsection{Test with active catcodes}
%
%    \begin{macrocode}
%<*active>
\catcode`\~=13 %
\def\ThPdNu@Temp#1{%
  \begingroup
    \lccode`\~=`#1 %
  \lowercase{\endgroup
    \edef~%
  }{\string#1}%
  \catcode`#1=13 %
}
\ThPdNu@Temp{.}
\ThPdNu@Temp{-}
\ThPdNu@Temp{+}
%</active>
%    \end{macrocode}
%
% \subsubsection{Test cases for \cs{thepdfnumber}}
%
%    \begin{macro}{\TestCommand}
%    \begin{macrocode}
\let\TestCommand\thepdfnumber
%    \end{macrocode}
%    \end{macro}
%
%    \begin{macrocode}
\Test{0}{0}
\Test{1}{1}
\Test{23}{23}
\Test{0045}{45}
\Test{4500}{4500}
\Test{-0}{0}
\Test{-1}{-1}
\Test{-00002}{-2}
\Test{-010203}{-10203}
\Test{+0}{0}
\Test{+123}{123}

\Test{+-4}{-4}
\Test{+-+-4}{4}
\Test{++--++--++4}{4}
\Test{+++---+++---+++---4}{-4}

\Test{.}{0}
\Test{0.}{0}
\Test{.0}{0}
\Test{0.}{0}
\Test{-.}{0}
\Test{-0.}{0}
\Test{-.0}{0}
\Test{-0.}{0}
\Test{010.020}{10.02}
\Test{123.456}{123.456}
\Test{12.}{12}
\Test{.123}{.123}
\Test{.0001}{.0001}
\Test{.00100}{.001}
\Test{.12003400560078009}{.12003400560078009}
\Test{-.12003400560078009}{-.12003400560078009}
\Test{04.0500000}{4.05}

\Test{}{0\thepdfnumber@ErrorUnexpectedEnd}
\Test{+}{0\thepdfnumber@ErrorUnexpectedEnd}
\Test{-}{0\thepdfnumber@ErrorUnexpectedEnd}
\Test{a}{0\thepdfnumber@ErrorInvalidToken}
\Test{0x1}{0\thepdfnumber@ErrorInvalidToken}
\Test{4x56}{4\thepdfnumber@ErrorInvalidToken}
\Test{012,34}{12\thepdfnumber@ErrorInvalidToken}
\Test{0.12x4}{.12\thepdfnumber@ErrorInvalidToken}
%    \end{macrocode}
%
% \subsubsection{Test cases for \cs{thepdfnumberNormZeroOne}}
%
%    \begin{macro}{\TestCommand}
%    \begin{macrocode}
\let\TestCommand\thepdfnumberNormZeroOne
%    \end{macrocode}
%    \end{macro}
%
%    \begin{macrocode}
\Test{0}{0}
\Test{1}{1}
\Test{23}{1}
\Test{12}{1}
\Test{10}{1}
\Test{0045}{1}
\Test{001}{1}
\Test{-0}{0}
\Test{-1}{0}
\Test{-01}{0}
\Test{-4}{0}

\Test{+-0}{0}
\Test{+-+-1}{1}
\Test{++--++--++1}{1}
\Test{+++---+++---+++---1}{0}

\Test{.}{0}
\Test{0.}{0}
\Test{.0}{0}
\Test{0.}{0}
\Test{-.}{0}
\Test{-0.}{0}
\Test{-.0}{0}
\Test{-0.}{0}
\Test{010.020}{1}
\Test{123.456}{1}
\Test{12.}{1}
\Test{.123}{.123}
\Test{.0001}{.0001}
\Test{.00100}{.001}
\Test{.12003400560078009}{.12003400560078009}
\Test{-.12003400560078009}{0}
\Test{04.0500000}{1}
\Test{0.1200340056}{.1200340056}
\Test{1.05}{1}

\Test{}{0\thepdfnumber@ErrorUnexpectedEnd}
\Test{+}{0\thepdfnumber@ErrorUnexpectedEnd}
\Test{-}{0\thepdfnumber@ErrorUnexpectedEnd}
\Test{a}{0\thepdfnumber@ErrorInvalidToken}
\Test{0x1}{0\thepdfnumber@ErrorInvalidToken}
\Test{4x56}{1}
\Test{012,34}{1}
\Test{.012x4}{.012\thepdfnumber@ErrorInvalidToken}
%    \end{macrocode}
%
%    \begin{macrocode}
\csname @@end\endcsname\end
%</test2>
%    \end{macrocode}
%
% \section{Installation}
%
% \subsection{Download}
%
% \paragraph{Package.} This package is available on
% CTAN\footnote{\url{http://ctan.org/pkg/thepdfnumber}}:
% \begin{description}
% \item[\CTAN{macros/latex/contrib/oberdiek/thepdfnumber.dtx}] The source file.
% \item[\CTAN{macros/latex/contrib/oberdiek/thepdfnumber.pdf}] Documentation.
% \end{description}
%
%
% \paragraph{Bundle.} All the packages of the bundle `oberdiek'
% are also available in a TDS compliant ZIP archive. There
% the packages are already unpacked and the documentation files
% are generated. The files and directories obey the TDS standard.
% \begin{description}
% \item[\CTAN{install/macros/latex/contrib/oberdiek.tds.zip}]
% \end{description}
% \emph{TDS} refers to the standard ``A Directory Structure
% for \TeX\ Files'' (\CTAN{tds/tds.pdf}). Directories
% with \xfile{texmf} in their name are usually organized this way.
%
% \subsection{Bundle installation}
%
% \paragraph{Unpacking.} Unpack the \xfile{oberdiek.tds.zip} in the
% TDS tree (also known as \xfile{texmf} tree) of your choice.
% Example (linux):
% \begin{quote}
%   |unzip oberdiek.tds.zip -d ~/texmf|
% \end{quote}
%
% \paragraph{Script installation.}
% Check the directory \xfile{TDS:scripts/oberdiek/} for
% scripts that need further installation steps.
% Package \xpackage{attachfile2} comes with the Perl script
% \xfile{pdfatfi.pl} that should be installed in such a way
% that it can be called as \texttt{pdfatfi}.
% Example (linux):
% \begin{quote}
%   |chmod +x scripts/oberdiek/pdfatfi.pl|\\
%   |cp scripts/oberdiek/pdfatfi.pl /usr/local/bin/|
% \end{quote}
%
% \subsection{Package installation}
%
% \paragraph{Unpacking.} The \xfile{.dtx} file is a self-extracting
% \docstrip\ archive. The files are extracted by running the
% \xfile{.dtx} through \plainTeX:
% \begin{quote}
%   \verb|tex thepdfnumber.dtx|
% \end{quote}
%
% \paragraph{TDS.} Now the different files must be moved into
% the different directories in your installation TDS tree
% (also known as \xfile{texmf} tree):
% \begin{quote}
% \def\t{^^A
% \begin{tabular}{@{}>{\ttfamily}l@{ $\rightarrow$ }>{\ttfamily}l@{}}
%   thepdfnumber.sty & tex/generic/oberdiek/thepdfnumber.sty\\
%   thepdfnumber.pdf & doc/latex/oberdiek/thepdfnumber.pdf\\
%   test/thepdfnumber-test1.tex & doc/latex/oberdiek/test/thepdfnumber-test1.tex\\
%   test/thepdfnumber-test2.tex & doc/latex/oberdiek/test/thepdfnumber-test2.tex\\
%   test/thepdfnumber-test3.tex & doc/latex/oberdiek/test/thepdfnumber-test3.tex\\
%   thepdfnumber.dtx & source/latex/oberdiek/thepdfnumber.dtx\\
% \end{tabular}^^A
% }^^A
% \sbox0{\t}^^A
% \ifdim\wd0>\linewidth
%   \begingroup
%     \advance\linewidth by\leftmargin
%     \advance\linewidth by\rightmargin
%   \edef\x{\endgroup
%     \def\noexpand\lw{\the\linewidth}^^A
%   }\x
%   \def\lwbox{^^A
%     \leavevmode
%     \hbox to \linewidth{^^A
%       \kern-\leftmargin\relax
%       \hss
%       \usebox0
%       \hss
%       \kern-\rightmargin\relax
%     }^^A
%   }^^A
%   \ifdim\wd0>\lw
%     \sbox0{\small\t}^^A
%     \ifdim\wd0>\linewidth
%       \ifdim\wd0>\lw
%         \sbox0{\footnotesize\t}^^A
%         \ifdim\wd0>\linewidth
%           \ifdim\wd0>\lw
%             \sbox0{\scriptsize\t}^^A
%             \ifdim\wd0>\linewidth
%               \ifdim\wd0>\lw
%                 \sbox0{\tiny\t}^^A
%                 \ifdim\wd0>\linewidth
%                   \lwbox
%                 \else
%                   \usebox0
%                 \fi
%               \else
%                 \lwbox
%               \fi
%             \else
%               \usebox0
%             \fi
%           \else
%             \lwbox
%           \fi
%         \else
%           \usebox0
%         \fi
%       \else
%         \lwbox
%       \fi
%     \else
%       \usebox0
%     \fi
%   \else
%     \lwbox
%   \fi
% \else
%   \usebox0
% \fi
% \end{quote}
% If you have a \xfile{docstrip.cfg} that configures and enables \docstrip's
% TDS installing feature, then some files can already be in the right
% place, see the documentation of \docstrip.
%
% \subsection{Refresh file name databases}
%
% If your \TeX~distribution
% (\teTeX, \mikTeX, \dots) relies on file name databases, you must refresh
% these. For example, \teTeX\ users run \verb|texhash| or
% \verb|mktexlsr|.
%
% \subsection{Some details for the interested}
%
% \paragraph{Attached source.}
%
% The PDF documentation on CTAN also includes the
% \xfile{.dtx} source file. It can be extracted by
% AcrobatReader 6 or higher. Another option is \textsf{pdftk},
% e.g. unpack the file into the current directory:
% \begin{quote}
%   \verb|pdftk thepdfnumber.pdf unpack_files output .|
% \end{quote}
%
% \paragraph{Unpacking with \LaTeX.}
% The \xfile{.dtx} chooses its action depending on the format:
% \begin{description}
% \item[\plainTeX:] Run \docstrip\ and extract the files.
% \item[\LaTeX:] Generate the documentation.
% \end{description}
% If you insist on using \LaTeX\ for \docstrip\ (really,
% \docstrip\ does not need \LaTeX), then inform the autodetect routine
% about your intention:
% \begin{quote}
%   \verb|latex \let\install=y\input{thepdfnumber.dtx}|
% \end{quote}
% Do not forget to quote the argument according to the demands
% of your shell.
%
% \paragraph{Generating the documentation.}
% You can use both the \xfile{.dtx} or the \xfile{.drv} to generate
% the documentation. The process can be configured by the
% configuration file \xfile{ltxdoc.cfg}. For instance, put this
% line into this file, if you want to have A4 as paper format:
% \begin{quote}
%   \verb|\PassOptionsToClass{a4paper}{article}|
% \end{quote}
% An example follows how to generate the
% documentation with pdf\LaTeX:
% \begin{quote}
%\begin{verbatim}
%pdflatex thepdfnumber.dtx
%makeindex -s gind.ist thepdfnumber.idx
%pdflatex thepdfnumber.dtx
%makeindex -s gind.ist thepdfnumber.idx
%pdflatex thepdfnumber.dtx
%\end{verbatim}
% \end{quote}
%
% \printbibliography[
%   heading=bibnumbered,
% ]
%
% \begin{History}
%   \begin{Version}{2011/11/24 v1.0}
%   \item
%     First version.
%   \end{Version}
%   \begin{Version}{2016/05/16 v1.1}
%   \item
%     Documentation updates.
%   \end{Version}
% \end{History}
%
% \PrintIndex
%
% \Finale
\endinput
|
% \end{quote}
% Do not forget to quote the argument according to the demands
% of your shell.
%
% \paragraph{Generating the documentation.}
% You can use both the \xfile{.dtx} or the \xfile{.drv} to generate
% the documentation. The process can be configured by the
% configuration file \xfile{ltxdoc.cfg}. For instance, put this
% line into this file, if you want to have A4 as paper format:
% \begin{quote}
%   \verb|\PassOptionsToClass{a4paper}{article}|
% \end{quote}
% An example follows how to generate the
% documentation with pdf\LaTeX:
% \begin{quote}
%\begin{verbatim}
%pdflatex thepdfnumber.dtx
%makeindex -s gind.ist thepdfnumber.idx
%pdflatex thepdfnumber.dtx
%makeindex -s gind.ist thepdfnumber.idx
%pdflatex thepdfnumber.dtx
%\end{verbatim}
% \end{quote}
%
% \printbibliography[
%   heading=bibnumbered,
% ]
%
% \begin{History}
%   \begin{Version}{2011/11/24 v1.0}
%   \item
%     First version.
%   \end{Version}
%   \begin{Version}{2016/05/16 v1.1}
%   \item
%     Documentation updates.
%   \end{Version}
% \end{History}
%
% \PrintIndex
%
% \Finale
\endinput
|
% \end{quote}
% Do not forget to quote the argument according to the demands
% of your shell.
%
% \paragraph{Generating the documentation.}
% You can use both the \xfile{.dtx} or the \xfile{.drv} to generate
% the documentation. The process can be configured by the
% configuration file \xfile{ltxdoc.cfg}. For instance, put this
% line into this file, if you want to have A4 as paper format:
% \begin{quote}
%   \verb|\PassOptionsToClass{a4paper}{article}|
% \end{quote}
% An example follows how to generate the
% documentation with pdf\LaTeX:
% \begin{quote}
%\begin{verbatim}
%pdflatex thepdfnumber.dtx
%makeindex -s gind.ist thepdfnumber.idx
%pdflatex thepdfnumber.dtx
%makeindex -s gind.ist thepdfnumber.idx
%pdflatex thepdfnumber.dtx
%\end{verbatim}
% \end{quote}
%
% \printbibliography[
%   heading=bibnumbered,
% ]
%
% \begin{History}
%   \begin{Version}{2011/11/24 v1.0}
%   \item
%     First version.
%   \end{Version}
%   \begin{Version}{2016/05/16 v1.1}
%   \item
%     Documentation updates.
%   \end{Version}
% \end{History}
%
% \PrintIndex
%
% \Finale
\endinput
|
% \end{quote}
% Do not forget to quote the argument according to the demands
% of your shell.
%
% \paragraph{Generating the documentation.}
% You can use both the \xfile{.dtx} or the \xfile{.drv} to generate
% the documentation. The process can be configured by the
% configuration file \xfile{ltxdoc.cfg}. For instance, put this
% line into this file, if you want to have A4 as paper format:
% \begin{quote}
%   \verb|\PassOptionsToClass{a4paper}{article}|
% \end{quote}
% An example follows how to generate the
% documentation with pdf\LaTeX:
% \begin{quote}
%\begin{verbatim}
%pdflatex thepdfnumber.dtx
%makeindex -s gind.ist thepdfnumber.idx
%pdflatex thepdfnumber.dtx
%makeindex -s gind.ist thepdfnumber.idx
%pdflatex thepdfnumber.dtx
%\end{verbatim}
% \end{quote}
%
% \printbibliography[
%   heading=bibnumbered,
% ]
%
% \begin{History}
%   \begin{Version}{2011/11/24 v1.0}
%   \item
%     First version.
%   \end{Version}
%   \begin{Version}{2016/05/16 v1.1}
%   \item
%     Documentation updates.
%   \end{Version}
% \end{History}
%
% \PrintIndex
%
% \Finale
\endinput
