% \iffalse meta-comment
%
% File: twoopt.dtx
% Version: 2008/08/11 v1.5
% Info: Definitions with two optional arguments
%
% Copyright (C) 1999, 2006, 2008 by
%    Heiko Oberdiek <heiko.oberdiek at googlemail.com>
%
% This work may be distributed and/or modified under the
% conditions of the LaTeX Project Public License, either
% version 1.3c of this license or (at your option) any later
% version. This version of this license is in
%    http://www.latex-project.org/lppl/lppl-1-3c.txt
% and the latest version of this license is in
%    http://www.latex-project.org/lppl.txt
% and version 1.3 or later is part of all distributions of
% LaTeX version 2005/12/01 or later.
%
% This work has the LPPL maintenance status "maintained".
%
% This Current Maintainer of this work is Heiko Oberdiek.
%
% This work consists of the main source file twoopt.dtx
% and the derived files
%    twoopt.sty, twoopt.pdf, twoopt.ins, twoopt.drv.
%
% Distribution:
%    CTAN:macros/latex/contrib/oberdiek/twoopt.dtx
%    CTAN:macros/latex/contrib/oberdiek/twoopt.pdf
%
% Unpacking:
%    (a) If twoopt.ins is present:
%           tex twoopt.ins
%    (b) Without twoopt.ins:
%           tex twoopt.dtx
%    (c) If you insist on using LaTeX
%           latex \let\install=y% \iffalse meta-comment
%
% File: twoopt.dtx
% Version: 2016/05/16 v1.6
% Info: Definitions with two optional arguments
%
% Copyright (C) 1999, 2006, 2008 by
%    Heiko Oberdiek <heiko.oberdiek at googlemail.com>
%    2016
%    https://github.com/ho-tex/oberdiek/issues
%
% This work may be distributed and/or modified under the
% conditions of the LaTeX Project Public License, either
% version 1.3c of this license or (at your option) any later
% version. This version of this license is in
%    http://www.latex-project.org/lppl/lppl-1-3c.txt
% and the latest version of this license is in
%    http://www.latex-project.org/lppl.txt
% and version 1.3 or later is part of all distributions of
% LaTeX version 2005/12/01 or later.
%
% This work has the LPPL maintenance status "maintained".
%
% This Current Maintainer of this work is Heiko Oberdiek.
%
% This work consists of the main source file twoopt.dtx
% and the derived files
%    twoopt.sty, twoopt.pdf, twoopt.ins, twoopt.drv.
%
% Distribution:
%    CTAN:macros/latex/contrib/oberdiek/twoopt.dtx
%    CTAN:macros/latex/contrib/oberdiek/twoopt.pdf
%
% Unpacking:
%    (a) If twoopt.ins is present:
%           tex twoopt.ins
%    (b) Without twoopt.ins:
%           tex twoopt.dtx
%    (c) If you insist on using LaTeX
%           latex \let\install=y% \iffalse meta-comment
%
% File: twoopt.dtx
% Version: 2016/05/16 v1.6
% Info: Definitions with two optional arguments
%
% Copyright (C) 1999, 2006, 2008 by
%    Heiko Oberdiek <heiko.oberdiek at googlemail.com>
%    2016
%    https://github.com/ho-tex/oberdiek/issues
%
% This work may be distributed and/or modified under the
% conditions of the LaTeX Project Public License, either
% version 1.3c of this license or (at your option) any later
% version. This version of this license is in
%    http://www.latex-project.org/lppl/lppl-1-3c.txt
% and the latest version of this license is in
%    http://www.latex-project.org/lppl.txt
% and version 1.3 or later is part of all distributions of
% LaTeX version 2005/12/01 or later.
%
% This work has the LPPL maintenance status "maintained".
%
% This Current Maintainer of this work is Heiko Oberdiek.
%
% This work consists of the main source file twoopt.dtx
% and the derived files
%    twoopt.sty, twoopt.pdf, twoopt.ins, twoopt.drv.
%
% Distribution:
%    CTAN:macros/latex/contrib/oberdiek/twoopt.dtx
%    CTAN:macros/latex/contrib/oberdiek/twoopt.pdf
%
% Unpacking:
%    (a) If twoopt.ins is present:
%           tex twoopt.ins
%    (b) Without twoopt.ins:
%           tex twoopt.dtx
%    (c) If you insist on using LaTeX
%           latex \let\install=y% \iffalse meta-comment
%
% File: twoopt.dtx
% Version: 2016/05/16 v1.6
% Info: Definitions with two optional arguments
%
% Copyright (C) 1999, 2006, 2008 by
%    Heiko Oberdiek <heiko.oberdiek at googlemail.com>
%    2016
%    https://github.com/ho-tex/oberdiek/issues
%
% This work may be distributed and/or modified under the
% conditions of the LaTeX Project Public License, either
% version 1.3c of this license or (at your option) any later
% version. This version of this license is in
%    http://www.latex-project.org/lppl/lppl-1-3c.txt
% and the latest version of this license is in
%    http://www.latex-project.org/lppl.txt
% and version 1.3 or later is part of all distributions of
% LaTeX version 2005/12/01 or later.
%
% This work has the LPPL maintenance status "maintained".
%
% This Current Maintainer of this work is Heiko Oberdiek.
%
% This work consists of the main source file twoopt.dtx
% and the derived files
%    twoopt.sty, twoopt.pdf, twoopt.ins, twoopt.drv.
%
% Distribution:
%    CTAN:macros/latex/contrib/oberdiek/twoopt.dtx
%    CTAN:macros/latex/contrib/oberdiek/twoopt.pdf
%
% Unpacking:
%    (a) If twoopt.ins is present:
%           tex twoopt.ins
%    (b) Without twoopt.ins:
%           tex twoopt.dtx
%    (c) If you insist on using LaTeX
%           latex \let\install=y\input{twoopt.dtx}
%        (quote the arguments according to the demands of your shell)
%
% Documentation:
%    (a) If twoopt.drv is present:
%           latex twoopt.drv
%    (b) Without twoopt.drv:
%           latex twoopt.dtx; ...
%    The class ltxdoc loads the configuration file ltxdoc.cfg
%    if available. Here you can specify further options, e.g.
%    use A4 as paper format:
%       \PassOptionsToClass{a4paper}{article}
%
%    Programm calls to get the documentation (example):
%       pdflatex twoopt.dtx
%       makeindex -s gind.ist twoopt.idx
%       pdflatex twoopt.dtx
%       makeindex -s gind.ist twoopt.idx
%       pdflatex twoopt.dtx
%
% Installation:
%    TDS:tex/latex/oberdiek/twoopt.sty
%    TDS:doc/latex/oberdiek/twoopt.pdf
%    TDS:source/latex/oberdiek/twoopt.dtx
%
%<*ignore>
\begingroup
  \catcode123=1 %
  \catcode125=2 %
  \def\x{LaTeX2e}%
\expandafter\endgroup
\ifcase 0\ifx\install y1\fi\expandafter
         \ifx\csname processbatchFile\endcsname\relax\else1\fi
         \ifx\fmtname\x\else 1\fi\relax
\else\csname fi\endcsname
%</ignore>
%<*install>
\input docstrip.tex
\Msg{************************************************************************}
\Msg{* Installation}
\Msg{* Package: twoopt 2016/05/16 v1.6 Definitions with two optional arguments (HO)}
\Msg{************************************************************************}

\keepsilent
\askforoverwritefalse

\let\MetaPrefix\relax
\preamble

This is a generated file.

Project: twoopt
Version: 2016/05/16 v1.6

Copyright (C) 1999, 2006, 2008 by
   Heiko Oberdiek <heiko.oberdiek at googlemail.com>

This work may be distributed and/or modified under the
conditions of the LaTeX Project Public License, either
version 1.3c of this license or (at your option) any later
version. This version of this license is in
   http://www.latex-project.org/lppl/lppl-1-3c.txt
and the latest version of this license is in
   http://www.latex-project.org/lppl.txt
and version 1.3 or later is part of all distributions of
LaTeX version 2005/12/01 or later.

This work has the LPPL maintenance status "maintained".

This Current Maintainer of this work is Heiko Oberdiek.

This work consists of the main source file twoopt.dtx
and the derived files
   twoopt.sty, twoopt.pdf, twoopt.ins, twoopt.drv.

\endpreamble
\let\MetaPrefix\DoubleperCent

\generate{%
  \file{twoopt.ins}{\from{twoopt.dtx}{install}}%
  \file{twoopt.drv}{\from{twoopt.dtx}{driver}}%
  \usedir{tex/latex/oberdiek}%
  \file{twoopt.sty}{\from{twoopt.dtx}{package}}%
  \nopreamble
  \nopostamble
%  \usedir{source/latex/oberdiek/catalogue}%
%  \file{twoopt.xml}{\from{twoopt.dtx}{catalogue}}%
}

\catcode32=13\relax% active space
\let =\space%
\Msg{************************************************************************}
\Msg{*}
\Msg{* To finish the installation you have to move the following}
\Msg{* file into a directory searched by TeX:}
\Msg{*}
\Msg{*     twoopt.sty}
\Msg{*}
\Msg{* To produce the documentation run the file `twoopt.drv'}
\Msg{* through LaTeX.}
\Msg{*}
\Msg{* Happy TeXing!}
\Msg{*}
\Msg{************************************************************************}

\endbatchfile
%</install>
%<*ignore>
\fi
%</ignore>
%<*driver>
\NeedsTeXFormat{LaTeX2e}
\ProvidesFile{twoopt.drv}%
  [2016/05/16 v1.6 Definitions with two optional arguments (HO)]%
\documentclass{ltxdoc}
\usepackage{holtxdoc}[2011/11/22]
\begin{document}
  \DocInput{twoopt.dtx}%
\end{document}
%</driver>
% \fi
%
%
% \CharacterTable
%  {Upper-case    \A\B\C\D\E\F\G\H\I\J\K\L\M\N\O\P\Q\R\S\T\U\V\W\X\Y\Z
%   Lower-case    \a\b\c\d\e\f\g\h\i\j\k\l\m\n\o\p\q\r\s\t\u\v\w\x\y\z
%   Digits        \0\1\2\3\4\5\6\7\8\9
%   Exclamation   \!     Double quote  \"     Hash (number) \#
%   Dollar        \$     Percent       \%     Ampersand     \&
%   Acute accent  \'     Left paren    \(     Right paren   \)
%   Asterisk      \*     Plus          \+     Comma         \,
%   Minus         \-     Point         \.     Solidus       \/
%   Colon         \:     Semicolon     \;     Less than     \<
%   Equals        \=     Greater than  \>     Question mark \?
%   Commercial at \@     Left bracket  \[     Backslash     \\
%   Right bracket \]     Circumflex    \^     Underscore    \_
%   Grave accent  \`     Left brace    \{     Vertical bar  \|
%   Right brace   \}     Tilde         \~}
%
% \GetFileInfo{twoopt.drv}
%
% \title{The \xpackage{twoopt} package}
% \date{2016/05/16 v1.6}
% \author{Heiko Oberdiek\thanks
% {Please report any issues at https://github.com/ho-tex/oberdiek/issues}\\
% \xemail{heiko.oberdiek at googlemail.com}}
%
% \maketitle
%
% \begin{abstract}
% This package provides commands to define macros with two
% optional arguments.
% \end{abstract}
%
% \tableofcontents
%
% \newenvironment{param}{^^A
%   \newcommand{\entry}[1]{\meta{\###1}:&}^^A
%   \begin{tabular}[t]{@{}l@{ }l@{}}^^A
% }{^^A
%   \end{tabular}^^A
% }
%
% \section{Usage}
%    \DescribeMacro{\newcommandtwoopt}
%    \DescribeMacro{\renewcommandtwoopt}
%    \DescribeMacro{\providecommandtwoopt}
%    Similar to \cmd{\newcommand}, \cmd{\renewcommand}
%    and \cmd{\providecommand} this package provides commands
%    to define macros with two optional arguments.
%    The names of the commands are built by appending the
%    package name to the \LaTeX-pendants:
%    \begingroup
%      \def\x{\marg{cmd} \oarg{num} \oarg{default1}^^A
%             \oarg{default2} \marg{def.}}^^A
%      \begin{tabbing}
%        \cmd{\providecommandtwoopt} \=\kill
%        \cmd{\newcommandtwoopt}\>\x\\
%        \cmd{\renewcommandtwoopt}\>\x\\
%        \cmd{\providecommandtwoopt}\>\x\\
%      \end{tabbing}
%    \endgroup
%
%    Also the |*|-forms are supported. Indeed it is better to
%    use this ones, unless it is intended to hold
%    whole paragraphs in some of the arguments. If the macro
%    is defined with the |*|-form, missing braces
%    can be detected earlier.
%
%    Example:
%    \begin{quote}
%      |\newcommandtwoopt{\bsp}[3][AA][BB]{%|\\
%      |  \typeout{\string\bsp: #1,#2,#3}%|\\
%      |}|\\
%      \begin{tabular}{@{}l@{\quad$\rightarrow$\quad}l@{}}
%      |\bsp[aa][bb]{cc}|&|\bsp: aa,bb,cc|\\
%      |\bsp[aa]{cc}|&|\bsp: aa,BB,cc|\\
%      |\bsp{cc}|&|\bsp: AA,BB,cc|\\
%      \end{tabular}
%    \end{quote}
%
% \StopEventually{
% }
%
% \section{Implementation}
%    \begin{macrocode}
%<*package>
\NeedsTeXFormat{LaTeX2e}
\ProvidesPackage{twoopt}
  [2016/05/16 v1.6 Definitions with two optional arguments (HO)]%
%    \end{macrocode}
%    \begin{macro}{\newcommandtwoopt}
%    \begin{macrocode}
\newcommand{\newcommandtwoopt}{%
  \@ifstar{\@newcommandtwoopt*}{\@newcommandtwoopt{}}%
}
%    \end{macrocode}
%    \end{macro}
%
%    \begin{macro}{\@newcommandtwoopt}
%    \begin{param}
%      \entry1 star\\
%      \entry2 macro name to be defined
%    \end{param}
%    \begin{macrocode}
\newcommand{\@newcommandtwoopt}{}
\long\def\@newcommandtwoopt#1#2{%
  \expandafter\@@newcommandtwoopt
    \csname2\string#2\endcsname{#1}{#2}%
}
%    \end{macrocode}
%    \end{macro}
%
%    \begin{macro}{\@@newcommandtwoopt}
%    \begin{param}
%      \entry1 help command to be defined
%        (\expandafter\cmd\csname 2\bslash<name>\endcsname)\\
%      \entry2 star\\
%      \entry3 macro name to be defined\\
%      \entry4 number of total arguments\\
%      \entry5 default for optional argument one\\
%      \entry6 default for optional argument two
%    \end{param}
%    \begin{macrocode}
\newcommand{\@@newcommandtwoopt}{}
\long\def\@@newcommandtwoopt#1#2#3[#4][#5][#6]{%
  \newcommand#2#3[1][{#5}]{%
    \to@ScanSecondOptArg#1{##1}{#6}%
  }%
  \newcommand#2#1[{#4}]%
}
%    \end{macrocode}
%    \end{macro}
%
%    \begin{macro}{\renewcommandtwoopt}
%    \begin{macrocode}
\newcommand{\renewcommandtwoopt}{%
  \@ifstar{\@renewcommandtwoopt*}{\@renewcommandtwoopt{}}%
}
%    \end{macrocode}
%    \end{macro}
%
%    \begin{macro}{\@renewcommandtwoopt}
%    \begin{param}
%      \entry1 star\\
%      \entry2 command name to be defined
%    \end{param}
%    \begin{macrocode}
\newcommand{\@renewcommandtwoopt}{}
\long\def\@renewcommandtwoopt#1#2{%
  \begingroup
    \escapechar\m@ne
    \xdef\@gtempa{{\string#2}}%
  \endgroup
  \expandafter\@ifundefined\@gtempa{%
    \@latex@error{\noexpand#2undefined}\@ehc
  }{}%
  \let#2\@undefined
  \expandafter\let\csname2\string#2\endcsname\@undefined
  \expandafter\@@newcommandtwoopt
    \csname2\string#2\endcsname{#1}{#2}%
}
%    \end{macrocode}
%    \end{macro}
%
%    \begin{macro}{\providecommandtwoopt}
%    \begin{macrocode}
\newcommand{\providecommandtwoopt}{%
  \@ifstar{\@providecommandtwoopt*}{\@providecommandtwoopt{}}%
}
%    \end{macrocode}
%    \end{macro}
%
%    \begin{macro}{\@providecommandtwoopt}
%    \begin{param}
%      \entry1 star\\
%      \entry2 command name to be defined
%    \end{param}
%    \begin{macrocode}
\newcommand{\@providecommandtwoopt}{}
\long\def\@providecommandtwoopt#1#2{%
  \begingroup
    \escapechar\m@ne
    \xdef\@gtempa{{\string#2}}%
  \endgroup
  \expandafter\@ifundefined\@gtempa{%
    \expandafter\@@newcommandtwoopt
      \csname2\string#2\endcsname{#1}{#2}%
  }{%
    \let\to@dummyA\@undefined
    \let\to@dummyB\@undefined
    \@@newcommandtwoopt\to@dummyA{#1}\to@dummyB
  }%
}
%    \end{macrocode}
%    \end{macro}
%
%    \begin{macro}{\to@ScanSecondOptArg}
%    \begin{param}
%      \entry1 help command to be defined
%        (\expandafter\cmd\csname 2\bslash<name>\endcsname)\\
%      \entry2 first arg of command to be defined\\
%      \entry3 default for second opt. arg.
%    \end{param}
%    \begin{macrocode}
\newcommand{\to@ScanSecondOptArg}[3]{%
  \@ifnextchar[{%
    \expandafter#1\to@ArgOptToArgArg{#2}%
  }{%
    #1{#2}{#3}%
  }%
}
%    \end{macrocode}
%    \end{macro}
%
%    \begin{macro}{\to@ArgOptToArgArg}
%    \begin{macrocode}
\newcommand{\to@ArgOptToArgArg}{}
\long\def\to@ArgOptToArgArg#1[#2]{{#1}{#2}}
%    \end{macrocode}
%    \end{macro}
%
%    \begin{macrocode}
%</package>
%    \end{macrocode}
%
% \section{Installation}
%
% \subsection{Download}
%
% \paragraph{Package.} This package is available on
% CTAN\footnote{\url{http://ctan.org/pkg/twoopt}}:
% \begin{description}
% \item[\CTAN{macros/latex/contrib/oberdiek/twoopt.dtx}] The source file.
% \item[\CTAN{macros/latex/contrib/oberdiek/twoopt.pdf}] Documentation.
% \end{description}
%
%
% \paragraph{Bundle.} All the packages of the bundle `oberdiek'
% are also available in a TDS compliant ZIP archive. There
% the packages are already unpacked and the documentation files
% are generated. The files and directories obey the TDS standard.
% \begin{description}
% \item[\CTAN{install/macros/latex/contrib/oberdiek.tds.zip}]
% \end{description}
% \emph{TDS} refers to the standard ``A Directory Structure
% for \TeX\ Files'' (\CTAN{tds/tds.pdf}). Directories
% with \xfile{texmf} in their name are usually organized this way.
%
% \subsection{Bundle installation}
%
% \paragraph{Unpacking.} Unpack the \xfile{oberdiek.tds.zip} in the
% TDS tree (also known as \xfile{texmf} tree) of your choice.
% Example (linux):
% \begin{quote}
%   |unzip oberdiek.tds.zip -d ~/texmf|
% \end{quote}
%
% \paragraph{Script installation.}
% Check the directory \xfile{TDS:scripts/oberdiek/} for
% scripts that need further installation steps.
% Package \xpackage{attachfile2} comes with the Perl script
% \xfile{pdfatfi.pl} that should be installed in such a way
% that it can be called as \texttt{pdfatfi}.
% Example (linux):
% \begin{quote}
%   |chmod +x scripts/oberdiek/pdfatfi.pl|\\
%   |cp scripts/oberdiek/pdfatfi.pl /usr/local/bin/|
% \end{quote}
%
% \subsection{Package installation}
%
% \paragraph{Unpacking.} The \xfile{.dtx} file is a self-extracting
% \docstrip\ archive. The files are extracted by running the
% \xfile{.dtx} through \plainTeX:
% \begin{quote}
%   \verb|tex twoopt.dtx|
% \end{quote}
%
% \paragraph{TDS.} Now the different files must be moved into
% the different directories in your installation TDS tree
% (also known as \xfile{texmf} tree):
% \begin{quote}
% \def\t{^^A
% \begin{tabular}{@{}>{\ttfamily}l@{ $\rightarrow$ }>{\ttfamily}l@{}}
%   twoopt.sty & tex/latex/oberdiek/twoopt.sty\\
%   twoopt.pdf & doc/latex/oberdiek/twoopt.pdf\\
%   twoopt.dtx & source/latex/oberdiek/twoopt.dtx\\
% \end{tabular}^^A
% }^^A
% \sbox0{\t}^^A
% \ifdim\wd0>\linewidth
%   \begingroup
%     \advance\linewidth by\leftmargin
%     \advance\linewidth by\rightmargin
%   \edef\x{\endgroup
%     \def\noexpand\lw{\the\linewidth}^^A
%   }\x
%   \def\lwbox{^^A
%     \leavevmode
%     \hbox to \linewidth{^^A
%       \kern-\leftmargin\relax
%       \hss
%       \usebox0
%       \hss
%       \kern-\rightmargin\relax
%     }^^A
%   }^^A
%   \ifdim\wd0>\lw
%     \sbox0{\small\t}^^A
%     \ifdim\wd0>\linewidth
%       \ifdim\wd0>\lw
%         \sbox0{\footnotesize\t}^^A
%         \ifdim\wd0>\linewidth
%           \ifdim\wd0>\lw
%             \sbox0{\scriptsize\t}^^A
%             \ifdim\wd0>\linewidth
%               \ifdim\wd0>\lw
%                 \sbox0{\tiny\t}^^A
%                 \ifdim\wd0>\linewidth
%                   \lwbox
%                 \else
%                   \usebox0
%                 \fi
%               \else
%                 \lwbox
%               \fi
%             \else
%               \usebox0
%             \fi
%           \else
%             \lwbox
%           \fi
%         \else
%           \usebox0
%         \fi
%       \else
%         \lwbox
%       \fi
%     \else
%       \usebox0
%     \fi
%   \else
%     \lwbox
%   \fi
% \else
%   \usebox0
% \fi
% \end{quote}
% If you have a \xfile{docstrip.cfg} that configures and enables \docstrip's
% TDS installing feature, then some files can already be in the right
% place, see the documentation of \docstrip.
%
% \subsection{Refresh file name databases}
%
% If your \TeX~distribution
% (\teTeX, \mikTeX, \dots) relies on file name databases, you must refresh
% these. For example, \teTeX\ users run \verb|texhash| or
% \verb|mktexlsr|.
%
% \subsection{Some details for the interested}
%
% \paragraph{Attached source.}
%
% The PDF documentation on CTAN also includes the
% \xfile{.dtx} source file. It can be extracted by
% AcrobatReader 6 or higher. Another option is \textsf{pdftk},
% e.g. unpack the file into the current directory:
% \begin{quote}
%   \verb|pdftk twoopt.pdf unpack_files output .|
% \end{quote}
%
% \paragraph{Unpacking with \LaTeX.}
% The \xfile{.dtx} chooses its action depending on the format:
% \begin{description}
% \item[\plainTeX:] Run \docstrip\ and extract the files.
% \item[\LaTeX:] Generate the documentation.
% \end{description}
% If you insist on using \LaTeX\ for \docstrip\ (really,
% \docstrip\ does not need \LaTeX), then inform the autodetect routine
% about your intention:
% \begin{quote}
%   \verb|latex \let\install=y\input{twoopt.dtx}|
% \end{quote}
% Do not forget to quote the argument according to the demands
% of your shell.
%
% \paragraph{Generating the documentation.}
% You can use both the \xfile{.dtx} or the \xfile{.drv} to generate
% the documentation. The process can be configured by the
% configuration file \xfile{ltxdoc.cfg}. For instance, put this
% line into this file, if you want to have A4 as paper format:
% \begin{quote}
%   \verb|\PassOptionsToClass{a4paper}{article}|
% \end{quote}
% An example follows how to generate the
% documentation with pdf\LaTeX:
% \begin{quote}
%\begin{verbatim}
%pdflatex twoopt.dtx
%makeindex -s gind.ist twoopt.idx
%pdflatex twoopt.dtx
%makeindex -s gind.ist twoopt.idx
%pdflatex twoopt.dtx
%\end{verbatim}
% \end{quote}
%
% \section{Catalogue}
%
% The following XML file can be used as source for the
% \href{http://mirror.ctan.org/help/Catalogue/catalogue.html}{\TeX\ Catalogue}.
% The elements \texttt{caption} and \texttt{description} are imported
% from the original XML file from the Catalogue.
% The name of the XML file in the Catalogue is \xfile{twoopt.xml}.
%    \begin{macrocode}
%<*catalogue>
<?xml version='1.0' encoding='us-ascii'?>
<!DOCTYPE entry SYSTEM 'catalogue.dtd'>
<entry datestamp='$Date$' modifier='$Author$' id='twoopt'>
  <name>twoopt</name>
  <caption>Definitions with two optional arguments.</caption>
  <authorref id='auth:oberdiek'/>
  <copyright owner='Heiko Oberdiek' year='1999,2006,2008'/>
  <license type='lppl1.3'/>
  <version number='1.6'/>
  <description>
    Variants of <tt>\newcommand</tt>, <tt>\renewcommand</tt> and
    <tt>\providecommand</tt> are provided.
    <p/>
    The package is part of the <xref refid='oberdiek'>oberdiek</xref>
    bundle.
  </description>
  <documentation details='Package documentation'
      href='ctan:/macros/latex/contrib/oberdiek/twoopt.pdf'/>
  <ctan file='true' path='/macros/latex/contrib/oberdiek/twoopt.dtx'/>
  <miktex location='oberdiek'/>
  <texlive location='oberdiek'/>
  <install path='/macros/latex/contrib/oberdiek/oberdiek.tds.zip'/>
</entry>
%</catalogue>
%    \end{macrocode}
%
% \begin{History}
%   \begin{Version}{1998/10/30 v1.0}
%   \item
%     The first version was built as a response to a question
%     of \NameEmail{Rebecca and Rowland}{rebecca@astrid.u-net.com},
%     published in the newsgroup
%     \href{news:comp.text.tex}{comp.text.tex}:\\
%     \URL{``Re: [Q] LaTeX command with two optional arguments?''}^^A
%     {http://groups.google.com/group/comp.text.tex/msg/0ab1afde7b172d37}
%   \end{Version}
%   \begin{Version}{1998/10/30 v1.1}
%   \item
%     Improvements added in response to
%     \NameEmail{Stefan Ulrich}{ulrich@cis.uni-muenchen.de}
%     in the same thread:\\
%     \URL{``Re: [Q] LaTeX command with two optional arguments?''}^^A
%     {http://groups.google.com/group/comp.text.tex/msg/b8d84d4336f302c4}
%   \end{Version}
%   \begin{Version}{1998/11/04 v1.2}
%   \item
%     Fixes for LaTeX bugs 2896, 2901, 2902 added.
%   \end{Version}
%   \begin{Version}{1999/04/12 v1.3}
%   \item
%     Fixes removed because of LaTeX [1998/12/01].
%   \item
%     Documentation in dtx format.
%   \item
%     Copyright: LPPL (\CTAN{macros/latex/base/lppl.txt})
%   \item
%     First CTAN release.
%   \end{Version}
%   \begin{Version}{2006/02/20 v1.4}
%   \item
%     Code is not changed.
%   \item
%     New DTX framework.
%   \item
%     LPPL 1.3
%   \end{Version}
%   \begin{Version}{2008/08/11 v1.5}
%   \item
%     Code is not changed.
%   \item
%     URLs updated from \texttt{www.dejanews.com}
%     to \texttt{groups.google.com}.
%   \end{Version}
%   \begin{Version}{2016/05/16 v1.6}
%   \item
%     Documentation updates.
%   \end{Version}
% \end{History}
%
% \PrintIndex
%
% \Finale
\endinput

%        (quote the arguments according to the demands of your shell)
%
% Documentation:
%    (a) If twoopt.drv is present:
%           latex twoopt.drv
%    (b) Without twoopt.drv:
%           latex twoopt.dtx; ...
%    The class ltxdoc loads the configuration file ltxdoc.cfg
%    if available. Here you can specify further options, e.g.
%    use A4 as paper format:
%       \PassOptionsToClass{a4paper}{article}
%
%    Programm calls to get the documentation (example):
%       pdflatex twoopt.dtx
%       makeindex -s gind.ist twoopt.idx
%       pdflatex twoopt.dtx
%       makeindex -s gind.ist twoopt.idx
%       pdflatex twoopt.dtx
%
% Installation:
%    TDS:tex/latex/oberdiek/twoopt.sty
%    TDS:doc/latex/oberdiek/twoopt.pdf
%    TDS:source/latex/oberdiek/twoopt.dtx
%
%<*ignore>
\begingroup
  \catcode123=1 %
  \catcode125=2 %
  \def\x{LaTeX2e}%
\expandafter\endgroup
\ifcase 0\ifx\install y1\fi\expandafter
         \ifx\csname processbatchFile\endcsname\relax\else1\fi
         \ifx\fmtname\x\else 1\fi\relax
\else\csname fi\endcsname
%</ignore>
%<*install>
\input docstrip.tex
\Msg{************************************************************************}
\Msg{* Installation}
\Msg{* Package: twoopt 2016/05/16 v1.6 Definitions with two optional arguments (HO)}
\Msg{************************************************************************}

\keepsilent
\askforoverwritefalse

\let\MetaPrefix\relax
\preamble

This is a generated file.

Project: twoopt
Version: 2016/05/16 v1.6

Copyright (C) 1999, 2006, 2008 by
   Heiko Oberdiek <heiko.oberdiek at googlemail.com>

This work may be distributed and/or modified under the
conditions of the LaTeX Project Public License, either
version 1.3c of this license or (at your option) any later
version. This version of this license is in
   http://www.latex-project.org/lppl/lppl-1-3c.txt
and the latest version of this license is in
   http://www.latex-project.org/lppl.txt
and version 1.3 or later is part of all distributions of
LaTeX version 2005/12/01 or later.

This work has the LPPL maintenance status "maintained".

This Current Maintainer of this work is Heiko Oberdiek.

This work consists of the main source file twoopt.dtx
and the derived files
   twoopt.sty, twoopt.pdf, twoopt.ins, twoopt.drv.

\endpreamble
\let\MetaPrefix\DoubleperCent

\generate{%
  \file{twoopt.ins}{\from{twoopt.dtx}{install}}%
  \file{twoopt.drv}{\from{twoopt.dtx}{driver}}%
  \usedir{tex/latex/oberdiek}%
  \file{twoopt.sty}{\from{twoopt.dtx}{package}}%
  \nopreamble
  \nopostamble
%  \usedir{source/latex/oberdiek/catalogue}%
%  \file{twoopt.xml}{\from{twoopt.dtx}{catalogue}}%
}

\catcode32=13\relax% active space
\let =\space%
\Msg{************************************************************************}
\Msg{*}
\Msg{* To finish the installation you have to move the following}
\Msg{* file into a directory searched by TeX:}
\Msg{*}
\Msg{*     twoopt.sty}
\Msg{*}
\Msg{* To produce the documentation run the file `twoopt.drv'}
\Msg{* through LaTeX.}
\Msg{*}
\Msg{* Happy TeXing!}
\Msg{*}
\Msg{************************************************************************}

\endbatchfile
%</install>
%<*ignore>
\fi
%</ignore>
%<*driver>
\NeedsTeXFormat{LaTeX2e}
\ProvidesFile{twoopt.drv}%
  [2016/05/16 v1.6 Definitions with two optional arguments (HO)]%
\documentclass{ltxdoc}
\usepackage{holtxdoc}[2011/11/22]
\begin{document}
  \DocInput{twoopt.dtx}%
\end{document}
%</driver>
% \fi
%
%
% \CharacterTable
%  {Upper-case    \A\B\C\D\E\F\G\H\I\J\K\L\M\N\O\P\Q\R\S\T\U\V\W\X\Y\Z
%   Lower-case    \a\b\c\d\e\f\g\h\i\j\k\l\m\n\o\p\q\r\s\t\u\v\w\x\y\z
%   Digits        \0\1\2\3\4\5\6\7\8\9
%   Exclamation   \!     Double quote  \"     Hash (number) \#
%   Dollar        \$     Percent       \%     Ampersand     \&
%   Acute accent  \'     Left paren    \(     Right paren   \)
%   Asterisk      \*     Plus          \+     Comma         \,
%   Minus         \-     Point         \.     Solidus       \/
%   Colon         \:     Semicolon     \;     Less than     \<
%   Equals        \=     Greater than  \>     Question mark \?
%   Commercial at \@     Left bracket  \[     Backslash     \\
%   Right bracket \]     Circumflex    \^     Underscore    \_
%   Grave accent  \`     Left brace    \{     Vertical bar  \|
%   Right brace   \}     Tilde         \~}
%
% \GetFileInfo{twoopt.drv}
%
% \title{The \xpackage{twoopt} package}
% \date{2016/05/16 v1.6}
% \author{Heiko Oberdiek\thanks
% {Please report any issues at https://github.com/ho-tex/oberdiek/issues}\\
% \xemail{heiko.oberdiek at googlemail.com}}
%
% \maketitle
%
% \begin{abstract}
% This package provides commands to define macros with two
% optional arguments.
% \end{abstract}
%
% \tableofcontents
%
% \newenvironment{param}{^^A
%   \newcommand{\entry}[1]{\meta{\###1}:&}^^A
%   \begin{tabular}[t]{@{}l@{ }l@{}}^^A
% }{^^A
%   \end{tabular}^^A
% }
%
% \section{Usage}
%    \DescribeMacro{\newcommandtwoopt}
%    \DescribeMacro{\renewcommandtwoopt}
%    \DescribeMacro{\providecommandtwoopt}
%    Similar to \cmd{\newcommand}, \cmd{\renewcommand}
%    and \cmd{\providecommand} this package provides commands
%    to define macros with two optional arguments.
%    The names of the commands are built by appending the
%    package name to the \LaTeX-pendants:
%    \begingroup
%      \def\x{\marg{cmd} \oarg{num} \oarg{default1}^^A
%             \oarg{default2} \marg{def.}}^^A
%      \begin{tabbing}
%        \cmd{\providecommandtwoopt} \=\kill
%        \cmd{\newcommandtwoopt}\>\x\\
%        \cmd{\renewcommandtwoopt}\>\x\\
%        \cmd{\providecommandtwoopt}\>\x\\
%      \end{tabbing}
%    \endgroup
%
%    Also the |*|-forms are supported. Indeed it is better to
%    use this ones, unless it is intended to hold
%    whole paragraphs in some of the arguments. If the macro
%    is defined with the |*|-form, missing braces
%    can be detected earlier.
%
%    Example:
%    \begin{quote}
%      |\newcommandtwoopt{\bsp}[3][AA][BB]{%|\\
%      |  \typeout{\string\bsp: #1,#2,#3}%|\\
%      |}|\\
%      \begin{tabular}{@{}l@{\quad$\rightarrow$\quad}l@{}}
%      |\bsp[aa][bb]{cc}|&|\bsp: aa,bb,cc|\\
%      |\bsp[aa]{cc}|&|\bsp: aa,BB,cc|\\
%      |\bsp{cc}|&|\bsp: AA,BB,cc|\\
%      \end{tabular}
%    \end{quote}
%
% \StopEventually{
% }
%
% \section{Implementation}
%    \begin{macrocode}
%<*package>
\NeedsTeXFormat{LaTeX2e}
\ProvidesPackage{twoopt}
  [2016/05/16 v1.6 Definitions with two optional arguments (HO)]%
%    \end{macrocode}
%    \begin{macro}{\newcommandtwoopt}
%    \begin{macrocode}
\newcommand{\newcommandtwoopt}{%
  \@ifstar{\@newcommandtwoopt*}{\@newcommandtwoopt{}}%
}
%    \end{macrocode}
%    \end{macro}
%
%    \begin{macro}{\@newcommandtwoopt}
%    \begin{param}
%      \entry1 star\\
%      \entry2 macro name to be defined
%    \end{param}
%    \begin{macrocode}
\newcommand{\@newcommandtwoopt}{}
\long\def\@newcommandtwoopt#1#2{%
  \expandafter\@@newcommandtwoopt
    \csname2\string#2\endcsname{#1}{#2}%
}
%    \end{macrocode}
%    \end{macro}
%
%    \begin{macro}{\@@newcommandtwoopt}
%    \begin{param}
%      \entry1 help command to be defined
%        (\expandafter\cmd\csname 2\bslash<name>\endcsname)\\
%      \entry2 star\\
%      \entry3 macro name to be defined\\
%      \entry4 number of total arguments\\
%      \entry5 default for optional argument one\\
%      \entry6 default for optional argument two
%    \end{param}
%    \begin{macrocode}
\newcommand{\@@newcommandtwoopt}{}
\long\def\@@newcommandtwoopt#1#2#3[#4][#5][#6]{%
  \newcommand#2#3[1][{#5}]{%
    \to@ScanSecondOptArg#1{##1}{#6}%
  }%
  \newcommand#2#1[{#4}]%
}
%    \end{macrocode}
%    \end{macro}
%
%    \begin{macro}{\renewcommandtwoopt}
%    \begin{macrocode}
\newcommand{\renewcommandtwoopt}{%
  \@ifstar{\@renewcommandtwoopt*}{\@renewcommandtwoopt{}}%
}
%    \end{macrocode}
%    \end{macro}
%
%    \begin{macro}{\@renewcommandtwoopt}
%    \begin{param}
%      \entry1 star\\
%      \entry2 command name to be defined
%    \end{param}
%    \begin{macrocode}
\newcommand{\@renewcommandtwoopt}{}
\long\def\@renewcommandtwoopt#1#2{%
  \begingroup
    \escapechar\m@ne
    \xdef\@gtempa{{\string#2}}%
  \endgroup
  \expandafter\@ifundefined\@gtempa{%
    \@latex@error{\noexpand#2undefined}\@ehc
  }{}%
  \let#2\@undefined
  \expandafter\let\csname2\string#2\endcsname\@undefined
  \expandafter\@@newcommandtwoopt
    \csname2\string#2\endcsname{#1}{#2}%
}
%    \end{macrocode}
%    \end{macro}
%
%    \begin{macro}{\providecommandtwoopt}
%    \begin{macrocode}
\newcommand{\providecommandtwoopt}{%
  \@ifstar{\@providecommandtwoopt*}{\@providecommandtwoopt{}}%
}
%    \end{macrocode}
%    \end{macro}
%
%    \begin{macro}{\@providecommandtwoopt}
%    \begin{param}
%      \entry1 star\\
%      \entry2 command name to be defined
%    \end{param}
%    \begin{macrocode}
\newcommand{\@providecommandtwoopt}{}
\long\def\@providecommandtwoopt#1#2{%
  \begingroup
    \escapechar\m@ne
    \xdef\@gtempa{{\string#2}}%
  \endgroup
  \expandafter\@ifundefined\@gtempa{%
    \expandafter\@@newcommandtwoopt
      \csname2\string#2\endcsname{#1}{#2}%
  }{%
    \let\to@dummyA\@undefined
    \let\to@dummyB\@undefined
    \@@newcommandtwoopt\to@dummyA{#1}\to@dummyB
  }%
}
%    \end{macrocode}
%    \end{macro}
%
%    \begin{macro}{\to@ScanSecondOptArg}
%    \begin{param}
%      \entry1 help command to be defined
%        (\expandafter\cmd\csname 2\bslash<name>\endcsname)\\
%      \entry2 first arg of command to be defined\\
%      \entry3 default for second opt. arg.
%    \end{param}
%    \begin{macrocode}
\newcommand{\to@ScanSecondOptArg}[3]{%
  \@ifnextchar[{%
    \expandafter#1\to@ArgOptToArgArg{#2}%
  }{%
    #1{#2}{#3}%
  }%
}
%    \end{macrocode}
%    \end{macro}
%
%    \begin{macro}{\to@ArgOptToArgArg}
%    \begin{macrocode}
\newcommand{\to@ArgOptToArgArg}{}
\long\def\to@ArgOptToArgArg#1[#2]{{#1}{#2}}
%    \end{macrocode}
%    \end{macro}
%
%    \begin{macrocode}
%</package>
%    \end{macrocode}
%
% \section{Installation}
%
% \subsection{Download}
%
% \paragraph{Package.} This package is available on
% CTAN\footnote{\url{http://ctan.org/pkg/twoopt}}:
% \begin{description}
% \item[\CTAN{macros/latex/contrib/oberdiek/twoopt.dtx}] The source file.
% \item[\CTAN{macros/latex/contrib/oberdiek/twoopt.pdf}] Documentation.
% \end{description}
%
%
% \paragraph{Bundle.} All the packages of the bundle `oberdiek'
% are also available in a TDS compliant ZIP archive. There
% the packages are already unpacked and the documentation files
% are generated. The files and directories obey the TDS standard.
% \begin{description}
% \item[\CTAN{install/macros/latex/contrib/oberdiek.tds.zip}]
% \end{description}
% \emph{TDS} refers to the standard ``A Directory Structure
% for \TeX\ Files'' (\CTAN{tds/tds.pdf}). Directories
% with \xfile{texmf} in their name are usually organized this way.
%
% \subsection{Bundle installation}
%
% \paragraph{Unpacking.} Unpack the \xfile{oberdiek.tds.zip} in the
% TDS tree (also known as \xfile{texmf} tree) of your choice.
% Example (linux):
% \begin{quote}
%   |unzip oberdiek.tds.zip -d ~/texmf|
% \end{quote}
%
% \paragraph{Script installation.}
% Check the directory \xfile{TDS:scripts/oberdiek/} for
% scripts that need further installation steps.
% Package \xpackage{attachfile2} comes with the Perl script
% \xfile{pdfatfi.pl} that should be installed in such a way
% that it can be called as \texttt{pdfatfi}.
% Example (linux):
% \begin{quote}
%   |chmod +x scripts/oberdiek/pdfatfi.pl|\\
%   |cp scripts/oberdiek/pdfatfi.pl /usr/local/bin/|
% \end{quote}
%
% \subsection{Package installation}
%
% \paragraph{Unpacking.} The \xfile{.dtx} file is a self-extracting
% \docstrip\ archive. The files are extracted by running the
% \xfile{.dtx} through \plainTeX:
% \begin{quote}
%   \verb|tex twoopt.dtx|
% \end{quote}
%
% \paragraph{TDS.} Now the different files must be moved into
% the different directories in your installation TDS tree
% (also known as \xfile{texmf} tree):
% \begin{quote}
% \def\t{^^A
% \begin{tabular}{@{}>{\ttfamily}l@{ $\rightarrow$ }>{\ttfamily}l@{}}
%   twoopt.sty & tex/latex/oberdiek/twoopt.sty\\
%   twoopt.pdf & doc/latex/oberdiek/twoopt.pdf\\
%   twoopt.dtx & source/latex/oberdiek/twoopt.dtx\\
% \end{tabular}^^A
% }^^A
% \sbox0{\t}^^A
% \ifdim\wd0>\linewidth
%   \begingroup
%     \advance\linewidth by\leftmargin
%     \advance\linewidth by\rightmargin
%   \edef\x{\endgroup
%     \def\noexpand\lw{\the\linewidth}^^A
%   }\x
%   \def\lwbox{^^A
%     \leavevmode
%     \hbox to \linewidth{^^A
%       \kern-\leftmargin\relax
%       \hss
%       \usebox0
%       \hss
%       \kern-\rightmargin\relax
%     }^^A
%   }^^A
%   \ifdim\wd0>\lw
%     \sbox0{\small\t}^^A
%     \ifdim\wd0>\linewidth
%       \ifdim\wd0>\lw
%         \sbox0{\footnotesize\t}^^A
%         \ifdim\wd0>\linewidth
%           \ifdim\wd0>\lw
%             \sbox0{\scriptsize\t}^^A
%             \ifdim\wd0>\linewidth
%               \ifdim\wd0>\lw
%                 \sbox0{\tiny\t}^^A
%                 \ifdim\wd0>\linewidth
%                   \lwbox
%                 \else
%                   \usebox0
%                 \fi
%               \else
%                 \lwbox
%               \fi
%             \else
%               \usebox0
%             \fi
%           \else
%             \lwbox
%           \fi
%         \else
%           \usebox0
%         \fi
%       \else
%         \lwbox
%       \fi
%     \else
%       \usebox0
%     \fi
%   \else
%     \lwbox
%   \fi
% \else
%   \usebox0
% \fi
% \end{quote}
% If you have a \xfile{docstrip.cfg} that configures and enables \docstrip's
% TDS installing feature, then some files can already be in the right
% place, see the documentation of \docstrip.
%
% \subsection{Refresh file name databases}
%
% If your \TeX~distribution
% (\teTeX, \mikTeX, \dots) relies on file name databases, you must refresh
% these. For example, \teTeX\ users run \verb|texhash| or
% \verb|mktexlsr|.
%
% \subsection{Some details for the interested}
%
% \paragraph{Attached source.}
%
% The PDF documentation on CTAN also includes the
% \xfile{.dtx} source file. It can be extracted by
% AcrobatReader 6 or higher. Another option is \textsf{pdftk},
% e.g. unpack the file into the current directory:
% \begin{quote}
%   \verb|pdftk twoopt.pdf unpack_files output .|
% \end{quote}
%
% \paragraph{Unpacking with \LaTeX.}
% The \xfile{.dtx} chooses its action depending on the format:
% \begin{description}
% \item[\plainTeX:] Run \docstrip\ and extract the files.
% \item[\LaTeX:] Generate the documentation.
% \end{description}
% If you insist on using \LaTeX\ for \docstrip\ (really,
% \docstrip\ does not need \LaTeX), then inform the autodetect routine
% about your intention:
% \begin{quote}
%   \verb|latex \let\install=y% \iffalse meta-comment
%
% File: twoopt.dtx
% Version: 2016/05/16 v1.6
% Info: Definitions with two optional arguments
%
% Copyright (C) 1999, 2006, 2008 by
%    Heiko Oberdiek <heiko.oberdiek at googlemail.com>
%    2016
%    https://github.com/ho-tex/oberdiek/issues
%
% This work may be distributed and/or modified under the
% conditions of the LaTeX Project Public License, either
% version 1.3c of this license or (at your option) any later
% version. This version of this license is in
%    http://www.latex-project.org/lppl/lppl-1-3c.txt
% and the latest version of this license is in
%    http://www.latex-project.org/lppl.txt
% and version 1.3 or later is part of all distributions of
% LaTeX version 2005/12/01 or later.
%
% This work has the LPPL maintenance status "maintained".
%
% This Current Maintainer of this work is Heiko Oberdiek.
%
% This work consists of the main source file twoopt.dtx
% and the derived files
%    twoopt.sty, twoopt.pdf, twoopt.ins, twoopt.drv.
%
% Distribution:
%    CTAN:macros/latex/contrib/oberdiek/twoopt.dtx
%    CTAN:macros/latex/contrib/oberdiek/twoopt.pdf
%
% Unpacking:
%    (a) If twoopt.ins is present:
%           tex twoopt.ins
%    (b) Without twoopt.ins:
%           tex twoopt.dtx
%    (c) If you insist on using LaTeX
%           latex \let\install=y\input{twoopt.dtx}
%        (quote the arguments according to the demands of your shell)
%
% Documentation:
%    (a) If twoopt.drv is present:
%           latex twoopt.drv
%    (b) Without twoopt.drv:
%           latex twoopt.dtx; ...
%    The class ltxdoc loads the configuration file ltxdoc.cfg
%    if available. Here you can specify further options, e.g.
%    use A4 as paper format:
%       \PassOptionsToClass{a4paper}{article}
%
%    Programm calls to get the documentation (example):
%       pdflatex twoopt.dtx
%       makeindex -s gind.ist twoopt.idx
%       pdflatex twoopt.dtx
%       makeindex -s gind.ist twoopt.idx
%       pdflatex twoopt.dtx
%
% Installation:
%    TDS:tex/latex/oberdiek/twoopt.sty
%    TDS:doc/latex/oberdiek/twoopt.pdf
%    TDS:source/latex/oberdiek/twoopt.dtx
%
%<*ignore>
\begingroup
  \catcode123=1 %
  \catcode125=2 %
  \def\x{LaTeX2e}%
\expandafter\endgroup
\ifcase 0\ifx\install y1\fi\expandafter
         \ifx\csname processbatchFile\endcsname\relax\else1\fi
         \ifx\fmtname\x\else 1\fi\relax
\else\csname fi\endcsname
%</ignore>
%<*install>
\input docstrip.tex
\Msg{************************************************************************}
\Msg{* Installation}
\Msg{* Package: twoopt 2016/05/16 v1.6 Definitions with two optional arguments (HO)}
\Msg{************************************************************************}

\keepsilent
\askforoverwritefalse

\let\MetaPrefix\relax
\preamble

This is a generated file.

Project: twoopt
Version: 2016/05/16 v1.6

Copyright (C) 1999, 2006, 2008 by
   Heiko Oberdiek <heiko.oberdiek at googlemail.com>

This work may be distributed and/or modified under the
conditions of the LaTeX Project Public License, either
version 1.3c of this license or (at your option) any later
version. This version of this license is in
   http://www.latex-project.org/lppl/lppl-1-3c.txt
and the latest version of this license is in
   http://www.latex-project.org/lppl.txt
and version 1.3 or later is part of all distributions of
LaTeX version 2005/12/01 or later.

This work has the LPPL maintenance status "maintained".

This Current Maintainer of this work is Heiko Oberdiek.

This work consists of the main source file twoopt.dtx
and the derived files
   twoopt.sty, twoopt.pdf, twoopt.ins, twoopt.drv.

\endpreamble
\let\MetaPrefix\DoubleperCent

\generate{%
  \file{twoopt.ins}{\from{twoopt.dtx}{install}}%
  \file{twoopt.drv}{\from{twoopt.dtx}{driver}}%
  \usedir{tex/latex/oberdiek}%
  \file{twoopt.sty}{\from{twoopt.dtx}{package}}%
  \nopreamble
  \nopostamble
%  \usedir{source/latex/oberdiek/catalogue}%
%  \file{twoopt.xml}{\from{twoopt.dtx}{catalogue}}%
}

\catcode32=13\relax% active space
\let =\space%
\Msg{************************************************************************}
\Msg{*}
\Msg{* To finish the installation you have to move the following}
\Msg{* file into a directory searched by TeX:}
\Msg{*}
\Msg{*     twoopt.sty}
\Msg{*}
\Msg{* To produce the documentation run the file `twoopt.drv'}
\Msg{* through LaTeX.}
\Msg{*}
\Msg{* Happy TeXing!}
\Msg{*}
\Msg{************************************************************************}

\endbatchfile
%</install>
%<*ignore>
\fi
%</ignore>
%<*driver>
\NeedsTeXFormat{LaTeX2e}
\ProvidesFile{twoopt.drv}%
  [2016/05/16 v1.6 Definitions with two optional arguments (HO)]%
\documentclass{ltxdoc}
\usepackage{holtxdoc}[2011/11/22]
\begin{document}
  \DocInput{twoopt.dtx}%
\end{document}
%</driver>
% \fi
%
%
% \CharacterTable
%  {Upper-case    \A\B\C\D\E\F\G\H\I\J\K\L\M\N\O\P\Q\R\S\T\U\V\W\X\Y\Z
%   Lower-case    \a\b\c\d\e\f\g\h\i\j\k\l\m\n\o\p\q\r\s\t\u\v\w\x\y\z
%   Digits        \0\1\2\3\4\5\6\7\8\9
%   Exclamation   \!     Double quote  \"     Hash (number) \#
%   Dollar        \$     Percent       \%     Ampersand     \&
%   Acute accent  \'     Left paren    \(     Right paren   \)
%   Asterisk      \*     Plus          \+     Comma         \,
%   Minus         \-     Point         \.     Solidus       \/
%   Colon         \:     Semicolon     \;     Less than     \<
%   Equals        \=     Greater than  \>     Question mark \?
%   Commercial at \@     Left bracket  \[     Backslash     \\
%   Right bracket \]     Circumflex    \^     Underscore    \_
%   Grave accent  \`     Left brace    \{     Vertical bar  \|
%   Right brace   \}     Tilde         \~}
%
% \GetFileInfo{twoopt.drv}
%
% \title{The \xpackage{twoopt} package}
% \date{2016/05/16 v1.6}
% \author{Heiko Oberdiek\thanks
% {Please report any issues at https://github.com/ho-tex/oberdiek/issues}\\
% \xemail{heiko.oberdiek at googlemail.com}}
%
% \maketitle
%
% \begin{abstract}
% This package provides commands to define macros with two
% optional arguments.
% \end{abstract}
%
% \tableofcontents
%
% \newenvironment{param}{^^A
%   \newcommand{\entry}[1]{\meta{\###1}:&}^^A
%   \begin{tabular}[t]{@{}l@{ }l@{}}^^A
% }{^^A
%   \end{tabular}^^A
% }
%
% \section{Usage}
%    \DescribeMacro{\newcommandtwoopt}
%    \DescribeMacro{\renewcommandtwoopt}
%    \DescribeMacro{\providecommandtwoopt}
%    Similar to \cmd{\newcommand}, \cmd{\renewcommand}
%    and \cmd{\providecommand} this package provides commands
%    to define macros with two optional arguments.
%    The names of the commands are built by appending the
%    package name to the \LaTeX-pendants:
%    \begingroup
%      \def\x{\marg{cmd} \oarg{num} \oarg{default1}^^A
%             \oarg{default2} \marg{def.}}^^A
%      \begin{tabbing}
%        \cmd{\providecommandtwoopt} \=\kill
%        \cmd{\newcommandtwoopt}\>\x\\
%        \cmd{\renewcommandtwoopt}\>\x\\
%        \cmd{\providecommandtwoopt}\>\x\\
%      \end{tabbing}
%    \endgroup
%
%    Also the |*|-forms are supported. Indeed it is better to
%    use this ones, unless it is intended to hold
%    whole paragraphs in some of the arguments. If the macro
%    is defined with the |*|-form, missing braces
%    can be detected earlier.
%
%    Example:
%    \begin{quote}
%      |\newcommandtwoopt{\bsp}[3][AA][BB]{%|\\
%      |  \typeout{\string\bsp: #1,#2,#3}%|\\
%      |}|\\
%      \begin{tabular}{@{}l@{\quad$\rightarrow$\quad}l@{}}
%      |\bsp[aa][bb]{cc}|&|\bsp: aa,bb,cc|\\
%      |\bsp[aa]{cc}|&|\bsp: aa,BB,cc|\\
%      |\bsp{cc}|&|\bsp: AA,BB,cc|\\
%      \end{tabular}
%    \end{quote}
%
% \StopEventually{
% }
%
% \section{Implementation}
%    \begin{macrocode}
%<*package>
\NeedsTeXFormat{LaTeX2e}
\ProvidesPackage{twoopt}
  [2016/05/16 v1.6 Definitions with two optional arguments (HO)]%
%    \end{macrocode}
%    \begin{macro}{\newcommandtwoopt}
%    \begin{macrocode}
\newcommand{\newcommandtwoopt}{%
  \@ifstar{\@newcommandtwoopt*}{\@newcommandtwoopt{}}%
}
%    \end{macrocode}
%    \end{macro}
%
%    \begin{macro}{\@newcommandtwoopt}
%    \begin{param}
%      \entry1 star\\
%      \entry2 macro name to be defined
%    \end{param}
%    \begin{macrocode}
\newcommand{\@newcommandtwoopt}{}
\long\def\@newcommandtwoopt#1#2{%
  \expandafter\@@newcommandtwoopt
    \csname2\string#2\endcsname{#1}{#2}%
}
%    \end{macrocode}
%    \end{macro}
%
%    \begin{macro}{\@@newcommandtwoopt}
%    \begin{param}
%      \entry1 help command to be defined
%        (\expandafter\cmd\csname 2\bslash<name>\endcsname)\\
%      \entry2 star\\
%      \entry3 macro name to be defined\\
%      \entry4 number of total arguments\\
%      \entry5 default for optional argument one\\
%      \entry6 default for optional argument two
%    \end{param}
%    \begin{macrocode}
\newcommand{\@@newcommandtwoopt}{}
\long\def\@@newcommandtwoopt#1#2#3[#4][#5][#6]{%
  \newcommand#2#3[1][{#5}]{%
    \to@ScanSecondOptArg#1{##1}{#6}%
  }%
  \newcommand#2#1[{#4}]%
}
%    \end{macrocode}
%    \end{macro}
%
%    \begin{macro}{\renewcommandtwoopt}
%    \begin{macrocode}
\newcommand{\renewcommandtwoopt}{%
  \@ifstar{\@renewcommandtwoopt*}{\@renewcommandtwoopt{}}%
}
%    \end{macrocode}
%    \end{macro}
%
%    \begin{macro}{\@renewcommandtwoopt}
%    \begin{param}
%      \entry1 star\\
%      \entry2 command name to be defined
%    \end{param}
%    \begin{macrocode}
\newcommand{\@renewcommandtwoopt}{}
\long\def\@renewcommandtwoopt#1#2{%
  \begingroup
    \escapechar\m@ne
    \xdef\@gtempa{{\string#2}}%
  \endgroup
  \expandafter\@ifundefined\@gtempa{%
    \@latex@error{\noexpand#2undefined}\@ehc
  }{}%
  \let#2\@undefined
  \expandafter\let\csname2\string#2\endcsname\@undefined
  \expandafter\@@newcommandtwoopt
    \csname2\string#2\endcsname{#1}{#2}%
}
%    \end{macrocode}
%    \end{macro}
%
%    \begin{macro}{\providecommandtwoopt}
%    \begin{macrocode}
\newcommand{\providecommandtwoopt}{%
  \@ifstar{\@providecommandtwoopt*}{\@providecommandtwoopt{}}%
}
%    \end{macrocode}
%    \end{macro}
%
%    \begin{macro}{\@providecommandtwoopt}
%    \begin{param}
%      \entry1 star\\
%      \entry2 command name to be defined
%    \end{param}
%    \begin{macrocode}
\newcommand{\@providecommandtwoopt}{}
\long\def\@providecommandtwoopt#1#2{%
  \begingroup
    \escapechar\m@ne
    \xdef\@gtempa{{\string#2}}%
  \endgroup
  \expandafter\@ifundefined\@gtempa{%
    \expandafter\@@newcommandtwoopt
      \csname2\string#2\endcsname{#1}{#2}%
  }{%
    \let\to@dummyA\@undefined
    \let\to@dummyB\@undefined
    \@@newcommandtwoopt\to@dummyA{#1}\to@dummyB
  }%
}
%    \end{macrocode}
%    \end{macro}
%
%    \begin{macro}{\to@ScanSecondOptArg}
%    \begin{param}
%      \entry1 help command to be defined
%        (\expandafter\cmd\csname 2\bslash<name>\endcsname)\\
%      \entry2 first arg of command to be defined\\
%      \entry3 default for second opt. arg.
%    \end{param}
%    \begin{macrocode}
\newcommand{\to@ScanSecondOptArg}[3]{%
  \@ifnextchar[{%
    \expandafter#1\to@ArgOptToArgArg{#2}%
  }{%
    #1{#2}{#3}%
  }%
}
%    \end{macrocode}
%    \end{macro}
%
%    \begin{macro}{\to@ArgOptToArgArg}
%    \begin{macrocode}
\newcommand{\to@ArgOptToArgArg}{}
\long\def\to@ArgOptToArgArg#1[#2]{{#1}{#2}}
%    \end{macrocode}
%    \end{macro}
%
%    \begin{macrocode}
%</package>
%    \end{macrocode}
%
% \section{Installation}
%
% \subsection{Download}
%
% \paragraph{Package.} This package is available on
% CTAN\footnote{\url{http://ctan.org/pkg/twoopt}}:
% \begin{description}
% \item[\CTAN{macros/latex/contrib/oberdiek/twoopt.dtx}] The source file.
% \item[\CTAN{macros/latex/contrib/oberdiek/twoopt.pdf}] Documentation.
% \end{description}
%
%
% \paragraph{Bundle.} All the packages of the bundle `oberdiek'
% are also available in a TDS compliant ZIP archive. There
% the packages are already unpacked and the documentation files
% are generated. The files and directories obey the TDS standard.
% \begin{description}
% \item[\CTAN{install/macros/latex/contrib/oberdiek.tds.zip}]
% \end{description}
% \emph{TDS} refers to the standard ``A Directory Structure
% for \TeX\ Files'' (\CTAN{tds/tds.pdf}). Directories
% with \xfile{texmf} in their name are usually organized this way.
%
% \subsection{Bundle installation}
%
% \paragraph{Unpacking.} Unpack the \xfile{oberdiek.tds.zip} in the
% TDS tree (also known as \xfile{texmf} tree) of your choice.
% Example (linux):
% \begin{quote}
%   |unzip oberdiek.tds.zip -d ~/texmf|
% \end{quote}
%
% \paragraph{Script installation.}
% Check the directory \xfile{TDS:scripts/oberdiek/} for
% scripts that need further installation steps.
% Package \xpackage{attachfile2} comes with the Perl script
% \xfile{pdfatfi.pl} that should be installed in such a way
% that it can be called as \texttt{pdfatfi}.
% Example (linux):
% \begin{quote}
%   |chmod +x scripts/oberdiek/pdfatfi.pl|\\
%   |cp scripts/oberdiek/pdfatfi.pl /usr/local/bin/|
% \end{quote}
%
% \subsection{Package installation}
%
% \paragraph{Unpacking.} The \xfile{.dtx} file is a self-extracting
% \docstrip\ archive. The files are extracted by running the
% \xfile{.dtx} through \plainTeX:
% \begin{quote}
%   \verb|tex twoopt.dtx|
% \end{quote}
%
% \paragraph{TDS.} Now the different files must be moved into
% the different directories in your installation TDS tree
% (also known as \xfile{texmf} tree):
% \begin{quote}
% \def\t{^^A
% \begin{tabular}{@{}>{\ttfamily}l@{ $\rightarrow$ }>{\ttfamily}l@{}}
%   twoopt.sty & tex/latex/oberdiek/twoopt.sty\\
%   twoopt.pdf & doc/latex/oberdiek/twoopt.pdf\\
%   twoopt.dtx & source/latex/oberdiek/twoopt.dtx\\
% \end{tabular}^^A
% }^^A
% \sbox0{\t}^^A
% \ifdim\wd0>\linewidth
%   \begingroup
%     \advance\linewidth by\leftmargin
%     \advance\linewidth by\rightmargin
%   \edef\x{\endgroup
%     \def\noexpand\lw{\the\linewidth}^^A
%   }\x
%   \def\lwbox{^^A
%     \leavevmode
%     \hbox to \linewidth{^^A
%       \kern-\leftmargin\relax
%       \hss
%       \usebox0
%       \hss
%       \kern-\rightmargin\relax
%     }^^A
%   }^^A
%   \ifdim\wd0>\lw
%     \sbox0{\small\t}^^A
%     \ifdim\wd0>\linewidth
%       \ifdim\wd0>\lw
%         \sbox0{\footnotesize\t}^^A
%         \ifdim\wd0>\linewidth
%           \ifdim\wd0>\lw
%             \sbox0{\scriptsize\t}^^A
%             \ifdim\wd0>\linewidth
%               \ifdim\wd0>\lw
%                 \sbox0{\tiny\t}^^A
%                 \ifdim\wd0>\linewidth
%                   \lwbox
%                 \else
%                   \usebox0
%                 \fi
%               \else
%                 \lwbox
%               \fi
%             \else
%               \usebox0
%             \fi
%           \else
%             \lwbox
%           \fi
%         \else
%           \usebox0
%         \fi
%       \else
%         \lwbox
%       \fi
%     \else
%       \usebox0
%     \fi
%   \else
%     \lwbox
%   \fi
% \else
%   \usebox0
% \fi
% \end{quote}
% If you have a \xfile{docstrip.cfg} that configures and enables \docstrip's
% TDS installing feature, then some files can already be in the right
% place, see the documentation of \docstrip.
%
% \subsection{Refresh file name databases}
%
% If your \TeX~distribution
% (\teTeX, \mikTeX, \dots) relies on file name databases, you must refresh
% these. For example, \teTeX\ users run \verb|texhash| or
% \verb|mktexlsr|.
%
% \subsection{Some details for the interested}
%
% \paragraph{Attached source.}
%
% The PDF documentation on CTAN also includes the
% \xfile{.dtx} source file. It can be extracted by
% AcrobatReader 6 or higher. Another option is \textsf{pdftk},
% e.g. unpack the file into the current directory:
% \begin{quote}
%   \verb|pdftk twoopt.pdf unpack_files output .|
% \end{quote}
%
% \paragraph{Unpacking with \LaTeX.}
% The \xfile{.dtx} chooses its action depending on the format:
% \begin{description}
% \item[\plainTeX:] Run \docstrip\ and extract the files.
% \item[\LaTeX:] Generate the documentation.
% \end{description}
% If you insist on using \LaTeX\ for \docstrip\ (really,
% \docstrip\ does not need \LaTeX), then inform the autodetect routine
% about your intention:
% \begin{quote}
%   \verb|latex \let\install=y\input{twoopt.dtx}|
% \end{quote}
% Do not forget to quote the argument according to the demands
% of your shell.
%
% \paragraph{Generating the documentation.}
% You can use both the \xfile{.dtx} or the \xfile{.drv} to generate
% the documentation. The process can be configured by the
% configuration file \xfile{ltxdoc.cfg}. For instance, put this
% line into this file, if you want to have A4 as paper format:
% \begin{quote}
%   \verb|\PassOptionsToClass{a4paper}{article}|
% \end{quote}
% An example follows how to generate the
% documentation with pdf\LaTeX:
% \begin{quote}
%\begin{verbatim}
%pdflatex twoopt.dtx
%makeindex -s gind.ist twoopt.idx
%pdflatex twoopt.dtx
%makeindex -s gind.ist twoopt.idx
%pdflatex twoopt.dtx
%\end{verbatim}
% \end{quote}
%
% \section{Catalogue}
%
% The following XML file can be used as source for the
% \href{http://mirror.ctan.org/help/Catalogue/catalogue.html}{\TeX\ Catalogue}.
% The elements \texttt{caption} and \texttt{description} are imported
% from the original XML file from the Catalogue.
% The name of the XML file in the Catalogue is \xfile{twoopt.xml}.
%    \begin{macrocode}
%<*catalogue>
<?xml version='1.0' encoding='us-ascii'?>
<!DOCTYPE entry SYSTEM 'catalogue.dtd'>
<entry datestamp='$Date$' modifier='$Author$' id='twoopt'>
  <name>twoopt</name>
  <caption>Definitions with two optional arguments.</caption>
  <authorref id='auth:oberdiek'/>
  <copyright owner='Heiko Oberdiek' year='1999,2006,2008'/>
  <license type='lppl1.3'/>
  <version number='1.6'/>
  <description>
    Variants of <tt>\newcommand</tt>, <tt>\renewcommand</tt> and
    <tt>\providecommand</tt> are provided.
    <p/>
    The package is part of the <xref refid='oberdiek'>oberdiek</xref>
    bundle.
  </description>
  <documentation details='Package documentation'
      href='ctan:/macros/latex/contrib/oberdiek/twoopt.pdf'/>
  <ctan file='true' path='/macros/latex/contrib/oberdiek/twoopt.dtx'/>
  <miktex location='oberdiek'/>
  <texlive location='oberdiek'/>
  <install path='/macros/latex/contrib/oberdiek/oberdiek.tds.zip'/>
</entry>
%</catalogue>
%    \end{macrocode}
%
% \begin{History}
%   \begin{Version}{1998/10/30 v1.0}
%   \item
%     The first version was built as a response to a question
%     of \NameEmail{Rebecca and Rowland}{rebecca@astrid.u-net.com},
%     published in the newsgroup
%     \href{news:comp.text.tex}{comp.text.tex}:\\
%     \URL{``Re: [Q] LaTeX command with two optional arguments?''}^^A
%     {http://groups.google.com/group/comp.text.tex/msg/0ab1afde7b172d37}
%   \end{Version}
%   \begin{Version}{1998/10/30 v1.1}
%   \item
%     Improvements added in response to
%     \NameEmail{Stefan Ulrich}{ulrich@cis.uni-muenchen.de}
%     in the same thread:\\
%     \URL{``Re: [Q] LaTeX command with two optional arguments?''}^^A
%     {http://groups.google.com/group/comp.text.tex/msg/b8d84d4336f302c4}
%   \end{Version}
%   \begin{Version}{1998/11/04 v1.2}
%   \item
%     Fixes for LaTeX bugs 2896, 2901, 2902 added.
%   \end{Version}
%   \begin{Version}{1999/04/12 v1.3}
%   \item
%     Fixes removed because of LaTeX [1998/12/01].
%   \item
%     Documentation in dtx format.
%   \item
%     Copyright: LPPL (\CTAN{macros/latex/base/lppl.txt})
%   \item
%     First CTAN release.
%   \end{Version}
%   \begin{Version}{2006/02/20 v1.4}
%   \item
%     Code is not changed.
%   \item
%     New DTX framework.
%   \item
%     LPPL 1.3
%   \end{Version}
%   \begin{Version}{2008/08/11 v1.5}
%   \item
%     Code is not changed.
%   \item
%     URLs updated from \texttt{www.dejanews.com}
%     to \texttt{groups.google.com}.
%   \end{Version}
%   \begin{Version}{2016/05/16 v1.6}
%   \item
%     Documentation updates.
%   \end{Version}
% \end{History}
%
% \PrintIndex
%
% \Finale
\endinput
|
% \end{quote}
% Do not forget to quote the argument according to the demands
% of your shell.
%
% \paragraph{Generating the documentation.}
% You can use both the \xfile{.dtx} or the \xfile{.drv} to generate
% the documentation. The process can be configured by the
% configuration file \xfile{ltxdoc.cfg}. For instance, put this
% line into this file, if you want to have A4 as paper format:
% \begin{quote}
%   \verb|\PassOptionsToClass{a4paper}{article}|
% \end{quote}
% An example follows how to generate the
% documentation with pdf\LaTeX:
% \begin{quote}
%\begin{verbatim}
%pdflatex twoopt.dtx
%makeindex -s gind.ist twoopt.idx
%pdflatex twoopt.dtx
%makeindex -s gind.ist twoopt.idx
%pdflatex twoopt.dtx
%\end{verbatim}
% \end{quote}
%
% \section{Catalogue}
%
% The following XML file can be used as source for the
% \href{http://mirror.ctan.org/help/Catalogue/catalogue.html}{\TeX\ Catalogue}.
% The elements \texttt{caption} and \texttt{description} are imported
% from the original XML file from the Catalogue.
% The name of the XML file in the Catalogue is \xfile{twoopt.xml}.
%    \begin{macrocode}
%<*catalogue>
<?xml version='1.0' encoding='us-ascii'?>
<!DOCTYPE entry SYSTEM 'catalogue.dtd'>
<entry datestamp='$Date$' modifier='$Author$' id='twoopt'>
  <name>twoopt</name>
  <caption>Definitions with two optional arguments.</caption>
  <authorref id='auth:oberdiek'/>
  <copyright owner='Heiko Oberdiek' year='1999,2006,2008'/>
  <license type='lppl1.3'/>
  <version number='1.6'/>
  <description>
    Variants of <tt>\newcommand</tt>, <tt>\renewcommand</tt> and
    <tt>\providecommand</tt> are provided.
    <p/>
    The package is part of the <xref refid='oberdiek'>oberdiek</xref>
    bundle.
  </description>
  <documentation details='Package documentation'
      href='ctan:/macros/latex/contrib/oberdiek/twoopt.pdf'/>
  <ctan file='true' path='/macros/latex/contrib/oberdiek/twoopt.dtx'/>
  <miktex location='oberdiek'/>
  <texlive location='oberdiek'/>
  <install path='/macros/latex/contrib/oberdiek/oberdiek.tds.zip'/>
</entry>
%</catalogue>
%    \end{macrocode}
%
% \begin{History}
%   \begin{Version}{1998/10/30 v1.0}
%   \item
%     The first version was built as a response to a question
%     of \NameEmail{Rebecca and Rowland}{rebecca@astrid.u-net.com},
%     published in the newsgroup
%     \href{news:comp.text.tex}{comp.text.tex}:\\
%     \URL{``Re: [Q] LaTeX command with two optional arguments?''}^^A
%     {http://groups.google.com/group/comp.text.tex/msg/0ab1afde7b172d37}
%   \end{Version}
%   \begin{Version}{1998/10/30 v1.1}
%   \item
%     Improvements added in response to
%     \NameEmail{Stefan Ulrich}{ulrich@cis.uni-muenchen.de}
%     in the same thread:\\
%     \URL{``Re: [Q] LaTeX command with two optional arguments?''}^^A
%     {http://groups.google.com/group/comp.text.tex/msg/b8d84d4336f302c4}
%   \end{Version}
%   \begin{Version}{1998/11/04 v1.2}
%   \item
%     Fixes for LaTeX bugs 2896, 2901, 2902 added.
%   \end{Version}
%   \begin{Version}{1999/04/12 v1.3}
%   \item
%     Fixes removed because of LaTeX [1998/12/01].
%   \item
%     Documentation in dtx format.
%   \item
%     Copyright: LPPL (\CTAN{macros/latex/base/lppl.txt})
%   \item
%     First CTAN release.
%   \end{Version}
%   \begin{Version}{2006/02/20 v1.4}
%   \item
%     Code is not changed.
%   \item
%     New DTX framework.
%   \item
%     LPPL 1.3
%   \end{Version}
%   \begin{Version}{2008/08/11 v1.5}
%   \item
%     Code is not changed.
%   \item
%     URLs updated from \texttt{www.dejanews.com}
%     to \texttt{groups.google.com}.
%   \end{Version}
%   \begin{Version}{2016/05/16 v1.6}
%   \item
%     Documentation updates.
%   \end{Version}
% \end{History}
%
% \PrintIndex
%
% \Finale
\endinput

%        (quote the arguments according to the demands of your shell)
%
% Documentation:
%    (a) If twoopt.drv is present:
%           latex twoopt.drv
%    (b) Without twoopt.drv:
%           latex twoopt.dtx; ...
%    The class ltxdoc loads the configuration file ltxdoc.cfg
%    if available. Here you can specify further options, e.g.
%    use A4 as paper format:
%       \PassOptionsToClass{a4paper}{article}
%
%    Programm calls to get the documentation (example):
%       pdflatex twoopt.dtx
%       makeindex -s gind.ist twoopt.idx
%       pdflatex twoopt.dtx
%       makeindex -s gind.ist twoopt.idx
%       pdflatex twoopt.dtx
%
% Installation:
%    TDS:tex/latex/oberdiek/twoopt.sty
%    TDS:doc/latex/oberdiek/twoopt.pdf
%    TDS:source/latex/oberdiek/twoopt.dtx
%
%<*ignore>
\begingroup
  \catcode123=1 %
  \catcode125=2 %
  \def\x{LaTeX2e}%
\expandafter\endgroup
\ifcase 0\ifx\install y1\fi\expandafter
         \ifx\csname processbatchFile\endcsname\relax\else1\fi
         \ifx\fmtname\x\else 1\fi\relax
\else\csname fi\endcsname
%</ignore>
%<*install>
\input docstrip.tex
\Msg{************************************************************************}
\Msg{* Installation}
\Msg{* Package: twoopt 2016/05/16 v1.6 Definitions with two optional arguments (HO)}
\Msg{************************************************************************}

\keepsilent
\askforoverwritefalse

\let\MetaPrefix\relax
\preamble

This is a generated file.

Project: twoopt
Version: 2016/05/16 v1.6

Copyright (C) 1999, 2006, 2008 by
   Heiko Oberdiek <heiko.oberdiek at googlemail.com>

This work may be distributed and/or modified under the
conditions of the LaTeX Project Public License, either
version 1.3c of this license or (at your option) any later
version. This version of this license is in
   http://www.latex-project.org/lppl/lppl-1-3c.txt
and the latest version of this license is in
   http://www.latex-project.org/lppl.txt
and version 1.3 or later is part of all distributions of
LaTeX version 2005/12/01 or later.

This work has the LPPL maintenance status "maintained".

This Current Maintainer of this work is Heiko Oberdiek.

This work consists of the main source file twoopt.dtx
and the derived files
   twoopt.sty, twoopt.pdf, twoopt.ins, twoopt.drv.

\endpreamble
\let\MetaPrefix\DoubleperCent

\generate{%
  \file{twoopt.ins}{\from{twoopt.dtx}{install}}%
  \file{twoopt.drv}{\from{twoopt.dtx}{driver}}%
  \usedir{tex/latex/oberdiek}%
  \file{twoopt.sty}{\from{twoopt.dtx}{package}}%
  \nopreamble
  \nopostamble
%  \usedir{source/latex/oberdiek/catalogue}%
%  \file{twoopt.xml}{\from{twoopt.dtx}{catalogue}}%
}

\catcode32=13\relax% active space
\let =\space%
\Msg{************************************************************************}
\Msg{*}
\Msg{* To finish the installation you have to move the following}
\Msg{* file into a directory searched by TeX:}
\Msg{*}
\Msg{*     twoopt.sty}
\Msg{*}
\Msg{* To produce the documentation run the file `twoopt.drv'}
\Msg{* through LaTeX.}
\Msg{*}
\Msg{* Happy TeXing!}
\Msg{*}
\Msg{************************************************************************}

\endbatchfile
%</install>
%<*ignore>
\fi
%</ignore>
%<*driver>
\NeedsTeXFormat{LaTeX2e}
\ProvidesFile{twoopt.drv}%
  [2016/05/16 v1.6 Definitions with two optional arguments (HO)]%
\documentclass{ltxdoc}
\usepackage{holtxdoc}[2011/11/22]
\begin{document}
  \DocInput{twoopt.dtx}%
\end{document}
%</driver>
% \fi
%
%
% \CharacterTable
%  {Upper-case    \A\B\C\D\E\F\G\H\I\J\K\L\M\N\O\P\Q\R\S\T\U\V\W\X\Y\Z
%   Lower-case    \a\b\c\d\e\f\g\h\i\j\k\l\m\n\o\p\q\r\s\t\u\v\w\x\y\z
%   Digits        \0\1\2\3\4\5\6\7\8\9
%   Exclamation   \!     Double quote  \"     Hash (number) \#
%   Dollar        \$     Percent       \%     Ampersand     \&
%   Acute accent  \'     Left paren    \(     Right paren   \)
%   Asterisk      \*     Plus          \+     Comma         \,
%   Minus         \-     Point         \.     Solidus       \/
%   Colon         \:     Semicolon     \;     Less than     \<
%   Equals        \=     Greater than  \>     Question mark \?
%   Commercial at \@     Left bracket  \[     Backslash     \\
%   Right bracket \]     Circumflex    \^     Underscore    \_
%   Grave accent  \`     Left brace    \{     Vertical bar  \|
%   Right brace   \}     Tilde         \~}
%
% \GetFileInfo{twoopt.drv}
%
% \title{The \xpackage{twoopt} package}
% \date{2016/05/16 v1.6}
% \author{Heiko Oberdiek\thanks
% {Please report any issues at https://github.com/ho-tex/oberdiek/issues}\\
% \xemail{heiko.oberdiek at googlemail.com}}
%
% \maketitle
%
% \begin{abstract}
% This package provides commands to define macros with two
% optional arguments.
% \end{abstract}
%
% \tableofcontents
%
% \newenvironment{param}{^^A
%   \newcommand{\entry}[1]{\meta{\###1}:&}^^A
%   \begin{tabular}[t]{@{}l@{ }l@{}}^^A
% }{^^A
%   \end{tabular}^^A
% }
%
% \section{Usage}
%    \DescribeMacro{\newcommandtwoopt}
%    \DescribeMacro{\renewcommandtwoopt}
%    \DescribeMacro{\providecommandtwoopt}
%    Similar to \cmd{\newcommand}, \cmd{\renewcommand}
%    and \cmd{\providecommand} this package provides commands
%    to define macros with two optional arguments.
%    The names of the commands are built by appending the
%    package name to the \LaTeX-pendants:
%    \begingroup
%      \def\x{\marg{cmd} \oarg{num} \oarg{default1}^^A
%             \oarg{default2} \marg{def.}}^^A
%      \begin{tabbing}
%        \cmd{\providecommandtwoopt} \=\kill
%        \cmd{\newcommandtwoopt}\>\x\\
%        \cmd{\renewcommandtwoopt}\>\x\\
%        \cmd{\providecommandtwoopt}\>\x\\
%      \end{tabbing}
%    \endgroup
%
%    Also the |*|-forms are supported. Indeed it is better to
%    use this ones, unless it is intended to hold
%    whole paragraphs in some of the arguments. If the macro
%    is defined with the |*|-form, missing braces
%    can be detected earlier.
%
%    Example:
%    \begin{quote}
%      |\newcommandtwoopt{\bsp}[3][AA][BB]{%|\\
%      |  \typeout{\string\bsp: #1,#2,#3}%|\\
%      |}|\\
%      \begin{tabular}{@{}l@{\quad$\rightarrow$\quad}l@{}}
%      |\bsp[aa][bb]{cc}|&|\bsp: aa,bb,cc|\\
%      |\bsp[aa]{cc}|&|\bsp: aa,BB,cc|\\
%      |\bsp{cc}|&|\bsp: AA,BB,cc|\\
%      \end{tabular}
%    \end{quote}
%
% \StopEventually{
% }
%
% \section{Implementation}
%    \begin{macrocode}
%<*package>
\NeedsTeXFormat{LaTeX2e}
\ProvidesPackage{twoopt}
  [2016/05/16 v1.6 Definitions with two optional arguments (HO)]%
%    \end{macrocode}
%    \begin{macro}{\newcommandtwoopt}
%    \begin{macrocode}
\newcommand{\newcommandtwoopt}{%
  \@ifstar{\@newcommandtwoopt*}{\@newcommandtwoopt{}}%
}
%    \end{macrocode}
%    \end{macro}
%
%    \begin{macro}{\@newcommandtwoopt}
%    \begin{param}
%      \entry1 star\\
%      \entry2 macro name to be defined
%    \end{param}
%    \begin{macrocode}
\newcommand{\@newcommandtwoopt}{}
\long\def\@newcommandtwoopt#1#2{%
  \expandafter\@@newcommandtwoopt
    \csname2\string#2\endcsname{#1}{#2}%
}
%    \end{macrocode}
%    \end{macro}
%
%    \begin{macro}{\@@newcommandtwoopt}
%    \begin{param}
%      \entry1 help command to be defined
%        (\expandafter\cmd\csname 2\bslash<name>\endcsname)\\
%      \entry2 star\\
%      \entry3 macro name to be defined\\
%      \entry4 number of total arguments\\
%      \entry5 default for optional argument one\\
%      \entry6 default for optional argument two
%    \end{param}
%    \begin{macrocode}
\newcommand{\@@newcommandtwoopt}{}
\long\def\@@newcommandtwoopt#1#2#3[#4][#5][#6]{%
  \newcommand#2#3[1][{#5}]{%
    \to@ScanSecondOptArg#1{##1}{#6}%
  }%
  \newcommand#2#1[{#4}]%
}
%    \end{macrocode}
%    \end{macro}
%
%    \begin{macro}{\renewcommandtwoopt}
%    \begin{macrocode}
\newcommand{\renewcommandtwoopt}{%
  \@ifstar{\@renewcommandtwoopt*}{\@renewcommandtwoopt{}}%
}
%    \end{macrocode}
%    \end{macro}
%
%    \begin{macro}{\@renewcommandtwoopt}
%    \begin{param}
%      \entry1 star\\
%      \entry2 command name to be defined
%    \end{param}
%    \begin{macrocode}
\newcommand{\@renewcommandtwoopt}{}
\long\def\@renewcommandtwoopt#1#2{%
  \begingroup
    \escapechar\m@ne
    \xdef\@gtempa{{\string#2}}%
  \endgroup
  \expandafter\@ifundefined\@gtempa{%
    \@latex@error{\noexpand#2undefined}\@ehc
  }{}%
  \let#2\@undefined
  \expandafter\let\csname2\string#2\endcsname\@undefined
  \expandafter\@@newcommandtwoopt
    \csname2\string#2\endcsname{#1}{#2}%
}
%    \end{macrocode}
%    \end{macro}
%
%    \begin{macro}{\providecommandtwoopt}
%    \begin{macrocode}
\newcommand{\providecommandtwoopt}{%
  \@ifstar{\@providecommandtwoopt*}{\@providecommandtwoopt{}}%
}
%    \end{macrocode}
%    \end{macro}
%
%    \begin{macro}{\@providecommandtwoopt}
%    \begin{param}
%      \entry1 star\\
%      \entry2 command name to be defined
%    \end{param}
%    \begin{macrocode}
\newcommand{\@providecommandtwoopt}{}
\long\def\@providecommandtwoopt#1#2{%
  \begingroup
    \escapechar\m@ne
    \xdef\@gtempa{{\string#2}}%
  \endgroup
  \expandafter\@ifundefined\@gtempa{%
    \expandafter\@@newcommandtwoopt
      \csname2\string#2\endcsname{#1}{#2}%
  }{%
    \let\to@dummyA\@undefined
    \let\to@dummyB\@undefined
    \@@newcommandtwoopt\to@dummyA{#1}\to@dummyB
  }%
}
%    \end{macrocode}
%    \end{macro}
%
%    \begin{macro}{\to@ScanSecondOptArg}
%    \begin{param}
%      \entry1 help command to be defined
%        (\expandafter\cmd\csname 2\bslash<name>\endcsname)\\
%      \entry2 first arg of command to be defined\\
%      \entry3 default for second opt. arg.
%    \end{param}
%    \begin{macrocode}
\newcommand{\to@ScanSecondOptArg}[3]{%
  \@ifnextchar[{%
    \expandafter#1\to@ArgOptToArgArg{#2}%
  }{%
    #1{#2}{#3}%
  }%
}
%    \end{macrocode}
%    \end{macro}
%
%    \begin{macro}{\to@ArgOptToArgArg}
%    \begin{macrocode}
\newcommand{\to@ArgOptToArgArg}{}
\long\def\to@ArgOptToArgArg#1[#2]{{#1}{#2}}
%    \end{macrocode}
%    \end{macro}
%
%    \begin{macrocode}
%</package>
%    \end{macrocode}
%
% \section{Installation}
%
% \subsection{Download}
%
% \paragraph{Package.} This package is available on
% CTAN\footnote{\url{http://ctan.org/pkg/twoopt}}:
% \begin{description}
% \item[\CTAN{macros/latex/contrib/oberdiek/twoopt.dtx}] The source file.
% \item[\CTAN{macros/latex/contrib/oberdiek/twoopt.pdf}] Documentation.
% \end{description}
%
%
% \paragraph{Bundle.} All the packages of the bundle `oberdiek'
% are also available in a TDS compliant ZIP archive. There
% the packages are already unpacked and the documentation files
% are generated. The files and directories obey the TDS standard.
% \begin{description}
% \item[\CTAN{install/macros/latex/contrib/oberdiek.tds.zip}]
% \end{description}
% \emph{TDS} refers to the standard ``A Directory Structure
% for \TeX\ Files'' (\CTAN{tds/tds.pdf}). Directories
% with \xfile{texmf} in their name are usually organized this way.
%
% \subsection{Bundle installation}
%
% \paragraph{Unpacking.} Unpack the \xfile{oberdiek.tds.zip} in the
% TDS tree (also known as \xfile{texmf} tree) of your choice.
% Example (linux):
% \begin{quote}
%   |unzip oberdiek.tds.zip -d ~/texmf|
% \end{quote}
%
% \paragraph{Script installation.}
% Check the directory \xfile{TDS:scripts/oberdiek/} for
% scripts that need further installation steps.
% Package \xpackage{attachfile2} comes with the Perl script
% \xfile{pdfatfi.pl} that should be installed in such a way
% that it can be called as \texttt{pdfatfi}.
% Example (linux):
% \begin{quote}
%   |chmod +x scripts/oberdiek/pdfatfi.pl|\\
%   |cp scripts/oberdiek/pdfatfi.pl /usr/local/bin/|
% \end{quote}
%
% \subsection{Package installation}
%
% \paragraph{Unpacking.} The \xfile{.dtx} file is a self-extracting
% \docstrip\ archive. The files are extracted by running the
% \xfile{.dtx} through \plainTeX:
% \begin{quote}
%   \verb|tex twoopt.dtx|
% \end{quote}
%
% \paragraph{TDS.} Now the different files must be moved into
% the different directories in your installation TDS tree
% (also known as \xfile{texmf} tree):
% \begin{quote}
% \def\t{^^A
% \begin{tabular}{@{}>{\ttfamily}l@{ $\rightarrow$ }>{\ttfamily}l@{}}
%   twoopt.sty & tex/latex/oberdiek/twoopt.sty\\
%   twoopt.pdf & doc/latex/oberdiek/twoopt.pdf\\
%   twoopt.dtx & source/latex/oberdiek/twoopt.dtx\\
% \end{tabular}^^A
% }^^A
% \sbox0{\t}^^A
% \ifdim\wd0>\linewidth
%   \begingroup
%     \advance\linewidth by\leftmargin
%     \advance\linewidth by\rightmargin
%   \edef\x{\endgroup
%     \def\noexpand\lw{\the\linewidth}^^A
%   }\x
%   \def\lwbox{^^A
%     \leavevmode
%     \hbox to \linewidth{^^A
%       \kern-\leftmargin\relax
%       \hss
%       \usebox0
%       \hss
%       \kern-\rightmargin\relax
%     }^^A
%   }^^A
%   \ifdim\wd0>\lw
%     \sbox0{\small\t}^^A
%     \ifdim\wd0>\linewidth
%       \ifdim\wd0>\lw
%         \sbox0{\footnotesize\t}^^A
%         \ifdim\wd0>\linewidth
%           \ifdim\wd0>\lw
%             \sbox0{\scriptsize\t}^^A
%             \ifdim\wd0>\linewidth
%               \ifdim\wd0>\lw
%                 \sbox0{\tiny\t}^^A
%                 \ifdim\wd0>\linewidth
%                   \lwbox
%                 \else
%                   \usebox0
%                 \fi
%               \else
%                 \lwbox
%               \fi
%             \else
%               \usebox0
%             \fi
%           \else
%             \lwbox
%           \fi
%         \else
%           \usebox0
%         \fi
%       \else
%         \lwbox
%       \fi
%     \else
%       \usebox0
%     \fi
%   \else
%     \lwbox
%   \fi
% \else
%   \usebox0
% \fi
% \end{quote}
% If you have a \xfile{docstrip.cfg} that configures and enables \docstrip's
% TDS installing feature, then some files can already be in the right
% place, see the documentation of \docstrip.
%
% \subsection{Refresh file name databases}
%
% If your \TeX~distribution
% (\teTeX, \mikTeX, \dots) relies on file name databases, you must refresh
% these. For example, \teTeX\ users run \verb|texhash| or
% \verb|mktexlsr|.
%
% \subsection{Some details for the interested}
%
% \paragraph{Attached source.}
%
% The PDF documentation on CTAN also includes the
% \xfile{.dtx} source file. It can be extracted by
% AcrobatReader 6 or higher. Another option is \textsf{pdftk},
% e.g. unpack the file into the current directory:
% \begin{quote}
%   \verb|pdftk twoopt.pdf unpack_files output .|
% \end{quote}
%
% \paragraph{Unpacking with \LaTeX.}
% The \xfile{.dtx} chooses its action depending on the format:
% \begin{description}
% \item[\plainTeX:] Run \docstrip\ and extract the files.
% \item[\LaTeX:] Generate the documentation.
% \end{description}
% If you insist on using \LaTeX\ for \docstrip\ (really,
% \docstrip\ does not need \LaTeX), then inform the autodetect routine
% about your intention:
% \begin{quote}
%   \verb|latex \let\install=y% \iffalse meta-comment
%
% File: twoopt.dtx
% Version: 2016/05/16 v1.6
% Info: Definitions with two optional arguments
%
% Copyright (C) 1999, 2006, 2008 by
%    Heiko Oberdiek <heiko.oberdiek at googlemail.com>
%    2016
%    https://github.com/ho-tex/oberdiek/issues
%
% This work may be distributed and/or modified under the
% conditions of the LaTeX Project Public License, either
% version 1.3c of this license or (at your option) any later
% version. This version of this license is in
%    http://www.latex-project.org/lppl/lppl-1-3c.txt
% and the latest version of this license is in
%    http://www.latex-project.org/lppl.txt
% and version 1.3 or later is part of all distributions of
% LaTeX version 2005/12/01 or later.
%
% This work has the LPPL maintenance status "maintained".
%
% This Current Maintainer of this work is Heiko Oberdiek.
%
% This work consists of the main source file twoopt.dtx
% and the derived files
%    twoopt.sty, twoopt.pdf, twoopt.ins, twoopt.drv.
%
% Distribution:
%    CTAN:macros/latex/contrib/oberdiek/twoopt.dtx
%    CTAN:macros/latex/contrib/oberdiek/twoopt.pdf
%
% Unpacking:
%    (a) If twoopt.ins is present:
%           tex twoopt.ins
%    (b) Without twoopt.ins:
%           tex twoopt.dtx
%    (c) If you insist on using LaTeX
%           latex \let\install=y% \iffalse meta-comment
%
% File: twoopt.dtx
% Version: 2016/05/16 v1.6
% Info: Definitions with two optional arguments
%
% Copyright (C) 1999, 2006, 2008 by
%    Heiko Oberdiek <heiko.oberdiek at googlemail.com>
%    2016
%    https://github.com/ho-tex/oberdiek/issues
%
% This work may be distributed and/or modified under the
% conditions of the LaTeX Project Public License, either
% version 1.3c of this license or (at your option) any later
% version. This version of this license is in
%    http://www.latex-project.org/lppl/lppl-1-3c.txt
% and the latest version of this license is in
%    http://www.latex-project.org/lppl.txt
% and version 1.3 or later is part of all distributions of
% LaTeX version 2005/12/01 or later.
%
% This work has the LPPL maintenance status "maintained".
%
% This Current Maintainer of this work is Heiko Oberdiek.
%
% This work consists of the main source file twoopt.dtx
% and the derived files
%    twoopt.sty, twoopt.pdf, twoopt.ins, twoopt.drv.
%
% Distribution:
%    CTAN:macros/latex/contrib/oberdiek/twoopt.dtx
%    CTAN:macros/latex/contrib/oberdiek/twoopt.pdf
%
% Unpacking:
%    (a) If twoopt.ins is present:
%           tex twoopt.ins
%    (b) Without twoopt.ins:
%           tex twoopt.dtx
%    (c) If you insist on using LaTeX
%           latex \let\install=y\input{twoopt.dtx}
%        (quote the arguments according to the demands of your shell)
%
% Documentation:
%    (a) If twoopt.drv is present:
%           latex twoopt.drv
%    (b) Without twoopt.drv:
%           latex twoopt.dtx; ...
%    The class ltxdoc loads the configuration file ltxdoc.cfg
%    if available. Here you can specify further options, e.g.
%    use A4 as paper format:
%       \PassOptionsToClass{a4paper}{article}
%
%    Programm calls to get the documentation (example):
%       pdflatex twoopt.dtx
%       makeindex -s gind.ist twoopt.idx
%       pdflatex twoopt.dtx
%       makeindex -s gind.ist twoopt.idx
%       pdflatex twoopt.dtx
%
% Installation:
%    TDS:tex/latex/oberdiek/twoopt.sty
%    TDS:doc/latex/oberdiek/twoopt.pdf
%    TDS:source/latex/oberdiek/twoopt.dtx
%
%<*ignore>
\begingroup
  \catcode123=1 %
  \catcode125=2 %
  \def\x{LaTeX2e}%
\expandafter\endgroup
\ifcase 0\ifx\install y1\fi\expandafter
         \ifx\csname processbatchFile\endcsname\relax\else1\fi
         \ifx\fmtname\x\else 1\fi\relax
\else\csname fi\endcsname
%</ignore>
%<*install>
\input docstrip.tex
\Msg{************************************************************************}
\Msg{* Installation}
\Msg{* Package: twoopt 2016/05/16 v1.6 Definitions with two optional arguments (HO)}
\Msg{************************************************************************}

\keepsilent
\askforoverwritefalse

\let\MetaPrefix\relax
\preamble

This is a generated file.

Project: twoopt
Version: 2016/05/16 v1.6

Copyright (C) 1999, 2006, 2008 by
   Heiko Oberdiek <heiko.oberdiek at googlemail.com>

This work may be distributed and/or modified under the
conditions of the LaTeX Project Public License, either
version 1.3c of this license or (at your option) any later
version. This version of this license is in
   http://www.latex-project.org/lppl/lppl-1-3c.txt
and the latest version of this license is in
   http://www.latex-project.org/lppl.txt
and version 1.3 or later is part of all distributions of
LaTeX version 2005/12/01 or later.

This work has the LPPL maintenance status "maintained".

This Current Maintainer of this work is Heiko Oberdiek.

This work consists of the main source file twoopt.dtx
and the derived files
   twoopt.sty, twoopt.pdf, twoopt.ins, twoopt.drv.

\endpreamble
\let\MetaPrefix\DoubleperCent

\generate{%
  \file{twoopt.ins}{\from{twoopt.dtx}{install}}%
  \file{twoopt.drv}{\from{twoopt.dtx}{driver}}%
  \usedir{tex/latex/oberdiek}%
  \file{twoopt.sty}{\from{twoopt.dtx}{package}}%
  \nopreamble
  \nopostamble
%  \usedir{source/latex/oberdiek/catalogue}%
%  \file{twoopt.xml}{\from{twoopt.dtx}{catalogue}}%
}

\catcode32=13\relax% active space
\let =\space%
\Msg{************************************************************************}
\Msg{*}
\Msg{* To finish the installation you have to move the following}
\Msg{* file into a directory searched by TeX:}
\Msg{*}
\Msg{*     twoopt.sty}
\Msg{*}
\Msg{* To produce the documentation run the file `twoopt.drv'}
\Msg{* through LaTeX.}
\Msg{*}
\Msg{* Happy TeXing!}
\Msg{*}
\Msg{************************************************************************}

\endbatchfile
%</install>
%<*ignore>
\fi
%</ignore>
%<*driver>
\NeedsTeXFormat{LaTeX2e}
\ProvidesFile{twoopt.drv}%
  [2016/05/16 v1.6 Definitions with two optional arguments (HO)]%
\documentclass{ltxdoc}
\usepackage{holtxdoc}[2011/11/22]
\begin{document}
  \DocInput{twoopt.dtx}%
\end{document}
%</driver>
% \fi
%
%
% \CharacterTable
%  {Upper-case    \A\B\C\D\E\F\G\H\I\J\K\L\M\N\O\P\Q\R\S\T\U\V\W\X\Y\Z
%   Lower-case    \a\b\c\d\e\f\g\h\i\j\k\l\m\n\o\p\q\r\s\t\u\v\w\x\y\z
%   Digits        \0\1\2\3\4\5\6\7\8\9
%   Exclamation   \!     Double quote  \"     Hash (number) \#
%   Dollar        \$     Percent       \%     Ampersand     \&
%   Acute accent  \'     Left paren    \(     Right paren   \)
%   Asterisk      \*     Plus          \+     Comma         \,
%   Minus         \-     Point         \.     Solidus       \/
%   Colon         \:     Semicolon     \;     Less than     \<
%   Equals        \=     Greater than  \>     Question mark \?
%   Commercial at \@     Left bracket  \[     Backslash     \\
%   Right bracket \]     Circumflex    \^     Underscore    \_
%   Grave accent  \`     Left brace    \{     Vertical bar  \|
%   Right brace   \}     Tilde         \~}
%
% \GetFileInfo{twoopt.drv}
%
% \title{The \xpackage{twoopt} package}
% \date{2016/05/16 v1.6}
% \author{Heiko Oberdiek\thanks
% {Please report any issues at https://github.com/ho-tex/oberdiek/issues}\\
% \xemail{heiko.oberdiek at googlemail.com}}
%
% \maketitle
%
% \begin{abstract}
% This package provides commands to define macros with two
% optional arguments.
% \end{abstract}
%
% \tableofcontents
%
% \newenvironment{param}{^^A
%   \newcommand{\entry}[1]{\meta{\###1}:&}^^A
%   \begin{tabular}[t]{@{}l@{ }l@{}}^^A
% }{^^A
%   \end{tabular}^^A
% }
%
% \section{Usage}
%    \DescribeMacro{\newcommandtwoopt}
%    \DescribeMacro{\renewcommandtwoopt}
%    \DescribeMacro{\providecommandtwoopt}
%    Similar to \cmd{\newcommand}, \cmd{\renewcommand}
%    and \cmd{\providecommand} this package provides commands
%    to define macros with two optional arguments.
%    The names of the commands are built by appending the
%    package name to the \LaTeX-pendants:
%    \begingroup
%      \def\x{\marg{cmd} \oarg{num} \oarg{default1}^^A
%             \oarg{default2} \marg{def.}}^^A
%      \begin{tabbing}
%        \cmd{\providecommandtwoopt} \=\kill
%        \cmd{\newcommandtwoopt}\>\x\\
%        \cmd{\renewcommandtwoopt}\>\x\\
%        \cmd{\providecommandtwoopt}\>\x\\
%      \end{tabbing}
%    \endgroup
%
%    Also the |*|-forms are supported. Indeed it is better to
%    use this ones, unless it is intended to hold
%    whole paragraphs in some of the arguments. If the macro
%    is defined with the |*|-form, missing braces
%    can be detected earlier.
%
%    Example:
%    \begin{quote}
%      |\newcommandtwoopt{\bsp}[3][AA][BB]{%|\\
%      |  \typeout{\string\bsp: #1,#2,#3}%|\\
%      |}|\\
%      \begin{tabular}{@{}l@{\quad$\rightarrow$\quad}l@{}}
%      |\bsp[aa][bb]{cc}|&|\bsp: aa,bb,cc|\\
%      |\bsp[aa]{cc}|&|\bsp: aa,BB,cc|\\
%      |\bsp{cc}|&|\bsp: AA,BB,cc|\\
%      \end{tabular}
%    \end{quote}
%
% \StopEventually{
% }
%
% \section{Implementation}
%    \begin{macrocode}
%<*package>
\NeedsTeXFormat{LaTeX2e}
\ProvidesPackage{twoopt}
  [2016/05/16 v1.6 Definitions with two optional arguments (HO)]%
%    \end{macrocode}
%    \begin{macro}{\newcommandtwoopt}
%    \begin{macrocode}
\newcommand{\newcommandtwoopt}{%
  \@ifstar{\@newcommandtwoopt*}{\@newcommandtwoopt{}}%
}
%    \end{macrocode}
%    \end{macro}
%
%    \begin{macro}{\@newcommandtwoopt}
%    \begin{param}
%      \entry1 star\\
%      \entry2 macro name to be defined
%    \end{param}
%    \begin{macrocode}
\newcommand{\@newcommandtwoopt}{}
\long\def\@newcommandtwoopt#1#2{%
  \expandafter\@@newcommandtwoopt
    \csname2\string#2\endcsname{#1}{#2}%
}
%    \end{macrocode}
%    \end{macro}
%
%    \begin{macro}{\@@newcommandtwoopt}
%    \begin{param}
%      \entry1 help command to be defined
%        (\expandafter\cmd\csname 2\bslash<name>\endcsname)\\
%      \entry2 star\\
%      \entry3 macro name to be defined\\
%      \entry4 number of total arguments\\
%      \entry5 default for optional argument one\\
%      \entry6 default for optional argument two
%    \end{param}
%    \begin{macrocode}
\newcommand{\@@newcommandtwoopt}{}
\long\def\@@newcommandtwoopt#1#2#3[#4][#5][#6]{%
  \newcommand#2#3[1][{#5}]{%
    \to@ScanSecondOptArg#1{##1}{#6}%
  }%
  \newcommand#2#1[{#4}]%
}
%    \end{macrocode}
%    \end{macro}
%
%    \begin{macro}{\renewcommandtwoopt}
%    \begin{macrocode}
\newcommand{\renewcommandtwoopt}{%
  \@ifstar{\@renewcommandtwoopt*}{\@renewcommandtwoopt{}}%
}
%    \end{macrocode}
%    \end{macro}
%
%    \begin{macro}{\@renewcommandtwoopt}
%    \begin{param}
%      \entry1 star\\
%      \entry2 command name to be defined
%    \end{param}
%    \begin{macrocode}
\newcommand{\@renewcommandtwoopt}{}
\long\def\@renewcommandtwoopt#1#2{%
  \begingroup
    \escapechar\m@ne
    \xdef\@gtempa{{\string#2}}%
  \endgroup
  \expandafter\@ifundefined\@gtempa{%
    \@latex@error{\noexpand#2undefined}\@ehc
  }{}%
  \let#2\@undefined
  \expandafter\let\csname2\string#2\endcsname\@undefined
  \expandafter\@@newcommandtwoopt
    \csname2\string#2\endcsname{#1}{#2}%
}
%    \end{macrocode}
%    \end{macro}
%
%    \begin{macro}{\providecommandtwoopt}
%    \begin{macrocode}
\newcommand{\providecommandtwoopt}{%
  \@ifstar{\@providecommandtwoopt*}{\@providecommandtwoopt{}}%
}
%    \end{macrocode}
%    \end{macro}
%
%    \begin{macro}{\@providecommandtwoopt}
%    \begin{param}
%      \entry1 star\\
%      \entry2 command name to be defined
%    \end{param}
%    \begin{macrocode}
\newcommand{\@providecommandtwoopt}{}
\long\def\@providecommandtwoopt#1#2{%
  \begingroup
    \escapechar\m@ne
    \xdef\@gtempa{{\string#2}}%
  \endgroup
  \expandafter\@ifundefined\@gtempa{%
    \expandafter\@@newcommandtwoopt
      \csname2\string#2\endcsname{#1}{#2}%
  }{%
    \let\to@dummyA\@undefined
    \let\to@dummyB\@undefined
    \@@newcommandtwoopt\to@dummyA{#1}\to@dummyB
  }%
}
%    \end{macrocode}
%    \end{macro}
%
%    \begin{macro}{\to@ScanSecondOptArg}
%    \begin{param}
%      \entry1 help command to be defined
%        (\expandafter\cmd\csname 2\bslash<name>\endcsname)\\
%      \entry2 first arg of command to be defined\\
%      \entry3 default for second opt. arg.
%    \end{param}
%    \begin{macrocode}
\newcommand{\to@ScanSecondOptArg}[3]{%
  \@ifnextchar[{%
    \expandafter#1\to@ArgOptToArgArg{#2}%
  }{%
    #1{#2}{#3}%
  }%
}
%    \end{macrocode}
%    \end{macro}
%
%    \begin{macro}{\to@ArgOptToArgArg}
%    \begin{macrocode}
\newcommand{\to@ArgOptToArgArg}{}
\long\def\to@ArgOptToArgArg#1[#2]{{#1}{#2}}
%    \end{macrocode}
%    \end{macro}
%
%    \begin{macrocode}
%</package>
%    \end{macrocode}
%
% \section{Installation}
%
% \subsection{Download}
%
% \paragraph{Package.} This package is available on
% CTAN\footnote{\url{http://ctan.org/pkg/twoopt}}:
% \begin{description}
% \item[\CTAN{macros/latex/contrib/oberdiek/twoopt.dtx}] The source file.
% \item[\CTAN{macros/latex/contrib/oberdiek/twoopt.pdf}] Documentation.
% \end{description}
%
%
% \paragraph{Bundle.} All the packages of the bundle `oberdiek'
% are also available in a TDS compliant ZIP archive. There
% the packages are already unpacked and the documentation files
% are generated. The files and directories obey the TDS standard.
% \begin{description}
% \item[\CTAN{install/macros/latex/contrib/oberdiek.tds.zip}]
% \end{description}
% \emph{TDS} refers to the standard ``A Directory Structure
% for \TeX\ Files'' (\CTAN{tds/tds.pdf}). Directories
% with \xfile{texmf} in their name are usually organized this way.
%
% \subsection{Bundle installation}
%
% \paragraph{Unpacking.} Unpack the \xfile{oberdiek.tds.zip} in the
% TDS tree (also known as \xfile{texmf} tree) of your choice.
% Example (linux):
% \begin{quote}
%   |unzip oberdiek.tds.zip -d ~/texmf|
% \end{quote}
%
% \paragraph{Script installation.}
% Check the directory \xfile{TDS:scripts/oberdiek/} for
% scripts that need further installation steps.
% Package \xpackage{attachfile2} comes with the Perl script
% \xfile{pdfatfi.pl} that should be installed in such a way
% that it can be called as \texttt{pdfatfi}.
% Example (linux):
% \begin{quote}
%   |chmod +x scripts/oberdiek/pdfatfi.pl|\\
%   |cp scripts/oberdiek/pdfatfi.pl /usr/local/bin/|
% \end{quote}
%
% \subsection{Package installation}
%
% \paragraph{Unpacking.} The \xfile{.dtx} file is a self-extracting
% \docstrip\ archive. The files are extracted by running the
% \xfile{.dtx} through \plainTeX:
% \begin{quote}
%   \verb|tex twoopt.dtx|
% \end{quote}
%
% \paragraph{TDS.} Now the different files must be moved into
% the different directories in your installation TDS tree
% (also known as \xfile{texmf} tree):
% \begin{quote}
% \def\t{^^A
% \begin{tabular}{@{}>{\ttfamily}l@{ $\rightarrow$ }>{\ttfamily}l@{}}
%   twoopt.sty & tex/latex/oberdiek/twoopt.sty\\
%   twoopt.pdf & doc/latex/oberdiek/twoopt.pdf\\
%   twoopt.dtx & source/latex/oberdiek/twoopt.dtx\\
% \end{tabular}^^A
% }^^A
% \sbox0{\t}^^A
% \ifdim\wd0>\linewidth
%   \begingroup
%     \advance\linewidth by\leftmargin
%     \advance\linewidth by\rightmargin
%   \edef\x{\endgroup
%     \def\noexpand\lw{\the\linewidth}^^A
%   }\x
%   \def\lwbox{^^A
%     \leavevmode
%     \hbox to \linewidth{^^A
%       \kern-\leftmargin\relax
%       \hss
%       \usebox0
%       \hss
%       \kern-\rightmargin\relax
%     }^^A
%   }^^A
%   \ifdim\wd0>\lw
%     \sbox0{\small\t}^^A
%     \ifdim\wd0>\linewidth
%       \ifdim\wd0>\lw
%         \sbox0{\footnotesize\t}^^A
%         \ifdim\wd0>\linewidth
%           \ifdim\wd0>\lw
%             \sbox0{\scriptsize\t}^^A
%             \ifdim\wd0>\linewidth
%               \ifdim\wd0>\lw
%                 \sbox0{\tiny\t}^^A
%                 \ifdim\wd0>\linewidth
%                   \lwbox
%                 \else
%                   \usebox0
%                 \fi
%               \else
%                 \lwbox
%               \fi
%             \else
%               \usebox0
%             \fi
%           \else
%             \lwbox
%           \fi
%         \else
%           \usebox0
%         \fi
%       \else
%         \lwbox
%       \fi
%     \else
%       \usebox0
%     \fi
%   \else
%     \lwbox
%   \fi
% \else
%   \usebox0
% \fi
% \end{quote}
% If you have a \xfile{docstrip.cfg} that configures and enables \docstrip's
% TDS installing feature, then some files can already be in the right
% place, see the documentation of \docstrip.
%
% \subsection{Refresh file name databases}
%
% If your \TeX~distribution
% (\teTeX, \mikTeX, \dots) relies on file name databases, you must refresh
% these. For example, \teTeX\ users run \verb|texhash| or
% \verb|mktexlsr|.
%
% \subsection{Some details for the interested}
%
% \paragraph{Attached source.}
%
% The PDF documentation on CTAN also includes the
% \xfile{.dtx} source file. It can be extracted by
% AcrobatReader 6 or higher. Another option is \textsf{pdftk},
% e.g. unpack the file into the current directory:
% \begin{quote}
%   \verb|pdftk twoopt.pdf unpack_files output .|
% \end{quote}
%
% \paragraph{Unpacking with \LaTeX.}
% The \xfile{.dtx} chooses its action depending on the format:
% \begin{description}
% \item[\plainTeX:] Run \docstrip\ and extract the files.
% \item[\LaTeX:] Generate the documentation.
% \end{description}
% If you insist on using \LaTeX\ for \docstrip\ (really,
% \docstrip\ does not need \LaTeX), then inform the autodetect routine
% about your intention:
% \begin{quote}
%   \verb|latex \let\install=y\input{twoopt.dtx}|
% \end{quote}
% Do not forget to quote the argument according to the demands
% of your shell.
%
% \paragraph{Generating the documentation.}
% You can use both the \xfile{.dtx} or the \xfile{.drv} to generate
% the documentation. The process can be configured by the
% configuration file \xfile{ltxdoc.cfg}. For instance, put this
% line into this file, if you want to have A4 as paper format:
% \begin{quote}
%   \verb|\PassOptionsToClass{a4paper}{article}|
% \end{quote}
% An example follows how to generate the
% documentation with pdf\LaTeX:
% \begin{quote}
%\begin{verbatim}
%pdflatex twoopt.dtx
%makeindex -s gind.ist twoopt.idx
%pdflatex twoopt.dtx
%makeindex -s gind.ist twoopt.idx
%pdflatex twoopt.dtx
%\end{verbatim}
% \end{quote}
%
% \section{Catalogue}
%
% The following XML file can be used as source for the
% \href{http://mirror.ctan.org/help/Catalogue/catalogue.html}{\TeX\ Catalogue}.
% The elements \texttt{caption} and \texttt{description} are imported
% from the original XML file from the Catalogue.
% The name of the XML file in the Catalogue is \xfile{twoopt.xml}.
%    \begin{macrocode}
%<*catalogue>
<?xml version='1.0' encoding='us-ascii'?>
<!DOCTYPE entry SYSTEM 'catalogue.dtd'>
<entry datestamp='$Date$' modifier='$Author$' id='twoopt'>
  <name>twoopt</name>
  <caption>Definitions with two optional arguments.</caption>
  <authorref id='auth:oberdiek'/>
  <copyright owner='Heiko Oberdiek' year='1999,2006,2008'/>
  <license type='lppl1.3'/>
  <version number='1.6'/>
  <description>
    Variants of <tt>\newcommand</tt>, <tt>\renewcommand</tt> and
    <tt>\providecommand</tt> are provided.
    <p/>
    The package is part of the <xref refid='oberdiek'>oberdiek</xref>
    bundle.
  </description>
  <documentation details='Package documentation'
      href='ctan:/macros/latex/contrib/oberdiek/twoopt.pdf'/>
  <ctan file='true' path='/macros/latex/contrib/oberdiek/twoopt.dtx'/>
  <miktex location='oberdiek'/>
  <texlive location='oberdiek'/>
  <install path='/macros/latex/contrib/oberdiek/oberdiek.tds.zip'/>
</entry>
%</catalogue>
%    \end{macrocode}
%
% \begin{History}
%   \begin{Version}{1998/10/30 v1.0}
%   \item
%     The first version was built as a response to a question
%     of \NameEmail{Rebecca and Rowland}{rebecca@astrid.u-net.com},
%     published in the newsgroup
%     \href{news:comp.text.tex}{comp.text.tex}:\\
%     \URL{``Re: [Q] LaTeX command with two optional arguments?''}^^A
%     {http://groups.google.com/group/comp.text.tex/msg/0ab1afde7b172d37}
%   \end{Version}
%   \begin{Version}{1998/10/30 v1.1}
%   \item
%     Improvements added in response to
%     \NameEmail{Stefan Ulrich}{ulrich@cis.uni-muenchen.de}
%     in the same thread:\\
%     \URL{``Re: [Q] LaTeX command with two optional arguments?''}^^A
%     {http://groups.google.com/group/comp.text.tex/msg/b8d84d4336f302c4}
%   \end{Version}
%   \begin{Version}{1998/11/04 v1.2}
%   \item
%     Fixes for LaTeX bugs 2896, 2901, 2902 added.
%   \end{Version}
%   \begin{Version}{1999/04/12 v1.3}
%   \item
%     Fixes removed because of LaTeX [1998/12/01].
%   \item
%     Documentation in dtx format.
%   \item
%     Copyright: LPPL (\CTAN{macros/latex/base/lppl.txt})
%   \item
%     First CTAN release.
%   \end{Version}
%   \begin{Version}{2006/02/20 v1.4}
%   \item
%     Code is not changed.
%   \item
%     New DTX framework.
%   \item
%     LPPL 1.3
%   \end{Version}
%   \begin{Version}{2008/08/11 v1.5}
%   \item
%     Code is not changed.
%   \item
%     URLs updated from \texttt{www.dejanews.com}
%     to \texttt{groups.google.com}.
%   \end{Version}
%   \begin{Version}{2016/05/16 v1.6}
%   \item
%     Documentation updates.
%   \end{Version}
% \end{History}
%
% \PrintIndex
%
% \Finale
\endinput

%        (quote the arguments according to the demands of your shell)
%
% Documentation:
%    (a) If twoopt.drv is present:
%           latex twoopt.drv
%    (b) Without twoopt.drv:
%           latex twoopt.dtx; ...
%    The class ltxdoc loads the configuration file ltxdoc.cfg
%    if available. Here you can specify further options, e.g.
%    use A4 as paper format:
%       \PassOptionsToClass{a4paper}{article}
%
%    Programm calls to get the documentation (example):
%       pdflatex twoopt.dtx
%       makeindex -s gind.ist twoopt.idx
%       pdflatex twoopt.dtx
%       makeindex -s gind.ist twoopt.idx
%       pdflatex twoopt.dtx
%
% Installation:
%    TDS:tex/latex/oberdiek/twoopt.sty
%    TDS:doc/latex/oberdiek/twoopt.pdf
%    TDS:source/latex/oberdiek/twoopt.dtx
%
%<*ignore>
\begingroup
  \catcode123=1 %
  \catcode125=2 %
  \def\x{LaTeX2e}%
\expandafter\endgroup
\ifcase 0\ifx\install y1\fi\expandafter
         \ifx\csname processbatchFile\endcsname\relax\else1\fi
         \ifx\fmtname\x\else 1\fi\relax
\else\csname fi\endcsname
%</ignore>
%<*install>
\input docstrip.tex
\Msg{************************************************************************}
\Msg{* Installation}
\Msg{* Package: twoopt 2016/05/16 v1.6 Definitions with two optional arguments (HO)}
\Msg{************************************************************************}

\keepsilent
\askforoverwritefalse

\let\MetaPrefix\relax
\preamble

This is a generated file.

Project: twoopt
Version: 2016/05/16 v1.6

Copyright (C) 1999, 2006, 2008 by
   Heiko Oberdiek <heiko.oberdiek at googlemail.com>

This work may be distributed and/or modified under the
conditions of the LaTeX Project Public License, either
version 1.3c of this license or (at your option) any later
version. This version of this license is in
   http://www.latex-project.org/lppl/lppl-1-3c.txt
and the latest version of this license is in
   http://www.latex-project.org/lppl.txt
and version 1.3 or later is part of all distributions of
LaTeX version 2005/12/01 or later.

This work has the LPPL maintenance status "maintained".

This Current Maintainer of this work is Heiko Oberdiek.

This work consists of the main source file twoopt.dtx
and the derived files
   twoopt.sty, twoopt.pdf, twoopt.ins, twoopt.drv.

\endpreamble
\let\MetaPrefix\DoubleperCent

\generate{%
  \file{twoopt.ins}{\from{twoopt.dtx}{install}}%
  \file{twoopt.drv}{\from{twoopt.dtx}{driver}}%
  \usedir{tex/latex/oberdiek}%
  \file{twoopt.sty}{\from{twoopt.dtx}{package}}%
  \nopreamble
  \nopostamble
%  \usedir{source/latex/oberdiek/catalogue}%
%  \file{twoopt.xml}{\from{twoopt.dtx}{catalogue}}%
}

\catcode32=13\relax% active space
\let =\space%
\Msg{************************************************************************}
\Msg{*}
\Msg{* To finish the installation you have to move the following}
\Msg{* file into a directory searched by TeX:}
\Msg{*}
\Msg{*     twoopt.sty}
\Msg{*}
\Msg{* To produce the documentation run the file `twoopt.drv'}
\Msg{* through LaTeX.}
\Msg{*}
\Msg{* Happy TeXing!}
\Msg{*}
\Msg{************************************************************************}

\endbatchfile
%</install>
%<*ignore>
\fi
%</ignore>
%<*driver>
\NeedsTeXFormat{LaTeX2e}
\ProvidesFile{twoopt.drv}%
  [2016/05/16 v1.6 Definitions with two optional arguments (HO)]%
\documentclass{ltxdoc}
\usepackage{holtxdoc}[2011/11/22]
\begin{document}
  \DocInput{twoopt.dtx}%
\end{document}
%</driver>
% \fi
%
%
% \CharacterTable
%  {Upper-case    \A\B\C\D\E\F\G\H\I\J\K\L\M\N\O\P\Q\R\S\T\U\V\W\X\Y\Z
%   Lower-case    \a\b\c\d\e\f\g\h\i\j\k\l\m\n\o\p\q\r\s\t\u\v\w\x\y\z
%   Digits        \0\1\2\3\4\5\6\7\8\9
%   Exclamation   \!     Double quote  \"     Hash (number) \#
%   Dollar        \$     Percent       \%     Ampersand     \&
%   Acute accent  \'     Left paren    \(     Right paren   \)
%   Asterisk      \*     Plus          \+     Comma         \,
%   Minus         \-     Point         \.     Solidus       \/
%   Colon         \:     Semicolon     \;     Less than     \<
%   Equals        \=     Greater than  \>     Question mark \?
%   Commercial at \@     Left bracket  \[     Backslash     \\
%   Right bracket \]     Circumflex    \^     Underscore    \_
%   Grave accent  \`     Left brace    \{     Vertical bar  \|
%   Right brace   \}     Tilde         \~}
%
% \GetFileInfo{twoopt.drv}
%
% \title{The \xpackage{twoopt} package}
% \date{2016/05/16 v1.6}
% \author{Heiko Oberdiek\thanks
% {Please report any issues at https://github.com/ho-tex/oberdiek/issues}\\
% \xemail{heiko.oberdiek at googlemail.com}}
%
% \maketitle
%
% \begin{abstract}
% This package provides commands to define macros with two
% optional arguments.
% \end{abstract}
%
% \tableofcontents
%
% \newenvironment{param}{^^A
%   \newcommand{\entry}[1]{\meta{\###1}:&}^^A
%   \begin{tabular}[t]{@{}l@{ }l@{}}^^A
% }{^^A
%   \end{tabular}^^A
% }
%
% \section{Usage}
%    \DescribeMacro{\newcommandtwoopt}
%    \DescribeMacro{\renewcommandtwoopt}
%    \DescribeMacro{\providecommandtwoopt}
%    Similar to \cmd{\newcommand}, \cmd{\renewcommand}
%    and \cmd{\providecommand} this package provides commands
%    to define macros with two optional arguments.
%    The names of the commands are built by appending the
%    package name to the \LaTeX-pendants:
%    \begingroup
%      \def\x{\marg{cmd} \oarg{num} \oarg{default1}^^A
%             \oarg{default2} \marg{def.}}^^A
%      \begin{tabbing}
%        \cmd{\providecommandtwoopt} \=\kill
%        \cmd{\newcommandtwoopt}\>\x\\
%        \cmd{\renewcommandtwoopt}\>\x\\
%        \cmd{\providecommandtwoopt}\>\x\\
%      \end{tabbing}
%    \endgroup
%
%    Also the |*|-forms are supported. Indeed it is better to
%    use this ones, unless it is intended to hold
%    whole paragraphs in some of the arguments. If the macro
%    is defined with the |*|-form, missing braces
%    can be detected earlier.
%
%    Example:
%    \begin{quote}
%      |\newcommandtwoopt{\bsp}[3][AA][BB]{%|\\
%      |  \typeout{\string\bsp: #1,#2,#3}%|\\
%      |}|\\
%      \begin{tabular}{@{}l@{\quad$\rightarrow$\quad}l@{}}
%      |\bsp[aa][bb]{cc}|&|\bsp: aa,bb,cc|\\
%      |\bsp[aa]{cc}|&|\bsp: aa,BB,cc|\\
%      |\bsp{cc}|&|\bsp: AA,BB,cc|\\
%      \end{tabular}
%    \end{quote}
%
% \StopEventually{
% }
%
% \section{Implementation}
%    \begin{macrocode}
%<*package>
\NeedsTeXFormat{LaTeX2e}
\ProvidesPackage{twoopt}
  [2016/05/16 v1.6 Definitions with two optional arguments (HO)]%
%    \end{macrocode}
%    \begin{macro}{\newcommandtwoopt}
%    \begin{macrocode}
\newcommand{\newcommandtwoopt}{%
  \@ifstar{\@newcommandtwoopt*}{\@newcommandtwoopt{}}%
}
%    \end{macrocode}
%    \end{macro}
%
%    \begin{macro}{\@newcommandtwoopt}
%    \begin{param}
%      \entry1 star\\
%      \entry2 macro name to be defined
%    \end{param}
%    \begin{macrocode}
\newcommand{\@newcommandtwoopt}{}
\long\def\@newcommandtwoopt#1#2{%
  \expandafter\@@newcommandtwoopt
    \csname2\string#2\endcsname{#1}{#2}%
}
%    \end{macrocode}
%    \end{macro}
%
%    \begin{macro}{\@@newcommandtwoopt}
%    \begin{param}
%      \entry1 help command to be defined
%        (\expandafter\cmd\csname 2\bslash<name>\endcsname)\\
%      \entry2 star\\
%      \entry3 macro name to be defined\\
%      \entry4 number of total arguments\\
%      \entry5 default for optional argument one\\
%      \entry6 default for optional argument two
%    \end{param}
%    \begin{macrocode}
\newcommand{\@@newcommandtwoopt}{}
\long\def\@@newcommandtwoopt#1#2#3[#4][#5][#6]{%
  \newcommand#2#3[1][{#5}]{%
    \to@ScanSecondOptArg#1{##1}{#6}%
  }%
  \newcommand#2#1[{#4}]%
}
%    \end{macrocode}
%    \end{macro}
%
%    \begin{macro}{\renewcommandtwoopt}
%    \begin{macrocode}
\newcommand{\renewcommandtwoopt}{%
  \@ifstar{\@renewcommandtwoopt*}{\@renewcommandtwoopt{}}%
}
%    \end{macrocode}
%    \end{macro}
%
%    \begin{macro}{\@renewcommandtwoopt}
%    \begin{param}
%      \entry1 star\\
%      \entry2 command name to be defined
%    \end{param}
%    \begin{macrocode}
\newcommand{\@renewcommandtwoopt}{}
\long\def\@renewcommandtwoopt#1#2{%
  \begingroup
    \escapechar\m@ne
    \xdef\@gtempa{{\string#2}}%
  \endgroup
  \expandafter\@ifundefined\@gtempa{%
    \@latex@error{\noexpand#2undefined}\@ehc
  }{}%
  \let#2\@undefined
  \expandafter\let\csname2\string#2\endcsname\@undefined
  \expandafter\@@newcommandtwoopt
    \csname2\string#2\endcsname{#1}{#2}%
}
%    \end{macrocode}
%    \end{macro}
%
%    \begin{macro}{\providecommandtwoopt}
%    \begin{macrocode}
\newcommand{\providecommandtwoopt}{%
  \@ifstar{\@providecommandtwoopt*}{\@providecommandtwoopt{}}%
}
%    \end{macrocode}
%    \end{macro}
%
%    \begin{macro}{\@providecommandtwoopt}
%    \begin{param}
%      \entry1 star\\
%      \entry2 command name to be defined
%    \end{param}
%    \begin{macrocode}
\newcommand{\@providecommandtwoopt}{}
\long\def\@providecommandtwoopt#1#2{%
  \begingroup
    \escapechar\m@ne
    \xdef\@gtempa{{\string#2}}%
  \endgroup
  \expandafter\@ifundefined\@gtempa{%
    \expandafter\@@newcommandtwoopt
      \csname2\string#2\endcsname{#1}{#2}%
  }{%
    \let\to@dummyA\@undefined
    \let\to@dummyB\@undefined
    \@@newcommandtwoopt\to@dummyA{#1}\to@dummyB
  }%
}
%    \end{macrocode}
%    \end{macro}
%
%    \begin{macro}{\to@ScanSecondOptArg}
%    \begin{param}
%      \entry1 help command to be defined
%        (\expandafter\cmd\csname 2\bslash<name>\endcsname)\\
%      \entry2 first arg of command to be defined\\
%      \entry3 default for second opt. arg.
%    \end{param}
%    \begin{macrocode}
\newcommand{\to@ScanSecondOptArg}[3]{%
  \@ifnextchar[{%
    \expandafter#1\to@ArgOptToArgArg{#2}%
  }{%
    #1{#2}{#3}%
  }%
}
%    \end{macrocode}
%    \end{macro}
%
%    \begin{macro}{\to@ArgOptToArgArg}
%    \begin{macrocode}
\newcommand{\to@ArgOptToArgArg}{}
\long\def\to@ArgOptToArgArg#1[#2]{{#1}{#2}}
%    \end{macrocode}
%    \end{macro}
%
%    \begin{macrocode}
%</package>
%    \end{macrocode}
%
% \section{Installation}
%
% \subsection{Download}
%
% \paragraph{Package.} This package is available on
% CTAN\footnote{\url{http://ctan.org/pkg/twoopt}}:
% \begin{description}
% \item[\CTAN{macros/latex/contrib/oberdiek/twoopt.dtx}] The source file.
% \item[\CTAN{macros/latex/contrib/oberdiek/twoopt.pdf}] Documentation.
% \end{description}
%
%
% \paragraph{Bundle.} All the packages of the bundle `oberdiek'
% are also available in a TDS compliant ZIP archive. There
% the packages are already unpacked and the documentation files
% are generated. The files and directories obey the TDS standard.
% \begin{description}
% \item[\CTAN{install/macros/latex/contrib/oberdiek.tds.zip}]
% \end{description}
% \emph{TDS} refers to the standard ``A Directory Structure
% for \TeX\ Files'' (\CTAN{tds/tds.pdf}). Directories
% with \xfile{texmf} in their name are usually organized this way.
%
% \subsection{Bundle installation}
%
% \paragraph{Unpacking.} Unpack the \xfile{oberdiek.tds.zip} in the
% TDS tree (also known as \xfile{texmf} tree) of your choice.
% Example (linux):
% \begin{quote}
%   |unzip oberdiek.tds.zip -d ~/texmf|
% \end{quote}
%
% \paragraph{Script installation.}
% Check the directory \xfile{TDS:scripts/oberdiek/} for
% scripts that need further installation steps.
% Package \xpackage{attachfile2} comes with the Perl script
% \xfile{pdfatfi.pl} that should be installed in such a way
% that it can be called as \texttt{pdfatfi}.
% Example (linux):
% \begin{quote}
%   |chmod +x scripts/oberdiek/pdfatfi.pl|\\
%   |cp scripts/oberdiek/pdfatfi.pl /usr/local/bin/|
% \end{quote}
%
% \subsection{Package installation}
%
% \paragraph{Unpacking.} The \xfile{.dtx} file is a self-extracting
% \docstrip\ archive. The files are extracted by running the
% \xfile{.dtx} through \plainTeX:
% \begin{quote}
%   \verb|tex twoopt.dtx|
% \end{quote}
%
% \paragraph{TDS.} Now the different files must be moved into
% the different directories in your installation TDS tree
% (also known as \xfile{texmf} tree):
% \begin{quote}
% \def\t{^^A
% \begin{tabular}{@{}>{\ttfamily}l@{ $\rightarrow$ }>{\ttfamily}l@{}}
%   twoopt.sty & tex/latex/oberdiek/twoopt.sty\\
%   twoopt.pdf & doc/latex/oberdiek/twoopt.pdf\\
%   twoopt.dtx & source/latex/oberdiek/twoopt.dtx\\
% \end{tabular}^^A
% }^^A
% \sbox0{\t}^^A
% \ifdim\wd0>\linewidth
%   \begingroup
%     \advance\linewidth by\leftmargin
%     \advance\linewidth by\rightmargin
%   \edef\x{\endgroup
%     \def\noexpand\lw{\the\linewidth}^^A
%   }\x
%   \def\lwbox{^^A
%     \leavevmode
%     \hbox to \linewidth{^^A
%       \kern-\leftmargin\relax
%       \hss
%       \usebox0
%       \hss
%       \kern-\rightmargin\relax
%     }^^A
%   }^^A
%   \ifdim\wd0>\lw
%     \sbox0{\small\t}^^A
%     \ifdim\wd0>\linewidth
%       \ifdim\wd0>\lw
%         \sbox0{\footnotesize\t}^^A
%         \ifdim\wd0>\linewidth
%           \ifdim\wd0>\lw
%             \sbox0{\scriptsize\t}^^A
%             \ifdim\wd0>\linewidth
%               \ifdim\wd0>\lw
%                 \sbox0{\tiny\t}^^A
%                 \ifdim\wd0>\linewidth
%                   \lwbox
%                 \else
%                   \usebox0
%                 \fi
%               \else
%                 \lwbox
%               \fi
%             \else
%               \usebox0
%             \fi
%           \else
%             \lwbox
%           \fi
%         \else
%           \usebox0
%         \fi
%       \else
%         \lwbox
%       \fi
%     \else
%       \usebox0
%     \fi
%   \else
%     \lwbox
%   \fi
% \else
%   \usebox0
% \fi
% \end{quote}
% If you have a \xfile{docstrip.cfg} that configures and enables \docstrip's
% TDS installing feature, then some files can already be in the right
% place, see the documentation of \docstrip.
%
% \subsection{Refresh file name databases}
%
% If your \TeX~distribution
% (\teTeX, \mikTeX, \dots) relies on file name databases, you must refresh
% these. For example, \teTeX\ users run \verb|texhash| or
% \verb|mktexlsr|.
%
% \subsection{Some details for the interested}
%
% \paragraph{Attached source.}
%
% The PDF documentation on CTAN also includes the
% \xfile{.dtx} source file. It can be extracted by
% AcrobatReader 6 or higher. Another option is \textsf{pdftk},
% e.g. unpack the file into the current directory:
% \begin{quote}
%   \verb|pdftk twoopt.pdf unpack_files output .|
% \end{quote}
%
% \paragraph{Unpacking with \LaTeX.}
% The \xfile{.dtx} chooses its action depending on the format:
% \begin{description}
% \item[\plainTeX:] Run \docstrip\ and extract the files.
% \item[\LaTeX:] Generate the documentation.
% \end{description}
% If you insist on using \LaTeX\ for \docstrip\ (really,
% \docstrip\ does not need \LaTeX), then inform the autodetect routine
% about your intention:
% \begin{quote}
%   \verb|latex \let\install=y% \iffalse meta-comment
%
% File: twoopt.dtx
% Version: 2016/05/16 v1.6
% Info: Definitions with two optional arguments
%
% Copyright (C) 1999, 2006, 2008 by
%    Heiko Oberdiek <heiko.oberdiek at googlemail.com>
%    2016
%    https://github.com/ho-tex/oberdiek/issues
%
% This work may be distributed and/or modified under the
% conditions of the LaTeX Project Public License, either
% version 1.3c of this license or (at your option) any later
% version. This version of this license is in
%    http://www.latex-project.org/lppl/lppl-1-3c.txt
% and the latest version of this license is in
%    http://www.latex-project.org/lppl.txt
% and version 1.3 or later is part of all distributions of
% LaTeX version 2005/12/01 or later.
%
% This work has the LPPL maintenance status "maintained".
%
% This Current Maintainer of this work is Heiko Oberdiek.
%
% This work consists of the main source file twoopt.dtx
% and the derived files
%    twoopt.sty, twoopt.pdf, twoopt.ins, twoopt.drv.
%
% Distribution:
%    CTAN:macros/latex/contrib/oberdiek/twoopt.dtx
%    CTAN:macros/latex/contrib/oberdiek/twoopt.pdf
%
% Unpacking:
%    (a) If twoopt.ins is present:
%           tex twoopt.ins
%    (b) Without twoopt.ins:
%           tex twoopt.dtx
%    (c) If you insist on using LaTeX
%           latex \let\install=y\input{twoopt.dtx}
%        (quote the arguments according to the demands of your shell)
%
% Documentation:
%    (a) If twoopt.drv is present:
%           latex twoopt.drv
%    (b) Without twoopt.drv:
%           latex twoopt.dtx; ...
%    The class ltxdoc loads the configuration file ltxdoc.cfg
%    if available. Here you can specify further options, e.g.
%    use A4 as paper format:
%       \PassOptionsToClass{a4paper}{article}
%
%    Programm calls to get the documentation (example):
%       pdflatex twoopt.dtx
%       makeindex -s gind.ist twoopt.idx
%       pdflatex twoopt.dtx
%       makeindex -s gind.ist twoopt.idx
%       pdflatex twoopt.dtx
%
% Installation:
%    TDS:tex/latex/oberdiek/twoopt.sty
%    TDS:doc/latex/oberdiek/twoopt.pdf
%    TDS:source/latex/oberdiek/twoopt.dtx
%
%<*ignore>
\begingroup
  \catcode123=1 %
  \catcode125=2 %
  \def\x{LaTeX2e}%
\expandafter\endgroup
\ifcase 0\ifx\install y1\fi\expandafter
         \ifx\csname processbatchFile\endcsname\relax\else1\fi
         \ifx\fmtname\x\else 1\fi\relax
\else\csname fi\endcsname
%</ignore>
%<*install>
\input docstrip.tex
\Msg{************************************************************************}
\Msg{* Installation}
\Msg{* Package: twoopt 2016/05/16 v1.6 Definitions with two optional arguments (HO)}
\Msg{************************************************************************}

\keepsilent
\askforoverwritefalse

\let\MetaPrefix\relax
\preamble

This is a generated file.

Project: twoopt
Version: 2016/05/16 v1.6

Copyright (C) 1999, 2006, 2008 by
   Heiko Oberdiek <heiko.oberdiek at googlemail.com>

This work may be distributed and/or modified under the
conditions of the LaTeX Project Public License, either
version 1.3c of this license or (at your option) any later
version. This version of this license is in
   http://www.latex-project.org/lppl/lppl-1-3c.txt
and the latest version of this license is in
   http://www.latex-project.org/lppl.txt
and version 1.3 or later is part of all distributions of
LaTeX version 2005/12/01 or later.

This work has the LPPL maintenance status "maintained".

This Current Maintainer of this work is Heiko Oberdiek.

This work consists of the main source file twoopt.dtx
and the derived files
   twoopt.sty, twoopt.pdf, twoopt.ins, twoopt.drv.

\endpreamble
\let\MetaPrefix\DoubleperCent

\generate{%
  \file{twoopt.ins}{\from{twoopt.dtx}{install}}%
  \file{twoopt.drv}{\from{twoopt.dtx}{driver}}%
  \usedir{tex/latex/oberdiek}%
  \file{twoopt.sty}{\from{twoopt.dtx}{package}}%
  \nopreamble
  \nopostamble
%  \usedir{source/latex/oberdiek/catalogue}%
%  \file{twoopt.xml}{\from{twoopt.dtx}{catalogue}}%
}

\catcode32=13\relax% active space
\let =\space%
\Msg{************************************************************************}
\Msg{*}
\Msg{* To finish the installation you have to move the following}
\Msg{* file into a directory searched by TeX:}
\Msg{*}
\Msg{*     twoopt.sty}
\Msg{*}
\Msg{* To produce the documentation run the file `twoopt.drv'}
\Msg{* through LaTeX.}
\Msg{*}
\Msg{* Happy TeXing!}
\Msg{*}
\Msg{************************************************************************}

\endbatchfile
%</install>
%<*ignore>
\fi
%</ignore>
%<*driver>
\NeedsTeXFormat{LaTeX2e}
\ProvidesFile{twoopt.drv}%
  [2016/05/16 v1.6 Definitions with two optional arguments (HO)]%
\documentclass{ltxdoc}
\usepackage{holtxdoc}[2011/11/22]
\begin{document}
  \DocInput{twoopt.dtx}%
\end{document}
%</driver>
% \fi
%
%
% \CharacterTable
%  {Upper-case    \A\B\C\D\E\F\G\H\I\J\K\L\M\N\O\P\Q\R\S\T\U\V\W\X\Y\Z
%   Lower-case    \a\b\c\d\e\f\g\h\i\j\k\l\m\n\o\p\q\r\s\t\u\v\w\x\y\z
%   Digits        \0\1\2\3\4\5\6\7\8\9
%   Exclamation   \!     Double quote  \"     Hash (number) \#
%   Dollar        \$     Percent       \%     Ampersand     \&
%   Acute accent  \'     Left paren    \(     Right paren   \)
%   Asterisk      \*     Plus          \+     Comma         \,
%   Minus         \-     Point         \.     Solidus       \/
%   Colon         \:     Semicolon     \;     Less than     \<
%   Equals        \=     Greater than  \>     Question mark \?
%   Commercial at \@     Left bracket  \[     Backslash     \\
%   Right bracket \]     Circumflex    \^     Underscore    \_
%   Grave accent  \`     Left brace    \{     Vertical bar  \|
%   Right brace   \}     Tilde         \~}
%
% \GetFileInfo{twoopt.drv}
%
% \title{The \xpackage{twoopt} package}
% \date{2016/05/16 v1.6}
% \author{Heiko Oberdiek\thanks
% {Please report any issues at https://github.com/ho-tex/oberdiek/issues}\\
% \xemail{heiko.oberdiek at googlemail.com}}
%
% \maketitle
%
% \begin{abstract}
% This package provides commands to define macros with two
% optional arguments.
% \end{abstract}
%
% \tableofcontents
%
% \newenvironment{param}{^^A
%   \newcommand{\entry}[1]{\meta{\###1}:&}^^A
%   \begin{tabular}[t]{@{}l@{ }l@{}}^^A
% }{^^A
%   \end{tabular}^^A
% }
%
% \section{Usage}
%    \DescribeMacro{\newcommandtwoopt}
%    \DescribeMacro{\renewcommandtwoopt}
%    \DescribeMacro{\providecommandtwoopt}
%    Similar to \cmd{\newcommand}, \cmd{\renewcommand}
%    and \cmd{\providecommand} this package provides commands
%    to define macros with two optional arguments.
%    The names of the commands are built by appending the
%    package name to the \LaTeX-pendants:
%    \begingroup
%      \def\x{\marg{cmd} \oarg{num} \oarg{default1}^^A
%             \oarg{default2} \marg{def.}}^^A
%      \begin{tabbing}
%        \cmd{\providecommandtwoopt} \=\kill
%        \cmd{\newcommandtwoopt}\>\x\\
%        \cmd{\renewcommandtwoopt}\>\x\\
%        \cmd{\providecommandtwoopt}\>\x\\
%      \end{tabbing}
%    \endgroup
%
%    Also the |*|-forms are supported. Indeed it is better to
%    use this ones, unless it is intended to hold
%    whole paragraphs in some of the arguments. If the macro
%    is defined with the |*|-form, missing braces
%    can be detected earlier.
%
%    Example:
%    \begin{quote}
%      |\newcommandtwoopt{\bsp}[3][AA][BB]{%|\\
%      |  \typeout{\string\bsp: #1,#2,#3}%|\\
%      |}|\\
%      \begin{tabular}{@{}l@{\quad$\rightarrow$\quad}l@{}}
%      |\bsp[aa][bb]{cc}|&|\bsp: aa,bb,cc|\\
%      |\bsp[aa]{cc}|&|\bsp: aa,BB,cc|\\
%      |\bsp{cc}|&|\bsp: AA,BB,cc|\\
%      \end{tabular}
%    \end{quote}
%
% \StopEventually{
% }
%
% \section{Implementation}
%    \begin{macrocode}
%<*package>
\NeedsTeXFormat{LaTeX2e}
\ProvidesPackage{twoopt}
  [2016/05/16 v1.6 Definitions with two optional arguments (HO)]%
%    \end{macrocode}
%    \begin{macro}{\newcommandtwoopt}
%    \begin{macrocode}
\newcommand{\newcommandtwoopt}{%
  \@ifstar{\@newcommandtwoopt*}{\@newcommandtwoopt{}}%
}
%    \end{macrocode}
%    \end{macro}
%
%    \begin{macro}{\@newcommandtwoopt}
%    \begin{param}
%      \entry1 star\\
%      \entry2 macro name to be defined
%    \end{param}
%    \begin{macrocode}
\newcommand{\@newcommandtwoopt}{}
\long\def\@newcommandtwoopt#1#2{%
  \expandafter\@@newcommandtwoopt
    \csname2\string#2\endcsname{#1}{#2}%
}
%    \end{macrocode}
%    \end{macro}
%
%    \begin{macro}{\@@newcommandtwoopt}
%    \begin{param}
%      \entry1 help command to be defined
%        (\expandafter\cmd\csname 2\bslash<name>\endcsname)\\
%      \entry2 star\\
%      \entry3 macro name to be defined\\
%      \entry4 number of total arguments\\
%      \entry5 default for optional argument one\\
%      \entry6 default for optional argument two
%    \end{param}
%    \begin{macrocode}
\newcommand{\@@newcommandtwoopt}{}
\long\def\@@newcommandtwoopt#1#2#3[#4][#5][#6]{%
  \newcommand#2#3[1][{#5}]{%
    \to@ScanSecondOptArg#1{##1}{#6}%
  }%
  \newcommand#2#1[{#4}]%
}
%    \end{macrocode}
%    \end{macro}
%
%    \begin{macro}{\renewcommandtwoopt}
%    \begin{macrocode}
\newcommand{\renewcommandtwoopt}{%
  \@ifstar{\@renewcommandtwoopt*}{\@renewcommandtwoopt{}}%
}
%    \end{macrocode}
%    \end{macro}
%
%    \begin{macro}{\@renewcommandtwoopt}
%    \begin{param}
%      \entry1 star\\
%      \entry2 command name to be defined
%    \end{param}
%    \begin{macrocode}
\newcommand{\@renewcommandtwoopt}{}
\long\def\@renewcommandtwoopt#1#2{%
  \begingroup
    \escapechar\m@ne
    \xdef\@gtempa{{\string#2}}%
  \endgroup
  \expandafter\@ifundefined\@gtempa{%
    \@latex@error{\noexpand#2undefined}\@ehc
  }{}%
  \let#2\@undefined
  \expandafter\let\csname2\string#2\endcsname\@undefined
  \expandafter\@@newcommandtwoopt
    \csname2\string#2\endcsname{#1}{#2}%
}
%    \end{macrocode}
%    \end{macro}
%
%    \begin{macro}{\providecommandtwoopt}
%    \begin{macrocode}
\newcommand{\providecommandtwoopt}{%
  \@ifstar{\@providecommandtwoopt*}{\@providecommandtwoopt{}}%
}
%    \end{macrocode}
%    \end{macro}
%
%    \begin{macro}{\@providecommandtwoopt}
%    \begin{param}
%      \entry1 star\\
%      \entry2 command name to be defined
%    \end{param}
%    \begin{macrocode}
\newcommand{\@providecommandtwoopt}{}
\long\def\@providecommandtwoopt#1#2{%
  \begingroup
    \escapechar\m@ne
    \xdef\@gtempa{{\string#2}}%
  \endgroup
  \expandafter\@ifundefined\@gtempa{%
    \expandafter\@@newcommandtwoopt
      \csname2\string#2\endcsname{#1}{#2}%
  }{%
    \let\to@dummyA\@undefined
    \let\to@dummyB\@undefined
    \@@newcommandtwoopt\to@dummyA{#1}\to@dummyB
  }%
}
%    \end{macrocode}
%    \end{macro}
%
%    \begin{macro}{\to@ScanSecondOptArg}
%    \begin{param}
%      \entry1 help command to be defined
%        (\expandafter\cmd\csname 2\bslash<name>\endcsname)\\
%      \entry2 first arg of command to be defined\\
%      \entry3 default for second opt. arg.
%    \end{param}
%    \begin{macrocode}
\newcommand{\to@ScanSecondOptArg}[3]{%
  \@ifnextchar[{%
    \expandafter#1\to@ArgOptToArgArg{#2}%
  }{%
    #1{#2}{#3}%
  }%
}
%    \end{macrocode}
%    \end{macro}
%
%    \begin{macro}{\to@ArgOptToArgArg}
%    \begin{macrocode}
\newcommand{\to@ArgOptToArgArg}{}
\long\def\to@ArgOptToArgArg#1[#2]{{#1}{#2}}
%    \end{macrocode}
%    \end{macro}
%
%    \begin{macrocode}
%</package>
%    \end{macrocode}
%
% \section{Installation}
%
% \subsection{Download}
%
% \paragraph{Package.} This package is available on
% CTAN\footnote{\url{http://ctan.org/pkg/twoopt}}:
% \begin{description}
% \item[\CTAN{macros/latex/contrib/oberdiek/twoopt.dtx}] The source file.
% \item[\CTAN{macros/latex/contrib/oberdiek/twoopt.pdf}] Documentation.
% \end{description}
%
%
% \paragraph{Bundle.} All the packages of the bundle `oberdiek'
% are also available in a TDS compliant ZIP archive. There
% the packages are already unpacked and the documentation files
% are generated. The files and directories obey the TDS standard.
% \begin{description}
% \item[\CTAN{install/macros/latex/contrib/oberdiek.tds.zip}]
% \end{description}
% \emph{TDS} refers to the standard ``A Directory Structure
% for \TeX\ Files'' (\CTAN{tds/tds.pdf}). Directories
% with \xfile{texmf} in their name are usually organized this way.
%
% \subsection{Bundle installation}
%
% \paragraph{Unpacking.} Unpack the \xfile{oberdiek.tds.zip} in the
% TDS tree (also known as \xfile{texmf} tree) of your choice.
% Example (linux):
% \begin{quote}
%   |unzip oberdiek.tds.zip -d ~/texmf|
% \end{quote}
%
% \paragraph{Script installation.}
% Check the directory \xfile{TDS:scripts/oberdiek/} for
% scripts that need further installation steps.
% Package \xpackage{attachfile2} comes with the Perl script
% \xfile{pdfatfi.pl} that should be installed in such a way
% that it can be called as \texttt{pdfatfi}.
% Example (linux):
% \begin{quote}
%   |chmod +x scripts/oberdiek/pdfatfi.pl|\\
%   |cp scripts/oberdiek/pdfatfi.pl /usr/local/bin/|
% \end{quote}
%
% \subsection{Package installation}
%
% \paragraph{Unpacking.} The \xfile{.dtx} file is a self-extracting
% \docstrip\ archive. The files are extracted by running the
% \xfile{.dtx} through \plainTeX:
% \begin{quote}
%   \verb|tex twoopt.dtx|
% \end{quote}
%
% \paragraph{TDS.} Now the different files must be moved into
% the different directories in your installation TDS tree
% (also known as \xfile{texmf} tree):
% \begin{quote}
% \def\t{^^A
% \begin{tabular}{@{}>{\ttfamily}l@{ $\rightarrow$ }>{\ttfamily}l@{}}
%   twoopt.sty & tex/latex/oberdiek/twoopt.sty\\
%   twoopt.pdf & doc/latex/oberdiek/twoopt.pdf\\
%   twoopt.dtx & source/latex/oberdiek/twoopt.dtx\\
% \end{tabular}^^A
% }^^A
% \sbox0{\t}^^A
% \ifdim\wd0>\linewidth
%   \begingroup
%     \advance\linewidth by\leftmargin
%     \advance\linewidth by\rightmargin
%   \edef\x{\endgroup
%     \def\noexpand\lw{\the\linewidth}^^A
%   }\x
%   \def\lwbox{^^A
%     \leavevmode
%     \hbox to \linewidth{^^A
%       \kern-\leftmargin\relax
%       \hss
%       \usebox0
%       \hss
%       \kern-\rightmargin\relax
%     }^^A
%   }^^A
%   \ifdim\wd0>\lw
%     \sbox0{\small\t}^^A
%     \ifdim\wd0>\linewidth
%       \ifdim\wd0>\lw
%         \sbox0{\footnotesize\t}^^A
%         \ifdim\wd0>\linewidth
%           \ifdim\wd0>\lw
%             \sbox0{\scriptsize\t}^^A
%             \ifdim\wd0>\linewidth
%               \ifdim\wd0>\lw
%                 \sbox0{\tiny\t}^^A
%                 \ifdim\wd0>\linewidth
%                   \lwbox
%                 \else
%                   \usebox0
%                 \fi
%               \else
%                 \lwbox
%               \fi
%             \else
%               \usebox0
%             \fi
%           \else
%             \lwbox
%           \fi
%         \else
%           \usebox0
%         \fi
%       \else
%         \lwbox
%       \fi
%     \else
%       \usebox0
%     \fi
%   \else
%     \lwbox
%   \fi
% \else
%   \usebox0
% \fi
% \end{quote}
% If you have a \xfile{docstrip.cfg} that configures and enables \docstrip's
% TDS installing feature, then some files can already be in the right
% place, see the documentation of \docstrip.
%
% \subsection{Refresh file name databases}
%
% If your \TeX~distribution
% (\teTeX, \mikTeX, \dots) relies on file name databases, you must refresh
% these. For example, \teTeX\ users run \verb|texhash| or
% \verb|mktexlsr|.
%
% \subsection{Some details for the interested}
%
% \paragraph{Attached source.}
%
% The PDF documentation on CTAN also includes the
% \xfile{.dtx} source file. It can be extracted by
% AcrobatReader 6 or higher. Another option is \textsf{pdftk},
% e.g. unpack the file into the current directory:
% \begin{quote}
%   \verb|pdftk twoopt.pdf unpack_files output .|
% \end{quote}
%
% \paragraph{Unpacking with \LaTeX.}
% The \xfile{.dtx} chooses its action depending on the format:
% \begin{description}
% \item[\plainTeX:] Run \docstrip\ and extract the files.
% \item[\LaTeX:] Generate the documentation.
% \end{description}
% If you insist on using \LaTeX\ for \docstrip\ (really,
% \docstrip\ does not need \LaTeX), then inform the autodetect routine
% about your intention:
% \begin{quote}
%   \verb|latex \let\install=y\input{twoopt.dtx}|
% \end{quote}
% Do not forget to quote the argument according to the demands
% of your shell.
%
% \paragraph{Generating the documentation.}
% You can use both the \xfile{.dtx} or the \xfile{.drv} to generate
% the documentation. The process can be configured by the
% configuration file \xfile{ltxdoc.cfg}. For instance, put this
% line into this file, if you want to have A4 as paper format:
% \begin{quote}
%   \verb|\PassOptionsToClass{a4paper}{article}|
% \end{quote}
% An example follows how to generate the
% documentation with pdf\LaTeX:
% \begin{quote}
%\begin{verbatim}
%pdflatex twoopt.dtx
%makeindex -s gind.ist twoopt.idx
%pdflatex twoopt.dtx
%makeindex -s gind.ist twoopt.idx
%pdflatex twoopt.dtx
%\end{verbatim}
% \end{quote}
%
% \section{Catalogue}
%
% The following XML file can be used as source for the
% \href{http://mirror.ctan.org/help/Catalogue/catalogue.html}{\TeX\ Catalogue}.
% The elements \texttt{caption} and \texttt{description} are imported
% from the original XML file from the Catalogue.
% The name of the XML file in the Catalogue is \xfile{twoopt.xml}.
%    \begin{macrocode}
%<*catalogue>
<?xml version='1.0' encoding='us-ascii'?>
<!DOCTYPE entry SYSTEM 'catalogue.dtd'>
<entry datestamp='$Date$' modifier='$Author$' id='twoopt'>
  <name>twoopt</name>
  <caption>Definitions with two optional arguments.</caption>
  <authorref id='auth:oberdiek'/>
  <copyright owner='Heiko Oberdiek' year='1999,2006,2008'/>
  <license type='lppl1.3'/>
  <version number='1.6'/>
  <description>
    Variants of <tt>\newcommand</tt>, <tt>\renewcommand</tt> and
    <tt>\providecommand</tt> are provided.
    <p/>
    The package is part of the <xref refid='oberdiek'>oberdiek</xref>
    bundle.
  </description>
  <documentation details='Package documentation'
      href='ctan:/macros/latex/contrib/oberdiek/twoopt.pdf'/>
  <ctan file='true' path='/macros/latex/contrib/oberdiek/twoopt.dtx'/>
  <miktex location='oberdiek'/>
  <texlive location='oberdiek'/>
  <install path='/macros/latex/contrib/oberdiek/oberdiek.tds.zip'/>
</entry>
%</catalogue>
%    \end{macrocode}
%
% \begin{History}
%   \begin{Version}{1998/10/30 v1.0}
%   \item
%     The first version was built as a response to a question
%     of \NameEmail{Rebecca and Rowland}{rebecca@astrid.u-net.com},
%     published in the newsgroup
%     \href{news:comp.text.tex}{comp.text.tex}:\\
%     \URL{``Re: [Q] LaTeX command with two optional arguments?''}^^A
%     {http://groups.google.com/group/comp.text.tex/msg/0ab1afde7b172d37}
%   \end{Version}
%   \begin{Version}{1998/10/30 v1.1}
%   \item
%     Improvements added in response to
%     \NameEmail{Stefan Ulrich}{ulrich@cis.uni-muenchen.de}
%     in the same thread:\\
%     \URL{``Re: [Q] LaTeX command with two optional arguments?''}^^A
%     {http://groups.google.com/group/comp.text.tex/msg/b8d84d4336f302c4}
%   \end{Version}
%   \begin{Version}{1998/11/04 v1.2}
%   \item
%     Fixes for LaTeX bugs 2896, 2901, 2902 added.
%   \end{Version}
%   \begin{Version}{1999/04/12 v1.3}
%   \item
%     Fixes removed because of LaTeX [1998/12/01].
%   \item
%     Documentation in dtx format.
%   \item
%     Copyright: LPPL (\CTAN{macros/latex/base/lppl.txt})
%   \item
%     First CTAN release.
%   \end{Version}
%   \begin{Version}{2006/02/20 v1.4}
%   \item
%     Code is not changed.
%   \item
%     New DTX framework.
%   \item
%     LPPL 1.3
%   \end{Version}
%   \begin{Version}{2008/08/11 v1.5}
%   \item
%     Code is not changed.
%   \item
%     URLs updated from \texttt{www.dejanews.com}
%     to \texttt{groups.google.com}.
%   \end{Version}
%   \begin{Version}{2016/05/16 v1.6}
%   \item
%     Documentation updates.
%   \end{Version}
% \end{History}
%
% \PrintIndex
%
% \Finale
\endinput
|
% \end{quote}
% Do not forget to quote the argument according to the demands
% of your shell.
%
% \paragraph{Generating the documentation.}
% You can use both the \xfile{.dtx} or the \xfile{.drv} to generate
% the documentation. The process can be configured by the
% configuration file \xfile{ltxdoc.cfg}. For instance, put this
% line into this file, if you want to have A4 as paper format:
% \begin{quote}
%   \verb|\PassOptionsToClass{a4paper}{article}|
% \end{quote}
% An example follows how to generate the
% documentation with pdf\LaTeX:
% \begin{quote}
%\begin{verbatim}
%pdflatex twoopt.dtx
%makeindex -s gind.ist twoopt.idx
%pdflatex twoopt.dtx
%makeindex -s gind.ist twoopt.idx
%pdflatex twoopt.dtx
%\end{verbatim}
% \end{quote}
%
% \section{Catalogue}
%
% The following XML file can be used as source for the
% \href{http://mirror.ctan.org/help/Catalogue/catalogue.html}{\TeX\ Catalogue}.
% The elements \texttt{caption} and \texttt{description} are imported
% from the original XML file from the Catalogue.
% The name of the XML file in the Catalogue is \xfile{twoopt.xml}.
%    \begin{macrocode}
%<*catalogue>
<?xml version='1.0' encoding='us-ascii'?>
<!DOCTYPE entry SYSTEM 'catalogue.dtd'>
<entry datestamp='$Date$' modifier='$Author$' id='twoopt'>
  <name>twoopt</name>
  <caption>Definitions with two optional arguments.</caption>
  <authorref id='auth:oberdiek'/>
  <copyright owner='Heiko Oberdiek' year='1999,2006,2008'/>
  <license type='lppl1.3'/>
  <version number='1.6'/>
  <description>
    Variants of <tt>\newcommand</tt>, <tt>\renewcommand</tt> and
    <tt>\providecommand</tt> are provided.
    <p/>
    The package is part of the <xref refid='oberdiek'>oberdiek</xref>
    bundle.
  </description>
  <documentation details='Package documentation'
      href='ctan:/macros/latex/contrib/oberdiek/twoopt.pdf'/>
  <ctan file='true' path='/macros/latex/contrib/oberdiek/twoopt.dtx'/>
  <miktex location='oberdiek'/>
  <texlive location='oberdiek'/>
  <install path='/macros/latex/contrib/oberdiek/oberdiek.tds.zip'/>
</entry>
%</catalogue>
%    \end{macrocode}
%
% \begin{History}
%   \begin{Version}{1998/10/30 v1.0}
%   \item
%     The first version was built as a response to a question
%     of \NameEmail{Rebecca and Rowland}{rebecca@astrid.u-net.com},
%     published in the newsgroup
%     \href{news:comp.text.tex}{comp.text.tex}:\\
%     \URL{``Re: [Q] LaTeX command with two optional arguments?''}^^A
%     {http://groups.google.com/group/comp.text.tex/msg/0ab1afde7b172d37}
%   \end{Version}
%   \begin{Version}{1998/10/30 v1.1}
%   \item
%     Improvements added in response to
%     \NameEmail{Stefan Ulrich}{ulrich@cis.uni-muenchen.de}
%     in the same thread:\\
%     \URL{``Re: [Q] LaTeX command with two optional arguments?''}^^A
%     {http://groups.google.com/group/comp.text.tex/msg/b8d84d4336f302c4}
%   \end{Version}
%   \begin{Version}{1998/11/04 v1.2}
%   \item
%     Fixes for LaTeX bugs 2896, 2901, 2902 added.
%   \end{Version}
%   \begin{Version}{1999/04/12 v1.3}
%   \item
%     Fixes removed because of LaTeX [1998/12/01].
%   \item
%     Documentation in dtx format.
%   \item
%     Copyright: LPPL (\CTAN{macros/latex/base/lppl.txt})
%   \item
%     First CTAN release.
%   \end{Version}
%   \begin{Version}{2006/02/20 v1.4}
%   \item
%     Code is not changed.
%   \item
%     New DTX framework.
%   \item
%     LPPL 1.3
%   \end{Version}
%   \begin{Version}{2008/08/11 v1.5}
%   \item
%     Code is not changed.
%   \item
%     URLs updated from \texttt{www.dejanews.com}
%     to \texttt{groups.google.com}.
%   \end{Version}
%   \begin{Version}{2016/05/16 v1.6}
%   \item
%     Documentation updates.
%   \end{Version}
% \end{History}
%
% \PrintIndex
%
% \Finale
\endinput
|
% \end{quote}
% Do not forget to quote the argument according to the demands
% of your shell.
%
% \paragraph{Generating the documentation.}
% You can use both the \xfile{.dtx} or the \xfile{.drv} to generate
% the documentation. The process can be configured by the
% configuration file \xfile{ltxdoc.cfg}. For instance, put this
% line into this file, if you want to have A4 as paper format:
% \begin{quote}
%   \verb|\PassOptionsToClass{a4paper}{article}|
% \end{quote}
% An example follows how to generate the
% documentation with pdf\LaTeX:
% \begin{quote}
%\begin{verbatim}
%pdflatex twoopt.dtx
%makeindex -s gind.ist twoopt.idx
%pdflatex twoopt.dtx
%makeindex -s gind.ist twoopt.idx
%pdflatex twoopt.dtx
%\end{verbatim}
% \end{quote}
%
% \section{Catalogue}
%
% The following XML file can be used as source for the
% \href{http://mirror.ctan.org/help/Catalogue/catalogue.html}{\TeX\ Catalogue}.
% The elements \texttt{caption} and \texttt{description} are imported
% from the original XML file from the Catalogue.
% The name of the XML file in the Catalogue is \xfile{twoopt.xml}.
%    \begin{macrocode}
%<*catalogue>
<?xml version='1.0' encoding='us-ascii'?>
<!DOCTYPE entry SYSTEM 'catalogue.dtd'>
<entry datestamp='$Date$' modifier='$Author$' id='twoopt'>
  <name>twoopt</name>
  <caption>Definitions with two optional arguments.</caption>
  <authorref id='auth:oberdiek'/>
  <copyright owner='Heiko Oberdiek' year='1999,2006,2008'/>
  <license type='lppl1.3'/>
  <version number='1.6'/>
  <description>
    Variants of <tt>\newcommand</tt>, <tt>\renewcommand</tt> and
    <tt>\providecommand</tt> are provided.
    <p/>
    The package is part of the <xref refid='oberdiek'>oberdiek</xref>
    bundle.
  </description>
  <documentation details='Package documentation'
      href='ctan:/macros/latex/contrib/oberdiek/twoopt.pdf'/>
  <ctan file='true' path='/macros/latex/contrib/oberdiek/twoopt.dtx'/>
  <miktex location='oberdiek'/>
  <texlive location='oberdiek'/>
  <install path='/macros/latex/contrib/oberdiek/oberdiek.tds.zip'/>
</entry>
%</catalogue>
%    \end{macrocode}
%
% \begin{History}
%   \begin{Version}{1998/10/30 v1.0}
%   \item
%     The first version was built as a response to a question
%     of \NameEmail{Rebecca and Rowland}{rebecca@astrid.u-net.com},
%     published in the newsgroup
%     \href{news:comp.text.tex}{comp.text.tex}:\\
%     \URL{``Re: [Q] LaTeX command with two optional arguments?''}^^A
%     {http://groups.google.com/group/comp.text.tex/msg/0ab1afde7b172d37}
%   \end{Version}
%   \begin{Version}{1998/10/30 v1.1}
%   \item
%     Improvements added in response to
%     \NameEmail{Stefan Ulrich}{ulrich@cis.uni-muenchen.de}
%     in the same thread:\\
%     \URL{``Re: [Q] LaTeX command with two optional arguments?''}^^A
%     {http://groups.google.com/group/comp.text.tex/msg/b8d84d4336f302c4}
%   \end{Version}
%   \begin{Version}{1998/11/04 v1.2}
%   \item
%     Fixes for LaTeX bugs 2896, 2901, 2902 added.
%   \end{Version}
%   \begin{Version}{1999/04/12 v1.3}
%   \item
%     Fixes removed because of LaTeX [1998/12/01].
%   \item
%     Documentation in dtx format.
%   \item
%     Copyright: LPPL (\CTAN{macros/latex/base/lppl.txt})
%   \item
%     First CTAN release.
%   \end{Version}
%   \begin{Version}{2006/02/20 v1.4}
%   \item
%     Code is not changed.
%   \item
%     New DTX framework.
%   \item
%     LPPL 1.3
%   \end{Version}
%   \begin{Version}{2008/08/11 v1.5}
%   \item
%     Code is not changed.
%   \item
%     URLs updated from \texttt{www.dejanews.com}
%     to \texttt{groups.google.com}.
%   \end{Version}
%   \begin{Version}{2016/05/16 v1.6}
%   \item
%     Documentation updates.
%   \end{Version}
% \end{History}
%
% \PrintIndex
%
% \Finale
\endinput

%        (quote the arguments according to the demands of your shell)
%
% Documentation:
%    (a) If twoopt.drv is present:
%           latex twoopt.drv
%    (b) Without twoopt.drv:
%           latex twoopt.dtx; ...
%    The class ltxdoc loads the configuration file ltxdoc.cfg
%    if available. Here you can specify further options, e.g.
%    use A4 as paper format:
%       \PassOptionsToClass{a4paper}{article}
%
%    Programm calls to get the documentation (example):
%       pdflatex twoopt.dtx
%       makeindex -s gind.ist twoopt.idx
%       pdflatex twoopt.dtx
%       makeindex -s gind.ist twoopt.idx
%       pdflatex twoopt.dtx
%
% Installation:
%    TDS:tex/latex/oberdiek/twoopt.sty
%    TDS:doc/latex/oberdiek/twoopt.pdf
%    TDS:source/latex/oberdiek/twoopt.dtx
%
%<*ignore>
\begingroup
  \catcode123=1 %
  \catcode125=2 %
  \def\x{LaTeX2e}%
\expandafter\endgroup
\ifcase 0\ifx\install y1\fi\expandafter
         \ifx\csname processbatchFile\endcsname\relax\else1\fi
         \ifx\fmtname\x\else 1\fi\relax
\else\csname fi\endcsname
%</ignore>
%<*install>
\input docstrip.tex
\Msg{************************************************************************}
\Msg{* Installation}
\Msg{* Package: twoopt 2008/08/11 v1.5 Definitions with two optional arguments (HO)}
\Msg{************************************************************************}

\keepsilent
\askforoverwritefalse

\let\MetaPrefix\relax
\preamble

This is a generated file.

Project: twoopt
Version: 2008/08/11 v1.5

Copyright (C) 1999, 2006, 2008 by
   Heiko Oberdiek <heiko.oberdiek at googlemail.com>

This work may be distributed and/or modified under the
conditions of the LaTeX Project Public License, either
version 1.3c of this license or (at your option) any later
version. This version of this license is in
   http://www.latex-project.org/lppl/lppl-1-3c.txt
and the latest version of this license is in
   http://www.latex-project.org/lppl.txt
and version 1.3 or later is part of all distributions of
LaTeX version 2005/12/01 or later.

This work has the LPPL maintenance status "maintained".

This Current Maintainer of this work is Heiko Oberdiek.

This work consists of the main source file twoopt.dtx
and the derived files
   twoopt.sty, twoopt.pdf, twoopt.ins, twoopt.drv.

\endpreamble
\let\MetaPrefix\DoubleperCent

\generate{%
  \file{twoopt.ins}{\from{twoopt.dtx}{install}}%
  \file{twoopt.drv}{\from{twoopt.dtx}{driver}}%
  \usedir{tex/latex/oberdiek}%
  \file{twoopt.sty}{\from{twoopt.dtx}{package}}%
  \nopreamble
  \nopostamble
  \usedir{source/latex/oberdiek/catalogue}%
  \file{twoopt.xml}{\from{twoopt.dtx}{catalogue}}%
}

\catcode32=13\relax% active space
\let =\space%
\Msg{************************************************************************}
\Msg{*}
\Msg{* To finish the installation you have to move the following}
\Msg{* file into a directory searched by TeX:}
\Msg{*}
\Msg{*     twoopt.sty}
\Msg{*}
\Msg{* To produce the documentation run the file `twoopt.drv'}
\Msg{* through LaTeX.}
\Msg{*}
\Msg{* Happy TeXing!}
\Msg{*}
\Msg{************************************************************************}

\endbatchfile
%</install>
%<*ignore>
\fi
%</ignore>
%<*driver>
\NeedsTeXFormat{LaTeX2e}
\ProvidesFile{twoopt.drv}%
  [2008/08/11 v1.5 Definitions with two optional arguments (HO)]%
\documentclass{ltxdoc}
\usepackage{holtxdoc}[2011/11/22]
\begin{document}
  \DocInput{twoopt.dtx}%
\end{document}
%</driver>
% \fi
%
% \CheckSum{108}
%
% \CharacterTable
%  {Upper-case    \A\B\C\D\E\F\G\H\I\J\K\L\M\N\O\P\Q\R\S\T\U\V\W\X\Y\Z
%   Lower-case    \a\b\c\d\e\f\g\h\i\j\k\l\m\n\o\p\q\r\s\t\u\v\w\x\y\z
%   Digits        \0\1\2\3\4\5\6\7\8\9
%   Exclamation   \!     Double quote  \"     Hash (number) \#
%   Dollar        \$     Percent       \%     Ampersand     \&
%   Acute accent  \'     Left paren    \(     Right paren   \)
%   Asterisk      \*     Plus          \+     Comma         \,
%   Minus         \-     Point         \.     Solidus       \/
%   Colon         \:     Semicolon     \;     Less than     \<
%   Equals        \=     Greater than  \>     Question mark \?
%   Commercial at \@     Left bracket  \[     Backslash     \\
%   Right bracket \]     Circumflex    \^     Underscore    \_
%   Grave accent  \`     Left brace    \{     Vertical bar  \|
%   Right brace   \}     Tilde         \~}
%
% \GetFileInfo{twoopt.drv}
%
% \title{The \xpackage{twoopt} package}
% \date{2008/08/11 v1.5}
% \author{Heiko Oberdiek\\\xemail{heiko.oberdiek at googlemail.com}}
%
% \maketitle
%
% \begin{abstract}
% This package provides commands to define macros with two
% optional arguments.
% \end{abstract}
%
% \tableofcontents
%
% \newenvironment{param}{^^A
%   \newcommand{\entry}[1]{\meta{\###1}:&}^^A
%   \begin{tabular}[t]{@{}l@{ }l@{}}^^A
% }{^^A
%   \end{tabular}^^A
% }
%
% \section{Usage}
%    \DescribeMacro{\newcommandtwoopt}
%    \DescribeMacro{\renewcommandtwoopt}
%    \DescribeMacro{\providecommandtwoopt}
%    Similar to \cmd{\newcommand}, \cmd{\renewcommand}
%    and \cmd{\providecommand} this package provides commands
%    to define macros with two optional arguments.
%    The names of the commands are built by appending the
%    package name to the \LaTeX-pendants:
%    \begingroup
%      \def\x{\marg{cmd} \oarg{num} \oarg{default1}^^A
%             \oarg{default2} \marg{def.}}^^A
%      \begin{tabbing}
%        \cmd{\providecommandtwoopt} \=\kill
%        \cmd{\newcommandtwoopt}\>\x\\
%        \cmd{\renewcommandtwoopt}\>\x\\
%        \cmd{\providecommandtwoopt}\>\x\\
%      \end{tabbing}
%    \endgroup
%
%    Also the |*|-forms are supported. Indeed it is better to
%    use this ones, unless it is intended to hold
%    whole paragraphs in some of the arguments. If the macro
%    is defined with the |*|-form, missing braces
%    can be detected earlier.
%
%    Example:
%    \begin{quote}
%      |\newcommandtwoopt{\bsp}[3][AA][BB]{%|\\
%      |  \typeout{\string\bsp: #1,#2,#3}%|\\
%      |}|\\
%      \begin{tabular}{@{}l@{\quad$\rightarrow$\quad}l@{}}
%      |\bsp[aa][bb]{cc}|&|\bsp: aa,bb,cc|\\
%      |\bsp[aa]{cc}|&|\bsp: aa,BB,cc|\\
%      |\bsp{cc}|&|\bsp: AA,BB,cc|\\
%      \end{tabular}
%    \end{quote}
%
% \StopEventually{
% }
%
% \section{Implementation}
%    \begin{macrocode}
%<*package>
\NeedsTeXFormat{LaTeX2e}
\ProvidesPackage{twoopt}
  [2008/08/11 v1.5 Definitions with two optional arguments (HO)]%
%    \end{macrocode}
%    \begin{macro}{\newcommandtwoopt}
%    \begin{macrocode}
\newcommand{\newcommandtwoopt}{%
  \@ifstar{\@newcommandtwoopt*}{\@newcommandtwoopt{}}%
}
%    \end{macrocode}
%    \end{macro}
%
%    \begin{macro}{\@newcommandtwoopt}
%    \begin{param}
%      \entry1 star\\
%      \entry2 macro name to be defined
%    \end{param}
%    \begin{macrocode}
\newcommand{\@newcommandtwoopt}{}
\long\def\@newcommandtwoopt#1#2{%
  \expandafter\@@newcommandtwoopt
    \csname2\string#2\endcsname{#1}{#2}%
}
%    \end{macrocode}
%    \end{macro}
%
%    \begin{macro}{\@@newcommandtwoopt}
%    \begin{param}
%      \entry1 help command to be defined
%        (\expandafter\cmd\csname 2\bslash<name>\endcsname)\\
%      \entry2 star\\
%      \entry3 macro name to be defined\\
%      \entry4 number of total arguments\\
%      \entry5 default for optional argument one\\
%      \entry6 default for optional argument two
%    \end{param}
%    \begin{macrocode}
\newcommand{\@@newcommandtwoopt}{}
\long\def\@@newcommandtwoopt#1#2#3[#4][#5][#6]{%
  \newcommand#2#3[1][{#5}]{%
    \to@ScanSecondOptArg#1{##1}{#6}%
  }%
  \newcommand#2#1[{#4}]%
}
%    \end{macrocode}
%    \end{macro}
%
%    \begin{macro}{\renewcommandtwoopt}
%    \begin{macrocode}
\newcommand{\renewcommandtwoopt}{%
  \@ifstar{\@renewcommandtwoopt*}{\@renewcommandtwoopt{}}%
}
%    \end{macrocode}
%    \end{macro}
%
%    \begin{macro}{\@renewcommandtwoopt}
%    \begin{param}
%      \entry1 star\\
%      \entry2 command name to be defined
%    \end{param}
%    \begin{macrocode}
\newcommand{\@renewcommandtwoopt}{}
\long\def\@renewcommandtwoopt#1#2{%
  \begingroup
    \escapechar\m@ne
    \xdef\@gtempa{{\string#2}}%
  \endgroup
  \expandafter\@ifundefined\@gtempa{%
    \@latex@error{\noexpand#2undefined}\@ehc
  }{}%
  \let#2\@undefined
  \expandafter\let\csname2\string#2\endcsname\@undefined
  \expandafter\@@newcommandtwoopt
    \csname2\string#2\endcsname{#1}{#2}%
}
%    \end{macrocode}
%    \end{macro}
%
%    \begin{macro}{\providecommandtwoopt}
%    \begin{macrocode}
\newcommand{\providecommandtwoopt}{%
  \@ifstar{\@providecommandtwoopt*}{\@providecommandtwoopt{}}%
}
%    \end{macrocode}
%    \end{macro}
%
%    \begin{macro}{\@providecommandtwoopt}
%    \begin{param}
%      \entry1 star\\
%      \entry2 command name to be defined
%    \end{param}
%    \begin{macrocode}
\newcommand{\@providecommandtwoopt}{}
\long\def\@providecommandtwoopt#1#2{%
  \begingroup
    \escapechar\m@ne
    \xdef\@gtempa{{\string#2}}%
  \endgroup
  \expandafter\@ifundefined\@gtempa{%
    \expandafter\@@newcommandtwoopt
      \csname2\string#2\endcsname{#1}{#2}%
  }{%
    \let\to@dummyA\@undefined
    \let\to@dummyB\@undefined
    \@@newcommandtwoopt\to@dummyA{#1}\to@dummyB
  }%
}
%    \end{macrocode}
%    \end{macro}
%
%    \begin{macro}{\to@ScanSecondOptArg}
%    \begin{param}
%      \entry1 help command to be defined
%        (\expandafter\cmd\csname 2\bslash<name>\endcsname)\\
%      \entry2 first arg of command to be defined\\
%      \entry3 default for second opt. arg.
%    \end{param}
%    \begin{macrocode}
\newcommand{\to@ScanSecondOptArg}[3]{%
  \@ifnextchar[{%
    \expandafter#1\to@ArgOptToArgArg{#2}%
  }{%
    #1{#2}{#3}%
  }%
}
%    \end{macrocode}
%    \end{macro}
%
%    \begin{macro}{\to@ArgOptToArgArg}
%    \begin{macrocode}
\newcommand{\to@ArgOptToArgArg}{}
\long\def\to@ArgOptToArgArg#1[#2]{{#1}{#2}}
%    \end{macrocode}
%    \end{macro}
%
%    \begin{macrocode}
%</package>
%    \end{macrocode}
%
% \section{Installation}
%
% \subsection{Download}
%
% \paragraph{Package.} This package is available on
% CTAN\footnote{\url{ftp://ftp.ctan.org/tex-archive/}}:
% \begin{description}
% \item[\CTAN{macros/latex/contrib/oberdiek/twoopt.dtx}] The source file.
% \item[\CTAN{macros/latex/contrib/oberdiek/twoopt.pdf}] Documentation.
% \end{description}
%
%
% \paragraph{Bundle.} All the packages of the bundle `oberdiek'
% are also available in a TDS compliant ZIP archive. There
% the packages are already unpacked and the documentation files
% are generated. The files and directories obey the TDS standard.
% \begin{description}
% \item[\CTAN{install/macros/latex/contrib/oberdiek.tds.zip}]
% \end{description}
% \emph{TDS} refers to the standard ``A Directory Structure
% for \TeX\ Files'' (\CTAN{tds/tds.pdf}). Directories
% with \xfile{texmf} in their name are usually organized this way.
%
% \subsection{Bundle installation}
%
% \paragraph{Unpacking.} Unpack the \xfile{oberdiek.tds.zip} in the
% TDS tree (also known as \xfile{texmf} tree) of your choice.
% Example (linux):
% \begin{quote}
%   |unzip oberdiek.tds.zip -d ~/texmf|
% \end{quote}
%
% \paragraph{Script installation.}
% Check the directory \xfile{TDS:scripts/oberdiek/} for
% scripts that need further installation steps.
% Package \xpackage{attachfile2} comes with the Perl script
% \xfile{pdfatfi.pl} that should be installed in such a way
% that it can be called as \texttt{pdfatfi}.
% Example (linux):
% \begin{quote}
%   |chmod +x scripts/oberdiek/pdfatfi.pl|\\
%   |cp scripts/oberdiek/pdfatfi.pl /usr/local/bin/|
% \end{quote}
%
% \subsection{Package installation}
%
% \paragraph{Unpacking.} The \xfile{.dtx} file is a self-extracting
% \docstrip\ archive. The files are extracted by running the
% \xfile{.dtx} through \plainTeX:
% \begin{quote}
%   \verb|tex twoopt.dtx|
% \end{quote}
%
% \paragraph{TDS.} Now the different files must be moved into
% the different directories in your installation TDS tree
% (also known as \xfile{texmf} tree):
% \begin{quote}
% \def\t{^^A
% \begin{tabular}{@{}>{\ttfamily}l@{ $\rightarrow$ }>{\ttfamily}l@{}}
%   twoopt.sty & tex/latex/oberdiek/twoopt.sty\\
%   twoopt.pdf & doc/latex/oberdiek/twoopt.pdf\\
%   twoopt.dtx & source/latex/oberdiek/twoopt.dtx\\
% \end{tabular}^^A
% }^^A
% \sbox0{\t}^^A
% \ifdim\wd0>\linewidth
%   \begingroup
%     \advance\linewidth by\leftmargin
%     \advance\linewidth by\rightmargin
%   \edef\x{\endgroup
%     \def\noexpand\lw{\the\linewidth}^^A
%   }\x
%   \def\lwbox{^^A
%     \leavevmode
%     \hbox to \linewidth{^^A
%       \kern-\leftmargin\relax
%       \hss
%       \usebox0
%       \hss
%       \kern-\rightmargin\relax
%     }^^A
%   }^^A
%   \ifdim\wd0>\lw
%     \sbox0{\small\t}^^A
%     \ifdim\wd0>\linewidth
%       \ifdim\wd0>\lw
%         \sbox0{\footnotesize\t}^^A
%         \ifdim\wd0>\linewidth
%           \ifdim\wd0>\lw
%             \sbox0{\scriptsize\t}^^A
%             \ifdim\wd0>\linewidth
%               \ifdim\wd0>\lw
%                 \sbox0{\tiny\t}^^A
%                 \ifdim\wd0>\linewidth
%                   \lwbox
%                 \else
%                   \usebox0
%                 \fi
%               \else
%                 \lwbox
%               \fi
%             \else
%               \usebox0
%             \fi
%           \else
%             \lwbox
%           \fi
%         \else
%           \usebox0
%         \fi
%       \else
%         \lwbox
%       \fi
%     \else
%       \usebox0
%     \fi
%   \else
%     \lwbox
%   \fi
% \else
%   \usebox0
% \fi
% \end{quote}
% If you have a \xfile{docstrip.cfg} that configures and enables \docstrip's
% TDS installing feature, then some files can already be in the right
% place, see the documentation of \docstrip.
%
% \subsection{Refresh file name databases}
%
% If your \TeX~distribution
% (\teTeX, \mikTeX, \dots) relies on file name databases, you must refresh
% these. For example, \teTeX\ users run \verb|texhash| or
% \verb|mktexlsr|.
%
% \subsection{Some details for the interested}
%
% \paragraph{Attached source.}
%
% The PDF documentation on CTAN also includes the
% \xfile{.dtx} source file. It can be extracted by
% AcrobatReader 6 or higher. Another option is \textsf{pdftk},
% e.g. unpack the file into the current directory:
% \begin{quote}
%   \verb|pdftk twoopt.pdf unpack_files output .|
% \end{quote}
%
% \paragraph{Unpacking with \LaTeX.}
% The \xfile{.dtx} chooses its action depending on the format:
% \begin{description}
% \item[\plainTeX:] Run \docstrip\ and extract the files.
% \item[\LaTeX:] Generate the documentation.
% \end{description}
% If you insist on using \LaTeX\ for \docstrip\ (really,
% \docstrip\ does not need \LaTeX), then inform the autodetect routine
% about your intention:
% \begin{quote}
%   \verb|latex \let\install=y% \iffalse meta-comment
%
% File: twoopt.dtx
% Version: 2016/05/16 v1.6
% Info: Definitions with two optional arguments
%
% Copyright (C) 1999, 2006, 2008 by
%    Heiko Oberdiek <heiko.oberdiek at googlemail.com>
%    2016
%    https://github.com/ho-tex/oberdiek/issues
%
% This work may be distributed and/or modified under the
% conditions of the LaTeX Project Public License, either
% version 1.3c of this license or (at your option) any later
% version. This version of this license is in
%    http://www.latex-project.org/lppl/lppl-1-3c.txt
% and the latest version of this license is in
%    http://www.latex-project.org/lppl.txt
% and version 1.3 or later is part of all distributions of
% LaTeX version 2005/12/01 or later.
%
% This work has the LPPL maintenance status "maintained".
%
% This Current Maintainer of this work is Heiko Oberdiek.
%
% This work consists of the main source file twoopt.dtx
% and the derived files
%    twoopt.sty, twoopt.pdf, twoopt.ins, twoopt.drv.
%
% Distribution:
%    CTAN:macros/latex/contrib/oberdiek/twoopt.dtx
%    CTAN:macros/latex/contrib/oberdiek/twoopt.pdf
%
% Unpacking:
%    (a) If twoopt.ins is present:
%           tex twoopt.ins
%    (b) Without twoopt.ins:
%           tex twoopt.dtx
%    (c) If you insist on using LaTeX
%           latex \let\install=y% \iffalse meta-comment
%
% File: twoopt.dtx
% Version: 2016/05/16 v1.6
% Info: Definitions with two optional arguments
%
% Copyright (C) 1999, 2006, 2008 by
%    Heiko Oberdiek <heiko.oberdiek at googlemail.com>
%    2016
%    https://github.com/ho-tex/oberdiek/issues
%
% This work may be distributed and/or modified under the
% conditions of the LaTeX Project Public License, either
% version 1.3c of this license or (at your option) any later
% version. This version of this license is in
%    http://www.latex-project.org/lppl/lppl-1-3c.txt
% and the latest version of this license is in
%    http://www.latex-project.org/lppl.txt
% and version 1.3 or later is part of all distributions of
% LaTeX version 2005/12/01 or later.
%
% This work has the LPPL maintenance status "maintained".
%
% This Current Maintainer of this work is Heiko Oberdiek.
%
% This work consists of the main source file twoopt.dtx
% and the derived files
%    twoopt.sty, twoopt.pdf, twoopt.ins, twoopt.drv.
%
% Distribution:
%    CTAN:macros/latex/contrib/oberdiek/twoopt.dtx
%    CTAN:macros/latex/contrib/oberdiek/twoopt.pdf
%
% Unpacking:
%    (a) If twoopt.ins is present:
%           tex twoopt.ins
%    (b) Without twoopt.ins:
%           tex twoopt.dtx
%    (c) If you insist on using LaTeX
%           latex \let\install=y% \iffalse meta-comment
%
% File: twoopt.dtx
% Version: 2016/05/16 v1.6
% Info: Definitions with two optional arguments
%
% Copyright (C) 1999, 2006, 2008 by
%    Heiko Oberdiek <heiko.oberdiek at googlemail.com>
%    2016
%    https://github.com/ho-tex/oberdiek/issues
%
% This work may be distributed and/or modified under the
% conditions of the LaTeX Project Public License, either
% version 1.3c of this license or (at your option) any later
% version. This version of this license is in
%    http://www.latex-project.org/lppl/lppl-1-3c.txt
% and the latest version of this license is in
%    http://www.latex-project.org/lppl.txt
% and version 1.3 or later is part of all distributions of
% LaTeX version 2005/12/01 or later.
%
% This work has the LPPL maintenance status "maintained".
%
% This Current Maintainer of this work is Heiko Oberdiek.
%
% This work consists of the main source file twoopt.dtx
% and the derived files
%    twoopt.sty, twoopt.pdf, twoopt.ins, twoopt.drv.
%
% Distribution:
%    CTAN:macros/latex/contrib/oberdiek/twoopt.dtx
%    CTAN:macros/latex/contrib/oberdiek/twoopt.pdf
%
% Unpacking:
%    (a) If twoopt.ins is present:
%           tex twoopt.ins
%    (b) Without twoopt.ins:
%           tex twoopt.dtx
%    (c) If you insist on using LaTeX
%           latex \let\install=y\input{twoopt.dtx}
%        (quote the arguments according to the demands of your shell)
%
% Documentation:
%    (a) If twoopt.drv is present:
%           latex twoopt.drv
%    (b) Without twoopt.drv:
%           latex twoopt.dtx; ...
%    The class ltxdoc loads the configuration file ltxdoc.cfg
%    if available. Here you can specify further options, e.g.
%    use A4 as paper format:
%       \PassOptionsToClass{a4paper}{article}
%
%    Programm calls to get the documentation (example):
%       pdflatex twoopt.dtx
%       makeindex -s gind.ist twoopt.idx
%       pdflatex twoopt.dtx
%       makeindex -s gind.ist twoopt.idx
%       pdflatex twoopt.dtx
%
% Installation:
%    TDS:tex/latex/oberdiek/twoopt.sty
%    TDS:doc/latex/oberdiek/twoopt.pdf
%    TDS:source/latex/oberdiek/twoopt.dtx
%
%<*ignore>
\begingroup
  \catcode123=1 %
  \catcode125=2 %
  \def\x{LaTeX2e}%
\expandafter\endgroup
\ifcase 0\ifx\install y1\fi\expandafter
         \ifx\csname processbatchFile\endcsname\relax\else1\fi
         \ifx\fmtname\x\else 1\fi\relax
\else\csname fi\endcsname
%</ignore>
%<*install>
\input docstrip.tex
\Msg{************************************************************************}
\Msg{* Installation}
\Msg{* Package: twoopt 2016/05/16 v1.6 Definitions with two optional arguments (HO)}
\Msg{************************************************************************}

\keepsilent
\askforoverwritefalse

\let\MetaPrefix\relax
\preamble

This is a generated file.

Project: twoopt
Version: 2016/05/16 v1.6

Copyright (C) 1999, 2006, 2008 by
   Heiko Oberdiek <heiko.oberdiek at googlemail.com>

This work may be distributed and/or modified under the
conditions of the LaTeX Project Public License, either
version 1.3c of this license or (at your option) any later
version. This version of this license is in
   http://www.latex-project.org/lppl/lppl-1-3c.txt
and the latest version of this license is in
   http://www.latex-project.org/lppl.txt
and version 1.3 or later is part of all distributions of
LaTeX version 2005/12/01 or later.

This work has the LPPL maintenance status "maintained".

This Current Maintainer of this work is Heiko Oberdiek.

This work consists of the main source file twoopt.dtx
and the derived files
   twoopt.sty, twoopt.pdf, twoopt.ins, twoopt.drv.

\endpreamble
\let\MetaPrefix\DoubleperCent

\generate{%
  \file{twoopt.ins}{\from{twoopt.dtx}{install}}%
  \file{twoopt.drv}{\from{twoopt.dtx}{driver}}%
  \usedir{tex/latex/oberdiek}%
  \file{twoopt.sty}{\from{twoopt.dtx}{package}}%
  \nopreamble
  \nopostamble
%  \usedir{source/latex/oberdiek/catalogue}%
%  \file{twoopt.xml}{\from{twoopt.dtx}{catalogue}}%
}

\catcode32=13\relax% active space
\let =\space%
\Msg{************************************************************************}
\Msg{*}
\Msg{* To finish the installation you have to move the following}
\Msg{* file into a directory searched by TeX:}
\Msg{*}
\Msg{*     twoopt.sty}
\Msg{*}
\Msg{* To produce the documentation run the file `twoopt.drv'}
\Msg{* through LaTeX.}
\Msg{*}
\Msg{* Happy TeXing!}
\Msg{*}
\Msg{************************************************************************}

\endbatchfile
%</install>
%<*ignore>
\fi
%</ignore>
%<*driver>
\NeedsTeXFormat{LaTeX2e}
\ProvidesFile{twoopt.drv}%
  [2016/05/16 v1.6 Definitions with two optional arguments (HO)]%
\documentclass{ltxdoc}
\usepackage{holtxdoc}[2011/11/22]
\begin{document}
  \DocInput{twoopt.dtx}%
\end{document}
%</driver>
% \fi
%
%
% \CharacterTable
%  {Upper-case    \A\B\C\D\E\F\G\H\I\J\K\L\M\N\O\P\Q\R\S\T\U\V\W\X\Y\Z
%   Lower-case    \a\b\c\d\e\f\g\h\i\j\k\l\m\n\o\p\q\r\s\t\u\v\w\x\y\z
%   Digits        \0\1\2\3\4\5\6\7\8\9
%   Exclamation   \!     Double quote  \"     Hash (number) \#
%   Dollar        \$     Percent       \%     Ampersand     \&
%   Acute accent  \'     Left paren    \(     Right paren   \)
%   Asterisk      \*     Plus          \+     Comma         \,
%   Minus         \-     Point         \.     Solidus       \/
%   Colon         \:     Semicolon     \;     Less than     \<
%   Equals        \=     Greater than  \>     Question mark \?
%   Commercial at \@     Left bracket  \[     Backslash     \\
%   Right bracket \]     Circumflex    \^     Underscore    \_
%   Grave accent  \`     Left brace    \{     Vertical bar  \|
%   Right brace   \}     Tilde         \~}
%
% \GetFileInfo{twoopt.drv}
%
% \title{The \xpackage{twoopt} package}
% \date{2016/05/16 v1.6}
% \author{Heiko Oberdiek\thanks
% {Please report any issues at https://github.com/ho-tex/oberdiek/issues}\\
% \xemail{heiko.oberdiek at googlemail.com}}
%
% \maketitle
%
% \begin{abstract}
% This package provides commands to define macros with two
% optional arguments.
% \end{abstract}
%
% \tableofcontents
%
% \newenvironment{param}{^^A
%   \newcommand{\entry}[1]{\meta{\###1}:&}^^A
%   \begin{tabular}[t]{@{}l@{ }l@{}}^^A
% }{^^A
%   \end{tabular}^^A
% }
%
% \section{Usage}
%    \DescribeMacro{\newcommandtwoopt}
%    \DescribeMacro{\renewcommandtwoopt}
%    \DescribeMacro{\providecommandtwoopt}
%    Similar to \cmd{\newcommand}, \cmd{\renewcommand}
%    and \cmd{\providecommand} this package provides commands
%    to define macros with two optional arguments.
%    The names of the commands are built by appending the
%    package name to the \LaTeX-pendants:
%    \begingroup
%      \def\x{\marg{cmd} \oarg{num} \oarg{default1}^^A
%             \oarg{default2} \marg{def.}}^^A
%      \begin{tabbing}
%        \cmd{\providecommandtwoopt} \=\kill
%        \cmd{\newcommandtwoopt}\>\x\\
%        \cmd{\renewcommandtwoopt}\>\x\\
%        \cmd{\providecommandtwoopt}\>\x\\
%      \end{tabbing}
%    \endgroup
%
%    Also the |*|-forms are supported. Indeed it is better to
%    use this ones, unless it is intended to hold
%    whole paragraphs in some of the arguments. If the macro
%    is defined with the |*|-form, missing braces
%    can be detected earlier.
%
%    Example:
%    \begin{quote}
%      |\newcommandtwoopt{\bsp}[3][AA][BB]{%|\\
%      |  \typeout{\string\bsp: #1,#2,#3}%|\\
%      |}|\\
%      \begin{tabular}{@{}l@{\quad$\rightarrow$\quad}l@{}}
%      |\bsp[aa][bb]{cc}|&|\bsp: aa,bb,cc|\\
%      |\bsp[aa]{cc}|&|\bsp: aa,BB,cc|\\
%      |\bsp{cc}|&|\bsp: AA,BB,cc|\\
%      \end{tabular}
%    \end{quote}
%
% \StopEventually{
% }
%
% \section{Implementation}
%    \begin{macrocode}
%<*package>
\NeedsTeXFormat{LaTeX2e}
\ProvidesPackage{twoopt}
  [2016/05/16 v1.6 Definitions with two optional arguments (HO)]%
%    \end{macrocode}
%    \begin{macro}{\newcommandtwoopt}
%    \begin{macrocode}
\newcommand{\newcommandtwoopt}{%
  \@ifstar{\@newcommandtwoopt*}{\@newcommandtwoopt{}}%
}
%    \end{macrocode}
%    \end{macro}
%
%    \begin{macro}{\@newcommandtwoopt}
%    \begin{param}
%      \entry1 star\\
%      \entry2 macro name to be defined
%    \end{param}
%    \begin{macrocode}
\newcommand{\@newcommandtwoopt}{}
\long\def\@newcommandtwoopt#1#2{%
  \expandafter\@@newcommandtwoopt
    \csname2\string#2\endcsname{#1}{#2}%
}
%    \end{macrocode}
%    \end{macro}
%
%    \begin{macro}{\@@newcommandtwoopt}
%    \begin{param}
%      \entry1 help command to be defined
%        (\expandafter\cmd\csname 2\bslash<name>\endcsname)\\
%      \entry2 star\\
%      \entry3 macro name to be defined\\
%      \entry4 number of total arguments\\
%      \entry5 default for optional argument one\\
%      \entry6 default for optional argument two
%    \end{param}
%    \begin{macrocode}
\newcommand{\@@newcommandtwoopt}{}
\long\def\@@newcommandtwoopt#1#2#3[#4][#5][#6]{%
  \newcommand#2#3[1][{#5}]{%
    \to@ScanSecondOptArg#1{##1}{#6}%
  }%
  \newcommand#2#1[{#4}]%
}
%    \end{macrocode}
%    \end{macro}
%
%    \begin{macro}{\renewcommandtwoopt}
%    \begin{macrocode}
\newcommand{\renewcommandtwoopt}{%
  \@ifstar{\@renewcommandtwoopt*}{\@renewcommandtwoopt{}}%
}
%    \end{macrocode}
%    \end{macro}
%
%    \begin{macro}{\@renewcommandtwoopt}
%    \begin{param}
%      \entry1 star\\
%      \entry2 command name to be defined
%    \end{param}
%    \begin{macrocode}
\newcommand{\@renewcommandtwoopt}{}
\long\def\@renewcommandtwoopt#1#2{%
  \begingroup
    \escapechar\m@ne
    \xdef\@gtempa{{\string#2}}%
  \endgroup
  \expandafter\@ifundefined\@gtempa{%
    \@latex@error{\noexpand#2undefined}\@ehc
  }{}%
  \let#2\@undefined
  \expandafter\let\csname2\string#2\endcsname\@undefined
  \expandafter\@@newcommandtwoopt
    \csname2\string#2\endcsname{#1}{#2}%
}
%    \end{macrocode}
%    \end{macro}
%
%    \begin{macro}{\providecommandtwoopt}
%    \begin{macrocode}
\newcommand{\providecommandtwoopt}{%
  \@ifstar{\@providecommandtwoopt*}{\@providecommandtwoopt{}}%
}
%    \end{macrocode}
%    \end{macro}
%
%    \begin{macro}{\@providecommandtwoopt}
%    \begin{param}
%      \entry1 star\\
%      \entry2 command name to be defined
%    \end{param}
%    \begin{macrocode}
\newcommand{\@providecommandtwoopt}{}
\long\def\@providecommandtwoopt#1#2{%
  \begingroup
    \escapechar\m@ne
    \xdef\@gtempa{{\string#2}}%
  \endgroup
  \expandafter\@ifundefined\@gtempa{%
    \expandafter\@@newcommandtwoopt
      \csname2\string#2\endcsname{#1}{#2}%
  }{%
    \let\to@dummyA\@undefined
    \let\to@dummyB\@undefined
    \@@newcommandtwoopt\to@dummyA{#1}\to@dummyB
  }%
}
%    \end{macrocode}
%    \end{macro}
%
%    \begin{macro}{\to@ScanSecondOptArg}
%    \begin{param}
%      \entry1 help command to be defined
%        (\expandafter\cmd\csname 2\bslash<name>\endcsname)\\
%      \entry2 first arg of command to be defined\\
%      \entry3 default for second opt. arg.
%    \end{param}
%    \begin{macrocode}
\newcommand{\to@ScanSecondOptArg}[3]{%
  \@ifnextchar[{%
    \expandafter#1\to@ArgOptToArgArg{#2}%
  }{%
    #1{#2}{#3}%
  }%
}
%    \end{macrocode}
%    \end{macro}
%
%    \begin{macro}{\to@ArgOptToArgArg}
%    \begin{macrocode}
\newcommand{\to@ArgOptToArgArg}{}
\long\def\to@ArgOptToArgArg#1[#2]{{#1}{#2}}
%    \end{macrocode}
%    \end{macro}
%
%    \begin{macrocode}
%</package>
%    \end{macrocode}
%
% \section{Installation}
%
% \subsection{Download}
%
% \paragraph{Package.} This package is available on
% CTAN\footnote{\url{http://ctan.org/pkg/twoopt}}:
% \begin{description}
% \item[\CTAN{macros/latex/contrib/oberdiek/twoopt.dtx}] The source file.
% \item[\CTAN{macros/latex/contrib/oberdiek/twoopt.pdf}] Documentation.
% \end{description}
%
%
% \paragraph{Bundle.} All the packages of the bundle `oberdiek'
% are also available in a TDS compliant ZIP archive. There
% the packages are already unpacked and the documentation files
% are generated. The files and directories obey the TDS standard.
% \begin{description}
% \item[\CTAN{install/macros/latex/contrib/oberdiek.tds.zip}]
% \end{description}
% \emph{TDS} refers to the standard ``A Directory Structure
% for \TeX\ Files'' (\CTAN{tds/tds.pdf}). Directories
% with \xfile{texmf} in their name are usually organized this way.
%
% \subsection{Bundle installation}
%
% \paragraph{Unpacking.} Unpack the \xfile{oberdiek.tds.zip} in the
% TDS tree (also known as \xfile{texmf} tree) of your choice.
% Example (linux):
% \begin{quote}
%   |unzip oberdiek.tds.zip -d ~/texmf|
% \end{quote}
%
% \paragraph{Script installation.}
% Check the directory \xfile{TDS:scripts/oberdiek/} for
% scripts that need further installation steps.
% Package \xpackage{attachfile2} comes with the Perl script
% \xfile{pdfatfi.pl} that should be installed in such a way
% that it can be called as \texttt{pdfatfi}.
% Example (linux):
% \begin{quote}
%   |chmod +x scripts/oberdiek/pdfatfi.pl|\\
%   |cp scripts/oberdiek/pdfatfi.pl /usr/local/bin/|
% \end{quote}
%
% \subsection{Package installation}
%
% \paragraph{Unpacking.} The \xfile{.dtx} file is a self-extracting
% \docstrip\ archive. The files are extracted by running the
% \xfile{.dtx} through \plainTeX:
% \begin{quote}
%   \verb|tex twoopt.dtx|
% \end{quote}
%
% \paragraph{TDS.} Now the different files must be moved into
% the different directories in your installation TDS tree
% (also known as \xfile{texmf} tree):
% \begin{quote}
% \def\t{^^A
% \begin{tabular}{@{}>{\ttfamily}l@{ $\rightarrow$ }>{\ttfamily}l@{}}
%   twoopt.sty & tex/latex/oberdiek/twoopt.sty\\
%   twoopt.pdf & doc/latex/oberdiek/twoopt.pdf\\
%   twoopt.dtx & source/latex/oberdiek/twoopt.dtx\\
% \end{tabular}^^A
% }^^A
% \sbox0{\t}^^A
% \ifdim\wd0>\linewidth
%   \begingroup
%     \advance\linewidth by\leftmargin
%     \advance\linewidth by\rightmargin
%   \edef\x{\endgroup
%     \def\noexpand\lw{\the\linewidth}^^A
%   }\x
%   \def\lwbox{^^A
%     \leavevmode
%     \hbox to \linewidth{^^A
%       \kern-\leftmargin\relax
%       \hss
%       \usebox0
%       \hss
%       \kern-\rightmargin\relax
%     }^^A
%   }^^A
%   \ifdim\wd0>\lw
%     \sbox0{\small\t}^^A
%     \ifdim\wd0>\linewidth
%       \ifdim\wd0>\lw
%         \sbox0{\footnotesize\t}^^A
%         \ifdim\wd0>\linewidth
%           \ifdim\wd0>\lw
%             \sbox0{\scriptsize\t}^^A
%             \ifdim\wd0>\linewidth
%               \ifdim\wd0>\lw
%                 \sbox0{\tiny\t}^^A
%                 \ifdim\wd0>\linewidth
%                   \lwbox
%                 \else
%                   \usebox0
%                 \fi
%               \else
%                 \lwbox
%               \fi
%             \else
%               \usebox0
%             \fi
%           \else
%             \lwbox
%           \fi
%         \else
%           \usebox0
%         \fi
%       \else
%         \lwbox
%       \fi
%     \else
%       \usebox0
%     \fi
%   \else
%     \lwbox
%   \fi
% \else
%   \usebox0
% \fi
% \end{quote}
% If you have a \xfile{docstrip.cfg} that configures and enables \docstrip's
% TDS installing feature, then some files can already be in the right
% place, see the documentation of \docstrip.
%
% \subsection{Refresh file name databases}
%
% If your \TeX~distribution
% (\teTeX, \mikTeX, \dots) relies on file name databases, you must refresh
% these. For example, \teTeX\ users run \verb|texhash| or
% \verb|mktexlsr|.
%
% \subsection{Some details for the interested}
%
% \paragraph{Attached source.}
%
% The PDF documentation on CTAN also includes the
% \xfile{.dtx} source file. It can be extracted by
% AcrobatReader 6 or higher. Another option is \textsf{pdftk},
% e.g. unpack the file into the current directory:
% \begin{quote}
%   \verb|pdftk twoopt.pdf unpack_files output .|
% \end{quote}
%
% \paragraph{Unpacking with \LaTeX.}
% The \xfile{.dtx} chooses its action depending on the format:
% \begin{description}
% \item[\plainTeX:] Run \docstrip\ and extract the files.
% \item[\LaTeX:] Generate the documentation.
% \end{description}
% If you insist on using \LaTeX\ for \docstrip\ (really,
% \docstrip\ does not need \LaTeX), then inform the autodetect routine
% about your intention:
% \begin{quote}
%   \verb|latex \let\install=y\input{twoopt.dtx}|
% \end{quote}
% Do not forget to quote the argument according to the demands
% of your shell.
%
% \paragraph{Generating the documentation.}
% You can use both the \xfile{.dtx} or the \xfile{.drv} to generate
% the documentation. The process can be configured by the
% configuration file \xfile{ltxdoc.cfg}. For instance, put this
% line into this file, if you want to have A4 as paper format:
% \begin{quote}
%   \verb|\PassOptionsToClass{a4paper}{article}|
% \end{quote}
% An example follows how to generate the
% documentation with pdf\LaTeX:
% \begin{quote}
%\begin{verbatim}
%pdflatex twoopt.dtx
%makeindex -s gind.ist twoopt.idx
%pdflatex twoopt.dtx
%makeindex -s gind.ist twoopt.idx
%pdflatex twoopt.dtx
%\end{verbatim}
% \end{quote}
%
% \section{Catalogue}
%
% The following XML file can be used as source for the
% \href{http://mirror.ctan.org/help/Catalogue/catalogue.html}{\TeX\ Catalogue}.
% The elements \texttt{caption} and \texttt{description} are imported
% from the original XML file from the Catalogue.
% The name of the XML file in the Catalogue is \xfile{twoopt.xml}.
%    \begin{macrocode}
%<*catalogue>
<?xml version='1.0' encoding='us-ascii'?>
<!DOCTYPE entry SYSTEM 'catalogue.dtd'>
<entry datestamp='$Date$' modifier='$Author$' id='twoopt'>
  <name>twoopt</name>
  <caption>Definitions with two optional arguments.</caption>
  <authorref id='auth:oberdiek'/>
  <copyright owner='Heiko Oberdiek' year='1999,2006,2008'/>
  <license type='lppl1.3'/>
  <version number='1.6'/>
  <description>
    Variants of <tt>\newcommand</tt>, <tt>\renewcommand</tt> and
    <tt>\providecommand</tt> are provided.
    <p/>
    The package is part of the <xref refid='oberdiek'>oberdiek</xref>
    bundle.
  </description>
  <documentation details='Package documentation'
      href='ctan:/macros/latex/contrib/oberdiek/twoopt.pdf'/>
  <ctan file='true' path='/macros/latex/contrib/oberdiek/twoopt.dtx'/>
  <miktex location='oberdiek'/>
  <texlive location='oberdiek'/>
  <install path='/macros/latex/contrib/oberdiek/oberdiek.tds.zip'/>
</entry>
%</catalogue>
%    \end{macrocode}
%
% \begin{History}
%   \begin{Version}{1998/10/30 v1.0}
%   \item
%     The first version was built as a response to a question
%     of \NameEmail{Rebecca and Rowland}{rebecca@astrid.u-net.com},
%     published in the newsgroup
%     \href{news:comp.text.tex}{comp.text.tex}:\\
%     \URL{``Re: [Q] LaTeX command with two optional arguments?''}^^A
%     {http://groups.google.com/group/comp.text.tex/msg/0ab1afde7b172d37}
%   \end{Version}
%   \begin{Version}{1998/10/30 v1.1}
%   \item
%     Improvements added in response to
%     \NameEmail{Stefan Ulrich}{ulrich@cis.uni-muenchen.de}
%     in the same thread:\\
%     \URL{``Re: [Q] LaTeX command with two optional arguments?''}^^A
%     {http://groups.google.com/group/comp.text.tex/msg/b8d84d4336f302c4}
%   \end{Version}
%   \begin{Version}{1998/11/04 v1.2}
%   \item
%     Fixes for LaTeX bugs 2896, 2901, 2902 added.
%   \end{Version}
%   \begin{Version}{1999/04/12 v1.3}
%   \item
%     Fixes removed because of LaTeX [1998/12/01].
%   \item
%     Documentation in dtx format.
%   \item
%     Copyright: LPPL (\CTAN{macros/latex/base/lppl.txt})
%   \item
%     First CTAN release.
%   \end{Version}
%   \begin{Version}{2006/02/20 v1.4}
%   \item
%     Code is not changed.
%   \item
%     New DTX framework.
%   \item
%     LPPL 1.3
%   \end{Version}
%   \begin{Version}{2008/08/11 v1.5}
%   \item
%     Code is not changed.
%   \item
%     URLs updated from \texttt{www.dejanews.com}
%     to \texttt{groups.google.com}.
%   \end{Version}
%   \begin{Version}{2016/05/16 v1.6}
%   \item
%     Documentation updates.
%   \end{Version}
% \end{History}
%
% \PrintIndex
%
% \Finale
\endinput

%        (quote the arguments according to the demands of your shell)
%
% Documentation:
%    (a) If twoopt.drv is present:
%           latex twoopt.drv
%    (b) Without twoopt.drv:
%           latex twoopt.dtx; ...
%    The class ltxdoc loads the configuration file ltxdoc.cfg
%    if available. Here you can specify further options, e.g.
%    use A4 as paper format:
%       \PassOptionsToClass{a4paper}{article}
%
%    Programm calls to get the documentation (example):
%       pdflatex twoopt.dtx
%       makeindex -s gind.ist twoopt.idx
%       pdflatex twoopt.dtx
%       makeindex -s gind.ist twoopt.idx
%       pdflatex twoopt.dtx
%
% Installation:
%    TDS:tex/latex/oberdiek/twoopt.sty
%    TDS:doc/latex/oberdiek/twoopt.pdf
%    TDS:source/latex/oberdiek/twoopt.dtx
%
%<*ignore>
\begingroup
  \catcode123=1 %
  \catcode125=2 %
  \def\x{LaTeX2e}%
\expandafter\endgroup
\ifcase 0\ifx\install y1\fi\expandafter
         \ifx\csname processbatchFile\endcsname\relax\else1\fi
         \ifx\fmtname\x\else 1\fi\relax
\else\csname fi\endcsname
%</ignore>
%<*install>
\input docstrip.tex
\Msg{************************************************************************}
\Msg{* Installation}
\Msg{* Package: twoopt 2016/05/16 v1.6 Definitions with two optional arguments (HO)}
\Msg{************************************************************************}

\keepsilent
\askforoverwritefalse

\let\MetaPrefix\relax
\preamble

This is a generated file.

Project: twoopt
Version: 2016/05/16 v1.6

Copyright (C) 1999, 2006, 2008 by
   Heiko Oberdiek <heiko.oberdiek at googlemail.com>

This work may be distributed and/or modified under the
conditions of the LaTeX Project Public License, either
version 1.3c of this license or (at your option) any later
version. This version of this license is in
   http://www.latex-project.org/lppl/lppl-1-3c.txt
and the latest version of this license is in
   http://www.latex-project.org/lppl.txt
and version 1.3 or later is part of all distributions of
LaTeX version 2005/12/01 or later.

This work has the LPPL maintenance status "maintained".

This Current Maintainer of this work is Heiko Oberdiek.

This work consists of the main source file twoopt.dtx
and the derived files
   twoopt.sty, twoopt.pdf, twoopt.ins, twoopt.drv.

\endpreamble
\let\MetaPrefix\DoubleperCent

\generate{%
  \file{twoopt.ins}{\from{twoopt.dtx}{install}}%
  \file{twoopt.drv}{\from{twoopt.dtx}{driver}}%
  \usedir{tex/latex/oberdiek}%
  \file{twoopt.sty}{\from{twoopt.dtx}{package}}%
  \nopreamble
  \nopostamble
%  \usedir{source/latex/oberdiek/catalogue}%
%  \file{twoopt.xml}{\from{twoopt.dtx}{catalogue}}%
}

\catcode32=13\relax% active space
\let =\space%
\Msg{************************************************************************}
\Msg{*}
\Msg{* To finish the installation you have to move the following}
\Msg{* file into a directory searched by TeX:}
\Msg{*}
\Msg{*     twoopt.sty}
\Msg{*}
\Msg{* To produce the documentation run the file `twoopt.drv'}
\Msg{* through LaTeX.}
\Msg{*}
\Msg{* Happy TeXing!}
\Msg{*}
\Msg{************************************************************************}

\endbatchfile
%</install>
%<*ignore>
\fi
%</ignore>
%<*driver>
\NeedsTeXFormat{LaTeX2e}
\ProvidesFile{twoopt.drv}%
  [2016/05/16 v1.6 Definitions with two optional arguments (HO)]%
\documentclass{ltxdoc}
\usepackage{holtxdoc}[2011/11/22]
\begin{document}
  \DocInput{twoopt.dtx}%
\end{document}
%</driver>
% \fi
%
%
% \CharacterTable
%  {Upper-case    \A\B\C\D\E\F\G\H\I\J\K\L\M\N\O\P\Q\R\S\T\U\V\W\X\Y\Z
%   Lower-case    \a\b\c\d\e\f\g\h\i\j\k\l\m\n\o\p\q\r\s\t\u\v\w\x\y\z
%   Digits        \0\1\2\3\4\5\6\7\8\9
%   Exclamation   \!     Double quote  \"     Hash (number) \#
%   Dollar        \$     Percent       \%     Ampersand     \&
%   Acute accent  \'     Left paren    \(     Right paren   \)
%   Asterisk      \*     Plus          \+     Comma         \,
%   Minus         \-     Point         \.     Solidus       \/
%   Colon         \:     Semicolon     \;     Less than     \<
%   Equals        \=     Greater than  \>     Question mark \?
%   Commercial at \@     Left bracket  \[     Backslash     \\
%   Right bracket \]     Circumflex    \^     Underscore    \_
%   Grave accent  \`     Left brace    \{     Vertical bar  \|
%   Right brace   \}     Tilde         \~}
%
% \GetFileInfo{twoopt.drv}
%
% \title{The \xpackage{twoopt} package}
% \date{2016/05/16 v1.6}
% \author{Heiko Oberdiek\thanks
% {Please report any issues at https://github.com/ho-tex/oberdiek/issues}\\
% \xemail{heiko.oberdiek at googlemail.com}}
%
% \maketitle
%
% \begin{abstract}
% This package provides commands to define macros with two
% optional arguments.
% \end{abstract}
%
% \tableofcontents
%
% \newenvironment{param}{^^A
%   \newcommand{\entry}[1]{\meta{\###1}:&}^^A
%   \begin{tabular}[t]{@{}l@{ }l@{}}^^A
% }{^^A
%   \end{tabular}^^A
% }
%
% \section{Usage}
%    \DescribeMacro{\newcommandtwoopt}
%    \DescribeMacro{\renewcommandtwoopt}
%    \DescribeMacro{\providecommandtwoopt}
%    Similar to \cmd{\newcommand}, \cmd{\renewcommand}
%    and \cmd{\providecommand} this package provides commands
%    to define macros with two optional arguments.
%    The names of the commands are built by appending the
%    package name to the \LaTeX-pendants:
%    \begingroup
%      \def\x{\marg{cmd} \oarg{num} \oarg{default1}^^A
%             \oarg{default2} \marg{def.}}^^A
%      \begin{tabbing}
%        \cmd{\providecommandtwoopt} \=\kill
%        \cmd{\newcommandtwoopt}\>\x\\
%        \cmd{\renewcommandtwoopt}\>\x\\
%        \cmd{\providecommandtwoopt}\>\x\\
%      \end{tabbing}
%    \endgroup
%
%    Also the |*|-forms are supported. Indeed it is better to
%    use this ones, unless it is intended to hold
%    whole paragraphs in some of the arguments. If the macro
%    is defined with the |*|-form, missing braces
%    can be detected earlier.
%
%    Example:
%    \begin{quote}
%      |\newcommandtwoopt{\bsp}[3][AA][BB]{%|\\
%      |  \typeout{\string\bsp: #1,#2,#3}%|\\
%      |}|\\
%      \begin{tabular}{@{}l@{\quad$\rightarrow$\quad}l@{}}
%      |\bsp[aa][bb]{cc}|&|\bsp: aa,bb,cc|\\
%      |\bsp[aa]{cc}|&|\bsp: aa,BB,cc|\\
%      |\bsp{cc}|&|\bsp: AA,BB,cc|\\
%      \end{tabular}
%    \end{quote}
%
% \StopEventually{
% }
%
% \section{Implementation}
%    \begin{macrocode}
%<*package>
\NeedsTeXFormat{LaTeX2e}
\ProvidesPackage{twoopt}
  [2016/05/16 v1.6 Definitions with two optional arguments (HO)]%
%    \end{macrocode}
%    \begin{macro}{\newcommandtwoopt}
%    \begin{macrocode}
\newcommand{\newcommandtwoopt}{%
  \@ifstar{\@newcommandtwoopt*}{\@newcommandtwoopt{}}%
}
%    \end{macrocode}
%    \end{macro}
%
%    \begin{macro}{\@newcommandtwoopt}
%    \begin{param}
%      \entry1 star\\
%      \entry2 macro name to be defined
%    \end{param}
%    \begin{macrocode}
\newcommand{\@newcommandtwoopt}{}
\long\def\@newcommandtwoopt#1#2{%
  \expandafter\@@newcommandtwoopt
    \csname2\string#2\endcsname{#1}{#2}%
}
%    \end{macrocode}
%    \end{macro}
%
%    \begin{macro}{\@@newcommandtwoopt}
%    \begin{param}
%      \entry1 help command to be defined
%        (\expandafter\cmd\csname 2\bslash<name>\endcsname)\\
%      \entry2 star\\
%      \entry3 macro name to be defined\\
%      \entry4 number of total arguments\\
%      \entry5 default for optional argument one\\
%      \entry6 default for optional argument two
%    \end{param}
%    \begin{macrocode}
\newcommand{\@@newcommandtwoopt}{}
\long\def\@@newcommandtwoopt#1#2#3[#4][#5][#6]{%
  \newcommand#2#3[1][{#5}]{%
    \to@ScanSecondOptArg#1{##1}{#6}%
  }%
  \newcommand#2#1[{#4}]%
}
%    \end{macrocode}
%    \end{macro}
%
%    \begin{macro}{\renewcommandtwoopt}
%    \begin{macrocode}
\newcommand{\renewcommandtwoopt}{%
  \@ifstar{\@renewcommandtwoopt*}{\@renewcommandtwoopt{}}%
}
%    \end{macrocode}
%    \end{macro}
%
%    \begin{macro}{\@renewcommandtwoopt}
%    \begin{param}
%      \entry1 star\\
%      \entry2 command name to be defined
%    \end{param}
%    \begin{macrocode}
\newcommand{\@renewcommandtwoopt}{}
\long\def\@renewcommandtwoopt#1#2{%
  \begingroup
    \escapechar\m@ne
    \xdef\@gtempa{{\string#2}}%
  \endgroup
  \expandafter\@ifundefined\@gtempa{%
    \@latex@error{\noexpand#2undefined}\@ehc
  }{}%
  \let#2\@undefined
  \expandafter\let\csname2\string#2\endcsname\@undefined
  \expandafter\@@newcommandtwoopt
    \csname2\string#2\endcsname{#1}{#2}%
}
%    \end{macrocode}
%    \end{macro}
%
%    \begin{macro}{\providecommandtwoopt}
%    \begin{macrocode}
\newcommand{\providecommandtwoopt}{%
  \@ifstar{\@providecommandtwoopt*}{\@providecommandtwoopt{}}%
}
%    \end{macrocode}
%    \end{macro}
%
%    \begin{macro}{\@providecommandtwoopt}
%    \begin{param}
%      \entry1 star\\
%      \entry2 command name to be defined
%    \end{param}
%    \begin{macrocode}
\newcommand{\@providecommandtwoopt}{}
\long\def\@providecommandtwoopt#1#2{%
  \begingroup
    \escapechar\m@ne
    \xdef\@gtempa{{\string#2}}%
  \endgroup
  \expandafter\@ifundefined\@gtempa{%
    \expandafter\@@newcommandtwoopt
      \csname2\string#2\endcsname{#1}{#2}%
  }{%
    \let\to@dummyA\@undefined
    \let\to@dummyB\@undefined
    \@@newcommandtwoopt\to@dummyA{#1}\to@dummyB
  }%
}
%    \end{macrocode}
%    \end{macro}
%
%    \begin{macro}{\to@ScanSecondOptArg}
%    \begin{param}
%      \entry1 help command to be defined
%        (\expandafter\cmd\csname 2\bslash<name>\endcsname)\\
%      \entry2 first arg of command to be defined\\
%      \entry3 default for second opt. arg.
%    \end{param}
%    \begin{macrocode}
\newcommand{\to@ScanSecondOptArg}[3]{%
  \@ifnextchar[{%
    \expandafter#1\to@ArgOptToArgArg{#2}%
  }{%
    #1{#2}{#3}%
  }%
}
%    \end{macrocode}
%    \end{macro}
%
%    \begin{macro}{\to@ArgOptToArgArg}
%    \begin{macrocode}
\newcommand{\to@ArgOptToArgArg}{}
\long\def\to@ArgOptToArgArg#1[#2]{{#1}{#2}}
%    \end{macrocode}
%    \end{macro}
%
%    \begin{macrocode}
%</package>
%    \end{macrocode}
%
% \section{Installation}
%
% \subsection{Download}
%
% \paragraph{Package.} This package is available on
% CTAN\footnote{\url{http://ctan.org/pkg/twoopt}}:
% \begin{description}
% \item[\CTAN{macros/latex/contrib/oberdiek/twoopt.dtx}] The source file.
% \item[\CTAN{macros/latex/contrib/oberdiek/twoopt.pdf}] Documentation.
% \end{description}
%
%
% \paragraph{Bundle.} All the packages of the bundle `oberdiek'
% are also available in a TDS compliant ZIP archive. There
% the packages are already unpacked and the documentation files
% are generated. The files and directories obey the TDS standard.
% \begin{description}
% \item[\CTAN{install/macros/latex/contrib/oberdiek.tds.zip}]
% \end{description}
% \emph{TDS} refers to the standard ``A Directory Structure
% for \TeX\ Files'' (\CTAN{tds/tds.pdf}). Directories
% with \xfile{texmf} in their name are usually organized this way.
%
% \subsection{Bundle installation}
%
% \paragraph{Unpacking.} Unpack the \xfile{oberdiek.tds.zip} in the
% TDS tree (also known as \xfile{texmf} tree) of your choice.
% Example (linux):
% \begin{quote}
%   |unzip oberdiek.tds.zip -d ~/texmf|
% \end{quote}
%
% \paragraph{Script installation.}
% Check the directory \xfile{TDS:scripts/oberdiek/} for
% scripts that need further installation steps.
% Package \xpackage{attachfile2} comes with the Perl script
% \xfile{pdfatfi.pl} that should be installed in such a way
% that it can be called as \texttt{pdfatfi}.
% Example (linux):
% \begin{quote}
%   |chmod +x scripts/oberdiek/pdfatfi.pl|\\
%   |cp scripts/oberdiek/pdfatfi.pl /usr/local/bin/|
% \end{quote}
%
% \subsection{Package installation}
%
% \paragraph{Unpacking.} The \xfile{.dtx} file is a self-extracting
% \docstrip\ archive. The files are extracted by running the
% \xfile{.dtx} through \plainTeX:
% \begin{quote}
%   \verb|tex twoopt.dtx|
% \end{quote}
%
% \paragraph{TDS.} Now the different files must be moved into
% the different directories in your installation TDS tree
% (also known as \xfile{texmf} tree):
% \begin{quote}
% \def\t{^^A
% \begin{tabular}{@{}>{\ttfamily}l@{ $\rightarrow$ }>{\ttfamily}l@{}}
%   twoopt.sty & tex/latex/oberdiek/twoopt.sty\\
%   twoopt.pdf & doc/latex/oberdiek/twoopt.pdf\\
%   twoopt.dtx & source/latex/oberdiek/twoopt.dtx\\
% \end{tabular}^^A
% }^^A
% \sbox0{\t}^^A
% \ifdim\wd0>\linewidth
%   \begingroup
%     \advance\linewidth by\leftmargin
%     \advance\linewidth by\rightmargin
%   \edef\x{\endgroup
%     \def\noexpand\lw{\the\linewidth}^^A
%   }\x
%   \def\lwbox{^^A
%     \leavevmode
%     \hbox to \linewidth{^^A
%       \kern-\leftmargin\relax
%       \hss
%       \usebox0
%       \hss
%       \kern-\rightmargin\relax
%     }^^A
%   }^^A
%   \ifdim\wd0>\lw
%     \sbox0{\small\t}^^A
%     \ifdim\wd0>\linewidth
%       \ifdim\wd0>\lw
%         \sbox0{\footnotesize\t}^^A
%         \ifdim\wd0>\linewidth
%           \ifdim\wd0>\lw
%             \sbox0{\scriptsize\t}^^A
%             \ifdim\wd0>\linewidth
%               \ifdim\wd0>\lw
%                 \sbox0{\tiny\t}^^A
%                 \ifdim\wd0>\linewidth
%                   \lwbox
%                 \else
%                   \usebox0
%                 \fi
%               \else
%                 \lwbox
%               \fi
%             \else
%               \usebox0
%             \fi
%           \else
%             \lwbox
%           \fi
%         \else
%           \usebox0
%         \fi
%       \else
%         \lwbox
%       \fi
%     \else
%       \usebox0
%     \fi
%   \else
%     \lwbox
%   \fi
% \else
%   \usebox0
% \fi
% \end{quote}
% If you have a \xfile{docstrip.cfg} that configures and enables \docstrip's
% TDS installing feature, then some files can already be in the right
% place, see the documentation of \docstrip.
%
% \subsection{Refresh file name databases}
%
% If your \TeX~distribution
% (\teTeX, \mikTeX, \dots) relies on file name databases, you must refresh
% these. For example, \teTeX\ users run \verb|texhash| or
% \verb|mktexlsr|.
%
% \subsection{Some details for the interested}
%
% \paragraph{Attached source.}
%
% The PDF documentation on CTAN also includes the
% \xfile{.dtx} source file. It can be extracted by
% AcrobatReader 6 or higher. Another option is \textsf{pdftk},
% e.g. unpack the file into the current directory:
% \begin{quote}
%   \verb|pdftk twoopt.pdf unpack_files output .|
% \end{quote}
%
% \paragraph{Unpacking with \LaTeX.}
% The \xfile{.dtx} chooses its action depending on the format:
% \begin{description}
% \item[\plainTeX:] Run \docstrip\ and extract the files.
% \item[\LaTeX:] Generate the documentation.
% \end{description}
% If you insist on using \LaTeX\ for \docstrip\ (really,
% \docstrip\ does not need \LaTeX), then inform the autodetect routine
% about your intention:
% \begin{quote}
%   \verb|latex \let\install=y% \iffalse meta-comment
%
% File: twoopt.dtx
% Version: 2016/05/16 v1.6
% Info: Definitions with two optional arguments
%
% Copyright (C) 1999, 2006, 2008 by
%    Heiko Oberdiek <heiko.oberdiek at googlemail.com>
%    2016
%    https://github.com/ho-tex/oberdiek/issues
%
% This work may be distributed and/or modified under the
% conditions of the LaTeX Project Public License, either
% version 1.3c of this license or (at your option) any later
% version. This version of this license is in
%    http://www.latex-project.org/lppl/lppl-1-3c.txt
% and the latest version of this license is in
%    http://www.latex-project.org/lppl.txt
% and version 1.3 or later is part of all distributions of
% LaTeX version 2005/12/01 or later.
%
% This work has the LPPL maintenance status "maintained".
%
% This Current Maintainer of this work is Heiko Oberdiek.
%
% This work consists of the main source file twoopt.dtx
% and the derived files
%    twoopt.sty, twoopt.pdf, twoopt.ins, twoopt.drv.
%
% Distribution:
%    CTAN:macros/latex/contrib/oberdiek/twoopt.dtx
%    CTAN:macros/latex/contrib/oberdiek/twoopt.pdf
%
% Unpacking:
%    (a) If twoopt.ins is present:
%           tex twoopt.ins
%    (b) Without twoopt.ins:
%           tex twoopt.dtx
%    (c) If you insist on using LaTeX
%           latex \let\install=y\input{twoopt.dtx}
%        (quote the arguments according to the demands of your shell)
%
% Documentation:
%    (a) If twoopt.drv is present:
%           latex twoopt.drv
%    (b) Without twoopt.drv:
%           latex twoopt.dtx; ...
%    The class ltxdoc loads the configuration file ltxdoc.cfg
%    if available. Here you can specify further options, e.g.
%    use A4 as paper format:
%       \PassOptionsToClass{a4paper}{article}
%
%    Programm calls to get the documentation (example):
%       pdflatex twoopt.dtx
%       makeindex -s gind.ist twoopt.idx
%       pdflatex twoopt.dtx
%       makeindex -s gind.ist twoopt.idx
%       pdflatex twoopt.dtx
%
% Installation:
%    TDS:tex/latex/oberdiek/twoopt.sty
%    TDS:doc/latex/oberdiek/twoopt.pdf
%    TDS:source/latex/oberdiek/twoopt.dtx
%
%<*ignore>
\begingroup
  \catcode123=1 %
  \catcode125=2 %
  \def\x{LaTeX2e}%
\expandafter\endgroup
\ifcase 0\ifx\install y1\fi\expandafter
         \ifx\csname processbatchFile\endcsname\relax\else1\fi
         \ifx\fmtname\x\else 1\fi\relax
\else\csname fi\endcsname
%</ignore>
%<*install>
\input docstrip.tex
\Msg{************************************************************************}
\Msg{* Installation}
\Msg{* Package: twoopt 2016/05/16 v1.6 Definitions with two optional arguments (HO)}
\Msg{************************************************************************}

\keepsilent
\askforoverwritefalse

\let\MetaPrefix\relax
\preamble

This is a generated file.

Project: twoopt
Version: 2016/05/16 v1.6

Copyright (C) 1999, 2006, 2008 by
   Heiko Oberdiek <heiko.oberdiek at googlemail.com>

This work may be distributed and/or modified under the
conditions of the LaTeX Project Public License, either
version 1.3c of this license or (at your option) any later
version. This version of this license is in
   http://www.latex-project.org/lppl/lppl-1-3c.txt
and the latest version of this license is in
   http://www.latex-project.org/lppl.txt
and version 1.3 or later is part of all distributions of
LaTeX version 2005/12/01 or later.

This work has the LPPL maintenance status "maintained".

This Current Maintainer of this work is Heiko Oberdiek.

This work consists of the main source file twoopt.dtx
and the derived files
   twoopt.sty, twoopt.pdf, twoopt.ins, twoopt.drv.

\endpreamble
\let\MetaPrefix\DoubleperCent

\generate{%
  \file{twoopt.ins}{\from{twoopt.dtx}{install}}%
  \file{twoopt.drv}{\from{twoopt.dtx}{driver}}%
  \usedir{tex/latex/oberdiek}%
  \file{twoopt.sty}{\from{twoopt.dtx}{package}}%
  \nopreamble
  \nopostamble
%  \usedir{source/latex/oberdiek/catalogue}%
%  \file{twoopt.xml}{\from{twoopt.dtx}{catalogue}}%
}

\catcode32=13\relax% active space
\let =\space%
\Msg{************************************************************************}
\Msg{*}
\Msg{* To finish the installation you have to move the following}
\Msg{* file into a directory searched by TeX:}
\Msg{*}
\Msg{*     twoopt.sty}
\Msg{*}
\Msg{* To produce the documentation run the file `twoopt.drv'}
\Msg{* through LaTeX.}
\Msg{*}
\Msg{* Happy TeXing!}
\Msg{*}
\Msg{************************************************************************}

\endbatchfile
%</install>
%<*ignore>
\fi
%</ignore>
%<*driver>
\NeedsTeXFormat{LaTeX2e}
\ProvidesFile{twoopt.drv}%
  [2016/05/16 v1.6 Definitions with two optional arguments (HO)]%
\documentclass{ltxdoc}
\usepackage{holtxdoc}[2011/11/22]
\begin{document}
  \DocInput{twoopt.dtx}%
\end{document}
%</driver>
% \fi
%
%
% \CharacterTable
%  {Upper-case    \A\B\C\D\E\F\G\H\I\J\K\L\M\N\O\P\Q\R\S\T\U\V\W\X\Y\Z
%   Lower-case    \a\b\c\d\e\f\g\h\i\j\k\l\m\n\o\p\q\r\s\t\u\v\w\x\y\z
%   Digits        \0\1\2\3\4\5\6\7\8\9
%   Exclamation   \!     Double quote  \"     Hash (number) \#
%   Dollar        \$     Percent       \%     Ampersand     \&
%   Acute accent  \'     Left paren    \(     Right paren   \)
%   Asterisk      \*     Plus          \+     Comma         \,
%   Minus         \-     Point         \.     Solidus       \/
%   Colon         \:     Semicolon     \;     Less than     \<
%   Equals        \=     Greater than  \>     Question mark \?
%   Commercial at \@     Left bracket  \[     Backslash     \\
%   Right bracket \]     Circumflex    \^     Underscore    \_
%   Grave accent  \`     Left brace    \{     Vertical bar  \|
%   Right brace   \}     Tilde         \~}
%
% \GetFileInfo{twoopt.drv}
%
% \title{The \xpackage{twoopt} package}
% \date{2016/05/16 v1.6}
% \author{Heiko Oberdiek\thanks
% {Please report any issues at https://github.com/ho-tex/oberdiek/issues}\\
% \xemail{heiko.oberdiek at googlemail.com}}
%
% \maketitle
%
% \begin{abstract}
% This package provides commands to define macros with two
% optional arguments.
% \end{abstract}
%
% \tableofcontents
%
% \newenvironment{param}{^^A
%   \newcommand{\entry}[1]{\meta{\###1}:&}^^A
%   \begin{tabular}[t]{@{}l@{ }l@{}}^^A
% }{^^A
%   \end{tabular}^^A
% }
%
% \section{Usage}
%    \DescribeMacro{\newcommandtwoopt}
%    \DescribeMacro{\renewcommandtwoopt}
%    \DescribeMacro{\providecommandtwoopt}
%    Similar to \cmd{\newcommand}, \cmd{\renewcommand}
%    and \cmd{\providecommand} this package provides commands
%    to define macros with two optional arguments.
%    The names of the commands are built by appending the
%    package name to the \LaTeX-pendants:
%    \begingroup
%      \def\x{\marg{cmd} \oarg{num} \oarg{default1}^^A
%             \oarg{default2} \marg{def.}}^^A
%      \begin{tabbing}
%        \cmd{\providecommandtwoopt} \=\kill
%        \cmd{\newcommandtwoopt}\>\x\\
%        \cmd{\renewcommandtwoopt}\>\x\\
%        \cmd{\providecommandtwoopt}\>\x\\
%      \end{tabbing}
%    \endgroup
%
%    Also the |*|-forms are supported. Indeed it is better to
%    use this ones, unless it is intended to hold
%    whole paragraphs in some of the arguments. If the macro
%    is defined with the |*|-form, missing braces
%    can be detected earlier.
%
%    Example:
%    \begin{quote}
%      |\newcommandtwoopt{\bsp}[3][AA][BB]{%|\\
%      |  \typeout{\string\bsp: #1,#2,#3}%|\\
%      |}|\\
%      \begin{tabular}{@{}l@{\quad$\rightarrow$\quad}l@{}}
%      |\bsp[aa][bb]{cc}|&|\bsp: aa,bb,cc|\\
%      |\bsp[aa]{cc}|&|\bsp: aa,BB,cc|\\
%      |\bsp{cc}|&|\bsp: AA,BB,cc|\\
%      \end{tabular}
%    \end{quote}
%
% \StopEventually{
% }
%
% \section{Implementation}
%    \begin{macrocode}
%<*package>
\NeedsTeXFormat{LaTeX2e}
\ProvidesPackage{twoopt}
  [2016/05/16 v1.6 Definitions with two optional arguments (HO)]%
%    \end{macrocode}
%    \begin{macro}{\newcommandtwoopt}
%    \begin{macrocode}
\newcommand{\newcommandtwoopt}{%
  \@ifstar{\@newcommandtwoopt*}{\@newcommandtwoopt{}}%
}
%    \end{macrocode}
%    \end{macro}
%
%    \begin{macro}{\@newcommandtwoopt}
%    \begin{param}
%      \entry1 star\\
%      \entry2 macro name to be defined
%    \end{param}
%    \begin{macrocode}
\newcommand{\@newcommandtwoopt}{}
\long\def\@newcommandtwoopt#1#2{%
  \expandafter\@@newcommandtwoopt
    \csname2\string#2\endcsname{#1}{#2}%
}
%    \end{macrocode}
%    \end{macro}
%
%    \begin{macro}{\@@newcommandtwoopt}
%    \begin{param}
%      \entry1 help command to be defined
%        (\expandafter\cmd\csname 2\bslash<name>\endcsname)\\
%      \entry2 star\\
%      \entry3 macro name to be defined\\
%      \entry4 number of total arguments\\
%      \entry5 default for optional argument one\\
%      \entry6 default for optional argument two
%    \end{param}
%    \begin{macrocode}
\newcommand{\@@newcommandtwoopt}{}
\long\def\@@newcommandtwoopt#1#2#3[#4][#5][#6]{%
  \newcommand#2#3[1][{#5}]{%
    \to@ScanSecondOptArg#1{##1}{#6}%
  }%
  \newcommand#2#1[{#4}]%
}
%    \end{macrocode}
%    \end{macro}
%
%    \begin{macro}{\renewcommandtwoopt}
%    \begin{macrocode}
\newcommand{\renewcommandtwoopt}{%
  \@ifstar{\@renewcommandtwoopt*}{\@renewcommandtwoopt{}}%
}
%    \end{macrocode}
%    \end{macro}
%
%    \begin{macro}{\@renewcommandtwoopt}
%    \begin{param}
%      \entry1 star\\
%      \entry2 command name to be defined
%    \end{param}
%    \begin{macrocode}
\newcommand{\@renewcommandtwoopt}{}
\long\def\@renewcommandtwoopt#1#2{%
  \begingroup
    \escapechar\m@ne
    \xdef\@gtempa{{\string#2}}%
  \endgroup
  \expandafter\@ifundefined\@gtempa{%
    \@latex@error{\noexpand#2undefined}\@ehc
  }{}%
  \let#2\@undefined
  \expandafter\let\csname2\string#2\endcsname\@undefined
  \expandafter\@@newcommandtwoopt
    \csname2\string#2\endcsname{#1}{#2}%
}
%    \end{macrocode}
%    \end{macro}
%
%    \begin{macro}{\providecommandtwoopt}
%    \begin{macrocode}
\newcommand{\providecommandtwoopt}{%
  \@ifstar{\@providecommandtwoopt*}{\@providecommandtwoopt{}}%
}
%    \end{macrocode}
%    \end{macro}
%
%    \begin{macro}{\@providecommandtwoopt}
%    \begin{param}
%      \entry1 star\\
%      \entry2 command name to be defined
%    \end{param}
%    \begin{macrocode}
\newcommand{\@providecommandtwoopt}{}
\long\def\@providecommandtwoopt#1#2{%
  \begingroup
    \escapechar\m@ne
    \xdef\@gtempa{{\string#2}}%
  \endgroup
  \expandafter\@ifundefined\@gtempa{%
    \expandafter\@@newcommandtwoopt
      \csname2\string#2\endcsname{#1}{#2}%
  }{%
    \let\to@dummyA\@undefined
    \let\to@dummyB\@undefined
    \@@newcommandtwoopt\to@dummyA{#1}\to@dummyB
  }%
}
%    \end{macrocode}
%    \end{macro}
%
%    \begin{macro}{\to@ScanSecondOptArg}
%    \begin{param}
%      \entry1 help command to be defined
%        (\expandafter\cmd\csname 2\bslash<name>\endcsname)\\
%      \entry2 first arg of command to be defined\\
%      \entry3 default for second opt. arg.
%    \end{param}
%    \begin{macrocode}
\newcommand{\to@ScanSecondOptArg}[3]{%
  \@ifnextchar[{%
    \expandafter#1\to@ArgOptToArgArg{#2}%
  }{%
    #1{#2}{#3}%
  }%
}
%    \end{macrocode}
%    \end{macro}
%
%    \begin{macro}{\to@ArgOptToArgArg}
%    \begin{macrocode}
\newcommand{\to@ArgOptToArgArg}{}
\long\def\to@ArgOptToArgArg#1[#2]{{#1}{#2}}
%    \end{macrocode}
%    \end{macro}
%
%    \begin{macrocode}
%</package>
%    \end{macrocode}
%
% \section{Installation}
%
% \subsection{Download}
%
% \paragraph{Package.} This package is available on
% CTAN\footnote{\url{http://ctan.org/pkg/twoopt}}:
% \begin{description}
% \item[\CTAN{macros/latex/contrib/oberdiek/twoopt.dtx}] The source file.
% \item[\CTAN{macros/latex/contrib/oberdiek/twoopt.pdf}] Documentation.
% \end{description}
%
%
% \paragraph{Bundle.} All the packages of the bundle `oberdiek'
% are also available in a TDS compliant ZIP archive. There
% the packages are already unpacked and the documentation files
% are generated. The files and directories obey the TDS standard.
% \begin{description}
% \item[\CTAN{install/macros/latex/contrib/oberdiek.tds.zip}]
% \end{description}
% \emph{TDS} refers to the standard ``A Directory Structure
% for \TeX\ Files'' (\CTAN{tds/tds.pdf}). Directories
% with \xfile{texmf} in their name are usually organized this way.
%
% \subsection{Bundle installation}
%
% \paragraph{Unpacking.} Unpack the \xfile{oberdiek.tds.zip} in the
% TDS tree (also known as \xfile{texmf} tree) of your choice.
% Example (linux):
% \begin{quote}
%   |unzip oberdiek.tds.zip -d ~/texmf|
% \end{quote}
%
% \paragraph{Script installation.}
% Check the directory \xfile{TDS:scripts/oberdiek/} for
% scripts that need further installation steps.
% Package \xpackage{attachfile2} comes with the Perl script
% \xfile{pdfatfi.pl} that should be installed in such a way
% that it can be called as \texttt{pdfatfi}.
% Example (linux):
% \begin{quote}
%   |chmod +x scripts/oberdiek/pdfatfi.pl|\\
%   |cp scripts/oberdiek/pdfatfi.pl /usr/local/bin/|
% \end{quote}
%
% \subsection{Package installation}
%
% \paragraph{Unpacking.} The \xfile{.dtx} file is a self-extracting
% \docstrip\ archive. The files are extracted by running the
% \xfile{.dtx} through \plainTeX:
% \begin{quote}
%   \verb|tex twoopt.dtx|
% \end{quote}
%
% \paragraph{TDS.} Now the different files must be moved into
% the different directories in your installation TDS tree
% (also known as \xfile{texmf} tree):
% \begin{quote}
% \def\t{^^A
% \begin{tabular}{@{}>{\ttfamily}l@{ $\rightarrow$ }>{\ttfamily}l@{}}
%   twoopt.sty & tex/latex/oberdiek/twoopt.sty\\
%   twoopt.pdf & doc/latex/oberdiek/twoopt.pdf\\
%   twoopt.dtx & source/latex/oberdiek/twoopt.dtx\\
% \end{tabular}^^A
% }^^A
% \sbox0{\t}^^A
% \ifdim\wd0>\linewidth
%   \begingroup
%     \advance\linewidth by\leftmargin
%     \advance\linewidth by\rightmargin
%   \edef\x{\endgroup
%     \def\noexpand\lw{\the\linewidth}^^A
%   }\x
%   \def\lwbox{^^A
%     \leavevmode
%     \hbox to \linewidth{^^A
%       \kern-\leftmargin\relax
%       \hss
%       \usebox0
%       \hss
%       \kern-\rightmargin\relax
%     }^^A
%   }^^A
%   \ifdim\wd0>\lw
%     \sbox0{\small\t}^^A
%     \ifdim\wd0>\linewidth
%       \ifdim\wd0>\lw
%         \sbox0{\footnotesize\t}^^A
%         \ifdim\wd0>\linewidth
%           \ifdim\wd0>\lw
%             \sbox0{\scriptsize\t}^^A
%             \ifdim\wd0>\linewidth
%               \ifdim\wd0>\lw
%                 \sbox0{\tiny\t}^^A
%                 \ifdim\wd0>\linewidth
%                   \lwbox
%                 \else
%                   \usebox0
%                 \fi
%               \else
%                 \lwbox
%               \fi
%             \else
%               \usebox0
%             \fi
%           \else
%             \lwbox
%           \fi
%         \else
%           \usebox0
%         \fi
%       \else
%         \lwbox
%       \fi
%     \else
%       \usebox0
%     \fi
%   \else
%     \lwbox
%   \fi
% \else
%   \usebox0
% \fi
% \end{quote}
% If you have a \xfile{docstrip.cfg} that configures and enables \docstrip's
% TDS installing feature, then some files can already be in the right
% place, see the documentation of \docstrip.
%
% \subsection{Refresh file name databases}
%
% If your \TeX~distribution
% (\teTeX, \mikTeX, \dots) relies on file name databases, you must refresh
% these. For example, \teTeX\ users run \verb|texhash| or
% \verb|mktexlsr|.
%
% \subsection{Some details for the interested}
%
% \paragraph{Attached source.}
%
% The PDF documentation on CTAN also includes the
% \xfile{.dtx} source file. It can be extracted by
% AcrobatReader 6 or higher. Another option is \textsf{pdftk},
% e.g. unpack the file into the current directory:
% \begin{quote}
%   \verb|pdftk twoopt.pdf unpack_files output .|
% \end{quote}
%
% \paragraph{Unpacking with \LaTeX.}
% The \xfile{.dtx} chooses its action depending on the format:
% \begin{description}
% \item[\plainTeX:] Run \docstrip\ and extract the files.
% \item[\LaTeX:] Generate the documentation.
% \end{description}
% If you insist on using \LaTeX\ for \docstrip\ (really,
% \docstrip\ does not need \LaTeX), then inform the autodetect routine
% about your intention:
% \begin{quote}
%   \verb|latex \let\install=y\input{twoopt.dtx}|
% \end{quote}
% Do not forget to quote the argument according to the demands
% of your shell.
%
% \paragraph{Generating the documentation.}
% You can use both the \xfile{.dtx} or the \xfile{.drv} to generate
% the documentation. The process can be configured by the
% configuration file \xfile{ltxdoc.cfg}. For instance, put this
% line into this file, if you want to have A4 as paper format:
% \begin{quote}
%   \verb|\PassOptionsToClass{a4paper}{article}|
% \end{quote}
% An example follows how to generate the
% documentation with pdf\LaTeX:
% \begin{quote}
%\begin{verbatim}
%pdflatex twoopt.dtx
%makeindex -s gind.ist twoopt.idx
%pdflatex twoopt.dtx
%makeindex -s gind.ist twoopt.idx
%pdflatex twoopt.dtx
%\end{verbatim}
% \end{quote}
%
% \section{Catalogue}
%
% The following XML file can be used as source for the
% \href{http://mirror.ctan.org/help/Catalogue/catalogue.html}{\TeX\ Catalogue}.
% The elements \texttt{caption} and \texttt{description} are imported
% from the original XML file from the Catalogue.
% The name of the XML file in the Catalogue is \xfile{twoopt.xml}.
%    \begin{macrocode}
%<*catalogue>
<?xml version='1.0' encoding='us-ascii'?>
<!DOCTYPE entry SYSTEM 'catalogue.dtd'>
<entry datestamp='$Date$' modifier='$Author$' id='twoopt'>
  <name>twoopt</name>
  <caption>Definitions with two optional arguments.</caption>
  <authorref id='auth:oberdiek'/>
  <copyright owner='Heiko Oberdiek' year='1999,2006,2008'/>
  <license type='lppl1.3'/>
  <version number='1.6'/>
  <description>
    Variants of <tt>\newcommand</tt>, <tt>\renewcommand</tt> and
    <tt>\providecommand</tt> are provided.
    <p/>
    The package is part of the <xref refid='oberdiek'>oberdiek</xref>
    bundle.
  </description>
  <documentation details='Package documentation'
      href='ctan:/macros/latex/contrib/oberdiek/twoopt.pdf'/>
  <ctan file='true' path='/macros/latex/contrib/oberdiek/twoopt.dtx'/>
  <miktex location='oberdiek'/>
  <texlive location='oberdiek'/>
  <install path='/macros/latex/contrib/oberdiek/oberdiek.tds.zip'/>
</entry>
%</catalogue>
%    \end{macrocode}
%
% \begin{History}
%   \begin{Version}{1998/10/30 v1.0}
%   \item
%     The first version was built as a response to a question
%     of \NameEmail{Rebecca and Rowland}{rebecca@astrid.u-net.com},
%     published in the newsgroup
%     \href{news:comp.text.tex}{comp.text.tex}:\\
%     \URL{``Re: [Q] LaTeX command with two optional arguments?''}^^A
%     {http://groups.google.com/group/comp.text.tex/msg/0ab1afde7b172d37}
%   \end{Version}
%   \begin{Version}{1998/10/30 v1.1}
%   \item
%     Improvements added in response to
%     \NameEmail{Stefan Ulrich}{ulrich@cis.uni-muenchen.de}
%     in the same thread:\\
%     \URL{``Re: [Q] LaTeX command with two optional arguments?''}^^A
%     {http://groups.google.com/group/comp.text.tex/msg/b8d84d4336f302c4}
%   \end{Version}
%   \begin{Version}{1998/11/04 v1.2}
%   \item
%     Fixes for LaTeX bugs 2896, 2901, 2902 added.
%   \end{Version}
%   \begin{Version}{1999/04/12 v1.3}
%   \item
%     Fixes removed because of LaTeX [1998/12/01].
%   \item
%     Documentation in dtx format.
%   \item
%     Copyright: LPPL (\CTAN{macros/latex/base/lppl.txt})
%   \item
%     First CTAN release.
%   \end{Version}
%   \begin{Version}{2006/02/20 v1.4}
%   \item
%     Code is not changed.
%   \item
%     New DTX framework.
%   \item
%     LPPL 1.3
%   \end{Version}
%   \begin{Version}{2008/08/11 v1.5}
%   \item
%     Code is not changed.
%   \item
%     URLs updated from \texttt{www.dejanews.com}
%     to \texttt{groups.google.com}.
%   \end{Version}
%   \begin{Version}{2016/05/16 v1.6}
%   \item
%     Documentation updates.
%   \end{Version}
% \end{History}
%
% \PrintIndex
%
% \Finale
\endinput
|
% \end{quote}
% Do not forget to quote the argument according to the demands
% of your shell.
%
% \paragraph{Generating the documentation.}
% You can use both the \xfile{.dtx} or the \xfile{.drv} to generate
% the documentation. The process can be configured by the
% configuration file \xfile{ltxdoc.cfg}. For instance, put this
% line into this file, if you want to have A4 as paper format:
% \begin{quote}
%   \verb|\PassOptionsToClass{a4paper}{article}|
% \end{quote}
% An example follows how to generate the
% documentation with pdf\LaTeX:
% \begin{quote}
%\begin{verbatim}
%pdflatex twoopt.dtx
%makeindex -s gind.ist twoopt.idx
%pdflatex twoopt.dtx
%makeindex -s gind.ist twoopt.idx
%pdflatex twoopt.dtx
%\end{verbatim}
% \end{quote}
%
% \section{Catalogue}
%
% The following XML file can be used as source for the
% \href{http://mirror.ctan.org/help/Catalogue/catalogue.html}{\TeX\ Catalogue}.
% The elements \texttt{caption} and \texttt{description} are imported
% from the original XML file from the Catalogue.
% The name of the XML file in the Catalogue is \xfile{twoopt.xml}.
%    \begin{macrocode}
%<*catalogue>
<?xml version='1.0' encoding='us-ascii'?>
<!DOCTYPE entry SYSTEM 'catalogue.dtd'>
<entry datestamp='$Date$' modifier='$Author$' id='twoopt'>
  <name>twoopt</name>
  <caption>Definitions with two optional arguments.</caption>
  <authorref id='auth:oberdiek'/>
  <copyright owner='Heiko Oberdiek' year='1999,2006,2008'/>
  <license type='lppl1.3'/>
  <version number='1.6'/>
  <description>
    Variants of <tt>\newcommand</tt>, <tt>\renewcommand</tt> and
    <tt>\providecommand</tt> are provided.
    <p/>
    The package is part of the <xref refid='oberdiek'>oberdiek</xref>
    bundle.
  </description>
  <documentation details='Package documentation'
      href='ctan:/macros/latex/contrib/oberdiek/twoopt.pdf'/>
  <ctan file='true' path='/macros/latex/contrib/oberdiek/twoopt.dtx'/>
  <miktex location='oberdiek'/>
  <texlive location='oberdiek'/>
  <install path='/macros/latex/contrib/oberdiek/oberdiek.tds.zip'/>
</entry>
%</catalogue>
%    \end{macrocode}
%
% \begin{History}
%   \begin{Version}{1998/10/30 v1.0}
%   \item
%     The first version was built as a response to a question
%     of \NameEmail{Rebecca and Rowland}{rebecca@astrid.u-net.com},
%     published in the newsgroup
%     \href{news:comp.text.tex}{comp.text.tex}:\\
%     \URL{``Re: [Q] LaTeX command with two optional arguments?''}^^A
%     {http://groups.google.com/group/comp.text.tex/msg/0ab1afde7b172d37}
%   \end{Version}
%   \begin{Version}{1998/10/30 v1.1}
%   \item
%     Improvements added in response to
%     \NameEmail{Stefan Ulrich}{ulrich@cis.uni-muenchen.de}
%     in the same thread:\\
%     \URL{``Re: [Q] LaTeX command with two optional arguments?''}^^A
%     {http://groups.google.com/group/comp.text.tex/msg/b8d84d4336f302c4}
%   \end{Version}
%   \begin{Version}{1998/11/04 v1.2}
%   \item
%     Fixes for LaTeX bugs 2896, 2901, 2902 added.
%   \end{Version}
%   \begin{Version}{1999/04/12 v1.3}
%   \item
%     Fixes removed because of LaTeX [1998/12/01].
%   \item
%     Documentation in dtx format.
%   \item
%     Copyright: LPPL (\CTAN{macros/latex/base/lppl.txt})
%   \item
%     First CTAN release.
%   \end{Version}
%   \begin{Version}{2006/02/20 v1.4}
%   \item
%     Code is not changed.
%   \item
%     New DTX framework.
%   \item
%     LPPL 1.3
%   \end{Version}
%   \begin{Version}{2008/08/11 v1.5}
%   \item
%     Code is not changed.
%   \item
%     URLs updated from \texttt{www.dejanews.com}
%     to \texttt{groups.google.com}.
%   \end{Version}
%   \begin{Version}{2016/05/16 v1.6}
%   \item
%     Documentation updates.
%   \end{Version}
% \end{History}
%
% \PrintIndex
%
% \Finale
\endinput

%        (quote the arguments according to the demands of your shell)
%
% Documentation:
%    (a) If twoopt.drv is present:
%           latex twoopt.drv
%    (b) Without twoopt.drv:
%           latex twoopt.dtx; ...
%    The class ltxdoc loads the configuration file ltxdoc.cfg
%    if available. Here you can specify further options, e.g.
%    use A4 as paper format:
%       \PassOptionsToClass{a4paper}{article}
%
%    Programm calls to get the documentation (example):
%       pdflatex twoopt.dtx
%       makeindex -s gind.ist twoopt.idx
%       pdflatex twoopt.dtx
%       makeindex -s gind.ist twoopt.idx
%       pdflatex twoopt.dtx
%
% Installation:
%    TDS:tex/latex/oberdiek/twoopt.sty
%    TDS:doc/latex/oberdiek/twoopt.pdf
%    TDS:source/latex/oberdiek/twoopt.dtx
%
%<*ignore>
\begingroup
  \catcode123=1 %
  \catcode125=2 %
  \def\x{LaTeX2e}%
\expandafter\endgroup
\ifcase 0\ifx\install y1\fi\expandafter
         \ifx\csname processbatchFile\endcsname\relax\else1\fi
         \ifx\fmtname\x\else 1\fi\relax
\else\csname fi\endcsname
%</ignore>
%<*install>
\input docstrip.tex
\Msg{************************************************************************}
\Msg{* Installation}
\Msg{* Package: twoopt 2016/05/16 v1.6 Definitions with two optional arguments (HO)}
\Msg{************************************************************************}

\keepsilent
\askforoverwritefalse

\let\MetaPrefix\relax
\preamble

This is a generated file.

Project: twoopt
Version: 2016/05/16 v1.6

Copyright (C) 1999, 2006, 2008 by
   Heiko Oberdiek <heiko.oberdiek at googlemail.com>

This work may be distributed and/or modified under the
conditions of the LaTeX Project Public License, either
version 1.3c of this license or (at your option) any later
version. This version of this license is in
   http://www.latex-project.org/lppl/lppl-1-3c.txt
and the latest version of this license is in
   http://www.latex-project.org/lppl.txt
and version 1.3 or later is part of all distributions of
LaTeX version 2005/12/01 or later.

This work has the LPPL maintenance status "maintained".

This Current Maintainer of this work is Heiko Oberdiek.

This work consists of the main source file twoopt.dtx
and the derived files
   twoopt.sty, twoopt.pdf, twoopt.ins, twoopt.drv.

\endpreamble
\let\MetaPrefix\DoubleperCent

\generate{%
  \file{twoopt.ins}{\from{twoopt.dtx}{install}}%
  \file{twoopt.drv}{\from{twoopt.dtx}{driver}}%
  \usedir{tex/latex/oberdiek}%
  \file{twoopt.sty}{\from{twoopt.dtx}{package}}%
  \nopreamble
  \nopostamble
%  \usedir{source/latex/oberdiek/catalogue}%
%  \file{twoopt.xml}{\from{twoopt.dtx}{catalogue}}%
}

\catcode32=13\relax% active space
\let =\space%
\Msg{************************************************************************}
\Msg{*}
\Msg{* To finish the installation you have to move the following}
\Msg{* file into a directory searched by TeX:}
\Msg{*}
\Msg{*     twoopt.sty}
\Msg{*}
\Msg{* To produce the documentation run the file `twoopt.drv'}
\Msg{* through LaTeX.}
\Msg{*}
\Msg{* Happy TeXing!}
\Msg{*}
\Msg{************************************************************************}

\endbatchfile
%</install>
%<*ignore>
\fi
%</ignore>
%<*driver>
\NeedsTeXFormat{LaTeX2e}
\ProvidesFile{twoopt.drv}%
  [2016/05/16 v1.6 Definitions with two optional arguments (HO)]%
\documentclass{ltxdoc}
\usepackage{holtxdoc}[2011/11/22]
\begin{document}
  \DocInput{twoopt.dtx}%
\end{document}
%</driver>
% \fi
%
%
% \CharacterTable
%  {Upper-case    \A\B\C\D\E\F\G\H\I\J\K\L\M\N\O\P\Q\R\S\T\U\V\W\X\Y\Z
%   Lower-case    \a\b\c\d\e\f\g\h\i\j\k\l\m\n\o\p\q\r\s\t\u\v\w\x\y\z
%   Digits        \0\1\2\3\4\5\6\7\8\9
%   Exclamation   \!     Double quote  \"     Hash (number) \#
%   Dollar        \$     Percent       \%     Ampersand     \&
%   Acute accent  \'     Left paren    \(     Right paren   \)
%   Asterisk      \*     Plus          \+     Comma         \,
%   Minus         \-     Point         \.     Solidus       \/
%   Colon         \:     Semicolon     \;     Less than     \<
%   Equals        \=     Greater than  \>     Question mark \?
%   Commercial at \@     Left bracket  \[     Backslash     \\
%   Right bracket \]     Circumflex    \^     Underscore    \_
%   Grave accent  \`     Left brace    \{     Vertical bar  \|
%   Right brace   \}     Tilde         \~}
%
% \GetFileInfo{twoopt.drv}
%
% \title{The \xpackage{twoopt} package}
% \date{2016/05/16 v1.6}
% \author{Heiko Oberdiek\thanks
% {Please report any issues at https://github.com/ho-tex/oberdiek/issues}\\
% \xemail{heiko.oberdiek at googlemail.com}}
%
% \maketitle
%
% \begin{abstract}
% This package provides commands to define macros with two
% optional arguments.
% \end{abstract}
%
% \tableofcontents
%
% \newenvironment{param}{^^A
%   \newcommand{\entry}[1]{\meta{\###1}:&}^^A
%   \begin{tabular}[t]{@{}l@{ }l@{}}^^A
% }{^^A
%   \end{tabular}^^A
% }
%
% \section{Usage}
%    \DescribeMacro{\newcommandtwoopt}
%    \DescribeMacro{\renewcommandtwoopt}
%    \DescribeMacro{\providecommandtwoopt}
%    Similar to \cmd{\newcommand}, \cmd{\renewcommand}
%    and \cmd{\providecommand} this package provides commands
%    to define macros with two optional arguments.
%    The names of the commands are built by appending the
%    package name to the \LaTeX-pendants:
%    \begingroup
%      \def\x{\marg{cmd} \oarg{num} \oarg{default1}^^A
%             \oarg{default2} \marg{def.}}^^A
%      \begin{tabbing}
%        \cmd{\providecommandtwoopt} \=\kill
%        \cmd{\newcommandtwoopt}\>\x\\
%        \cmd{\renewcommandtwoopt}\>\x\\
%        \cmd{\providecommandtwoopt}\>\x\\
%      \end{tabbing}
%    \endgroup
%
%    Also the |*|-forms are supported. Indeed it is better to
%    use this ones, unless it is intended to hold
%    whole paragraphs in some of the arguments. If the macro
%    is defined with the |*|-form, missing braces
%    can be detected earlier.
%
%    Example:
%    \begin{quote}
%      |\newcommandtwoopt{\bsp}[3][AA][BB]{%|\\
%      |  \typeout{\string\bsp: #1,#2,#3}%|\\
%      |}|\\
%      \begin{tabular}{@{}l@{\quad$\rightarrow$\quad}l@{}}
%      |\bsp[aa][bb]{cc}|&|\bsp: aa,bb,cc|\\
%      |\bsp[aa]{cc}|&|\bsp: aa,BB,cc|\\
%      |\bsp{cc}|&|\bsp: AA,BB,cc|\\
%      \end{tabular}
%    \end{quote}
%
% \StopEventually{
% }
%
% \section{Implementation}
%    \begin{macrocode}
%<*package>
\NeedsTeXFormat{LaTeX2e}
\ProvidesPackage{twoopt}
  [2016/05/16 v1.6 Definitions with two optional arguments (HO)]%
%    \end{macrocode}
%    \begin{macro}{\newcommandtwoopt}
%    \begin{macrocode}
\newcommand{\newcommandtwoopt}{%
  \@ifstar{\@newcommandtwoopt*}{\@newcommandtwoopt{}}%
}
%    \end{macrocode}
%    \end{macro}
%
%    \begin{macro}{\@newcommandtwoopt}
%    \begin{param}
%      \entry1 star\\
%      \entry2 macro name to be defined
%    \end{param}
%    \begin{macrocode}
\newcommand{\@newcommandtwoopt}{}
\long\def\@newcommandtwoopt#1#2{%
  \expandafter\@@newcommandtwoopt
    \csname2\string#2\endcsname{#1}{#2}%
}
%    \end{macrocode}
%    \end{macro}
%
%    \begin{macro}{\@@newcommandtwoopt}
%    \begin{param}
%      \entry1 help command to be defined
%        (\expandafter\cmd\csname 2\bslash<name>\endcsname)\\
%      \entry2 star\\
%      \entry3 macro name to be defined\\
%      \entry4 number of total arguments\\
%      \entry5 default for optional argument one\\
%      \entry6 default for optional argument two
%    \end{param}
%    \begin{macrocode}
\newcommand{\@@newcommandtwoopt}{}
\long\def\@@newcommandtwoopt#1#2#3[#4][#5][#6]{%
  \newcommand#2#3[1][{#5}]{%
    \to@ScanSecondOptArg#1{##1}{#6}%
  }%
  \newcommand#2#1[{#4}]%
}
%    \end{macrocode}
%    \end{macro}
%
%    \begin{macro}{\renewcommandtwoopt}
%    \begin{macrocode}
\newcommand{\renewcommandtwoopt}{%
  \@ifstar{\@renewcommandtwoopt*}{\@renewcommandtwoopt{}}%
}
%    \end{macrocode}
%    \end{macro}
%
%    \begin{macro}{\@renewcommandtwoopt}
%    \begin{param}
%      \entry1 star\\
%      \entry2 command name to be defined
%    \end{param}
%    \begin{macrocode}
\newcommand{\@renewcommandtwoopt}{}
\long\def\@renewcommandtwoopt#1#2{%
  \begingroup
    \escapechar\m@ne
    \xdef\@gtempa{{\string#2}}%
  \endgroup
  \expandafter\@ifundefined\@gtempa{%
    \@latex@error{\noexpand#2undefined}\@ehc
  }{}%
  \let#2\@undefined
  \expandafter\let\csname2\string#2\endcsname\@undefined
  \expandafter\@@newcommandtwoopt
    \csname2\string#2\endcsname{#1}{#2}%
}
%    \end{macrocode}
%    \end{macro}
%
%    \begin{macro}{\providecommandtwoopt}
%    \begin{macrocode}
\newcommand{\providecommandtwoopt}{%
  \@ifstar{\@providecommandtwoopt*}{\@providecommandtwoopt{}}%
}
%    \end{macrocode}
%    \end{macro}
%
%    \begin{macro}{\@providecommandtwoopt}
%    \begin{param}
%      \entry1 star\\
%      \entry2 command name to be defined
%    \end{param}
%    \begin{macrocode}
\newcommand{\@providecommandtwoopt}{}
\long\def\@providecommandtwoopt#1#2{%
  \begingroup
    \escapechar\m@ne
    \xdef\@gtempa{{\string#2}}%
  \endgroup
  \expandafter\@ifundefined\@gtempa{%
    \expandafter\@@newcommandtwoopt
      \csname2\string#2\endcsname{#1}{#2}%
  }{%
    \let\to@dummyA\@undefined
    \let\to@dummyB\@undefined
    \@@newcommandtwoopt\to@dummyA{#1}\to@dummyB
  }%
}
%    \end{macrocode}
%    \end{macro}
%
%    \begin{macro}{\to@ScanSecondOptArg}
%    \begin{param}
%      \entry1 help command to be defined
%        (\expandafter\cmd\csname 2\bslash<name>\endcsname)\\
%      \entry2 first arg of command to be defined\\
%      \entry3 default for second opt. arg.
%    \end{param}
%    \begin{macrocode}
\newcommand{\to@ScanSecondOptArg}[3]{%
  \@ifnextchar[{%
    \expandafter#1\to@ArgOptToArgArg{#2}%
  }{%
    #1{#2}{#3}%
  }%
}
%    \end{macrocode}
%    \end{macro}
%
%    \begin{macro}{\to@ArgOptToArgArg}
%    \begin{macrocode}
\newcommand{\to@ArgOptToArgArg}{}
\long\def\to@ArgOptToArgArg#1[#2]{{#1}{#2}}
%    \end{macrocode}
%    \end{macro}
%
%    \begin{macrocode}
%</package>
%    \end{macrocode}
%
% \section{Installation}
%
% \subsection{Download}
%
% \paragraph{Package.} This package is available on
% CTAN\footnote{\url{http://ctan.org/pkg/twoopt}}:
% \begin{description}
% \item[\CTAN{macros/latex/contrib/oberdiek/twoopt.dtx}] The source file.
% \item[\CTAN{macros/latex/contrib/oberdiek/twoopt.pdf}] Documentation.
% \end{description}
%
%
% \paragraph{Bundle.} All the packages of the bundle `oberdiek'
% are also available in a TDS compliant ZIP archive. There
% the packages are already unpacked and the documentation files
% are generated. The files and directories obey the TDS standard.
% \begin{description}
% \item[\CTAN{install/macros/latex/contrib/oberdiek.tds.zip}]
% \end{description}
% \emph{TDS} refers to the standard ``A Directory Structure
% for \TeX\ Files'' (\CTAN{tds/tds.pdf}). Directories
% with \xfile{texmf} in their name are usually organized this way.
%
% \subsection{Bundle installation}
%
% \paragraph{Unpacking.} Unpack the \xfile{oberdiek.tds.zip} in the
% TDS tree (also known as \xfile{texmf} tree) of your choice.
% Example (linux):
% \begin{quote}
%   |unzip oberdiek.tds.zip -d ~/texmf|
% \end{quote}
%
% \paragraph{Script installation.}
% Check the directory \xfile{TDS:scripts/oberdiek/} for
% scripts that need further installation steps.
% Package \xpackage{attachfile2} comes with the Perl script
% \xfile{pdfatfi.pl} that should be installed in such a way
% that it can be called as \texttt{pdfatfi}.
% Example (linux):
% \begin{quote}
%   |chmod +x scripts/oberdiek/pdfatfi.pl|\\
%   |cp scripts/oberdiek/pdfatfi.pl /usr/local/bin/|
% \end{quote}
%
% \subsection{Package installation}
%
% \paragraph{Unpacking.} The \xfile{.dtx} file is a self-extracting
% \docstrip\ archive. The files are extracted by running the
% \xfile{.dtx} through \plainTeX:
% \begin{quote}
%   \verb|tex twoopt.dtx|
% \end{quote}
%
% \paragraph{TDS.} Now the different files must be moved into
% the different directories in your installation TDS tree
% (also known as \xfile{texmf} tree):
% \begin{quote}
% \def\t{^^A
% \begin{tabular}{@{}>{\ttfamily}l@{ $\rightarrow$ }>{\ttfamily}l@{}}
%   twoopt.sty & tex/latex/oberdiek/twoopt.sty\\
%   twoopt.pdf & doc/latex/oberdiek/twoopt.pdf\\
%   twoopt.dtx & source/latex/oberdiek/twoopt.dtx\\
% \end{tabular}^^A
% }^^A
% \sbox0{\t}^^A
% \ifdim\wd0>\linewidth
%   \begingroup
%     \advance\linewidth by\leftmargin
%     \advance\linewidth by\rightmargin
%   \edef\x{\endgroup
%     \def\noexpand\lw{\the\linewidth}^^A
%   }\x
%   \def\lwbox{^^A
%     \leavevmode
%     \hbox to \linewidth{^^A
%       \kern-\leftmargin\relax
%       \hss
%       \usebox0
%       \hss
%       \kern-\rightmargin\relax
%     }^^A
%   }^^A
%   \ifdim\wd0>\lw
%     \sbox0{\small\t}^^A
%     \ifdim\wd0>\linewidth
%       \ifdim\wd0>\lw
%         \sbox0{\footnotesize\t}^^A
%         \ifdim\wd0>\linewidth
%           \ifdim\wd0>\lw
%             \sbox0{\scriptsize\t}^^A
%             \ifdim\wd0>\linewidth
%               \ifdim\wd0>\lw
%                 \sbox0{\tiny\t}^^A
%                 \ifdim\wd0>\linewidth
%                   \lwbox
%                 \else
%                   \usebox0
%                 \fi
%               \else
%                 \lwbox
%               \fi
%             \else
%               \usebox0
%             \fi
%           \else
%             \lwbox
%           \fi
%         \else
%           \usebox0
%         \fi
%       \else
%         \lwbox
%       \fi
%     \else
%       \usebox0
%     \fi
%   \else
%     \lwbox
%   \fi
% \else
%   \usebox0
% \fi
% \end{quote}
% If you have a \xfile{docstrip.cfg} that configures and enables \docstrip's
% TDS installing feature, then some files can already be in the right
% place, see the documentation of \docstrip.
%
% \subsection{Refresh file name databases}
%
% If your \TeX~distribution
% (\teTeX, \mikTeX, \dots) relies on file name databases, you must refresh
% these. For example, \teTeX\ users run \verb|texhash| or
% \verb|mktexlsr|.
%
% \subsection{Some details for the interested}
%
% \paragraph{Attached source.}
%
% The PDF documentation on CTAN also includes the
% \xfile{.dtx} source file. It can be extracted by
% AcrobatReader 6 or higher. Another option is \textsf{pdftk},
% e.g. unpack the file into the current directory:
% \begin{quote}
%   \verb|pdftk twoopt.pdf unpack_files output .|
% \end{quote}
%
% \paragraph{Unpacking with \LaTeX.}
% The \xfile{.dtx} chooses its action depending on the format:
% \begin{description}
% \item[\plainTeX:] Run \docstrip\ and extract the files.
% \item[\LaTeX:] Generate the documentation.
% \end{description}
% If you insist on using \LaTeX\ for \docstrip\ (really,
% \docstrip\ does not need \LaTeX), then inform the autodetect routine
% about your intention:
% \begin{quote}
%   \verb|latex \let\install=y% \iffalse meta-comment
%
% File: twoopt.dtx
% Version: 2016/05/16 v1.6
% Info: Definitions with two optional arguments
%
% Copyright (C) 1999, 2006, 2008 by
%    Heiko Oberdiek <heiko.oberdiek at googlemail.com>
%    2016
%    https://github.com/ho-tex/oberdiek/issues
%
% This work may be distributed and/or modified under the
% conditions of the LaTeX Project Public License, either
% version 1.3c of this license or (at your option) any later
% version. This version of this license is in
%    http://www.latex-project.org/lppl/lppl-1-3c.txt
% and the latest version of this license is in
%    http://www.latex-project.org/lppl.txt
% and version 1.3 or later is part of all distributions of
% LaTeX version 2005/12/01 or later.
%
% This work has the LPPL maintenance status "maintained".
%
% This Current Maintainer of this work is Heiko Oberdiek.
%
% This work consists of the main source file twoopt.dtx
% and the derived files
%    twoopt.sty, twoopt.pdf, twoopt.ins, twoopt.drv.
%
% Distribution:
%    CTAN:macros/latex/contrib/oberdiek/twoopt.dtx
%    CTAN:macros/latex/contrib/oberdiek/twoopt.pdf
%
% Unpacking:
%    (a) If twoopt.ins is present:
%           tex twoopt.ins
%    (b) Without twoopt.ins:
%           tex twoopt.dtx
%    (c) If you insist on using LaTeX
%           latex \let\install=y% \iffalse meta-comment
%
% File: twoopt.dtx
% Version: 2016/05/16 v1.6
% Info: Definitions with two optional arguments
%
% Copyright (C) 1999, 2006, 2008 by
%    Heiko Oberdiek <heiko.oberdiek at googlemail.com>
%    2016
%    https://github.com/ho-tex/oberdiek/issues
%
% This work may be distributed and/or modified under the
% conditions of the LaTeX Project Public License, either
% version 1.3c of this license or (at your option) any later
% version. This version of this license is in
%    http://www.latex-project.org/lppl/lppl-1-3c.txt
% and the latest version of this license is in
%    http://www.latex-project.org/lppl.txt
% and version 1.3 or later is part of all distributions of
% LaTeX version 2005/12/01 or later.
%
% This work has the LPPL maintenance status "maintained".
%
% This Current Maintainer of this work is Heiko Oberdiek.
%
% This work consists of the main source file twoopt.dtx
% and the derived files
%    twoopt.sty, twoopt.pdf, twoopt.ins, twoopt.drv.
%
% Distribution:
%    CTAN:macros/latex/contrib/oberdiek/twoopt.dtx
%    CTAN:macros/latex/contrib/oberdiek/twoopt.pdf
%
% Unpacking:
%    (a) If twoopt.ins is present:
%           tex twoopt.ins
%    (b) Without twoopt.ins:
%           tex twoopt.dtx
%    (c) If you insist on using LaTeX
%           latex \let\install=y\input{twoopt.dtx}
%        (quote the arguments according to the demands of your shell)
%
% Documentation:
%    (a) If twoopt.drv is present:
%           latex twoopt.drv
%    (b) Without twoopt.drv:
%           latex twoopt.dtx; ...
%    The class ltxdoc loads the configuration file ltxdoc.cfg
%    if available. Here you can specify further options, e.g.
%    use A4 as paper format:
%       \PassOptionsToClass{a4paper}{article}
%
%    Programm calls to get the documentation (example):
%       pdflatex twoopt.dtx
%       makeindex -s gind.ist twoopt.idx
%       pdflatex twoopt.dtx
%       makeindex -s gind.ist twoopt.idx
%       pdflatex twoopt.dtx
%
% Installation:
%    TDS:tex/latex/oberdiek/twoopt.sty
%    TDS:doc/latex/oberdiek/twoopt.pdf
%    TDS:source/latex/oberdiek/twoopt.dtx
%
%<*ignore>
\begingroup
  \catcode123=1 %
  \catcode125=2 %
  \def\x{LaTeX2e}%
\expandafter\endgroup
\ifcase 0\ifx\install y1\fi\expandafter
         \ifx\csname processbatchFile\endcsname\relax\else1\fi
         \ifx\fmtname\x\else 1\fi\relax
\else\csname fi\endcsname
%</ignore>
%<*install>
\input docstrip.tex
\Msg{************************************************************************}
\Msg{* Installation}
\Msg{* Package: twoopt 2016/05/16 v1.6 Definitions with two optional arguments (HO)}
\Msg{************************************************************************}

\keepsilent
\askforoverwritefalse

\let\MetaPrefix\relax
\preamble

This is a generated file.

Project: twoopt
Version: 2016/05/16 v1.6

Copyright (C) 1999, 2006, 2008 by
   Heiko Oberdiek <heiko.oberdiek at googlemail.com>

This work may be distributed and/or modified under the
conditions of the LaTeX Project Public License, either
version 1.3c of this license or (at your option) any later
version. This version of this license is in
   http://www.latex-project.org/lppl/lppl-1-3c.txt
and the latest version of this license is in
   http://www.latex-project.org/lppl.txt
and version 1.3 or later is part of all distributions of
LaTeX version 2005/12/01 or later.

This work has the LPPL maintenance status "maintained".

This Current Maintainer of this work is Heiko Oberdiek.

This work consists of the main source file twoopt.dtx
and the derived files
   twoopt.sty, twoopt.pdf, twoopt.ins, twoopt.drv.

\endpreamble
\let\MetaPrefix\DoubleperCent

\generate{%
  \file{twoopt.ins}{\from{twoopt.dtx}{install}}%
  \file{twoopt.drv}{\from{twoopt.dtx}{driver}}%
  \usedir{tex/latex/oberdiek}%
  \file{twoopt.sty}{\from{twoopt.dtx}{package}}%
  \nopreamble
  \nopostamble
%  \usedir{source/latex/oberdiek/catalogue}%
%  \file{twoopt.xml}{\from{twoopt.dtx}{catalogue}}%
}

\catcode32=13\relax% active space
\let =\space%
\Msg{************************************************************************}
\Msg{*}
\Msg{* To finish the installation you have to move the following}
\Msg{* file into a directory searched by TeX:}
\Msg{*}
\Msg{*     twoopt.sty}
\Msg{*}
\Msg{* To produce the documentation run the file `twoopt.drv'}
\Msg{* through LaTeX.}
\Msg{*}
\Msg{* Happy TeXing!}
\Msg{*}
\Msg{************************************************************************}

\endbatchfile
%</install>
%<*ignore>
\fi
%</ignore>
%<*driver>
\NeedsTeXFormat{LaTeX2e}
\ProvidesFile{twoopt.drv}%
  [2016/05/16 v1.6 Definitions with two optional arguments (HO)]%
\documentclass{ltxdoc}
\usepackage{holtxdoc}[2011/11/22]
\begin{document}
  \DocInput{twoopt.dtx}%
\end{document}
%</driver>
% \fi
%
%
% \CharacterTable
%  {Upper-case    \A\B\C\D\E\F\G\H\I\J\K\L\M\N\O\P\Q\R\S\T\U\V\W\X\Y\Z
%   Lower-case    \a\b\c\d\e\f\g\h\i\j\k\l\m\n\o\p\q\r\s\t\u\v\w\x\y\z
%   Digits        \0\1\2\3\4\5\6\7\8\9
%   Exclamation   \!     Double quote  \"     Hash (number) \#
%   Dollar        \$     Percent       \%     Ampersand     \&
%   Acute accent  \'     Left paren    \(     Right paren   \)
%   Asterisk      \*     Plus          \+     Comma         \,
%   Minus         \-     Point         \.     Solidus       \/
%   Colon         \:     Semicolon     \;     Less than     \<
%   Equals        \=     Greater than  \>     Question mark \?
%   Commercial at \@     Left bracket  \[     Backslash     \\
%   Right bracket \]     Circumflex    \^     Underscore    \_
%   Grave accent  \`     Left brace    \{     Vertical bar  \|
%   Right brace   \}     Tilde         \~}
%
% \GetFileInfo{twoopt.drv}
%
% \title{The \xpackage{twoopt} package}
% \date{2016/05/16 v1.6}
% \author{Heiko Oberdiek\thanks
% {Please report any issues at https://github.com/ho-tex/oberdiek/issues}\\
% \xemail{heiko.oberdiek at googlemail.com}}
%
% \maketitle
%
% \begin{abstract}
% This package provides commands to define macros with two
% optional arguments.
% \end{abstract}
%
% \tableofcontents
%
% \newenvironment{param}{^^A
%   \newcommand{\entry}[1]{\meta{\###1}:&}^^A
%   \begin{tabular}[t]{@{}l@{ }l@{}}^^A
% }{^^A
%   \end{tabular}^^A
% }
%
% \section{Usage}
%    \DescribeMacro{\newcommandtwoopt}
%    \DescribeMacro{\renewcommandtwoopt}
%    \DescribeMacro{\providecommandtwoopt}
%    Similar to \cmd{\newcommand}, \cmd{\renewcommand}
%    and \cmd{\providecommand} this package provides commands
%    to define macros with two optional arguments.
%    The names of the commands are built by appending the
%    package name to the \LaTeX-pendants:
%    \begingroup
%      \def\x{\marg{cmd} \oarg{num} \oarg{default1}^^A
%             \oarg{default2} \marg{def.}}^^A
%      \begin{tabbing}
%        \cmd{\providecommandtwoopt} \=\kill
%        \cmd{\newcommandtwoopt}\>\x\\
%        \cmd{\renewcommandtwoopt}\>\x\\
%        \cmd{\providecommandtwoopt}\>\x\\
%      \end{tabbing}
%    \endgroup
%
%    Also the |*|-forms are supported. Indeed it is better to
%    use this ones, unless it is intended to hold
%    whole paragraphs in some of the arguments. If the macro
%    is defined with the |*|-form, missing braces
%    can be detected earlier.
%
%    Example:
%    \begin{quote}
%      |\newcommandtwoopt{\bsp}[3][AA][BB]{%|\\
%      |  \typeout{\string\bsp: #1,#2,#3}%|\\
%      |}|\\
%      \begin{tabular}{@{}l@{\quad$\rightarrow$\quad}l@{}}
%      |\bsp[aa][bb]{cc}|&|\bsp: aa,bb,cc|\\
%      |\bsp[aa]{cc}|&|\bsp: aa,BB,cc|\\
%      |\bsp{cc}|&|\bsp: AA,BB,cc|\\
%      \end{tabular}
%    \end{quote}
%
% \StopEventually{
% }
%
% \section{Implementation}
%    \begin{macrocode}
%<*package>
\NeedsTeXFormat{LaTeX2e}
\ProvidesPackage{twoopt}
  [2016/05/16 v1.6 Definitions with two optional arguments (HO)]%
%    \end{macrocode}
%    \begin{macro}{\newcommandtwoopt}
%    \begin{macrocode}
\newcommand{\newcommandtwoopt}{%
  \@ifstar{\@newcommandtwoopt*}{\@newcommandtwoopt{}}%
}
%    \end{macrocode}
%    \end{macro}
%
%    \begin{macro}{\@newcommandtwoopt}
%    \begin{param}
%      \entry1 star\\
%      \entry2 macro name to be defined
%    \end{param}
%    \begin{macrocode}
\newcommand{\@newcommandtwoopt}{}
\long\def\@newcommandtwoopt#1#2{%
  \expandafter\@@newcommandtwoopt
    \csname2\string#2\endcsname{#1}{#2}%
}
%    \end{macrocode}
%    \end{macro}
%
%    \begin{macro}{\@@newcommandtwoopt}
%    \begin{param}
%      \entry1 help command to be defined
%        (\expandafter\cmd\csname 2\bslash<name>\endcsname)\\
%      \entry2 star\\
%      \entry3 macro name to be defined\\
%      \entry4 number of total arguments\\
%      \entry5 default for optional argument one\\
%      \entry6 default for optional argument two
%    \end{param}
%    \begin{macrocode}
\newcommand{\@@newcommandtwoopt}{}
\long\def\@@newcommandtwoopt#1#2#3[#4][#5][#6]{%
  \newcommand#2#3[1][{#5}]{%
    \to@ScanSecondOptArg#1{##1}{#6}%
  }%
  \newcommand#2#1[{#4}]%
}
%    \end{macrocode}
%    \end{macro}
%
%    \begin{macro}{\renewcommandtwoopt}
%    \begin{macrocode}
\newcommand{\renewcommandtwoopt}{%
  \@ifstar{\@renewcommandtwoopt*}{\@renewcommandtwoopt{}}%
}
%    \end{macrocode}
%    \end{macro}
%
%    \begin{macro}{\@renewcommandtwoopt}
%    \begin{param}
%      \entry1 star\\
%      \entry2 command name to be defined
%    \end{param}
%    \begin{macrocode}
\newcommand{\@renewcommandtwoopt}{}
\long\def\@renewcommandtwoopt#1#2{%
  \begingroup
    \escapechar\m@ne
    \xdef\@gtempa{{\string#2}}%
  \endgroup
  \expandafter\@ifundefined\@gtempa{%
    \@latex@error{\noexpand#2undefined}\@ehc
  }{}%
  \let#2\@undefined
  \expandafter\let\csname2\string#2\endcsname\@undefined
  \expandafter\@@newcommandtwoopt
    \csname2\string#2\endcsname{#1}{#2}%
}
%    \end{macrocode}
%    \end{macro}
%
%    \begin{macro}{\providecommandtwoopt}
%    \begin{macrocode}
\newcommand{\providecommandtwoopt}{%
  \@ifstar{\@providecommandtwoopt*}{\@providecommandtwoopt{}}%
}
%    \end{macrocode}
%    \end{macro}
%
%    \begin{macro}{\@providecommandtwoopt}
%    \begin{param}
%      \entry1 star\\
%      \entry2 command name to be defined
%    \end{param}
%    \begin{macrocode}
\newcommand{\@providecommandtwoopt}{}
\long\def\@providecommandtwoopt#1#2{%
  \begingroup
    \escapechar\m@ne
    \xdef\@gtempa{{\string#2}}%
  \endgroup
  \expandafter\@ifundefined\@gtempa{%
    \expandafter\@@newcommandtwoopt
      \csname2\string#2\endcsname{#1}{#2}%
  }{%
    \let\to@dummyA\@undefined
    \let\to@dummyB\@undefined
    \@@newcommandtwoopt\to@dummyA{#1}\to@dummyB
  }%
}
%    \end{macrocode}
%    \end{macro}
%
%    \begin{macro}{\to@ScanSecondOptArg}
%    \begin{param}
%      \entry1 help command to be defined
%        (\expandafter\cmd\csname 2\bslash<name>\endcsname)\\
%      \entry2 first arg of command to be defined\\
%      \entry3 default for second opt. arg.
%    \end{param}
%    \begin{macrocode}
\newcommand{\to@ScanSecondOptArg}[3]{%
  \@ifnextchar[{%
    \expandafter#1\to@ArgOptToArgArg{#2}%
  }{%
    #1{#2}{#3}%
  }%
}
%    \end{macrocode}
%    \end{macro}
%
%    \begin{macro}{\to@ArgOptToArgArg}
%    \begin{macrocode}
\newcommand{\to@ArgOptToArgArg}{}
\long\def\to@ArgOptToArgArg#1[#2]{{#1}{#2}}
%    \end{macrocode}
%    \end{macro}
%
%    \begin{macrocode}
%</package>
%    \end{macrocode}
%
% \section{Installation}
%
% \subsection{Download}
%
% \paragraph{Package.} This package is available on
% CTAN\footnote{\url{http://ctan.org/pkg/twoopt}}:
% \begin{description}
% \item[\CTAN{macros/latex/contrib/oberdiek/twoopt.dtx}] The source file.
% \item[\CTAN{macros/latex/contrib/oberdiek/twoopt.pdf}] Documentation.
% \end{description}
%
%
% \paragraph{Bundle.} All the packages of the bundle `oberdiek'
% are also available in a TDS compliant ZIP archive. There
% the packages are already unpacked and the documentation files
% are generated. The files and directories obey the TDS standard.
% \begin{description}
% \item[\CTAN{install/macros/latex/contrib/oberdiek.tds.zip}]
% \end{description}
% \emph{TDS} refers to the standard ``A Directory Structure
% for \TeX\ Files'' (\CTAN{tds/tds.pdf}). Directories
% with \xfile{texmf} in their name are usually organized this way.
%
% \subsection{Bundle installation}
%
% \paragraph{Unpacking.} Unpack the \xfile{oberdiek.tds.zip} in the
% TDS tree (also known as \xfile{texmf} tree) of your choice.
% Example (linux):
% \begin{quote}
%   |unzip oberdiek.tds.zip -d ~/texmf|
% \end{quote}
%
% \paragraph{Script installation.}
% Check the directory \xfile{TDS:scripts/oberdiek/} for
% scripts that need further installation steps.
% Package \xpackage{attachfile2} comes with the Perl script
% \xfile{pdfatfi.pl} that should be installed in such a way
% that it can be called as \texttt{pdfatfi}.
% Example (linux):
% \begin{quote}
%   |chmod +x scripts/oberdiek/pdfatfi.pl|\\
%   |cp scripts/oberdiek/pdfatfi.pl /usr/local/bin/|
% \end{quote}
%
% \subsection{Package installation}
%
% \paragraph{Unpacking.} The \xfile{.dtx} file is a self-extracting
% \docstrip\ archive. The files are extracted by running the
% \xfile{.dtx} through \plainTeX:
% \begin{quote}
%   \verb|tex twoopt.dtx|
% \end{quote}
%
% \paragraph{TDS.} Now the different files must be moved into
% the different directories in your installation TDS tree
% (also known as \xfile{texmf} tree):
% \begin{quote}
% \def\t{^^A
% \begin{tabular}{@{}>{\ttfamily}l@{ $\rightarrow$ }>{\ttfamily}l@{}}
%   twoopt.sty & tex/latex/oberdiek/twoopt.sty\\
%   twoopt.pdf & doc/latex/oberdiek/twoopt.pdf\\
%   twoopt.dtx & source/latex/oberdiek/twoopt.dtx\\
% \end{tabular}^^A
% }^^A
% \sbox0{\t}^^A
% \ifdim\wd0>\linewidth
%   \begingroup
%     \advance\linewidth by\leftmargin
%     \advance\linewidth by\rightmargin
%   \edef\x{\endgroup
%     \def\noexpand\lw{\the\linewidth}^^A
%   }\x
%   \def\lwbox{^^A
%     \leavevmode
%     \hbox to \linewidth{^^A
%       \kern-\leftmargin\relax
%       \hss
%       \usebox0
%       \hss
%       \kern-\rightmargin\relax
%     }^^A
%   }^^A
%   \ifdim\wd0>\lw
%     \sbox0{\small\t}^^A
%     \ifdim\wd0>\linewidth
%       \ifdim\wd0>\lw
%         \sbox0{\footnotesize\t}^^A
%         \ifdim\wd0>\linewidth
%           \ifdim\wd0>\lw
%             \sbox0{\scriptsize\t}^^A
%             \ifdim\wd0>\linewidth
%               \ifdim\wd0>\lw
%                 \sbox0{\tiny\t}^^A
%                 \ifdim\wd0>\linewidth
%                   \lwbox
%                 \else
%                   \usebox0
%                 \fi
%               \else
%                 \lwbox
%               \fi
%             \else
%               \usebox0
%             \fi
%           \else
%             \lwbox
%           \fi
%         \else
%           \usebox0
%         \fi
%       \else
%         \lwbox
%       \fi
%     \else
%       \usebox0
%     \fi
%   \else
%     \lwbox
%   \fi
% \else
%   \usebox0
% \fi
% \end{quote}
% If you have a \xfile{docstrip.cfg} that configures and enables \docstrip's
% TDS installing feature, then some files can already be in the right
% place, see the documentation of \docstrip.
%
% \subsection{Refresh file name databases}
%
% If your \TeX~distribution
% (\teTeX, \mikTeX, \dots) relies on file name databases, you must refresh
% these. For example, \teTeX\ users run \verb|texhash| or
% \verb|mktexlsr|.
%
% \subsection{Some details for the interested}
%
% \paragraph{Attached source.}
%
% The PDF documentation on CTAN also includes the
% \xfile{.dtx} source file. It can be extracted by
% AcrobatReader 6 or higher. Another option is \textsf{pdftk},
% e.g. unpack the file into the current directory:
% \begin{quote}
%   \verb|pdftk twoopt.pdf unpack_files output .|
% \end{quote}
%
% \paragraph{Unpacking with \LaTeX.}
% The \xfile{.dtx} chooses its action depending on the format:
% \begin{description}
% \item[\plainTeX:] Run \docstrip\ and extract the files.
% \item[\LaTeX:] Generate the documentation.
% \end{description}
% If you insist on using \LaTeX\ for \docstrip\ (really,
% \docstrip\ does not need \LaTeX), then inform the autodetect routine
% about your intention:
% \begin{quote}
%   \verb|latex \let\install=y\input{twoopt.dtx}|
% \end{quote}
% Do not forget to quote the argument according to the demands
% of your shell.
%
% \paragraph{Generating the documentation.}
% You can use both the \xfile{.dtx} or the \xfile{.drv} to generate
% the documentation. The process can be configured by the
% configuration file \xfile{ltxdoc.cfg}. For instance, put this
% line into this file, if you want to have A4 as paper format:
% \begin{quote}
%   \verb|\PassOptionsToClass{a4paper}{article}|
% \end{quote}
% An example follows how to generate the
% documentation with pdf\LaTeX:
% \begin{quote}
%\begin{verbatim}
%pdflatex twoopt.dtx
%makeindex -s gind.ist twoopt.idx
%pdflatex twoopt.dtx
%makeindex -s gind.ist twoopt.idx
%pdflatex twoopt.dtx
%\end{verbatim}
% \end{quote}
%
% \section{Catalogue}
%
% The following XML file can be used as source for the
% \href{http://mirror.ctan.org/help/Catalogue/catalogue.html}{\TeX\ Catalogue}.
% The elements \texttt{caption} and \texttt{description} are imported
% from the original XML file from the Catalogue.
% The name of the XML file in the Catalogue is \xfile{twoopt.xml}.
%    \begin{macrocode}
%<*catalogue>
<?xml version='1.0' encoding='us-ascii'?>
<!DOCTYPE entry SYSTEM 'catalogue.dtd'>
<entry datestamp='$Date$' modifier='$Author$' id='twoopt'>
  <name>twoopt</name>
  <caption>Definitions with two optional arguments.</caption>
  <authorref id='auth:oberdiek'/>
  <copyright owner='Heiko Oberdiek' year='1999,2006,2008'/>
  <license type='lppl1.3'/>
  <version number='1.6'/>
  <description>
    Variants of <tt>\newcommand</tt>, <tt>\renewcommand</tt> and
    <tt>\providecommand</tt> are provided.
    <p/>
    The package is part of the <xref refid='oberdiek'>oberdiek</xref>
    bundle.
  </description>
  <documentation details='Package documentation'
      href='ctan:/macros/latex/contrib/oberdiek/twoopt.pdf'/>
  <ctan file='true' path='/macros/latex/contrib/oberdiek/twoopt.dtx'/>
  <miktex location='oberdiek'/>
  <texlive location='oberdiek'/>
  <install path='/macros/latex/contrib/oberdiek/oberdiek.tds.zip'/>
</entry>
%</catalogue>
%    \end{macrocode}
%
% \begin{History}
%   \begin{Version}{1998/10/30 v1.0}
%   \item
%     The first version was built as a response to a question
%     of \NameEmail{Rebecca and Rowland}{rebecca@astrid.u-net.com},
%     published in the newsgroup
%     \href{news:comp.text.tex}{comp.text.tex}:\\
%     \URL{``Re: [Q] LaTeX command with two optional arguments?''}^^A
%     {http://groups.google.com/group/comp.text.tex/msg/0ab1afde7b172d37}
%   \end{Version}
%   \begin{Version}{1998/10/30 v1.1}
%   \item
%     Improvements added in response to
%     \NameEmail{Stefan Ulrich}{ulrich@cis.uni-muenchen.de}
%     in the same thread:\\
%     \URL{``Re: [Q] LaTeX command with two optional arguments?''}^^A
%     {http://groups.google.com/group/comp.text.tex/msg/b8d84d4336f302c4}
%   \end{Version}
%   \begin{Version}{1998/11/04 v1.2}
%   \item
%     Fixes for LaTeX bugs 2896, 2901, 2902 added.
%   \end{Version}
%   \begin{Version}{1999/04/12 v1.3}
%   \item
%     Fixes removed because of LaTeX [1998/12/01].
%   \item
%     Documentation in dtx format.
%   \item
%     Copyright: LPPL (\CTAN{macros/latex/base/lppl.txt})
%   \item
%     First CTAN release.
%   \end{Version}
%   \begin{Version}{2006/02/20 v1.4}
%   \item
%     Code is not changed.
%   \item
%     New DTX framework.
%   \item
%     LPPL 1.3
%   \end{Version}
%   \begin{Version}{2008/08/11 v1.5}
%   \item
%     Code is not changed.
%   \item
%     URLs updated from \texttt{www.dejanews.com}
%     to \texttt{groups.google.com}.
%   \end{Version}
%   \begin{Version}{2016/05/16 v1.6}
%   \item
%     Documentation updates.
%   \end{Version}
% \end{History}
%
% \PrintIndex
%
% \Finale
\endinput

%        (quote the arguments according to the demands of your shell)
%
% Documentation:
%    (a) If twoopt.drv is present:
%           latex twoopt.drv
%    (b) Without twoopt.drv:
%           latex twoopt.dtx; ...
%    The class ltxdoc loads the configuration file ltxdoc.cfg
%    if available. Here you can specify further options, e.g.
%    use A4 as paper format:
%       \PassOptionsToClass{a4paper}{article}
%
%    Programm calls to get the documentation (example):
%       pdflatex twoopt.dtx
%       makeindex -s gind.ist twoopt.idx
%       pdflatex twoopt.dtx
%       makeindex -s gind.ist twoopt.idx
%       pdflatex twoopt.dtx
%
% Installation:
%    TDS:tex/latex/oberdiek/twoopt.sty
%    TDS:doc/latex/oberdiek/twoopt.pdf
%    TDS:source/latex/oberdiek/twoopt.dtx
%
%<*ignore>
\begingroup
  \catcode123=1 %
  \catcode125=2 %
  \def\x{LaTeX2e}%
\expandafter\endgroup
\ifcase 0\ifx\install y1\fi\expandafter
         \ifx\csname processbatchFile\endcsname\relax\else1\fi
         \ifx\fmtname\x\else 1\fi\relax
\else\csname fi\endcsname
%</ignore>
%<*install>
\input docstrip.tex
\Msg{************************************************************************}
\Msg{* Installation}
\Msg{* Package: twoopt 2016/05/16 v1.6 Definitions with two optional arguments (HO)}
\Msg{************************************************************************}

\keepsilent
\askforoverwritefalse

\let\MetaPrefix\relax
\preamble

This is a generated file.

Project: twoopt
Version: 2016/05/16 v1.6

Copyright (C) 1999, 2006, 2008 by
   Heiko Oberdiek <heiko.oberdiek at googlemail.com>

This work may be distributed and/or modified under the
conditions of the LaTeX Project Public License, either
version 1.3c of this license or (at your option) any later
version. This version of this license is in
   http://www.latex-project.org/lppl/lppl-1-3c.txt
and the latest version of this license is in
   http://www.latex-project.org/lppl.txt
and version 1.3 or later is part of all distributions of
LaTeX version 2005/12/01 or later.

This work has the LPPL maintenance status "maintained".

This Current Maintainer of this work is Heiko Oberdiek.

This work consists of the main source file twoopt.dtx
and the derived files
   twoopt.sty, twoopt.pdf, twoopt.ins, twoopt.drv.

\endpreamble
\let\MetaPrefix\DoubleperCent

\generate{%
  \file{twoopt.ins}{\from{twoopt.dtx}{install}}%
  \file{twoopt.drv}{\from{twoopt.dtx}{driver}}%
  \usedir{tex/latex/oberdiek}%
  \file{twoopt.sty}{\from{twoopt.dtx}{package}}%
  \nopreamble
  \nopostamble
%  \usedir{source/latex/oberdiek/catalogue}%
%  \file{twoopt.xml}{\from{twoopt.dtx}{catalogue}}%
}

\catcode32=13\relax% active space
\let =\space%
\Msg{************************************************************************}
\Msg{*}
\Msg{* To finish the installation you have to move the following}
\Msg{* file into a directory searched by TeX:}
\Msg{*}
\Msg{*     twoopt.sty}
\Msg{*}
\Msg{* To produce the documentation run the file `twoopt.drv'}
\Msg{* through LaTeX.}
\Msg{*}
\Msg{* Happy TeXing!}
\Msg{*}
\Msg{************************************************************************}

\endbatchfile
%</install>
%<*ignore>
\fi
%</ignore>
%<*driver>
\NeedsTeXFormat{LaTeX2e}
\ProvidesFile{twoopt.drv}%
  [2016/05/16 v1.6 Definitions with two optional arguments (HO)]%
\documentclass{ltxdoc}
\usepackage{holtxdoc}[2011/11/22]
\begin{document}
  \DocInput{twoopt.dtx}%
\end{document}
%</driver>
% \fi
%
%
% \CharacterTable
%  {Upper-case    \A\B\C\D\E\F\G\H\I\J\K\L\M\N\O\P\Q\R\S\T\U\V\W\X\Y\Z
%   Lower-case    \a\b\c\d\e\f\g\h\i\j\k\l\m\n\o\p\q\r\s\t\u\v\w\x\y\z
%   Digits        \0\1\2\3\4\5\6\7\8\9
%   Exclamation   \!     Double quote  \"     Hash (number) \#
%   Dollar        \$     Percent       \%     Ampersand     \&
%   Acute accent  \'     Left paren    \(     Right paren   \)
%   Asterisk      \*     Plus          \+     Comma         \,
%   Minus         \-     Point         \.     Solidus       \/
%   Colon         \:     Semicolon     \;     Less than     \<
%   Equals        \=     Greater than  \>     Question mark \?
%   Commercial at \@     Left bracket  \[     Backslash     \\
%   Right bracket \]     Circumflex    \^     Underscore    \_
%   Grave accent  \`     Left brace    \{     Vertical bar  \|
%   Right brace   \}     Tilde         \~}
%
% \GetFileInfo{twoopt.drv}
%
% \title{The \xpackage{twoopt} package}
% \date{2016/05/16 v1.6}
% \author{Heiko Oberdiek\thanks
% {Please report any issues at https://github.com/ho-tex/oberdiek/issues}\\
% \xemail{heiko.oberdiek at googlemail.com}}
%
% \maketitle
%
% \begin{abstract}
% This package provides commands to define macros with two
% optional arguments.
% \end{abstract}
%
% \tableofcontents
%
% \newenvironment{param}{^^A
%   \newcommand{\entry}[1]{\meta{\###1}:&}^^A
%   \begin{tabular}[t]{@{}l@{ }l@{}}^^A
% }{^^A
%   \end{tabular}^^A
% }
%
% \section{Usage}
%    \DescribeMacro{\newcommandtwoopt}
%    \DescribeMacro{\renewcommandtwoopt}
%    \DescribeMacro{\providecommandtwoopt}
%    Similar to \cmd{\newcommand}, \cmd{\renewcommand}
%    and \cmd{\providecommand} this package provides commands
%    to define macros with two optional arguments.
%    The names of the commands are built by appending the
%    package name to the \LaTeX-pendants:
%    \begingroup
%      \def\x{\marg{cmd} \oarg{num} \oarg{default1}^^A
%             \oarg{default2} \marg{def.}}^^A
%      \begin{tabbing}
%        \cmd{\providecommandtwoopt} \=\kill
%        \cmd{\newcommandtwoopt}\>\x\\
%        \cmd{\renewcommandtwoopt}\>\x\\
%        \cmd{\providecommandtwoopt}\>\x\\
%      \end{tabbing}
%    \endgroup
%
%    Also the |*|-forms are supported. Indeed it is better to
%    use this ones, unless it is intended to hold
%    whole paragraphs in some of the arguments. If the macro
%    is defined with the |*|-form, missing braces
%    can be detected earlier.
%
%    Example:
%    \begin{quote}
%      |\newcommandtwoopt{\bsp}[3][AA][BB]{%|\\
%      |  \typeout{\string\bsp: #1,#2,#3}%|\\
%      |}|\\
%      \begin{tabular}{@{}l@{\quad$\rightarrow$\quad}l@{}}
%      |\bsp[aa][bb]{cc}|&|\bsp: aa,bb,cc|\\
%      |\bsp[aa]{cc}|&|\bsp: aa,BB,cc|\\
%      |\bsp{cc}|&|\bsp: AA,BB,cc|\\
%      \end{tabular}
%    \end{quote}
%
% \StopEventually{
% }
%
% \section{Implementation}
%    \begin{macrocode}
%<*package>
\NeedsTeXFormat{LaTeX2e}
\ProvidesPackage{twoopt}
  [2016/05/16 v1.6 Definitions with two optional arguments (HO)]%
%    \end{macrocode}
%    \begin{macro}{\newcommandtwoopt}
%    \begin{macrocode}
\newcommand{\newcommandtwoopt}{%
  \@ifstar{\@newcommandtwoopt*}{\@newcommandtwoopt{}}%
}
%    \end{macrocode}
%    \end{macro}
%
%    \begin{macro}{\@newcommandtwoopt}
%    \begin{param}
%      \entry1 star\\
%      \entry2 macro name to be defined
%    \end{param}
%    \begin{macrocode}
\newcommand{\@newcommandtwoopt}{}
\long\def\@newcommandtwoopt#1#2{%
  \expandafter\@@newcommandtwoopt
    \csname2\string#2\endcsname{#1}{#2}%
}
%    \end{macrocode}
%    \end{macro}
%
%    \begin{macro}{\@@newcommandtwoopt}
%    \begin{param}
%      \entry1 help command to be defined
%        (\expandafter\cmd\csname 2\bslash<name>\endcsname)\\
%      \entry2 star\\
%      \entry3 macro name to be defined\\
%      \entry4 number of total arguments\\
%      \entry5 default for optional argument one\\
%      \entry6 default for optional argument two
%    \end{param}
%    \begin{macrocode}
\newcommand{\@@newcommandtwoopt}{}
\long\def\@@newcommandtwoopt#1#2#3[#4][#5][#6]{%
  \newcommand#2#3[1][{#5}]{%
    \to@ScanSecondOptArg#1{##1}{#6}%
  }%
  \newcommand#2#1[{#4}]%
}
%    \end{macrocode}
%    \end{macro}
%
%    \begin{macro}{\renewcommandtwoopt}
%    \begin{macrocode}
\newcommand{\renewcommandtwoopt}{%
  \@ifstar{\@renewcommandtwoopt*}{\@renewcommandtwoopt{}}%
}
%    \end{macrocode}
%    \end{macro}
%
%    \begin{macro}{\@renewcommandtwoopt}
%    \begin{param}
%      \entry1 star\\
%      \entry2 command name to be defined
%    \end{param}
%    \begin{macrocode}
\newcommand{\@renewcommandtwoopt}{}
\long\def\@renewcommandtwoopt#1#2{%
  \begingroup
    \escapechar\m@ne
    \xdef\@gtempa{{\string#2}}%
  \endgroup
  \expandafter\@ifundefined\@gtempa{%
    \@latex@error{\noexpand#2undefined}\@ehc
  }{}%
  \let#2\@undefined
  \expandafter\let\csname2\string#2\endcsname\@undefined
  \expandafter\@@newcommandtwoopt
    \csname2\string#2\endcsname{#1}{#2}%
}
%    \end{macrocode}
%    \end{macro}
%
%    \begin{macro}{\providecommandtwoopt}
%    \begin{macrocode}
\newcommand{\providecommandtwoopt}{%
  \@ifstar{\@providecommandtwoopt*}{\@providecommandtwoopt{}}%
}
%    \end{macrocode}
%    \end{macro}
%
%    \begin{macro}{\@providecommandtwoopt}
%    \begin{param}
%      \entry1 star\\
%      \entry2 command name to be defined
%    \end{param}
%    \begin{macrocode}
\newcommand{\@providecommandtwoopt}{}
\long\def\@providecommandtwoopt#1#2{%
  \begingroup
    \escapechar\m@ne
    \xdef\@gtempa{{\string#2}}%
  \endgroup
  \expandafter\@ifundefined\@gtempa{%
    \expandafter\@@newcommandtwoopt
      \csname2\string#2\endcsname{#1}{#2}%
  }{%
    \let\to@dummyA\@undefined
    \let\to@dummyB\@undefined
    \@@newcommandtwoopt\to@dummyA{#1}\to@dummyB
  }%
}
%    \end{macrocode}
%    \end{macro}
%
%    \begin{macro}{\to@ScanSecondOptArg}
%    \begin{param}
%      \entry1 help command to be defined
%        (\expandafter\cmd\csname 2\bslash<name>\endcsname)\\
%      \entry2 first arg of command to be defined\\
%      \entry3 default for second opt. arg.
%    \end{param}
%    \begin{macrocode}
\newcommand{\to@ScanSecondOptArg}[3]{%
  \@ifnextchar[{%
    \expandafter#1\to@ArgOptToArgArg{#2}%
  }{%
    #1{#2}{#3}%
  }%
}
%    \end{macrocode}
%    \end{macro}
%
%    \begin{macro}{\to@ArgOptToArgArg}
%    \begin{macrocode}
\newcommand{\to@ArgOptToArgArg}{}
\long\def\to@ArgOptToArgArg#1[#2]{{#1}{#2}}
%    \end{macrocode}
%    \end{macro}
%
%    \begin{macrocode}
%</package>
%    \end{macrocode}
%
% \section{Installation}
%
% \subsection{Download}
%
% \paragraph{Package.} This package is available on
% CTAN\footnote{\url{http://ctan.org/pkg/twoopt}}:
% \begin{description}
% \item[\CTAN{macros/latex/contrib/oberdiek/twoopt.dtx}] The source file.
% \item[\CTAN{macros/latex/contrib/oberdiek/twoopt.pdf}] Documentation.
% \end{description}
%
%
% \paragraph{Bundle.} All the packages of the bundle `oberdiek'
% are also available in a TDS compliant ZIP archive. There
% the packages are already unpacked and the documentation files
% are generated. The files and directories obey the TDS standard.
% \begin{description}
% \item[\CTAN{install/macros/latex/contrib/oberdiek.tds.zip}]
% \end{description}
% \emph{TDS} refers to the standard ``A Directory Structure
% for \TeX\ Files'' (\CTAN{tds/tds.pdf}). Directories
% with \xfile{texmf} in their name are usually organized this way.
%
% \subsection{Bundle installation}
%
% \paragraph{Unpacking.} Unpack the \xfile{oberdiek.tds.zip} in the
% TDS tree (also known as \xfile{texmf} tree) of your choice.
% Example (linux):
% \begin{quote}
%   |unzip oberdiek.tds.zip -d ~/texmf|
% \end{quote}
%
% \paragraph{Script installation.}
% Check the directory \xfile{TDS:scripts/oberdiek/} for
% scripts that need further installation steps.
% Package \xpackage{attachfile2} comes with the Perl script
% \xfile{pdfatfi.pl} that should be installed in such a way
% that it can be called as \texttt{pdfatfi}.
% Example (linux):
% \begin{quote}
%   |chmod +x scripts/oberdiek/pdfatfi.pl|\\
%   |cp scripts/oberdiek/pdfatfi.pl /usr/local/bin/|
% \end{quote}
%
% \subsection{Package installation}
%
% \paragraph{Unpacking.} The \xfile{.dtx} file is a self-extracting
% \docstrip\ archive. The files are extracted by running the
% \xfile{.dtx} through \plainTeX:
% \begin{quote}
%   \verb|tex twoopt.dtx|
% \end{quote}
%
% \paragraph{TDS.} Now the different files must be moved into
% the different directories in your installation TDS tree
% (also known as \xfile{texmf} tree):
% \begin{quote}
% \def\t{^^A
% \begin{tabular}{@{}>{\ttfamily}l@{ $\rightarrow$ }>{\ttfamily}l@{}}
%   twoopt.sty & tex/latex/oberdiek/twoopt.sty\\
%   twoopt.pdf & doc/latex/oberdiek/twoopt.pdf\\
%   twoopt.dtx & source/latex/oberdiek/twoopt.dtx\\
% \end{tabular}^^A
% }^^A
% \sbox0{\t}^^A
% \ifdim\wd0>\linewidth
%   \begingroup
%     \advance\linewidth by\leftmargin
%     \advance\linewidth by\rightmargin
%   \edef\x{\endgroup
%     \def\noexpand\lw{\the\linewidth}^^A
%   }\x
%   \def\lwbox{^^A
%     \leavevmode
%     \hbox to \linewidth{^^A
%       \kern-\leftmargin\relax
%       \hss
%       \usebox0
%       \hss
%       \kern-\rightmargin\relax
%     }^^A
%   }^^A
%   \ifdim\wd0>\lw
%     \sbox0{\small\t}^^A
%     \ifdim\wd0>\linewidth
%       \ifdim\wd0>\lw
%         \sbox0{\footnotesize\t}^^A
%         \ifdim\wd0>\linewidth
%           \ifdim\wd0>\lw
%             \sbox0{\scriptsize\t}^^A
%             \ifdim\wd0>\linewidth
%               \ifdim\wd0>\lw
%                 \sbox0{\tiny\t}^^A
%                 \ifdim\wd0>\linewidth
%                   \lwbox
%                 \else
%                   \usebox0
%                 \fi
%               \else
%                 \lwbox
%               \fi
%             \else
%               \usebox0
%             \fi
%           \else
%             \lwbox
%           \fi
%         \else
%           \usebox0
%         \fi
%       \else
%         \lwbox
%       \fi
%     \else
%       \usebox0
%     \fi
%   \else
%     \lwbox
%   \fi
% \else
%   \usebox0
% \fi
% \end{quote}
% If you have a \xfile{docstrip.cfg} that configures and enables \docstrip's
% TDS installing feature, then some files can already be in the right
% place, see the documentation of \docstrip.
%
% \subsection{Refresh file name databases}
%
% If your \TeX~distribution
% (\teTeX, \mikTeX, \dots) relies on file name databases, you must refresh
% these. For example, \teTeX\ users run \verb|texhash| or
% \verb|mktexlsr|.
%
% \subsection{Some details for the interested}
%
% \paragraph{Attached source.}
%
% The PDF documentation on CTAN also includes the
% \xfile{.dtx} source file. It can be extracted by
% AcrobatReader 6 or higher. Another option is \textsf{pdftk},
% e.g. unpack the file into the current directory:
% \begin{quote}
%   \verb|pdftk twoopt.pdf unpack_files output .|
% \end{quote}
%
% \paragraph{Unpacking with \LaTeX.}
% The \xfile{.dtx} chooses its action depending on the format:
% \begin{description}
% \item[\plainTeX:] Run \docstrip\ and extract the files.
% \item[\LaTeX:] Generate the documentation.
% \end{description}
% If you insist on using \LaTeX\ for \docstrip\ (really,
% \docstrip\ does not need \LaTeX), then inform the autodetect routine
% about your intention:
% \begin{quote}
%   \verb|latex \let\install=y% \iffalse meta-comment
%
% File: twoopt.dtx
% Version: 2016/05/16 v1.6
% Info: Definitions with two optional arguments
%
% Copyright (C) 1999, 2006, 2008 by
%    Heiko Oberdiek <heiko.oberdiek at googlemail.com>
%    2016
%    https://github.com/ho-tex/oberdiek/issues
%
% This work may be distributed and/or modified under the
% conditions of the LaTeX Project Public License, either
% version 1.3c of this license or (at your option) any later
% version. This version of this license is in
%    http://www.latex-project.org/lppl/lppl-1-3c.txt
% and the latest version of this license is in
%    http://www.latex-project.org/lppl.txt
% and version 1.3 or later is part of all distributions of
% LaTeX version 2005/12/01 or later.
%
% This work has the LPPL maintenance status "maintained".
%
% This Current Maintainer of this work is Heiko Oberdiek.
%
% This work consists of the main source file twoopt.dtx
% and the derived files
%    twoopt.sty, twoopt.pdf, twoopt.ins, twoopt.drv.
%
% Distribution:
%    CTAN:macros/latex/contrib/oberdiek/twoopt.dtx
%    CTAN:macros/latex/contrib/oberdiek/twoopt.pdf
%
% Unpacking:
%    (a) If twoopt.ins is present:
%           tex twoopt.ins
%    (b) Without twoopt.ins:
%           tex twoopt.dtx
%    (c) If you insist on using LaTeX
%           latex \let\install=y\input{twoopt.dtx}
%        (quote the arguments according to the demands of your shell)
%
% Documentation:
%    (a) If twoopt.drv is present:
%           latex twoopt.drv
%    (b) Without twoopt.drv:
%           latex twoopt.dtx; ...
%    The class ltxdoc loads the configuration file ltxdoc.cfg
%    if available. Here you can specify further options, e.g.
%    use A4 as paper format:
%       \PassOptionsToClass{a4paper}{article}
%
%    Programm calls to get the documentation (example):
%       pdflatex twoopt.dtx
%       makeindex -s gind.ist twoopt.idx
%       pdflatex twoopt.dtx
%       makeindex -s gind.ist twoopt.idx
%       pdflatex twoopt.dtx
%
% Installation:
%    TDS:tex/latex/oberdiek/twoopt.sty
%    TDS:doc/latex/oberdiek/twoopt.pdf
%    TDS:source/latex/oberdiek/twoopt.dtx
%
%<*ignore>
\begingroup
  \catcode123=1 %
  \catcode125=2 %
  \def\x{LaTeX2e}%
\expandafter\endgroup
\ifcase 0\ifx\install y1\fi\expandafter
         \ifx\csname processbatchFile\endcsname\relax\else1\fi
         \ifx\fmtname\x\else 1\fi\relax
\else\csname fi\endcsname
%</ignore>
%<*install>
\input docstrip.tex
\Msg{************************************************************************}
\Msg{* Installation}
\Msg{* Package: twoopt 2016/05/16 v1.6 Definitions with two optional arguments (HO)}
\Msg{************************************************************************}

\keepsilent
\askforoverwritefalse

\let\MetaPrefix\relax
\preamble

This is a generated file.

Project: twoopt
Version: 2016/05/16 v1.6

Copyright (C) 1999, 2006, 2008 by
   Heiko Oberdiek <heiko.oberdiek at googlemail.com>

This work may be distributed and/or modified under the
conditions of the LaTeX Project Public License, either
version 1.3c of this license or (at your option) any later
version. This version of this license is in
   http://www.latex-project.org/lppl/lppl-1-3c.txt
and the latest version of this license is in
   http://www.latex-project.org/lppl.txt
and version 1.3 or later is part of all distributions of
LaTeX version 2005/12/01 or later.

This work has the LPPL maintenance status "maintained".

This Current Maintainer of this work is Heiko Oberdiek.

This work consists of the main source file twoopt.dtx
and the derived files
   twoopt.sty, twoopt.pdf, twoopt.ins, twoopt.drv.

\endpreamble
\let\MetaPrefix\DoubleperCent

\generate{%
  \file{twoopt.ins}{\from{twoopt.dtx}{install}}%
  \file{twoopt.drv}{\from{twoopt.dtx}{driver}}%
  \usedir{tex/latex/oberdiek}%
  \file{twoopt.sty}{\from{twoopt.dtx}{package}}%
  \nopreamble
  \nopostamble
%  \usedir{source/latex/oberdiek/catalogue}%
%  \file{twoopt.xml}{\from{twoopt.dtx}{catalogue}}%
}

\catcode32=13\relax% active space
\let =\space%
\Msg{************************************************************************}
\Msg{*}
\Msg{* To finish the installation you have to move the following}
\Msg{* file into a directory searched by TeX:}
\Msg{*}
\Msg{*     twoopt.sty}
\Msg{*}
\Msg{* To produce the documentation run the file `twoopt.drv'}
\Msg{* through LaTeX.}
\Msg{*}
\Msg{* Happy TeXing!}
\Msg{*}
\Msg{************************************************************************}

\endbatchfile
%</install>
%<*ignore>
\fi
%</ignore>
%<*driver>
\NeedsTeXFormat{LaTeX2e}
\ProvidesFile{twoopt.drv}%
  [2016/05/16 v1.6 Definitions with two optional arguments (HO)]%
\documentclass{ltxdoc}
\usepackage{holtxdoc}[2011/11/22]
\begin{document}
  \DocInput{twoopt.dtx}%
\end{document}
%</driver>
% \fi
%
%
% \CharacterTable
%  {Upper-case    \A\B\C\D\E\F\G\H\I\J\K\L\M\N\O\P\Q\R\S\T\U\V\W\X\Y\Z
%   Lower-case    \a\b\c\d\e\f\g\h\i\j\k\l\m\n\o\p\q\r\s\t\u\v\w\x\y\z
%   Digits        \0\1\2\3\4\5\6\7\8\9
%   Exclamation   \!     Double quote  \"     Hash (number) \#
%   Dollar        \$     Percent       \%     Ampersand     \&
%   Acute accent  \'     Left paren    \(     Right paren   \)
%   Asterisk      \*     Plus          \+     Comma         \,
%   Minus         \-     Point         \.     Solidus       \/
%   Colon         \:     Semicolon     \;     Less than     \<
%   Equals        \=     Greater than  \>     Question mark \?
%   Commercial at \@     Left bracket  \[     Backslash     \\
%   Right bracket \]     Circumflex    \^     Underscore    \_
%   Grave accent  \`     Left brace    \{     Vertical bar  \|
%   Right brace   \}     Tilde         \~}
%
% \GetFileInfo{twoopt.drv}
%
% \title{The \xpackage{twoopt} package}
% \date{2016/05/16 v1.6}
% \author{Heiko Oberdiek\thanks
% {Please report any issues at https://github.com/ho-tex/oberdiek/issues}\\
% \xemail{heiko.oberdiek at googlemail.com}}
%
% \maketitle
%
% \begin{abstract}
% This package provides commands to define macros with two
% optional arguments.
% \end{abstract}
%
% \tableofcontents
%
% \newenvironment{param}{^^A
%   \newcommand{\entry}[1]{\meta{\###1}:&}^^A
%   \begin{tabular}[t]{@{}l@{ }l@{}}^^A
% }{^^A
%   \end{tabular}^^A
% }
%
% \section{Usage}
%    \DescribeMacro{\newcommandtwoopt}
%    \DescribeMacro{\renewcommandtwoopt}
%    \DescribeMacro{\providecommandtwoopt}
%    Similar to \cmd{\newcommand}, \cmd{\renewcommand}
%    and \cmd{\providecommand} this package provides commands
%    to define macros with two optional arguments.
%    The names of the commands are built by appending the
%    package name to the \LaTeX-pendants:
%    \begingroup
%      \def\x{\marg{cmd} \oarg{num} \oarg{default1}^^A
%             \oarg{default2} \marg{def.}}^^A
%      \begin{tabbing}
%        \cmd{\providecommandtwoopt} \=\kill
%        \cmd{\newcommandtwoopt}\>\x\\
%        \cmd{\renewcommandtwoopt}\>\x\\
%        \cmd{\providecommandtwoopt}\>\x\\
%      \end{tabbing}
%    \endgroup
%
%    Also the |*|-forms are supported. Indeed it is better to
%    use this ones, unless it is intended to hold
%    whole paragraphs in some of the arguments. If the macro
%    is defined with the |*|-form, missing braces
%    can be detected earlier.
%
%    Example:
%    \begin{quote}
%      |\newcommandtwoopt{\bsp}[3][AA][BB]{%|\\
%      |  \typeout{\string\bsp: #1,#2,#3}%|\\
%      |}|\\
%      \begin{tabular}{@{}l@{\quad$\rightarrow$\quad}l@{}}
%      |\bsp[aa][bb]{cc}|&|\bsp: aa,bb,cc|\\
%      |\bsp[aa]{cc}|&|\bsp: aa,BB,cc|\\
%      |\bsp{cc}|&|\bsp: AA,BB,cc|\\
%      \end{tabular}
%    \end{quote}
%
% \StopEventually{
% }
%
% \section{Implementation}
%    \begin{macrocode}
%<*package>
\NeedsTeXFormat{LaTeX2e}
\ProvidesPackage{twoopt}
  [2016/05/16 v1.6 Definitions with two optional arguments (HO)]%
%    \end{macrocode}
%    \begin{macro}{\newcommandtwoopt}
%    \begin{macrocode}
\newcommand{\newcommandtwoopt}{%
  \@ifstar{\@newcommandtwoopt*}{\@newcommandtwoopt{}}%
}
%    \end{macrocode}
%    \end{macro}
%
%    \begin{macro}{\@newcommandtwoopt}
%    \begin{param}
%      \entry1 star\\
%      \entry2 macro name to be defined
%    \end{param}
%    \begin{macrocode}
\newcommand{\@newcommandtwoopt}{}
\long\def\@newcommandtwoopt#1#2{%
  \expandafter\@@newcommandtwoopt
    \csname2\string#2\endcsname{#1}{#2}%
}
%    \end{macrocode}
%    \end{macro}
%
%    \begin{macro}{\@@newcommandtwoopt}
%    \begin{param}
%      \entry1 help command to be defined
%        (\expandafter\cmd\csname 2\bslash<name>\endcsname)\\
%      \entry2 star\\
%      \entry3 macro name to be defined\\
%      \entry4 number of total arguments\\
%      \entry5 default for optional argument one\\
%      \entry6 default for optional argument two
%    \end{param}
%    \begin{macrocode}
\newcommand{\@@newcommandtwoopt}{}
\long\def\@@newcommandtwoopt#1#2#3[#4][#5][#6]{%
  \newcommand#2#3[1][{#5}]{%
    \to@ScanSecondOptArg#1{##1}{#6}%
  }%
  \newcommand#2#1[{#4}]%
}
%    \end{macrocode}
%    \end{macro}
%
%    \begin{macro}{\renewcommandtwoopt}
%    \begin{macrocode}
\newcommand{\renewcommandtwoopt}{%
  \@ifstar{\@renewcommandtwoopt*}{\@renewcommandtwoopt{}}%
}
%    \end{macrocode}
%    \end{macro}
%
%    \begin{macro}{\@renewcommandtwoopt}
%    \begin{param}
%      \entry1 star\\
%      \entry2 command name to be defined
%    \end{param}
%    \begin{macrocode}
\newcommand{\@renewcommandtwoopt}{}
\long\def\@renewcommandtwoopt#1#2{%
  \begingroup
    \escapechar\m@ne
    \xdef\@gtempa{{\string#2}}%
  \endgroup
  \expandafter\@ifundefined\@gtempa{%
    \@latex@error{\noexpand#2undefined}\@ehc
  }{}%
  \let#2\@undefined
  \expandafter\let\csname2\string#2\endcsname\@undefined
  \expandafter\@@newcommandtwoopt
    \csname2\string#2\endcsname{#1}{#2}%
}
%    \end{macrocode}
%    \end{macro}
%
%    \begin{macro}{\providecommandtwoopt}
%    \begin{macrocode}
\newcommand{\providecommandtwoopt}{%
  \@ifstar{\@providecommandtwoopt*}{\@providecommandtwoopt{}}%
}
%    \end{macrocode}
%    \end{macro}
%
%    \begin{macro}{\@providecommandtwoopt}
%    \begin{param}
%      \entry1 star\\
%      \entry2 command name to be defined
%    \end{param}
%    \begin{macrocode}
\newcommand{\@providecommandtwoopt}{}
\long\def\@providecommandtwoopt#1#2{%
  \begingroup
    \escapechar\m@ne
    \xdef\@gtempa{{\string#2}}%
  \endgroup
  \expandafter\@ifundefined\@gtempa{%
    \expandafter\@@newcommandtwoopt
      \csname2\string#2\endcsname{#1}{#2}%
  }{%
    \let\to@dummyA\@undefined
    \let\to@dummyB\@undefined
    \@@newcommandtwoopt\to@dummyA{#1}\to@dummyB
  }%
}
%    \end{macrocode}
%    \end{macro}
%
%    \begin{macro}{\to@ScanSecondOptArg}
%    \begin{param}
%      \entry1 help command to be defined
%        (\expandafter\cmd\csname 2\bslash<name>\endcsname)\\
%      \entry2 first arg of command to be defined\\
%      \entry3 default for second opt. arg.
%    \end{param}
%    \begin{macrocode}
\newcommand{\to@ScanSecondOptArg}[3]{%
  \@ifnextchar[{%
    \expandafter#1\to@ArgOptToArgArg{#2}%
  }{%
    #1{#2}{#3}%
  }%
}
%    \end{macrocode}
%    \end{macro}
%
%    \begin{macro}{\to@ArgOptToArgArg}
%    \begin{macrocode}
\newcommand{\to@ArgOptToArgArg}{}
\long\def\to@ArgOptToArgArg#1[#2]{{#1}{#2}}
%    \end{macrocode}
%    \end{macro}
%
%    \begin{macrocode}
%</package>
%    \end{macrocode}
%
% \section{Installation}
%
% \subsection{Download}
%
% \paragraph{Package.} This package is available on
% CTAN\footnote{\url{http://ctan.org/pkg/twoopt}}:
% \begin{description}
% \item[\CTAN{macros/latex/contrib/oberdiek/twoopt.dtx}] The source file.
% \item[\CTAN{macros/latex/contrib/oberdiek/twoopt.pdf}] Documentation.
% \end{description}
%
%
% \paragraph{Bundle.} All the packages of the bundle `oberdiek'
% are also available in a TDS compliant ZIP archive. There
% the packages are already unpacked and the documentation files
% are generated. The files and directories obey the TDS standard.
% \begin{description}
% \item[\CTAN{install/macros/latex/contrib/oberdiek.tds.zip}]
% \end{description}
% \emph{TDS} refers to the standard ``A Directory Structure
% for \TeX\ Files'' (\CTAN{tds/tds.pdf}). Directories
% with \xfile{texmf} in their name are usually organized this way.
%
% \subsection{Bundle installation}
%
% \paragraph{Unpacking.} Unpack the \xfile{oberdiek.tds.zip} in the
% TDS tree (also known as \xfile{texmf} tree) of your choice.
% Example (linux):
% \begin{quote}
%   |unzip oberdiek.tds.zip -d ~/texmf|
% \end{quote}
%
% \paragraph{Script installation.}
% Check the directory \xfile{TDS:scripts/oberdiek/} for
% scripts that need further installation steps.
% Package \xpackage{attachfile2} comes with the Perl script
% \xfile{pdfatfi.pl} that should be installed in such a way
% that it can be called as \texttt{pdfatfi}.
% Example (linux):
% \begin{quote}
%   |chmod +x scripts/oberdiek/pdfatfi.pl|\\
%   |cp scripts/oberdiek/pdfatfi.pl /usr/local/bin/|
% \end{quote}
%
% \subsection{Package installation}
%
% \paragraph{Unpacking.} The \xfile{.dtx} file is a self-extracting
% \docstrip\ archive. The files are extracted by running the
% \xfile{.dtx} through \plainTeX:
% \begin{quote}
%   \verb|tex twoopt.dtx|
% \end{quote}
%
% \paragraph{TDS.} Now the different files must be moved into
% the different directories in your installation TDS tree
% (also known as \xfile{texmf} tree):
% \begin{quote}
% \def\t{^^A
% \begin{tabular}{@{}>{\ttfamily}l@{ $\rightarrow$ }>{\ttfamily}l@{}}
%   twoopt.sty & tex/latex/oberdiek/twoopt.sty\\
%   twoopt.pdf & doc/latex/oberdiek/twoopt.pdf\\
%   twoopt.dtx & source/latex/oberdiek/twoopt.dtx\\
% \end{tabular}^^A
% }^^A
% \sbox0{\t}^^A
% \ifdim\wd0>\linewidth
%   \begingroup
%     \advance\linewidth by\leftmargin
%     \advance\linewidth by\rightmargin
%   \edef\x{\endgroup
%     \def\noexpand\lw{\the\linewidth}^^A
%   }\x
%   \def\lwbox{^^A
%     \leavevmode
%     \hbox to \linewidth{^^A
%       \kern-\leftmargin\relax
%       \hss
%       \usebox0
%       \hss
%       \kern-\rightmargin\relax
%     }^^A
%   }^^A
%   \ifdim\wd0>\lw
%     \sbox0{\small\t}^^A
%     \ifdim\wd0>\linewidth
%       \ifdim\wd0>\lw
%         \sbox0{\footnotesize\t}^^A
%         \ifdim\wd0>\linewidth
%           \ifdim\wd0>\lw
%             \sbox0{\scriptsize\t}^^A
%             \ifdim\wd0>\linewidth
%               \ifdim\wd0>\lw
%                 \sbox0{\tiny\t}^^A
%                 \ifdim\wd0>\linewidth
%                   \lwbox
%                 \else
%                   \usebox0
%                 \fi
%               \else
%                 \lwbox
%               \fi
%             \else
%               \usebox0
%             \fi
%           \else
%             \lwbox
%           \fi
%         \else
%           \usebox0
%         \fi
%       \else
%         \lwbox
%       \fi
%     \else
%       \usebox0
%     \fi
%   \else
%     \lwbox
%   \fi
% \else
%   \usebox0
% \fi
% \end{quote}
% If you have a \xfile{docstrip.cfg} that configures and enables \docstrip's
% TDS installing feature, then some files can already be in the right
% place, see the documentation of \docstrip.
%
% \subsection{Refresh file name databases}
%
% If your \TeX~distribution
% (\teTeX, \mikTeX, \dots) relies on file name databases, you must refresh
% these. For example, \teTeX\ users run \verb|texhash| or
% \verb|mktexlsr|.
%
% \subsection{Some details for the interested}
%
% \paragraph{Attached source.}
%
% The PDF documentation on CTAN also includes the
% \xfile{.dtx} source file. It can be extracted by
% AcrobatReader 6 or higher. Another option is \textsf{pdftk},
% e.g. unpack the file into the current directory:
% \begin{quote}
%   \verb|pdftk twoopt.pdf unpack_files output .|
% \end{quote}
%
% \paragraph{Unpacking with \LaTeX.}
% The \xfile{.dtx} chooses its action depending on the format:
% \begin{description}
% \item[\plainTeX:] Run \docstrip\ and extract the files.
% \item[\LaTeX:] Generate the documentation.
% \end{description}
% If you insist on using \LaTeX\ for \docstrip\ (really,
% \docstrip\ does not need \LaTeX), then inform the autodetect routine
% about your intention:
% \begin{quote}
%   \verb|latex \let\install=y\input{twoopt.dtx}|
% \end{quote}
% Do not forget to quote the argument according to the demands
% of your shell.
%
% \paragraph{Generating the documentation.}
% You can use both the \xfile{.dtx} or the \xfile{.drv} to generate
% the documentation. The process can be configured by the
% configuration file \xfile{ltxdoc.cfg}. For instance, put this
% line into this file, if you want to have A4 as paper format:
% \begin{quote}
%   \verb|\PassOptionsToClass{a4paper}{article}|
% \end{quote}
% An example follows how to generate the
% documentation with pdf\LaTeX:
% \begin{quote}
%\begin{verbatim}
%pdflatex twoopt.dtx
%makeindex -s gind.ist twoopt.idx
%pdflatex twoopt.dtx
%makeindex -s gind.ist twoopt.idx
%pdflatex twoopt.dtx
%\end{verbatim}
% \end{quote}
%
% \section{Catalogue}
%
% The following XML file can be used as source for the
% \href{http://mirror.ctan.org/help/Catalogue/catalogue.html}{\TeX\ Catalogue}.
% The elements \texttt{caption} and \texttt{description} are imported
% from the original XML file from the Catalogue.
% The name of the XML file in the Catalogue is \xfile{twoopt.xml}.
%    \begin{macrocode}
%<*catalogue>
<?xml version='1.0' encoding='us-ascii'?>
<!DOCTYPE entry SYSTEM 'catalogue.dtd'>
<entry datestamp='$Date$' modifier='$Author$' id='twoopt'>
  <name>twoopt</name>
  <caption>Definitions with two optional arguments.</caption>
  <authorref id='auth:oberdiek'/>
  <copyright owner='Heiko Oberdiek' year='1999,2006,2008'/>
  <license type='lppl1.3'/>
  <version number='1.6'/>
  <description>
    Variants of <tt>\newcommand</tt>, <tt>\renewcommand</tt> and
    <tt>\providecommand</tt> are provided.
    <p/>
    The package is part of the <xref refid='oberdiek'>oberdiek</xref>
    bundle.
  </description>
  <documentation details='Package documentation'
      href='ctan:/macros/latex/contrib/oberdiek/twoopt.pdf'/>
  <ctan file='true' path='/macros/latex/contrib/oberdiek/twoopt.dtx'/>
  <miktex location='oberdiek'/>
  <texlive location='oberdiek'/>
  <install path='/macros/latex/contrib/oberdiek/oberdiek.tds.zip'/>
</entry>
%</catalogue>
%    \end{macrocode}
%
% \begin{History}
%   \begin{Version}{1998/10/30 v1.0}
%   \item
%     The first version was built as a response to a question
%     of \NameEmail{Rebecca and Rowland}{rebecca@astrid.u-net.com},
%     published in the newsgroup
%     \href{news:comp.text.tex}{comp.text.tex}:\\
%     \URL{``Re: [Q] LaTeX command with two optional arguments?''}^^A
%     {http://groups.google.com/group/comp.text.tex/msg/0ab1afde7b172d37}
%   \end{Version}
%   \begin{Version}{1998/10/30 v1.1}
%   \item
%     Improvements added in response to
%     \NameEmail{Stefan Ulrich}{ulrich@cis.uni-muenchen.de}
%     in the same thread:\\
%     \URL{``Re: [Q] LaTeX command with two optional arguments?''}^^A
%     {http://groups.google.com/group/comp.text.tex/msg/b8d84d4336f302c4}
%   \end{Version}
%   \begin{Version}{1998/11/04 v1.2}
%   \item
%     Fixes for LaTeX bugs 2896, 2901, 2902 added.
%   \end{Version}
%   \begin{Version}{1999/04/12 v1.3}
%   \item
%     Fixes removed because of LaTeX [1998/12/01].
%   \item
%     Documentation in dtx format.
%   \item
%     Copyright: LPPL (\CTAN{macros/latex/base/lppl.txt})
%   \item
%     First CTAN release.
%   \end{Version}
%   \begin{Version}{2006/02/20 v1.4}
%   \item
%     Code is not changed.
%   \item
%     New DTX framework.
%   \item
%     LPPL 1.3
%   \end{Version}
%   \begin{Version}{2008/08/11 v1.5}
%   \item
%     Code is not changed.
%   \item
%     URLs updated from \texttt{www.dejanews.com}
%     to \texttt{groups.google.com}.
%   \end{Version}
%   \begin{Version}{2016/05/16 v1.6}
%   \item
%     Documentation updates.
%   \end{Version}
% \end{History}
%
% \PrintIndex
%
% \Finale
\endinput
|
% \end{quote}
% Do not forget to quote the argument according to the demands
% of your shell.
%
% \paragraph{Generating the documentation.}
% You can use both the \xfile{.dtx} or the \xfile{.drv} to generate
% the documentation. The process can be configured by the
% configuration file \xfile{ltxdoc.cfg}. For instance, put this
% line into this file, if you want to have A4 as paper format:
% \begin{quote}
%   \verb|\PassOptionsToClass{a4paper}{article}|
% \end{quote}
% An example follows how to generate the
% documentation with pdf\LaTeX:
% \begin{quote}
%\begin{verbatim}
%pdflatex twoopt.dtx
%makeindex -s gind.ist twoopt.idx
%pdflatex twoopt.dtx
%makeindex -s gind.ist twoopt.idx
%pdflatex twoopt.dtx
%\end{verbatim}
% \end{quote}
%
% \section{Catalogue}
%
% The following XML file can be used as source for the
% \href{http://mirror.ctan.org/help/Catalogue/catalogue.html}{\TeX\ Catalogue}.
% The elements \texttt{caption} and \texttt{description} are imported
% from the original XML file from the Catalogue.
% The name of the XML file in the Catalogue is \xfile{twoopt.xml}.
%    \begin{macrocode}
%<*catalogue>
<?xml version='1.0' encoding='us-ascii'?>
<!DOCTYPE entry SYSTEM 'catalogue.dtd'>
<entry datestamp='$Date$' modifier='$Author$' id='twoopt'>
  <name>twoopt</name>
  <caption>Definitions with two optional arguments.</caption>
  <authorref id='auth:oberdiek'/>
  <copyright owner='Heiko Oberdiek' year='1999,2006,2008'/>
  <license type='lppl1.3'/>
  <version number='1.6'/>
  <description>
    Variants of <tt>\newcommand</tt>, <tt>\renewcommand</tt> and
    <tt>\providecommand</tt> are provided.
    <p/>
    The package is part of the <xref refid='oberdiek'>oberdiek</xref>
    bundle.
  </description>
  <documentation details='Package documentation'
      href='ctan:/macros/latex/contrib/oberdiek/twoopt.pdf'/>
  <ctan file='true' path='/macros/latex/contrib/oberdiek/twoopt.dtx'/>
  <miktex location='oberdiek'/>
  <texlive location='oberdiek'/>
  <install path='/macros/latex/contrib/oberdiek/oberdiek.tds.zip'/>
</entry>
%</catalogue>
%    \end{macrocode}
%
% \begin{History}
%   \begin{Version}{1998/10/30 v1.0}
%   \item
%     The first version was built as a response to a question
%     of \NameEmail{Rebecca and Rowland}{rebecca@astrid.u-net.com},
%     published in the newsgroup
%     \href{news:comp.text.tex}{comp.text.tex}:\\
%     \URL{``Re: [Q] LaTeX command with two optional arguments?''}^^A
%     {http://groups.google.com/group/comp.text.tex/msg/0ab1afde7b172d37}
%   \end{Version}
%   \begin{Version}{1998/10/30 v1.1}
%   \item
%     Improvements added in response to
%     \NameEmail{Stefan Ulrich}{ulrich@cis.uni-muenchen.de}
%     in the same thread:\\
%     \URL{``Re: [Q] LaTeX command with two optional arguments?''}^^A
%     {http://groups.google.com/group/comp.text.tex/msg/b8d84d4336f302c4}
%   \end{Version}
%   \begin{Version}{1998/11/04 v1.2}
%   \item
%     Fixes for LaTeX bugs 2896, 2901, 2902 added.
%   \end{Version}
%   \begin{Version}{1999/04/12 v1.3}
%   \item
%     Fixes removed because of LaTeX [1998/12/01].
%   \item
%     Documentation in dtx format.
%   \item
%     Copyright: LPPL (\CTAN{macros/latex/base/lppl.txt})
%   \item
%     First CTAN release.
%   \end{Version}
%   \begin{Version}{2006/02/20 v1.4}
%   \item
%     Code is not changed.
%   \item
%     New DTX framework.
%   \item
%     LPPL 1.3
%   \end{Version}
%   \begin{Version}{2008/08/11 v1.5}
%   \item
%     Code is not changed.
%   \item
%     URLs updated from \texttt{www.dejanews.com}
%     to \texttt{groups.google.com}.
%   \end{Version}
%   \begin{Version}{2016/05/16 v1.6}
%   \item
%     Documentation updates.
%   \end{Version}
% \end{History}
%
% \PrintIndex
%
% \Finale
\endinput
|
% \end{quote}
% Do not forget to quote the argument according to the demands
% of your shell.
%
% \paragraph{Generating the documentation.}
% You can use both the \xfile{.dtx} or the \xfile{.drv} to generate
% the documentation. The process can be configured by the
% configuration file \xfile{ltxdoc.cfg}. For instance, put this
% line into this file, if you want to have A4 as paper format:
% \begin{quote}
%   \verb|\PassOptionsToClass{a4paper}{article}|
% \end{quote}
% An example follows how to generate the
% documentation with pdf\LaTeX:
% \begin{quote}
%\begin{verbatim}
%pdflatex twoopt.dtx
%makeindex -s gind.ist twoopt.idx
%pdflatex twoopt.dtx
%makeindex -s gind.ist twoopt.idx
%pdflatex twoopt.dtx
%\end{verbatim}
% \end{quote}
%
% \section{Catalogue}
%
% The following XML file can be used as source for the
% \href{http://mirror.ctan.org/help/Catalogue/catalogue.html}{\TeX\ Catalogue}.
% The elements \texttt{caption} and \texttt{description} are imported
% from the original XML file from the Catalogue.
% The name of the XML file in the Catalogue is \xfile{twoopt.xml}.
%    \begin{macrocode}
%<*catalogue>
<?xml version='1.0' encoding='us-ascii'?>
<!DOCTYPE entry SYSTEM 'catalogue.dtd'>
<entry datestamp='$Date$' modifier='$Author$' id='twoopt'>
  <name>twoopt</name>
  <caption>Definitions with two optional arguments.</caption>
  <authorref id='auth:oberdiek'/>
  <copyright owner='Heiko Oberdiek' year='1999,2006,2008'/>
  <license type='lppl1.3'/>
  <version number='1.6'/>
  <description>
    Variants of <tt>\newcommand</tt>, <tt>\renewcommand</tt> and
    <tt>\providecommand</tt> are provided.
    <p/>
    The package is part of the <xref refid='oberdiek'>oberdiek</xref>
    bundle.
  </description>
  <documentation details='Package documentation'
      href='ctan:/macros/latex/contrib/oberdiek/twoopt.pdf'/>
  <ctan file='true' path='/macros/latex/contrib/oberdiek/twoopt.dtx'/>
  <miktex location='oberdiek'/>
  <texlive location='oberdiek'/>
  <install path='/macros/latex/contrib/oberdiek/oberdiek.tds.zip'/>
</entry>
%</catalogue>
%    \end{macrocode}
%
% \begin{History}
%   \begin{Version}{1998/10/30 v1.0}
%   \item
%     The first version was built as a response to a question
%     of \NameEmail{Rebecca and Rowland}{rebecca@astrid.u-net.com},
%     published in the newsgroup
%     \href{news:comp.text.tex}{comp.text.tex}:\\
%     \URL{``Re: [Q] LaTeX command with two optional arguments?''}^^A
%     {http://groups.google.com/group/comp.text.tex/msg/0ab1afde7b172d37}
%   \end{Version}
%   \begin{Version}{1998/10/30 v1.1}
%   \item
%     Improvements added in response to
%     \NameEmail{Stefan Ulrich}{ulrich@cis.uni-muenchen.de}
%     in the same thread:\\
%     \URL{``Re: [Q] LaTeX command with two optional arguments?''}^^A
%     {http://groups.google.com/group/comp.text.tex/msg/b8d84d4336f302c4}
%   \end{Version}
%   \begin{Version}{1998/11/04 v1.2}
%   \item
%     Fixes for LaTeX bugs 2896, 2901, 2902 added.
%   \end{Version}
%   \begin{Version}{1999/04/12 v1.3}
%   \item
%     Fixes removed because of LaTeX [1998/12/01].
%   \item
%     Documentation in dtx format.
%   \item
%     Copyright: LPPL (\CTAN{macros/latex/base/lppl.txt})
%   \item
%     First CTAN release.
%   \end{Version}
%   \begin{Version}{2006/02/20 v1.4}
%   \item
%     Code is not changed.
%   \item
%     New DTX framework.
%   \item
%     LPPL 1.3
%   \end{Version}
%   \begin{Version}{2008/08/11 v1.5}
%   \item
%     Code is not changed.
%   \item
%     URLs updated from \texttt{www.dejanews.com}
%     to \texttt{groups.google.com}.
%   \end{Version}
%   \begin{Version}{2016/05/16 v1.6}
%   \item
%     Documentation updates.
%   \end{Version}
% \end{History}
%
% \PrintIndex
%
% \Finale
\endinput
|
% \end{quote}
% Do not forget to quote the argument according to the demands
% of your shell.
%
% \paragraph{Generating the documentation.}
% You can use both the \xfile{.dtx} or the \xfile{.drv} to generate
% the documentation. The process can be configured by the
% configuration file \xfile{ltxdoc.cfg}. For instance, put this
% line into this file, if you want to have A4 as paper format:
% \begin{quote}
%   \verb|\PassOptionsToClass{a4paper}{article}|
% \end{quote}
% An example follows how to generate the
% documentation with pdf\LaTeX:
% \begin{quote}
%\begin{verbatim}
%pdflatex twoopt.dtx
%makeindex -s gind.ist twoopt.idx
%pdflatex twoopt.dtx
%makeindex -s gind.ist twoopt.idx
%pdflatex twoopt.dtx
%\end{verbatim}
% \end{quote}
%
% \section{Catalogue}
%
% The following XML file can be used as source for the
% \href{http://mirror.ctan.org/help/Catalogue/catalogue.html}{\TeX\ Catalogue}.
% The elements \texttt{caption} and \texttt{description} are imported
% from the original XML file from the Catalogue.
% The name of the XML file in the Catalogue is \xfile{twoopt.xml}.
%    \begin{macrocode}
%<*catalogue>
<?xml version='1.0' encoding='us-ascii'?>
<!DOCTYPE entry SYSTEM 'catalogue.dtd'>
<entry datestamp='$Date$' modifier='$Author$' id='twoopt'>
  <name>twoopt</name>
  <caption>Definitions with two optional arguments.</caption>
  <authorref id='auth:oberdiek'/>
  <copyright owner='Heiko Oberdiek' year='1999,2006,2008'/>
  <license type='lppl1.3'/>
  <version number='1.5'/>
  <description>
    Variants of <tt>\newcommand</tt>, <tt>\renewcommand</tt> and
    <tt>\providecommand</tt> are provided.
    <p/>
    The package is part of the <xref refid='oberdiek'>oberdiek</xref>
    bundle.
  </description>
  <documentation details='Package documentation'
      href='ctan:/macros/latex/contrib/oberdiek/twoopt.pdf'/>
  <ctan file='true' path='/macros/latex/contrib/oberdiek/twoopt.dtx'/>
  <miktex location='oberdiek'/>
  <texlive location='oberdiek'/>
  <install path='/macros/latex/contrib/oberdiek/oberdiek.tds.zip'/>
</entry>
%</catalogue>
%    \end{macrocode}
%
% \begin{History}
%   \begin{Version}{1998/10/30 v1.0}
%   \item
%     The first version was built as a response to a question
%     of \NameEmail{Rebecca and Rowland}{rebecca@astrid.u-net.com},
%     published in the newsgroup
%     \href{news:comp.text.tex}{comp.text.tex}:\\
%     \URL{``Re: [Q] LaTeX command with two optional arguments?''}^^A
%     {http://groups.google.com/group/comp.text.tex/msg/0ab1afde7b172d37}
%   \end{Version}
%   \begin{Version}{1998/10/30 v1.1}
%   \item
%     Improvements added in response to
%     \NameEmail{Stefan Ulrich}{ulrich@cis.uni-muenchen.de}
%     in the same thread:\\
%     \URL{``Re: [Q] LaTeX command with two optional arguments?''}^^A
%     {http://groups.google.com/group/comp.text.tex/msg/b8d84d4336f302c4}
%   \end{Version}
%   \begin{Version}{1998/11/04 v1.2}
%   \item
%     Fixes for LaTeX bugs 2896, 2901, 2902 added.
%   \end{Version}
%   \begin{Version}{1999/04/12 v1.3}
%   \item
%     Fixes removed because of LaTeX [1998/12/01].
%   \item
%     Documentation in dtx format.
%   \item
%     Copyright: LPPL (\CTAN{macros/latex/base/lppl.txt})
%   \item
%     First CTAN release.
%   \end{Version}
%   \begin{Version}{2006/02/20 v1.4}
%   \item
%     Code is not changed.
%   \item
%     New DTX framework.
%   \item
%     LPPL 1.3
%   \end{Version}
%   \begin{Version}{2008/08/11 v1.5}
%   \item
%     Code is not changed.
%   \item
%     URLs updated from \texttt{www.dejanews.com}
%     to \texttt{groups.google.com}.
%   \end{Version}
% \end{History}
%
% \PrintIndex
%
% \Finale
\endinput
