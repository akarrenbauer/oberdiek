% \iffalse meta-comment
%
% File: stampinclude.dtx
% Version: 2008/07/14 v1.0
% Info: Include files based on time stamps
%
% Copyright (C) 2008 by
%    Heiko Oberdiek <heiko.oberdiek at googlemail.com>
%
% This work may be distributed and/or modified under the
% conditions of the LaTeX Project Public License, either
% version 1.3c of this license or (at your option) any later
% version. This version of this license is in
%    http://www.latex-project.org/lppl/lppl-1-3c.txt
% and the latest version of this license is in
%    http://www.latex-project.org/lppl.txt
% and version 1.3 or later is part of all distributions of
% LaTeX version 2005/12/01 or later.
%
% This work has the LPPL maintenance status "maintained".
%
% This Current Maintainer of this work is Heiko Oberdiek.
%
% This work consists of the main source file stampinclude.dtx
% and the derived files
%    stampinclude.sty, stampinclude.pdf, stampinclude.ins, stampinclude.drv.
%
% Distribution:
%    CTAN:macros/latex/contrib/oberdiek/stampinclude.dtx
%    CTAN:macros/latex/contrib/oberdiek/stampinclude.pdf
%
% Unpacking:
%    (a) If stampinclude.ins is present:
%           tex stampinclude.ins
%    (b) Without stampinclude.ins:
%           tex stampinclude.dtx
%    (c) If you insist on using LaTeX
%           latex \let\install=y% \iffalse meta-comment
%
% File: stampinclude.dtx
% Version: 2016/05/16 v1.1
% Info: Include files based on time stamps
%
% Copyright (C) 2008 by
%    Heiko Oberdiek <heiko.oberdiek at googlemail.com>
%    2016
%    https://github.com/ho-tex/oberdiek/issues
%
% This work may be distributed and/or modified under the
% conditions of the LaTeX Project Public License, either
% version 1.3c of this license or (at your option) any later
% version. This version of this license is in
%    http://www.latex-project.org/lppl/lppl-1-3c.txt
% and the latest version of this license is in
%    http://www.latex-project.org/lppl.txt
% and version 1.3 or later is part of all distributions of
% LaTeX version 2005/12/01 or later.
%
% This work has the LPPL maintenance status "maintained".
%
% This Current Maintainer of this work is Heiko Oberdiek.
%
% This work consists of the main source file stampinclude.dtx
% and the derived files
%    stampinclude.sty, stampinclude.pdf, stampinclude.ins, stampinclude.drv.
%
% Distribution:
%    CTAN:macros/latex/contrib/oberdiek/stampinclude.dtx
%    CTAN:macros/latex/contrib/oberdiek/stampinclude.pdf
%
% Unpacking:
%    (a) If stampinclude.ins is present:
%           tex stampinclude.ins
%    (b) Without stampinclude.ins:
%           tex stampinclude.dtx
%    (c) If you insist on using LaTeX
%           latex \let\install=y% \iffalse meta-comment
%
% File: stampinclude.dtx
% Version: 2016/05/16 v1.1
% Info: Include files based on time stamps
%
% Copyright (C) 2008 by
%    Heiko Oberdiek <heiko.oberdiek at googlemail.com>
%    2016
%    https://github.com/ho-tex/oberdiek/issues
%
% This work may be distributed and/or modified under the
% conditions of the LaTeX Project Public License, either
% version 1.3c of this license or (at your option) any later
% version. This version of this license is in
%    http://www.latex-project.org/lppl/lppl-1-3c.txt
% and the latest version of this license is in
%    http://www.latex-project.org/lppl.txt
% and version 1.3 or later is part of all distributions of
% LaTeX version 2005/12/01 or later.
%
% This work has the LPPL maintenance status "maintained".
%
% This Current Maintainer of this work is Heiko Oberdiek.
%
% This work consists of the main source file stampinclude.dtx
% and the derived files
%    stampinclude.sty, stampinclude.pdf, stampinclude.ins, stampinclude.drv.
%
% Distribution:
%    CTAN:macros/latex/contrib/oberdiek/stampinclude.dtx
%    CTAN:macros/latex/contrib/oberdiek/stampinclude.pdf
%
% Unpacking:
%    (a) If stampinclude.ins is present:
%           tex stampinclude.ins
%    (b) Without stampinclude.ins:
%           tex stampinclude.dtx
%    (c) If you insist on using LaTeX
%           latex \let\install=y% \iffalse meta-comment
%
% File: stampinclude.dtx
% Version: 2016/05/16 v1.1
% Info: Include files based on time stamps
%
% Copyright (C) 2008 by
%    Heiko Oberdiek <heiko.oberdiek at googlemail.com>
%    2016
%    https://github.com/ho-tex/oberdiek/issues
%
% This work may be distributed and/or modified under the
% conditions of the LaTeX Project Public License, either
% version 1.3c of this license or (at your option) any later
% version. This version of this license is in
%    http://www.latex-project.org/lppl/lppl-1-3c.txt
% and the latest version of this license is in
%    http://www.latex-project.org/lppl.txt
% and version 1.3 or later is part of all distributions of
% LaTeX version 2005/12/01 or later.
%
% This work has the LPPL maintenance status "maintained".
%
% This Current Maintainer of this work is Heiko Oberdiek.
%
% This work consists of the main source file stampinclude.dtx
% and the derived files
%    stampinclude.sty, stampinclude.pdf, stampinclude.ins, stampinclude.drv.
%
% Distribution:
%    CTAN:macros/latex/contrib/oberdiek/stampinclude.dtx
%    CTAN:macros/latex/contrib/oberdiek/stampinclude.pdf
%
% Unpacking:
%    (a) If stampinclude.ins is present:
%           tex stampinclude.ins
%    (b) Without stampinclude.ins:
%           tex stampinclude.dtx
%    (c) If you insist on using LaTeX
%           latex \let\install=y\input{stampinclude.dtx}
%        (quote the arguments according to the demands of your shell)
%
% Documentation:
%    (a) If stampinclude.drv is present:
%           latex stampinclude.drv
%    (b) Without stampinclude.drv:
%           latex stampinclude.dtx; ...
%    The class ltxdoc loads the configuration file ltxdoc.cfg
%    if available. Here you can specify further options, e.g.
%    use A4 as paper format:
%       \PassOptionsToClass{a4paper}{article}
%
%    Programm calls to get the documentation (example):
%       pdflatex stampinclude.dtx
%       makeindex -s gind.ist stampinclude.idx
%       pdflatex stampinclude.dtx
%       makeindex -s gind.ist stampinclude.idx
%       pdflatex stampinclude.dtx
%
% Installation:
%    TDS:tex/latex/oberdiek/stampinclude.sty
%    TDS:doc/latex/oberdiek/stampinclude.pdf
%    TDS:source/latex/oberdiek/stampinclude.dtx
%
%<*ignore>
\begingroup
  \catcode123=1 %
  \catcode125=2 %
  \def\x{LaTeX2e}%
\expandafter\endgroup
\ifcase 0\ifx\install y1\fi\expandafter
         \ifx\csname processbatchFile\endcsname\relax\else1\fi
         \ifx\fmtname\x\else 1\fi\relax
\else\csname fi\endcsname
%</ignore>
%<*install>
\input docstrip.tex
\Msg{************************************************************************}
\Msg{* Installation}
\Msg{* Package: stampinclude 2016/05/16 v1.1 Include files based on time stamps (HO)}
\Msg{************************************************************************}

\keepsilent
\askforoverwritefalse

\let\MetaPrefix\relax
\preamble

This is a generated file.

Project: stampinclude
Version: 2016/05/16 v1.1

Copyright (C) 2008 by
   Heiko Oberdiek <heiko.oberdiek at googlemail.com>

This work may be distributed and/or modified under the
conditions of the LaTeX Project Public License, either
version 1.3c of this license or (at your option) any later
version. This version of this license is in
   http://www.latex-project.org/lppl/lppl-1-3c.txt
and the latest version of this license is in
   http://www.latex-project.org/lppl.txt
and version 1.3 or later is part of all distributions of
LaTeX version 2005/12/01 or later.

This work has the LPPL maintenance status "maintained".

This Current Maintainer of this work is Heiko Oberdiek.

This work consists of the main source file stampinclude.dtx
and the derived files
   stampinclude.sty, stampinclude.pdf, stampinclude.ins, stampinclude.drv.

\endpreamble
\let\MetaPrefix\DoubleperCent

\generate{%
  \file{stampinclude.ins}{\from{stampinclude.dtx}{install}}%
  \file{stampinclude.drv}{\from{stampinclude.dtx}{driver}}%
  \usedir{tex/latex/oberdiek}%
  \file{stampinclude.sty}{\from{stampinclude.dtx}{package}}%
  \nopreamble
  \nopostamble
  \usedir{source/latex/oberdiek/catalogue}%
  \file{stampinclude.xml}{\from{stampinclude.dtx}{catalogue}}%
}

\catcode32=13\relax% active space
\let =\space%
\Msg{************************************************************************}
\Msg{*}
\Msg{* To finish the installation you have to move the following}
\Msg{* file into a directory searched by TeX:}
\Msg{*}
\Msg{*     stampinclude.sty}
\Msg{*}
\Msg{* To produce the documentation run the file `stampinclude.drv'}
\Msg{* through LaTeX.}
\Msg{*}
\Msg{* Happy TeXing!}
\Msg{*}
\Msg{************************************************************************}

\endbatchfile
%</install>
%<*ignore>
\fi
%</ignore>
%<*driver>
\NeedsTeXFormat{LaTeX2e}
\ProvidesFile{stampinclude.drv}%
  [2016/05/16 v1.1 Include files based on time stamps (HO)]%
\documentclass{ltxdoc}
\usepackage{holtxdoc}[2011/11/22]
\begin{document}
  \DocInput{stampinclude.dtx}%
\end{document}
%</driver>
% \fi
%
%
% \CharacterTable
%  {Upper-case    \A\B\C\D\E\F\G\H\I\J\K\L\M\N\O\P\Q\R\S\T\U\V\W\X\Y\Z
%   Lower-case    \a\b\c\d\e\f\g\h\i\j\k\l\m\n\o\p\q\r\s\t\u\v\w\x\y\z
%   Digits        \0\1\2\3\4\5\6\7\8\9
%   Exclamation   \!     Double quote  \"     Hash (number) \#
%   Dollar        \$     Percent       \%     Ampersand     \&
%   Acute accent  \'     Left paren    \(     Right paren   \)
%   Asterisk      \*     Plus          \+     Comma         \,
%   Minus         \-     Point         \.     Solidus       \/
%   Colon         \:     Semicolon     \;     Less than     \<
%   Equals        \=     Greater than  \>     Question mark \?
%   Commercial at \@     Left bracket  \[     Backslash     \\
%   Right bracket \]     Circumflex    \^     Underscore    \_
%   Grave accent  \`     Left brace    \{     Vertical bar  \|
%   Right brace   \}     Tilde         \~}
%
% \GetFileInfo{stampinclude.drv}
%
% \title{The \xpackage{stampinclude} package}
% \date{2016/05/16 v1.1}
% \author{Heiko Oberdiek\thanks
% {Please report any issues at https://github.com/ho-tex/oberdiek/issues}\\
% \xemail{heiko.oberdiek at googlemail.com}}
%
% \maketitle
%
% \begin{abstract}
% The package replaces \cs{includeonly} and selects the files for
% \cs{include} by inspecting the time stamp of the \xext{aux} file.
% The file is selected for inclusion if the \xext{aux} file does
% not yet exist or is older than the corresponding \xext{tex} file.
% \end{abstract}
%
% \tableofcontents
%
% \section{Documentation}
%
% \subsection{Introduction}
% \label{sec:intro}
%
% \LaTeX\ provides two commands \cs{include} and \cs{includeonly}
% that helps in organizing large projects.
% Example for a master file:
%\begin{quote}
%\begin{verbatim}
%\documentclass{book}
%  % \includeonly{}
%\begin{document}
% \include{fileA}
% \include{fileB}
% \include{fileC}
%\end{document}
%\end{verbatim}
%\end{quote}
% All files are read and compiled if \cs{includeonly} is not
% executed. Otherwise you can give \cs{includeonly} a list
% of files in the preamble, e.g.:
% \begin{quote}
%   |\includeonly{fileA,fileC}|
% \end{quote}
% Now only files \xfile{fileA.tex} and \xfile{fileC.tex} are read
% and compiled.
%
% If you change file \xfile{fileB.tex} and want to see only this
% file, then you must change the line with \cs{includeonly} to
% \begin{quote}
%   |\includeonly{fileB}|
% \end{quote}
% It is tedious to do this again and again, if different files
% are changed.
%
% Package \xpackage{askinclude} \cite{askinclude}
% offers a solution for this problem. It interactively asks
% for the files to be included and saves the user from
% editing the master file.
%
% This package \xpackage{stampinclude} goes another way.
% \LaTeX\ reads and writes a separate \xext{aux} file for each
% file that is included by \cs{include}. There \LaTeX\ remembers
% counter valuses. Changed \xext{tex}
% files can therefore be detected by comparing the file date stamp of
% the \xext{tex} file with the date stamp of its \xext{aux} file.
% Since version 1.30.0 \pdfTeX\ provides \cs{pdffilemoddate}
% that reads the file date stamp. Thus this package uses this
% command and redefines
% \cs{include} to include the files that do not have \xext{aux}
% files yet or that are newer than its \xext{aux} file.
% \cs{includeonly} is ignored.
%
% \subsection{Usage}
%
% The package is loaded as normal \LaTeX\ package without options:
% \begin{quote}
%   |\usepackage{stampinclude}|
% \end{quote}
% Alternatively the package may be loaded on the command line
% (Example for shell `bash'):
% \begin{center}
%   |latex '\AtBeginDocument{\usepackage{stampinclude}}\input{master}'|
% \end{center}
% Without \cs{AtBeginDocument} (and \cs{RequirePackage} instead of
% \cs{usepackage}) \TeX\ would name the document \xfile{stampinclude.dvi}
% instead of \xfile{master.dvi}.
%
% \subsection{Limitations}
%
% \subsubsection{Other file dependencies}
%
% A file that is included by \cs{include} may input ore reference
% other files:
% \begin{itemize}
% \item other \TeX\ files using \cs{input},
% \item graphics files (\cs{includegraphics}),
% \item listings of external files,
% \item ...
% \end{itemize}
% Updates of those files are not detected by this package.
% It limits the date stamp comparison of an \xext{aux} file
% to its \xext{tex} file.
%
% \subsubsection{\cs{include} dependencies}
%
% In the example, given in the introduction \ref{sec:intro},
% three files \xfile{fileA}, \xfile{fileB}, and \xfile{fileC}
% are included in this order. Now file \xfile{fileA} is changed by adding
% four pages, \xfile{fileB} remains untouched, and \xfile{fileC} is
% also updated. Then the package only selects \xfile{fileA} and
% \xfile{fileC} for inclusion. File \xfile{fileB} is not included.
% But \LaTeX\ has stored the counter values that are active
% at the end of \xfile{fileB} in \xfile{fileB.aux} in one of the
% previous runs when \xfile{fileB} was included.
% However the later addition of four pages in \xfile{fileA}
% was not known at that time. Therefore \xfile{fileB.aux}
% is out of date and the inclusion of file \xfile{fileC}
% starts with wrong counter values (especially the page counter).
%
% \subsubsection{Summary}
%
% This package \xpackage{stampinclude} and the \cs{include} feature
% helps in accelerating the \LaTeX\ compilation.
% But it is not intended for generating the final version.
% For the final version of the document it is better to include
% \emph{all} files to get all counter values right.
% Then this package and any \cs{includeonly} lines should be commented out:
%\begin{quote}
%  |% \usepackage{stampinclude}|\\
%  |% \includeonly{...}|
%\end{quote}
%
% \subsection{Requirements}
%
% \begin{itemize}
% \item \pdfTeX\ v1.30.0 (because of \cs{pdffilemoddate}
%   and \cs{pdfstrcmp}),\\
%   both modes for DVI and PDF are supported.
% \item Alternatively Lua\TeX\ may be used.
%   It lacks \cs{pdffilemoddate} and \cs{pdfstrcmp}. But its services
%   are provided by package \xpackage{pdftexcmds} \cite{pdftexcmds}
%   that is automatically loaded.
% \end{itemize}
%
% \StopEventually{
% }
%
% \section{Implementation}
%
%    \begin{macrocode}
%<*package>
\NeedsTeXFormat{LaTeX2e}
\ProvidesPackage{stampinclude}
  [2016/05/16 v1.1 Include files based on time stamps (HO)]%
%    \end{macrocode}
%
%    \begin{macrocode}
\RequirePackage{pdftexcmds}[2007/12/12]%
%    \end{macrocode}
%
%    \begin{macrocode}
\begingroup
  \chardef\x=1 %
  \expandafter\ifx\csname pdf@filemoddate\endcsname\relax
    \chardef\x=0 %
  \fi
  \expandafter\ifx\csname pdf@strcmp\endcsname\relax
    \chardef\x=0 %
  \fi
\expandafter\endgroup\ifcase\x
  \PackageWarningNoLine{stampinclude}{%
    \string\pdffilemoddate\space or %
    \string\pdfstrcmp\space are not found,\MessageBreak
    that are provided by pdfTeX >= 1.30.0.\MessageBreak
    Also LuaTeX is not detected.\MessageBreak
    Therefore package loading is aborted%
  }%
  \expandafter\endinput
\fi
%    \end{macrocode}
%
%    \begin{macro}{\SInc@org@include}
%    \begin{macrocode}
\let\SInc@org@include\@include
%    \end{macrocode}
%    \end{macro}
%    \begin{macro}{\@include}
%    \begin{macrocode}
\def\@include#1 {%
  \IfFileExists{#1.aux}{%
    \ifnum\pdf@strcmp{\pdf@filemoddate{#1.aux}}%
                     {\pdf@filemoddate{#1.tex}}<0 %
      \ifx\@partlist\@empty
        \gdef\@partlist{{#1}}%
      \else
        \g@addto@macro\@partlist{,{#1}}%
      \fi
    \fi
  }{%
    \ifx\@partlist\@empty
      \gdef\@partlist{{#1}}%
    \else
      \g@addto@macro\@partlist{,{#1}}%
    \fi
  }%
  \SInc@org@include{#1} \relax
}
%    \end{macrocode}
%    \end{macro}
%
%    \begin{macro}{\includeonly}
%    Macro \cs{includeonly} is ignored.
%    \begin{macrocode}
\renewcommand*{\includeonly}[1]{%
  \PackageInfo{stampinclude}{%
    Ignoring \string\includeonly
  }%
}
%    \end{macrocode}
%    \end{macro}
%
%    Simulate \cs{includeonly}.
%    \begin{macrocode}
\@partswtrue
\gdef\@partlist{}
%    \end{macrocode}
%
%    Print included files at end of document.
%    \begin{macrocode}
\AtEndDocument{%
  \begingroup
    \expandafter\let\expandafter\@partlist\expandafter\@empty
    \expandafter\@for\expandafter\reserved@a
    \expandafter:\expandafter=\@partlist\do{%
      \ifx\@partlist\@empty
        \edef\@partlist{\reserved@a}%
      \else
        \edef\@partlist{\@partlist, \reserved@a}%
      \fi
    }%
    \typeout{********************%
             ********************%
             ********************%
             ******************%
    }%
    \ifx\@partlist\@empty
      \typeout{[stampinclude] No included files.}%
    \else
      \typeout{[stampinclude] Included files:}%
      \typeout{\@partlist}%
    \fi
    \typeout{********************%
             ********************%
             ********************%
             ******************%
    }%
  \endgroup
}
%    \end{macrocode}
%
%    \begin{macrocode}
%</package>
%    \end{macrocode}
%
% \section{Installation}
%
% \subsection{Download}
%
% \paragraph{Package.} This package is available on
% CTAN\footnote{\url{http://ctan.org/pkg/stampinclude}}:
% \begin{description}
% \item[\CTAN{macros/latex/contrib/oberdiek/stampinclude.dtx}] The source file.
% \item[\CTAN{macros/latex/contrib/oberdiek/stampinclude.pdf}] Documentation.
% \end{description}
%
%
% \paragraph{Bundle.} All the packages of the bundle `oberdiek'
% are also available in a TDS compliant ZIP archive. There
% the packages are already unpacked and the documentation files
% are generated. The files and directories obey the TDS standard.
% \begin{description}
% \item[\CTAN{install/macros/latex/contrib/oberdiek.tds.zip}]
% \end{description}
% \emph{TDS} refers to the standard ``A Directory Structure
% for \TeX\ Files'' (\CTAN{tds/tds.pdf}). Directories
% with \xfile{texmf} in their name are usually organized this way.
%
% \subsection{Bundle installation}
%
% \paragraph{Unpacking.} Unpack the \xfile{oberdiek.tds.zip} in the
% TDS tree (also known as \xfile{texmf} tree) of your choice.
% Example (linux):
% \begin{quote}
%   |unzip oberdiek.tds.zip -d ~/texmf|
% \end{quote}
%
% \paragraph{Script installation.}
% Check the directory \xfile{TDS:scripts/oberdiek/} for
% scripts that need further installation steps.
% Package \xpackage{attachfile2} comes with the Perl script
% \xfile{pdfatfi.pl} that should be installed in such a way
% that it can be called as \texttt{pdfatfi}.
% Example (linux):
% \begin{quote}
%   |chmod +x scripts/oberdiek/pdfatfi.pl|\\
%   |cp scripts/oberdiek/pdfatfi.pl /usr/local/bin/|
% \end{quote}
%
% \subsection{Package installation}
%
% \paragraph{Unpacking.} The \xfile{.dtx} file is a self-extracting
% \docstrip\ archive. The files are extracted by running the
% \xfile{.dtx} through \plainTeX:
% \begin{quote}
%   \verb|tex stampinclude.dtx|
% \end{quote}
%
% \paragraph{TDS.} Now the different files must be moved into
% the different directories in your installation TDS tree
% (also known as \xfile{texmf} tree):
% \begin{quote}
% \def\t{^^A
% \begin{tabular}{@{}>{\ttfamily}l@{ $\rightarrow$ }>{\ttfamily}l@{}}
%   stampinclude.sty & tex/latex/oberdiek/stampinclude.sty\\
%   stampinclude.pdf & doc/latex/oberdiek/stampinclude.pdf\\
%   stampinclude.dtx & source/latex/oberdiek/stampinclude.dtx\\
% \end{tabular}^^A
% }^^A
% \sbox0{\t}^^A
% \ifdim\wd0>\linewidth
%   \begingroup
%     \advance\linewidth by\leftmargin
%     \advance\linewidth by\rightmargin
%   \edef\x{\endgroup
%     \def\noexpand\lw{\the\linewidth}^^A
%   }\x
%   \def\lwbox{^^A
%     \leavevmode
%     \hbox to \linewidth{^^A
%       \kern-\leftmargin\relax
%       \hss
%       \usebox0
%       \hss
%       \kern-\rightmargin\relax
%     }^^A
%   }^^A
%   \ifdim\wd0>\lw
%     \sbox0{\small\t}^^A
%     \ifdim\wd0>\linewidth
%       \ifdim\wd0>\lw
%         \sbox0{\footnotesize\t}^^A
%         \ifdim\wd0>\linewidth
%           \ifdim\wd0>\lw
%             \sbox0{\scriptsize\t}^^A
%             \ifdim\wd0>\linewidth
%               \ifdim\wd0>\lw
%                 \sbox0{\tiny\t}^^A
%                 \ifdim\wd0>\linewidth
%                   \lwbox
%                 \else
%                   \usebox0
%                 \fi
%               \else
%                 \lwbox
%               \fi
%             \else
%               \usebox0
%             \fi
%           \else
%             \lwbox
%           \fi
%         \else
%           \usebox0
%         \fi
%       \else
%         \lwbox
%       \fi
%     \else
%       \usebox0
%     \fi
%   \else
%     \lwbox
%   \fi
% \else
%   \usebox0
% \fi
% \end{quote}
% If you have a \xfile{docstrip.cfg} that configures and enables \docstrip's
% TDS installing feature, then some files can already be in the right
% place, see the documentation of \docstrip.
%
% \subsection{Refresh file name databases}
%
% If your \TeX~distribution
% (\teTeX, \mikTeX, \dots) relies on file name databases, you must refresh
% these. For example, \teTeX\ users run \verb|texhash| or
% \verb|mktexlsr|.
%
% \subsection{Some details for the interested}
%
% \paragraph{Attached source.}
%
% The PDF documentation on CTAN also includes the
% \xfile{.dtx} source file. It can be extracted by
% AcrobatReader 6 or higher. Another option is \textsf{pdftk},
% e.g. unpack the file into the current directory:
% \begin{quote}
%   \verb|pdftk stampinclude.pdf unpack_files output .|
% \end{quote}
%
% \paragraph{Unpacking with \LaTeX.}
% The \xfile{.dtx} chooses its action depending on the format:
% \begin{description}
% \item[\plainTeX:] Run \docstrip\ and extract the files.
% \item[\LaTeX:] Generate the documentation.
% \end{description}
% If you insist on using \LaTeX\ for \docstrip\ (really,
% \docstrip\ does not need \LaTeX), then inform the autodetect routine
% about your intention:
% \begin{quote}
%   \verb|latex \let\install=y\input{stampinclude.dtx}|
% \end{quote}
% Do not forget to quote the argument according to the demands
% of your shell.
%
% \paragraph{Generating the documentation.}
% You can use both the \xfile{.dtx} or the \xfile{.drv} to generate
% the documentation. The process can be configured by the
% configuration file \xfile{ltxdoc.cfg}. For instance, put this
% line into this file, if you want to have A4 as paper format:
% \begin{quote}
%   \verb|\PassOptionsToClass{a4paper}{article}|
% \end{quote}
% An example follows how to generate the
% documentation with pdf\LaTeX:
% \begin{quote}
%\begin{verbatim}
%pdflatex stampinclude.dtx
%makeindex -s gind.ist stampinclude.idx
%pdflatex stampinclude.dtx
%makeindex -s gind.ist stampinclude.idx
%pdflatex stampinclude.dtx
%\end{verbatim}
% \end{quote}
%
% \section{Catalogue}
%
% The following XML file can be used as source for the
% \href{http://mirror.ctan.org/help/Catalogue/catalogue.html}{\TeX\ Catalogue}.
% The elements \texttt{caption} and \texttt{description} are imported
% from the original XML file from the Catalogue.
% The name of the XML file in the Catalogue is \xfile{stampinclude.xml}.
%    \begin{macrocode}
%<*catalogue>
<?xml version='1.0' encoding='us-ascii'?>
<!DOCTYPE entry SYSTEM 'catalogue.dtd'>
<entry datestamp='$Date$' modifier='$Author$' id='stampinclude'>
  <name>stampinclude</name>
  <caption>Inclusion based on .aux file date stamps.</caption>
  <authorref id='auth:oberdiek'/>
  <copyright owner='Heiko Oberdiek' year='2008'/>
  <license type='lppl1.3'/>
  <version number='1.1'/>
  <description>
    This package replaces <tt>\includeonly</tt> and selects the files for
    <tt>\include</tt> by inspecting the timestamp of the <tt>.aux</tt> file.
    The file is selected for inclusion if the <tt>.aux</tt> file does
    not yet exist or is older than the corresponding <tt>.tex</tt> file.
    <p/>
    The package is part of the <xref refid='oberdiek'>oberdiek</xref>
    bundle.
  </description>
  <documentation details='Package documentation'
      href='ctan:/macros/latex/contrib/oberdiek/stampinclude.pdf'/>
  <ctan file='true' path='/macros/latex/contrib/oberdiek/stampinclude.dtx'/>
  <miktex location='oberdiek'/>
  <texlive location='oberdiek'/>
  <install path='/macros/latex/contrib/oberdiek/oberdiek.tds.zip'/>
</entry>
%</catalogue>
%    \end{macrocode}
%
% \begin{thebibliography}{9}
% \bibitem{askinclude}
%   Pablo A. Straub, Heiko Oberdiek:
%   \textit{The \xpackage{askinclude} package};
%   2007/10/23 v2.0;
%   \CTAN{macros/latex/contrib/oberdiek/askinclude.pdf}.
%
% \bibitem{pdftexcmds}
%   Heiko Oberdiek:
%   \textit{The \xpackage{pdftexcmds} package};
%   2007/12/12 v0.3;
%   \CTAN{macros/latex/contrib/oberdiek/pdftexcmds.pdf}.
%
% \end{thebibliography}
%
% \begin{History}
%   \begin{Version}{2008/07/14 v1.0}
%   \item
%     First version.
%   \end{Version}
%   \begin{Version}{2016/05/16 v1.1}
%   \item
%     Documentation updates.
%   \end{Version}
% \end{History}
%
% \PrintIndex
%
% \Finale
\endinput

%        (quote the arguments according to the demands of your shell)
%
% Documentation:
%    (a) If stampinclude.drv is present:
%           latex stampinclude.drv
%    (b) Without stampinclude.drv:
%           latex stampinclude.dtx; ...
%    The class ltxdoc loads the configuration file ltxdoc.cfg
%    if available. Here you can specify further options, e.g.
%    use A4 as paper format:
%       \PassOptionsToClass{a4paper}{article}
%
%    Programm calls to get the documentation (example):
%       pdflatex stampinclude.dtx
%       makeindex -s gind.ist stampinclude.idx
%       pdflatex stampinclude.dtx
%       makeindex -s gind.ist stampinclude.idx
%       pdflatex stampinclude.dtx
%
% Installation:
%    TDS:tex/latex/oberdiek/stampinclude.sty
%    TDS:doc/latex/oberdiek/stampinclude.pdf
%    TDS:source/latex/oberdiek/stampinclude.dtx
%
%<*ignore>
\begingroup
  \catcode123=1 %
  \catcode125=2 %
  \def\x{LaTeX2e}%
\expandafter\endgroup
\ifcase 0\ifx\install y1\fi\expandafter
         \ifx\csname processbatchFile\endcsname\relax\else1\fi
         \ifx\fmtname\x\else 1\fi\relax
\else\csname fi\endcsname
%</ignore>
%<*install>
\input docstrip.tex
\Msg{************************************************************************}
\Msg{* Installation}
\Msg{* Package: stampinclude 2016/05/16 v1.1 Include files based on time stamps (HO)}
\Msg{************************************************************************}

\keepsilent
\askforoverwritefalse

\let\MetaPrefix\relax
\preamble

This is a generated file.

Project: stampinclude
Version: 2016/05/16 v1.1

Copyright (C) 2008 by
   Heiko Oberdiek <heiko.oberdiek at googlemail.com>

This work may be distributed and/or modified under the
conditions of the LaTeX Project Public License, either
version 1.3c of this license or (at your option) any later
version. This version of this license is in
   http://www.latex-project.org/lppl/lppl-1-3c.txt
and the latest version of this license is in
   http://www.latex-project.org/lppl.txt
and version 1.3 or later is part of all distributions of
LaTeX version 2005/12/01 or later.

This work has the LPPL maintenance status "maintained".

This Current Maintainer of this work is Heiko Oberdiek.

This work consists of the main source file stampinclude.dtx
and the derived files
   stampinclude.sty, stampinclude.pdf, stampinclude.ins, stampinclude.drv.

\endpreamble
\let\MetaPrefix\DoubleperCent

\generate{%
  \file{stampinclude.ins}{\from{stampinclude.dtx}{install}}%
  \file{stampinclude.drv}{\from{stampinclude.dtx}{driver}}%
  \usedir{tex/latex/oberdiek}%
  \file{stampinclude.sty}{\from{stampinclude.dtx}{package}}%
  \nopreamble
  \nopostamble
  \usedir{source/latex/oberdiek/catalogue}%
  \file{stampinclude.xml}{\from{stampinclude.dtx}{catalogue}}%
}

\catcode32=13\relax% active space
\let =\space%
\Msg{************************************************************************}
\Msg{*}
\Msg{* To finish the installation you have to move the following}
\Msg{* file into a directory searched by TeX:}
\Msg{*}
\Msg{*     stampinclude.sty}
\Msg{*}
\Msg{* To produce the documentation run the file `stampinclude.drv'}
\Msg{* through LaTeX.}
\Msg{*}
\Msg{* Happy TeXing!}
\Msg{*}
\Msg{************************************************************************}

\endbatchfile
%</install>
%<*ignore>
\fi
%</ignore>
%<*driver>
\NeedsTeXFormat{LaTeX2e}
\ProvidesFile{stampinclude.drv}%
  [2016/05/16 v1.1 Include files based on time stamps (HO)]%
\documentclass{ltxdoc}
\usepackage{holtxdoc}[2011/11/22]
\begin{document}
  \DocInput{stampinclude.dtx}%
\end{document}
%</driver>
% \fi
%
%
% \CharacterTable
%  {Upper-case    \A\B\C\D\E\F\G\H\I\J\K\L\M\N\O\P\Q\R\S\T\U\V\W\X\Y\Z
%   Lower-case    \a\b\c\d\e\f\g\h\i\j\k\l\m\n\o\p\q\r\s\t\u\v\w\x\y\z
%   Digits        \0\1\2\3\4\5\6\7\8\9
%   Exclamation   \!     Double quote  \"     Hash (number) \#
%   Dollar        \$     Percent       \%     Ampersand     \&
%   Acute accent  \'     Left paren    \(     Right paren   \)
%   Asterisk      \*     Plus          \+     Comma         \,
%   Minus         \-     Point         \.     Solidus       \/
%   Colon         \:     Semicolon     \;     Less than     \<
%   Equals        \=     Greater than  \>     Question mark \?
%   Commercial at \@     Left bracket  \[     Backslash     \\
%   Right bracket \]     Circumflex    \^     Underscore    \_
%   Grave accent  \`     Left brace    \{     Vertical bar  \|
%   Right brace   \}     Tilde         \~}
%
% \GetFileInfo{stampinclude.drv}
%
% \title{The \xpackage{stampinclude} package}
% \date{2016/05/16 v1.1}
% \author{Heiko Oberdiek\thanks
% {Please report any issues at https://github.com/ho-tex/oberdiek/issues}\\
% \xemail{heiko.oberdiek at googlemail.com}}
%
% \maketitle
%
% \begin{abstract}
% The package replaces \cs{includeonly} and selects the files for
% \cs{include} by inspecting the time stamp of the \xext{aux} file.
% The file is selected for inclusion if the \xext{aux} file does
% not yet exist or is older than the corresponding \xext{tex} file.
% \end{abstract}
%
% \tableofcontents
%
% \section{Documentation}
%
% \subsection{Introduction}
% \label{sec:intro}
%
% \LaTeX\ provides two commands \cs{include} and \cs{includeonly}
% that helps in organizing large projects.
% Example for a master file:
%\begin{quote}
%\begin{verbatim}
%\documentclass{book}
%  % \includeonly{}
%\begin{document}
% \include{fileA}
% \include{fileB}
% \include{fileC}
%\end{document}
%\end{verbatim}
%\end{quote}
% All files are read and compiled if \cs{includeonly} is not
% executed. Otherwise you can give \cs{includeonly} a list
% of files in the preamble, e.g.:
% \begin{quote}
%   |\includeonly{fileA,fileC}|
% \end{quote}
% Now only files \xfile{fileA.tex} and \xfile{fileC.tex} are read
% and compiled.
%
% If you change file \xfile{fileB.tex} and want to see only this
% file, then you must change the line with \cs{includeonly} to
% \begin{quote}
%   |\includeonly{fileB}|
% \end{quote}
% It is tedious to do this again and again, if different files
% are changed.
%
% Package \xpackage{askinclude} \cite{askinclude}
% offers a solution for this problem. It interactively asks
% for the files to be included and saves the user from
% editing the master file.
%
% This package \xpackage{stampinclude} goes another way.
% \LaTeX\ reads and writes a separate \xext{aux} file for each
% file that is included by \cs{include}. There \LaTeX\ remembers
% counter valuses. Changed \xext{tex}
% files can therefore be detected by comparing the file date stamp of
% the \xext{tex} file with the date stamp of its \xext{aux} file.
% Since version 1.30.0 \pdfTeX\ provides \cs{pdffilemoddate}
% that reads the file date stamp. Thus this package uses this
% command and redefines
% \cs{include} to include the files that do not have \xext{aux}
% files yet or that are newer than its \xext{aux} file.
% \cs{includeonly} is ignored.
%
% \subsection{Usage}
%
% The package is loaded as normal \LaTeX\ package without options:
% \begin{quote}
%   |\usepackage{stampinclude}|
% \end{quote}
% Alternatively the package may be loaded on the command line
% (Example for shell `bash'):
% \begin{center}
%   |latex '\AtBeginDocument{\usepackage{stampinclude}}\input{master}'|
% \end{center}
% Without \cs{AtBeginDocument} (and \cs{RequirePackage} instead of
% \cs{usepackage}) \TeX\ would name the document \xfile{stampinclude.dvi}
% instead of \xfile{master.dvi}.
%
% \subsection{Limitations}
%
% \subsubsection{Other file dependencies}
%
% A file that is included by \cs{include} may input ore reference
% other files:
% \begin{itemize}
% \item other \TeX\ files using \cs{input},
% \item graphics files (\cs{includegraphics}),
% \item listings of external files,
% \item ...
% \end{itemize}
% Updates of those files are not detected by this package.
% It limits the date stamp comparison of an \xext{aux} file
% to its \xext{tex} file.
%
% \subsubsection{\cs{include} dependencies}
%
% In the example, given in the introduction \ref{sec:intro},
% three files \xfile{fileA}, \xfile{fileB}, and \xfile{fileC}
% are included in this order. Now file \xfile{fileA} is changed by adding
% four pages, \xfile{fileB} remains untouched, and \xfile{fileC} is
% also updated. Then the package only selects \xfile{fileA} and
% \xfile{fileC} for inclusion. File \xfile{fileB} is not included.
% But \LaTeX\ has stored the counter values that are active
% at the end of \xfile{fileB} in \xfile{fileB.aux} in one of the
% previous runs when \xfile{fileB} was included.
% However the later addition of four pages in \xfile{fileA}
% was not known at that time. Therefore \xfile{fileB.aux}
% is out of date and the inclusion of file \xfile{fileC}
% starts with wrong counter values (especially the page counter).
%
% \subsubsection{Summary}
%
% This package \xpackage{stampinclude} and the \cs{include} feature
% helps in accelerating the \LaTeX\ compilation.
% But it is not intended for generating the final version.
% For the final version of the document it is better to include
% \emph{all} files to get all counter values right.
% Then this package and any \cs{includeonly} lines should be commented out:
%\begin{quote}
%  |% \usepackage{stampinclude}|\\
%  |% \includeonly{...}|
%\end{quote}
%
% \subsection{Requirements}
%
% \begin{itemize}
% \item \pdfTeX\ v1.30.0 (because of \cs{pdffilemoddate}
%   and \cs{pdfstrcmp}),\\
%   both modes for DVI and PDF are supported.
% \item Alternatively Lua\TeX\ may be used.
%   It lacks \cs{pdffilemoddate} and \cs{pdfstrcmp}. But its services
%   are provided by package \xpackage{pdftexcmds} \cite{pdftexcmds}
%   that is automatically loaded.
% \end{itemize}
%
% \StopEventually{
% }
%
% \section{Implementation}
%
%    \begin{macrocode}
%<*package>
\NeedsTeXFormat{LaTeX2e}
\ProvidesPackage{stampinclude}
  [2016/05/16 v1.1 Include files based on time stamps (HO)]%
%    \end{macrocode}
%
%    \begin{macrocode}
\RequirePackage{pdftexcmds}[2007/12/12]%
%    \end{macrocode}
%
%    \begin{macrocode}
\begingroup
  \chardef\x=1 %
  \expandafter\ifx\csname pdf@filemoddate\endcsname\relax
    \chardef\x=0 %
  \fi
  \expandafter\ifx\csname pdf@strcmp\endcsname\relax
    \chardef\x=0 %
  \fi
\expandafter\endgroup\ifcase\x
  \PackageWarningNoLine{stampinclude}{%
    \string\pdffilemoddate\space or %
    \string\pdfstrcmp\space are not found,\MessageBreak
    that are provided by pdfTeX >= 1.30.0.\MessageBreak
    Also LuaTeX is not detected.\MessageBreak
    Therefore package loading is aborted%
  }%
  \expandafter\endinput
\fi
%    \end{macrocode}
%
%    \begin{macro}{\SInc@org@include}
%    \begin{macrocode}
\let\SInc@org@include\@include
%    \end{macrocode}
%    \end{macro}
%    \begin{macro}{\@include}
%    \begin{macrocode}
\def\@include#1 {%
  \IfFileExists{#1.aux}{%
    \ifnum\pdf@strcmp{\pdf@filemoddate{#1.aux}}%
                     {\pdf@filemoddate{#1.tex}}<0 %
      \ifx\@partlist\@empty
        \gdef\@partlist{{#1}}%
      \else
        \g@addto@macro\@partlist{,{#1}}%
      \fi
    \fi
  }{%
    \ifx\@partlist\@empty
      \gdef\@partlist{{#1}}%
    \else
      \g@addto@macro\@partlist{,{#1}}%
    \fi
  }%
  \SInc@org@include{#1} \relax
}
%    \end{macrocode}
%    \end{macro}
%
%    \begin{macro}{\includeonly}
%    Macro \cs{includeonly} is ignored.
%    \begin{macrocode}
\renewcommand*{\includeonly}[1]{%
  \PackageInfo{stampinclude}{%
    Ignoring \string\includeonly
  }%
}
%    \end{macrocode}
%    \end{macro}
%
%    Simulate \cs{includeonly}.
%    \begin{macrocode}
\@partswtrue
\gdef\@partlist{}
%    \end{macrocode}
%
%    Print included files at end of document.
%    \begin{macrocode}
\AtEndDocument{%
  \begingroup
    \expandafter\let\expandafter\@partlist\expandafter\@empty
    \expandafter\@for\expandafter\reserved@a
    \expandafter:\expandafter=\@partlist\do{%
      \ifx\@partlist\@empty
        \edef\@partlist{\reserved@a}%
      \else
        \edef\@partlist{\@partlist, \reserved@a}%
      \fi
    }%
    \typeout{********************%
             ********************%
             ********************%
             ******************%
    }%
    \ifx\@partlist\@empty
      \typeout{[stampinclude] No included files.}%
    \else
      \typeout{[stampinclude] Included files:}%
      \typeout{\@partlist}%
    \fi
    \typeout{********************%
             ********************%
             ********************%
             ******************%
    }%
  \endgroup
}
%    \end{macrocode}
%
%    \begin{macrocode}
%</package>
%    \end{macrocode}
%
% \section{Installation}
%
% \subsection{Download}
%
% \paragraph{Package.} This package is available on
% CTAN\footnote{\url{http://ctan.org/pkg/stampinclude}}:
% \begin{description}
% \item[\CTAN{macros/latex/contrib/oberdiek/stampinclude.dtx}] The source file.
% \item[\CTAN{macros/latex/contrib/oberdiek/stampinclude.pdf}] Documentation.
% \end{description}
%
%
% \paragraph{Bundle.} All the packages of the bundle `oberdiek'
% are also available in a TDS compliant ZIP archive. There
% the packages are already unpacked and the documentation files
% are generated. The files and directories obey the TDS standard.
% \begin{description}
% \item[\CTAN{install/macros/latex/contrib/oberdiek.tds.zip}]
% \end{description}
% \emph{TDS} refers to the standard ``A Directory Structure
% for \TeX\ Files'' (\CTAN{tds/tds.pdf}). Directories
% with \xfile{texmf} in their name are usually organized this way.
%
% \subsection{Bundle installation}
%
% \paragraph{Unpacking.} Unpack the \xfile{oberdiek.tds.zip} in the
% TDS tree (also known as \xfile{texmf} tree) of your choice.
% Example (linux):
% \begin{quote}
%   |unzip oberdiek.tds.zip -d ~/texmf|
% \end{quote}
%
% \paragraph{Script installation.}
% Check the directory \xfile{TDS:scripts/oberdiek/} for
% scripts that need further installation steps.
% Package \xpackage{attachfile2} comes with the Perl script
% \xfile{pdfatfi.pl} that should be installed in such a way
% that it can be called as \texttt{pdfatfi}.
% Example (linux):
% \begin{quote}
%   |chmod +x scripts/oberdiek/pdfatfi.pl|\\
%   |cp scripts/oberdiek/pdfatfi.pl /usr/local/bin/|
% \end{quote}
%
% \subsection{Package installation}
%
% \paragraph{Unpacking.} The \xfile{.dtx} file is a self-extracting
% \docstrip\ archive. The files are extracted by running the
% \xfile{.dtx} through \plainTeX:
% \begin{quote}
%   \verb|tex stampinclude.dtx|
% \end{quote}
%
% \paragraph{TDS.} Now the different files must be moved into
% the different directories in your installation TDS tree
% (also known as \xfile{texmf} tree):
% \begin{quote}
% \def\t{^^A
% \begin{tabular}{@{}>{\ttfamily}l@{ $\rightarrow$ }>{\ttfamily}l@{}}
%   stampinclude.sty & tex/latex/oberdiek/stampinclude.sty\\
%   stampinclude.pdf & doc/latex/oberdiek/stampinclude.pdf\\
%   stampinclude.dtx & source/latex/oberdiek/stampinclude.dtx\\
% \end{tabular}^^A
% }^^A
% \sbox0{\t}^^A
% \ifdim\wd0>\linewidth
%   \begingroup
%     \advance\linewidth by\leftmargin
%     \advance\linewidth by\rightmargin
%   \edef\x{\endgroup
%     \def\noexpand\lw{\the\linewidth}^^A
%   }\x
%   \def\lwbox{^^A
%     \leavevmode
%     \hbox to \linewidth{^^A
%       \kern-\leftmargin\relax
%       \hss
%       \usebox0
%       \hss
%       \kern-\rightmargin\relax
%     }^^A
%   }^^A
%   \ifdim\wd0>\lw
%     \sbox0{\small\t}^^A
%     \ifdim\wd0>\linewidth
%       \ifdim\wd0>\lw
%         \sbox0{\footnotesize\t}^^A
%         \ifdim\wd0>\linewidth
%           \ifdim\wd0>\lw
%             \sbox0{\scriptsize\t}^^A
%             \ifdim\wd0>\linewidth
%               \ifdim\wd0>\lw
%                 \sbox0{\tiny\t}^^A
%                 \ifdim\wd0>\linewidth
%                   \lwbox
%                 \else
%                   \usebox0
%                 \fi
%               \else
%                 \lwbox
%               \fi
%             \else
%               \usebox0
%             \fi
%           \else
%             \lwbox
%           \fi
%         \else
%           \usebox0
%         \fi
%       \else
%         \lwbox
%       \fi
%     \else
%       \usebox0
%     \fi
%   \else
%     \lwbox
%   \fi
% \else
%   \usebox0
% \fi
% \end{quote}
% If you have a \xfile{docstrip.cfg} that configures and enables \docstrip's
% TDS installing feature, then some files can already be in the right
% place, see the documentation of \docstrip.
%
% \subsection{Refresh file name databases}
%
% If your \TeX~distribution
% (\teTeX, \mikTeX, \dots) relies on file name databases, you must refresh
% these. For example, \teTeX\ users run \verb|texhash| or
% \verb|mktexlsr|.
%
% \subsection{Some details for the interested}
%
% \paragraph{Attached source.}
%
% The PDF documentation on CTAN also includes the
% \xfile{.dtx} source file. It can be extracted by
% AcrobatReader 6 or higher. Another option is \textsf{pdftk},
% e.g. unpack the file into the current directory:
% \begin{quote}
%   \verb|pdftk stampinclude.pdf unpack_files output .|
% \end{quote}
%
% \paragraph{Unpacking with \LaTeX.}
% The \xfile{.dtx} chooses its action depending on the format:
% \begin{description}
% \item[\plainTeX:] Run \docstrip\ and extract the files.
% \item[\LaTeX:] Generate the documentation.
% \end{description}
% If you insist on using \LaTeX\ for \docstrip\ (really,
% \docstrip\ does not need \LaTeX), then inform the autodetect routine
% about your intention:
% \begin{quote}
%   \verb|latex \let\install=y% \iffalse meta-comment
%
% File: stampinclude.dtx
% Version: 2016/05/16 v1.1
% Info: Include files based on time stamps
%
% Copyright (C) 2008 by
%    Heiko Oberdiek <heiko.oberdiek at googlemail.com>
%    2016
%    https://github.com/ho-tex/oberdiek/issues
%
% This work may be distributed and/or modified under the
% conditions of the LaTeX Project Public License, either
% version 1.3c of this license or (at your option) any later
% version. This version of this license is in
%    http://www.latex-project.org/lppl/lppl-1-3c.txt
% and the latest version of this license is in
%    http://www.latex-project.org/lppl.txt
% and version 1.3 or later is part of all distributions of
% LaTeX version 2005/12/01 or later.
%
% This work has the LPPL maintenance status "maintained".
%
% This Current Maintainer of this work is Heiko Oberdiek.
%
% This work consists of the main source file stampinclude.dtx
% and the derived files
%    stampinclude.sty, stampinclude.pdf, stampinclude.ins, stampinclude.drv.
%
% Distribution:
%    CTAN:macros/latex/contrib/oberdiek/stampinclude.dtx
%    CTAN:macros/latex/contrib/oberdiek/stampinclude.pdf
%
% Unpacking:
%    (a) If stampinclude.ins is present:
%           tex stampinclude.ins
%    (b) Without stampinclude.ins:
%           tex stampinclude.dtx
%    (c) If you insist on using LaTeX
%           latex \let\install=y\input{stampinclude.dtx}
%        (quote the arguments according to the demands of your shell)
%
% Documentation:
%    (a) If stampinclude.drv is present:
%           latex stampinclude.drv
%    (b) Without stampinclude.drv:
%           latex stampinclude.dtx; ...
%    The class ltxdoc loads the configuration file ltxdoc.cfg
%    if available. Here you can specify further options, e.g.
%    use A4 as paper format:
%       \PassOptionsToClass{a4paper}{article}
%
%    Programm calls to get the documentation (example):
%       pdflatex stampinclude.dtx
%       makeindex -s gind.ist stampinclude.idx
%       pdflatex stampinclude.dtx
%       makeindex -s gind.ist stampinclude.idx
%       pdflatex stampinclude.dtx
%
% Installation:
%    TDS:tex/latex/oberdiek/stampinclude.sty
%    TDS:doc/latex/oberdiek/stampinclude.pdf
%    TDS:source/latex/oberdiek/stampinclude.dtx
%
%<*ignore>
\begingroup
  \catcode123=1 %
  \catcode125=2 %
  \def\x{LaTeX2e}%
\expandafter\endgroup
\ifcase 0\ifx\install y1\fi\expandafter
         \ifx\csname processbatchFile\endcsname\relax\else1\fi
         \ifx\fmtname\x\else 1\fi\relax
\else\csname fi\endcsname
%</ignore>
%<*install>
\input docstrip.tex
\Msg{************************************************************************}
\Msg{* Installation}
\Msg{* Package: stampinclude 2016/05/16 v1.1 Include files based on time stamps (HO)}
\Msg{************************************************************************}

\keepsilent
\askforoverwritefalse

\let\MetaPrefix\relax
\preamble

This is a generated file.

Project: stampinclude
Version: 2016/05/16 v1.1

Copyright (C) 2008 by
   Heiko Oberdiek <heiko.oberdiek at googlemail.com>

This work may be distributed and/or modified under the
conditions of the LaTeX Project Public License, either
version 1.3c of this license or (at your option) any later
version. This version of this license is in
   http://www.latex-project.org/lppl/lppl-1-3c.txt
and the latest version of this license is in
   http://www.latex-project.org/lppl.txt
and version 1.3 or later is part of all distributions of
LaTeX version 2005/12/01 or later.

This work has the LPPL maintenance status "maintained".

This Current Maintainer of this work is Heiko Oberdiek.

This work consists of the main source file stampinclude.dtx
and the derived files
   stampinclude.sty, stampinclude.pdf, stampinclude.ins, stampinclude.drv.

\endpreamble
\let\MetaPrefix\DoubleperCent

\generate{%
  \file{stampinclude.ins}{\from{stampinclude.dtx}{install}}%
  \file{stampinclude.drv}{\from{stampinclude.dtx}{driver}}%
  \usedir{tex/latex/oberdiek}%
  \file{stampinclude.sty}{\from{stampinclude.dtx}{package}}%
  \nopreamble
  \nopostamble
  \usedir{source/latex/oberdiek/catalogue}%
  \file{stampinclude.xml}{\from{stampinclude.dtx}{catalogue}}%
}

\catcode32=13\relax% active space
\let =\space%
\Msg{************************************************************************}
\Msg{*}
\Msg{* To finish the installation you have to move the following}
\Msg{* file into a directory searched by TeX:}
\Msg{*}
\Msg{*     stampinclude.sty}
\Msg{*}
\Msg{* To produce the documentation run the file `stampinclude.drv'}
\Msg{* through LaTeX.}
\Msg{*}
\Msg{* Happy TeXing!}
\Msg{*}
\Msg{************************************************************************}

\endbatchfile
%</install>
%<*ignore>
\fi
%</ignore>
%<*driver>
\NeedsTeXFormat{LaTeX2e}
\ProvidesFile{stampinclude.drv}%
  [2016/05/16 v1.1 Include files based on time stamps (HO)]%
\documentclass{ltxdoc}
\usepackage{holtxdoc}[2011/11/22]
\begin{document}
  \DocInput{stampinclude.dtx}%
\end{document}
%</driver>
% \fi
%
%
% \CharacterTable
%  {Upper-case    \A\B\C\D\E\F\G\H\I\J\K\L\M\N\O\P\Q\R\S\T\U\V\W\X\Y\Z
%   Lower-case    \a\b\c\d\e\f\g\h\i\j\k\l\m\n\o\p\q\r\s\t\u\v\w\x\y\z
%   Digits        \0\1\2\3\4\5\6\7\8\9
%   Exclamation   \!     Double quote  \"     Hash (number) \#
%   Dollar        \$     Percent       \%     Ampersand     \&
%   Acute accent  \'     Left paren    \(     Right paren   \)
%   Asterisk      \*     Plus          \+     Comma         \,
%   Minus         \-     Point         \.     Solidus       \/
%   Colon         \:     Semicolon     \;     Less than     \<
%   Equals        \=     Greater than  \>     Question mark \?
%   Commercial at \@     Left bracket  \[     Backslash     \\
%   Right bracket \]     Circumflex    \^     Underscore    \_
%   Grave accent  \`     Left brace    \{     Vertical bar  \|
%   Right brace   \}     Tilde         \~}
%
% \GetFileInfo{stampinclude.drv}
%
% \title{The \xpackage{stampinclude} package}
% \date{2016/05/16 v1.1}
% \author{Heiko Oberdiek\thanks
% {Please report any issues at https://github.com/ho-tex/oberdiek/issues}\\
% \xemail{heiko.oberdiek at googlemail.com}}
%
% \maketitle
%
% \begin{abstract}
% The package replaces \cs{includeonly} and selects the files for
% \cs{include} by inspecting the time stamp of the \xext{aux} file.
% The file is selected for inclusion if the \xext{aux} file does
% not yet exist or is older than the corresponding \xext{tex} file.
% \end{abstract}
%
% \tableofcontents
%
% \section{Documentation}
%
% \subsection{Introduction}
% \label{sec:intro}
%
% \LaTeX\ provides two commands \cs{include} and \cs{includeonly}
% that helps in organizing large projects.
% Example for a master file:
%\begin{quote}
%\begin{verbatim}
%\documentclass{book}
%  % \includeonly{}
%\begin{document}
% \include{fileA}
% \include{fileB}
% \include{fileC}
%\end{document}
%\end{verbatim}
%\end{quote}
% All files are read and compiled if \cs{includeonly} is not
% executed. Otherwise you can give \cs{includeonly} a list
% of files in the preamble, e.g.:
% \begin{quote}
%   |\includeonly{fileA,fileC}|
% \end{quote}
% Now only files \xfile{fileA.tex} and \xfile{fileC.tex} are read
% and compiled.
%
% If you change file \xfile{fileB.tex} and want to see only this
% file, then you must change the line with \cs{includeonly} to
% \begin{quote}
%   |\includeonly{fileB}|
% \end{quote}
% It is tedious to do this again and again, if different files
% are changed.
%
% Package \xpackage{askinclude} \cite{askinclude}
% offers a solution for this problem. It interactively asks
% for the files to be included and saves the user from
% editing the master file.
%
% This package \xpackage{stampinclude} goes another way.
% \LaTeX\ reads and writes a separate \xext{aux} file for each
% file that is included by \cs{include}. There \LaTeX\ remembers
% counter valuses. Changed \xext{tex}
% files can therefore be detected by comparing the file date stamp of
% the \xext{tex} file with the date stamp of its \xext{aux} file.
% Since version 1.30.0 \pdfTeX\ provides \cs{pdffilemoddate}
% that reads the file date stamp. Thus this package uses this
% command and redefines
% \cs{include} to include the files that do not have \xext{aux}
% files yet or that are newer than its \xext{aux} file.
% \cs{includeonly} is ignored.
%
% \subsection{Usage}
%
% The package is loaded as normal \LaTeX\ package without options:
% \begin{quote}
%   |\usepackage{stampinclude}|
% \end{quote}
% Alternatively the package may be loaded on the command line
% (Example for shell `bash'):
% \begin{center}
%   |latex '\AtBeginDocument{\usepackage{stampinclude}}\input{master}'|
% \end{center}
% Without \cs{AtBeginDocument} (and \cs{RequirePackage} instead of
% \cs{usepackage}) \TeX\ would name the document \xfile{stampinclude.dvi}
% instead of \xfile{master.dvi}.
%
% \subsection{Limitations}
%
% \subsubsection{Other file dependencies}
%
% A file that is included by \cs{include} may input ore reference
% other files:
% \begin{itemize}
% \item other \TeX\ files using \cs{input},
% \item graphics files (\cs{includegraphics}),
% \item listings of external files,
% \item ...
% \end{itemize}
% Updates of those files are not detected by this package.
% It limits the date stamp comparison of an \xext{aux} file
% to its \xext{tex} file.
%
% \subsubsection{\cs{include} dependencies}
%
% In the example, given in the introduction \ref{sec:intro},
% three files \xfile{fileA}, \xfile{fileB}, and \xfile{fileC}
% are included in this order. Now file \xfile{fileA} is changed by adding
% four pages, \xfile{fileB} remains untouched, and \xfile{fileC} is
% also updated. Then the package only selects \xfile{fileA} and
% \xfile{fileC} for inclusion. File \xfile{fileB} is not included.
% But \LaTeX\ has stored the counter values that are active
% at the end of \xfile{fileB} in \xfile{fileB.aux} in one of the
% previous runs when \xfile{fileB} was included.
% However the later addition of four pages in \xfile{fileA}
% was not known at that time. Therefore \xfile{fileB.aux}
% is out of date and the inclusion of file \xfile{fileC}
% starts with wrong counter values (especially the page counter).
%
% \subsubsection{Summary}
%
% This package \xpackage{stampinclude} and the \cs{include} feature
% helps in accelerating the \LaTeX\ compilation.
% But it is not intended for generating the final version.
% For the final version of the document it is better to include
% \emph{all} files to get all counter values right.
% Then this package and any \cs{includeonly} lines should be commented out:
%\begin{quote}
%  |% \usepackage{stampinclude}|\\
%  |% \includeonly{...}|
%\end{quote}
%
% \subsection{Requirements}
%
% \begin{itemize}
% \item \pdfTeX\ v1.30.0 (because of \cs{pdffilemoddate}
%   and \cs{pdfstrcmp}),\\
%   both modes for DVI and PDF are supported.
% \item Alternatively Lua\TeX\ may be used.
%   It lacks \cs{pdffilemoddate} and \cs{pdfstrcmp}. But its services
%   are provided by package \xpackage{pdftexcmds} \cite{pdftexcmds}
%   that is automatically loaded.
% \end{itemize}
%
% \StopEventually{
% }
%
% \section{Implementation}
%
%    \begin{macrocode}
%<*package>
\NeedsTeXFormat{LaTeX2e}
\ProvidesPackage{stampinclude}
  [2016/05/16 v1.1 Include files based on time stamps (HO)]%
%    \end{macrocode}
%
%    \begin{macrocode}
\RequirePackage{pdftexcmds}[2007/12/12]%
%    \end{macrocode}
%
%    \begin{macrocode}
\begingroup
  \chardef\x=1 %
  \expandafter\ifx\csname pdf@filemoddate\endcsname\relax
    \chardef\x=0 %
  \fi
  \expandafter\ifx\csname pdf@strcmp\endcsname\relax
    \chardef\x=0 %
  \fi
\expandafter\endgroup\ifcase\x
  \PackageWarningNoLine{stampinclude}{%
    \string\pdffilemoddate\space or %
    \string\pdfstrcmp\space are not found,\MessageBreak
    that are provided by pdfTeX >= 1.30.0.\MessageBreak
    Also LuaTeX is not detected.\MessageBreak
    Therefore package loading is aborted%
  }%
  \expandafter\endinput
\fi
%    \end{macrocode}
%
%    \begin{macro}{\SInc@org@include}
%    \begin{macrocode}
\let\SInc@org@include\@include
%    \end{macrocode}
%    \end{macro}
%    \begin{macro}{\@include}
%    \begin{macrocode}
\def\@include#1 {%
  \IfFileExists{#1.aux}{%
    \ifnum\pdf@strcmp{\pdf@filemoddate{#1.aux}}%
                     {\pdf@filemoddate{#1.tex}}<0 %
      \ifx\@partlist\@empty
        \gdef\@partlist{{#1}}%
      \else
        \g@addto@macro\@partlist{,{#1}}%
      \fi
    \fi
  }{%
    \ifx\@partlist\@empty
      \gdef\@partlist{{#1}}%
    \else
      \g@addto@macro\@partlist{,{#1}}%
    \fi
  }%
  \SInc@org@include{#1} \relax
}
%    \end{macrocode}
%    \end{macro}
%
%    \begin{macro}{\includeonly}
%    Macro \cs{includeonly} is ignored.
%    \begin{macrocode}
\renewcommand*{\includeonly}[1]{%
  \PackageInfo{stampinclude}{%
    Ignoring \string\includeonly
  }%
}
%    \end{macrocode}
%    \end{macro}
%
%    Simulate \cs{includeonly}.
%    \begin{macrocode}
\@partswtrue
\gdef\@partlist{}
%    \end{macrocode}
%
%    Print included files at end of document.
%    \begin{macrocode}
\AtEndDocument{%
  \begingroup
    \expandafter\let\expandafter\@partlist\expandafter\@empty
    \expandafter\@for\expandafter\reserved@a
    \expandafter:\expandafter=\@partlist\do{%
      \ifx\@partlist\@empty
        \edef\@partlist{\reserved@a}%
      \else
        \edef\@partlist{\@partlist, \reserved@a}%
      \fi
    }%
    \typeout{********************%
             ********************%
             ********************%
             ******************%
    }%
    \ifx\@partlist\@empty
      \typeout{[stampinclude] No included files.}%
    \else
      \typeout{[stampinclude] Included files:}%
      \typeout{\@partlist}%
    \fi
    \typeout{********************%
             ********************%
             ********************%
             ******************%
    }%
  \endgroup
}
%    \end{macrocode}
%
%    \begin{macrocode}
%</package>
%    \end{macrocode}
%
% \section{Installation}
%
% \subsection{Download}
%
% \paragraph{Package.} This package is available on
% CTAN\footnote{\url{http://ctan.org/pkg/stampinclude}}:
% \begin{description}
% \item[\CTAN{macros/latex/contrib/oberdiek/stampinclude.dtx}] The source file.
% \item[\CTAN{macros/latex/contrib/oberdiek/stampinclude.pdf}] Documentation.
% \end{description}
%
%
% \paragraph{Bundle.} All the packages of the bundle `oberdiek'
% are also available in a TDS compliant ZIP archive. There
% the packages are already unpacked and the documentation files
% are generated. The files and directories obey the TDS standard.
% \begin{description}
% \item[\CTAN{install/macros/latex/contrib/oberdiek.tds.zip}]
% \end{description}
% \emph{TDS} refers to the standard ``A Directory Structure
% for \TeX\ Files'' (\CTAN{tds/tds.pdf}). Directories
% with \xfile{texmf} in their name are usually organized this way.
%
% \subsection{Bundle installation}
%
% \paragraph{Unpacking.} Unpack the \xfile{oberdiek.tds.zip} in the
% TDS tree (also known as \xfile{texmf} tree) of your choice.
% Example (linux):
% \begin{quote}
%   |unzip oberdiek.tds.zip -d ~/texmf|
% \end{quote}
%
% \paragraph{Script installation.}
% Check the directory \xfile{TDS:scripts/oberdiek/} for
% scripts that need further installation steps.
% Package \xpackage{attachfile2} comes with the Perl script
% \xfile{pdfatfi.pl} that should be installed in such a way
% that it can be called as \texttt{pdfatfi}.
% Example (linux):
% \begin{quote}
%   |chmod +x scripts/oberdiek/pdfatfi.pl|\\
%   |cp scripts/oberdiek/pdfatfi.pl /usr/local/bin/|
% \end{quote}
%
% \subsection{Package installation}
%
% \paragraph{Unpacking.} The \xfile{.dtx} file is a self-extracting
% \docstrip\ archive. The files are extracted by running the
% \xfile{.dtx} through \plainTeX:
% \begin{quote}
%   \verb|tex stampinclude.dtx|
% \end{quote}
%
% \paragraph{TDS.} Now the different files must be moved into
% the different directories in your installation TDS tree
% (also known as \xfile{texmf} tree):
% \begin{quote}
% \def\t{^^A
% \begin{tabular}{@{}>{\ttfamily}l@{ $\rightarrow$ }>{\ttfamily}l@{}}
%   stampinclude.sty & tex/latex/oberdiek/stampinclude.sty\\
%   stampinclude.pdf & doc/latex/oberdiek/stampinclude.pdf\\
%   stampinclude.dtx & source/latex/oberdiek/stampinclude.dtx\\
% \end{tabular}^^A
% }^^A
% \sbox0{\t}^^A
% \ifdim\wd0>\linewidth
%   \begingroup
%     \advance\linewidth by\leftmargin
%     \advance\linewidth by\rightmargin
%   \edef\x{\endgroup
%     \def\noexpand\lw{\the\linewidth}^^A
%   }\x
%   \def\lwbox{^^A
%     \leavevmode
%     \hbox to \linewidth{^^A
%       \kern-\leftmargin\relax
%       \hss
%       \usebox0
%       \hss
%       \kern-\rightmargin\relax
%     }^^A
%   }^^A
%   \ifdim\wd0>\lw
%     \sbox0{\small\t}^^A
%     \ifdim\wd0>\linewidth
%       \ifdim\wd0>\lw
%         \sbox0{\footnotesize\t}^^A
%         \ifdim\wd0>\linewidth
%           \ifdim\wd0>\lw
%             \sbox0{\scriptsize\t}^^A
%             \ifdim\wd0>\linewidth
%               \ifdim\wd0>\lw
%                 \sbox0{\tiny\t}^^A
%                 \ifdim\wd0>\linewidth
%                   \lwbox
%                 \else
%                   \usebox0
%                 \fi
%               \else
%                 \lwbox
%               \fi
%             \else
%               \usebox0
%             \fi
%           \else
%             \lwbox
%           \fi
%         \else
%           \usebox0
%         \fi
%       \else
%         \lwbox
%       \fi
%     \else
%       \usebox0
%     \fi
%   \else
%     \lwbox
%   \fi
% \else
%   \usebox0
% \fi
% \end{quote}
% If you have a \xfile{docstrip.cfg} that configures and enables \docstrip's
% TDS installing feature, then some files can already be in the right
% place, see the documentation of \docstrip.
%
% \subsection{Refresh file name databases}
%
% If your \TeX~distribution
% (\teTeX, \mikTeX, \dots) relies on file name databases, you must refresh
% these. For example, \teTeX\ users run \verb|texhash| or
% \verb|mktexlsr|.
%
% \subsection{Some details for the interested}
%
% \paragraph{Attached source.}
%
% The PDF documentation on CTAN also includes the
% \xfile{.dtx} source file. It can be extracted by
% AcrobatReader 6 or higher. Another option is \textsf{pdftk},
% e.g. unpack the file into the current directory:
% \begin{quote}
%   \verb|pdftk stampinclude.pdf unpack_files output .|
% \end{quote}
%
% \paragraph{Unpacking with \LaTeX.}
% The \xfile{.dtx} chooses its action depending on the format:
% \begin{description}
% \item[\plainTeX:] Run \docstrip\ and extract the files.
% \item[\LaTeX:] Generate the documentation.
% \end{description}
% If you insist on using \LaTeX\ for \docstrip\ (really,
% \docstrip\ does not need \LaTeX), then inform the autodetect routine
% about your intention:
% \begin{quote}
%   \verb|latex \let\install=y\input{stampinclude.dtx}|
% \end{quote}
% Do not forget to quote the argument according to the demands
% of your shell.
%
% \paragraph{Generating the documentation.}
% You can use both the \xfile{.dtx} or the \xfile{.drv} to generate
% the documentation. The process can be configured by the
% configuration file \xfile{ltxdoc.cfg}. For instance, put this
% line into this file, if you want to have A4 as paper format:
% \begin{quote}
%   \verb|\PassOptionsToClass{a4paper}{article}|
% \end{quote}
% An example follows how to generate the
% documentation with pdf\LaTeX:
% \begin{quote}
%\begin{verbatim}
%pdflatex stampinclude.dtx
%makeindex -s gind.ist stampinclude.idx
%pdflatex stampinclude.dtx
%makeindex -s gind.ist stampinclude.idx
%pdflatex stampinclude.dtx
%\end{verbatim}
% \end{quote}
%
% \section{Catalogue}
%
% The following XML file can be used as source for the
% \href{http://mirror.ctan.org/help/Catalogue/catalogue.html}{\TeX\ Catalogue}.
% The elements \texttt{caption} and \texttt{description} are imported
% from the original XML file from the Catalogue.
% The name of the XML file in the Catalogue is \xfile{stampinclude.xml}.
%    \begin{macrocode}
%<*catalogue>
<?xml version='1.0' encoding='us-ascii'?>
<!DOCTYPE entry SYSTEM 'catalogue.dtd'>
<entry datestamp='$Date$' modifier='$Author$' id='stampinclude'>
  <name>stampinclude</name>
  <caption>Inclusion based on .aux file date stamps.</caption>
  <authorref id='auth:oberdiek'/>
  <copyright owner='Heiko Oberdiek' year='2008'/>
  <license type='lppl1.3'/>
  <version number='1.1'/>
  <description>
    This package replaces <tt>\includeonly</tt> and selects the files for
    <tt>\include</tt> by inspecting the timestamp of the <tt>.aux</tt> file.
    The file is selected for inclusion if the <tt>.aux</tt> file does
    not yet exist or is older than the corresponding <tt>.tex</tt> file.
    <p/>
    The package is part of the <xref refid='oberdiek'>oberdiek</xref>
    bundle.
  </description>
  <documentation details='Package documentation'
      href='ctan:/macros/latex/contrib/oberdiek/stampinclude.pdf'/>
  <ctan file='true' path='/macros/latex/contrib/oberdiek/stampinclude.dtx'/>
  <miktex location='oberdiek'/>
  <texlive location='oberdiek'/>
  <install path='/macros/latex/contrib/oberdiek/oberdiek.tds.zip'/>
</entry>
%</catalogue>
%    \end{macrocode}
%
% \begin{thebibliography}{9}
% \bibitem{askinclude}
%   Pablo A. Straub, Heiko Oberdiek:
%   \textit{The \xpackage{askinclude} package};
%   2007/10/23 v2.0;
%   \CTAN{macros/latex/contrib/oberdiek/askinclude.pdf}.
%
% \bibitem{pdftexcmds}
%   Heiko Oberdiek:
%   \textit{The \xpackage{pdftexcmds} package};
%   2007/12/12 v0.3;
%   \CTAN{macros/latex/contrib/oberdiek/pdftexcmds.pdf}.
%
% \end{thebibliography}
%
% \begin{History}
%   \begin{Version}{2008/07/14 v1.0}
%   \item
%     First version.
%   \end{Version}
%   \begin{Version}{2016/05/16 v1.1}
%   \item
%     Documentation updates.
%   \end{Version}
% \end{History}
%
% \PrintIndex
%
% \Finale
\endinput
|
% \end{quote}
% Do not forget to quote the argument according to the demands
% of your shell.
%
% \paragraph{Generating the documentation.}
% You can use both the \xfile{.dtx} or the \xfile{.drv} to generate
% the documentation. The process can be configured by the
% configuration file \xfile{ltxdoc.cfg}. For instance, put this
% line into this file, if you want to have A4 as paper format:
% \begin{quote}
%   \verb|\PassOptionsToClass{a4paper}{article}|
% \end{quote}
% An example follows how to generate the
% documentation with pdf\LaTeX:
% \begin{quote}
%\begin{verbatim}
%pdflatex stampinclude.dtx
%makeindex -s gind.ist stampinclude.idx
%pdflatex stampinclude.dtx
%makeindex -s gind.ist stampinclude.idx
%pdflatex stampinclude.dtx
%\end{verbatim}
% \end{quote}
%
% \section{Catalogue}
%
% The following XML file can be used as source for the
% \href{http://mirror.ctan.org/help/Catalogue/catalogue.html}{\TeX\ Catalogue}.
% The elements \texttt{caption} and \texttt{description} are imported
% from the original XML file from the Catalogue.
% The name of the XML file in the Catalogue is \xfile{stampinclude.xml}.
%    \begin{macrocode}
%<*catalogue>
<?xml version='1.0' encoding='us-ascii'?>
<!DOCTYPE entry SYSTEM 'catalogue.dtd'>
<entry datestamp='$Date$' modifier='$Author$' id='stampinclude'>
  <name>stampinclude</name>
  <caption>Inclusion based on .aux file date stamps.</caption>
  <authorref id='auth:oberdiek'/>
  <copyright owner='Heiko Oberdiek' year='2008'/>
  <license type='lppl1.3'/>
  <version number='1.1'/>
  <description>
    This package replaces <tt>\includeonly</tt> and selects the files for
    <tt>\include</tt> by inspecting the timestamp of the <tt>.aux</tt> file.
    The file is selected for inclusion if the <tt>.aux</tt> file does
    not yet exist or is older than the corresponding <tt>.tex</tt> file.
    <p/>
    The package is part of the <xref refid='oberdiek'>oberdiek</xref>
    bundle.
  </description>
  <documentation details='Package documentation'
      href='ctan:/macros/latex/contrib/oberdiek/stampinclude.pdf'/>
  <ctan file='true' path='/macros/latex/contrib/oberdiek/stampinclude.dtx'/>
  <miktex location='oberdiek'/>
  <texlive location='oberdiek'/>
  <install path='/macros/latex/contrib/oberdiek/oberdiek.tds.zip'/>
</entry>
%</catalogue>
%    \end{macrocode}
%
% \begin{thebibliography}{9}
% \bibitem{askinclude}
%   Pablo A. Straub, Heiko Oberdiek:
%   \textit{The \xpackage{askinclude} package};
%   2007/10/23 v2.0;
%   \CTAN{macros/latex/contrib/oberdiek/askinclude.pdf}.
%
% \bibitem{pdftexcmds}
%   Heiko Oberdiek:
%   \textit{The \xpackage{pdftexcmds} package};
%   2007/12/12 v0.3;
%   \CTAN{macros/latex/contrib/oberdiek/pdftexcmds.pdf}.
%
% \end{thebibliography}
%
% \begin{History}
%   \begin{Version}{2008/07/14 v1.0}
%   \item
%     First version.
%   \end{Version}
%   \begin{Version}{2016/05/16 v1.1}
%   \item
%     Documentation updates.
%   \end{Version}
% \end{History}
%
% \PrintIndex
%
% \Finale
\endinput

%        (quote the arguments according to the demands of your shell)
%
% Documentation:
%    (a) If stampinclude.drv is present:
%           latex stampinclude.drv
%    (b) Without stampinclude.drv:
%           latex stampinclude.dtx; ...
%    The class ltxdoc loads the configuration file ltxdoc.cfg
%    if available. Here you can specify further options, e.g.
%    use A4 as paper format:
%       \PassOptionsToClass{a4paper}{article}
%
%    Programm calls to get the documentation (example):
%       pdflatex stampinclude.dtx
%       makeindex -s gind.ist stampinclude.idx
%       pdflatex stampinclude.dtx
%       makeindex -s gind.ist stampinclude.idx
%       pdflatex stampinclude.dtx
%
% Installation:
%    TDS:tex/latex/oberdiek/stampinclude.sty
%    TDS:doc/latex/oberdiek/stampinclude.pdf
%    TDS:source/latex/oberdiek/stampinclude.dtx
%
%<*ignore>
\begingroup
  \catcode123=1 %
  \catcode125=2 %
  \def\x{LaTeX2e}%
\expandafter\endgroup
\ifcase 0\ifx\install y1\fi\expandafter
         \ifx\csname processbatchFile\endcsname\relax\else1\fi
         \ifx\fmtname\x\else 1\fi\relax
\else\csname fi\endcsname
%</ignore>
%<*install>
\input docstrip.tex
\Msg{************************************************************************}
\Msg{* Installation}
\Msg{* Package: stampinclude 2016/05/16 v1.1 Include files based on time stamps (HO)}
\Msg{************************************************************************}

\keepsilent
\askforoverwritefalse

\let\MetaPrefix\relax
\preamble

This is a generated file.

Project: stampinclude
Version: 2016/05/16 v1.1

Copyright (C) 2008 by
   Heiko Oberdiek <heiko.oberdiek at googlemail.com>

This work may be distributed and/or modified under the
conditions of the LaTeX Project Public License, either
version 1.3c of this license or (at your option) any later
version. This version of this license is in
   http://www.latex-project.org/lppl/lppl-1-3c.txt
and the latest version of this license is in
   http://www.latex-project.org/lppl.txt
and version 1.3 or later is part of all distributions of
LaTeX version 2005/12/01 or later.

This work has the LPPL maintenance status "maintained".

This Current Maintainer of this work is Heiko Oberdiek.

This work consists of the main source file stampinclude.dtx
and the derived files
   stampinclude.sty, stampinclude.pdf, stampinclude.ins, stampinclude.drv.

\endpreamble
\let\MetaPrefix\DoubleperCent

\generate{%
  \file{stampinclude.ins}{\from{stampinclude.dtx}{install}}%
  \file{stampinclude.drv}{\from{stampinclude.dtx}{driver}}%
  \usedir{tex/latex/oberdiek}%
  \file{stampinclude.sty}{\from{stampinclude.dtx}{package}}%
  \nopreamble
  \nopostamble
  \usedir{source/latex/oberdiek/catalogue}%
  \file{stampinclude.xml}{\from{stampinclude.dtx}{catalogue}}%
}

\catcode32=13\relax% active space
\let =\space%
\Msg{************************************************************************}
\Msg{*}
\Msg{* To finish the installation you have to move the following}
\Msg{* file into a directory searched by TeX:}
\Msg{*}
\Msg{*     stampinclude.sty}
\Msg{*}
\Msg{* To produce the documentation run the file `stampinclude.drv'}
\Msg{* through LaTeX.}
\Msg{*}
\Msg{* Happy TeXing!}
\Msg{*}
\Msg{************************************************************************}

\endbatchfile
%</install>
%<*ignore>
\fi
%</ignore>
%<*driver>
\NeedsTeXFormat{LaTeX2e}
\ProvidesFile{stampinclude.drv}%
  [2016/05/16 v1.1 Include files based on time stamps (HO)]%
\documentclass{ltxdoc}
\usepackage{holtxdoc}[2011/11/22]
\begin{document}
  \DocInput{stampinclude.dtx}%
\end{document}
%</driver>
% \fi
%
%
% \CharacterTable
%  {Upper-case    \A\B\C\D\E\F\G\H\I\J\K\L\M\N\O\P\Q\R\S\T\U\V\W\X\Y\Z
%   Lower-case    \a\b\c\d\e\f\g\h\i\j\k\l\m\n\o\p\q\r\s\t\u\v\w\x\y\z
%   Digits        \0\1\2\3\4\5\6\7\8\9
%   Exclamation   \!     Double quote  \"     Hash (number) \#
%   Dollar        \$     Percent       \%     Ampersand     \&
%   Acute accent  \'     Left paren    \(     Right paren   \)
%   Asterisk      \*     Plus          \+     Comma         \,
%   Minus         \-     Point         \.     Solidus       \/
%   Colon         \:     Semicolon     \;     Less than     \<
%   Equals        \=     Greater than  \>     Question mark \?
%   Commercial at \@     Left bracket  \[     Backslash     \\
%   Right bracket \]     Circumflex    \^     Underscore    \_
%   Grave accent  \`     Left brace    \{     Vertical bar  \|
%   Right brace   \}     Tilde         \~}
%
% \GetFileInfo{stampinclude.drv}
%
% \title{The \xpackage{stampinclude} package}
% \date{2016/05/16 v1.1}
% \author{Heiko Oberdiek\thanks
% {Please report any issues at https://github.com/ho-tex/oberdiek/issues}\\
% \xemail{heiko.oberdiek at googlemail.com}}
%
% \maketitle
%
% \begin{abstract}
% The package replaces \cs{includeonly} and selects the files for
% \cs{include} by inspecting the time stamp of the \xext{aux} file.
% The file is selected for inclusion if the \xext{aux} file does
% not yet exist or is older than the corresponding \xext{tex} file.
% \end{abstract}
%
% \tableofcontents
%
% \section{Documentation}
%
% \subsection{Introduction}
% \label{sec:intro}
%
% \LaTeX\ provides two commands \cs{include} and \cs{includeonly}
% that helps in organizing large projects.
% Example for a master file:
%\begin{quote}
%\begin{verbatim}
%\documentclass{book}
%  % \includeonly{}
%\begin{document}
% \include{fileA}
% \include{fileB}
% \include{fileC}
%\end{document}
%\end{verbatim}
%\end{quote}
% All files are read and compiled if \cs{includeonly} is not
% executed. Otherwise you can give \cs{includeonly} a list
% of files in the preamble, e.g.:
% \begin{quote}
%   |\includeonly{fileA,fileC}|
% \end{quote}
% Now only files \xfile{fileA.tex} and \xfile{fileC.tex} are read
% and compiled.
%
% If you change file \xfile{fileB.tex} and want to see only this
% file, then you must change the line with \cs{includeonly} to
% \begin{quote}
%   |\includeonly{fileB}|
% \end{quote}
% It is tedious to do this again and again, if different files
% are changed.
%
% Package \xpackage{askinclude} \cite{askinclude}
% offers a solution for this problem. It interactively asks
% for the files to be included and saves the user from
% editing the master file.
%
% This package \xpackage{stampinclude} goes another way.
% \LaTeX\ reads and writes a separate \xext{aux} file for each
% file that is included by \cs{include}. There \LaTeX\ remembers
% counter valuses. Changed \xext{tex}
% files can therefore be detected by comparing the file date stamp of
% the \xext{tex} file with the date stamp of its \xext{aux} file.
% Since version 1.30.0 \pdfTeX\ provides \cs{pdffilemoddate}
% that reads the file date stamp. Thus this package uses this
% command and redefines
% \cs{include} to include the files that do not have \xext{aux}
% files yet or that are newer than its \xext{aux} file.
% \cs{includeonly} is ignored.
%
% \subsection{Usage}
%
% The package is loaded as normal \LaTeX\ package without options:
% \begin{quote}
%   |\usepackage{stampinclude}|
% \end{quote}
% Alternatively the package may be loaded on the command line
% (Example for shell `bash'):
% \begin{center}
%   |latex '\AtBeginDocument{\usepackage{stampinclude}}\input{master}'|
% \end{center}
% Without \cs{AtBeginDocument} (and \cs{RequirePackage} instead of
% \cs{usepackage}) \TeX\ would name the document \xfile{stampinclude.dvi}
% instead of \xfile{master.dvi}.
%
% \subsection{Limitations}
%
% \subsubsection{Other file dependencies}
%
% A file that is included by \cs{include} may input ore reference
% other files:
% \begin{itemize}
% \item other \TeX\ files using \cs{input},
% \item graphics files (\cs{includegraphics}),
% \item listings of external files,
% \item ...
% \end{itemize}
% Updates of those files are not detected by this package.
% It limits the date stamp comparison of an \xext{aux} file
% to its \xext{tex} file.
%
% \subsubsection{\cs{include} dependencies}
%
% In the example, given in the introduction \ref{sec:intro},
% three files \xfile{fileA}, \xfile{fileB}, and \xfile{fileC}
% are included in this order. Now file \xfile{fileA} is changed by adding
% four pages, \xfile{fileB} remains untouched, and \xfile{fileC} is
% also updated. Then the package only selects \xfile{fileA} and
% \xfile{fileC} for inclusion. File \xfile{fileB} is not included.
% But \LaTeX\ has stored the counter values that are active
% at the end of \xfile{fileB} in \xfile{fileB.aux} in one of the
% previous runs when \xfile{fileB} was included.
% However the later addition of four pages in \xfile{fileA}
% was not known at that time. Therefore \xfile{fileB.aux}
% is out of date and the inclusion of file \xfile{fileC}
% starts with wrong counter values (especially the page counter).
%
% \subsubsection{Summary}
%
% This package \xpackage{stampinclude} and the \cs{include} feature
% helps in accelerating the \LaTeX\ compilation.
% But it is not intended for generating the final version.
% For the final version of the document it is better to include
% \emph{all} files to get all counter values right.
% Then this package and any \cs{includeonly} lines should be commented out:
%\begin{quote}
%  |% \usepackage{stampinclude}|\\
%  |% \includeonly{...}|
%\end{quote}
%
% \subsection{Requirements}
%
% \begin{itemize}
% \item \pdfTeX\ v1.30.0 (because of \cs{pdffilemoddate}
%   and \cs{pdfstrcmp}),\\
%   both modes for DVI and PDF are supported.
% \item Alternatively Lua\TeX\ may be used.
%   It lacks \cs{pdffilemoddate} and \cs{pdfstrcmp}. But its services
%   are provided by package \xpackage{pdftexcmds} \cite{pdftexcmds}
%   that is automatically loaded.
% \end{itemize}
%
% \StopEventually{
% }
%
% \section{Implementation}
%
%    \begin{macrocode}
%<*package>
\NeedsTeXFormat{LaTeX2e}
\ProvidesPackage{stampinclude}
  [2016/05/16 v1.1 Include files based on time stamps (HO)]%
%    \end{macrocode}
%
%    \begin{macrocode}
\RequirePackage{pdftexcmds}[2007/12/12]%
%    \end{macrocode}
%
%    \begin{macrocode}
\begingroup
  \chardef\x=1 %
  \expandafter\ifx\csname pdf@filemoddate\endcsname\relax
    \chardef\x=0 %
  \fi
  \expandafter\ifx\csname pdf@strcmp\endcsname\relax
    \chardef\x=0 %
  \fi
\expandafter\endgroup\ifcase\x
  \PackageWarningNoLine{stampinclude}{%
    \string\pdffilemoddate\space or %
    \string\pdfstrcmp\space are not found,\MessageBreak
    that are provided by pdfTeX >= 1.30.0.\MessageBreak
    Also LuaTeX is not detected.\MessageBreak
    Therefore package loading is aborted%
  }%
  \expandafter\endinput
\fi
%    \end{macrocode}
%
%    \begin{macro}{\SInc@org@include}
%    \begin{macrocode}
\let\SInc@org@include\@include
%    \end{macrocode}
%    \end{macro}
%    \begin{macro}{\@include}
%    \begin{macrocode}
\def\@include#1 {%
  \IfFileExists{#1.aux}{%
    \ifnum\pdf@strcmp{\pdf@filemoddate{#1.aux}}%
                     {\pdf@filemoddate{#1.tex}}<0 %
      \ifx\@partlist\@empty
        \gdef\@partlist{{#1}}%
      \else
        \g@addto@macro\@partlist{,{#1}}%
      \fi
    \fi
  }{%
    \ifx\@partlist\@empty
      \gdef\@partlist{{#1}}%
    \else
      \g@addto@macro\@partlist{,{#1}}%
    \fi
  }%
  \SInc@org@include{#1} \relax
}
%    \end{macrocode}
%    \end{macro}
%
%    \begin{macro}{\includeonly}
%    Macro \cs{includeonly} is ignored.
%    \begin{macrocode}
\renewcommand*{\includeonly}[1]{%
  \PackageInfo{stampinclude}{%
    Ignoring \string\includeonly
  }%
}
%    \end{macrocode}
%    \end{macro}
%
%    Simulate \cs{includeonly}.
%    \begin{macrocode}
\@partswtrue
\gdef\@partlist{}
%    \end{macrocode}
%
%    Print included files at end of document.
%    \begin{macrocode}
\AtEndDocument{%
  \begingroup
    \expandafter\let\expandafter\@partlist\expandafter\@empty
    \expandafter\@for\expandafter\reserved@a
    \expandafter:\expandafter=\@partlist\do{%
      \ifx\@partlist\@empty
        \edef\@partlist{\reserved@a}%
      \else
        \edef\@partlist{\@partlist, \reserved@a}%
      \fi
    }%
    \typeout{********************%
             ********************%
             ********************%
             ******************%
    }%
    \ifx\@partlist\@empty
      \typeout{[stampinclude] No included files.}%
    \else
      \typeout{[stampinclude] Included files:}%
      \typeout{\@partlist}%
    \fi
    \typeout{********************%
             ********************%
             ********************%
             ******************%
    }%
  \endgroup
}
%    \end{macrocode}
%
%    \begin{macrocode}
%</package>
%    \end{macrocode}
%
% \section{Installation}
%
% \subsection{Download}
%
% \paragraph{Package.} This package is available on
% CTAN\footnote{\url{http://ctan.org/pkg/stampinclude}}:
% \begin{description}
% \item[\CTAN{macros/latex/contrib/oberdiek/stampinclude.dtx}] The source file.
% \item[\CTAN{macros/latex/contrib/oberdiek/stampinclude.pdf}] Documentation.
% \end{description}
%
%
% \paragraph{Bundle.} All the packages of the bundle `oberdiek'
% are also available in a TDS compliant ZIP archive. There
% the packages are already unpacked and the documentation files
% are generated. The files and directories obey the TDS standard.
% \begin{description}
% \item[\CTAN{install/macros/latex/contrib/oberdiek.tds.zip}]
% \end{description}
% \emph{TDS} refers to the standard ``A Directory Structure
% for \TeX\ Files'' (\CTAN{tds/tds.pdf}). Directories
% with \xfile{texmf} in their name are usually organized this way.
%
% \subsection{Bundle installation}
%
% \paragraph{Unpacking.} Unpack the \xfile{oberdiek.tds.zip} in the
% TDS tree (also known as \xfile{texmf} tree) of your choice.
% Example (linux):
% \begin{quote}
%   |unzip oberdiek.tds.zip -d ~/texmf|
% \end{quote}
%
% \paragraph{Script installation.}
% Check the directory \xfile{TDS:scripts/oberdiek/} for
% scripts that need further installation steps.
% Package \xpackage{attachfile2} comes with the Perl script
% \xfile{pdfatfi.pl} that should be installed in such a way
% that it can be called as \texttt{pdfatfi}.
% Example (linux):
% \begin{quote}
%   |chmod +x scripts/oberdiek/pdfatfi.pl|\\
%   |cp scripts/oberdiek/pdfatfi.pl /usr/local/bin/|
% \end{quote}
%
% \subsection{Package installation}
%
% \paragraph{Unpacking.} The \xfile{.dtx} file is a self-extracting
% \docstrip\ archive. The files are extracted by running the
% \xfile{.dtx} through \plainTeX:
% \begin{quote}
%   \verb|tex stampinclude.dtx|
% \end{quote}
%
% \paragraph{TDS.} Now the different files must be moved into
% the different directories in your installation TDS tree
% (also known as \xfile{texmf} tree):
% \begin{quote}
% \def\t{^^A
% \begin{tabular}{@{}>{\ttfamily}l@{ $\rightarrow$ }>{\ttfamily}l@{}}
%   stampinclude.sty & tex/latex/oberdiek/stampinclude.sty\\
%   stampinclude.pdf & doc/latex/oberdiek/stampinclude.pdf\\
%   stampinclude.dtx & source/latex/oberdiek/stampinclude.dtx\\
% \end{tabular}^^A
% }^^A
% \sbox0{\t}^^A
% \ifdim\wd0>\linewidth
%   \begingroup
%     \advance\linewidth by\leftmargin
%     \advance\linewidth by\rightmargin
%   \edef\x{\endgroup
%     \def\noexpand\lw{\the\linewidth}^^A
%   }\x
%   \def\lwbox{^^A
%     \leavevmode
%     \hbox to \linewidth{^^A
%       \kern-\leftmargin\relax
%       \hss
%       \usebox0
%       \hss
%       \kern-\rightmargin\relax
%     }^^A
%   }^^A
%   \ifdim\wd0>\lw
%     \sbox0{\small\t}^^A
%     \ifdim\wd0>\linewidth
%       \ifdim\wd0>\lw
%         \sbox0{\footnotesize\t}^^A
%         \ifdim\wd0>\linewidth
%           \ifdim\wd0>\lw
%             \sbox0{\scriptsize\t}^^A
%             \ifdim\wd0>\linewidth
%               \ifdim\wd0>\lw
%                 \sbox0{\tiny\t}^^A
%                 \ifdim\wd0>\linewidth
%                   \lwbox
%                 \else
%                   \usebox0
%                 \fi
%               \else
%                 \lwbox
%               \fi
%             \else
%               \usebox0
%             \fi
%           \else
%             \lwbox
%           \fi
%         \else
%           \usebox0
%         \fi
%       \else
%         \lwbox
%       \fi
%     \else
%       \usebox0
%     \fi
%   \else
%     \lwbox
%   \fi
% \else
%   \usebox0
% \fi
% \end{quote}
% If you have a \xfile{docstrip.cfg} that configures and enables \docstrip's
% TDS installing feature, then some files can already be in the right
% place, see the documentation of \docstrip.
%
% \subsection{Refresh file name databases}
%
% If your \TeX~distribution
% (\teTeX, \mikTeX, \dots) relies on file name databases, you must refresh
% these. For example, \teTeX\ users run \verb|texhash| or
% \verb|mktexlsr|.
%
% \subsection{Some details for the interested}
%
% \paragraph{Attached source.}
%
% The PDF documentation on CTAN also includes the
% \xfile{.dtx} source file. It can be extracted by
% AcrobatReader 6 or higher. Another option is \textsf{pdftk},
% e.g. unpack the file into the current directory:
% \begin{quote}
%   \verb|pdftk stampinclude.pdf unpack_files output .|
% \end{quote}
%
% \paragraph{Unpacking with \LaTeX.}
% The \xfile{.dtx} chooses its action depending on the format:
% \begin{description}
% \item[\plainTeX:] Run \docstrip\ and extract the files.
% \item[\LaTeX:] Generate the documentation.
% \end{description}
% If you insist on using \LaTeX\ for \docstrip\ (really,
% \docstrip\ does not need \LaTeX), then inform the autodetect routine
% about your intention:
% \begin{quote}
%   \verb|latex \let\install=y% \iffalse meta-comment
%
% File: stampinclude.dtx
% Version: 2016/05/16 v1.1
% Info: Include files based on time stamps
%
% Copyright (C) 2008 by
%    Heiko Oberdiek <heiko.oberdiek at googlemail.com>
%    2016
%    https://github.com/ho-tex/oberdiek/issues
%
% This work may be distributed and/or modified under the
% conditions of the LaTeX Project Public License, either
% version 1.3c of this license or (at your option) any later
% version. This version of this license is in
%    http://www.latex-project.org/lppl/lppl-1-3c.txt
% and the latest version of this license is in
%    http://www.latex-project.org/lppl.txt
% and version 1.3 or later is part of all distributions of
% LaTeX version 2005/12/01 or later.
%
% This work has the LPPL maintenance status "maintained".
%
% This Current Maintainer of this work is Heiko Oberdiek.
%
% This work consists of the main source file stampinclude.dtx
% and the derived files
%    stampinclude.sty, stampinclude.pdf, stampinclude.ins, stampinclude.drv.
%
% Distribution:
%    CTAN:macros/latex/contrib/oberdiek/stampinclude.dtx
%    CTAN:macros/latex/contrib/oberdiek/stampinclude.pdf
%
% Unpacking:
%    (a) If stampinclude.ins is present:
%           tex stampinclude.ins
%    (b) Without stampinclude.ins:
%           tex stampinclude.dtx
%    (c) If you insist on using LaTeX
%           latex \let\install=y% \iffalse meta-comment
%
% File: stampinclude.dtx
% Version: 2016/05/16 v1.1
% Info: Include files based on time stamps
%
% Copyright (C) 2008 by
%    Heiko Oberdiek <heiko.oberdiek at googlemail.com>
%    2016
%    https://github.com/ho-tex/oberdiek/issues
%
% This work may be distributed and/or modified under the
% conditions of the LaTeX Project Public License, either
% version 1.3c of this license or (at your option) any later
% version. This version of this license is in
%    http://www.latex-project.org/lppl/lppl-1-3c.txt
% and the latest version of this license is in
%    http://www.latex-project.org/lppl.txt
% and version 1.3 or later is part of all distributions of
% LaTeX version 2005/12/01 or later.
%
% This work has the LPPL maintenance status "maintained".
%
% This Current Maintainer of this work is Heiko Oberdiek.
%
% This work consists of the main source file stampinclude.dtx
% and the derived files
%    stampinclude.sty, stampinclude.pdf, stampinclude.ins, stampinclude.drv.
%
% Distribution:
%    CTAN:macros/latex/contrib/oberdiek/stampinclude.dtx
%    CTAN:macros/latex/contrib/oberdiek/stampinclude.pdf
%
% Unpacking:
%    (a) If stampinclude.ins is present:
%           tex stampinclude.ins
%    (b) Without stampinclude.ins:
%           tex stampinclude.dtx
%    (c) If you insist on using LaTeX
%           latex \let\install=y\input{stampinclude.dtx}
%        (quote the arguments according to the demands of your shell)
%
% Documentation:
%    (a) If stampinclude.drv is present:
%           latex stampinclude.drv
%    (b) Without stampinclude.drv:
%           latex stampinclude.dtx; ...
%    The class ltxdoc loads the configuration file ltxdoc.cfg
%    if available. Here you can specify further options, e.g.
%    use A4 as paper format:
%       \PassOptionsToClass{a4paper}{article}
%
%    Programm calls to get the documentation (example):
%       pdflatex stampinclude.dtx
%       makeindex -s gind.ist stampinclude.idx
%       pdflatex stampinclude.dtx
%       makeindex -s gind.ist stampinclude.idx
%       pdflatex stampinclude.dtx
%
% Installation:
%    TDS:tex/latex/oberdiek/stampinclude.sty
%    TDS:doc/latex/oberdiek/stampinclude.pdf
%    TDS:source/latex/oberdiek/stampinclude.dtx
%
%<*ignore>
\begingroup
  \catcode123=1 %
  \catcode125=2 %
  \def\x{LaTeX2e}%
\expandafter\endgroup
\ifcase 0\ifx\install y1\fi\expandafter
         \ifx\csname processbatchFile\endcsname\relax\else1\fi
         \ifx\fmtname\x\else 1\fi\relax
\else\csname fi\endcsname
%</ignore>
%<*install>
\input docstrip.tex
\Msg{************************************************************************}
\Msg{* Installation}
\Msg{* Package: stampinclude 2016/05/16 v1.1 Include files based on time stamps (HO)}
\Msg{************************************************************************}

\keepsilent
\askforoverwritefalse

\let\MetaPrefix\relax
\preamble

This is a generated file.

Project: stampinclude
Version: 2016/05/16 v1.1

Copyright (C) 2008 by
   Heiko Oberdiek <heiko.oberdiek at googlemail.com>

This work may be distributed and/or modified under the
conditions of the LaTeX Project Public License, either
version 1.3c of this license or (at your option) any later
version. This version of this license is in
   http://www.latex-project.org/lppl/lppl-1-3c.txt
and the latest version of this license is in
   http://www.latex-project.org/lppl.txt
and version 1.3 or later is part of all distributions of
LaTeX version 2005/12/01 or later.

This work has the LPPL maintenance status "maintained".

This Current Maintainer of this work is Heiko Oberdiek.

This work consists of the main source file stampinclude.dtx
and the derived files
   stampinclude.sty, stampinclude.pdf, stampinclude.ins, stampinclude.drv.

\endpreamble
\let\MetaPrefix\DoubleperCent

\generate{%
  \file{stampinclude.ins}{\from{stampinclude.dtx}{install}}%
  \file{stampinclude.drv}{\from{stampinclude.dtx}{driver}}%
  \usedir{tex/latex/oberdiek}%
  \file{stampinclude.sty}{\from{stampinclude.dtx}{package}}%
  \nopreamble
  \nopostamble
  \usedir{source/latex/oberdiek/catalogue}%
  \file{stampinclude.xml}{\from{stampinclude.dtx}{catalogue}}%
}

\catcode32=13\relax% active space
\let =\space%
\Msg{************************************************************************}
\Msg{*}
\Msg{* To finish the installation you have to move the following}
\Msg{* file into a directory searched by TeX:}
\Msg{*}
\Msg{*     stampinclude.sty}
\Msg{*}
\Msg{* To produce the documentation run the file `stampinclude.drv'}
\Msg{* through LaTeX.}
\Msg{*}
\Msg{* Happy TeXing!}
\Msg{*}
\Msg{************************************************************************}

\endbatchfile
%</install>
%<*ignore>
\fi
%</ignore>
%<*driver>
\NeedsTeXFormat{LaTeX2e}
\ProvidesFile{stampinclude.drv}%
  [2016/05/16 v1.1 Include files based on time stamps (HO)]%
\documentclass{ltxdoc}
\usepackage{holtxdoc}[2011/11/22]
\begin{document}
  \DocInput{stampinclude.dtx}%
\end{document}
%</driver>
% \fi
%
%
% \CharacterTable
%  {Upper-case    \A\B\C\D\E\F\G\H\I\J\K\L\M\N\O\P\Q\R\S\T\U\V\W\X\Y\Z
%   Lower-case    \a\b\c\d\e\f\g\h\i\j\k\l\m\n\o\p\q\r\s\t\u\v\w\x\y\z
%   Digits        \0\1\2\3\4\5\6\7\8\9
%   Exclamation   \!     Double quote  \"     Hash (number) \#
%   Dollar        \$     Percent       \%     Ampersand     \&
%   Acute accent  \'     Left paren    \(     Right paren   \)
%   Asterisk      \*     Plus          \+     Comma         \,
%   Minus         \-     Point         \.     Solidus       \/
%   Colon         \:     Semicolon     \;     Less than     \<
%   Equals        \=     Greater than  \>     Question mark \?
%   Commercial at \@     Left bracket  \[     Backslash     \\
%   Right bracket \]     Circumflex    \^     Underscore    \_
%   Grave accent  \`     Left brace    \{     Vertical bar  \|
%   Right brace   \}     Tilde         \~}
%
% \GetFileInfo{stampinclude.drv}
%
% \title{The \xpackage{stampinclude} package}
% \date{2016/05/16 v1.1}
% \author{Heiko Oberdiek\thanks
% {Please report any issues at https://github.com/ho-tex/oberdiek/issues}\\
% \xemail{heiko.oberdiek at googlemail.com}}
%
% \maketitle
%
% \begin{abstract}
% The package replaces \cs{includeonly} and selects the files for
% \cs{include} by inspecting the time stamp of the \xext{aux} file.
% The file is selected for inclusion if the \xext{aux} file does
% not yet exist or is older than the corresponding \xext{tex} file.
% \end{abstract}
%
% \tableofcontents
%
% \section{Documentation}
%
% \subsection{Introduction}
% \label{sec:intro}
%
% \LaTeX\ provides two commands \cs{include} and \cs{includeonly}
% that helps in organizing large projects.
% Example for a master file:
%\begin{quote}
%\begin{verbatim}
%\documentclass{book}
%  % \includeonly{}
%\begin{document}
% \include{fileA}
% \include{fileB}
% \include{fileC}
%\end{document}
%\end{verbatim}
%\end{quote}
% All files are read and compiled if \cs{includeonly} is not
% executed. Otherwise you can give \cs{includeonly} a list
% of files in the preamble, e.g.:
% \begin{quote}
%   |\includeonly{fileA,fileC}|
% \end{quote}
% Now only files \xfile{fileA.tex} and \xfile{fileC.tex} are read
% and compiled.
%
% If you change file \xfile{fileB.tex} and want to see only this
% file, then you must change the line with \cs{includeonly} to
% \begin{quote}
%   |\includeonly{fileB}|
% \end{quote}
% It is tedious to do this again and again, if different files
% are changed.
%
% Package \xpackage{askinclude} \cite{askinclude}
% offers a solution for this problem. It interactively asks
% for the files to be included and saves the user from
% editing the master file.
%
% This package \xpackage{stampinclude} goes another way.
% \LaTeX\ reads and writes a separate \xext{aux} file for each
% file that is included by \cs{include}. There \LaTeX\ remembers
% counter valuses. Changed \xext{tex}
% files can therefore be detected by comparing the file date stamp of
% the \xext{tex} file with the date stamp of its \xext{aux} file.
% Since version 1.30.0 \pdfTeX\ provides \cs{pdffilemoddate}
% that reads the file date stamp. Thus this package uses this
% command and redefines
% \cs{include} to include the files that do not have \xext{aux}
% files yet or that are newer than its \xext{aux} file.
% \cs{includeonly} is ignored.
%
% \subsection{Usage}
%
% The package is loaded as normal \LaTeX\ package without options:
% \begin{quote}
%   |\usepackage{stampinclude}|
% \end{quote}
% Alternatively the package may be loaded on the command line
% (Example for shell `bash'):
% \begin{center}
%   |latex '\AtBeginDocument{\usepackage{stampinclude}}\input{master}'|
% \end{center}
% Without \cs{AtBeginDocument} (and \cs{RequirePackage} instead of
% \cs{usepackage}) \TeX\ would name the document \xfile{stampinclude.dvi}
% instead of \xfile{master.dvi}.
%
% \subsection{Limitations}
%
% \subsubsection{Other file dependencies}
%
% A file that is included by \cs{include} may input ore reference
% other files:
% \begin{itemize}
% \item other \TeX\ files using \cs{input},
% \item graphics files (\cs{includegraphics}),
% \item listings of external files,
% \item ...
% \end{itemize}
% Updates of those files are not detected by this package.
% It limits the date stamp comparison of an \xext{aux} file
% to its \xext{tex} file.
%
% \subsubsection{\cs{include} dependencies}
%
% In the example, given in the introduction \ref{sec:intro},
% three files \xfile{fileA}, \xfile{fileB}, and \xfile{fileC}
% are included in this order. Now file \xfile{fileA} is changed by adding
% four pages, \xfile{fileB} remains untouched, and \xfile{fileC} is
% also updated. Then the package only selects \xfile{fileA} and
% \xfile{fileC} for inclusion. File \xfile{fileB} is not included.
% But \LaTeX\ has stored the counter values that are active
% at the end of \xfile{fileB} in \xfile{fileB.aux} in one of the
% previous runs when \xfile{fileB} was included.
% However the later addition of four pages in \xfile{fileA}
% was not known at that time. Therefore \xfile{fileB.aux}
% is out of date and the inclusion of file \xfile{fileC}
% starts with wrong counter values (especially the page counter).
%
% \subsubsection{Summary}
%
% This package \xpackage{stampinclude} and the \cs{include} feature
% helps in accelerating the \LaTeX\ compilation.
% But it is not intended for generating the final version.
% For the final version of the document it is better to include
% \emph{all} files to get all counter values right.
% Then this package and any \cs{includeonly} lines should be commented out:
%\begin{quote}
%  |% \usepackage{stampinclude}|\\
%  |% \includeonly{...}|
%\end{quote}
%
% \subsection{Requirements}
%
% \begin{itemize}
% \item \pdfTeX\ v1.30.0 (because of \cs{pdffilemoddate}
%   and \cs{pdfstrcmp}),\\
%   both modes for DVI and PDF are supported.
% \item Alternatively Lua\TeX\ may be used.
%   It lacks \cs{pdffilemoddate} and \cs{pdfstrcmp}. But its services
%   are provided by package \xpackage{pdftexcmds} \cite{pdftexcmds}
%   that is automatically loaded.
% \end{itemize}
%
% \StopEventually{
% }
%
% \section{Implementation}
%
%    \begin{macrocode}
%<*package>
\NeedsTeXFormat{LaTeX2e}
\ProvidesPackage{stampinclude}
  [2016/05/16 v1.1 Include files based on time stamps (HO)]%
%    \end{macrocode}
%
%    \begin{macrocode}
\RequirePackage{pdftexcmds}[2007/12/12]%
%    \end{macrocode}
%
%    \begin{macrocode}
\begingroup
  \chardef\x=1 %
  \expandafter\ifx\csname pdf@filemoddate\endcsname\relax
    \chardef\x=0 %
  \fi
  \expandafter\ifx\csname pdf@strcmp\endcsname\relax
    \chardef\x=0 %
  \fi
\expandafter\endgroup\ifcase\x
  \PackageWarningNoLine{stampinclude}{%
    \string\pdffilemoddate\space or %
    \string\pdfstrcmp\space are not found,\MessageBreak
    that are provided by pdfTeX >= 1.30.0.\MessageBreak
    Also LuaTeX is not detected.\MessageBreak
    Therefore package loading is aborted%
  }%
  \expandafter\endinput
\fi
%    \end{macrocode}
%
%    \begin{macro}{\SInc@org@include}
%    \begin{macrocode}
\let\SInc@org@include\@include
%    \end{macrocode}
%    \end{macro}
%    \begin{macro}{\@include}
%    \begin{macrocode}
\def\@include#1 {%
  \IfFileExists{#1.aux}{%
    \ifnum\pdf@strcmp{\pdf@filemoddate{#1.aux}}%
                     {\pdf@filemoddate{#1.tex}}<0 %
      \ifx\@partlist\@empty
        \gdef\@partlist{{#1}}%
      \else
        \g@addto@macro\@partlist{,{#1}}%
      \fi
    \fi
  }{%
    \ifx\@partlist\@empty
      \gdef\@partlist{{#1}}%
    \else
      \g@addto@macro\@partlist{,{#1}}%
    \fi
  }%
  \SInc@org@include{#1} \relax
}
%    \end{macrocode}
%    \end{macro}
%
%    \begin{macro}{\includeonly}
%    Macro \cs{includeonly} is ignored.
%    \begin{macrocode}
\renewcommand*{\includeonly}[1]{%
  \PackageInfo{stampinclude}{%
    Ignoring \string\includeonly
  }%
}
%    \end{macrocode}
%    \end{macro}
%
%    Simulate \cs{includeonly}.
%    \begin{macrocode}
\@partswtrue
\gdef\@partlist{}
%    \end{macrocode}
%
%    Print included files at end of document.
%    \begin{macrocode}
\AtEndDocument{%
  \begingroup
    \expandafter\let\expandafter\@partlist\expandafter\@empty
    \expandafter\@for\expandafter\reserved@a
    \expandafter:\expandafter=\@partlist\do{%
      \ifx\@partlist\@empty
        \edef\@partlist{\reserved@a}%
      \else
        \edef\@partlist{\@partlist, \reserved@a}%
      \fi
    }%
    \typeout{********************%
             ********************%
             ********************%
             ******************%
    }%
    \ifx\@partlist\@empty
      \typeout{[stampinclude] No included files.}%
    \else
      \typeout{[stampinclude] Included files:}%
      \typeout{\@partlist}%
    \fi
    \typeout{********************%
             ********************%
             ********************%
             ******************%
    }%
  \endgroup
}
%    \end{macrocode}
%
%    \begin{macrocode}
%</package>
%    \end{macrocode}
%
% \section{Installation}
%
% \subsection{Download}
%
% \paragraph{Package.} This package is available on
% CTAN\footnote{\url{http://ctan.org/pkg/stampinclude}}:
% \begin{description}
% \item[\CTAN{macros/latex/contrib/oberdiek/stampinclude.dtx}] The source file.
% \item[\CTAN{macros/latex/contrib/oberdiek/stampinclude.pdf}] Documentation.
% \end{description}
%
%
% \paragraph{Bundle.} All the packages of the bundle `oberdiek'
% are also available in a TDS compliant ZIP archive. There
% the packages are already unpacked and the documentation files
% are generated. The files and directories obey the TDS standard.
% \begin{description}
% \item[\CTAN{install/macros/latex/contrib/oberdiek.tds.zip}]
% \end{description}
% \emph{TDS} refers to the standard ``A Directory Structure
% for \TeX\ Files'' (\CTAN{tds/tds.pdf}). Directories
% with \xfile{texmf} in their name are usually organized this way.
%
% \subsection{Bundle installation}
%
% \paragraph{Unpacking.} Unpack the \xfile{oberdiek.tds.zip} in the
% TDS tree (also known as \xfile{texmf} tree) of your choice.
% Example (linux):
% \begin{quote}
%   |unzip oberdiek.tds.zip -d ~/texmf|
% \end{quote}
%
% \paragraph{Script installation.}
% Check the directory \xfile{TDS:scripts/oberdiek/} for
% scripts that need further installation steps.
% Package \xpackage{attachfile2} comes with the Perl script
% \xfile{pdfatfi.pl} that should be installed in such a way
% that it can be called as \texttt{pdfatfi}.
% Example (linux):
% \begin{quote}
%   |chmod +x scripts/oberdiek/pdfatfi.pl|\\
%   |cp scripts/oberdiek/pdfatfi.pl /usr/local/bin/|
% \end{quote}
%
% \subsection{Package installation}
%
% \paragraph{Unpacking.} The \xfile{.dtx} file is a self-extracting
% \docstrip\ archive. The files are extracted by running the
% \xfile{.dtx} through \plainTeX:
% \begin{quote}
%   \verb|tex stampinclude.dtx|
% \end{quote}
%
% \paragraph{TDS.} Now the different files must be moved into
% the different directories in your installation TDS tree
% (also known as \xfile{texmf} tree):
% \begin{quote}
% \def\t{^^A
% \begin{tabular}{@{}>{\ttfamily}l@{ $\rightarrow$ }>{\ttfamily}l@{}}
%   stampinclude.sty & tex/latex/oberdiek/stampinclude.sty\\
%   stampinclude.pdf & doc/latex/oberdiek/stampinclude.pdf\\
%   stampinclude.dtx & source/latex/oberdiek/stampinclude.dtx\\
% \end{tabular}^^A
% }^^A
% \sbox0{\t}^^A
% \ifdim\wd0>\linewidth
%   \begingroup
%     \advance\linewidth by\leftmargin
%     \advance\linewidth by\rightmargin
%   \edef\x{\endgroup
%     \def\noexpand\lw{\the\linewidth}^^A
%   }\x
%   \def\lwbox{^^A
%     \leavevmode
%     \hbox to \linewidth{^^A
%       \kern-\leftmargin\relax
%       \hss
%       \usebox0
%       \hss
%       \kern-\rightmargin\relax
%     }^^A
%   }^^A
%   \ifdim\wd0>\lw
%     \sbox0{\small\t}^^A
%     \ifdim\wd0>\linewidth
%       \ifdim\wd0>\lw
%         \sbox0{\footnotesize\t}^^A
%         \ifdim\wd0>\linewidth
%           \ifdim\wd0>\lw
%             \sbox0{\scriptsize\t}^^A
%             \ifdim\wd0>\linewidth
%               \ifdim\wd0>\lw
%                 \sbox0{\tiny\t}^^A
%                 \ifdim\wd0>\linewidth
%                   \lwbox
%                 \else
%                   \usebox0
%                 \fi
%               \else
%                 \lwbox
%               \fi
%             \else
%               \usebox0
%             \fi
%           \else
%             \lwbox
%           \fi
%         \else
%           \usebox0
%         \fi
%       \else
%         \lwbox
%       \fi
%     \else
%       \usebox0
%     \fi
%   \else
%     \lwbox
%   \fi
% \else
%   \usebox0
% \fi
% \end{quote}
% If you have a \xfile{docstrip.cfg} that configures and enables \docstrip's
% TDS installing feature, then some files can already be in the right
% place, see the documentation of \docstrip.
%
% \subsection{Refresh file name databases}
%
% If your \TeX~distribution
% (\teTeX, \mikTeX, \dots) relies on file name databases, you must refresh
% these. For example, \teTeX\ users run \verb|texhash| or
% \verb|mktexlsr|.
%
% \subsection{Some details for the interested}
%
% \paragraph{Attached source.}
%
% The PDF documentation on CTAN also includes the
% \xfile{.dtx} source file. It can be extracted by
% AcrobatReader 6 or higher. Another option is \textsf{pdftk},
% e.g. unpack the file into the current directory:
% \begin{quote}
%   \verb|pdftk stampinclude.pdf unpack_files output .|
% \end{quote}
%
% \paragraph{Unpacking with \LaTeX.}
% The \xfile{.dtx} chooses its action depending on the format:
% \begin{description}
% \item[\plainTeX:] Run \docstrip\ and extract the files.
% \item[\LaTeX:] Generate the documentation.
% \end{description}
% If you insist on using \LaTeX\ for \docstrip\ (really,
% \docstrip\ does not need \LaTeX), then inform the autodetect routine
% about your intention:
% \begin{quote}
%   \verb|latex \let\install=y\input{stampinclude.dtx}|
% \end{quote}
% Do not forget to quote the argument according to the demands
% of your shell.
%
% \paragraph{Generating the documentation.}
% You can use both the \xfile{.dtx} or the \xfile{.drv} to generate
% the documentation. The process can be configured by the
% configuration file \xfile{ltxdoc.cfg}. For instance, put this
% line into this file, if you want to have A4 as paper format:
% \begin{quote}
%   \verb|\PassOptionsToClass{a4paper}{article}|
% \end{quote}
% An example follows how to generate the
% documentation with pdf\LaTeX:
% \begin{quote}
%\begin{verbatim}
%pdflatex stampinclude.dtx
%makeindex -s gind.ist stampinclude.idx
%pdflatex stampinclude.dtx
%makeindex -s gind.ist stampinclude.idx
%pdflatex stampinclude.dtx
%\end{verbatim}
% \end{quote}
%
% \section{Catalogue}
%
% The following XML file can be used as source for the
% \href{http://mirror.ctan.org/help/Catalogue/catalogue.html}{\TeX\ Catalogue}.
% The elements \texttt{caption} and \texttt{description} are imported
% from the original XML file from the Catalogue.
% The name of the XML file in the Catalogue is \xfile{stampinclude.xml}.
%    \begin{macrocode}
%<*catalogue>
<?xml version='1.0' encoding='us-ascii'?>
<!DOCTYPE entry SYSTEM 'catalogue.dtd'>
<entry datestamp='$Date$' modifier='$Author$' id='stampinclude'>
  <name>stampinclude</name>
  <caption>Inclusion based on .aux file date stamps.</caption>
  <authorref id='auth:oberdiek'/>
  <copyright owner='Heiko Oberdiek' year='2008'/>
  <license type='lppl1.3'/>
  <version number='1.1'/>
  <description>
    This package replaces <tt>\includeonly</tt> and selects the files for
    <tt>\include</tt> by inspecting the timestamp of the <tt>.aux</tt> file.
    The file is selected for inclusion if the <tt>.aux</tt> file does
    not yet exist or is older than the corresponding <tt>.tex</tt> file.
    <p/>
    The package is part of the <xref refid='oberdiek'>oberdiek</xref>
    bundle.
  </description>
  <documentation details='Package documentation'
      href='ctan:/macros/latex/contrib/oberdiek/stampinclude.pdf'/>
  <ctan file='true' path='/macros/latex/contrib/oberdiek/stampinclude.dtx'/>
  <miktex location='oberdiek'/>
  <texlive location='oberdiek'/>
  <install path='/macros/latex/contrib/oberdiek/oberdiek.tds.zip'/>
</entry>
%</catalogue>
%    \end{macrocode}
%
% \begin{thebibliography}{9}
% \bibitem{askinclude}
%   Pablo A. Straub, Heiko Oberdiek:
%   \textit{The \xpackage{askinclude} package};
%   2007/10/23 v2.0;
%   \CTAN{macros/latex/contrib/oberdiek/askinclude.pdf}.
%
% \bibitem{pdftexcmds}
%   Heiko Oberdiek:
%   \textit{The \xpackage{pdftexcmds} package};
%   2007/12/12 v0.3;
%   \CTAN{macros/latex/contrib/oberdiek/pdftexcmds.pdf}.
%
% \end{thebibliography}
%
% \begin{History}
%   \begin{Version}{2008/07/14 v1.0}
%   \item
%     First version.
%   \end{Version}
%   \begin{Version}{2016/05/16 v1.1}
%   \item
%     Documentation updates.
%   \end{Version}
% \end{History}
%
% \PrintIndex
%
% \Finale
\endinput

%        (quote the arguments according to the demands of your shell)
%
% Documentation:
%    (a) If stampinclude.drv is present:
%           latex stampinclude.drv
%    (b) Without stampinclude.drv:
%           latex stampinclude.dtx; ...
%    The class ltxdoc loads the configuration file ltxdoc.cfg
%    if available. Here you can specify further options, e.g.
%    use A4 as paper format:
%       \PassOptionsToClass{a4paper}{article}
%
%    Programm calls to get the documentation (example):
%       pdflatex stampinclude.dtx
%       makeindex -s gind.ist stampinclude.idx
%       pdflatex stampinclude.dtx
%       makeindex -s gind.ist stampinclude.idx
%       pdflatex stampinclude.dtx
%
% Installation:
%    TDS:tex/latex/oberdiek/stampinclude.sty
%    TDS:doc/latex/oberdiek/stampinclude.pdf
%    TDS:source/latex/oberdiek/stampinclude.dtx
%
%<*ignore>
\begingroup
  \catcode123=1 %
  \catcode125=2 %
  \def\x{LaTeX2e}%
\expandafter\endgroup
\ifcase 0\ifx\install y1\fi\expandafter
         \ifx\csname processbatchFile\endcsname\relax\else1\fi
         \ifx\fmtname\x\else 1\fi\relax
\else\csname fi\endcsname
%</ignore>
%<*install>
\input docstrip.tex
\Msg{************************************************************************}
\Msg{* Installation}
\Msg{* Package: stampinclude 2016/05/16 v1.1 Include files based on time stamps (HO)}
\Msg{************************************************************************}

\keepsilent
\askforoverwritefalse

\let\MetaPrefix\relax
\preamble

This is a generated file.

Project: stampinclude
Version: 2016/05/16 v1.1

Copyright (C) 2008 by
   Heiko Oberdiek <heiko.oberdiek at googlemail.com>

This work may be distributed and/or modified under the
conditions of the LaTeX Project Public License, either
version 1.3c of this license or (at your option) any later
version. This version of this license is in
   http://www.latex-project.org/lppl/lppl-1-3c.txt
and the latest version of this license is in
   http://www.latex-project.org/lppl.txt
and version 1.3 or later is part of all distributions of
LaTeX version 2005/12/01 or later.

This work has the LPPL maintenance status "maintained".

This Current Maintainer of this work is Heiko Oberdiek.

This work consists of the main source file stampinclude.dtx
and the derived files
   stampinclude.sty, stampinclude.pdf, stampinclude.ins, stampinclude.drv.

\endpreamble
\let\MetaPrefix\DoubleperCent

\generate{%
  \file{stampinclude.ins}{\from{stampinclude.dtx}{install}}%
  \file{stampinclude.drv}{\from{stampinclude.dtx}{driver}}%
  \usedir{tex/latex/oberdiek}%
  \file{stampinclude.sty}{\from{stampinclude.dtx}{package}}%
  \nopreamble
  \nopostamble
  \usedir{source/latex/oberdiek/catalogue}%
  \file{stampinclude.xml}{\from{stampinclude.dtx}{catalogue}}%
}

\catcode32=13\relax% active space
\let =\space%
\Msg{************************************************************************}
\Msg{*}
\Msg{* To finish the installation you have to move the following}
\Msg{* file into a directory searched by TeX:}
\Msg{*}
\Msg{*     stampinclude.sty}
\Msg{*}
\Msg{* To produce the documentation run the file `stampinclude.drv'}
\Msg{* through LaTeX.}
\Msg{*}
\Msg{* Happy TeXing!}
\Msg{*}
\Msg{************************************************************************}

\endbatchfile
%</install>
%<*ignore>
\fi
%</ignore>
%<*driver>
\NeedsTeXFormat{LaTeX2e}
\ProvidesFile{stampinclude.drv}%
  [2016/05/16 v1.1 Include files based on time stamps (HO)]%
\documentclass{ltxdoc}
\usepackage{holtxdoc}[2011/11/22]
\begin{document}
  \DocInput{stampinclude.dtx}%
\end{document}
%</driver>
% \fi
%
%
% \CharacterTable
%  {Upper-case    \A\B\C\D\E\F\G\H\I\J\K\L\M\N\O\P\Q\R\S\T\U\V\W\X\Y\Z
%   Lower-case    \a\b\c\d\e\f\g\h\i\j\k\l\m\n\o\p\q\r\s\t\u\v\w\x\y\z
%   Digits        \0\1\2\3\4\5\6\7\8\9
%   Exclamation   \!     Double quote  \"     Hash (number) \#
%   Dollar        \$     Percent       \%     Ampersand     \&
%   Acute accent  \'     Left paren    \(     Right paren   \)
%   Asterisk      \*     Plus          \+     Comma         \,
%   Minus         \-     Point         \.     Solidus       \/
%   Colon         \:     Semicolon     \;     Less than     \<
%   Equals        \=     Greater than  \>     Question mark \?
%   Commercial at \@     Left bracket  \[     Backslash     \\
%   Right bracket \]     Circumflex    \^     Underscore    \_
%   Grave accent  \`     Left brace    \{     Vertical bar  \|
%   Right brace   \}     Tilde         \~}
%
% \GetFileInfo{stampinclude.drv}
%
% \title{The \xpackage{stampinclude} package}
% \date{2016/05/16 v1.1}
% \author{Heiko Oberdiek\thanks
% {Please report any issues at https://github.com/ho-tex/oberdiek/issues}\\
% \xemail{heiko.oberdiek at googlemail.com}}
%
% \maketitle
%
% \begin{abstract}
% The package replaces \cs{includeonly} and selects the files for
% \cs{include} by inspecting the time stamp of the \xext{aux} file.
% The file is selected for inclusion if the \xext{aux} file does
% not yet exist or is older than the corresponding \xext{tex} file.
% \end{abstract}
%
% \tableofcontents
%
% \section{Documentation}
%
% \subsection{Introduction}
% \label{sec:intro}
%
% \LaTeX\ provides two commands \cs{include} and \cs{includeonly}
% that helps in organizing large projects.
% Example for a master file:
%\begin{quote}
%\begin{verbatim}
%\documentclass{book}
%  % \includeonly{}
%\begin{document}
% \include{fileA}
% \include{fileB}
% \include{fileC}
%\end{document}
%\end{verbatim}
%\end{quote}
% All files are read and compiled if \cs{includeonly} is not
% executed. Otherwise you can give \cs{includeonly} a list
% of files in the preamble, e.g.:
% \begin{quote}
%   |\includeonly{fileA,fileC}|
% \end{quote}
% Now only files \xfile{fileA.tex} and \xfile{fileC.tex} are read
% and compiled.
%
% If you change file \xfile{fileB.tex} and want to see only this
% file, then you must change the line with \cs{includeonly} to
% \begin{quote}
%   |\includeonly{fileB}|
% \end{quote}
% It is tedious to do this again and again, if different files
% are changed.
%
% Package \xpackage{askinclude} \cite{askinclude}
% offers a solution for this problem. It interactively asks
% for the files to be included and saves the user from
% editing the master file.
%
% This package \xpackage{stampinclude} goes another way.
% \LaTeX\ reads and writes a separate \xext{aux} file for each
% file that is included by \cs{include}. There \LaTeX\ remembers
% counter valuses. Changed \xext{tex}
% files can therefore be detected by comparing the file date stamp of
% the \xext{tex} file with the date stamp of its \xext{aux} file.
% Since version 1.30.0 \pdfTeX\ provides \cs{pdffilemoddate}
% that reads the file date stamp. Thus this package uses this
% command and redefines
% \cs{include} to include the files that do not have \xext{aux}
% files yet or that are newer than its \xext{aux} file.
% \cs{includeonly} is ignored.
%
% \subsection{Usage}
%
% The package is loaded as normal \LaTeX\ package without options:
% \begin{quote}
%   |\usepackage{stampinclude}|
% \end{quote}
% Alternatively the package may be loaded on the command line
% (Example for shell `bash'):
% \begin{center}
%   |latex '\AtBeginDocument{\usepackage{stampinclude}}\input{master}'|
% \end{center}
% Without \cs{AtBeginDocument} (and \cs{RequirePackage} instead of
% \cs{usepackage}) \TeX\ would name the document \xfile{stampinclude.dvi}
% instead of \xfile{master.dvi}.
%
% \subsection{Limitations}
%
% \subsubsection{Other file dependencies}
%
% A file that is included by \cs{include} may input ore reference
% other files:
% \begin{itemize}
% \item other \TeX\ files using \cs{input},
% \item graphics files (\cs{includegraphics}),
% \item listings of external files,
% \item ...
% \end{itemize}
% Updates of those files are not detected by this package.
% It limits the date stamp comparison of an \xext{aux} file
% to its \xext{tex} file.
%
% \subsubsection{\cs{include} dependencies}
%
% In the example, given in the introduction \ref{sec:intro},
% three files \xfile{fileA}, \xfile{fileB}, and \xfile{fileC}
% are included in this order. Now file \xfile{fileA} is changed by adding
% four pages, \xfile{fileB} remains untouched, and \xfile{fileC} is
% also updated. Then the package only selects \xfile{fileA} and
% \xfile{fileC} for inclusion. File \xfile{fileB} is not included.
% But \LaTeX\ has stored the counter values that are active
% at the end of \xfile{fileB} in \xfile{fileB.aux} in one of the
% previous runs when \xfile{fileB} was included.
% However the later addition of four pages in \xfile{fileA}
% was not known at that time. Therefore \xfile{fileB.aux}
% is out of date and the inclusion of file \xfile{fileC}
% starts with wrong counter values (especially the page counter).
%
% \subsubsection{Summary}
%
% This package \xpackage{stampinclude} and the \cs{include} feature
% helps in accelerating the \LaTeX\ compilation.
% But it is not intended for generating the final version.
% For the final version of the document it is better to include
% \emph{all} files to get all counter values right.
% Then this package and any \cs{includeonly} lines should be commented out:
%\begin{quote}
%  |% \usepackage{stampinclude}|\\
%  |% \includeonly{...}|
%\end{quote}
%
% \subsection{Requirements}
%
% \begin{itemize}
% \item \pdfTeX\ v1.30.0 (because of \cs{pdffilemoddate}
%   and \cs{pdfstrcmp}),\\
%   both modes for DVI and PDF are supported.
% \item Alternatively Lua\TeX\ may be used.
%   It lacks \cs{pdffilemoddate} and \cs{pdfstrcmp}. But its services
%   are provided by package \xpackage{pdftexcmds} \cite{pdftexcmds}
%   that is automatically loaded.
% \end{itemize}
%
% \StopEventually{
% }
%
% \section{Implementation}
%
%    \begin{macrocode}
%<*package>
\NeedsTeXFormat{LaTeX2e}
\ProvidesPackage{stampinclude}
  [2016/05/16 v1.1 Include files based on time stamps (HO)]%
%    \end{macrocode}
%
%    \begin{macrocode}
\RequirePackage{pdftexcmds}[2007/12/12]%
%    \end{macrocode}
%
%    \begin{macrocode}
\begingroup
  \chardef\x=1 %
  \expandafter\ifx\csname pdf@filemoddate\endcsname\relax
    \chardef\x=0 %
  \fi
  \expandafter\ifx\csname pdf@strcmp\endcsname\relax
    \chardef\x=0 %
  \fi
\expandafter\endgroup\ifcase\x
  \PackageWarningNoLine{stampinclude}{%
    \string\pdffilemoddate\space or %
    \string\pdfstrcmp\space are not found,\MessageBreak
    that are provided by pdfTeX >= 1.30.0.\MessageBreak
    Also LuaTeX is not detected.\MessageBreak
    Therefore package loading is aborted%
  }%
  \expandafter\endinput
\fi
%    \end{macrocode}
%
%    \begin{macro}{\SInc@org@include}
%    \begin{macrocode}
\let\SInc@org@include\@include
%    \end{macrocode}
%    \end{macro}
%    \begin{macro}{\@include}
%    \begin{macrocode}
\def\@include#1 {%
  \IfFileExists{#1.aux}{%
    \ifnum\pdf@strcmp{\pdf@filemoddate{#1.aux}}%
                     {\pdf@filemoddate{#1.tex}}<0 %
      \ifx\@partlist\@empty
        \gdef\@partlist{{#1}}%
      \else
        \g@addto@macro\@partlist{,{#1}}%
      \fi
    \fi
  }{%
    \ifx\@partlist\@empty
      \gdef\@partlist{{#1}}%
    \else
      \g@addto@macro\@partlist{,{#1}}%
    \fi
  }%
  \SInc@org@include{#1} \relax
}
%    \end{macrocode}
%    \end{macro}
%
%    \begin{macro}{\includeonly}
%    Macro \cs{includeonly} is ignored.
%    \begin{macrocode}
\renewcommand*{\includeonly}[1]{%
  \PackageInfo{stampinclude}{%
    Ignoring \string\includeonly
  }%
}
%    \end{macrocode}
%    \end{macro}
%
%    Simulate \cs{includeonly}.
%    \begin{macrocode}
\@partswtrue
\gdef\@partlist{}
%    \end{macrocode}
%
%    Print included files at end of document.
%    \begin{macrocode}
\AtEndDocument{%
  \begingroup
    \expandafter\let\expandafter\@partlist\expandafter\@empty
    \expandafter\@for\expandafter\reserved@a
    \expandafter:\expandafter=\@partlist\do{%
      \ifx\@partlist\@empty
        \edef\@partlist{\reserved@a}%
      \else
        \edef\@partlist{\@partlist, \reserved@a}%
      \fi
    }%
    \typeout{********************%
             ********************%
             ********************%
             ******************%
    }%
    \ifx\@partlist\@empty
      \typeout{[stampinclude] No included files.}%
    \else
      \typeout{[stampinclude] Included files:}%
      \typeout{\@partlist}%
    \fi
    \typeout{********************%
             ********************%
             ********************%
             ******************%
    }%
  \endgroup
}
%    \end{macrocode}
%
%    \begin{macrocode}
%</package>
%    \end{macrocode}
%
% \section{Installation}
%
% \subsection{Download}
%
% \paragraph{Package.} This package is available on
% CTAN\footnote{\url{http://ctan.org/pkg/stampinclude}}:
% \begin{description}
% \item[\CTAN{macros/latex/contrib/oberdiek/stampinclude.dtx}] The source file.
% \item[\CTAN{macros/latex/contrib/oberdiek/stampinclude.pdf}] Documentation.
% \end{description}
%
%
% \paragraph{Bundle.} All the packages of the bundle `oberdiek'
% are also available in a TDS compliant ZIP archive. There
% the packages are already unpacked and the documentation files
% are generated. The files and directories obey the TDS standard.
% \begin{description}
% \item[\CTAN{install/macros/latex/contrib/oberdiek.tds.zip}]
% \end{description}
% \emph{TDS} refers to the standard ``A Directory Structure
% for \TeX\ Files'' (\CTAN{tds/tds.pdf}). Directories
% with \xfile{texmf} in their name are usually organized this way.
%
% \subsection{Bundle installation}
%
% \paragraph{Unpacking.} Unpack the \xfile{oberdiek.tds.zip} in the
% TDS tree (also known as \xfile{texmf} tree) of your choice.
% Example (linux):
% \begin{quote}
%   |unzip oberdiek.tds.zip -d ~/texmf|
% \end{quote}
%
% \paragraph{Script installation.}
% Check the directory \xfile{TDS:scripts/oberdiek/} for
% scripts that need further installation steps.
% Package \xpackage{attachfile2} comes with the Perl script
% \xfile{pdfatfi.pl} that should be installed in such a way
% that it can be called as \texttt{pdfatfi}.
% Example (linux):
% \begin{quote}
%   |chmod +x scripts/oberdiek/pdfatfi.pl|\\
%   |cp scripts/oberdiek/pdfatfi.pl /usr/local/bin/|
% \end{quote}
%
% \subsection{Package installation}
%
% \paragraph{Unpacking.} The \xfile{.dtx} file is a self-extracting
% \docstrip\ archive. The files are extracted by running the
% \xfile{.dtx} through \plainTeX:
% \begin{quote}
%   \verb|tex stampinclude.dtx|
% \end{quote}
%
% \paragraph{TDS.} Now the different files must be moved into
% the different directories in your installation TDS tree
% (also known as \xfile{texmf} tree):
% \begin{quote}
% \def\t{^^A
% \begin{tabular}{@{}>{\ttfamily}l@{ $\rightarrow$ }>{\ttfamily}l@{}}
%   stampinclude.sty & tex/latex/oberdiek/stampinclude.sty\\
%   stampinclude.pdf & doc/latex/oberdiek/stampinclude.pdf\\
%   stampinclude.dtx & source/latex/oberdiek/stampinclude.dtx\\
% \end{tabular}^^A
% }^^A
% \sbox0{\t}^^A
% \ifdim\wd0>\linewidth
%   \begingroup
%     \advance\linewidth by\leftmargin
%     \advance\linewidth by\rightmargin
%   \edef\x{\endgroup
%     \def\noexpand\lw{\the\linewidth}^^A
%   }\x
%   \def\lwbox{^^A
%     \leavevmode
%     \hbox to \linewidth{^^A
%       \kern-\leftmargin\relax
%       \hss
%       \usebox0
%       \hss
%       \kern-\rightmargin\relax
%     }^^A
%   }^^A
%   \ifdim\wd0>\lw
%     \sbox0{\small\t}^^A
%     \ifdim\wd0>\linewidth
%       \ifdim\wd0>\lw
%         \sbox0{\footnotesize\t}^^A
%         \ifdim\wd0>\linewidth
%           \ifdim\wd0>\lw
%             \sbox0{\scriptsize\t}^^A
%             \ifdim\wd0>\linewidth
%               \ifdim\wd0>\lw
%                 \sbox0{\tiny\t}^^A
%                 \ifdim\wd0>\linewidth
%                   \lwbox
%                 \else
%                   \usebox0
%                 \fi
%               \else
%                 \lwbox
%               \fi
%             \else
%               \usebox0
%             \fi
%           \else
%             \lwbox
%           \fi
%         \else
%           \usebox0
%         \fi
%       \else
%         \lwbox
%       \fi
%     \else
%       \usebox0
%     \fi
%   \else
%     \lwbox
%   \fi
% \else
%   \usebox0
% \fi
% \end{quote}
% If you have a \xfile{docstrip.cfg} that configures and enables \docstrip's
% TDS installing feature, then some files can already be in the right
% place, see the documentation of \docstrip.
%
% \subsection{Refresh file name databases}
%
% If your \TeX~distribution
% (\teTeX, \mikTeX, \dots) relies on file name databases, you must refresh
% these. For example, \teTeX\ users run \verb|texhash| or
% \verb|mktexlsr|.
%
% \subsection{Some details for the interested}
%
% \paragraph{Attached source.}
%
% The PDF documentation on CTAN also includes the
% \xfile{.dtx} source file. It can be extracted by
% AcrobatReader 6 or higher. Another option is \textsf{pdftk},
% e.g. unpack the file into the current directory:
% \begin{quote}
%   \verb|pdftk stampinclude.pdf unpack_files output .|
% \end{quote}
%
% \paragraph{Unpacking with \LaTeX.}
% The \xfile{.dtx} chooses its action depending on the format:
% \begin{description}
% \item[\plainTeX:] Run \docstrip\ and extract the files.
% \item[\LaTeX:] Generate the documentation.
% \end{description}
% If you insist on using \LaTeX\ for \docstrip\ (really,
% \docstrip\ does not need \LaTeX), then inform the autodetect routine
% about your intention:
% \begin{quote}
%   \verb|latex \let\install=y% \iffalse meta-comment
%
% File: stampinclude.dtx
% Version: 2016/05/16 v1.1
% Info: Include files based on time stamps
%
% Copyright (C) 2008 by
%    Heiko Oberdiek <heiko.oberdiek at googlemail.com>
%    2016
%    https://github.com/ho-tex/oberdiek/issues
%
% This work may be distributed and/or modified under the
% conditions of the LaTeX Project Public License, either
% version 1.3c of this license or (at your option) any later
% version. This version of this license is in
%    http://www.latex-project.org/lppl/lppl-1-3c.txt
% and the latest version of this license is in
%    http://www.latex-project.org/lppl.txt
% and version 1.3 or later is part of all distributions of
% LaTeX version 2005/12/01 or later.
%
% This work has the LPPL maintenance status "maintained".
%
% This Current Maintainer of this work is Heiko Oberdiek.
%
% This work consists of the main source file stampinclude.dtx
% and the derived files
%    stampinclude.sty, stampinclude.pdf, stampinclude.ins, stampinclude.drv.
%
% Distribution:
%    CTAN:macros/latex/contrib/oberdiek/stampinclude.dtx
%    CTAN:macros/latex/contrib/oberdiek/stampinclude.pdf
%
% Unpacking:
%    (a) If stampinclude.ins is present:
%           tex stampinclude.ins
%    (b) Without stampinclude.ins:
%           tex stampinclude.dtx
%    (c) If you insist on using LaTeX
%           latex \let\install=y\input{stampinclude.dtx}
%        (quote the arguments according to the demands of your shell)
%
% Documentation:
%    (a) If stampinclude.drv is present:
%           latex stampinclude.drv
%    (b) Without stampinclude.drv:
%           latex stampinclude.dtx; ...
%    The class ltxdoc loads the configuration file ltxdoc.cfg
%    if available. Here you can specify further options, e.g.
%    use A4 as paper format:
%       \PassOptionsToClass{a4paper}{article}
%
%    Programm calls to get the documentation (example):
%       pdflatex stampinclude.dtx
%       makeindex -s gind.ist stampinclude.idx
%       pdflatex stampinclude.dtx
%       makeindex -s gind.ist stampinclude.idx
%       pdflatex stampinclude.dtx
%
% Installation:
%    TDS:tex/latex/oberdiek/stampinclude.sty
%    TDS:doc/latex/oberdiek/stampinclude.pdf
%    TDS:source/latex/oberdiek/stampinclude.dtx
%
%<*ignore>
\begingroup
  \catcode123=1 %
  \catcode125=2 %
  \def\x{LaTeX2e}%
\expandafter\endgroup
\ifcase 0\ifx\install y1\fi\expandafter
         \ifx\csname processbatchFile\endcsname\relax\else1\fi
         \ifx\fmtname\x\else 1\fi\relax
\else\csname fi\endcsname
%</ignore>
%<*install>
\input docstrip.tex
\Msg{************************************************************************}
\Msg{* Installation}
\Msg{* Package: stampinclude 2016/05/16 v1.1 Include files based on time stamps (HO)}
\Msg{************************************************************************}

\keepsilent
\askforoverwritefalse

\let\MetaPrefix\relax
\preamble

This is a generated file.

Project: stampinclude
Version: 2016/05/16 v1.1

Copyright (C) 2008 by
   Heiko Oberdiek <heiko.oberdiek at googlemail.com>

This work may be distributed and/or modified under the
conditions of the LaTeX Project Public License, either
version 1.3c of this license or (at your option) any later
version. This version of this license is in
   http://www.latex-project.org/lppl/lppl-1-3c.txt
and the latest version of this license is in
   http://www.latex-project.org/lppl.txt
and version 1.3 or later is part of all distributions of
LaTeX version 2005/12/01 or later.

This work has the LPPL maintenance status "maintained".

This Current Maintainer of this work is Heiko Oberdiek.

This work consists of the main source file stampinclude.dtx
and the derived files
   stampinclude.sty, stampinclude.pdf, stampinclude.ins, stampinclude.drv.

\endpreamble
\let\MetaPrefix\DoubleperCent

\generate{%
  \file{stampinclude.ins}{\from{stampinclude.dtx}{install}}%
  \file{stampinclude.drv}{\from{stampinclude.dtx}{driver}}%
  \usedir{tex/latex/oberdiek}%
  \file{stampinclude.sty}{\from{stampinclude.dtx}{package}}%
  \nopreamble
  \nopostamble
  \usedir{source/latex/oberdiek/catalogue}%
  \file{stampinclude.xml}{\from{stampinclude.dtx}{catalogue}}%
}

\catcode32=13\relax% active space
\let =\space%
\Msg{************************************************************************}
\Msg{*}
\Msg{* To finish the installation you have to move the following}
\Msg{* file into a directory searched by TeX:}
\Msg{*}
\Msg{*     stampinclude.sty}
\Msg{*}
\Msg{* To produce the documentation run the file `stampinclude.drv'}
\Msg{* through LaTeX.}
\Msg{*}
\Msg{* Happy TeXing!}
\Msg{*}
\Msg{************************************************************************}

\endbatchfile
%</install>
%<*ignore>
\fi
%</ignore>
%<*driver>
\NeedsTeXFormat{LaTeX2e}
\ProvidesFile{stampinclude.drv}%
  [2016/05/16 v1.1 Include files based on time stamps (HO)]%
\documentclass{ltxdoc}
\usepackage{holtxdoc}[2011/11/22]
\begin{document}
  \DocInput{stampinclude.dtx}%
\end{document}
%</driver>
% \fi
%
%
% \CharacterTable
%  {Upper-case    \A\B\C\D\E\F\G\H\I\J\K\L\M\N\O\P\Q\R\S\T\U\V\W\X\Y\Z
%   Lower-case    \a\b\c\d\e\f\g\h\i\j\k\l\m\n\o\p\q\r\s\t\u\v\w\x\y\z
%   Digits        \0\1\2\3\4\5\6\7\8\9
%   Exclamation   \!     Double quote  \"     Hash (number) \#
%   Dollar        \$     Percent       \%     Ampersand     \&
%   Acute accent  \'     Left paren    \(     Right paren   \)
%   Asterisk      \*     Plus          \+     Comma         \,
%   Minus         \-     Point         \.     Solidus       \/
%   Colon         \:     Semicolon     \;     Less than     \<
%   Equals        \=     Greater than  \>     Question mark \?
%   Commercial at \@     Left bracket  \[     Backslash     \\
%   Right bracket \]     Circumflex    \^     Underscore    \_
%   Grave accent  \`     Left brace    \{     Vertical bar  \|
%   Right brace   \}     Tilde         \~}
%
% \GetFileInfo{stampinclude.drv}
%
% \title{The \xpackage{stampinclude} package}
% \date{2016/05/16 v1.1}
% \author{Heiko Oberdiek\thanks
% {Please report any issues at https://github.com/ho-tex/oberdiek/issues}\\
% \xemail{heiko.oberdiek at googlemail.com}}
%
% \maketitle
%
% \begin{abstract}
% The package replaces \cs{includeonly} and selects the files for
% \cs{include} by inspecting the time stamp of the \xext{aux} file.
% The file is selected for inclusion if the \xext{aux} file does
% not yet exist or is older than the corresponding \xext{tex} file.
% \end{abstract}
%
% \tableofcontents
%
% \section{Documentation}
%
% \subsection{Introduction}
% \label{sec:intro}
%
% \LaTeX\ provides two commands \cs{include} and \cs{includeonly}
% that helps in organizing large projects.
% Example for a master file:
%\begin{quote}
%\begin{verbatim}
%\documentclass{book}
%  % \includeonly{}
%\begin{document}
% \include{fileA}
% \include{fileB}
% \include{fileC}
%\end{document}
%\end{verbatim}
%\end{quote}
% All files are read and compiled if \cs{includeonly} is not
% executed. Otherwise you can give \cs{includeonly} a list
% of files in the preamble, e.g.:
% \begin{quote}
%   |\includeonly{fileA,fileC}|
% \end{quote}
% Now only files \xfile{fileA.tex} and \xfile{fileC.tex} are read
% and compiled.
%
% If you change file \xfile{fileB.tex} and want to see only this
% file, then you must change the line with \cs{includeonly} to
% \begin{quote}
%   |\includeonly{fileB}|
% \end{quote}
% It is tedious to do this again and again, if different files
% are changed.
%
% Package \xpackage{askinclude} \cite{askinclude}
% offers a solution for this problem. It interactively asks
% for the files to be included and saves the user from
% editing the master file.
%
% This package \xpackage{stampinclude} goes another way.
% \LaTeX\ reads and writes a separate \xext{aux} file for each
% file that is included by \cs{include}. There \LaTeX\ remembers
% counter valuses. Changed \xext{tex}
% files can therefore be detected by comparing the file date stamp of
% the \xext{tex} file with the date stamp of its \xext{aux} file.
% Since version 1.30.0 \pdfTeX\ provides \cs{pdffilemoddate}
% that reads the file date stamp. Thus this package uses this
% command and redefines
% \cs{include} to include the files that do not have \xext{aux}
% files yet or that are newer than its \xext{aux} file.
% \cs{includeonly} is ignored.
%
% \subsection{Usage}
%
% The package is loaded as normal \LaTeX\ package without options:
% \begin{quote}
%   |\usepackage{stampinclude}|
% \end{quote}
% Alternatively the package may be loaded on the command line
% (Example for shell `bash'):
% \begin{center}
%   |latex '\AtBeginDocument{\usepackage{stampinclude}}\input{master}'|
% \end{center}
% Without \cs{AtBeginDocument} (and \cs{RequirePackage} instead of
% \cs{usepackage}) \TeX\ would name the document \xfile{stampinclude.dvi}
% instead of \xfile{master.dvi}.
%
% \subsection{Limitations}
%
% \subsubsection{Other file dependencies}
%
% A file that is included by \cs{include} may input ore reference
% other files:
% \begin{itemize}
% \item other \TeX\ files using \cs{input},
% \item graphics files (\cs{includegraphics}),
% \item listings of external files,
% \item ...
% \end{itemize}
% Updates of those files are not detected by this package.
% It limits the date stamp comparison of an \xext{aux} file
% to its \xext{tex} file.
%
% \subsubsection{\cs{include} dependencies}
%
% In the example, given in the introduction \ref{sec:intro},
% three files \xfile{fileA}, \xfile{fileB}, and \xfile{fileC}
% are included in this order. Now file \xfile{fileA} is changed by adding
% four pages, \xfile{fileB} remains untouched, and \xfile{fileC} is
% also updated. Then the package only selects \xfile{fileA} and
% \xfile{fileC} for inclusion. File \xfile{fileB} is not included.
% But \LaTeX\ has stored the counter values that are active
% at the end of \xfile{fileB} in \xfile{fileB.aux} in one of the
% previous runs when \xfile{fileB} was included.
% However the later addition of four pages in \xfile{fileA}
% was not known at that time. Therefore \xfile{fileB.aux}
% is out of date and the inclusion of file \xfile{fileC}
% starts with wrong counter values (especially the page counter).
%
% \subsubsection{Summary}
%
% This package \xpackage{stampinclude} and the \cs{include} feature
% helps in accelerating the \LaTeX\ compilation.
% But it is not intended for generating the final version.
% For the final version of the document it is better to include
% \emph{all} files to get all counter values right.
% Then this package and any \cs{includeonly} lines should be commented out:
%\begin{quote}
%  |% \usepackage{stampinclude}|\\
%  |% \includeonly{...}|
%\end{quote}
%
% \subsection{Requirements}
%
% \begin{itemize}
% \item \pdfTeX\ v1.30.0 (because of \cs{pdffilemoddate}
%   and \cs{pdfstrcmp}),\\
%   both modes for DVI and PDF are supported.
% \item Alternatively Lua\TeX\ may be used.
%   It lacks \cs{pdffilemoddate} and \cs{pdfstrcmp}. But its services
%   are provided by package \xpackage{pdftexcmds} \cite{pdftexcmds}
%   that is automatically loaded.
% \end{itemize}
%
% \StopEventually{
% }
%
% \section{Implementation}
%
%    \begin{macrocode}
%<*package>
\NeedsTeXFormat{LaTeX2e}
\ProvidesPackage{stampinclude}
  [2016/05/16 v1.1 Include files based on time stamps (HO)]%
%    \end{macrocode}
%
%    \begin{macrocode}
\RequirePackage{pdftexcmds}[2007/12/12]%
%    \end{macrocode}
%
%    \begin{macrocode}
\begingroup
  \chardef\x=1 %
  \expandafter\ifx\csname pdf@filemoddate\endcsname\relax
    \chardef\x=0 %
  \fi
  \expandafter\ifx\csname pdf@strcmp\endcsname\relax
    \chardef\x=0 %
  \fi
\expandafter\endgroup\ifcase\x
  \PackageWarningNoLine{stampinclude}{%
    \string\pdffilemoddate\space or %
    \string\pdfstrcmp\space are not found,\MessageBreak
    that are provided by pdfTeX >= 1.30.0.\MessageBreak
    Also LuaTeX is not detected.\MessageBreak
    Therefore package loading is aborted%
  }%
  \expandafter\endinput
\fi
%    \end{macrocode}
%
%    \begin{macro}{\SInc@org@include}
%    \begin{macrocode}
\let\SInc@org@include\@include
%    \end{macrocode}
%    \end{macro}
%    \begin{macro}{\@include}
%    \begin{macrocode}
\def\@include#1 {%
  \IfFileExists{#1.aux}{%
    \ifnum\pdf@strcmp{\pdf@filemoddate{#1.aux}}%
                     {\pdf@filemoddate{#1.tex}}<0 %
      \ifx\@partlist\@empty
        \gdef\@partlist{{#1}}%
      \else
        \g@addto@macro\@partlist{,{#1}}%
      \fi
    \fi
  }{%
    \ifx\@partlist\@empty
      \gdef\@partlist{{#1}}%
    \else
      \g@addto@macro\@partlist{,{#1}}%
    \fi
  }%
  \SInc@org@include{#1} \relax
}
%    \end{macrocode}
%    \end{macro}
%
%    \begin{macro}{\includeonly}
%    Macro \cs{includeonly} is ignored.
%    \begin{macrocode}
\renewcommand*{\includeonly}[1]{%
  \PackageInfo{stampinclude}{%
    Ignoring \string\includeonly
  }%
}
%    \end{macrocode}
%    \end{macro}
%
%    Simulate \cs{includeonly}.
%    \begin{macrocode}
\@partswtrue
\gdef\@partlist{}
%    \end{macrocode}
%
%    Print included files at end of document.
%    \begin{macrocode}
\AtEndDocument{%
  \begingroup
    \expandafter\let\expandafter\@partlist\expandafter\@empty
    \expandafter\@for\expandafter\reserved@a
    \expandafter:\expandafter=\@partlist\do{%
      \ifx\@partlist\@empty
        \edef\@partlist{\reserved@a}%
      \else
        \edef\@partlist{\@partlist, \reserved@a}%
      \fi
    }%
    \typeout{********************%
             ********************%
             ********************%
             ******************%
    }%
    \ifx\@partlist\@empty
      \typeout{[stampinclude] No included files.}%
    \else
      \typeout{[stampinclude] Included files:}%
      \typeout{\@partlist}%
    \fi
    \typeout{********************%
             ********************%
             ********************%
             ******************%
    }%
  \endgroup
}
%    \end{macrocode}
%
%    \begin{macrocode}
%</package>
%    \end{macrocode}
%
% \section{Installation}
%
% \subsection{Download}
%
% \paragraph{Package.} This package is available on
% CTAN\footnote{\url{http://ctan.org/pkg/stampinclude}}:
% \begin{description}
% \item[\CTAN{macros/latex/contrib/oberdiek/stampinclude.dtx}] The source file.
% \item[\CTAN{macros/latex/contrib/oberdiek/stampinclude.pdf}] Documentation.
% \end{description}
%
%
% \paragraph{Bundle.} All the packages of the bundle `oberdiek'
% are also available in a TDS compliant ZIP archive. There
% the packages are already unpacked and the documentation files
% are generated. The files and directories obey the TDS standard.
% \begin{description}
% \item[\CTAN{install/macros/latex/contrib/oberdiek.tds.zip}]
% \end{description}
% \emph{TDS} refers to the standard ``A Directory Structure
% for \TeX\ Files'' (\CTAN{tds/tds.pdf}). Directories
% with \xfile{texmf} in their name are usually organized this way.
%
% \subsection{Bundle installation}
%
% \paragraph{Unpacking.} Unpack the \xfile{oberdiek.tds.zip} in the
% TDS tree (also known as \xfile{texmf} tree) of your choice.
% Example (linux):
% \begin{quote}
%   |unzip oberdiek.tds.zip -d ~/texmf|
% \end{quote}
%
% \paragraph{Script installation.}
% Check the directory \xfile{TDS:scripts/oberdiek/} for
% scripts that need further installation steps.
% Package \xpackage{attachfile2} comes with the Perl script
% \xfile{pdfatfi.pl} that should be installed in such a way
% that it can be called as \texttt{pdfatfi}.
% Example (linux):
% \begin{quote}
%   |chmod +x scripts/oberdiek/pdfatfi.pl|\\
%   |cp scripts/oberdiek/pdfatfi.pl /usr/local/bin/|
% \end{quote}
%
% \subsection{Package installation}
%
% \paragraph{Unpacking.} The \xfile{.dtx} file is a self-extracting
% \docstrip\ archive. The files are extracted by running the
% \xfile{.dtx} through \plainTeX:
% \begin{quote}
%   \verb|tex stampinclude.dtx|
% \end{quote}
%
% \paragraph{TDS.} Now the different files must be moved into
% the different directories in your installation TDS tree
% (also known as \xfile{texmf} tree):
% \begin{quote}
% \def\t{^^A
% \begin{tabular}{@{}>{\ttfamily}l@{ $\rightarrow$ }>{\ttfamily}l@{}}
%   stampinclude.sty & tex/latex/oberdiek/stampinclude.sty\\
%   stampinclude.pdf & doc/latex/oberdiek/stampinclude.pdf\\
%   stampinclude.dtx & source/latex/oberdiek/stampinclude.dtx\\
% \end{tabular}^^A
% }^^A
% \sbox0{\t}^^A
% \ifdim\wd0>\linewidth
%   \begingroup
%     \advance\linewidth by\leftmargin
%     \advance\linewidth by\rightmargin
%   \edef\x{\endgroup
%     \def\noexpand\lw{\the\linewidth}^^A
%   }\x
%   \def\lwbox{^^A
%     \leavevmode
%     \hbox to \linewidth{^^A
%       \kern-\leftmargin\relax
%       \hss
%       \usebox0
%       \hss
%       \kern-\rightmargin\relax
%     }^^A
%   }^^A
%   \ifdim\wd0>\lw
%     \sbox0{\small\t}^^A
%     \ifdim\wd0>\linewidth
%       \ifdim\wd0>\lw
%         \sbox0{\footnotesize\t}^^A
%         \ifdim\wd0>\linewidth
%           \ifdim\wd0>\lw
%             \sbox0{\scriptsize\t}^^A
%             \ifdim\wd0>\linewidth
%               \ifdim\wd0>\lw
%                 \sbox0{\tiny\t}^^A
%                 \ifdim\wd0>\linewidth
%                   \lwbox
%                 \else
%                   \usebox0
%                 \fi
%               \else
%                 \lwbox
%               \fi
%             \else
%               \usebox0
%             \fi
%           \else
%             \lwbox
%           \fi
%         \else
%           \usebox0
%         \fi
%       \else
%         \lwbox
%       \fi
%     \else
%       \usebox0
%     \fi
%   \else
%     \lwbox
%   \fi
% \else
%   \usebox0
% \fi
% \end{quote}
% If you have a \xfile{docstrip.cfg} that configures and enables \docstrip's
% TDS installing feature, then some files can already be in the right
% place, see the documentation of \docstrip.
%
% \subsection{Refresh file name databases}
%
% If your \TeX~distribution
% (\teTeX, \mikTeX, \dots) relies on file name databases, you must refresh
% these. For example, \teTeX\ users run \verb|texhash| or
% \verb|mktexlsr|.
%
% \subsection{Some details for the interested}
%
% \paragraph{Attached source.}
%
% The PDF documentation on CTAN also includes the
% \xfile{.dtx} source file. It can be extracted by
% AcrobatReader 6 or higher. Another option is \textsf{pdftk},
% e.g. unpack the file into the current directory:
% \begin{quote}
%   \verb|pdftk stampinclude.pdf unpack_files output .|
% \end{quote}
%
% \paragraph{Unpacking with \LaTeX.}
% The \xfile{.dtx} chooses its action depending on the format:
% \begin{description}
% \item[\plainTeX:] Run \docstrip\ and extract the files.
% \item[\LaTeX:] Generate the documentation.
% \end{description}
% If you insist on using \LaTeX\ for \docstrip\ (really,
% \docstrip\ does not need \LaTeX), then inform the autodetect routine
% about your intention:
% \begin{quote}
%   \verb|latex \let\install=y\input{stampinclude.dtx}|
% \end{quote}
% Do not forget to quote the argument according to the demands
% of your shell.
%
% \paragraph{Generating the documentation.}
% You can use both the \xfile{.dtx} or the \xfile{.drv} to generate
% the documentation. The process can be configured by the
% configuration file \xfile{ltxdoc.cfg}. For instance, put this
% line into this file, if you want to have A4 as paper format:
% \begin{quote}
%   \verb|\PassOptionsToClass{a4paper}{article}|
% \end{quote}
% An example follows how to generate the
% documentation with pdf\LaTeX:
% \begin{quote}
%\begin{verbatim}
%pdflatex stampinclude.dtx
%makeindex -s gind.ist stampinclude.idx
%pdflatex stampinclude.dtx
%makeindex -s gind.ist stampinclude.idx
%pdflatex stampinclude.dtx
%\end{verbatim}
% \end{quote}
%
% \section{Catalogue}
%
% The following XML file can be used as source for the
% \href{http://mirror.ctan.org/help/Catalogue/catalogue.html}{\TeX\ Catalogue}.
% The elements \texttt{caption} and \texttt{description} are imported
% from the original XML file from the Catalogue.
% The name of the XML file in the Catalogue is \xfile{stampinclude.xml}.
%    \begin{macrocode}
%<*catalogue>
<?xml version='1.0' encoding='us-ascii'?>
<!DOCTYPE entry SYSTEM 'catalogue.dtd'>
<entry datestamp='$Date$' modifier='$Author$' id='stampinclude'>
  <name>stampinclude</name>
  <caption>Inclusion based on .aux file date stamps.</caption>
  <authorref id='auth:oberdiek'/>
  <copyright owner='Heiko Oberdiek' year='2008'/>
  <license type='lppl1.3'/>
  <version number='1.1'/>
  <description>
    This package replaces <tt>\includeonly</tt> and selects the files for
    <tt>\include</tt> by inspecting the timestamp of the <tt>.aux</tt> file.
    The file is selected for inclusion if the <tt>.aux</tt> file does
    not yet exist or is older than the corresponding <tt>.tex</tt> file.
    <p/>
    The package is part of the <xref refid='oberdiek'>oberdiek</xref>
    bundle.
  </description>
  <documentation details='Package documentation'
      href='ctan:/macros/latex/contrib/oberdiek/stampinclude.pdf'/>
  <ctan file='true' path='/macros/latex/contrib/oberdiek/stampinclude.dtx'/>
  <miktex location='oberdiek'/>
  <texlive location='oberdiek'/>
  <install path='/macros/latex/contrib/oberdiek/oberdiek.tds.zip'/>
</entry>
%</catalogue>
%    \end{macrocode}
%
% \begin{thebibliography}{9}
% \bibitem{askinclude}
%   Pablo A. Straub, Heiko Oberdiek:
%   \textit{The \xpackage{askinclude} package};
%   2007/10/23 v2.0;
%   \CTAN{macros/latex/contrib/oberdiek/askinclude.pdf}.
%
% \bibitem{pdftexcmds}
%   Heiko Oberdiek:
%   \textit{The \xpackage{pdftexcmds} package};
%   2007/12/12 v0.3;
%   \CTAN{macros/latex/contrib/oberdiek/pdftexcmds.pdf}.
%
% \end{thebibliography}
%
% \begin{History}
%   \begin{Version}{2008/07/14 v1.0}
%   \item
%     First version.
%   \end{Version}
%   \begin{Version}{2016/05/16 v1.1}
%   \item
%     Documentation updates.
%   \end{Version}
% \end{History}
%
% \PrintIndex
%
% \Finale
\endinput
|
% \end{quote}
% Do not forget to quote the argument according to the demands
% of your shell.
%
% \paragraph{Generating the documentation.}
% You can use both the \xfile{.dtx} or the \xfile{.drv} to generate
% the documentation. The process can be configured by the
% configuration file \xfile{ltxdoc.cfg}. For instance, put this
% line into this file, if you want to have A4 as paper format:
% \begin{quote}
%   \verb|\PassOptionsToClass{a4paper}{article}|
% \end{quote}
% An example follows how to generate the
% documentation with pdf\LaTeX:
% \begin{quote}
%\begin{verbatim}
%pdflatex stampinclude.dtx
%makeindex -s gind.ist stampinclude.idx
%pdflatex stampinclude.dtx
%makeindex -s gind.ist stampinclude.idx
%pdflatex stampinclude.dtx
%\end{verbatim}
% \end{quote}
%
% \section{Catalogue}
%
% The following XML file can be used as source for the
% \href{http://mirror.ctan.org/help/Catalogue/catalogue.html}{\TeX\ Catalogue}.
% The elements \texttt{caption} and \texttt{description} are imported
% from the original XML file from the Catalogue.
% The name of the XML file in the Catalogue is \xfile{stampinclude.xml}.
%    \begin{macrocode}
%<*catalogue>
<?xml version='1.0' encoding='us-ascii'?>
<!DOCTYPE entry SYSTEM 'catalogue.dtd'>
<entry datestamp='$Date$' modifier='$Author$' id='stampinclude'>
  <name>stampinclude</name>
  <caption>Inclusion based on .aux file date stamps.</caption>
  <authorref id='auth:oberdiek'/>
  <copyright owner='Heiko Oberdiek' year='2008'/>
  <license type='lppl1.3'/>
  <version number='1.1'/>
  <description>
    This package replaces <tt>\includeonly</tt> and selects the files for
    <tt>\include</tt> by inspecting the timestamp of the <tt>.aux</tt> file.
    The file is selected for inclusion if the <tt>.aux</tt> file does
    not yet exist or is older than the corresponding <tt>.tex</tt> file.
    <p/>
    The package is part of the <xref refid='oberdiek'>oberdiek</xref>
    bundle.
  </description>
  <documentation details='Package documentation'
      href='ctan:/macros/latex/contrib/oberdiek/stampinclude.pdf'/>
  <ctan file='true' path='/macros/latex/contrib/oberdiek/stampinclude.dtx'/>
  <miktex location='oberdiek'/>
  <texlive location='oberdiek'/>
  <install path='/macros/latex/contrib/oberdiek/oberdiek.tds.zip'/>
</entry>
%</catalogue>
%    \end{macrocode}
%
% \begin{thebibliography}{9}
% \bibitem{askinclude}
%   Pablo A. Straub, Heiko Oberdiek:
%   \textit{The \xpackage{askinclude} package};
%   2007/10/23 v2.0;
%   \CTAN{macros/latex/contrib/oberdiek/askinclude.pdf}.
%
% \bibitem{pdftexcmds}
%   Heiko Oberdiek:
%   \textit{The \xpackage{pdftexcmds} package};
%   2007/12/12 v0.3;
%   \CTAN{macros/latex/contrib/oberdiek/pdftexcmds.pdf}.
%
% \end{thebibliography}
%
% \begin{History}
%   \begin{Version}{2008/07/14 v1.0}
%   \item
%     First version.
%   \end{Version}
%   \begin{Version}{2016/05/16 v1.1}
%   \item
%     Documentation updates.
%   \end{Version}
% \end{History}
%
% \PrintIndex
%
% \Finale
\endinput
|
% \end{quote}
% Do not forget to quote the argument according to the demands
% of your shell.
%
% \paragraph{Generating the documentation.}
% You can use both the \xfile{.dtx} or the \xfile{.drv} to generate
% the documentation. The process can be configured by the
% configuration file \xfile{ltxdoc.cfg}. For instance, put this
% line into this file, if you want to have A4 as paper format:
% \begin{quote}
%   \verb|\PassOptionsToClass{a4paper}{article}|
% \end{quote}
% An example follows how to generate the
% documentation with pdf\LaTeX:
% \begin{quote}
%\begin{verbatim}
%pdflatex stampinclude.dtx
%makeindex -s gind.ist stampinclude.idx
%pdflatex stampinclude.dtx
%makeindex -s gind.ist stampinclude.idx
%pdflatex stampinclude.dtx
%\end{verbatim}
% \end{quote}
%
% \section{Catalogue}
%
% The following XML file can be used as source for the
% \href{http://mirror.ctan.org/help/Catalogue/catalogue.html}{\TeX\ Catalogue}.
% The elements \texttt{caption} and \texttt{description} are imported
% from the original XML file from the Catalogue.
% The name of the XML file in the Catalogue is \xfile{stampinclude.xml}.
%    \begin{macrocode}
%<*catalogue>
<?xml version='1.0' encoding='us-ascii'?>
<!DOCTYPE entry SYSTEM 'catalogue.dtd'>
<entry datestamp='$Date$' modifier='$Author$' id='stampinclude'>
  <name>stampinclude</name>
  <caption>Inclusion based on .aux file date stamps.</caption>
  <authorref id='auth:oberdiek'/>
  <copyright owner='Heiko Oberdiek' year='2008'/>
  <license type='lppl1.3'/>
  <version number='1.1'/>
  <description>
    This package replaces <tt>\includeonly</tt> and selects the files for
    <tt>\include</tt> by inspecting the timestamp of the <tt>.aux</tt> file.
    The file is selected for inclusion if the <tt>.aux</tt> file does
    not yet exist or is older than the corresponding <tt>.tex</tt> file.
    <p/>
    The package is part of the <xref refid='oberdiek'>oberdiek</xref>
    bundle.
  </description>
  <documentation details='Package documentation'
      href='ctan:/macros/latex/contrib/oberdiek/stampinclude.pdf'/>
  <ctan file='true' path='/macros/latex/contrib/oberdiek/stampinclude.dtx'/>
  <miktex location='oberdiek'/>
  <texlive location='oberdiek'/>
  <install path='/macros/latex/contrib/oberdiek/oberdiek.tds.zip'/>
</entry>
%</catalogue>
%    \end{macrocode}
%
% \begin{thebibliography}{9}
% \bibitem{askinclude}
%   Pablo A. Straub, Heiko Oberdiek:
%   \textit{The \xpackage{askinclude} package};
%   2007/10/23 v2.0;
%   \CTAN{macros/latex/contrib/oberdiek/askinclude.pdf}.
%
% \bibitem{pdftexcmds}
%   Heiko Oberdiek:
%   \textit{The \xpackage{pdftexcmds} package};
%   2007/12/12 v0.3;
%   \CTAN{macros/latex/contrib/oberdiek/pdftexcmds.pdf}.
%
% \end{thebibliography}
%
% \begin{History}
%   \begin{Version}{2008/07/14 v1.0}
%   \item
%     First version.
%   \end{Version}
%   \begin{Version}{2016/05/16 v1.1}
%   \item
%     Documentation updates.
%   \end{Version}
% \end{History}
%
% \PrintIndex
%
% \Finale
\endinput

%        (quote the arguments according to the demands of your shell)
%
% Documentation:
%    (a) If stampinclude.drv is present:
%           latex stampinclude.drv
%    (b) Without stampinclude.drv:
%           latex stampinclude.dtx; ...
%    The class ltxdoc loads the configuration file ltxdoc.cfg
%    if available. Here you can specify further options, e.g.
%    use A4 as paper format:
%       \PassOptionsToClass{a4paper}{article}
%
%    Programm calls to get the documentation (example):
%       pdflatex stampinclude.dtx
%       makeindex -s gind.ist stampinclude.idx
%       pdflatex stampinclude.dtx
%       makeindex -s gind.ist stampinclude.idx
%       pdflatex stampinclude.dtx
%
% Installation:
%    TDS:tex/latex/oberdiek/stampinclude.sty
%    TDS:doc/latex/oberdiek/stampinclude.pdf
%    TDS:source/latex/oberdiek/stampinclude.dtx
%
%<*ignore>
\begingroup
  \catcode123=1 %
  \catcode125=2 %
  \def\x{LaTeX2e}%
\expandafter\endgroup
\ifcase 0\ifx\install y1\fi\expandafter
         \ifx\csname processbatchFile\endcsname\relax\else1\fi
         \ifx\fmtname\x\else 1\fi\relax
\else\csname fi\endcsname
%</ignore>
%<*install>
\input docstrip.tex
\Msg{************************************************************************}
\Msg{* Installation}
\Msg{* Package: stampinclude 2008/07/14 v1.0 Include files based on time stamps (HO)}
\Msg{************************************************************************}

\keepsilent
\askforoverwritefalse

\let\MetaPrefix\relax
\preamble

This is a generated file.

Project: stampinclude
Version: 2008/07/14 v1.0

Copyright (C) 2008 by
   Heiko Oberdiek <heiko.oberdiek at googlemail.com>

This work may be distributed and/or modified under the
conditions of the LaTeX Project Public License, either
version 1.3c of this license or (at your option) any later
version. This version of this license is in
   http://www.latex-project.org/lppl/lppl-1-3c.txt
and the latest version of this license is in
   http://www.latex-project.org/lppl.txt
and version 1.3 or later is part of all distributions of
LaTeX version 2005/12/01 or later.

This work has the LPPL maintenance status "maintained".

This Current Maintainer of this work is Heiko Oberdiek.

This work consists of the main source file stampinclude.dtx
and the derived files
   stampinclude.sty, stampinclude.pdf, stampinclude.ins, stampinclude.drv.

\endpreamble
\let\MetaPrefix\DoubleperCent

\generate{%
  \file{stampinclude.ins}{\from{stampinclude.dtx}{install}}%
  \file{stampinclude.drv}{\from{stampinclude.dtx}{driver}}%
  \usedir{tex/latex/oberdiek}%
  \file{stampinclude.sty}{\from{stampinclude.dtx}{package}}%
  \nopreamble
  \nopostamble
  \usedir{source/latex/oberdiek/catalogue}%
  \file{stampinclude.xml}{\from{stampinclude.dtx}{catalogue}}%
}

\catcode32=13\relax% active space
\let =\space%
\Msg{************************************************************************}
\Msg{*}
\Msg{* To finish the installation you have to move the following}
\Msg{* file into a directory searched by TeX:}
\Msg{*}
\Msg{*     stampinclude.sty}
\Msg{*}
\Msg{* To produce the documentation run the file `stampinclude.drv'}
\Msg{* through LaTeX.}
\Msg{*}
\Msg{* Happy TeXing!}
\Msg{*}
\Msg{************************************************************************}

\endbatchfile
%</install>
%<*ignore>
\fi
%</ignore>
%<*driver>
\NeedsTeXFormat{LaTeX2e}
\ProvidesFile{stampinclude.drv}%
  [2008/07/14 v1.0 Include files based on time stamps (HO)]%
\documentclass{ltxdoc}
\usepackage{holtxdoc}[2011/11/22]
\begin{document}
  \DocInput{stampinclude.dtx}%
\end{document}
%</driver>
% \fi
%
% \CheckSum{120}
%
% \CharacterTable
%  {Upper-case    \A\B\C\D\E\F\G\H\I\J\K\L\M\N\O\P\Q\R\S\T\U\V\W\X\Y\Z
%   Lower-case    \a\b\c\d\e\f\g\h\i\j\k\l\m\n\o\p\q\r\s\t\u\v\w\x\y\z
%   Digits        \0\1\2\3\4\5\6\7\8\9
%   Exclamation   \!     Double quote  \"     Hash (number) \#
%   Dollar        \$     Percent       \%     Ampersand     \&
%   Acute accent  \'     Left paren    \(     Right paren   \)
%   Asterisk      \*     Plus          \+     Comma         \,
%   Minus         \-     Point         \.     Solidus       \/
%   Colon         \:     Semicolon     \;     Less than     \<
%   Equals        \=     Greater than  \>     Question mark \?
%   Commercial at \@     Left bracket  \[     Backslash     \\
%   Right bracket \]     Circumflex    \^     Underscore    \_
%   Grave accent  \`     Left brace    \{     Vertical bar  \|
%   Right brace   \}     Tilde         \~}
%
% \GetFileInfo{stampinclude.drv}
%
% \title{The \xpackage{stampinclude} package}
% \date{2008/07/14 v1.0}
% \author{Heiko Oberdiek\\\xemail{heiko.oberdiek at googlemail.com}}
%
% \maketitle
%
% \begin{abstract}
% The package replaces \cs{includeonly} and selects the files for
% \cs{include} by inspecting the time stamp of the \xext{aux} file.
% The file is selected for inclusion if the \xext{aux} file does
% not yet exist or is older than the corresponding \xext{tex} file.
% \end{abstract}
%
% \tableofcontents
%
% \section{Documentation}
%
% \subsection{Introduction}
% \label{sec:intro}
%
% \LaTeX\ provides two commands \cs{include} and \cs{includeonly}
% that helps in organizing large projects.
% Example for a master file:
%\begin{quote}
%\begin{verbatim}
%\documentclass{book}
%  % \includeonly{}
%\begin{document}
% \include{fileA}
% \include{fileB}
% \include{fileC}
%\end{document}
%\end{verbatim}
%\end{quote}
% All files are read and compiled if \cs{includeonly} is not
% executed. Otherwise you can give \cs{includeonly} a list
% of files in the preamble, e.g.:
% \begin{quote}
%   |\includeonly{fileA,fileC}|
% \end{quote}
% Now only files \xfile{fileA.tex} and \xfile{fileC.tex} are read
% and compiled.
%
% If you change file \xfile{fileB.tex} and want to see only this
% file, then you must change the line with \cs{includeonly} to
% \begin{quote}
%   |\includeonly{fileB}|
% \end{quote}
% It is tedious to do this again and again, if different files
% are changed.
%
% Package \xpackage{askinclude} \cite{askinclude}
% offers a solution for this problem. It interactively asks
% for the files to be included and saves the user from
% editing the master file.
%
% This package \xpackage{stampinclude} goes another way.
% \LaTeX\ reads and writes a separate \xext{aux} file for each
% file that is included by \cs{include}. There \LaTeX\ remembers
% counter valuses. Changed \xext{tex}
% files can therefore be detected by comparing the file date stamp of
% the \xext{tex} file with the date stamp of its \xext{aux} file.
% Since version 1.30.0 \pdfTeX\ provides \cs{pdffilemoddate}
% that reads the file date stamp. Thus this package uses this
% command and redefines
% \cs{include} to include the files that do not have \xext{aux}
% files yet or that are newer than its \xext{aux} file.
% \cs{includeonly} is ignored.
%
% \subsection{Usage}
%
% The package is loaded as normal \LaTeX\ package without options:
% \begin{quote}
%   |\usepackage{stampinclude}|
% \end{quote}
% Alternatively the package may be loaded on the command line
% (Example for shell `bash'):
% \begin{center}
%   |latex '\AtBeginDocument{\usepackage{stampinclude}}\input{master}'|
% \end{center}
% Without \cs{AtBeginDocument} (and \cs{RequirePackage} instead of
% \cs{usepackage}) \TeX\ would name the document \xfile{stampinclude.dvi}
% instead of \xfile{master.dvi}.
%
% \subsection{Limitations}
%
% \subsubsection{Other file dependencies}
%
% A file that is included by \cs{include} may input ore reference
% other files:
% \begin{itemize}
% \item other \TeX\ files using \cs{input},
% \item graphics files (\cs{includegraphics}),
% \item listings of external files,
% \item ...
% \end{itemize}
% Updates of those files are not detected by this package.
% It limits the date stamp comparison of an \xext{aux} file
% to its \xext{tex} file.
%
% \subsubsection{\cs{include} dependencies}
%
% In the example, given in the introduction \ref{sec:intro},
% three files \xfile{fileA}, \xfile{fileB}, and \xfile{fileC}
% are included in this order. Now file \xfile{fileA} is changed by adding
% four pages, \xfile{fileB} remains untouched, and \xfile{fileC} is
% also updated. Then the package only selects \xfile{fileA} and
% \xfile{fileC} for inclusion. File \xfile{fileB} is not included.
% But \LaTeX\ has stored the counter values that are active
% at the end of \xfile{fileB} in \xfile{fileB.aux} in one of the
% previous runs when \xfile{fileB} was included.
% However the later addition of four pages in \xfile{fileA}
% was not known at that time. Therefore \xfile{fileB.aux}
% is out of date and the inclusion of file \xfile{fileC}
% starts with wrong counter values (especially the page counter).
%
% \subsubsection{Summary}
%
% This package \xpackage{stampinclude} and the \cs{include} feature
% helps in accelerating the \LaTeX\ compilation.
% But it is not intended for generating the final version.
% For the final version of the document it is better to include
% \emph{all} files to get all counter values right.
% Then this package and any \cs{includeonly} lines should be commented out:
%\begin{quote}
%  |% \usepackage{stampinclude}|\\
%  |% \includeonly{...}|
%\end{quote}
%
% \subsection{Requirements}
%
% \begin{itemize}
% \item \pdfTeX\ v1.30.0 (because of \cs{pdffilemoddate}
%   and \cs{pdfstrcmp}),\\
%   both modes for DVI and PDF are supported.
% \item Alternatively Lua\TeX\ may be used.
%   It lacks \cs{pdffilemoddate} and \cs{pdfstrcmp}. But its services
%   are provided by package \xpackage{pdftexcmds} \cite{pdftexcmds}
%   that is automatically loaded.
% \end{itemize}
%
% \StopEventually{
% }
%
% \section{Implementation}
%
%    \begin{macrocode}
%<*package>
\NeedsTeXFormat{LaTeX2e}
\ProvidesPackage{stampinclude}
  [2008/07/14 v1.0 Include files based on time stamps (HO)]%
%    \end{macrocode}
%
%    \begin{macrocode}
\RequirePackage{pdftexcmds}[2007/12/12]%
%    \end{macrocode}
%
%    \begin{macrocode}
\begingroup
  \chardef\x=1 %
  \expandafter\ifx\csname pdf@filemoddate\endcsname\relax
    \chardef\x=0 %
  \fi
  \expandafter\ifx\csname pdf@strcmp\endcsname\relax
    \chardef\x=0 %
  \fi
\expandafter\endgroup\ifcase\x
  \PackageWarningNoLine{stampinclude}{%
    \string\pdffilemoddate\space or %
    \string\pdfstrcmp\space are not found,\MessageBreak
    that are provided by pdfTeX >= 1.30.0.\MessageBreak
    Also LuaTeX is not detected.\MessageBreak
    Therefore package loading is aborted%
  }%
  \expandafter\endinput
\fi
%    \end{macrocode}
%
%    \begin{macro}{\SInc@org@include}
%    \begin{macrocode}
\let\SInc@org@include\@include
%    \end{macrocode}
%    \end{macro}
%    \begin{macro}{\@include}
%    \begin{macrocode}
\def\@include#1 {%
  \IfFileExists{#1.aux}{%
    \ifnum\pdf@strcmp{\pdf@filemoddate{#1.aux}}%
                     {\pdf@filemoddate{#1.tex}}<0 %
      \ifx\@partlist\@empty
        \gdef\@partlist{{#1}}%
      \else
        \g@addto@macro\@partlist{,{#1}}%
      \fi
    \fi
  }{%
    \ifx\@partlist\@empty
      \gdef\@partlist{{#1}}%
    \else
      \g@addto@macro\@partlist{,{#1}}%
    \fi
  }%
  \SInc@org@include{#1} \relax
}
%    \end{macrocode}
%    \end{macro}
%
%    \begin{macro}{\includeonly}
%    Macro \cs{includeonly} is ignored.
%    \begin{macrocode}
\renewcommand*{\includeonly}[1]{%
  \PackageInfo{stampinclude}{%
    Ignoring \string\includeonly
  }%
}
%    \end{macrocode}
%    \end{macro}
%
%    Simulate \cs{includeonly}.
%    \begin{macrocode}
\@partswtrue
\gdef\@partlist{}
%    \end{macrocode}
%
%    Print included files at end of document.
%    \begin{macrocode}
\AtEndDocument{%
  \begingroup
    \expandafter\let\expandafter\@partlist\expandafter\@empty
    \expandafter\@for\expandafter\reserved@a
    \expandafter:\expandafter=\@partlist\do{%
      \ifx\@partlist\@empty
        \edef\@partlist{\reserved@a}%
      \else
        \edef\@partlist{\@partlist, \reserved@a}%
      \fi
    }%
    \typeout{********************%
             ********************%
             ********************%
             ******************%
    }%
    \ifx\@partlist\@empty
      \typeout{[stampinclude] No included files.}%
    \else
      \typeout{[stampinclude] Included files:}%
      \typeout{\@partlist}%
    \fi
    \typeout{********************%
             ********************%
             ********************%
             ******************%
    }%
  \endgroup
}
%    \end{macrocode}
%
%    \begin{macrocode}
%</package>
%    \end{macrocode}
%
% \section{Installation}
%
% \subsection{Download}
%
% \paragraph{Package.} This package is available on
% CTAN\footnote{\url{ftp://ftp.ctan.org/tex-archive/}}:
% \begin{description}
% \item[\CTAN{macros/latex/contrib/oberdiek/stampinclude.dtx}] The source file.
% \item[\CTAN{macros/latex/contrib/oberdiek/stampinclude.pdf}] Documentation.
% \end{description}
%
%
% \paragraph{Bundle.} All the packages of the bundle `oberdiek'
% are also available in a TDS compliant ZIP archive. There
% the packages are already unpacked and the documentation files
% are generated. The files and directories obey the TDS standard.
% \begin{description}
% \item[\CTAN{install/macros/latex/contrib/oberdiek.tds.zip}]
% \end{description}
% \emph{TDS} refers to the standard ``A Directory Structure
% for \TeX\ Files'' (\CTAN{tds/tds.pdf}). Directories
% with \xfile{texmf} in their name are usually organized this way.
%
% \subsection{Bundle installation}
%
% \paragraph{Unpacking.} Unpack the \xfile{oberdiek.tds.zip} in the
% TDS tree (also known as \xfile{texmf} tree) of your choice.
% Example (linux):
% \begin{quote}
%   |unzip oberdiek.tds.zip -d ~/texmf|
% \end{quote}
%
% \paragraph{Script installation.}
% Check the directory \xfile{TDS:scripts/oberdiek/} for
% scripts that need further installation steps.
% Package \xpackage{attachfile2} comes with the Perl script
% \xfile{pdfatfi.pl} that should be installed in such a way
% that it can be called as \texttt{pdfatfi}.
% Example (linux):
% \begin{quote}
%   |chmod +x scripts/oberdiek/pdfatfi.pl|\\
%   |cp scripts/oberdiek/pdfatfi.pl /usr/local/bin/|
% \end{quote}
%
% \subsection{Package installation}
%
% \paragraph{Unpacking.} The \xfile{.dtx} file is a self-extracting
% \docstrip\ archive. The files are extracted by running the
% \xfile{.dtx} through \plainTeX:
% \begin{quote}
%   \verb|tex stampinclude.dtx|
% \end{quote}
%
% \paragraph{TDS.} Now the different files must be moved into
% the different directories in your installation TDS tree
% (also known as \xfile{texmf} tree):
% \begin{quote}
% \def\t{^^A
% \begin{tabular}{@{}>{\ttfamily}l@{ $\rightarrow$ }>{\ttfamily}l@{}}
%   stampinclude.sty & tex/latex/oberdiek/stampinclude.sty\\
%   stampinclude.pdf & doc/latex/oberdiek/stampinclude.pdf\\
%   stampinclude.dtx & source/latex/oberdiek/stampinclude.dtx\\
% \end{tabular}^^A
% }^^A
% \sbox0{\t}^^A
% \ifdim\wd0>\linewidth
%   \begingroup
%     \advance\linewidth by\leftmargin
%     \advance\linewidth by\rightmargin
%   \edef\x{\endgroup
%     \def\noexpand\lw{\the\linewidth}^^A
%   }\x
%   \def\lwbox{^^A
%     \leavevmode
%     \hbox to \linewidth{^^A
%       \kern-\leftmargin\relax
%       \hss
%       \usebox0
%       \hss
%       \kern-\rightmargin\relax
%     }^^A
%   }^^A
%   \ifdim\wd0>\lw
%     \sbox0{\small\t}^^A
%     \ifdim\wd0>\linewidth
%       \ifdim\wd0>\lw
%         \sbox0{\footnotesize\t}^^A
%         \ifdim\wd0>\linewidth
%           \ifdim\wd0>\lw
%             \sbox0{\scriptsize\t}^^A
%             \ifdim\wd0>\linewidth
%               \ifdim\wd0>\lw
%                 \sbox0{\tiny\t}^^A
%                 \ifdim\wd0>\linewidth
%                   \lwbox
%                 \else
%                   \usebox0
%                 \fi
%               \else
%                 \lwbox
%               \fi
%             \else
%               \usebox0
%             \fi
%           \else
%             \lwbox
%           \fi
%         \else
%           \usebox0
%         \fi
%       \else
%         \lwbox
%       \fi
%     \else
%       \usebox0
%     \fi
%   \else
%     \lwbox
%   \fi
% \else
%   \usebox0
% \fi
% \end{quote}
% If you have a \xfile{docstrip.cfg} that configures and enables \docstrip's
% TDS installing feature, then some files can already be in the right
% place, see the documentation of \docstrip.
%
% \subsection{Refresh file name databases}
%
% If your \TeX~distribution
% (\teTeX, \mikTeX, \dots) relies on file name databases, you must refresh
% these. For example, \teTeX\ users run \verb|texhash| or
% \verb|mktexlsr|.
%
% \subsection{Some details for the interested}
%
% \paragraph{Attached source.}
%
% The PDF documentation on CTAN also includes the
% \xfile{.dtx} source file. It can be extracted by
% AcrobatReader 6 or higher. Another option is \textsf{pdftk},
% e.g. unpack the file into the current directory:
% \begin{quote}
%   \verb|pdftk stampinclude.pdf unpack_files output .|
% \end{quote}
%
% \paragraph{Unpacking with \LaTeX.}
% The \xfile{.dtx} chooses its action depending on the format:
% \begin{description}
% \item[\plainTeX:] Run \docstrip\ and extract the files.
% \item[\LaTeX:] Generate the documentation.
% \end{description}
% If you insist on using \LaTeX\ for \docstrip\ (really,
% \docstrip\ does not need \LaTeX), then inform the autodetect routine
% about your intention:
% \begin{quote}
%   \verb|latex \let\install=y% \iffalse meta-comment
%
% File: stampinclude.dtx
% Version: 2016/05/16 v1.1
% Info: Include files based on time stamps
%
% Copyright (C) 2008 by
%    Heiko Oberdiek <heiko.oberdiek at googlemail.com>
%    2016
%    https://github.com/ho-tex/oberdiek/issues
%
% This work may be distributed and/or modified under the
% conditions of the LaTeX Project Public License, either
% version 1.3c of this license or (at your option) any later
% version. This version of this license is in
%    http://www.latex-project.org/lppl/lppl-1-3c.txt
% and the latest version of this license is in
%    http://www.latex-project.org/lppl.txt
% and version 1.3 or later is part of all distributions of
% LaTeX version 2005/12/01 or later.
%
% This work has the LPPL maintenance status "maintained".
%
% This Current Maintainer of this work is Heiko Oberdiek.
%
% This work consists of the main source file stampinclude.dtx
% and the derived files
%    stampinclude.sty, stampinclude.pdf, stampinclude.ins, stampinclude.drv.
%
% Distribution:
%    CTAN:macros/latex/contrib/oberdiek/stampinclude.dtx
%    CTAN:macros/latex/contrib/oberdiek/stampinclude.pdf
%
% Unpacking:
%    (a) If stampinclude.ins is present:
%           tex stampinclude.ins
%    (b) Without stampinclude.ins:
%           tex stampinclude.dtx
%    (c) If you insist on using LaTeX
%           latex \let\install=y% \iffalse meta-comment
%
% File: stampinclude.dtx
% Version: 2016/05/16 v1.1
% Info: Include files based on time stamps
%
% Copyright (C) 2008 by
%    Heiko Oberdiek <heiko.oberdiek at googlemail.com>
%    2016
%    https://github.com/ho-tex/oberdiek/issues
%
% This work may be distributed and/or modified under the
% conditions of the LaTeX Project Public License, either
% version 1.3c of this license or (at your option) any later
% version. This version of this license is in
%    http://www.latex-project.org/lppl/lppl-1-3c.txt
% and the latest version of this license is in
%    http://www.latex-project.org/lppl.txt
% and version 1.3 or later is part of all distributions of
% LaTeX version 2005/12/01 or later.
%
% This work has the LPPL maintenance status "maintained".
%
% This Current Maintainer of this work is Heiko Oberdiek.
%
% This work consists of the main source file stampinclude.dtx
% and the derived files
%    stampinclude.sty, stampinclude.pdf, stampinclude.ins, stampinclude.drv.
%
% Distribution:
%    CTAN:macros/latex/contrib/oberdiek/stampinclude.dtx
%    CTAN:macros/latex/contrib/oberdiek/stampinclude.pdf
%
% Unpacking:
%    (a) If stampinclude.ins is present:
%           tex stampinclude.ins
%    (b) Without stampinclude.ins:
%           tex stampinclude.dtx
%    (c) If you insist on using LaTeX
%           latex \let\install=y% \iffalse meta-comment
%
% File: stampinclude.dtx
% Version: 2016/05/16 v1.1
% Info: Include files based on time stamps
%
% Copyright (C) 2008 by
%    Heiko Oberdiek <heiko.oberdiek at googlemail.com>
%    2016
%    https://github.com/ho-tex/oberdiek/issues
%
% This work may be distributed and/or modified under the
% conditions of the LaTeX Project Public License, either
% version 1.3c of this license or (at your option) any later
% version. This version of this license is in
%    http://www.latex-project.org/lppl/lppl-1-3c.txt
% and the latest version of this license is in
%    http://www.latex-project.org/lppl.txt
% and version 1.3 or later is part of all distributions of
% LaTeX version 2005/12/01 or later.
%
% This work has the LPPL maintenance status "maintained".
%
% This Current Maintainer of this work is Heiko Oberdiek.
%
% This work consists of the main source file stampinclude.dtx
% and the derived files
%    stampinclude.sty, stampinclude.pdf, stampinclude.ins, stampinclude.drv.
%
% Distribution:
%    CTAN:macros/latex/contrib/oberdiek/stampinclude.dtx
%    CTAN:macros/latex/contrib/oberdiek/stampinclude.pdf
%
% Unpacking:
%    (a) If stampinclude.ins is present:
%           tex stampinclude.ins
%    (b) Without stampinclude.ins:
%           tex stampinclude.dtx
%    (c) If you insist on using LaTeX
%           latex \let\install=y\input{stampinclude.dtx}
%        (quote the arguments according to the demands of your shell)
%
% Documentation:
%    (a) If stampinclude.drv is present:
%           latex stampinclude.drv
%    (b) Without stampinclude.drv:
%           latex stampinclude.dtx; ...
%    The class ltxdoc loads the configuration file ltxdoc.cfg
%    if available. Here you can specify further options, e.g.
%    use A4 as paper format:
%       \PassOptionsToClass{a4paper}{article}
%
%    Programm calls to get the documentation (example):
%       pdflatex stampinclude.dtx
%       makeindex -s gind.ist stampinclude.idx
%       pdflatex stampinclude.dtx
%       makeindex -s gind.ist stampinclude.idx
%       pdflatex stampinclude.dtx
%
% Installation:
%    TDS:tex/latex/oberdiek/stampinclude.sty
%    TDS:doc/latex/oberdiek/stampinclude.pdf
%    TDS:source/latex/oberdiek/stampinclude.dtx
%
%<*ignore>
\begingroup
  \catcode123=1 %
  \catcode125=2 %
  \def\x{LaTeX2e}%
\expandafter\endgroup
\ifcase 0\ifx\install y1\fi\expandafter
         \ifx\csname processbatchFile\endcsname\relax\else1\fi
         \ifx\fmtname\x\else 1\fi\relax
\else\csname fi\endcsname
%</ignore>
%<*install>
\input docstrip.tex
\Msg{************************************************************************}
\Msg{* Installation}
\Msg{* Package: stampinclude 2016/05/16 v1.1 Include files based on time stamps (HO)}
\Msg{************************************************************************}

\keepsilent
\askforoverwritefalse

\let\MetaPrefix\relax
\preamble

This is a generated file.

Project: stampinclude
Version: 2016/05/16 v1.1

Copyright (C) 2008 by
   Heiko Oberdiek <heiko.oberdiek at googlemail.com>

This work may be distributed and/or modified under the
conditions of the LaTeX Project Public License, either
version 1.3c of this license or (at your option) any later
version. This version of this license is in
   http://www.latex-project.org/lppl/lppl-1-3c.txt
and the latest version of this license is in
   http://www.latex-project.org/lppl.txt
and version 1.3 or later is part of all distributions of
LaTeX version 2005/12/01 or later.

This work has the LPPL maintenance status "maintained".

This Current Maintainer of this work is Heiko Oberdiek.

This work consists of the main source file stampinclude.dtx
and the derived files
   stampinclude.sty, stampinclude.pdf, stampinclude.ins, stampinclude.drv.

\endpreamble
\let\MetaPrefix\DoubleperCent

\generate{%
  \file{stampinclude.ins}{\from{stampinclude.dtx}{install}}%
  \file{stampinclude.drv}{\from{stampinclude.dtx}{driver}}%
  \usedir{tex/latex/oberdiek}%
  \file{stampinclude.sty}{\from{stampinclude.dtx}{package}}%
  \nopreamble
  \nopostamble
  \usedir{source/latex/oberdiek/catalogue}%
  \file{stampinclude.xml}{\from{stampinclude.dtx}{catalogue}}%
}

\catcode32=13\relax% active space
\let =\space%
\Msg{************************************************************************}
\Msg{*}
\Msg{* To finish the installation you have to move the following}
\Msg{* file into a directory searched by TeX:}
\Msg{*}
\Msg{*     stampinclude.sty}
\Msg{*}
\Msg{* To produce the documentation run the file `stampinclude.drv'}
\Msg{* through LaTeX.}
\Msg{*}
\Msg{* Happy TeXing!}
\Msg{*}
\Msg{************************************************************************}

\endbatchfile
%</install>
%<*ignore>
\fi
%</ignore>
%<*driver>
\NeedsTeXFormat{LaTeX2e}
\ProvidesFile{stampinclude.drv}%
  [2016/05/16 v1.1 Include files based on time stamps (HO)]%
\documentclass{ltxdoc}
\usepackage{holtxdoc}[2011/11/22]
\begin{document}
  \DocInput{stampinclude.dtx}%
\end{document}
%</driver>
% \fi
%
%
% \CharacterTable
%  {Upper-case    \A\B\C\D\E\F\G\H\I\J\K\L\M\N\O\P\Q\R\S\T\U\V\W\X\Y\Z
%   Lower-case    \a\b\c\d\e\f\g\h\i\j\k\l\m\n\o\p\q\r\s\t\u\v\w\x\y\z
%   Digits        \0\1\2\3\4\5\6\7\8\9
%   Exclamation   \!     Double quote  \"     Hash (number) \#
%   Dollar        \$     Percent       \%     Ampersand     \&
%   Acute accent  \'     Left paren    \(     Right paren   \)
%   Asterisk      \*     Plus          \+     Comma         \,
%   Minus         \-     Point         \.     Solidus       \/
%   Colon         \:     Semicolon     \;     Less than     \<
%   Equals        \=     Greater than  \>     Question mark \?
%   Commercial at \@     Left bracket  \[     Backslash     \\
%   Right bracket \]     Circumflex    \^     Underscore    \_
%   Grave accent  \`     Left brace    \{     Vertical bar  \|
%   Right brace   \}     Tilde         \~}
%
% \GetFileInfo{stampinclude.drv}
%
% \title{The \xpackage{stampinclude} package}
% \date{2016/05/16 v1.1}
% \author{Heiko Oberdiek\thanks
% {Please report any issues at https://github.com/ho-tex/oberdiek/issues}\\
% \xemail{heiko.oberdiek at googlemail.com}}
%
% \maketitle
%
% \begin{abstract}
% The package replaces \cs{includeonly} and selects the files for
% \cs{include} by inspecting the time stamp of the \xext{aux} file.
% The file is selected for inclusion if the \xext{aux} file does
% not yet exist or is older than the corresponding \xext{tex} file.
% \end{abstract}
%
% \tableofcontents
%
% \section{Documentation}
%
% \subsection{Introduction}
% \label{sec:intro}
%
% \LaTeX\ provides two commands \cs{include} and \cs{includeonly}
% that helps in organizing large projects.
% Example for a master file:
%\begin{quote}
%\begin{verbatim}
%\documentclass{book}
%  % \includeonly{}
%\begin{document}
% \include{fileA}
% \include{fileB}
% \include{fileC}
%\end{document}
%\end{verbatim}
%\end{quote}
% All files are read and compiled if \cs{includeonly} is not
% executed. Otherwise you can give \cs{includeonly} a list
% of files in the preamble, e.g.:
% \begin{quote}
%   |\includeonly{fileA,fileC}|
% \end{quote}
% Now only files \xfile{fileA.tex} and \xfile{fileC.tex} are read
% and compiled.
%
% If you change file \xfile{fileB.tex} and want to see only this
% file, then you must change the line with \cs{includeonly} to
% \begin{quote}
%   |\includeonly{fileB}|
% \end{quote}
% It is tedious to do this again and again, if different files
% are changed.
%
% Package \xpackage{askinclude} \cite{askinclude}
% offers a solution for this problem. It interactively asks
% for the files to be included and saves the user from
% editing the master file.
%
% This package \xpackage{stampinclude} goes another way.
% \LaTeX\ reads and writes a separate \xext{aux} file for each
% file that is included by \cs{include}. There \LaTeX\ remembers
% counter valuses. Changed \xext{tex}
% files can therefore be detected by comparing the file date stamp of
% the \xext{tex} file with the date stamp of its \xext{aux} file.
% Since version 1.30.0 \pdfTeX\ provides \cs{pdffilemoddate}
% that reads the file date stamp. Thus this package uses this
% command and redefines
% \cs{include} to include the files that do not have \xext{aux}
% files yet or that are newer than its \xext{aux} file.
% \cs{includeonly} is ignored.
%
% \subsection{Usage}
%
% The package is loaded as normal \LaTeX\ package without options:
% \begin{quote}
%   |\usepackage{stampinclude}|
% \end{quote}
% Alternatively the package may be loaded on the command line
% (Example for shell `bash'):
% \begin{center}
%   |latex '\AtBeginDocument{\usepackage{stampinclude}}\input{master}'|
% \end{center}
% Without \cs{AtBeginDocument} (and \cs{RequirePackage} instead of
% \cs{usepackage}) \TeX\ would name the document \xfile{stampinclude.dvi}
% instead of \xfile{master.dvi}.
%
% \subsection{Limitations}
%
% \subsubsection{Other file dependencies}
%
% A file that is included by \cs{include} may input ore reference
% other files:
% \begin{itemize}
% \item other \TeX\ files using \cs{input},
% \item graphics files (\cs{includegraphics}),
% \item listings of external files,
% \item ...
% \end{itemize}
% Updates of those files are not detected by this package.
% It limits the date stamp comparison of an \xext{aux} file
% to its \xext{tex} file.
%
% \subsubsection{\cs{include} dependencies}
%
% In the example, given in the introduction \ref{sec:intro},
% three files \xfile{fileA}, \xfile{fileB}, and \xfile{fileC}
% are included in this order. Now file \xfile{fileA} is changed by adding
% four pages, \xfile{fileB} remains untouched, and \xfile{fileC} is
% also updated. Then the package only selects \xfile{fileA} and
% \xfile{fileC} for inclusion. File \xfile{fileB} is not included.
% But \LaTeX\ has stored the counter values that are active
% at the end of \xfile{fileB} in \xfile{fileB.aux} in one of the
% previous runs when \xfile{fileB} was included.
% However the later addition of four pages in \xfile{fileA}
% was not known at that time. Therefore \xfile{fileB.aux}
% is out of date and the inclusion of file \xfile{fileC}
% starts with wrong counter values (especially the page counter).
%
% \subsubsection{Summary}
%
% This package \xpackage{stampinclude} and the \cs{include} feature
% helps in accelerating the \LaTeX\ compilation.
% But it is not intended for generating the final version.
% For the final version of the document it is better to include
% \emph{all} files to get all counter values right.
% Then this package and any \cs{includeonly} lines should be commented out:
%\begin{quote}
%  |% \usepackage{stampinclude}|\\
%  |% \includeonly{...}|
%\end{quote}
%
% \subsection{Requirements}
%
% \begin{itemize}
% \item \pdfTeX\ v1.30.0 (because of \cs{pdffilemoddate}
%   and \cs{pdfstrcmp}),\\
%   both modes for DVI and PDF are supported.
% \item Alternatively Lua\TeX\ may be used.
%   It lacks \cs{pdffilemoddate} and \cs{pdfstrcmp}. But its services
%   are provided by package \xpackage{pdftexcmds} \cite{pdftexcmds}
%   that is automatically loaded.
% \end{itemize}
%
% \StopEventually{
% }
%
% \section{Implementation}
%
%    \begin{macrocode}
%<*package>
\NeedsTeXFormat{LaTeX2e}
\ProvidesPackage{stampinclude}
  [2016/05/16 v1.1 Include files based on time stamps (HO)]%
%    \end{macrocode}
%
%    \begin{macrocode}
\RequirePackage{pdftexcmds}[2007/12/12]%
%    \end{macrocode}
%
%    \begin{macrocode}
\begingroup
  \chardef\x=1 %
  \expandafter\ifx\csname pdf@filemoddate\endcsname\relax
    \chardef\x=0 %
  \fi
  \expandafter\ifx\csname pdf@strcmp\endcsname\relax
    \chardef\x=0 %
  \fi
\expandafter\endgroup\ifcase\x
  \PackageWarningNoLine{stampinclude}{%
    \string\pdffilemoddate\space or %
    \string\pdfstrcmp\space are not found,\MessageBreak
    that are provided by pdfTeX >= 1.30.0.\MessageBreak
    Also LuaTeX is not detected.\MessageBreak
    Therefore package loading is aborted%
  }%
  \expandafter\endinput
\fi
%    \end{macrocode}
%
%    \begin{macro}{\SInc@org@include}
%    \begin{macrocode}
\let\SInc@org@include\@include
%    \end{macrocode}
%    \end{macro}
%    \begin{macro}{\@include}
%    \begin{macrocode}
\def\@include#1 {%
  \IfFileExists{#1.aux}{%
    \ifnum\pdf@strcmp{\pdf@filemoddate{#1.aux}}%
                     {\pdf@filemoddate{#1.tex}}<0 %
      \ifx\@partlist\@empty
        \gdef\@partlist{{#1}}%
      \else
        \g@addto@macro\@partlist{,{#1}}%
      \fi
    \fi
  }{%
    \ifx\@partlist\@empty
      \gdef\@partlist{{#1}}%
    \else
      \g@addto@macro\@partlist{,{#1}}%
    \fi
  }%
  \SInc@org@include{#1} \relax
}
%    \end{macrocode}
%    \end{macro}
%
%    \begin{macro}{\includeonly}
%    Macro \cs{includeonly} is ignored.
%    \begin{macrocode}
\renewcommand*{\includeonly}[1]{%
  \PackageInfo{stampinclude}{%
    Ignoring \string\includeonly
  }%
}
%    \end{macrocode}
%    \end{macro}
%
%    Simulate \cs{includeonly}.
%    \begin{macrocode}
\@partswtrue
\gdef\@partlist{}
%    \end{macrocode}
%
%    Print included files at end of document.
%    \begin{macrocode}
\AtEndDocument{%
  \begingroup
    \expandafter\let\expandafter\@partlist\expandafter\@empty
    \expandafter\@for\expandafter\reserved@a
    \expandafter:\expandafter=\@partlist\do{%
      \ifx\@partlist\@empty
        \edef\@partlist{\reserved@a}%
      \else
        \edef\@partlist{\@partlist, \reserved@a}%
      \fi
    }%
    \typeout{********************%
             ********************%
             ********************%
             ******************%
    }%
    \ifx\@partlist\@empty
      \typeout{[stampinclude] No included files.}%
    \else
      \typeout{[stampinclude] Included files:}%
      \typeout{\@partlist}%
    \fi
    \typeout{********************%
             ********************%
             ********************%
             ******************%
    }%
  \endgroup
}
%    \end{macrocode}
%
%    \begin{macrocode}
%</package>
%    \end{macrocode}
%
% \section{Installation}
%
% \subsection{Download}
%
% \paragraph{Package.} This package is available on
% CTAN\footnote{\url{http://ctan.org/pkg/stampinclude}}:
% \begin{description}
% \item[\CTAN{macros/latex/contrib/oberdiek/stampinclude.dtx}] The source file.
% \item[\CTAN{macros/latex/contrib/oberdiek/stampinclude.pdf}] Documentation.
% \end{description}
%
%
% \paragraph{Bundle.} All the packages of the bundle `oberdiek'
% are also available in a TDS compliant ZIP archive. There
% the packages are already unpacked and the documentation files
% are generated. The files and directories obey the TDS standard.
% \begin{description}
% \item[\CTAN{install/macros/latex/contrib/oberdiek.tds.zip}]
% \end{description}
% \emph{TDS} refers to the standard ``A Directory Structure
% for \TeX\ Files'' (\CTAN{tds/tds.pdf}). Directories
% with \xfile{texmf} in their name are usually organized this way.
%
% \subsection{Bundle installation}
%
% \paragraph{Unpacking.} Unpack the \xfile{oberdiek.tds.zip} in the
% TDS tree (also known as \xfile{texmf} tree) of your choice.
% Example (linux):
% \begin{quote}
%   |unzip oberdiek.tds.zip -d ~/texmf|
% \end{quote}
%
% \paragraph{Script installation.}
% Check the directory \xfile{TDS:scripts/oberdiek/} for
% scripts that need further installation steps.
% Package \xpackage{attachfile2} comes with the Perl script
% \xfile{pdfatfi.pl} that should be installed in such a way
% that it can be called as \texttt{pdfatfi}.
% Example (linux):
% \begin{quote}
%   |chmod +x scripts/oberdiek/pdfatfi.pl|\\
%   |cp scripts/oberdiek/pdfatfi.pl /usr/local/bin/|
% \end{quote}
%
% \subsection{Package installation}
%
% \paragraph{Unpacking.} The \xfile{.dtx} file is a self-extracting
% \docstrip\ archive. The files are extracted by running the
% \xfile{.dtx} through \plainTeX:
% \begin{quote}
%   \verb|tex stampinclude.dtx|
% \end{quote}
%
% \paragraph{TDS.} Now the different files must be moved into
% the different directories in your installation TDS tree
% (also known as \xfile{texmf} tree):
% \begin{quote}
% \def\t{^^A
% \begin{tabular}{@{}>{\ttfamily}l@{ $\rightarrow$ }>{\ttfamily}l@{}}
%   stampinclude.sty & tex/latex/oberdiek/stampinclude.sty\\
%   stampinclude.pdf & doc/latex/oberdiek/stampinclude.pdf\\
%   stampinclude.dtx & source/latex/oberdiek/stampinclude.dtx\\
% \end{tabular}^^A
% }^^A
% \sbox0{\t}^^A
% \ifdim\wd0>\linewidth
%   \begingroup
%     \advance\linewidth by\leftmargin
%     \advance\linewidth by\rightmargin
%   \edef\x{\endgroup
%     \def\noexpand\lw{\the\linewidth}^^A
%   }\x
%   \def\lwbox{^^A
%     \leavevmode
%     \hbox to \linewidth{^^A
%       \kern-\leftmargin\relax
%       \hss
%       \usebox0
%       \hss
%       \kern-\rightmargin\relax
%     }^^A
%   }^^A
%   \ifdim\wd0>\lw
%     \sbox0{\small\t}^^A
%     \ifdim\wd0>\linewidth
%       \ifdim\wd0>\lw
%         \sbox0{\footnotesize\t}^^A
%         \ifdim\wd0>\linewidth
%           \ifdim\wd0>\lw
%             \sbox0{\scriptsize\t}^^A
%             \ifdim\wd0>\linewidth
%               \ifdim\wd0>\lw
%                 \sbox0{\tiny\t}^^A
%                 \ifdim\wd0>\linewidth
%                   \lwbox
%                 \else
%                   \usebox0
%                 \fi
%               \else
%                 \lwbox
%               \fi
%             \else
%               \usebox0
%             \fi
%           \else
%             \lwbox
%           \fi
%         \else
%           \usebox0
%         \fi
%       \else
%         \lwbox
%       \fi
%     \else
%       \usebox0
%     \fi
%   \else
%     \lwbox
%   \fi
% \else
%   \usebox0
% \fi
% \end{quote}
% If you have a \xfile{docstrip.cfg} that configures and enables \docstrip's
% TDS installing feature, then some files can already be in the right
% place, see the documentation of \docstrip.
%
% \subsection{Refresh file name databases}
%
% If your \TeX~distribution
% (\teTeX, \mikTeX, \dots) relies on file name databases, you must refresh
% these. For example, \teTeX\ users run \verb|texhash| or
% \verb|mktexlsr|.
%
% \subsection{Some details for the interested}
%
% \paragraph{Attached source.}
%
% The PDF documentation on CTAN also includes the
% \xfile{.dtx} source file. It can be extracted by
% AcrobatReader 6 or higher. Another option is \textsf{pdftk},
% e.g. unpack the file into the current directory:
% \begin{quote}
%   \verb|pdftk stampinclude.pdf unpack_files output .|
% \end{quote}
%
% \paragraph{Unpacking with \LaTeX.}
% The \xfile{.dtx} chooses its action depending on the format:
% \begin{description}
% \item[\plainTeX:] Run \docstrip\ and extract the files.
% \item[\LaTeX:] Generate the documentation.
% \end{description}
% If you insist on using \LaTeX\ for \docstrip\ (really,
% \docstrip\ does not need \LaTeX), then inform the autodetect routine
% about your intention:
% \begin{quote}
%   \verb|latex \let\install=y\input{stampinclude.dtx}|
% \end{quote}
% Do not forget to quote the argument according to the demands
% of your shell.
%
% \paragraph{Generating the documentation.}
% You can use both the \xfile{.dtx} or the \xfile{.drv} to generate
% the documentation. The process can be configured by the
% configuration file \xfile{ltxdoc.cfg}. For instance, put this
% line into this file, if you want to have A4 as paper format:
% \begin{quote}
%   \verb|\PassOptionsToClass{a4paper}{article}|
% \end{quote}
% An example follows how to generate the
% documentation with pdf\LaTeX:
% \begin{quote}
%\begin{verbatim}
%pdflatex stampinclude.dtx
%makeindex -s gind.ist stampinclude.idx
%pdflatex stampinclude.dtx
%makeindex -s gind.ist stampinclude.idx
%pdflatex stampinclude.dtx
%\end{verbatim}
% \end{quote}
%
% \section{Catalogue}
%
% The following XML file can be used as source for the
% \href{http://mirror.ctan.org/help/Catalogue/catalogue.html}{\TeX\ Catalogue}.
% The elements \texttt{caption} and \texttt{description} are imported
% from the original XML file from the Catalogue.
% The name of the XML file in the Catalogue is \xfile{stampinclude.xml}.
%    \begin{macrocode}
%<*catalogue>
<?xml version='1.0' encoding='us-ascii'?>
<!DOCTYPE entry SYSTEM 'catalogue.dtd'>
<entry datestamp='$Date$' modifier='$Author$' id='stampinclude'>
  <name>stampinclude</name>
  <caption>Inclusion based on .aux file date stamps.</caption>
  <authorref id='auth:oberdiek'/>
  <copyright owner='Heiko Oberdiek' year='2008'/>
  <license type='lppl1.3'/>
  <version number='1.1'/>
  <description>
    This package replaces <tt>\includeonly</tt> and selects the files for
    <tt>\include</tt> by inspecting the timestamp of the <tt>.aux</tt> file.
    The file is selected for inclusion if the <tt>.aux</tt> file does
    not yet exist or is older than the corresponding <tt>.tex</tt> file.
    <p/>
    The package is part of the <xref refid='oberdiek'>oberdiek</xref>
    bundle.
  </description>
  <documentation details='Package documentation'
      href='ctan:/macros/latex/contrib/oberdiek/stampinclude.pdf'/>
  <ctan file='true' path='/macros/latex/contrib/oberdiek/stampinclude.dtx'/>
  <miktex location='oberdiek'/>
  <texlive location='oberdiek'/>
  <install path='/macros/latex/contrib/oberdiek/oberdiek.tds.zip'/>
</entry>
%</catalogue>
%    \end{macrocode}
%
% \begin{thebibliography}{9}
% \bibitem{askinclude}
%   Pablo A. Straub, Heiko Oberdiek:
%   \textit{The \xpackage{askinclude} package};
%   2007/10/23 v2.0;
%   \CTAN{macros/latex/contrib/oberdiek/askinclude.pdf}.
%
% \bibitem{pdftexcmds}
%   Heiko Oberdiek:
%   \textit{The \xpackage{pdftexcmds} package};
%   2007/12/12 v0.3;
%   \CTAN{macros/latex/contrib/oberdiek/pdftexcmds.pdf}.
%
% \end{thebibliography}
%
% \begin{History}
%   \begin{Version}{2008/07/14 v1.0}
%   \item
%     First version.
%   \end{Version}
%   \begin{Version}{2016/05/16 v1.1}
%   \item
%     Documentation updates.
%   \end{Version}
% \end{History}
%
% \PrintIndex
%
% \Finale
\endinput

%        (quote the arguments according to the demands of your shell)
%
% Documentation:
%    (a) If stampinclude.drv is present:
%           latex stampinclude.drv
%    (b) Without stampinclude.drv:
%           latex stampinclude.dtx; ...
%    The class ltxdoc loads the configuration file ltxdoc.cfg
%    if available. Here you can specify further options, e.g.
%    use A4 as paper format:
%       \PassOptionsToClass{a4paper}{article}
%
%    Programm calls to get the documentation (example):
%       pdflatex stampinclude.dtx
%       makeindex -s gind.ist stampinclude.idx
%       pdflatex stampinclude.dtx
%       makeindex -s gind.ist stampinclude.idx
%       pdflatex stampinclude.dtx
%
% Installation:
%    TDS:tex/latex/oberdiek/stampinclude.sty
%    TDS:doc/latex/oberdiek/stampinclude.pdf
%    TDS:source/latex/oberdiek/stampinclude.dtx
%
%<*ignore>
\begingroup
  \catcode123=1 %
  \catcode125=2 %
  \def\x{LaTeX2e}%
\expandafter\endgroup
\ifcase 0\ifx\install y1\fi\expandafter
         \ifx\csname processbatchFile\endcsname\relax\else1\fi
         \ifx\fmtname\x\else 1\fi\relax
\else\csname fi\endcsname
%</ignore>
%<*install>
\input docstrip.tex
\Msg{************************************************************************}
\Msg{* Installation}
\Msg{* Package: stampinclude 2016/05/16 v1.1 Include files based on time stamps (HO)}
\Msg{************************************************************************}

\keepsilent
\askforoverwritefalse

\let\MetaPrefix\relax
\preamble

This is a generated file.

Project: stampinclude
Version: 2016/05/16 v1.1

Copyright (C) 2008 by
   Heiko Oberdiek <heiko.oberdiek at googlemail.com>

This work may be distributed and/or modified under the
conditions of the LaTeX Project Public License, either
version 1.3c of this license or (at your option) any later
version. This version of this license is in
   http://www.latex-project.org/lppl/lppl-1-3c.txt
and the latest version of this license is in
   http://www.latex-project.org/lppl.txt
and version 1.3 or later is part of all distributions of
LaTeX version 2005/12/01 or later.

This work has the LPPL maintenance status "maintained".

This Current Maintainer of this work is Heiko Oberdiek.

This work consists of the main source file stampinclude.dtx
and the derived files
   stampinclude.sty, stampinclude.pdf, stampinclude.ins, stampinclude.drv.

\endpreamble
\let\MetaPrefix\DoubleperCent

\generate{%
  \file{stampinclude.ins}{\from{stampinclude.dtx}{install}}%
  \file{stampinclude.drv}{\from{stampinclude.dtx}{driver}}%
  \usedir{tex/latex/oberdiek}%
  \file{stampinclude.sty}{\from{stampinclude.dtx}{package}}%
  \nopreamble
  \nopostamble
  \usedir{source/latex/oberdiek/catalogue}%
  \file{stampinclude.xml}{\from{stampinclude.dtx}{catalogue}}%
}

\catcode32=13\relax% active space
\let =\space%
\Msg{************************************************************************}
\Msg{*}
\Msg{* To finish the installation you have to move the following}
\Msg{* file into a directory searched by TeX:}
\Msg{*}
\Msg{*     stampinclude.sty}
\Msg{*}
\Msg{* To produce the documentation run the file `stampinclude.drv'}
\Msg{* through LaTeX.}
\Msg{*}
\Msg{* Happy TeXing!}
\Msg{*}
\Msg{************************************************************************}

\endbatchfile
%</install>
%<*ignore>
\fi
%</ignore>
%<*driver>
\NeedsTeXFormat{LaTeX2e}
\ProvidesFile{stampinclude.drv}%
  [2016/05/16 v1.1 Include files based on time stamps (HO)]%
\documentclass{ltxdoc}
\usepackage{holtxdoc}[2011/11/22]
\begin{document}
  \DocInput{stampinclude.dtx}%
\end{document}
%</driver>
% \fi
%
%
% \CharacterTable
%  {Upper-case    \A\B\C\D\E\F\G\H\I\J\K\L\M\N\O\P\Q\R\S\T\U\V\W\X\Y\Z
%   Lower-case    \a\b\c\d\e\f\g\h\i\j\k\l\m\n\o\p\q\r\s\t\u\v\w\x\y\z
%   Digits        \0\1\2\3\4\5\6\7\8\9
%   Exclamation   \!     Double quote  \"     Hash (number) \#
%   Dollar        \$     Percent       \%     Ampersand     \&
%   Acute accent  \'     Left paren    \(     Right paren   \)
%   Asterisk      \*     Plus          \+     Comma         \,
%   Minus         \-     Point         \.     Solidus       \/
%   Colon         \:     Semicolon     \;     Less than     \<
%   Equals        \=     Greater than  \>     Question mark \?
%   Commercial at \@     Left bracket  \[     Backslash     \\
%   Right bracket \]     Circumflex    \^     Underscore    \_
%   Grave accent  \`     Left brace    \{     Vertical bar  \|
%   Right brace   \}     Tilde         \~}
%
% \GetFileInfo{stampinclude.drv}
%
% \title{The \xpackage{stampinclude} package}
% \date{2016/05/16 v1.1}
% \author{Heiko Oberdiek\thanks
% {Please report any issues at https://github.com/ho-tex/oberdiek/issues}\\
% \xemail{heiko.oberdiek at googlemail.com}}
%
% \maketitle
%
% \begin{abstract}
% The package replaces \cs{includeonly} and selects the files for
% \cs{include} by inspecting the time stamp of the \xext{aux} file.
% The file is selected for inclusion if the \xext{aux} file does
% not yet exist or is older than the corresponding \xext{tex} file.
% \end{abstract}
%
% \tableofcontents
%
% \section{Documentation}
%
% \subsection{Introduction}
% \label{sec:intro}
%
% \LaTeX\ provides two commands \cs{include} and \cs{includeonly}
% that helps in organizing large projects.
% Example for a master file:
%\begin{quote}
%\begin{verbatim}
%\documentclass{book}
%  % \includeonly{}
%\begin{document}
% \include{fileA}
% \include{fileB}
% \include{fileC}
%\end{document}
%\end{verbatim}
%\end{quote}
% All files are read and compiled if \cs{includeonly} is not
% executed. Otherwise you can give \cs{includeonly} a list
% of files in the preamble, e.g.:
% \begin{quote}
%   |\includeonly{fileA,fileC}|
% \end{quote}
% Now only files \xfile{fileA.tex} and \xfile{fileC.tex} are read
% and compiled.
%
% If you change file \xfile{fileB.tex} and want to see only this
% file, then you must change the line with \cs{includeonly} to
% \begin{quote}
%   |\includeonly{fileB}|
% \end{quote}
% It is tedious to do this again and again, if different files
% are changed.
%
% Package \xpackage{askinclude} \cite{askinclude}
% offers a solution for this problem. It interactively asks
% for the files to be included and saves the user from
% editing the master file.
%
% This package \xpackage{stampinclude} goes another way.
% \LaTeX\ reads and writes a separate \xext{aux} file for each
% file that is included by \cs{include}. There \LaTeX\ remembers
% counter valuses. Changed \xext{tex}
% files can therefore be detected by comparing the file date stamp of
% the \xext{tex} file with the date stamp of its \xext{aux} file.
% Since version 1.30.0 \pdfTeX\ provides \cs{pdffilemoddate}
% that reads the file date stamp. Thus this package uses this
% command and redefines
% \cs{include} to include the files that do not have \xext{aux}
% files yet or that are newer than its \xext{aux} file.
% \cs{includeonly} is ignored.
%
% \subsection{Usage}
%
% The package is loaded as normal \LaTeX\ package without options:
% \begin{quote}
%   |\usepackage{stampinclude}|
% \end{quote}
% Alternatively the package may be loaded on the command line
% (Example for shell `bash'):
% \begin{center}
%   |latex '\AtBeginDocument{\usepackage{stampinclude}}\input{master}'|
% \end{center}
% Without \cs{AtBeginDocument} (and \cs{RequirePackage} instead of
% \cs{usepackage}) \TeX\ would name the document \xfile{stampinclude.dvi}
% instead of \xfile{master.dvi}.
%
% \subsection{Limitations}
%
% \subsubsection{Other file dependencies}
%
% A file that is included by \cs{include} may input ore reference
% other files:
% \begin{itemize}
% \item other \TeX\ files using \cs{input},
% \item graphics files (\cs{includegraphics}),
% \item listings of external files,
% \item ...
% \end{itemize}
% Updates of those files are not detected by this package.
% It limits the date stamp comparison of an \xext{aux} file
% to its \xext{tex} file.
%
% \subsubsection{\cs{include} dependencies}
%
% In the example, given in the introduction \ref{sec:intro},
% three files \xfile{fileA}, \xfile{fileB}, and \xfile{fileC}
% are included in this order. Now file \xfile{fileA} is changed by adding
% four pages, \xfile{fileB} remains untouched, and \xfile{fileC} is
% also updated. Then the package only selects \xfile{fileA} and
% \xfile{fileC} for inclusion. File \xfile{fileB} is not included.
% But \LaTeX\ has stored the counter values that are active
% at the end of \xfile{fileB} in \xfile{fileB.aux} in one of the
% previous runs when \xfile{fileB} was included.
% However the later addition of four pages in \xfile{fileA}
% was not known at that time. Therefore \xfile{fileB.aux}
% is out of date and the inclusion of file \xfile{fileC}
% starts with wrong counter values (especially the page counter).
%
% \subsubsection{Summary}
%
% This package \xpackage{stampinclude} and the \cs{include} feature
% helps in accelerating the \LaTeX\ compilation.
% But it is not intended for generating the final version.
% For the final version of the document it is better to include
% \emph{all} files to get all counter values right.
% Then this package and any \cs{includeonly} lines should be commented out:
%\begin{quote}
%  |% \usepackage{stampinclude}|\\
%  |% \includeonly{...}|
%\end{quote}
%
% \subsection{Requirements}
%
% \begin{itemize}
% \item \pdfTeX\ v1.30.0 (because of \cs{pdffilemoddate}
%   and \cs{pdfstrcmp}),\\
%   both modes for DVI and PDF are supported.
% \item Alternatively Lua\TeX\ may be used.
%   It lacks \cs{pdffilemoddate} and \cs{pdfstrcmp}. But its services
%   are provided by package \xpackage{pdftexcmds} \cite{pdftexcmds}
%   that is automatically loaded.
% \end{itemize}
%
% \StopEventually{
% }
%
% \section{Implementation}
%
%    \begin{macrocode}
%<*package>
\NeedsTeXFormat{LaTeX2e}
\ProvidesPackage{stampinclude}
  [2016/05/16 v1.1 Include files based on time stamps (HO)]%
%    \end{macrocode}
%
%    \begin{macrocode}
\RequirePackage{pdftexcmds}[2007/12/12]%
%    \end{macrocode}
%
%    \begin{macrocode}
\begingroup
  \chardef\x=1 %
  \expandafter\ifx\csname pdf@filemoddate\endcsname\relax
    \chardef\x=0 %
  \fi
  \expandafter\ifx\csname pdf@strcmp\endcsname\relax
    \chardef\x=0 %
  \fi
\expandafter\endgroup\ifcase\x
  \PackageWarningNoLine{stampinclude}{%
    \string\pdffilemoddate\space or %
    \string\pdfstrcmp\space are not found,\MessageBreak
    that are provided by pdfTeX >= 1.30.0.\MessageBreak
    Also LuaTeX is not detected.\MessageBreak
    Therefore package loading is aborted%
  }%
  \expandafter\endinput
\fi
%    \end{macrocode}
%
%    \begin{macro}{\SInc@org@include}
%    \begin{macrocode}
\let\SInc@org@include\@include
%    \end{macrocode}
%    \end{macro}
%    \begin{macro}{\@include}
%    \begin{macrocode}
\def\@include#1 {%
  \IfFileExists{#1.aux}{%
    \ifnum\pdf@strcmp{\pdf@filemoddate{#1.aux}}%
                     {\pdf@filemoddate{#1.tex}}<0 %
      \ifx\@partlist\@empty
        \gdef\@partlist{{#1}}%
      \else
        \g@addto@macro\@partlist{,{#1}}%
      \fi
    \fi
  }{%
    \ifx\@partlist\@empty
      \gdef\@partlist{{#1}}%
    \else
      \g@addto@macro\@partlist{,{#1}}%
    \fi
  }%
  \SInc@org@include{#1} \relax
}
%    \end{macrocode}
%    \end{macro}
%
%    \begin{macro}{\includeonly}
%    Macro \cs{includeonly} is ignored.
%    \begin{macrocode}
\renewcommand*{\includeonly}[1]{%
  \PackageInfo{stampinclude}{%
    Ignoring \string\includeonly
  }%
}
%    \end{macrocode}
%    \end{macro}
%
%    Simulate \cs{includeonly}.
%    \begin{macrocode}
\@partswtrue
\gdef\@partlist{}
%    \end{macrocode}
%
%    Print included files at end of document.
%    \begin{macrocode}
\AtEndDocument{%
  \begingroup
    \expandafter\let\expandafter\@partlist\expandafter\@empty
    \expandafter\@for\expandafter\reserved@a
    \expandafter:\expandafter=\@partlist\do{%
      \ifx\@partlist\@empty
        \edef\@partlist{\reserved@a}%
      \else
        \edef\@partlist{\@partlist, \reserved@a}%
      \fi
    }%
    \typeout{********************%
             ********************%
             ********************%
             ******************%
    }%
    \ifx\@partlist\@empty
      \typeout{[stampinclude] No included files.}%
    \else
      \typeout{[stampinclude] Included files:}%
      \typeout{\@partlist}%
    \fi
    \typeout{********************%
             ********************%
             ********************%
             ******************%
    }%
  \endgroup
}
%    \end{macrocode}
%
%    \begin{macrocode}
%</package>
%    \end{macrocode}
%
% \section{Installation}
%
% \subsection{Download}
%
% \paragraph{Package.} This package is available on
% CTAN\footnote{\url{http://ctan.org/pkg/stampinclude}}:
% \begin{description}
% \item[\CTAN{macros/latex/contrib/oberdiek/stampinclude.dtx}] The source file.
% \item[\CTAN{macros/latex/contrib/oberdiek/stampinclude.pdf}] Documentation.
% \end{description}
%
%
% \paragraph{Bundle.} All the packages of the bundle `oberdiek'
% are also available in a TDS compliant ZIP archive. There
% the packages are already unpacked and the documentation files
% are generated. The files and directories obey the TDS standard.
% \begin{description}
% \item[\CTAN{install/macros/latex/contrib/oberdiek.tds.zip}]
% \end{description}
% \emph{TDS} refers to the standard ``A Directory Structure
% for \TeX\ Files'' (\CTAN{tds/tds.pdf}). Directories
% with \xfile{texmf} in their name are usually organized this way.
%
% \subsection{Bundle installation}
%
% \paragraph{Unpacking.} Unpack the \xfile{oberdiek.tds.zip} in the
% TDS tree (also known as \xfile{texmf} tree) of your choice.
% Example (linux):
% \begin{quote}
%   |unzip oberdiek.tds.zip -d ~/texmf|
% \end{quote}
%
% \paragraph{Script installation.}
% Check the directory \xfile{TDS:scripts/oberdiek/} for
% scripts that need further installation steps.
% Package \xpackage{attachfile2} comes with the Perl script
% \xfile{pdfatfi.pl} that should be installed in such a way
% that it can be called as \texttt{pdfatfi}.
% Example (linux):
% \begin{quote}
%   |chmod +x scripts/oberdiek/pdfatfi.pl|\\
%   |cp scripts/oberdiek/pdfatfi.pl /usr/local/bin/|
% \end{quote}
%
% \subsection{Package installation}
%
% \paragraph{Unpacking.} The \xfile{.dtx} file is a self-extracting
% \docstrip\ archive. The files are extracted by running the
% \xfile{.dtx} through \plainTeX:
% \begin{quote}
%   \verb|tex stampinclude.dtx|
% \end{quote}
%
% \paragraph{TDS.} Now the different files must be moved into
% the different directories in your installation TDS tree
% (also known as \xfile{texmf} tree):
% \begin{quote}
% \def\t{^^A
% \begin{tabular}{@{}>{\ttfamily}l@{ $\rightarrow$ }>{\ttfamily}l@{}}
%   stampinclude.sty & tex/latex/oberdiek/stampinclude.sty\\
%   stampinclude.pdf & doc/latex/oberdiek/stampinclude.pdf\\
%   stampinclude.dtx & source/latex/oberdiek/stampinclude.dtx\\
% \end{tabular}^^A
% }^^A
% \sbox0{\t}^^A
% \ifdim\wd0>\linewidth
%   \begingroup
%     \advance\linewidth by\leftmargin
%     \advance\linewidth by\rightmargin
%   \edef\x{\endgroup
%     \def\noexpand\lw{\the\linewidth}^^A
%   }\x
%   \def\lwbox{^^A
%     \leavevmode
%     \hbox to \linewidth{^^A
%       \kern-\leftmargin\relax
%       \hss
%       \usebox0
%       \hss
%       \kern-\rightmargin\relax
%     }^^A
%   }^^A
%   \ifdim\wd0>\lw
%     \sbox0{\small\t}^^A
%     \ifdim\wd0>\linewidth
%       \ifdim\wd0>\lw
%         \sbox0{\footnotesize\t}^^A
%         \ifdim\wd0>\linewidth
%           \ifdim\wd0>\lw
%             \sbox0{\scriptsize\t}^^A
%             \ifdim\wd0>\linewidth
%               \ifdim\wd0>\lw
%                 \sbox0{\tiny\t}^^A
%                 \ifdim\wd0>\linewidth
%                   \lwbox
%                 \else
%                   \usebox0
%                 \fi
%               \else
%                 \lwbox
%               \fi
%             \else
%               \usebox0
%             \fi
%           \else
%             \lwbox
%           \fi
%         \else
%           \usebox0
%         \fi
%       \else
%         \lwbox
%       \fi
%     \else
%       \usebox0
%     \fi
%   \else
%     \lwbox
%   \fi
% \else
%   \usebox0
% \fi
% \end{quote}
% If you have a \xfile{docstrip.cfg} that configures and enables \docstrip's
% TDS installing feature, then some files can already be in the right
% place, see the documentation of \docstrip.
%
% \subsection{Refresh file name databases}
%
% If your \TeX~distribution
% (\teTeX, \mikTeX, \dots) relies on file name databases, you must refresh
% these. For example, \teTeX\ users run \verb|texhash| or
% \verb|mktexlsr|.
%
% \subsection{Some details for the interested}
%
% \paragraph{Attached source.}
%
% The PDF documentation on CTAN also includes the
% \xfile{.dtx} source file. It can be extracted by
% AcrobatReader 6 or higher. Another option is \textsf{pdftk},
% e.g. unpack the file into the current directory:
% \begin{quote}
%   \verb|pdftk stampinclude.pdf unpack_files output .|
% \end{quote}
%
% \paragraph{Unpacking with \LaTeX.}
% The \xfile{.dtx} chooses its action depending on the format:
% \begin{description}
% \item[\plainTeX:] Run \docstrip\ and extract the files.
% \item[\LaTeX:] Generate the documentation.
% \end{description}
% If you insist on using \LaTeX\ for \docstrip\ (really,
% \docstrip\ does not need \LaTeX), then inform the autodetect routine
% about your intention:
% \begin{quote}
%   \verb|latex \let\install=y% \iffalse meta-comment
%
% File: stampinclude.dtx
% Version: 2016/05/16 v1.1
% Info: Include files based on time stamps
%
% Copyright (C) 2008 by
%    Heiko Oberdiek <heiko.oberdiek at googlemail.com>
%    2016
%    https://github.com/ho-tex/oberdiek/issues
%
% This work may be distributed and/or modified under the
% conditions of the LaTeX Project Public License, either
% version 1.3c of this license or (at your option) any later
% version. This version of this license is in
%    http://www.latex-project.org/lppl/lppl-1-3c.txt
% and the latest version of this license is in
%    http://www.latex-project.org/lppl.txt
% and version 1.3 or later is part of all distributions of
% LaTeX version 2005/12/01 or later.
%
% This work has the LPPL maintenance status "maintained".
%
% This Current Maintainer of this work is Heiko Oberdiek.
%
% This work consists of the main source file stampinclude.dtx
% and the derived files
%    stampinclude.sty, stampinclude.pdf, stampinclude.ins, stampinclude.drv.
%
% Distribution:
%    CTAN:macros/latex/contrib/oberdiek/stampinclude.dtx
%    CTAN:macros/latex/contrib/oberdiek/stampinclude.pdf
%
% Unpacking:
%    (a) If stampinclude.ins is present:
%           tex stampinclude.ins
%    (b) Without stampinclude.ins:
%           tex stampinclude.dtx
%    (c) If you insist on using LaTeX
%           latex \let\install=y\input{stampinclude.dtx}
%        (quote the arguments according to the demands of your shell)
%
% Documentation:
%    (a) If stampinclude.drv is present:
%           latex stampinclude.drv
%    (b) Without stampinclude.drv:
%           latex stampinclude.dtx; ...
%    The class ltxdoc loads the configuration file ltxdoc.cfg
%    if available. Here you can specify further options, e.g.
%    use A4 as paper format:
%       \PassOptionsToClass{a4paper}{article}
%
%    Programm calls to get the documentation (example):
%       pdflatex stampinclude.dtx
%       makeindex -s gind.ist stampinclude.idx
%       pdflatex stampinclude.dtx
%       makeindex -s gind.ist stampinclude.idx
%       pdflatex stampinclude.dtx
%
% Installation:
%    TDS:tex/latex/oberdiek/stampinclude.sty
%    TDS:doc/latex/oberdiek/stampinclude.pdf
%    TDS:source/latex/oberdiek/stampinclude.dtx
%
%<*ignore>
\begingroup
  \catcode123=1 %
  \catcode125=2 %
  \def\x{LaTeX2e}%
\expandafter\endgroup
\ifcase 0\ifx\install y1\fi\expandafter
         \ifx\csname processbatchFile\endcsname\relax\else1\fi
         \ifx\fmtname\x\else 1\fi\relax
\else\csname fi\endcsname
%</ignore>
%<*install>
\input docstrip.tex
\Msg{************************************************************************}
\Msg{* Installation}
\Msg{* Package: stampinclude 2016/05/16 v1.1 Include files based on time stamps (HO)}
\Msg{************************************************************************}

\keepsilent
\askforoverwritefalse

\let\MetaPrefix\relax
\preamble

This is a generated file.

Project: stampinclude
Version: 2016/05/16 v1.1

Copyright (C) 2008 by
   Heiko Oberdiek <heiko.oberdiek at googlemail.com>

This work may be distributed and/or modified under the
conditions of the LaTeX Project Public License, either
version 1.3c of this license or (at your option) any later
version. This version of this license is in
   http://www.latex-project.org/lppl/lppl-1-3c.txt
and the latest version of this license is in
   http://www.latex-project.org/lppl.txt
and version 1.3 or later is part of all distributions of
LaTeX version 2005/12/01 or later.

This work has the LPPL maintenance status "maintained".

This Current Maintainer of this work is Heiko Oberdiek.

This work consists of the main source file stampinclude.dtx
and the derived files
   stampinclude.sty, stampinclude.pdf, stampinclude.ins, stampinclude.drv.

\endpreamble
\let\MetaPrefix\DoubleperCent

\generate{%
  \file{stampinclude.ins}{\from{stampinclude.dtx}{install}}%
  \file{stampinclude.drv}{\from{stampinclude.dtx}{driver}}%
  \usedir{tex/latex/oberdiek}%
  \file{stampinclude.sty}{\from{stampinclude.dtx}{package}}%
  \nopreamble
  \nopostamble
  \usedir{source/latex/oberdiek/catalogue}%
  \file{stampinclude.xml}{\from{stampinclude.dtx}{catalogue}}%
}

\catcode32=13\relax% active space
\let =\space%
\Msg{************************************************************************}
\Msg{*}
\Msg{* To finish the installation you have to move the following}
\Msg{* file into a directory searched by TeX:}
\Msg{*}
\Msg{*     stampinclude.sty}
\Msg{*}
\Msg{* To produce the documentation run the file `stampinclude.drv'}
\Msg{* through LaTeX.}
\Msg{*}
\Msg{* Happy TeXing!}
\Msg{*}
\Msg{************************************************************************}

\endbatchfile
%</install>
%<*ignore>
\fi
%</ignore>
%<*driver>
\NeedsTeXFormat{LaTeX2e}
\ProvidesFile{stampinclude.drv}%
  [2016/05/16 v1.1 Include files based on time stamps (HO)]%
\documentclass{ltxdoc}
\usepackage{holtxdoc}[2011/11/22]
\begin{document}
  \DocInput{stampinclude.dtx}%
\end{document}
%</driver>
% \fi
%
%
% \CharacterTable
%  {Upper-case    \A\B\C\D\E\F\G\H\I\J\K\L\M\N\O\P\Q\R\S\T\U\V\W\X\Y\Z
%   Lower-case    \a\b\c\d\e\f\g\h\i\j\k\l\m\n\o\p\q\r\s\t\u\v\w\x\y\z
%   Digits        \0\1\2\3\4\5\6\7\8\9
%   Exclamation   \!     Double quote  \"     Hash (number) \#
%   Dollar        \$     Percent       \%     Ampersand     \&
%   Acute accent  \'     Left paren    \(     Right paren   \)
%   Asterisk      \*     Plus          \+     Comma         \,
%   Minus         \-     Point         \.     Solidus       \/
%   Colon         \:     Semicolon     \;     Less than     \<
%   Equals        \=     Greater than  \>     Question mark \?
%   Commercial at \@     Left bracket  \[     Backslash     \\
%   Right bracket \]     Circumflex    \^     Underscore    \_
%   Grave accent  \`     Left brace    \{     Vertical bar  \|
%   Right brace   \}     Tilde         \~}
%
% \GetFileInfo{stampinclude.drv}
%
% \title{The \xpackage{stampinclude} package}
% \date{2016/05/16 v1.1}
% \author{Heiko Oberdiek\thanks
% {Please report any issues at https://github.com/ho-tex/oberdiek/issues}\\
% \xemail{heiko.oberdiek at googlemail.com}}
%
% \maketitle
%
% \begin{abstract}
% The package replaces \cs{includeonly} and selects the files for
% \cs{include} by inspecting the time stamp of the \xext{aux} file.
% The file is selected for inclusion if the \xext{aux} file does
% not yet exist or is older than the corresponding \xext{tex} file.
% \end{abstract}
%
% \tableofcontents
%
% \section{Documentation}
%
% \subsection{Introduction}
% \label{sec:intro}
%
% \LaTeX\ provides two commands \cs{include} and \cs{includeonly}
% that helps in organizing large projects.
% Example for a master file:
%\begin{quote}
%\begin{verbatim}
%\documentclass{book}
%  % \includeonly{}
%\begin{document}
% \include{fileA}
% \include{fileB}
% \include{fileC}
%\end{document}
%\end{verbatim}
%\end{quote}
% All files are read and compiled if \cs{includeonly} is not
% executed. Otherwise you can give \cs{includeonly} a list
% of files in the preamble, e.g.:
% \begin{quote}
%   |\includeonly{fileA,fileC}|
% \end{quote}
% Now only files \xfile{fileA.tex} and \xfile{fileC.tex} are read
% and compiled.
%
% If you change file \xfile{fileB.tex} and want to see only this
% file, then you must change the line with \cs{includeonly} to
% \begin{quote}
%   |\includeonly{fileB}|
% \end{quote}
% It is tedious to do this again and again, if different files
% are changed.
%
% Package \xpackage{askinclude} \cite{askinclude}
% offers a solution for this problem. It interactively asks
% for the files to be included and saves the user from
% editing the master file.
%
% This package \xpackage{stampinclude} goes another way.
% \LaTeX\ reads and writes a separate \xext{aux} file for each
% file that is included by \cs{include}. There \LaTeX\ remembers
% counter valuses. Changed \xext{tex}
% files can therefore be detected by comparing the file date stamp of
% the \xext{tex} file with the date stamp of its \xext{aux} file.
% Since version 1.30.0 \pdfTeX\ provides \cs{pdffilemoddate}
% that reads the file date stamp. Thus this package uses this
% command and redefines
% \cs{include} to include the files that do not have \xext{aux}
% files yet or that are newer than its \xext{aux} file.
% \cs{includeonly} is ignored.
%
% \subsection{Usage}
%
% The package is loaded as normal \LaTeX\ package without options:
% \begin{quote}
%   |\usepackage{stampinclude}|
% \end{quote}
% Alternatively the package may be loaded on the command line
% (Example for shell `bash'):
% \begin{center}
%   |latex '\AtBeginDocument{\usepackage{stampinclude}}\input{master}'|
% \end{center}
% Without \cs{AtBeginDocument} (and \cs{RequirePackage} instead of
% \cs{usepackage}) \TeX\ would name the document \xfile{stampinclude.dvi}
% instead of \xfile{master.dvi}.
%
% \subsection{Limitations}
%
% \subsubsection{Other file dependencies}
%
% A file that is included by \cs{include} may input ore reference
% other files:
% \begin{itemize}
% \item other \TeX\ files using \cs{input},
% \item graphics files (\cs{includegraphics}),
% \item listings of external files,
% \item ...
% \end{itemize}
% Updates of those files are not detected by this package.
% It limits the date stamp comparison of an \xext{aux} file
% to its \xext{tex} file.
%
% \subsubsection{\cs{include} dependencies}
%
% In the example, given in the introduction \ref{sec:intro},
% three files \xfile{fileA}, \xfile{fileB}, and \xfile{fileC}
% are included in this order. Now file \xfile{fileA} is changed by adding
% four pages, \xfile{fileB} remains untouched, and \xfile{fileC} is
% also updated. Then the package only selects \xfile{fileA} and
% \xfile{fileC} for inclusion. File \xfile{fileB} is not included.
% But \LaTeX\ has stored the counter values that are active
% at the end of \xfile{fileB} in \xfile{fileB.aux} in one of the
% previous runs when \xfile{fileB} was included.
% However the later addition of four pages in \xfile{fileA}
% was not known at that time. Therefore \xfile{fileB.aux}
% is out of date and the inclusion of file \xfile{fileC}
% starts with wrong counter values (especially the page counter).
%
% \subsubsection{Summary}
%
% This package \xpackage{stampinclude} and the \cs{include} feature
% helps in accelerating the \LaTeX\ compilation.
% But it is not intended for generating the final version.
% For the final version of the document it is better to include
% \emph{all} files to get all counter values right.
% Then this package and any \cs{includeonly} lines should be commented out:
%\begin{quote}
%  |% \usepackage{stampinclude}|\\
%  |% \includeonly{...}|
%\end{quote}
%
% \subsection{Requirements}
%
% \begin{itemize}
% \item \pdfTeX\ v1.30.0 (because of \cs{pdffilemoddate}
%   and \cs{pdfstrcmp}),\\
%   both modes for DVI and PDF are supported.
% \item Alternatively Lua\TeX\ may be used.
%   It lacks \cs{pdffilemoddate} and \cs{pdfstrcmp}. But its services
%   are provided by package \xpackage{pdftexcmds} \cite{pdftexcmds}
%   that is automatically loaded.
% \end{itemize}
%
% \StopEventually{
% }
%
% \section{Implementation}
%
%    \begin{macrocode}
%<*package>
\NeedsTeXFormat{LaTeX2e}
\ProvidesPackage{stampinclude}
  [2016/05/16 v1.1 Include files based on time stamps (HO)]%
%    \end{macrocode}
%
%    \begin{macrocode}
\RequirePackage{pdftexcmds}[2007/12/12]%
%    \end{macrocode}
%
%    \begin{macrocode}
\begingroup
  \chardef\x=1 %
  \expandafter\ifx\csname pdf@filemoddate\endcsname\relax
    \chardef\x=0 %
  \fi
  \expandafter\ifx\csname pdf@strcmp\endcsname\relax
    \chardef\x=0 %
  \fi
\expandafter\endgroup\ifcase\x
  \PackageWarningNoLine{stampinclude}{%
    \string\pdffilemoddate\space or %
    \string\pdfstrcmp\space are not found,\MessageBreak
    that are provided by pdfTeX >= 1.30.0.\MessageBreak
    Also LuaTeX is not detected.\MessageBreak
    Therefore package loading is aborted%
  }%
  \expandafter\endinput
\fi
%    \end{macrocode}
%
%    \begin{macro}{\SInc@org@include}
%    \begin{macrocode}
\let\SInc@org@include\@include
%    \end{macrocode}
%    \end{macro}
%    \begin{macro}{\@include}
%    \begin{macrocode}
\def\@include#1 {%
  \IfFileExists{#1.aux}{%
    \ifnum\pdf@strcmp{\pdf@filemoddate{#1.aux}}%
                     {\pdf@filemoddate{#1.tex}}<0 %
      \ifx\@partlist\@empty
        \gdef\@partlist{{#1}}%
      \else
        \g@addto@macro\@partlist{,{#1}}%
      \fi
    \fi
  }{%
    \ifx\@partlist\@empty
      \gdef\@partlist{{#1}}%
    \else
      \g@addto@macro\@partlist{,{#1}}%
    \fi
  }%
  \SInc@org@include{#1} \relax
}
%    \end{macrocode}
%    \end{macro}
%
%    \begin{macro}{\includeonly}
%    Macro \cs{includeonly} is ignored.
%    \begin{macrocode}
\renewcommand*{\includeonly}[1]{%
  \PackageInfo{stampinclude}{%
    Ignoring \string\includeonly
  }%
}
%    \end{macrocode}
%    \end{macro}
%
%    Simulate \cs{includeonly}.
%    \begin{macrocode}
\@partswtrue
\gdef\@partlist{}
%    \end{macrocode}
%
%    Print included files at end of document.
%    \begin{macrocode}
\AtEndDocument{%
  \begingroup
    \expandafter\let\expandafter\@partlist\expandafter\@empty
    \expandafter\@for\expandafter\reserved@a
    \expandafter:\expandafter=\@partlist\do{%
      \ifx\@partlist\@empty
        \edef\@partlist{\reserved@a}%
      \else
        \edef\@partlist{\@partlist, \reserved@a}%
      \fi
    }%
    \typeout{********************%
             ********************%
             ********************%
             ******************%
    }%
    \ifx\@partlist\@empty
      \typeout{[stampinclude] No included files.}%
    \else
      \typeout{[stampinclude] Included files:}%
      \typeout{\@partlist}%
    \fi
    \typeout{********************%
             ********************%
             ********************%
             ******************%
    }%
  \endgroup
}
%    \end{macrocode}
%
%    \begin{macrocode}
%</package>
%    \end{macrocode}
%
% \section{Installation}
%
% \subsection{Download}
%
% \paragraph{Package.} This package is available on
% CTAN\footnote{\url{http://ctan.org/pkg/stampinclude}}:
% \begin{description}
% \item[\CTAN{macros/latex/contrib/oberdiek/stampinclude.dtx}] The source file.
% \item[\CTAN{macros/latex/contrib/oberdiek/stampinclude.pdf}] Documentation.
% \end{description}
%
%
% \paragraph{Bundle.} All the packages of the bundle `oberdiek'
% are also available in a TDS compliant ZIP archive. There
% the packages are already unpacked and the documentation files
% are generated. The files and directories obey the TDS standard.
% \begin{description}
% \item[\CTAN{install/macros/latex/contrib/oberdiek.tds.zip}]
% \end{description}
% \emph{TDS} refers to the standard ``A Directory Structure
% for \TeX\ Files'' (\CTAN{tds/tds.pdf}). Directories
% with \xfile{texmf} in their name are usually organized this way.
%
% \subsection{Bundle installation}
%
% \paragraph{Unpacking.} Unpack the \xfile{oberdiek.tds.zip} in the
% TDS tree (also known as \xfile{texmf} tree) of your choice.
% Example (linux):
% \begin{quote}
%   |unzip oberdiek.tds.zip -d ~/texmf|
% \end{quote}
%
% \paragraph{Script installation.}
% Check the directory \xfile{TDS:scripts/oberdiek/} for
% scripts that need further installation steps.
% Package \xpackage{attachfile2} comes with the Perl script
% \xfile{pdfatfi.pl} that should be installed in such a way
% that it can be called as \texttt{pdfatfi}.
% Example (linux):
% \begin{quote}
%   |chmod +x scripts/oberdiek/pdfatfi.pl|\\
%   |cp scripts/oberdiek/pdfatfi.pl /usr/local/bin/|
% \end{quote}
%
% \subsection{Package installation}
%
% \paragraph{Unpacking.} The \xfile{.dtx} file is a self-extracting
% \docstrip\ archive. The files are extracted by running the
% \xfile{.dtx} through \plainTeX:
% \begin{quote}
%   \verb|tex stampinclude.dtx|
% \end{quote}
%
% \paragraph{TDS.} Now the different files must be moved into
% the different directories in your installation TDS tree
% (also known as \xfile{texmf} tree):
% \begin{quote}
% \def\t{^^A
% \begin{tabular}{@{}>{\ttfamily}l@{ $\rightarrow$ }>{\ttfamily}l@{}}
%   stampinclude.sty & tex/latex/oberdiek/stampinclude.sty\\
%   stampinclude.pdf & doc/latex/oberdiek/stampinclude.pdf\\
%   stampinclude.dtx & source/latex/oberdiek/stampinclude.dtx\\
% \end{tabular}^^A
% }^^A
% \sbox0{\t}^^A
% \ifdim\wd0>\linewidth
%   \begingroup
%     \advance\linewidth by\leftmargin
%     \advance\linewidth by\rightmargin
%   \edef\x{\endgroup
%     \def\noexpand\lw{\the\linewidth}^^A
%   }\x
%   \def\lwbox{^^A
%     \leavevmode
%     \hbox to \linewidth{^^A
%       \kern-\leftmargin\relax
%       \hss
%       \usebox0
%       \hss
%       \kern-\rightmargin\relax
%     }^^A
%   }^^A
%   \ifdim\wd0>\lw
%     \sbox0{\small\t}^^A
%     \ifdim\wd0>\linewidth
%       \ifdim\wd0>\lw
%         \sbox0{\footnotesize\t}^^A
%         \ifdim\wd0>\linewidth
%           \ifdim\wd0>\lw
%             \sbox0{\scriptsize\t}^^A
%             \ifdim\wd0>\linewidth
%               \ifdim\wd0>\lw
%                 \sbox0{\tiny\t}^^A
%                 \ifdim\wd0>\linewidth
%                   \lwbox
%                 \else
%                   \usebox0
%                 \fi
%               \else
%                 \lwbox
%               \fi
%             \else
%               \usebox0
%             \fi
%           \else
%             \lwbox
%           \fi
%         \else
%           \usebox0
%         \fi
%       \else
%         \lwbox
%       \fi
%     \else
%       \usebox0
%     \fi
%   \else
%     \lwbox
%   \fi
% \else
%   \usebox0
% \fi
% \end{quote}
% If you have a \xfile{docstrip.cfg} that configures and enables \docstrip's
% TDS installing feature, then some files can already be in the right
% place, see the documentation of \docstrip.
%
% \subsection{Refresh file name databases}
%
% If your \TeX~distribution
% (\teTeX, \mikTeX, \dots) relies on file name databases, you must refresh
% these. For example, \teTeX\ users run \verb|texhash| or
% \verb|mktexlsr|.
%
% \subsection{Some details for the interested}
%
% \paragraph{Attached source.}
%
% The PDF documentation on CTAN also includes the
% \xfile{.dtx} source file. It can be extracted by
% AcrobatReader 6 or higher. Another option is \textsf{pdftk},
% e.g. unpack the file into the current directory:
% \begin{quote}
%   \verb|pdftk stampinclude.pdf unpack_files output .|
% \end{quote}
%
% \paragraph{Unpacking with \LaTeX.}
% The \xfile{.dtx} chooses its action depending on the format:
% \begin{description}
% \item[\plainTeX:] Run \docstrip\ and extract the files.
% \item[\LaTeX:] Generate the documentation.
% \end{description}
% If you insist on using \LaTeX\ for \docstrip\ (really,
% \docstrip\ does not need \LaTeX), then inform the autodetect routine
% about your intention:
% \begin{quote}
%   \verb|latex \let\install=y\input{stampinclude.dtx}|
% \end{quote}
% Do not forget to quote the argument according to the demands
% of your shell.
%
% \paragraph{Generating the documentation.}
% You can use both the \xfile{.dtx} or the \xfile{.drv} to generate
% the documentation. The process can be configured by the
% configuration file \xfile{ltxdoc.cfg}. For instance, put this
% line into this file, if you want to have A4 as paper format:
% \begin{quote}
%   \verb|\PassOptionsToClass{a4paper}{article}|
% \end{quote}
% An example follows how to generate the
% documentation with pdf\LaTeX:
% \begin{quote}
%\begin{verbatim}
%pdflatex stampinclude.dtx
%makeindex -s gind.ist stampinclude.idx
%pdflatex stampinclude.dtx
%makeindex -s gind.ist stampinclude.idx
%pdflatex stampinclude.dtx
%\end{verbatim}
% \end{quote}
%
% \section{Catalogue}
%
% The following XML file can be used as source for the
% \href{http://mirror.ctan.org/help/Catalogue/catalogue.html}{\TeX\ Catalogue}.
% The elements \texttt{caption} and \texttt{description} are imported
% from the original XML file from the Catalogue.
% The name of the XML file in the Catalogue is \xfile{stampinclude.xml}.
%    \begin{macrocode}
%<*catalogue>
<?xml version='1.0' encoding='us-ascii'?>
<!DOCTYPE entry SYSTEM 'catalogue.dtd'>
<entry datestamp='$Date$' modifier='$Author$' id='stampinclude'>
  <name>stampinclude</name>
  <caption>Inclusion based on .aux file date stamps.</caption>
  <authorref id='auth:oberdiek'/>
  <copyright owner='Heiko Oberdiek' year='2008'/>
  <license type='lppl1.3'/>
  <version number='1.1'/>
  <description>
    This package replaces <tt>\includeonly</tt> and selects the files for
    <tt>\include</tt> by inspecting the timestamp of the <tt>.aux</tt> file.
    The file is selected for inclusion if the <tt>.aux</tt> file does
    not yet exist or is older than the corresponding <tt>.tex</tt> file.
    <p/>
    The package is part of the <xref refid='oberdiek'>oberdiek</xref>
    bundle.
  </description>
  <documentation details='Package documentation'
      href='ctan:/macros/latex/contrib/oberdiek/stampinclude.pdf'/>
  <ctan file='true' path='/macros/latex/contrib/oberdiek/stampinclude.dtx'/>
  <miktex location='oberdiek'/>
  <texlive location='oberdiek'/>
  <install path='/macros/latex/contrib/oberdiek/oberdiek.tds.zip'/>
</entry>
%</catalogue>
%    \end{macrocode}
%
% \begin{thebibliography}{9}
% \bibitem{askinclude}
%   Pablo A. Straub, Heiko Oberdiek:
%   \textit{The \xpackage{askinclude} package};
%   2007/10/23 v2.0;
%   \CTAN{macros/latex/contrib/oberdiek/askinclude.pdf}.
%
% \bibitem{pdftexcmds}
%   Heiko Oberdiek:
%   \textit{The \xpackage{pdftexcmds} package};
%   2007/12/12 v0.3;
%   \CTAN{macros/latex/contrib/oberdiek/pdftexcmds.pdf}.
%
% \end{thebibliography}
%
% \begin{History}
%   \begin{Version}{2008/07/14 v1.0}
%   \item
%     First version.
%   \end{Version}
%   \begin{Version}{2016/05/16 v1.1}
%   \item
%     Documentation updates.
%   \end{Version}
% \end{History}
%
% \PrintIndex
%
% \Finale
\endinput
|
% \end{quote}
% Do not forget to quote the argument according to the demands
% of your shell.
%
% \paragraph{Generating the documentation.}
% You can use both the \xfile{.dtx} or the \xfile{.drv} to generate
% the documentation. The process can be configured by the
% configuration file \xfile{ltxdoc.cfg}. For instance, put this
% line into this file, if you want to have A4 as paper format:
% \begin{quote}
%   \verb|\PassOptionsToClass{a4paper}{article}|
% \end{quote}
% An example follows how to generate the
% documentation with pdf\LaTeX:
% \begin{quote}
%\begin{verbatim}
%pdflatex stampinclude.dtx
%makeindex -s gind.ist stampinclude.idx
%pdflatex stampinclude.dtx
%makeindex -s gind.ist stampinclude.idx
%pdflatex stampinclude.dtx
%\end{verbatim}
% \end{quote}
%
% \section{Catalogue}
%
% The following XML file can be used as source for the
% \href{http://mirror.ctan.org/help/Catalogue/catalogue.html}{\TeX\ Catalogue}.
% The elements \texttt{caption} and \texttt{description} are imported
% from the original XML file from the Catalogue.
% The name of the XML file in the Catalogue is \xfile{stampinclude.xml}.
%    \begin{macrocode}
%<*catalogue>
<?xml version='1.0' encoding='us-ascii'?>
<!DOCTYPE entry SYSTEM 'catalogue.dtd'>
<entry datestamp='$Date$' modifier='$Author$' id='stampinclude'>
  <name>stampinclude</name>
  <caption>Inclusion based on .aux file date stamps.</caption>
  <authorref id='auth:oberdiek'/>
  <copyright owner='Heiko Oberdiek' year='2008'/>
  <license type='lppl1.3'/>
  <version number='1.1'/>
  <description>
    This package replaces <tt>\includeonly</tt> and selects the files for
    <tt>\include</tt> by inspecting the timestamp of the <tt>.aux</tt> file.
    The file is selected for inclusion if the <tt>.aux</tt> file does
    not yet exist or is older than the corresponding <tt>.tex</tt> file.
    <p/>
    The package is part of the <xref refid='oberdiek'>oberdiek</xref>
    bundle.
  </description>
  <documentation details='Package documentation'
      href='ctan:/macros/latex/contrib/oberdiek/stampinclude.pdf'/>
  <ctan file='true' path='/macros/latex/contrib/oberdiek/stampinclude.dtx'/>
  <miktex location='oberdiek'/>
  <texlive location='oberdiek'/>
  <install path='/macros/latex/contrib/oberdiek/oberdiek.tds.zip'/>
</entry>
%</catalogue>
%    \end{macrocode}
%
% \begin{thebibliography}{9}
% \bibitem{askinclude}
%   Pablo A. Straub, Heiko Oberdiek:
%   \textit{The \xpackage{askinclude} package};
%   2007/10/23 v2.0;
%   \CTAN{macros/latex/contrib/oberdiek/askinclude.pdf}.
%
% \bibitem{pdftexcmds}
%   Heiko Oberdiek:
%   \textit{The \xpackage{pdftexcmds} package};
%   2007/12/12 v0.3;
%   \CTAN{macros/latex/contrib/oberdiek/pdftexcmds.pdf}.
%
% \end{thebibliography}
%
% \begin{History}
%   \begin{Version}{2008/07/14 v1.0}
%   \item
%     First version.
%   \end{Version}
%   \begin{Version}{2016/05/16 v1.1}
%   \item
%     Documentation updates.
%   \end{Version}
% \end{History}
%
% \PrintIndex
%
% \Finale
\endinput

%        (quote the arguments according to the demands of your shell)
%
% Documentation:
%    (a) If stampinclude.drv is present:
%           latex stampinclude.drv
%    (b) Without stampinclude.drv:
%           latex stampinclude.dtx; ...
%    The class ltxdoc loads the configuration file ltxdoc.cfg
%    if available. Here you can specify further options, e.g.
%    use A4 as paper format:
%       \PassOptionsToClass{a4paper}{article}
%
%    Programm calls to get the documentation (example):
%       pdflatex stampinclude.dtx
%       makeindex -s gind.ist stampinclude.idx
%       pdflatex stampinclude.dtx
%       makeindex -s gind.ist stampinclude.idx
%       pdflatex stampinclude.dtx
%
% Installation:
%    TDS:tex/latex/oberdiek/stampinclude.sty
%    TDS:doc/latex/oberdiek/stampinclude.pdf
%    TDS:source/latex/oberdiek/stampinclude.dtx
%
%<*ignore>
\begingroup
  \catcode123=1 %
  \catcode125=2 %
  \def\x{LaTeX2e}%
\expandafter\endgroup
\ifcase 0\ifx\install y1\fi\expandafter
         \ifx\csname processbatchFile\endcsname\relax\else1\fi
         \ifx\fmtname\x\else 1\fi\relax
\else\csname fi\endcsname
%</ignore>
%<*install>
\input docstrip.tex
\Msg{************************************************************************}
\Msg{* Installation}
\Msg{* Package: stampinclude 2016/05/16 v1.1 Include files based on time stamps (HO)}
\Msg{************************************************************************}

\keepsilent
\askforoverwritefalse

\let\MetaPrefix\relax
\preamble

This is a generated file.

Project: stampinclude
Version: 2016/05/16 v1.1

Copyright (C) 2008 by
   Heiko Oberdiek <heiko.oberdiek at googlemail.com>

This work may be distributed and/or modified under the
conditions of the LaTeX Project Public License, either
version 1.3c of this license or (at your option) any later
version. This version of this license is in
   http://www.latex-project.org/lppl/lppl-1-3c.txt
and the latest version of this license is in
   http://www.latex-project.org/lppl.txt
and version 1.3 or later is part of all distributions of
LaTeX version 2005/12/01 or later.

This work has the LPPL maintenance status "maintained".

This Current Maintainer of this work is Heiko Oberdiek.

This work consists of the main source file stampinclude.dtx
and the derived files
   stampinclude.sty, stampinclude.pdf, stampinclude.ins, stampinclude.drv.

\endpreamble
\let\MetaPrefix\DoubleperCent

\generate{%
  \file{stampinclude.ins}{\from{stampinclude.dtx}{install}}%
  \file{stampinclude.drv}{\from{stampinclude.dtx}{driver}}%
  \usedir{tex/latex/oberdiek}%
  \file{stampinclude.sty}{\from{stampinclude.dtx}{package}}%
  \nopreamble
  \nopostamble
  \usedir{source/latex/oberdiek/catalogue}%
  \file{stampinclude.xml}{\from{stampinclude.dtx}{catalogue}}%
}

\catcode32=13\relax% active space
\let =\space%
\Msg{************************************************************************}
\Msg{*}
\Msg{* To finish the installation you have to move the following}
\Msg{* file into a directory searched by TeX:}
\Msg{*}
\Msg{*     stampinclude.sty}
\Msg{*}
\Msg{* To produce the documentation run the file `stampinclude.drv'}
\Msg{* through LaTeX.}
\Msg{*}
\Msg{* Happy TeXing!}
\Msg{*}
\Msg{************************************************************************}

\endbatchfile
%</install>
%<*ignore>
\fi
%</ignore>
%<*driver>
\NeedsTeXFormat{LaTeX2e}
\ProvidesFile{stampinclude.drv}%
  [2016/05/16 v1.1 Include files based on time stamps (HO)]%
\documentclass{ltxdoc}
\usepackage{holtxdoc}[2011/11/22]
\begin{document}
  \DocInput{stampinclude.dtx}%
\end{document}
%</driver>
% \fi
%
%
% \CharacterTable
%  {Upper-case    \A\B\C\D\E\F\G\H\I\J\K\L\M\N\O\P\Q\R\S\T\U\V\W\X\Y\Z
%   Lower-case    \a\b\c\d\e\f\g\h\i\j\k\l\m\n\o\p\q\r\s\t\u\v\w\x\y\z
%   Digits        \0\1\2\3\4\5\6\7\8\9
%   Exclamation   \!     Double quote  \"     Hash (number) \#
%   Dollar        \$     Percent       \%     Ampersand     \&
%   Acute accent  \'     Left paren    \(     Right paren   \)
%   Asterisk      \*     Plus          \+     Comma         \,
%   Minus         \-     Point         \.     Solidus       \/
%   Colon         \:     Semicolon     \;     Less than     \<
%   Equals        \=     Greater than  \>     Question mark \?
%   Commercial at \@     Left bracket  \[     Backslash     \\
%   Right bracket \]     Circumflex    \^     Underscore    \_
%   Grave accent  \`     Left brace    \{     Vertical bar  \|
%   Right brace   \}     Tilde         \~}
%
% \GetFileInfo{stampinclude.drv}
%
% \title{The \xpackage{stampinclude} package}
% \date{2016/05/16 v1.1}
% \author{Heiko Oberdiek\thanks
% {Please report any issues at https://github.com/ho-tex/oberdiek/issues}\\
% \xemail{heiko.oberdiek at googlemail.com}}
%
% \maketitle
%
% \begin{abstract}
% The package replaces \cs{includeonly} and selects the files for
% \cs{include} by inspecting the time stamp of the \xext{aux} file.
% The file is selected for inclusion if the \xext{aux} file does
% not yet exist or is older than the corresponding \xext{tex} file.
% \end{abstract}
%
% \tableofcontents
%
% \section{Documentation}
%
% \subsection{Introduction}
% \label{sec:intro}
%
% \LaTeX\ provides two commands \cs{include} and \cs{includeonly}
% that helps in organizing large projects.
% Example for a master file:
%\begin{quote}
%\begin{verbatim}
%\documentclass{book}
%  % \includeonly{}
%\begin{document}
% \include{fileA}
% \include{fileB}
% \include{fileC}
%\end{document}
%\end{verbatim}
%\end{quote}
% All files are read and compiled if \cs{includeonly} is not
% executed. Otherwise you can give \cs{includeonly} a list
% of files in the preamble, e.g.:
% \begin{quote}
%   |\includeonly{fileA,fileC}|
% \end{quote}
% Now only files \xfile{fileA.tex} and \xfile{fileC.tex} are read
% and compiled.
%
% If you change file \xfile{fileB.tex} and want to see only this
% file, then you must change the line with \cs{includeonly} to
% \begin{quote}
%   |\includeonly{fileB}|
% \end{quote}
% It is tedious to do this again and again, if different files
% are changed.
%
% Package \xpackage{askinclude} \cite{askinclude}
% offers a solution for this problem. It interactively asks
% for the files to be included and saves the user from
% editing the master file.
%
% This package \xpackage{stampinclude} goes another way.
% \LaTeX\ reads and writes a separate \xext{aux} file for each
% file that is included by \cs{include}. There \LaTeX\ remembers
% counter valuses. Changed \xext{tex}
% files can therefore be detected by comparing the file date stamp of
% the \xext{tex} file with the date stamp of its \xext{aux} file.
% Since version 1.30.0 \pdfTeX\ provides \cs{pdffilemoddate}
% that reads the file date stamp. Thus this package uses this
% command and redefines
% \cs{include} to include the files that do not have \xext{aux}
% files yet or that are newer than its \xext{aux} file.
% \cs{includeonly} is ignored.
%
% \subsection{Usage}
%
% The package is loaded as normal \LaTeX\ package without options:
% \begin{quote}
%   |\usepackage{stampinclude}|
% \end{quote}
% Alternatively the package may be loaded on the command line
% (Example for shell `bash'):
% \begin{center}
%   |latex '\AtBeginDocument{\usepackage{stampinclude}}\input{master}'|
% \end{center}
% Without \cs{AtBeginDocument} (and \cs{RequirePackage} instead of
% \cs{usepackage}) \TeX\ would name the document \xfile{stampinclude.dvi}
% instead of \xfile{master.dvi}.
%
% \subsection{Limitations}
%
% \subsubsection{Other file dependencies}
%
% A file that is included by \cs{include} may input ore reference
% other files:
% \begin{itemize}
% \item other \TeX\ files using \cs{input},
% \item graphics files (\cs{includegraphics}),
% \item listings of external files,
% \item ...
% \end{itemize}
% Updates of those files are not detected by this package.
% It limits the date stamp comparison of an \xext{aux} file
% to its \xext{tex} file.
%
% \subsubsection{\cs{include} dependencies}
%
% In the example, given in the introduction \ref{sec:intro},
% three files \xfile{fileA}, \xfile{fileB}, and \xfile{fileC}
% are included in this order. Now file \xfile{fileA} is changed by adding
% four pages, \xfile{fileB} remains untouched, and \xfile{fileC} is
% also updated. Then the package only selects \xfile{fileA} and
% \xfile{fileC} for inclusion. File \xfile{fileB} is not included.
% But \LaTeX\ has stored the counter values that are active
% at the end of \xfile{fileB} in \xfile{fileB.aux} in one of the
% previous runs when \xfile{fileB} was included.
% However the later addition of four pages in \xfile{fileA}
% was not known at that time. Therefore \xfile{fileB.aux}
% is out of date and the inclusion of file \xfile{fileC}
% starts with wrong counter values (especially the page counter).
%
% \subsubsection{Summary}
%
% This package \xpackage{stampinclude} and the \cs{include} feature
% helps in accelerating the \LaTeX\ compilation.
% But it is not intended for generating the final version.
% For the final version of the document it is better to include
% \emph{all} files to get all counter values right.
% Then this package and any \cs{includeonly} lines should be commented out:
%\begin{quote}
%  |% \usepackage{stampinclude}|\\
%  |% \includeonly{...}|
%\end{quote}
%
% \subsection{Requirements}
%
% \begin{itemize}
% \item \pdfTeX\ v1.30.0 (because of \cs{pdffilemoddate}
%   and \cs{pdfstrcmp}),\\
%   both modes for DVI and PDF are supported.
% \item Alternatively Lua\TeX\ may be used.
%   It lacks \cs{pdffilemoddate} and \cs{pdfstrcmp}. But its services
%   are provided by package \xpackage{pdftexcmds} \cite{pdftexcmds}
%   that is automatically loaded.
% \end{itemize}
%
% \StopEventually{
% }
%
% \section{Implementation}
%
%    \begin{macrocode}
%<*package>
\NeedsTeXFormat{LaTeX2e}
\ProvidesPackage{stampinclude}
  [2016/05/16 v1.1 Include files based on time stamps (HO)]%
%    \end{macrocode}
%
%    \begin{macrocode}
\RequirePackage{pdftexcmds}[2007/12/12]%
%    \end{macrocode}
%
%    \begin{macrocode}
\begingroup
  \chardef\x=1 %
  \expandafter\ifx\csname pdf@filemoddate\endcsname\relax
    \chardef\x=0 %
  \fi
  \expandafter\ifx\csname pdf@strcmp\endcsname\relax
    \chardef\x=0 %
  \fi
\expandafter\endgroup\ifcase\x
  \PackageWarningNoLine{stampinclude}{%
    \string\pdffilemoddate\space or %
    \string\pdfstrcmp\space are not found,\MessageBreak
    that are provided by pdfTeX >= 1.30.0.\MessageBreak
    Also LuaTeX is not detected.\MessageBreak
    Therefore package loading is aborted%
  }%
  \expandafter\endinput
\fi
%    \end{macrocode}
%
%    \begin{macro}{\SInc@org@include}
%    \begin{macrocode}
\let\SInc@org@include\@include
%    \end{macrocode}
%    \end{macro}
%    \begin{macro}{\@include}
%    \begin{macrocode}
\def\@include#1 {%
  \IfFileExists{#1.aux}{%
    \ifnum\pdf@strcmp{\pdf@filemoddate{#1.aux}}%
                     {\pdf@filemoddate{#1.tex}}<0 %
      \ifx\@partlist\@empty
        \gdef\@partlist{{#1}}%
      \else
        \g@addto@macro\@partlist{,{#1}}%
      \fi
    \fi
  }{%
    \ifx\@partlist\@empty
      \gdef\@partlist{{#1}}%
    \else
      \g@addto@macro\@partlist{,{#1}}%
    \fi
  }%
  \SInc@org@include{#1} \relax
}
%    \end{macrocode}
%    \end{macro}
%
%    \begin{macro}{\includeonly}
%    Macro \cs{includeonly} is ignored.
%    \begin{macrocode}
\renewcommand*{\includeonly}[1]{%
  \PackageInfo{stampinclude}{%
    Ignoring \string\includeonly
  }%
}
%    \end{macrocode}
%    \end{macro}
%
%    Simulate \cs{includeonly}.
%    \begin{macrocode}
\@partswtrue
\gdef\@partlist{}
%    \end{macrocode}
%
%    Print included files at end of document.
%    \begin{macrocode}
\AtEndDocument{%
  \begingroup
    \expandafter\let\expandafter\@partlist\expandafter\@empty
    \expandafter\@for\expandafter\reserved@a
    \expandafter:\expandafter=\@partlist\do{%
      \ifx\@partlist\@empty
        \edef\@partlist{\reserved@a}%
      \else
        \edef\@partlist{\@partlist, \reserved@a}%
      \fi
    }%
    \typeout{********************%
             ********************%
             ********************%
             ******************%
    }%
    \ifx\@partlist\@empty
      \typeout{[stampinclude] No included files.}%
    \else
      \typeout{[stampinclude] Included files:}%
      \typeout{\@partlist}%
    \fi
    \typeout{********************%
             ********************%
             ********************%
             ******************%
    }%
  \endgroup
}
%    \end{macrocode}
%
%    \begin{macrocode}
%</package>
%    \end{macrocode}
%
% \section{Installation}
%
% \subsection{Download}
%
% \paragraph{Package.} This package is available on
% CTAN\footnote{\url{http://ctan.org/pkg/stampinclude}}:
% \begin{description}
% \item[\CTAN{macros/latex/contrib/oberdiek/stampinclude.dtx}] The source file.
% \item[\CTAN{macros/latex/contrib/oberdiek/stampinclude.pdf}] Documentation.
% \end{description}
%
%
% \paragraph{Bundle.} All the packages of the bundle `oberdiek'
% are also available in a TDS compliant ZIP archive. There
% the packages are already unpacked and the documentation files
% are generated. The files and directories obey the TDS standard.
% \begin{description}
% \item[\CTAN{install/macros/latex/contrib/oberdiek.tds.zip}]
% \end{description}
% \emph{TDS} refers to the standard ``A Directory Structure
% for \TeX\ Files'' (\CTAN{tds/tds.pdf}). Directories
% with \xfile{texmf} in their name are usually organized this way.
%
% \subsection{Bundle installation}
%
% \paragraph{Unpacking.} Unpack the \xfile{oberdiek.tds.zip} in the
% TDS tree (also known as \xfile{texmf} tree) of your choice.
% Example (linux):
% \begin{quote}
%   |unzip oberdiek.tds.zip -d ~/texmf|
% \end{quote}
%
% \paragraph{Script installation.}
% Check the directory \xfile{TDS:scripts/oberdiek/} for
% scripts that need further installation steps.
% Package \xpackage{attachfile2} comes with the Perl script
% \xfile{pdfatfi.pl} that should be installed in such a way
% that it can be called as \texttt{pdfatfi}.
% Example (linux):
% \begin{quote}
%   |chmod +x scripts/oberdiek/pdfatfi.pl|\\
%   |cp scripts/oberdiek/pdfatfi.pl /usr/local/bin/|
% \end{quote}
%
% \subsection{Package installation}
%
% \paragraph{Unpacking.} The \xfile{.dtx} file is a self-extracting
% \docstrip\ archive. The files are extracted by running the
% \xfile{.dtx} through \plainTeX:
% \begin{quote}
%   \verb|tex stampinclude.dtx|
% \end{quote}
%
% \paragraph{TDS.} Now the different files must be moved into
% the different directories in your installation TDS tree
% (also known as \xfile{texmf} tree):
% \begin{quote}
% \def\t{^^A
% \begin{tabular}{@{}>{\ttfamily}l@{ $\rightarrow$ }>{\ttfamily}l@{}}
%   stampinclude.sty & tex/latex/oberdiek/stampinclude.sty\\
%   stampinclude.pdf & doc/latex/oberdiek/stampinclude.pdf\\
%   stampinclude.dtx & source/latex/oberdiek/stampinclude.dtx\\
% \end{tabular}^^A
% }^^A
% \sbox0{\t}^^A
% \ifdim\wd0>\linewidth
%   \begingroup
%     \advance\linewidth by\leftmargin
%     \advance\linewidth by\rightmargin
%   \edef\x{\endgroup
%     \def\noexpand\lw{\the\linewidth}^^A
%   }\x
%   \def\lwbox{^^A
%     \leavevmode
%     \hbox to \linewidth{^^A
%       \kern-\leftmargin\relax
%       \hss
%       \usebox0
%       \hss
%       \kern-\rightmargin\relax
%     }^^A
%   }^^A
%   \ifdim\wd0>\lw
%     \sbox0{\small\t}^^A
%     \ifdim\wd0>\linewidth
%       \ifdim\wd0>\lw
%         \sbox0{\footnotesize\t}^^A
%         \ifdim\wd0>\linewidth
%           \ifdim\wd0>\lw
%             \sbox0{\scriptsize\t}^^A
%             \ifdim\wd0>\linewidth
%               \ifdim\wd0>\lw
%                 \sbox0{\tiny\t}^^A
%                 \ifdim\wd0>\linewidth
%                   \lwbox
%                 \else
%                   \usebox0
%                 \fi
%               \else
%                 \lwbox
%               \fi
%             \else
%               \usebox0
%             \fi
%           \else
%             \lwbox
%           \fi
%         \else
%           \usebox0
%         \fi
%       \else
%         \lwbox
%       \fi
%     \else
%       \usebox0
%     \fi
%   \else
%     \lwbox
%   \fi
% \else
%   \usebox0
% \fi
% \end{quote}
% If you have a \xfile{docstrip.cfg} that configures and enables \docstrip's
% TDS installing feature, then some files can already be in the right
% place, see the documentation of \docstrip.
%
% \subsection{Refresh file name databases}
%
% If your \TeX~distribution
% (\teTeX, \mikTeX, \dots) relies on file name databases, you must refresh
% these. For example, \teTeX\ users run \verb|texhash| or
% \verb|mktexlsr|.
%
% \subsection{Some details for the interested}
%
% \paragraph{Attached source.}
%
% The PDF documentation on CTAN also includes the
% \xfile{.dtx} source file. It can be extracted by
% AcrobatReader 6 or higher. Another option is \textsf{pdftk},
% e.g. unpack the file into the current directory:
% \begin{quote}
%   \verb|pdftk stampinclude.pdf unpack_files output .|
% \end{quote}
%
% \paragraph{Unpacking with \LaTeX.}
% The \xfile{.dtx} chooses its action depending on the format:
% \begin{description}
% \item[\plainTeX:] Run \docstrip\ and extract the files.
% \item[\LaTeX:] Generate the documentation.
% \end{description}
% If you insist on using \LaTeX\ for \docstrip\ (really,
% \docstrip\ does not need \LaTeX), then inform the autodetect routine
% about your intention:
% \begin{quote}
%   \verb|latex \let\install=y% \iffalse meta-comment
%
% File: stampinclude.dtx
% Version: 2016/05/16 v1.1
% Info: Include files based on time stamps
%
% Copyright (C) 2008 by
%    Heiko Oberdiek <heiko.oberdiek at googlemail.com>
%    2016
%    https://github.com/ho-tex/oberdiek/issues
%
% This work may be distributed and/or modified under the
% conditions of the LaTeX Project Public License, either
% version 1.3c of this license or (at your option) any later
% version. This version of this license is in
%    http://www.latex-project.org/lppl/lppl-1-3c.txt
% and the latest version of this license is in
%    http://www.latex-project.org/lppl.txt
% and version 1.3 or later is part of all distributions of
% LaTeX version 2005/12/01 or later.
%
% This work has the LPPL maintenance status "maintained".
%
% This Current Maintainer of this work is Heiko Oberdiek.
%
% This work consists of the main source file stampinclude.dtx
% and the derived files
%    stampinclude.sty, stampinclude.pdf, stampinclude.ins, stampinclude.drv.
%
% Distribution:
%    CTAN:macros/latex/contrib/oberdiek/stampinclude.dtx
%    CTAN:macros/latex/contrib/oberdiek/stampinclude.pdf
%
% Unpacking:
%    (a) If stampinclude.ins is present:
%           tex stampinclude.ins
%    (b) Without stampinclude.ins:
%           tex stampinclude.dtx
%    (c) If you insist on using LaTeX
%           latex \let\install=y% \iffalse meta-comment
%
% File: stampinclude.dtx
% Version: 2016/05/16 v1.1
% Info: Include files based on time stamps
%
% Copyright (C) 2008 by
%    Heiko Oberdiek <heiko.oberdiek at googlemail.com>
%    2016
%    https://github.com/ho-tex/oberdiek/issues
%
% This work may be distributed and/or modified under the
% conditions of the LaTeX Project Public License, either
% version 1.3c of this license or (at your option) any later
% version. This version of this license is in
%    http://www.latex-project.org/lppl/lppl-1-3c.txt
% and the latest version of this license is in
%    http://www.latex-project.org/lppl.txt
% and version 1.3 or later is part of all distributions of
% LaTeX version 2005/12/01 or later.
%
% This work has the LPPL maintenance status "maintained".
%
% This Current Maintainer of this work is Heiko Oberdiek.
%
% This work consists of the main source file stampinclude.dtx
% and the derived files
%    stampinclude.sty, stampinclude.pdf, stampinclude.ins, stampinclude.drv.
%
% Distribution:
%    CTAN:macros/latex/contrib/oberdiek/stampinclude.dtx
%    CTAN:macros/latex/contrib/oberdiek/stampinclude.pdf
%
% Unpacking:
%    (a) If stampinclude.ins is present:
%           tex stampinclude.ins
%    (b) Without stampinclude.ins:
%           tex stampinclude.dtx
%    (c) If you insist on using LaTeX
%           latex \let\install=y\input{stampinclude.dtx}
%        (quote the arguments according to the demands of your shell)
%
% Documentation:
%    (a) If stampinclude.drv is present:
%           latex stampinclude.drv
%    (b) Without stampinclude.drv:
%           latex stampinclude.dtx; ...
%    The class ltxdoc loads the configuration file ltxdoc.cfg
%    if available. Here you can specify further options, e.g.
%    use A4 as paper format:
%       \PassOptionsToClass{a4paper}{article}
%
%    Programm calls to get the documentation (example):
%       pdflatex stampinclude.dtx
%       makeindex -s gind.ist stampinclude.idx
%       pdflatex stampinclude.dtx
%       makeindex -s gind.ist stampinclude.idx
%       pdflatex stampinclude.dtx
%
% Installation:
%    TDS:tex/latex/oberdiek/stampinclude.sty
%    TDS:doc/latex/oberdiek/stampinclude.pdf
%    TDS:source/latex/oberdiek/stampinclude.dtx
%
%<*ignore>
\begingroup
  \catcode123=1 %
  \catcode125=2 %
  \def\x{LaTeX2e}%
\expandafter\endgroup
\ifcase 0\ifx\install y1\fi\expandafter
         \ifx\csname processbatchFile\endcsname\relax\else1\fi
         \ifx\fmtname\x\else 1\fi\relax
\else\csname fi\endcsname
%</ignore>
%<*install>
\input docstrip.tex
\Msg{************************************************************************}
\Msg{* Installation}
\Msg{* Package: stampinclude 2016/05/16 v1.1 Include files based on time stamps (HO)}
\Msg{************************************************************************}

\keepsilent
\askforoverwritefalse

\let\MetaPrefix\relax
\preamble

This is a generated file.

Project: stampinclude
Version: 2016/05/16 v1.1

Copyright (C) 2008 by
   Heiko Oberdiek <heiko.oberdiek at googlemail.com>

This work may be distributed and/or modified under the
conditions of the LaTeX Project Public License, either
version 1.3c of this license or (at your option) any later
version. This version of this license is in
   http://www.latex-project.org/lppl/lppl-1-3c.txt
and the latest version of this license is in
   http://www.latex-project.org/lppl.txt
and version 1.3 or later is part of all distributions of
LaTeX version 2005/12/01 or later.

This work has the LPPL maintenance status "maintained".

This Current Maintainer of this work is Heiko Oberdiek.

This work consists of the main source file stampinclude.dtx
and the derived files
   stampinclude.sty, stampinclude.pdf, stampinclude.ins, stampinclude.drv.

\endpreamble
\let\MetaPrefix\DoubleperCent

\generate{%
  \file{stampinclude.ins}{\from{stampinclude.dtx}{install}}%
  \file{stampinclude.drv}{\from{stampinclude.dtx}{driver}}%
  \usedir{tex/latex/oberdiek}%
  \file{stampinclude.sty}{\from{stampinclude.dtx}{package}}%
  \nopreamble
  \nopostamble
  \usedir{source/latex/oberdiek/catalogue}%
  \file{stampinclude.xml}{\from{stampinclude.dtx}{catalogue}}%
}

\catcode32=13\relax% active space
\let =\space%
\Msg{************************************************************************}
\Msg{*}
\Msg{* To finish the installation you have to move the following}
\Msg{* file into a directory searched by TeX:}
\Msg{*}
\Msg{*     stampinclude.sty}
\Msg{*}
\Msg{* To produce the documentation run the file `stampinclude.drv'}
\Msg{* through LaTeX.}
\Msg{*}
\Msg{* Happy TeXing!}
\Msg{*}
\Msg{************************************************************************}

\endbatchfile
%</install>
%<*ignore>
\fi
%</ignore>
%<*driver>
\NeedsTeXFormat{LaTeX2e}
\ProvidesFile{stampinclude.drv}%
  [2016/05/16 v1.1 Include files based on time stamps (HO)]%
\documentclass{ltxdoc}
\usepackage{holtxdoc}[2011/11/22]
\begin{document}
  \DocInput{stampinclude.dtx}%
\end{document}
%</driver>
% \fi
%
%
% \CharacterTable
%  {Upper-case    \A\B\C\D\E\F\G\H\I\J\K\L\M\N\O\P\Q\R\S\T\U\V\W\X\Y\Z
%   Lower-case    \a\b\c\d\e\f\g\h\i\j\k\l\m\n\o\p\q\r\s\t\u\v\w\x\y\z
%   Digits        \0\1\2\3\4\5\6\7\8\9
%   Exclamation   \!     Double quote  \"     Hash (number) \#
%   Dollar        \$     Percent       \%     Ampersand     \&
%   Acute accent  \'     Left paren    \(     Right paren   \)
%   Asterisk      \*     Plus          \+     Comma         \,
%   Minus         \-     Point         \.     Solidus       \/
%   Colon         \:     Semicolon     \;     Less than     \<
%   Equals        \=     Greater than  \>     Question mark \?
%   Commercial at \@     Left bracket  \[     Backslash     \\
%   Right bracket \]     Circumflex    \^     Underscore    \_
%   Grave accent  \`     Left brace    \{     Vertical bar  \|
%   Right brace   \}     Tilde         \~}
%
% \GetFileInfo{stampinclude.drv}
%
% \title{The \xpackage{stampinclude} package}
% \date{2016/05/16 v1.1}
% \author{Heiko Oberdiek\thanks
% {Please report any issues at https://github.com/ho-tex/oberdiek/issues}\\
% \xemail{heiko.oberdiek at googlemail.com}}
%
% \maketitle
%
% \begin{abstract}
% The package replaces \cs{includeonly} and selects the files for
% \cs{include} by inspecting the time stamp of the \xext{aux} file.
% The file is selected for inclusion if the \xext{aux} file does
% not yet exist or is older than the corresponding \xext{tex} file.
% \end{abstract}
%
% \tableofcontents
%
% \section{Documentation}
%
% \subsection{Introduction}
% \label{sec:intro}
%
% \LaTeX\ provides two commands \cs{include} and \cs{includeonly}
% that helps in organizing large projects.
% Example for a master file:
%\begin{quote}
%\begin{verbatim}
%\documentclass{book}
%  % \includeonly{}
%\begin{document}
% \include{fileA}
% \include{fileB}
% \include{fileC}
%\end{document}
%\end{verbatim}
%\end{quote}
% All files are read and compiled if \cs{includeonly} is not
% executed. Otherwise you can give \cs{includeonly} a list
% of files in the preamble, e.g.:
% \begin{quote}
%   |\includeonly{fileA,fileC}|
% \end{quote}
% Now only files \xfile{fileA.tex} and \xfile{fileC.tex} are read
% and compiled.
%
% If you change file \xfile{fileB.tex} and want to see only this
% file, then you must change the line with \cs{includeonly} to
% \begin{quote}
%   |\includeonly{fileB}|
% \end{quote}
% It is tedious to do this again and again, if different files
% are changed.
%
% Package \xpackage{askinclude} \cite{askinclude}
% offers a solution for this problem. It interactively asks
% for the files to be included and saves the user from
% editing the master file.
%
% This package \xpackage{stampinclude} goes another way.
% \LaTeX\ reads and writes a separate \xext{aux} file for each
% file that is included by \cs{include}. There \LaTeX\ remembers
% counter valuses. Changed \xext{tex}
% files can therefore be detected by comparing the file date stamp of
% the \xext{tex} file with the date stamp of its \xext{aux} file.
% Since version 1.30.0 \pdfTeX\ provides \cs{pdffilemoddate}
% that reads the file date stamp. Thus this package uses this
% command and redefines
% \cs{include} to include the files that do not have \xext{aux}
% files yet or that are newer than its \xext{aux} file.
% \cs{includeonly} is ignored.
%
% \subsection{Usage}
%
% The package is loaded as normal \LaTeX\ package without options:
% \begin{quote}
%   |\usepackage{stampinclude}|
% \end{quote}
% Alternatively the package may be loaded on the command line
% (Example for shell `bash'):
% \begin{center}
%   |latex '\AtBeginDocument{\usepackage{stampinclude}}\input{master}'|
% \end{center}
% Without \cs{AtBeginDocument} (and \cs{RequirePackage} instead of
% \cs{usepackage}) \TeX\ would name the document \xfile{stampinclude.dvi}
% instead of \xfile{master.dvi}.
%
% \subsection{Limitations}
%
% \subsubsection{Other file dependencies}
%
% A file that is included by \cs{include} may input ore reference
% other files:
% \begin{itemize}
% \item other \TeX\ files using \cs{input},
% \item graphics files (\cs{includegraphics}),
% \item listings of external files,
% \item ...
% \end{itemize}
% Updates of those files are not detected by this package.
% It limits the date stamp comparison of an \xext{aux} file
% to its \xext{tex} file.
%
% \subsubsection{\cs{include} dependencies}
%
% In the example, given in the introduction \ref{sec:intro},
% three files \xfile{fileA}, \xfile{fileB}, and \xfile{fileC}
% are included in this order. Now file \xfile{fileA} is changed by adding
% four pages, \xfile{fileB} remains untouched, and \xfile{fileC} is
% also updated. Then the package only selects \xfile{fileA} and
% \xfile{fileC} for inclusion. File \xfile{fileB} is not included.
% But \LaTeX\ has stored the counter values that are active
% at the end of \xfile{fileB} in \xfile{fileB.aux} in one of the
% previous runs when \xfile{fileB} was included.
% However the later addition of four pages in \xfile{fileA}
% was not known at that time. Therefore \xfile{fileB.aux}
% is out of date and the inclusion of file \xfile{fileC}
% starts with wrong counter values (especially the page counter).
%
% \subsubsection{Summary}
%
% This package \xpackage{stampinclude} and the \cs{include} feature
% helps in accelerating the \LaTeX\ compilation.
% But it is not intended for generating the final version.
% For the final version of the document it is better to include
% \emph{all} files to get all counter values right.
% Then this package and any \cs{includeonly} lines should be commented out:
%\begin{quote}
%  |% \usepackage{stampinclude}|\\
%  |% \includeonly{...}|
%\end{quote}
%
% \subsection{Requirements}
%
% \begin{itemize}
% \item \pdfTeX\ v1.30.0 (because of \cs{pdffilemoddate}
%   and \cs{pdfstrcmp}),\\
%   both modes for DVI and PDF are supported.
% \item Alternatively Lua\TeX\ may be used.
%   It lacks \cs{pdffilemoddate} and \cs{pdfstrcmp}. But its services
%   are provided by package \xpackage{pdftexcmds} \cite{pdftexcmds}
%   that is automatically loaded.
% \end{itemize}
%
% \StopEventually{
% }
%
% \section{Implementation}
%
%    \begin{macrocode}
%<*package>
\NeedsTeXFormat{LaTeX2e}
\ProvidesPackage{stampinclude}
  [2016/05/16 v1.1 Include files based on time stamps (HO)]%
%    \end{macrocode}
%
%    \begin{macrocode}
\RequirePackage{pdftexcmds}[2007/12/12]%
%    \end{macrocode}
%
%    \begin{macrocode}
\begingroup
  \chardef\x=1 %
  \expandafter\ifx\csname pdf@filemoddate\endcsname\relax
    \chardef\x=0 %
  \fi
  \expandafter\ifx\csname pdf@strcmp\endcsname\relax
    \chardef\x=0 %
  \fi
\expandafter\endgroup\ifcase\x
  \PackageWarningNoLine{stampinclude}{%
    \string\pdffilemoddate\space or %
    \string\pdfstrcmp\space are not found,\MessageBreak
    that are provided by pdfTeX >= 1.30.0.\MessageBreak
    Also LuaTeX is not detected.\MessageBreak
    Therefore package loading is aborted%
  }%
  \expandafter\endinput
\fi
%    \end{macrocode}
%
%    \begin{macro}{\SInc@org@include}
%    \begin{macrocode}
\let\SInc@org@include\@include
%    \end{macrocode}
%    \end{macro}
%    \begin{macro}{\@include}
%    \begin{macrocode}
\def\@include#1 {%
  \IfFileExists{#1.aux}{%
    \ifnum\pdf@strcmp{\pdf@filemoddate{#1.aux}}%
                     {\pdf@filemoddate{#1.tex}}<0 %
      \ifx\@partlist\@empty
        \gdef\@partlist{{#1}}%
      \else
        \g@addto@macro\@partlist{,{#1}}%
      \fi
    \fi
  }{%
    \ifx\@partlist\@empty
      \gdef\@partlist{{#1}}%
    \else
      \g@addto@macro\@partlist{,{#1}}%
    \fi
  }%
  \SInc@org@include{#1} \relax
}
%    \end{macrocode}
%    \end{macro}
%
%    \begin{macro}{\includeonly}
%    Macro \cs{includeonly} is ignored.
%    \begin{macrocode}
\renewcommand*{\includeonly}[1]{%
  \PackageInfo{stampinclude}{%
    Ignoring \string\includeonly
  }%
}
%    \end{macrocode}
%    \end{macro}
%
%    Simulate \cs{includeonly}.
%    \begin{macrocode}
\@partswtrue
\gdef\@partlist{}
%    \end{macrocode}
%
%    Print included files at end of document.
%    \begin{macrocode}
\AtEndDocument{%
  \begingroup
    \expandafter\let\expandafter\@partlist\expandafter\@empty
    \expandafter\@for\expandafter\reserved@a
    \expandafter:\expandafter=\@partlist\do{%
      \ifx\@partlist\@empty
        \edef\@partlist{\reserved@a}%
      \else
        \edef\@partlist{\@partlist, \reserved@a}%
      \fi
    }%
    \typeout{********************%
             ********************%
             ********************%
             ******************%
    }%
    \ifx\@partlist\@empty
      \typeout{[stampinclude] No included files.}%
    \else
      \typeout{[stampinclude] Included files:}%
      \typeout{\@partlist}%
    \fi
    \typeout{********************%
             ********************%
             ********************%
             ******************%
    }%
  \endgroup
}
%    \end{macrocode}
%
%    \begin{macrocode}
%</package>
%    \end{macrocode}
%
% \section{Installation}
%
% \subsection{Download}
%
% \paragraph{Package.} This package is available on
% CTAN\footnote{\url{http://ctan.org/pkg/stampinclude}}:
% \begin{description}
% \item[\CTAN{macros/latex/contrib/oberdiek/stampinclude.dtx}] The source file.
% \item[\CTAN{macros/latex/contrib/oberdiek/stampinclude.pdf}] Documentation.
% \end{description}
%
%
% \paragraph{Bundle.} All the packages of the bundle `oberdiek'
% are also available in a TDS compliant ZIP archive. There
% the packages are already unpacked and the documentation files
% are generated. The files and directories obey the TDS standard.
% \begin{description}
% \item[\CTAN{install/macros/latex/contrib/oberdiek.tds.zip}]
% \end{description}
% \emph{TDS} refers to the standard ``A Directory Structure
% for \TeX\ Files'' (\CTAN{tds/tds.pdf}). Directories
% with \xfile{texmf} in their name are usually organized this way.
%
% \subsection{Bundle installation}
%
% \paragraph{Unpacking.} Unpack the \xfile{oberdiek.tds.zip} in the
% TDS tree (also known as \xfile{texmf} tree) of your choice.
% Example (linux):
% \begin{quote}
%   |unzip oberdiek.tds.zip -d ~/texmf|
% \end{quote}
%
% \paragraph{Script installation.}
% Check the directory \xfile{TDS:scripts/oberdiek/} for
% scripts that need further installation steps.
% Package \xpackage{attachfile2} comes with the Perl script
% \xfile{pdfatfi.pl} that should be installed in such a way
% that it can be called as \texttt{pdfatfi}.
% Example (linux):
% \begin{quote}
%   |chmod +x scripts/oberdiek/pdfatfi.pl|\\
%   |cp scripts/oberdiek/pdfatfi.pl /usr/local/bin/|
% \end{quote}
%
% \subsection{Package installation}
%
% \paragraph{Unpacking.} The \xfile{.dtx} file is a self-extracting
% \docstrip\ archive. The files are extracted by running the
% \xfile{.dtx} through \plainTeX:
% \begin{quote}
%   \verb|tex stampinclude.dtx|
% \end{quote}
%
% \paragraph{TDS.} Now the different files must be moved into
% the different directories in your installation TDS tree
% (also known as \xfile{texmf} tree):
% \begin{quote}
% \def\t{^^A
% \begin{tabular}{@{}>{\ttfamily}l@{ $\rightarrow$ }>{\ttfamily}l@{}}
%   stampinclude.sty & tex/latex/oberdiek/stampinclude.sty\\
%   stampinclude.pdf & doc/latex/oberdiek/stampinclude.pdf\\
%   stampinclude.dtx & source/latex/oberdiek/stampinclude.dtx\\
% \end{tabular}^^A
% }^^A
% \sbox0{\t}^^A
% \ifdim\wd0>\linewidth
%   \begingroup
%     \advance\linewidth by\leftmargin
%     \advance\linewidth by\rightmargin
%   \edef\x{\endgroup
%     \def\noexpand\lw{\the\linewidth}^^A
%   }\x
%   \def\lwbox{^^A
%     \leavevmode
%     \hbox to \linewidth{^^A
%       \kern-\leftmargin\relax
%       \hss
%       \usebox0
%       \hss
%       \kern-\rightmargin\relax
%     }^^A
%   }^^A
%   \ifdim\wd0>\lw
%     \sbox0{\small\t}^^A
%     \ifdim\wd0>\linewidth
%       \ifdim\wd0>\lw
%         \sbox0{\footnotesize\t}^^A
%         \ifdim\wd0>\linewidth
%           \ifdim\wd0>\lw
%             \sbox0{\scriptsize\t}^^A
%             \ifdim\wd0>\linewidth
%               \ifdim\wd0>\lw
%                 \sbox0{\tiny\t}^^A
%                 \ifdim\wd0>\linewidth
%                   \lwbox
%                 \else
%                   \usebox0
%                 \fi
%               \else
%                 \lwbox
%               \fi
%             \else
%               \usebox0
%             \fi
%           \else
%             \lwbox
%           \fi
%         \else
%           \usebox0
%         \fi
%       \else
%         \lwbox
%       \fi
%     \else
%       \usebox0
%     \fi
%   \else
%     \lwbox
%   \fi
% \else
%   \usebox0
% \fi
% \end{quote}
% If you have a \xfile{docstrip.cfg} that configures and enables \docstrip's
% TDS installing feature, then some files can already be in the right
% place, see the documentation of \docstrip.
%
% \subsection{Refresh file name databases}
%
% If your \TeX~distribution
% (\teTeX, \mikTeX, \dots) relies on file name databases, you must refresh
% these. For example, \teTeX\ users run \verb|texhash| or
% \verb|mktexlsr|.
%
% \subsection{Some details for the interested}
%
% \paragraph{Attached source.}
%
% The PDF documentation on CTAN also includes the
% \xfile{.dtx} source file. It can be extracted by
% AcrobatReader 6 or higher. Another option is \textsf{pdftk},
% e.g. unpack the file into the current directory:
% \begin{quote}
%   \verb|pdftk stampinclude.pdf unpack_files output .|
% \end{quote}
%
% \paragraph{Unpacking with \LaTeX.}
% The \xfile{.dtx} chooses its action depending on the format:
% \begin{description}
% \item[\plainTeX:] Run \docstrip\ and extract the files.
% \item[\LaTeX:] Generate the documentation.
% \end{description}
% If you insist on using \LaTeX\ for \docstrip\ (really,
% \docstrip\ does not need \LaTeX), then inform the autodetect routine
% about your intention:
% \begin{quote}
%   \verb|latex \let\install=y\input{stampinclude.dtx}|
% \end{quote}
% Do not forget to quote the argument according to the demands
% of your shell.
%
% \paragraph{Generating the documentation.}
% You can use both the \xfile{.dtx} or the \xfile{.drv} to generate
% the documentation. The process can be configured by the
% configuration file \xfile{ltxdoc.cfg}. For instance, put this
% line into this file, if you want to have A4 as paper format:
% \begin{quote}
%   \verb|\PassOptionsToClass{a4paper}{article}|
% \end{quote}
% An example follows how to generate the
% documentation with pdf\LaTeX:
% \begin{quote}
%\begin{verbatim}
%pdflatex stampinclude.dtx
%makeindex -s gind.ist stampinclude.idx
%pdflatex stampinclude.dtx
%makeindex -s gind.ist stampinclude.idx
%pdflatex stampinclude.dtx
%\end{verbatim}
% \end{quote}
%
% \section{Catalogue}
%
% The following XML file can be used as source for the
% \href{http://mirror.ctan.org/help/Catalogue/catalogue.html}{\TeX\ Catalogue}.
% The elements \texttt{caption} and \texttt{description} are imported
% from the original XML file from the Catalogue.
% The name of the XML file in the Catalogue is \xfile{stampinclude.xml}.
%    \begin{macrocode}
%<*catalogue>
<?xml version='1.0' encoding='us-ascii'?>
<!DOCTYPE entry SYSTEM 'catalogue.dtd'>
<entry datestamp='$Date$' modifier='$Author$' id='stampinclude'>
  <name>stampinclude</name>
  <caption>Inclusion based on .aux file date stamps.</caption>
  <authorref id='auth:oberdiek'/>
  <copyright owner='Heiko Oberdiek' year='2008'/>
  <license type='lppl1.3'/>
  <version number='1.1'/>
  <description>
    This package replaces <tt>\includeonly</tt> and selects the files for
    <tt>\include</tt> by inspecting the timestamp of the <tt>.aux</tt> file.
    The file is selected for inclusion if the <tt>.aux</tt> file does
    not yet exist or is older than the corresponding <tt>.tex</tt> file.
    <p/>
    The package is part of the <xref refid='oberdiek'>oberdiek</xref>
    bundle.
  </description>
  <documentation details='Package documentation'
      href='ctan:/macros/latex/contrib/oberdiek/stampinclude.pdf'/>
  <ctan file='true' path='/macros/latex/contrib/oberdiek/stampinclude.dtx'/>
  <miktex location='oberdiek'/>
  <texlive location='oberdiek'/>
  <install path='/macros/latex/contrib/oberdiek/oberdiek.tds.zip'/>
</entry>
%</catalogue>
%    \end{macrocode}
%
% \begin{thebibliography}{9}
% \bibitem{askinclude}
%   Pablo A. Straub, Heiko Oberdiek:
%   \textit{The \xpackage{askinclude} package};
%   2007/10/23 v2.0;
%   \CTAN{macros/latex/contrib/oberdiek/askinclude.pdf}.
%
% \bibitem{pdftexcmds}
%   Heiko Oberdiek:
%   \textit{The \xpackage{pdftexcmds} package};
%   2007/12/12 v0.3;
%   \CTAN{macros/latex/contrib/oberdiek/pdftexcmds.pdf}.
%
% \end{thebibliography}
%
% \begin{History}
%   \begin{Version}{2008/07/14 v1.0}
%   \item
%     First version.
%   \end{Version}
%   \begin{Version}{2016/05/16 v1.1}
%   \item
%     Documentation updates.
%   \end{Version}
% \end{History}
%
% \PrintIndex
%
% \Finale
\endinput

%        (quote the arguments according to the demands of your shell)
%
% Documentation:
%    (a) If stampinclude.drv is present:
%           latex stampinclude.drv
%    (b) Without stampinclude.drv:
%           latex stampinclude.dtx; ...
%    The class ltxdoc loads the configuration file ltxdoc.cfg
%    if available. Here you can specify further options, e.g.
%    use A4 as paper format:
%       \PassOptionsToClass{a4paper}{article}
%
%    Programm calls to get the documentation (example):
%       pdflatex stampinclude.dtx
%       makeindex -s gind.ist stampinclude.idx
%       pdflatex stampinclude.dtx
%       makeindex -s gind.ist stampinclude.idx
%       pdflatex stampinclude.dtx
%
% Installation:
%    TDS:tex/latex/oberdiek/stampinclude.sty
%    TDS:doc/latex/oberdiek/stampinclude.pdf
%    TDS:source/latex/oberdiek/stampinclude.dtx
%
%<*ignore>
\begingroup
  \catcode123=1 %
  \catcode125=2 %
  \def\x{LaTeX2e}%
\expandafter\endgroup
\ifcase 0\ifx\install y1\fi\expandafter
         \ifx\csname processbatchFile\endcsname\relax\else1\fi
         \ifx\fmtname\x\else 1\fi\relax
\else\csname fi\endcsname
%</ignore>
%<*install>
\input docstrip.tex
\Msg{************************************************************************}
\Msg{* Installation}
\Msg{* Package: stampinclude 2016/05/16 v1.1 Include files based on time stamps (HO)}
\Msg{************************************************************************}

\keepsilent
\askforoverwritefalse

\let\MetaPrefix\relax
\preamble

This is a generated file.

Project: stampinclude
Version: 2016/05/16 v1.1

Copyright (C) 2008 by
   Heiko Oberdiek <heiko.oberdiek at googlemail.com>

This work may be distributed and/or modified under the
conditions of the LaTeX Project Public License, either
version 1.3c of this license or (at your option) any later
version. This version of this license is in
   http://www.latex-project.org/lppl/lppl-1-3c.txt
and the latest version of this license is in
   http://www.latex-project.org/lppl.txt
and version 1.3 or later is part of all distributions of
LaTeX version 2005/12/01 or later.

This work has the LPPL maintenance status "maintained".

This Current Maintainer of this work is Heiko Oberdiek.

This work consists of the main source file stampinclude.dtx
and the derived files
   stampinclude.sty, stampinclude.pdf, stampinclude.ins, stampinclude.drv.

\endpreamble
\let\MetaPrefix\DoubleperCent

\generate{%
  \file{stampinclude.ins}{\from{stampinclude.dtx}{install}}%
  \file{stampinclude.drv}{\from{stampinclude.dtx}{driver}}%
  \usedir{tex/latex/oberdiek}%
  \file{stampinclude.sty}{\from{stampinclude.dtx}{package}}%
  \nopreamble
  \nopostamble
  \usedir{source/latex/oberdiek/catalogue}%
  \file{stampinclude.xml}{\from{stampinclude.dtx}{catalogue}}%
}

\catcode32=13\relax% active space
\let =\space%
\Msg{************************************************************************}
\Msg{*}
\Msg{* To finish the installation you have to move the following}
\Msg{* file into a directory searched by TeX:}
\Msg{*}
\Msg{*     stampinclude.sty}
\Msg{*}
\Msg{* To produce the documentation run the file `stampinclude.drv'}
\Msg{* through LaTeX.}
\Msg{*}
\Msg{* Happy TeXing!}
\Msg{*}
\Msg{************************************************************************}

\endbatchfile
%</install>
%<*ignore>
\fi
%</ignore>
%<*driver>
\NeedsTeXFormat{LaTeX2e}
\ProvidesFile{stampinclude.drv}%
  [2016/05/16 v1.1 Include files based on time stamps (HO)]%
\documentclass{ltxdoc}
\usepackage{holtxdoc}[2011/11/22]
\begin{document}
  \DocInput{stampinclude.dtx}%
\end{document}
%</driver>
% \fi
%
%
% \CharacterTable
%  {Upper-case    \A\B\C\D\E\F\G\H\I\J\K\L\M\N\O\P\Q\R\S\T\U\V\W\X\Y\Z
%   Lower-case    \a\b\c\d\e\f\g\h\i\j\k\l\m\n\o\p\q\r\s\t\u\v\w\x\y\z
%   Digits        \0\1\2\3\4\5\6\7\8\9
%   Exclamation   \!     Double quote  \"     Hash (number) \#
%   Dollar        \$     Percent       \%     Ampersand     \&
%   Acute accent  \'     Left paren    \(     Right paren   \)
%   Asterisk      \*     Plus          \+     Comma         \,
%   Minus         \-     Point         \.     Solidus       \/
%   Colon         \:     Semicolon     \;     Less than     \<
%   Equals        \=     Greater than  \>     Question mark \?
%   Commercial at \@     Left bracket  \[     Backslash     \\
%   Right bracket \]     Circumflex    \^     Underscore    \_
%   Grave accent  \`     Left brace    \{     Vertical bar  \|
%   Right brace   \}     Tilde         \~}
%
% \GetFileInfo{stampinclude.drv}
%
% \title{The \xpackage{stampinclude} package}
% \date{2016/05/16 v1.1}
% \author{Heiko Oberdiek\thanks
% {Please report any issues at https://github.com/ho-tex/oberdiek/issues}\\
% \xemail{heiko.oberdiek at googlemail.com}}
%
% \maketitle
%
% \begin{abstract}
% The package replaces \cs{includeonly} and selects the files for
% \cs{include} by inspecting the time stamp of the \xext{aux} file.
% The file is selected for inclusion if the \xext{aux} file does
% not yet exist or is older than the corresponding \xext{tex} file.
% \end{abstract}
%
% \tableofcontents
%
% \section{Documentation}
%
% \subsection{Introduction}
% \label{sec:intro}
%
% \LaTeX\ provides two commands \cs{include} and \cs{includeonly}
% that helps in organizing large projects.
% Example for a master file:
%\begin{quote}
%\begin{verbatim}
%\documentclass{book}
%  % \includeonly{}
%\begin{document}
% \include{fileA}
% \include{fileB}
% \include{fileC}
%\end{document}
%\end{verbatim}
%\end{quote}
% All files are read and compiled if \cs{includeonly} is not
% executed. Otherwise you can give \cs{includeonly} a list
% of files in the preamble, e.g.:
% \begin{quote}
%   |\includeonly{fileA,fileC}|
% \end{quote}
% Now only files \xfile{fileA.tex} and \xfile{fileC.tex} are read
% and compiled.
%
% If you change file \xfile{fileB.tex} and want to see only this
% file, then you must change the line with \cs{includeonly} to
% \begin{quote}
%   |\includeonly{fileB}|
% \end{quote}
% It is tedious to do this again and again, if different files
% are changed.
%
% Package \xpackage{askinclude} \cite{askinclude}
% offers a solution for this problem. It interactively asks
% for the files to be included and saves the user from
% editing the master file.
%
% This package \xpackage{stampinclude} goes another way.
% \LaTeX\ reads and writes a separate \xext{aux} file for each
% file that is included by \cs{include}. There \LaTeX\ remembers
% counter valuses. Changed \xext{tex}
% files can therefore be detected by comparing the file date stamp of
% the \xext{tex} file with the date stamp of its \xext{aux} file.
% Since version 1.30.0 \pdfTeX\ provides \cs{pdffilemoddate}
% that reads the file date stamp. Thus this package uses this
% command and redefines
% \cs{include} to include the files that do not have \xext{aux}
% files yet or that are newer than its \xext{aux} file.
% \cs{includeonly} is ignored.
%
% \subsection{Usage}
%
% The package is loaded as normal \LaTeX\ package without options:
% \begin{quote}
%   |\usepackage{stampinclude}|
% \end{quote}
% Alternatively the package may be loaded on the command line
% (Example for shell `bash'):
% \begin{center}
%   |latex '\AtBeginDocument{\usepackage{stampinclude}}\input{master}'|
% \end{center}
% Without \cs{AtBeginDocument} (and \cs{RequirePackage} instead of
% \cs{usepackage}) \TeX\ would name the document \xfile{stampinclude.dvi}
% instead of \xfile{master.dvi}.
%
% \subsection{Limitations}
%
% \subsubsection{Other file dependencies}
%
% A file that is included by \cs{include} may input ore reference
% other files:
% \begin{itemize}
% \item other \TeX\ files using \cs{input},
% \item graphics files (\cs{includegraphics}),
% \item listings of external files,
% \item ...
% \end{itemize}
% Updates of those files are not detected by this package.
% It limits the date stamp comparison of an \xext{aux} file
% to its \xext{tex} file.
%
% \subsubsection{\cs{include} dependencies}
%
% In the example, given in the introduction \ref{sec:intro},
% three files \xfile{fileA}, \xfile{fileB}, and \xfile{fileC}
% are included in this order. Now file \xfile{fileA} is changed by adding
% four pages, \xfile{fileB} remains untouched, and \xfile{fileC} is
% also updated. Then the package only selects \xfile{fileA} and
% \xfile{fileC} for inclusion. File \xfile{fileB} is not included.
% But \LaTeX\ has stored the counter values that are active
% at the end of \xfile{fileB} in \xfile{fileB.aux} in one of the
% previous runs when \xfile{fileB} was included.
% However the later addition of four pages in \xfile{fileA}
% was not known at that time. Therefore \xfile{fileB.aux}
% is out of date and the inclusion of file \xfile{fileC}
% starts with wrong counter values (especially the page counter).
%
% \subsubsection{Summary}
%
% This package \xpackage{stampinclude} and the \cs{include} feature
% helps in accelerating the \LaTeX\ compilation.
% But it is not intended for generating the final version.
% For the final version of the document it is better to include
% \emph{all} files to get all counter values right.
% Then this package and any \cs{includeonly} lines should be commented out:
%\begin{quote}
%  |% \usepackage{stampinclude}|\\
%  |% \includeonly{...}|
%\end{quote}
%
% \subsection{Requirements}
%
% \begin{itemize}
% \item \pdfTeX\ v1.30.0 (because of \cs{pdffilemoddate}
%   and \cs{pdfstrcmp}),\\
%   both modes for DVI and PDF are supported.
% \item Alternatively Lua\TeX\ may be used.
%   It lacks \cs{pdffilemoddate} and \cs{pdfstrcmp}. But its services
%   are provided by package \xpackage{pdftexcmds} \cite{pdftexcmds}
%   that is automatically loaded.
% \end{itemize}
%
% \StopEventually{
% }
%
% \section{Implementation}
%
%    \begin{macrocode}
%<*package>
\NeedsTeXFormat{LaTeX2e}
\ProvidesPackage{stampinclude}
  [2016/05/16 v1.1 Include files based on time stamps (HO)]%
%    \end{macrocode}
%
%    \begin{macrocode}
\RequirePackage{pdftexcmds}[2007/12/12]%
%    \end{macrocode}
%
%    \begin{macrocode}
\begingroup
  \chardef\x=1 %
  \expandafter\ifx\csname pdf@filemoddate\endcsname\relax
    \chardef\x=0 %
  \fi
  \expandafter\ifx\csname pdf@strcmp\endcsname\relax
    \chardef\x=0 %
  \fi
\expandafter\endgroup\ifcase\x
  \PackageWarningNoLine{stampinclude}{%
    \string\pdffilemoddate\space or %
    \string\pdfstrcmp\space are not found,\MessageBreak
    that are provided by pdfTeX >= 1.30.0.\MessageBreak
    Also LuaTeX is not detected.\MessageBreak
    Therefore package loading is aborted%
  }%
  \expandafter\endinput
\fi
%    \end{macrocode}
%
%    \begin{macro}{\SInc@org@include}
%    \begin{macrocode}
\let\SInc@org@include\@include
%    \end{macrocode}
%    \end{macro}
%    \begin{macro}{\@include}
%    \begin{macrocode}
\def\@include#1 {%
  \IfFileExists{#1.aux}{%
    \ifnum\pdf@strcmp{\pdf@filemoddate{#1.aux}}%
                     {\pdf@filemoddate{#1.tex}}<0 %
      \ifx\@partlist\@empty
        \gdef\@partlist{{#1}}%
      \else
        \g@addto@macro\@partlist{,{#1}}%
      \fi
    \fi
  }{%
    \ifx\@partlist\@empty
      \gdef\@partlist{{#1}}%
    \else
      \g@addto@macro\@partlist{,{#1}}%
    \fi
  }%
  \SInc@org@include{#1} \relax
}
%    \end{macrocode}
%    \end{macro}
%
%    \begin{macro}{\includeonly}
%    Macro \cs{includeonly} is ignored.
%    \begin{macrocode}
\renewcommand*{\includeonly}[1]{%
  \PackageInfo{stampinclude}{%
    Ignoring \string\includeonly
  }%
}
%    \end{macrocode}
%    \end{macro}
%
%    Simulate \cs{includeonly}.
%    \begin{macrocode}
\@partswtrue
\gdef\@partlist{}
%    \end{macrocode}
%
%    Print included files at end of document.
%    \begin{macrocode}
\AtEndDocument{%
  \begingroup
    \expandafter\let\expandafter\@partlist\expandafter\@empty
    \expandafter\@for\expandafter\reserved@a
    \expandafter:\expandafter=\@partlist\do{%
      \ifx\@partlist\@empty
        \edef\@partlist{\reserved@a}%
      \else
        \edef\@partlist{\@partlist, \reserved@a}%
      \fi
    }%
    \typeout{********************%
             ********************%
             ********************%
             ******************%
    }%
    \ifx\@partlist\@empty
      \typeout{[stampinclude] No included files.}%
    \else
      \typeout{[stampinclude] Included files:}%
      \typeout{\@partlist}%
    \fi
    \typeout{********************%
             ********************%
             ********************%
             ******************%
    }%
  \endgroup
}
%    \end{macrocode}
%
%    \begin{macrocode}
%</package>
%    \end{macrocode}
%
% \section{Installation}
%
% \subsection{Download}
%
% \paragraph{Package.} This package is available on
% CTAN\footnote{\url{http://ctan.org/pkg/stampinclude}}:
% \begin{description}
% \item[\CTAN{macros/latex/contrib/oberdiek/stampinclude.dtx}] The source file.
% \item[\CTAN{macros/latex/contrib/oberdiek/stampinclude.pdf}] Documentation.
% \end{description}
%
%
% \paragraph{Bundle.} All the packages of the bundle `oberdiek'
% are also available in a TDS compliant ZIP archive. There
% the packages are already unpacked and the documentation files
% are generated. The files and directories obey the TDS standard.
% \begin{description}
% \item[\CTAN{install/macros/latex/contrib/oberdiek.tds.zip}]
% \end{description}
% \emph{TDS} refers to the standard ``A Directory Structure
% for \TeX\ Files'' (\CTAN{tds/tds.pdf}). Directories
% with \xfile{texmf} in their name are usually organized this way.
%
% \subsection{Bundle installation}
%
% \paragraph{Unpacking.} Unpack the \xfile{oberdiek.tds.zip} in the
% TDS tree (also known as \xfile{texmf} tree) of your choice.
% Example (linux):
% \begin{quote}
%   |unzip oberdiek.tds.zip -d ~/texmf|
% \end{quote}
%
% \paragraph{Script installation.}
% Check the directory \xfile{TDS:scripts/oberdiek/} for
% scripts that need further installation steps.
% Package \xpackage{attachfile2} comes with the Perl script
% \xfile{pdfatfi.pl} that should be installed in such a way
% that it can be called as \texttt{pdfatfi}.
% Example (linux):
% \begin{quote}
%   |chmod +x scripts/oberdiek/pdfatfi.pl|\\
%   |cp scripts/oberdiek/pdfatfi.pl /usr/local/bin/|
% \end{quote}
%
% \subsection{Package installation}
%
% \paragraph{Unpacking.} The \xfile{.dtx} file is a self-extracting
% \docstrip\ archive. The files are extracted by running the
% \xfile{.dtx} through \plainTeX:
% \begin{quote}
%   \verb|tex stampinclude.dtx|
% \end{quote}
%
% \paragraph{TDS.} Now the different files must be moved into
% the different directories in your installation TDS tree
% (also known as \xfile{texmf} tree):
% \begin{quote}
% \def\t{^^A
% \begin{tabular}{@{}>{\ttfamily}l@{ $\rightarrow$ }>{\ttfamily}l@{}}
%   stampinclude.sty & tex/latex/oberdiek/stampinclude.sty\\
%   stampinclude.pdf & doc/latex/oberdiek/stampinclude.pdf\\
%   stampinclude.dtx & source/latex/oberdiek/stampinclude.dtx\\
% \end{tabular}^^A
% }^^A
% \sbox0{\t}^^A
% \ifdim\wd0>\linewidth
%   \begingroup
%     \advance\linewidth by\leftmargin
%     \advance\linewidth by\rightmargin
%   \edef\x{\endgroup
%     \def\noexpand\lw{\the\linewidth}^^A
%   }\x
%   \def\lwbox{^^A
%     \leavevmode
%     \hbox to \linewidth{^^A
%       \kern-\leftmargin\relax
%       \hss
%       \usebox0
%       \hss
%       \kern-\rightmargin\relax
%     }^^A
%   }^^A
%   \ifdim\wd0>\lw
%     \sbox0{\small\t}^^A
%     \ifdim\wd0>\linewidth
%       \ifdim\wd0>\lw
%         \sbox0{\footnotesize\t}^^A
%         \ifdim\wd0>\linewidth
%           \ifdim\wd0>\lw
%             \sbox0{\scriptsize\t}^^A
%             \ifdim\wd0>\linewidth
%               \ifdim\wd0>\lw
%                 \sbox0{\tiny\t}^^A
%                 \ifdim\wd0>\linewidth
%                   \lwbox
%                 \else
%                   \usebox0
%                 \fi
%               \else
%                 \lwbox
%               \fi
%             \else
%               \usebox0
%             \fi
%           \else
%             \lwbox
%           \fi
%         \else
%           \usebox0
%         \fi
%       \else
%         \lwbox
%       \fi
%     \else
%       \usebox0
%     \fi
%   \else
%     \lwbox
%   \fi
% \else
%   \usebox0
% \fi
% \end{quote}
% If you have a \xfile{docstrip.cfg} that configures and enables \docstrip's
% TDS installing feature, then some files can already be in the right
% place, see the documentation of \docstrip.
%
% \subsection{Refresh file name databases}
%
% If your \TeX~distribution
% (\teTeX, \mikTeX, \dots) relies on file name databases, you must refresh
% these. For example, \teTeX\ users run \verb|texhash| or
% \verb|mktexlsr|.
%
% \subsection{Some details for the interested}
%
% \paragraph{Attached source.}
%
% The PDF documentation on CTAN also includes the
% \xfile{.dtx} source file. It can be extracted by
% AcrobatReader 6 or higher. Another option is \textsf{pdftk},
% e.g. unpack the file into the current directory:
% \begin{quote}
%   \verb|pdftk stampinclude.pdf unpack_files output .|
% \end{quote}
%
% \paragraph{Unpacking with \LaTeX.}
% The \xfile{.dtx} chooses its action depending on the format:
% \begin{description}
% \item[\plainTeX:] Run \docstrip\ and extract the files.
% \item[\LaTeX:] Generate the documentation.
% \end{description}
% If you insist on using \LaTeX\ for \docstrip\ (really,
% \docstrip\ does not need \LaTeX), then inform the autodetect routine
% about your intention:
% \begin{quote}
%   \verb|latex \let\install=y% \iffalse meta-comment
%
% File: stampinclude.dtx
% Version: 2016/05/16 v1.1
% Info: Include files based on time stamps
%
% Copyright (C) 2008 by
%    Heiko Oberdiek <heiko.oberdiek at googlemail.com>
%    2016
%    https://github.com/ho-tex/oberdiek/issues
%
% This work may be distributed and/or modified under the
% conditions of the LaTeX Project Public License, either
% version 1.3c of this license or (at your option) any later
% version. This version of this license is in
%    http://www.latex-project.org/lppl/lppl-1-3c.txt
% and the latest version of this license is in
%    http://www.latex-project.org/lppl.txt
% and version 1.3 or later is part of all distributions of
% LaTeX version 2005/12/01 or later.
%
% This work has the LPPL maintenance status "maintained".
%
% This Current Maintainer of this work is Heiko Oberdiek.
%
% This work consists of the main source file stampinclude.dtx
% and the derived files
%    stampinclude.sty, stampinclude.pdf, stampinclude.ins, stampinclude.drv.
%
% Distribution:
%    CTAN:macros/latex/contrib/oberdiek/stampinclude.dtx
%    CTAN:macros/latex/contrib/oberdiek/stampinclude.pdf
%
% Unpacking:
%    (a) If stampinclude.ins is present:
%           tex stampinclude.ins
%    (b) Without stampinclude.ins:
%           tex stampinclude.dtx
%    (c) If you insist on using LaTeX
%           latex \let\install=y\input{stampinclude.dtx}
%        (quote the arguments according to the demands of your shell)
%
% Documentation:
%    (a) If stampinclude.drv is present:
%           latex stampinclude.drv
%    (b) Without stampinclude.drv:
%           latex stampinclude.dtx; ...
%    The class ltxdoc loads the configuration file ltxdoc.cfg
%    if available. Here you can specify further options, e.g.
%    use A4 as paper format:
%       \PassOptionsToClass{a4paper}{article}
%
%    Programm calls to get the documentation (example):
%       pdflatex stampinclude.dtx
%       makeindex -s gind.ist stampinclude.idx
%       pdflatex stampinclude.dtx
%       makeindex -s gind.ist stampinclude.idx
%       pdflatex stampinclude.dtx
%
% Installation:
%    TDS:tex/latex/oberdiek/stampinclude.sty
%    TDS:doc/latex/oberdiek/stampinclude.pdf
%    TDS:source/latex/oberdiek/stampinclude.dtx
%
%<*ignore>
\begingroup
  \catcode123=1 %
  \catcode125=2 %
  \def\x{LaTeX2e}%
\expandafter\endgroup
\ifcase 0\ifx\install y1\fi\expandafter
         \ifx\csname processbatchFile\endcsname\relax\else1\fi
         \ifx\fmtname\x\else 1\fi\relax
\else\csname fi\endcsname
%</ignore>
%<*install>
\input docstrip.tex
\Msg{************************************************************************}
\Msg{* Installation}
\Msg{* Package: stampinclude 2016/05/16 v1.1 Include files based on time stamps (HO)}
\Msg{************************************************************************}

\keepsilent
\askforoverwritefalse

\let\MetaPrefix\relax
\preamble

This is a generated file.

Project: stampinclude
Version: 2016/05/16 v1.1

Copyright (C) 2008 by
   Heiko Oberdiek <heiko.oberdiek at googlemail.com>

This work may be distributed and/or modified under the
conditions of the LaTeX Project Public License, either
version 1.3c of this license or (at your option) any later
version. This version of this license is in
   http://www.latex-project.org/lppl/lppl-1-3c.txt
and the latest version of this license is in
   http://www.latex-project.org/lppl.txt
and version 1.3 or later is part of all distributions of
LaTeX version 2005/12/01 or later.

This work has the LPPL maintenance status "maintained".

This Current Maintainer of this work is Heiko Oberdiek.

This work consists of the main source file stampinclude.dtx
and the derived files
   stampinclude.sty, stampinclude.pdf, stampinclude.ins, stampinclude.drv.

\endpreamble
\let\MetaPrefix\DoubleperCent

\generate{%
  \file{stampinclude.ins}{\from{stampinclude.dtx}{install}}%
  \file{stampinclude.drv}{\from{stampinclude.dtx}{driver}}%
  \usedir{tex/latex/oberdiek}%
  \file{stampinclude.sty}{\from{stampinclude.dtx}{package}}%
  \nopreamble
  \nopostamble
  \usedir{source/latex/oberdiek/catalogue}%
  \file{stampinclude.xml}{\from{stampinclude.dtx}{catalogue}}%
}

\catcode32=13\relax% active space
\let =\space%
\Msg{************************************************************************}
\Msg{*}
\Msg{* To finish the installation you have to move the following}
\Msg{* file into a directory searched by TeX:}
\Msg{*}
\Msg{*     stampinclude.sty}
\Msg{*}
\Msg{* To produce the documentation run the file `stampinclude.drv'}
\Msg{* through LaTeX.}
\Msg{*}
\Msg{* Happy TeXing!}
\Msg{*}
\Msg{************************************************************************}

\endbatchfile
%</install>
%<*ignore>
\fi
%</ignore>
%<*driver>
\NeedsTeXFormat{LaTeX2e}
\ProvidesFile{stampinclude.drv}%
  [2016/05/16 v1.1 Include files based on time stamps (HO)]%
\documentclass{ltxdoc}
\usepackage{holtxdoc}[2011/11/22]
\begin{document}
  \DocInput{stampinclude.dtx}%
\end{document}
%</driver>
% \fi
%
%
% \CharacterTable
%  {Upper-case    \A\B\C\D\E\F\G\H\I\J\K\L\M\N\O\P\Q\R\S\T\U\V\W\X\Y\Z
%   Lower-case    \a\b\c\d\e\f\g\h\i\j\k\l\m\n\o\p\q\r\s\t\u\v\w\x\y\z
%   Digits        \0\1\2\3\4\5\6\7\8\9
%   Exclamation   \!     Double quote  \"     Hash (number) \#
%   Dollar        \$     Percent       \%     Ampersand     \&
%   Acute accent  \'     Left paren    \(     Right paren   \)
%   Asterisk      \*     Plus          \+     Comma         \,
%   Minus         \-     Point         \.     Solidus       \/
%   Colon         \:     Semicolon     \;     Less than     \<
%   Equals        \=     Greater than  \>     Question mark \?
%   Commercial at \@     Left bracket  \[     Backslash     \\
%   Right bracket \]     Circumflex    \^     Underscore    \_
%   Grave accent  \`     Left brace    \{     Vertical bar  \|
%   Right brace   \}     Tilde         \~}
%
% \GetFileInfo{stampinclude.drv}
%
% \title{The \xpackage{stampinclude} package}
% \date{2016/05/16 v1.1}
% \author{Heiko Oberdiek\thanks
% {Please report any issues at https://github.com/ho-tex/oberdiek/issues}\\
% \xemail{heiko.oberdiek at googlemail.com}}
%
% \maketitle
%
% \begin{abstract}
% The package replaces \cs{includeonly} and selects the files for
% \cs{include} by inspecting the time stamp of the \xext{aux} file.
% The file is selected for inclusion if the \xext{aux} file does
% not yet exist or is older than the corresponding \xext{tex} file.
% \end{abstract}
%
% \tableofcontents
%
% \section{Documentation}
%
% \subsection{Introduction}
% \label{sec:intro}
%
% \LaTeX\ provides two commands \cs{include} and \cs{includeonly}
% that helps in organizing large projects.
% Example for a master file:
%\begin{quote}
%\begin{verbatim}
%\documentclass{book}
%  % \includeonly{}
%\begin{document}
% \include{fileA}
% \include{fileB}
% \include{fileC}
%\end{document}
%\end{verbatim}
%\end{quote}
% All files are read and compiled if \cs{includeonly} is not
% executed. Otherwise you can give \cs{includeonly} a list
% of files in the preamble, e.g.:
% \begin{quote}
%   |\includeonly{fileA,fileC}|
% \end{quote}
% Now only files \xfile{fileA.tex} and \xfile{fileC.tex} are read
% and compiled.
%
% If you change file \xfile{fileB.tex} and want to see only this
% file, then you must change the line with \cs{includeonly} to
% \begin{quote}
%   |\includeonly{fileB}|
% \end{quote}
% It is tedious to do this again and again, if different files
% are changed.
%
% Package \xpackage{askinclude} \cite{askinclude}
% offers a solution for this problem. It interactively asks
% for the files to be included and saves the user from
% editing the master file.
%
% This package \xpackage{stampinclude} goes another way.
% \LaTeX\ reads and writes a separate \xext{aux} file for each
% file that is included by \cs{include}. There \LaTeX\ remembers
% counter valuses. Changed \xext{tex}
% files can therefore be detected by comparing the file date stamp of
% the \xext{tex} file with the date stamp of its \xext{aux} file.
% Since version 1.30.0 \pdfTeX\ provides \cs{pdffilemoddate}
% that reads the file date stamp. Thus this package uses this
% command and redefines
% \cs{include} to include the files that do not have \xext{aux}
% files yet or that are newer than its \xext{aux} file.
% \cs{includeonly} is ignored.
%
% \subsection{Usage}
%
% The package is loaded as normal \LaTeX\ package without options:
% \begin{quote}
%   |\usepackage{stampinclude}|
% \end{quote}
% Alternatively the package may be loaded on the command line
% (Example for shell `bash'):
% \begin{center}
%   |latex '\AtBeginDocument{\usepackage{stampinclude}}\input{master}'|
% \end{center}
% Without \cs{AtBeginDocument} (and \cs{RequirePackage} instead of
% \cs{usepackage}) \TeX\ would name the document \xfile{stampinclude.dvi}
% instead of \xfile{master.dvi}.
%
% \subsection{Limitations}
%
% \subsubsection{Other file dependencies}
%
% A file that is included by \cs{include} may input ore reference
% other files:
% \begin{itemize}
% \item other \TeX\ files using \cs{input},
% \item graphics files (\cs{includegraphics}),
% \item listings of external files,
% \item ...
% \end{itemize}
% Updates of those files are not detected by this package.
% It limits the date stamp comparison of an \xext{aux} file
% to its \xext{tex} file.
%
% \subsubsection{\cs{include} dependencies}
%
% In the example, given in the introduction \ref{sec:intro},
% three files \xfile{fileA}, \xfile{fileB}, and \xfile{fileC}
% are included in this order. Now file \xfile{fileA} is changed by adding
% four pages, \xfile{fileB} remains untouched, and \xfile{fileC} is
% also updated. Then the package only selects \xfile{fileA} and
% \xfile{fileC} for inclusion. File \xfile{fileB} is not included.
% But \LaTeX\ has stored the counter values that are active
% at the end of \xfile{fileB} in \xfile{fileB.aux} in one of the
% previous runs when \xfile{fileB} was included.
% However the later addition of four pages in \xfile{fileA}
% was not known at that time. Therefore \xfile{fileB.aux}
% is out of date and the inclusion of file \xfile{fileC}
% starts with wrong counter values (especially the page counter).
%
% \subsubsection{Summary}
%
% This package \xpackage{stampinclude} and the \cs{include} feature
% helps in accelerating the \LaTeX\ compilation.
% But it is not intended for generating the final version.
% For the final version of the document it is better to include
% \emph{all} files to get all counter values right.
% Then this package and any \cs{includeonly} lines should be commented out:
%\begin{quote}
%  |% \usepackage{stampinclude}|\\
%  |% \includeonly{...}|
%\end{quote}
%
% \subsection{Requirements}
%
% \begin{itemize}
% \item \pdfTeX\ v1.30.0 (because of \cs{pdffilemoddate}
%   and \cs{pdfstrcmp}),\\
%   both modes for DVI and PDF are supported.
% \item Alternatively Lua\TeX\ may be used.
%   It lacks \cs{pdffilemoddate} and \cs{pdfstrcmp}. But its services
%   are provided by package \xpackage{pdftexcmds} \cite{pdftexcmds}
%   that is automatically loaded.
% \end{itemize}
%
% \StopEventually{
% }
%
% \section{Implementation}
%
%    \begin{macrocode}
%<*package>
\NeedsTeXFormat{LaTeX2e}
\ProvidesPackage{stampinclude}
  [2016/05/16 v1.1 Include files based on time stamps (HO)]%
%    \end{macrocode}
%
%    \begin{macrocode}
\RequirePackage{pdftexcmds}[2007/12/12]%
%    \end{macrocode}
%
%    \begin{macrocode}
\begingroup
  \chardef\x=1 %
  \expandafter\ifx\csname pdf@filemoddate\endcsname\relax
    \chardef\x=0 %
  \fi
  \expandafter\ifx\csname pdf@strcmp\endcsname\relax
    \chardef\x=0 %
  \fi
\expandafter\endgroup\ifcase\x
  \PackageWarningNoLine{stampinclude}{%
    \string\pdffilemoddate\space or %
    \string\pdfstrcmp\space are not found,\MessageBreak
    that are provided by pdfTeX >= 1.30.0.\MessageBreak
    Also LuaTeX is not detected.\MessageBreak
    Therefore package loading is aborted%
  }%
  \expandafter\endinput
\fi
%    \end{macrocode}
%
%    \begin{macro}{\SInc@org@include}
%    \begin{macrocode}
\let\SInc@org@include\@include
%    \end{macrocode}
%    \end{macro}
%    \begin{macro}{\@include}
%    \begin{macrocode}
\def\@include#1 {%
  \IfFileExists{#1.aux}{%
    \ifnum\pdf@strcmp{\pdf@filemoddate{#1.aux}}%
                     {\pdf@filemoddate{#1.tex}}<0 %
      \ifx\@partlist\@empty
        \gdef\@partlist{{#1}}%
      \else
        \g@addto@macro\@partlist{,{#1}}%
      \fi
    \fi
  }{%
    \ifx\@partlist\@empty
      \gdef\@partlist{{#1}}%
    \else
      \g@addto@macro\@partlist{,{#1}}%
    \fi
  }%
  \SInc@org@include{#1} \relax
}
%    \end{macrocode}
%    \end{macro}
%
%    \begin{macro}{\includeonly}
%    Macro \cs{includeonly} is ignored.
%    \begin{macrocode}
\renewcommand*{\includeonly}[1]{%
  \PackageInfo{stampinclude}{%
    Ignoring \string\includeonly
  }%
}
%    \end{macrocode}
%    \end{macro}
%
%    Simulate \cs{includeonly}.
%    \begin{macrocode}
\@partswtrue
\gdef\@partlist{}
%    \end{macrocode}
%
%    Print included files at end of document.
%    \begin{macrocode}
\AtEndDocument{%
  \begingroup
    \expandafter\let\expandafter\@partlist\expandafter\@empty
    \expandafter\@for\expandafter\reserved@a
    \expandafter:\expandafter=\@partlist\do{%
      \ifx\@partlist\@empty
        \edef\@partlist{\reserved@a}%
      \else
        \edef\@partlist{\@partlist, \reserved@a}%
      \fi
    }%
    \typeout{********************%
             ********************%
             ********************%
             ******************%
    }%
    \ifx\@partlist\@empty
      \typeout{[stampinclude] No included files.}%
    \else
      \typeout{[stampinclude] Included files:}%
      \typeout{\@partlist}%
    \fi
    \typeout{********************%
             ********************%
             ********************%
             ******************%
    }%
  \endgroup
}
%    \end{macrocode}
%
%    \begin{macrocode}
%</package>
%    \end{macrocode}
%
% \section{Installation}
%
% \subsection{Download}
%
% \paragraph{Package.} This package is available on
% CTAN\footnote{\url{http://ctan.org/pkg/stampinclude}}:
% \begin{description}
% \item[\CTAN{macros/latex/contrib/oberdiek/stampinclude.dtx}] The source file.
% \item[\CTAN{macros/latex/contrib/oberdiek/stampinclude.pdf}] Documentation.
% \end{description}
%
%
% \paragraph{Bundle.} All the packages of the bundle `oberdiek'
% are also available in a TDS compliant ZIP archive. There
% the packages are already unpacked and the documentation files
% are generated. The files and directories obey the TDS standard.
% \begin{description}
% \item[\CTAN{install/macros/latex/contrib/oberdiek.tds.zip}]
% \end{description}
% \emph{TDS} refers to the standard ``A Directory Structure
% for \TeX\ Files'' (\CTAN{tds/tds.pdf}). Directories
% with \xfile{texmf} in their name are usually organized this way.
%
% \subsection{Bundle installation}
%
% \paragraph{Unpacking.} Unpack the \xfile{oberdiek.tds.zip} in the
% TDS tree (also known as \xfile{texmf} tree) of your choice.
% Example (linux):
% \begin{quote}
%   |unzip oberdiek.tds.zip -d ~/texmf|
% \end{quote}
%
% \paragraph{Script installation.}
% Check the directory \xfile{TDS:scripts/oberdiek/} for
% scripts that need further installation steps.
% Package \xpackage{attachfile2} comes with the Perl script
% \xfile{pdfatfi.pl} that should be installed in such a way
% that it can be called as \texttt{pdfatfi}.
% Example (linux):
% \begin{quote}
%   |chmod +x scripts/oberdiek/pdfatfi.pl|\\
%   |cp scripts/oberdiek/pdfatfi.pl /usr/local/bin/|
% \end{quote}
%
% \subsection{Package installation}
%
% \paragraph{Unpacking.} The \xfile{.dtx} file is a self-extracting
% \docstrip\ archive. The files are extracted by running the
% \xfile{.dtx} through \plainTeX:
% \begin{quote}
%   \verb|tex stampinclude.dtx|
% \end{quote}
%
% \paragraph{TDS.} Now the different files must be moved into
% the different directories in your installation TDS tree
% (also known as \xfile{texmf} tree):
% \begin{quote}
% \def\t{^^A
% \begin{tabular}{@{}>{\ttfamily}l@{ $\rightarrow$ }>{\ttfamily}l@{}}
%   stampinclude.sty & tex/latex/oberdiek/stampinclude.sty\\
%   stampinclude.pdf & doc/latex/oberdiek/stampinclude.pdf\\
%   stampinclude.dtx & source/latex/oberdiek/stampinclude.dtx\\
% \end{tabular}^^A
% }^^A
% \sbox0{\t}^^A
% \ifdim\wd0>\linewidth
%   \begingroup
%     \advance\linewidth by\leftmargin
%     \advance\linewidth by\rightmargin
%   \edef\x{\endgroup
%     \def\noexpand\lw{\the\linewidth}^^A
%   }\x
%   \def\lwbox{^^A
%     \leavevmode
%     \hbox to \linewidth{^^A
%       \kern-\leftmargin\relax
%       \hss
%       \usebox0
%       \hss
%       \kern-\rightmargin\relax
%     }^^A
%   }^^A
%   \ifdim\wd0>\lw
%     \sbox0{\small\t}^^A
%     \ifdim\wd0>\linewidth
%       \ifdim\wd0>\lw
%         \sbox0{\footnotesize\t}^^A
%         \ifdim\wd0>\linewidth
%           \ifdim\wd0>\lw
%             \sbox0{\scriptsize\t}^^A
%             \ifdim\wd0>\linewidth
%               \ifdim\wd0>\lw
%                 \sbox0{\tiny\t}^^A
%                 \ifdim\wd0>\linewidth
%                   \lwbox
%                 \else
%                   \usebox0
%                 \fi
%               \else
%                 \lwbox
%               \fi
%             \else
%               \usebox0
%             \fi
%           \else
%             \lwbox
%           \fi
%         \else
%           \usebox0
%         \fi
%       \else
%         \lwbox
%       \fi
%     \else
%       \usebox0
%     \fi
%   \else
%     \lwbox
%   \fi
% \else
%   \usebox0
% \fi
% \end{quote}
% If you have a \xfile{docstrip.cfg} that configures and enables \docstrip's
% TDS installing feature, then some files can already be in the right
% place, see the documentation of \docstrip.
%
% \subsection{Refresh file name databases}
%
% If your \TeX~distribution
% (\teTeX, \mikTeX, \dots) relies on file name databases, you must refresh
% these. For example, \teTeX\ users run \verb|texhash| or
% \verb|mktexlsr|.
%
% \subsection{Some details for the interested}
%
% \paragraph{Attached source.}
%
% The PDF documentation on CTAN also includes the
% \xfile{.dtx} source file. It can be extracted by
% AcrobatReader 6 or higher. Another option is \textsf{pdftk},
% e.g. unpack the file into the current directory:
% \begin{quote}
%   \verb|pdftk stampinclude.pdf unpack_files output .|
% \end{quote}
%
% \paragraph{Unpacking with \LaTeX.}
% The \xfile{.dtx} chooses its action depending on the format:
% \begin{description}
% \item[\plainTeX:] Run \docstrip\ and extract the files.
% \item[\LaTeX:] Generate the documentation.
% \end{description}
% If you insist on using \LaTeX\ for \docstrip\ (really,
% \docstrip\ does not need \LaTeX), then inform the autodetect routine
% about your intention:
% \begin{quote}
%   \verb|latex \let\install=y\input{stampinclude.dtx}|
% \end{quote}
% Do not forget to quote the argument according to the demands
% of your shell.
%
% \paragraph{Generating the documentation.}
% You can use both the \xfile{.dtx} or the \xfile{.drv} to generate
% the documentation. The process can be configured by the
% configuration file \xfile{ltxdoc.cfg}. For instance, put this
% line into this file, if you want to have A4 as paper format:
% \begin{quote}
%   \verb|\PassOptionsToClass{a4paper}{article}|
% \end{quote}
% An example follows how to generate the
% documentation with pdf\LaTeX:
% \begin{quote}
%\begin{verbatim}
%pdflatex stampinclude.dtx
%makeindex -s gind.ist stampinclude.idx
%pdflatex stampinclude.dtx
%makeindex -s gind.ist stampinclude.idx
%pdflatex stampinclude.dtx
%\end{verbatim}
% \end{quote}
%
% \section{Catalogue}
%
% The following XML file can be used as source for the
% \href{http://mirror.ctan.org/help/Catalogue/catalogue.html}{\TeX\ Catalogue}.
% The elements \texttt{caption} and \texttt{description} are imported
% from the original XML file from the Catalogue.
% The name of the XML file in the Catalogue is \xfile{stampinclude.xml}.
%    \begin{macrocode}
%<*catalogue>
<?xml version='1.0' encoding='us-ascii'?>
<!DOCTYPE entry SYSTEM 'catalogue.dtd'>
<entry datestamp='$Date$' modifier='$Author$' id='stampinclude'>
  <name>stampinclude</name>
  <caption>Inclusion based on .aux file date stamps.</caption>
  <authorref id='auth:oberdiek'/>
  <copyright owner='Heiko Oberdiek' year='2008'/>
  <license type='lppl1.3'/>
  <version number='1.1'/>
  <description>
    This package replaces <tt>\includeonly</tt> and selects the files for
    <tt>\include</tt> by inspecting the timestamp of the <tt>.aux</tt> file.
    The file is selected for inclusion if the <tt>.aux</tt> file does
    not yet exist or is older than the corresponding <tt>.tex</tt> file.
    <p/>
    The package is part of the <xref refid='oberdiek'>oberdiek</xref>
    bundle.
  </description>
  <documentation details='Package documentation'
      href='ctan:/macros/latex/contrib/oberdiek/stampinclude.pdf'/>
  <ctan file='true' path='/macros/latex/contrib/oberdiek/stampinclude.dtx'/>
  <miktex location='oberdiek'/>
  <texlive location='oberdiek'/>
  <install path='/macros/latex/contrib/oberdiek/oberdiek.tds.zip'/>
</entry>
%</catalogue>
%    \end{macrocode}
%
% \begin{thebibliography}{9}
% \bibitem{askinclude}
%   Pablo A. Straub, Heiko Oberdiek:
%   \textit{The \xpackage{askinclude} package};
%   2007/10/23 v2.0;
%   \CTAN{macros/latex/contrib/oberdiek/askinclude.pdf}.
%
% \bibitem{pdftexcmds}
%   Heiko Oberdiek:
%   \textit{The \xpackage{pdftexcmds} package};
%   2007/12/12 v0.3;
%   \CTAN{macros/latex/contrib/oberdiek/pdftexcmds.pdf}.
%
% \end{thebibliography}
%
% \begin{History}
%   \begin{Version}{2008/07/14 v1.0}
%   \item
%     First version.
%   \end{Version}
%   \begin{Version}{2016/05/16 v1.1}
%   \item
%     Documentation updates.
%   \end{Version}
% \end{History}
%
% \PrintIndex
%
% \Finale
\endinput
|
% \end{quote}
% Do not forget to quote the argument according to the demands
% of your shell.
%
% \paragraph{Generating the documentation.}
% You can use both the \xfile{.dtx} or the \xfile{.drv} to generate
% the documentation. The process can be configured by the
% configuration file \xfile{ltxdoc.cfg}. For instance, put this
% line into this file, if you want to have A4 as paper format:
% \begin{quote}
%   \verb|\PassOptionsToClass{a4paper}{article}|
% \end{quote}
% An example follows how to generate the
% documentation with pdf\LaTeX:
% \begin{quote}
%\begin{verbatim}
%pdflatex stampinclude.dtx
%makeindex -s gind.ist stampinclude.idx
%pdflatex stampinclude.dtx
%makeindex -s gind.ist stampinclude.idx
%pdflatex stampinclude.dtx
%\end{verbatim}
% \end{quote}
%
% \section{Catalogue}
%
% The following XML file can be used as source for the
% \href{http://mirror.ctan.org/help/Catalogue/catalogue.html}{\TeX\ Catalogue}.
% The elements \texttt{caption} and \texttt{description} are imported
% from the original XML file from the Catalogue.
% The name of the XML file in the Catalogue is \xfile{stampinclude.xml}.
%    \begin{macrocode}
%<*catalogue>
<?xml version='1.0' encoding='us-ascii'?>
<!DOCTYPE entry SYSTEM 'catalogue.dtd'>
<entry datestamp='$Date$' modifier='$Author$' id='stampinclude'>
  <name>stampinclude</name>
  <caption>Inclusion based on .aux file date stamps.</caption>
  <authorref id='auth:oberdiek'/>
  <copyright owner='Heiko Oberdiek' year='2008'/>
  <license type='lppl1.3'/>
  <version number='1.1'/>
  <description>
    This package replaces <tt>\includeonly</tt> and selects the files for
    <tt>\include</tt> by inspecting the timestamp of the <tt>.aux</tt> file.
    The file is selected for inclusion if the <tt>.aux</tt> file does
    not yet exist or is older than the corresponding <tt>.tex</tt> file.
    <p/>
    The package is part of the <xref refid='oberdiek'>oberdiek</xref>
    bundle.
  </description>
  <documentation details='Package documentation'
      href='ctan:/macros/latex/contrib/oberdiek/stampinclude.pdf'/>
  <ctan file='true' path='/macros/latex/contrib/oberdiek/stampinclude.dtx'/>
  <miktex location='oberdiek'/>
  <texlive location='oberdiek'/>
  <install path='/macros/latex/contrib/oberdiek/oberdiek.tds.zip'/>
</entry>
%</catalogue>
%    \end{macrocode}
%
% \begin{thebibliography}{9}
% \bibitem{askinclude}
%   Pablo A. Straub, Heiko Oberdiek:
%   \textit{The \xpackage{askinclude} package};
%   2007/10/23 v2.0;
%   \CTAN{macros/latex/contrib/oberdiek/askinclude.pdf}.
%
% \bibitem{pdftexcmds}
%   Heiko Oberdiek:
%   \textit{The \xpackage{pdftexcmds} package};
%   2007/12/12 v0.3;
%   \CTAN{macros/latex/contrib/oberdiek/pdftexcmds.pdf}.
%
% \end{thebibliography}
%
% \begin{History}
%   \begin{Version}{2008/07/14 v1.0}
%   \item
%     First version.
%   \end{Version}
%   \begin{Version}{2016/05/16 v1.1}
%   \item
%     Documentation updates.
%   \end{Version}
% \end{History}
%
% \PrintIndex
%
% \Finale
\endinput
|
% \end{quote}
% Do not forget to quote the argument according to the demands
% of your shell.
%
% \paragraph{Generating the documentation.}
% You can use both the \xfile{.dtx} or the \xfile{.drv} to generate
% the documentation. The process can be configured by the
% configuration file \xfile{ltxdoc.cfg}. For instance, put this
% line into this file, if you want to have A4 as paper format:
% \begin{quote}
%   \verb|\PassOptionsToClass{a4paper}{article}|
% \end{quote}
% An example follows how to generate the
% documentation with pdf\LaTeX:
% \begin{quote}
%\begin{verbatim}
%pdflatex stampinclude.dtx
%makeindex -s gind.ist stampinclude.idx
%pdflatex stampinclude.dtx
%makeindex -s gind.ist stampinclude.idx
%pdflatex stampinclude.dtx
%\end{verbatim}
% \end{quote}
%
% \section{Catalogue}
%
% The following XML file can be used as source for the
% \href{http://mirror.ctan.org/help/Catalogue/catalogue.html}{\TeX\ Catalogue}.
% The elements \texttt{caption} and \texttt{description} are imported
% from the original XML file from the Catalogue.
% The name of the XML file in the Catalogue is \xfile{stampinclude.xml}.
%    \begin{macrocode}
%<*catalogue>
<?xml version='1.0' encoding='us-ascii'?>
<!DOCTYPE entry SYSTEM 'catalogue.dtd'>
<entry datestamp='$Date$' modifier='$Author$' id='stampinclude'>
  <name>stampinclude</name>
  <caption>Inclusion based on .aux file date stamps.</caption>
  <authorref id='auth:oberdiek'/>
  <copyright owner='Heiko Oberdiek' year='2008'/>
  <license type='lppl1.3'/>
  <version number='1.1'/>
  <description>
    This package replaces <tt>\includeonly</tt> and selects the files for
    <tt>\include</tt> by inspecting the timestamp of the <tt>.aux</tt> file.
    The file is selected for inclusion if the <tt>.aux</tt> file does
    not yet exist or is older than the corresponding <tt>.tex</tt> file.
    <p/>
    The package is part of the <xref refid='oberdiek'>oberdiek</xref>
    bundle.
  </description>
  <documentation details='Package documentation'
      href='ctan:/macros/latex/contrib/oberdiek/stampinclude.pdf'/>
  <ctan file='true' path='/macros/latex/contrib/oberdiek/stampinclude.dtx'/>
  <miktex location='oberdiek'/>
  <texlive location='oberdiek'/>
  <install path='/macros/latex/contrib/oberdiek/oberdiek.tds.zip'/>
</entry>
%</catalogue>
%    \end{macrocode}
%
% \begin{thebibliography}{9}
% \bibitem{askinclude}
%   Pablo A. Straub, Heiko Oberdiek:
%   \textit{The \xpackage{askinclude} package};
%   2007/10/23 v2.0;
%   \CTAN{macros/latex/contrib/oberdiek/askinclude.pdf}.
%
% \bibitem{pdftexcmds}
%   Heiko Oberdiek:
%   \textit{The \xpackage{pdftexcmds} package};
%   2007/12/12 v0.3;
%   \CTAN{macros/latex/contrib/oberdiek/pdftexcmds.pdf}.
%
% \end{thebibliography}
%
% \begin{History}
%   \begin{Version}{2008/07/14 v1.0}
%   \item
%     First version.
%   \end{Version}
%   \begin{Version}{2016/05/16 v1.1}
%   \item
%     Documentation updates.
%   \end{Version}
% \end{History}
%
% \PrintIndex
%
% \Finale
\endinput
|
% \end{quote}
% Do not forget to quote the argument according to the demands
% of your shell.
%
% \paragraph{Generating the documentation.}
% You can use both the \xfile{.dtx} or the \xfile{.drv} to generate
% the documentation. The process can be configured by the
% configuration file \xfile{ltxdoc.cfg}. For instance, put this
% line into this file, if you want to have A4 as paper format:
% \begin{quote}
%   \verb|\PassOptionsToClass{a4paper}{article}|
% \end{quote}
% An example follows how to generate the
% documentation with pdf\LaTeX:
% \begin{quote}
%\begin{verbatim}
%pdflatex stampinclude.dtx
%makeindex -s gind.ist stampinclude.idx
%pdflatex stampinclude.dtx
%makeindex -s gind.ist stampinclude.idx
%pdflatex stampinclude.dtx
%\end{verbatim}
% \end{quote}
%
% \section{Catalogue}
%
% The following XML file can be used as source for the
% \href{http://mirror.ctan.org/help/Catalogue/catalogue.html}{\TeX\ Catalogue}.
% The elements \texttt{caption} and \texttt{description} are imported
% from the original XML file from the Catalogue.
% The name of the XML file in the Catalogue is \xfile{stampinclude.xml}.
%    \begin{macrocode}
%<*catalogue>
<?xml version='1.0' encoding='us-ascii'?>
<!DOCTYPE entry SYSTEM 'catalogue.dtd'>
<entry datestamp='$Date$' modifier='$Author$' id='stampinclude'>
  <name>stampinclude</name>
  <caption>Inclusion based on .aux file date stamps.</caption>
  <authorref id='auth:oberdiek'/>
  <copyright owner='Heiko Oberdiek' year='2008'/>
  <license type='lppl1.3'/>
  <version number='1.0'/>
  <description>
    This package replaces <tt>\includeonly</tt> and selects the files for
    <tt>\include</tt> by inspecting the timestamp of the <tt>.aux</tt> file.
    The file is selected for inclusion if the <tt>.aux</tt> file does
    not yet exist or is older than the corresponding <tt>.tex</tt> file.
    <p/>
    The package is part of the <xref refid='oberdiek'>oberdiek</xref>
    bundle.
  </description>
  <documentation details='Package documentation'
      href='ctan:/macros/latex/contrib/oberdiek/stampinclude.pdf'/>
  <ctan file='true' path='/macros/latex/contrib/oberdiek/stampinclude.dtx'/>
  <miktex location='oberdiek'/>
  <texlive location='oberdiek'/>
  <install path='/macros/latex/contrib/oberdiek/oberdiek.tds.zip'/>
</entry>
%</catalogue>
%    \end{macrocode}
%
% \begin{thebibliography}{9}
% \bibitem{askinclude}
%   Pablo A. Straub, Heiko Oberdiek:
%   \textit{The \xpackage{askinclude} package};
%   2007/10/23 v2.0;
%   \CTAN{macros/latex/contrib/oberdiek/askinclude.pdf}.
%
% \bibitem{pdftexcmds}
%   Heiko Oberdiek:
%   \textit{The \xpackage{pdftexcmds} package};
%   2007/12/12 v0.3;
%   \CTAN{macros/latex/contrib/oberdiek/pdftexcmds.pdf}.
%
% \end{thebibliography}
%
% \begin{History}
%   \begin{Version}{2008/07/14 v1.0}
%   \item
%     First version.
%   \end{Version}
% \end{History}
%
% \PrintIndex
%
% \Finale
\endinput
