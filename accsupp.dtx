% \iffalse meta-comment
%
% File: accsupp.dtx
% Version: 2016/05/16 v0.4
% Info: Accessibility support by marked content
%
% Copyright (C) 2007, 2010 by
%    Heiko Oberdiek <heiko.oberdiek at googlemail.com>
%    2016
%    https://github.com/ho-tex/oberdiek/issues
%
% This work may be distributed and/or modified under the
% conditions of the LaTeX Project Public License, either
% version 1.3c of this license or (at your option) any later
% version. This version of this license is in
%    http://www.latex-project.org/lppl/lppl-1-3c.txt
% and the latest version of this license is in
%    http://www.latex-project.org/lppl.txt
% and version 1.3 or later is part of all distributions of
% LaTeX version 2005/12/01 or later.
%
% This work has the LPPL maintenance status "maintained".
%
% This Current Maintainer of this work is Heiko Oberdiek.
%
% This work consists of the main source file accsupp.dtx
% and the derived files
%    accsupp.sty, accsupp.pdf, accsupp.ins, accsupp.drv,
%    accsupp-pdftex.def, accsupp-dvipdfm.def, accsupp-dvips.def,
%    accsupp-example1.tex, accsupp-example2.tex,
%    accsupp-test1.tex.
%
% Distribution:
%    CTAN:macros/latex/contrib/oberdiek/accsupp.dtx
%    CTAN:macros/latex/contrib/oberdiek/accsupp.pdf
%
% Unpacking:
%    (a) If accsupp.ins is present:
%           tex accsupp.ins
%    (b) Without accsupp.ins:
%           tex accsupp.dtx
%    (c) If you insist on using LaTeX
%           latex \let\install=y% \iffalse meta-comment
%
% File: accsupp.dtx
% Version: 2018/01/30 v0.5
% Info: Accessibility support by marked content
%
% Copyright (C) 2007, 2010 by
%    Heiko Oberdiek <heiko.oberdiek at googlemail.com>
%    2016
%    https://github.com/ho-tex/oberdiek/issues
%
% This work may be distributed and/or modified under the
% conditions of the LaTeX Project Public License, either
% version 1.3c of this license or (at your option) any later
% version. This version of this license is in
%    http://www.latex-project.org/lppl/lppl-1-3c.txt
% and the latest version of this license is in
%    http://www.latex-project.org/lppl.txt
% and version 1.3 or later is part of all distributions of
% LaTeX version 2005/12/01 or later.
%
% This work has the LPPL maintenance status "maintained".
%
% This Current Maintainer of this work is Heiko Oberdiek.
%
% This work consists of the main source file accsupp.dtx
% and the derived files
%    accsupp.sty, accsupp.pdf, accsupp.ins, accsupp.drv,
%    accsupp-pdftex.def, \ccsupp-luatex.def, accsupp-dvipdfm.def,
%    accsupp-dvips.def,
%    accsupp-example1.tex, accsupp-example2.tex,
%    accsupp-test1.tex.
%
% Distribution:
%    CTAN:macros/latex/contrib/oberdiek/accsupp.dtx
%    CTAN:macros/latex/contrib/oberdiek/accsupp.pdf
%
% Unpacking:
%    (a) If accsupp.ins is present:
%           tex accsupp.ins
%    (b) Without accsupp.ins:
%           tex accsupp.dtx
%    (c) If you insist on using LaTeX
%           latex \let\install=y% \iffalse meta-comment
%
% File: accsupp.dtx
% Version: 2018/01/30 v0.5
% Info: Accessibility support by marked content
%
% Copyright (C) 2007, 2010 by
%    Heiko Oberdiek <heiko.oberdiek at googlemail.com>
%    2016
%    https://github.com/ho-tex/oberdiek/issues
%
% This work may be distributed and/or modified under the
% conditions of the LaTeX Project Public License, either
% version 1.3c of this license or (at your option) any later
% version. This version of this license is in
%    http://www.latex-project.org/lppl/lppl-1-3c.txt
% and the latest version of this license is in
%    http://www.latex-project.org/lppl.txt
% and version 1.3 or later is part of all distributions of
% LaTeX version 2005/12/01 or later.
%
% This work has the LPPL maintenance status "maintained".
%
% This Current Maintainer of this work is Heiko Oberdiek.
%
% This work consists of the main source file accsupp.dtx
% and the derived files
%    accsupp.sty, accsupp.pdf, accsupp.ins, accsupp.drv,
%    accsupp-pdftex.def, \ccsupp-luatex.def, accsupp-dvipdfm.def,
%    accsupp-dvips.def,
%    accsupp-example1.tex, accsupp-example2.tex,
%    accsupp-test1.tex.
%
% Distribution:
%    CTAN:macros/latex/contrib/oberdiek/accsupp.dtx
%    CTAN:macros/latex/contrib/oberdiek/accsupp.pdf
%
% Unpacking:
%    (a) If accsupp.ins is present:
%           tex accsupp.ins
%    (b) Without accsupp.ins:
%           tex accsupp.dtx
%    (c) If you insist on using LaTeX
%           latex \let\install=y% \iffalse meta-comment
%
% File: accsupp.dtx
% Version: 2018/01/30 v0.5
% Info: Accessibility support by marked content
%
% Copyright (C) 2007, 2010 by
%    Heiko Oberdiek <heiko.oberdiek at googlemail.com>
%    2016
%    https://github.com/ho-tex/oberdiek/issues
%
% This work may be distributed and/or modified under the
% conditions of the LaTeX Project Public License, either
% version 1.3c of this license or (at your option) any later
% version. This version of this license is in
%    http://www.latex-project.org/lppl/lppl-1-3c.txt
% and the latest version of this license is in
%    http://www.latex-project.org/lppl.txt
% and version 1.3 or later is part of all distributions of
% LaTeX version 2005/12/01 or later.
%
% This work has the LPPL maintenance status "maintained".
%
% This Current Maintainer of this work is Heiko Oberdiek.
%
% This work consists of the main source file accsupp.dtx
% and the derived files
%    accsupp.sty, accsupp.pdf, accsupp.ins, accsupp.drv,
%    accsupp-pdftex.def, \ccsupp-luatex.def, accsupp-dvipdfm.def,
%    accsupp-dvips.def,
%    accsupp-example1.tex, accsupp-example2.tex,
%    accsupp-test1.tex.
%
% Distribution:
%    CTAN:macros/latex/contrib/oberdiek/accsupp.dtx
%    CTAN:macros/latex/contrib/oberdiek/accsupp.pdf
%
% Unpacking:
%    (a) If accsupp.ins is present:
%           tex accsupp.ins
%    (b) Without accsupp.ins:
%           tex accsupp.dtx
%    (c) If you insist on using LaTeX
%           latex \let\install=y\input{accsupp.dtx}
%        (quote the arguments according to the demands of your shell)
%
% Documentation:
%    (a) If accsupp.drv is present:
%           latex accsupp.drv
%    (b) Without accsupp.drv:
%           latex accsupp.dtx; ...
%    The class ltxdoc loads the configuration file ltxdoc.cfg
%    if available. Here you can specify further options, e.g.
%    use A4 as paper format:
%       \PassOptionsToClass{a4paper}{article}
%
%    Programm calls to get the documentation (example):
%       pdflatex accsupp.dtx
%       makeindex -s gind.ist accsupp.idx
%       pdflatex accsupp.dtx
%       makeindex -s gind.ist accsupp.idx
%       pdflatex accsupp.dtx
%
% Installation:
%    TDS:tex/latex/oberdiek/accsupp.sty
%    TDS:tex/latex/oberdiek/accsupp-pdftex.def
%    TDS:tex/latex/oberdiek/accsupp-luatex.def
%    TDS:tex/latex/oberdiek/accsupp-dvipdfm.def
%    TDS:tex/latex/oberdiek/accsupp-dvips.def
%    TDS:doc/latex/oberdiek/accsupp.pdf
%    TDS:doc/latex/oberdiek/accsupp-example1.tex
%    TDS:doc/latex/oberdiek/accsupp-example2.tex
%    TDS:doc/latex/oberdiek/test/accsupp-test1.tex
%    TDS:source/latex/oberdiek/accsupp.dtx
%
%<*ignore>
\begingroup
  \catcode123=1 %
  \catcode125=2 %
  \def\x{LaTeX2e}%
\expandafter\endgroup
\ifcase 0\ifx\install y1\fi\expandafter
         \ifx\csname processbatchFile\endcsname\relax\else1\fi
         \ifx\fmtname\x\else 1\fi\relax
\else\csname fi\endcsname
%</ignore>
%<*install>
\input docstrip.tex
\Msg{************************************************************************}
\Msg{* Installation}
\Msg{* Package: accsupp 2018/01/30 v0.5 Accessibility support by marked content (HO)}
\Msg{************************************************************************}

\keepsilent
\askforoverwritefalse

\let\MetaPrefix\relax
\preamble

This is a generated file.

Project: accsupp
Version: 2018/01/30 v0.5

Copyright (C) 2007, 2010 by
   Heiko Oberdiek <heiko.oberdiek at googlemail.com>

This work may be distributed and/or modified under the
conditions of the LaTeX Project Public License, either
version 1.3c of this license or (at your option) any later
version. This version of this license is in
   http://www.latex-project.org/lppl/lppl-1-3c.txt
and the latest version of this license is in
   http://www.latex-project.org/lppl.txt
and version 1.3 or later is part of all distributions of
LaTeX version 2005/12/01 or later.

This work has the LPPL maintenance status "maintained".

This Current Maintainer of this work is Heiko Oberdiek.

This work consists of the main source file accsupp.dtx
and the derived files
   accsupp.sty, accsupp.pdf, accsupp.ins, accsupp.drv,
   accsupp-pdftex.def, accsupp-luatex.def, accsupp-dvipdfm.def,
   accsupp-dvips.def,
   accsupp-example1.tex, accsupp-example2.tex,
   accsupp-test1.tex.

\endpreamble
\let\MetaPrefix\DoubleperCent

\generate{%
  \file{accsupp.ins}{\from{accsupp.dtx}{install}}%
  \file{accsupp.drv}{\from{accsupp.dtx}{driver}}%
  \usedir{tex/latex/oberdiek}%
  \file{accsupp.sty}{\from{accsupp.dtx}{package}}%
  \file{accsupp-pdftex.def}{\from{accsupp.dtx}{pdftex}}%
  \file{accsupp-luatex.def}{\from{accsupp.dtx}{luatex}}%
  \file{accsupp-dvipdfm.def}{\from{accsupp.dtx}{dvipdfm}}%
  \file{accsupp-dvips.def}{\from{accsupp.dtx}{dvips}}%
  \usedir{doc/latex/oberdiek}%
  \file{accsupp-example1.tex}{\from{accsupp.dtx}{example1}}%
  \file{accsupp-example2.tex}{\from{accsupp.dtx}{example2}}%
  \usedir{doc/latex/oberdiek/test}%
  \file{accsupp-test1.tex}{\from{accsupp.dtx}{test1}}%
  \nopreamble
  \nopostamble
  \usedir{source/latex/oberdiek/catalogue}%
  \file{accsupp.xml}{\from{accsupp.dtx}{catalogue}}%
}

\catcode32=13\relax% active space
\let =\space%
\Msg{************************************************************************}
\Msg{*}
\Msg{* To finish the installation you have to move the following}
\Msg{* files into a directory searched by TeX:}
\Msg{*}
\Msg{*     accsupp.sty, accsupp-pdftex.def, accsupp-dvipdfm.def,}
\Msg{*     accsupp-dvips.def}
\Msg{*}
\Msg{* To produce the documentation run the file `accsupp.drv'}
\Msg{* through LaTeX.}
\Msg{*}
\Msg{* Happy TeXing!}
\Msg{*}
\Msg{************************************************************************}

\endbatchfile
%</install>
%<*ignore>
\fi
%</ignore>
%<*driver>
\NeedsTeXFormat{LaTeX2e}
\ProvidesFile{accsupp.drv}%
  [2018/01/30 v0.5 Accessibility support by marked content (HO)]%
\documentclass{ltxdoc}
\usepackage{holtxdoc}[2011/11/22]
\begin{document}
  \DocInput{accsupp.dtx}%
\end{document}
%</driver>
% \fi
%
%
% \CharacterTable
%  {Upper-case    \A\B\C\D\E\F\G\H\I\J\K\L\M\N\O\P\Q\R\S\T\U\V\W\X\Y\Z
%   Lower-case    \a\b\c\d\e\f\g\h\i\j\k\l\m\n\o\p\q\r\s\t\u\v\w\x\y\z
%   Digits        \0\1\2\3\4\5\6\7\8\9
%   Exclamation   \!     Double quote  \"     Hash (number) \#
%   Dollar        \$     Percent       \%     Ampersand     \&
%   Acute accent  \'     Left paren    \(     Right paren   \)
%   Asterisk      \*     Plus          \+     Comma         \,
%   Minus         \-     Point         \.     Solidus       \/
%   Colon         \:     Semicolon     \;     Less than     \<
%   Equals        \=     Greater than  \>     Question mark \?
%   Commercial at \@     Left bracket  \[     Backslash     \\
%   Right bracket \]     Circumflex    \^     Underscore    \_
%   Grave accent  \`     Left brace    \{     Vertical bar  \|
%   Right brace   \}     Tilde         \~}
%
% \GetFileInfo{accsupp.drv}
%
% \title{The \xpackage{accsupp} package}
% \date{2018/01/30 v0.5}
% \author{Heiko Oberdiek\thanks
% {Please report any issues at https://github.com/ho-tex/oberdiek/issues}\\
% \xemail{heiko.oberdiek at googlemail.com}}
%
% \maketitle
%
% \begin{abstract}
% Since PDF 1.5 portions of a page can be marked for better
% accessibility support.
% For example, replacement texts or expansions of abbreviations can be
% provided. Package \xpackage{accsupp} starts with providing a minimal
% low-level interface for programmers. Status is experimental.
% \end{abstract}
%
% \tableofcontents
%
% \section{Documentation}
%
% \subsection{Macros}
%
% Section ``10.8 Accessibility Support'' of the PDF reference
% \cite{pdfspec} lists some features that can be added by
% operators for marked content.
%
% \begin{declcs}{BeginAccSupp} \M{options}
% \end{declcs}
% It puts the operator |BDC| in the page stream:
% \begin{quote}
%   |/Span|\\
%   |<<|\dots|>>| \qquad \textit{\% property dictionary}\\
%   |BDC|
% \end{quote}
% The contents of the dictionary is controlled by \meta{options}.
% See sections \ref{sec:feature} and \ref{sec:method}.
%
% \begin{declcs}{EndAccSupp} \M{options}
% \end{declcs}
% It puts the operator |EMC| in the page stream.
% The only option is \xoption{pdfliteral}, see section \ref{sec:pdftex}.
%
% \bigskip
% \noindent\textbf{Note:}
% The caller is responsible for the placement of \cs{BeginAccSupp}
% and \cs{EndAccSupp} pairs. Especially page breaks are not allowed
% in between.
%
% \subsubsection{Feature options}\label{sec:feature}
%
% The PDF reference \cite{pdfspec} describes and explains the different
% features. The names of the feature options are the same as the
% key names for the property dictionary for operator |BDC|, see
% \cs{BeginAccSupp}.
%
% \begin{description}
% \item[\xoption{ActualText}:]
%   Provides a replacement text, see examples in
%   section \ref{sec:example}.
% \item[\xoption{Alt}:]
%   Provides an alternate description.
% \item[\xoption{E}:]
%   Provides the expansion of an abbreviation or an acronym.
% \item[\xoption{Lang}:]
%   Specifies the language.
% \end{description}
%
% \subsubsection{Input methods}\label{sec:method}
%
% Except for \xoption{Lang} option \xoption{method} controls
% how the argument for \xoption{ActualText}, \xoption{Alt},
% and \xoption{E} are interpreted.
% \begin{description}
% \item[\xoption{method=plain}:]
%   The string is only expanded and written without further
%   treatment. Special characters are not protected, thus this method
%   may result in an invalid PDF file.
% \item[\xoption{mehod=escape}:]
%   The string is expanded and special characters are escaped.
%   The result is a valid PDF string.
% \item[\xoption{method=hex}:]
%   The string is given in hexadecimal notation.
%   Section \ref{sec:example:notparallel} shows an example.
% \item[\xoption{method=pdfstringdef}:]
%   If package \xpackage{hyperref} is loaded, then its \cs{pdfstringdef}
%   is used. This method is slow, but useful if the string contains
%   arbitrary \hologo{TeX} code.
% \item[\xoption{unicode}:]
%   This option is needed, if the string is given as Unicode string
%   (16 bit). Internally it adds the string prefix for Unicode.
%   In case of \xoption{method=pdfstringdef} it passes the option
%   to \cs{hypersetup}.
% \end{description}
%
% \subsection{Workaround, option \xoption{space}}
%
% PDF specification says in ``10.8.3 Replacement Text''
% \begin{quote}
% Just as alternative descriptions can be provided for images and other
% items that do not translate naturally into text (\dots), replacement
% text can be specified for content that does translate into text but
% that is represented in a nonstandard way. These nonstandard representations
% might include, for example, glyphs for ligatures or custom characters, or
% inline graphics corresponding to letters in an illuminated manustript or to
% dropped capitals.
% \end{quote}
% However, the ActualText is ignored in Acrobat Reader
% (until version 9 at least), if the content does not contain glyphs.
% Option \xoption{space} adds such an invisible glyph, a space
% character. The font name can be configured by option \xoption{spacefont},
% the default font is \texttt{phvr8r}. The character slot is given
% by option \xoption{spacechar}, default is \texttt{32}, the usual
% position of the space character.
%
% These options can also be given as package options or in
% macro \cs{AccSuppSetup} that takes a key value list as argument.
% Usually only option \xoption{space} is necessary, if the
% contents does not contain glyphs otherwise. Option \xoption{space}
% is enabled by |space| or |space=true| and disabled by |space=false|.
% It is disabled as default. The option is evaluated by \cs{BeginAccSupp}
% and ignored by \cs{EndAccSupp}.
%
% Note: Even with option \xoption{space} I get sometimes wrong texts
% when cut \& paste from AR7/Linux or AR8/Linux, e.g.
% \begin{quote}
%   |Hello| $\rightarrow$ |Helo|, |Helol|, |Hell|, \dots\\
%   |Hello World| $\rightarrow$ |Helo WorldW|, |Helo World|, \dots
% \end{quote}
% I do not know what Acrobat Reader is doing here, thus feedback and
% insights are welcome.
%
% \subsection{Driver options}
%
% Driver options are package options only.
% The special \hologo{TeX} compilers \hologo{pdfTeX}, \hologo{LuaTeX} and \hologo{XeTeX}
% are detected automatically. The default for unrecognized drivers can be
% set by defining \cs{ActualTextDriverDefault}. This can be done
% in the configuration file \xfile{accsupp.cfg}.
%
% \subsubsection{Option \xoption{pdftex}}\label{sec:pdftex}
% Package option \xoption{pdftex} is used for \hologo{pdfTeX} in PDF mode.
% Additionally \cs{BeginAccSupp} and \cs{EndAccSupp} understand
% option \xoption{pdfliteral}. It controls
% the modifier keyword for \cs{pdfliteral}:
% \begin{quote}
%   |pdfliteral=direct| $\Rightarrow$ |\pdfliteral direct{...}|
% \end{quote}
%
% \subsubsection{Option \xoption{luatex}}\label{sec:luatex}
% Package option \xoption{luatex} is used for \hologo{LuaTeX} in PDF mode.
% Additionally \cs{BeginAccSupp} and \cs{EndAccSupp} understand
% option \xoption{pdfliteral}. It controls
% the modifier keyword for \cs{pdfextension} literal:
% \begin{quote}
%   |pdfliteral=direct| $\Rightarrow$ |\pdfextension literal direct{...}|
% \end{quote}
%
% \subsubsection{Option \xoption{dvipdfm}}
%
% Package option \xoption{dvipdfm} and its aliases \xoption{dvipdfmx}
% \xoption{xetex} are used for drivers that support
% \textsf{dvipdfm} specials.
%
% \subsubsection{Option \xoption{dvips}}
%
% Package option \xoption{dvips} and its alias \xoption{dvipsone}
% write pdfmark specials in the output. Unhappily these pdfmark
% operators are ignored by \textsf{ghostscript} (latest tested
% version is 8.54). Perhaps they are recognized by commercial
% distiller applications.
%
% \subsubsection{Turning off page stream compression}
%
% For debugging it is useful to have uncompressed page stream objects.
% This can be done afterwards via \textsf{pdftk}:
% \begin{quote}
%   |pdftk file.pdf output file-uncompressed.pdf uncompress|
% \end{quote}
% Or the PDF file is generated uncompressed:
% \begin{description}
% \item[\hologo{pdfTeX}:] |\pdfcompresslevel=0|
% \item[dvipdfm:] |dvipdfm -z0 ...|
% \item[dvipdfmx:] |dvipdfmx -z0 ...|
% \item[ghostscript/ps2pdf:]
%      |ps2pdf -dCompressPages=false input.ps output.pdf|
% \end{description}
%
% \section{Example}
% \label{sec:example}
%
% \subsection{Example \cs{notparallel}}
% \label{sec:example:notparallel}
%
%    \begin{macrocode}
%<*example1>
%    \end{macrocode}
%    \begin{macrocode}
%<<END
\documentclass{article}
\usepackage{accsupp}[2007/11/14]
\usepackage{centernot}
% U+2226 NOT PARALLEL
% \mathrel{...} prevents page break in between
\newcommand*{\notparallel}{%
  \ensuremath{%
    \mathrel{%
      \BeginAccSupp{method=hex,unicode,ActualText=2226}%
      \centernot{\parallel}%
      \EndAccSupp{}%
    }%
  }%
}
\begin{document}
\begin{equation}
A\notparallel B
\end{equation}
\end{document}
%END
%    \end{macrocode}
%    \begin{macrocode}
%</example1>
%    \end{macrocode}
%
% \subsection{Example with \xoption{pdfstringdef}}
%
%    \begin{macrocode}
%<*example2>
%    \end{macrocode}
%    \begin{macrocode}
%<<END
\documentclass{article}
\usepackage[unicode]{hyperref}
\usepackage{accsupp}[2007/11/14]
\begin{document}
  \begin{equation}
    \BeginAccSupp{
      method=pdfstringdef,
      unicode,
      ActualText={%
        a\texttwosuperior +b\texttwosuperior
        =c\texttwosuperior
      }
    }
    a^2 + b^2 = c^2
    \EndAccSupp{}
  \end{equation}
\end{document}
%END
%    \end{macrocode}
%    \begin{macrocode}
%</example2>
%    \end{macrocode}
%
% \StopEventually{
% }
%
% \section{Implementation}
%
% \subsection{Package}
%
%    \begin{macrocode}
%<*package>
%    \end{macrocode}
%
%    \begin{macrocode}
\begingroup\catcode61\catcode48\catcode32=10\relax%
  \catcode13=5 % ^^M
  \endlinechar=13 %
  \catcode123=1 % {
  \catcode125=2 % }
  \catcode64=11 % @
  \def\x{\endgroup
    \expandafter\edef\csname ACCSUPP@AtEnd\endcsname{%
      \endlinechar=\the\endlinechar\relax
      \catcode13=\the\catcode13\relax
      \catcode32=\the\catcode32\relax
      \catcode35=\the\catcode35\relax
      \catcode61=\the\catcode61\relax
      \catcode64=\the\catcode64\relax
      \catcode123=\the\catcode123\relax
      \catcode125=\the\catcode125\relax
    }%
  }%
\x\catcode61\catcode48\catcode32=10\relax%
\catcode13=5 % ^^M
\endlinechar=13 %
\catcode35=6 % #
\catcode64=11 % @
\catcode123=1 % {
\catcode125=2 % }
\def\TMP@EnsureCode#1#2{%
  \edef\ACCSUPP@AtEnd{%
    \ACCSUPP@AtEnd
    \catcode#1=\the\catcode#1\relax
  }%
  \catcode#1=#2\relax
}
\TMP@EnsureCode{10}{12}% ^^J
\TMP@EnsureCode{33}{12}% !
\TMP@EnsureCode{39}{12}% '
\TMP@EnsureCode{40}{12}% (
\TMP@EnsureCode{41}{12}% )
\TMP@EnsureCode{42}{12}% *
\TMP@EnsureCode{44}{12}% ,
\TMP@EnsureCode{45}{12}% -
\TMP@EnsureCode{46}{12}% .
\TMP@EnsureCode{47}{12}% /
\TMP@EnsureCode{58}{12}% :
\TMP@EnsureCode{60}{12}% <
\TMP@EnsureCode{62}{12}% >
\TMP@EnsureCode{94}{7}% ^ (superscript)
\TMP@EnsureCode{96}{12}% `
\TMP@EnsureCode{254}{12}% ^^fe
\TMP@EnsureCode{255}{12}% ^^ff
\edef\ACCSUPP@AtEnd{\ACCSUPP@AtEnd\noexpand\endinput}
%    \end{macrocode}
%
%    Package identification.
%    \begin{macrocode}
\NeedsTeXFormat{LaTeX2e}
\ProvidesPackage{accsupp}%
  [2018/01/30 v0.5 Accessibility support by marked content (HO)]
%    \end{macrocode}
%
%    \begin{macrocode}
\RequirePackage{pdfescape}[2007/02/25]
\RequirePackage{ifpdf}
\RequirePackage{ifxetex}
\RequirePackage{kvoptions}
%    \end{macrocode}
%    \begin{macrocode}
\SetupKeyvalOptions{%
  family=ACCSUPP,%
  prefix=ACCSUPP@%
}
%    \end{macrocode}
%
% \subsection{Space setup}
%
%    \begin{macrocode}
\DeclareBoolOption{space}
\DeclareStringOption[phvr8r]{spacefont}
\DeclareStringOption[32]{spacechar}
%    \end{macrocode}
%
% \subsection{Driver detection and setup}
%
% \paragraph{Driver declarations.}
%
%    \begin{macrocode}
\def\ACCSUPP@DefineDriverKey{%
  \@dblarg\ACCSUPP@@DefineDriverKey
}
\def\ACCSUPP@@DefineDriverKey[#1]#2{%
  \define@key{ACCSUPP}{#2}[]{%
    \def\ACCSUPP@driver{#1}%
  }%
  \g@addto@macro\ACCSUPP@DisableOptions{%
    \DisableKeyvalOption{ACCSUPP}{#2}%
  }%
}
\let\ACCSUPP@DisableOptions\@empty
\ACCSUPP@DefineDriverKey{pdftex}
\ACCSUPP@DefineDriverKey{luatex}
\ACCSUPP@DefineDriverKey{dvips}
\ACCSUPP@DefineDriverKey[dvips]{dvipsone}
\ACCSUPP@DefineDriverKey{dvipdfm}
\ACCSUPP@DefineDriverKey[dvipdfm]{dvipdfmx}
\ACCSUPP@DefineDriverKey[dvipdfm]{xetex}
\let\ACCSUPP@driver\relax
\InputIfFileExists{accsupp.cfg}{}{}
\providecommand*{\ActualTextDriverDefault}{dvips}
\ifpdf
  \ifx\pdfextension\@undefined
    \def\ACCSUPP@driver{pdftex}%
  \else
    \def\ACCSUPP@driver{luatex}%
  \fi
\else
  \ifxetex
    \def\ACCSUPP@driver{dvipdfm}%
  \else
    \ifx\ACCSUPP@driver\relax
      \let\ACCSUPP@driver\ActualTextDriverDefault
    \fi
  \fi
\fi
%    \end{macrocode}
%
% \paragraph{Process options.}
%
%    \begin{macrocode}
\ProcessKeyvalOptions*
\ACCSUPP@DisableOptions
%    \end{macrocode}
%
% \paragraph{Driver validation and loading.}
%
%    \begin{macrocode}
\def\ACCSUPP@temp{pdftex}%
\ifpdf
  \ifx\pdfextension\@undefined\else
    \def\ACCSUPP@temp{luatex}%
  \fi
  \ifx\ACCSUPP@temp\ACCSUPP@driver
  \else
    \PackageWarningNoLine{accsupp}{%
      Wrong driver `\ACCSUPP@driver', using `pdftex' instead%
    }%
ed as
    \let\ACCSUPP@driver\ACCSUPP@temp
  \fi
\else
  \ifx\ACCSUPP@temp\ACCSUPP@driver
    \PackageError{accsupp}{%
      Wrong driver, pdfTeX is not running in PDF mode.\MessageBreak
      Package loading is aborted%
    }\@ehc
    \expandafter\expandafter\expandafter\ACCSUPP@AtEnd
  \fi
  \def\ACCSUPP@temp{dvipdfm}%
  \ifxetex
    \ifx\ACCSUPP@temp\ACCSUPP@driver
    \else
      \PackageWarningNoLine{accsupp}{%
        Wrong driver `\ACCSUPP@driver',\MessageBreak
        using `dvipdfm' for XeTeX instead%
      }%
      \let\ACCSUPP@driver\ACCSUPP@temp
    \fi
  \fi
\fi%
\ifx\ACCSUPP@driver\relax
  \PackageError{accsupp}{%
    Missing driver option.\MessageBreak
    Package loading is aborted%
  }\@ehc
  \expandafter\ACCSUPP@AtEnd
\fi%
\InputIfFileExists{accsupp-\ACCSUPP@driver.def}{}{%
  \PackageError{accsupp}{%
    Unsupported driver `\ACCSUPP@driver'.\MessageBreak
    Package loading is aborted%
  }\@ehc
  \ACCSUPP@AtEnd
}%
%    \end{macrocode}
%
% \subsection{Main macro}
%
%    \begin{macrocode}
\DeclareBoolOption{unicode}
\DeclareStringOption[page]{pdfliteral}
\DeclareStringOption{Lang}
\def\ACCSUPP@method{escape}
\define@key{ACCSUPP}{method}{%
  \@ifundefined{ACCSUPP@method@#1}{%
    \PackageError{accsupp}{%
      Ignoring unknown method `#1'%
    }\@ehc
  }{%
    \edef\ACCSUPP@method{#1}%
  }%
}
\let\ACCSUPP@Lang\relax
\def\ACCSUPP@temp#1{%
  \expandafter\ACCSUPP@@temp\csname ACCSUPP@#1\endcsname{#1}%
}
\def\ACCSUPP@@temp#1#2{%
  \let#1\relax
  \define@key{ACCSUPP}{#2}{%
    \def#1{##1}%
    \ifx#1\@empty
      \def#1{()}%
    \else
      \csname ACCSUPP@method@\ACCSUPP@method\endcsname#1%
    \fi
  }%
}
\ACCSUPP@temp{Alt}
\ACCSUPP@temp{ActualText}
\ACCSUPP@temp{E}
%    \end{macrocode}
%
%    \begin{macrocode}
\newcommand*{\BeginAccSupp}[1]{%
  \begingroup
    \setkeys{ACCSUPP}{#1}%
    \edef\ACCSUPP@span{%
      /Span<<%
        \ifx\ACCSUPP@Lang\relax
        \else
          /Lang\ACCSUPP@Lang
        \fi
        \ifx\ACCSUPP@Alt\relax
        \else
          /Alt\ACCSUPP@Alt
        \fi
        \ifx\ACCSUPP@ActualText\relax
        \else
          /ActualText\ACCSUPP@ActualText
        \fi
        \ifx\ACCSUPP@E\relax
        \else
          /E\ACCSUPP@E
        \fi
      >>%
    }%
    \ACCSUPP@bdc
    \ACCSUPP@space
  \endgroup
}
\newcommand*{\EndAccSupp}[1]{%
  \begingroup
    \setkeys{ACCSUPP}{#1}%
    \ACCSUPP@emc
  \endgroup
}
\def\ACCSUPP@space{%
  \ifACCSUPP@space
    \begingroup
      \@ifundefined{ACCSUPP@Font}{%
        \global\font\ACCSUPP@Font=\ACCSUPP@spacefont\relax
      }{}%
      \leavevmode
      \setbox\z@\hbox{\ACCSUPP@Font\char\ACCSUPP@spacechar}%
      \wd\z@\z@
      \ht\z@\z@
      \dp\z@\z@
      \copy\z@
    \endgroup
  \fi
}
\newcommand*{\AccSuppSetup}{%
  \setkeys{ACCSUPP}%
}
%    \end{macrocode}
%
% \subsubsection{Input methods}
%
%    \begin{macrocode}
\def\ACCSUPP@method@plain#1{%
  \csname @safe@activestrue\endcsname
  \edef#1{%
    (%
    \ifACCSUPP@unicode
      \string\376\string\377%
    \fi
    #1%
    )%
  }%
  \@onelevel@sanitize#1%
}
%    \end{macrocode}
%    \begin{macrocode}
\def\ACCSUPP@method@escape#1{%
  \EdefEscapeString#1{%
    \ifACCSUPP@unicode
      ^^fe^^ff%
    \fi
    #1%
  }%
  \edef#1{(#1)}%
}%
%    \end{macrocode}
%    \begin{macrocode}
\def\ACCSUPP@method@hex#1{%
  \edef#1{%
    <%
    \ifACCSUPP@unicode
      FEFF%
    \fi
    #1%
    >%
  }%
}
%    \end{macrocode}
%    \begin{macrocode}
\def\ACCSUPP@method@pdfstringdef#1{%
  \ifACCSUPP@unicode
    \@ifundefined{hypersetup}{}{%
      \hypersetup{unicode}%
    }%
  \fi
  \@ifundefined{pdfstringdef}{%
    \PackageError{accsupp}{%
      Method `pdfstringdef' requires package `hyperref'%
    }\@ehc
    \let\ACCSUPP@temp\@empty
  }{%
    \begingroup
      \setbox0=\hbox{%
        \pdfstringdef\ACCSUPP@temp#1%
        \global\let\ACCSUPP@temp\ACCSUPP@temp
      }%
    \endgroup
  }%
  \edef#1{(\ACCSUPP@temp)}%
}
%    \end{macrocode}
%
%    \begin{macrocode}
\ACCSUPP@AtEnd%
%</package>
%    \end{macrocode}
%
% \subsection{Drivers}
%
% \subsubsection{Driver \xoption{pdftex}}
%
%    \begin{macrocode}
%<*pdftex>
\NeedsTeXFormat{LaTeX2e}
\ProvidesFile{accsupp-pdftex.def}%
  [2018/01/30 v0.5 accsupp driver for pdfTeX (HO)]%
%    \end{macrocode}
%
%    \begin{macrocode}
\def\ACCSUPP@bdc{%
  \pdfliteral\ACCSUPP@pdfliteral{\ACCSUPP@span BDC}%
}
\def\ACCSUPP@emc{%
  \pdfliteral\ACCSUPP@pdfliteral{EMC}%
}
%    \end{macrocode}
%    \begin{macrocode}
%</pdftex>
%    \end{macrocode}
%
% \subsubsection{Driver \xoption{luatex}}
%
%    \begin{macrocode}
%<*luatex>
\NeedsTeXFormat{LaTeX2e}
\ProvidesFile{accsupp-luatex.def}%
  [2018/01/30 v0.5 accsupp driver for pdfTeX (HO)]%
%    \end{macrocode}
%
%    \begin{macrocode}
\protected\def\ACCSUPP@pdf@literal{\pdfextension literal}
\def\ACCSUPP@bdc{%
  \ACCSUPP@pdf@literal\ACCSUPP@pdfliteral{\ACCSUPP@span BDC}%
}
\def\ACCSUPP@emc{%
  \ACCSUPP@pdf@literal\ACCSUPP@pdfliteral{EMC}%
}
%    \end{macrocode}
%    \begin{macrocode}
%</luatex>
%    \end{macrocode}
%
% \subsubsection{Driver \xoption{dvipdfm}}
%
%    \begin{macrocode}
%<*dvipdfm>
\NeedsTeXFormat{LaTeX2e}
\ProvidesFile{accsupp-dvipdfm.def}%
  [2018/01/30 v0.5 accsupp driver for dvipdfm (HO)]%
%    \end{macrocode}
%
%    \begin{macrocode}
\def\ACCSUPP@bdc{%
  \special{pdf:content \ACCSUPP@span BDC}%
}
\def\ACCSUPP@emc{%
  \special{pdf:content EMC}%
}
%    \end{macrocode}
%    \begin{macrocode}
%</dvipdfm>
%    \end{macrocode}
%
% \subsubsection{Driver \xoption{dvips}}
%
%    \begin{macrocode}
%<*dvips>
\NeedsTeXFormat{LaTeX2e}
\ProvidesFile{accsupp-dvips.def}%
  [2018/01/30 v0.5 accsupp driver for dvips (HO)]%
%    \end{macrocode}
%
%    \begin{macrocode}
\def\ACCSUPP@bdc{%
  \special{ps:[\ACCSUPP@span/BDC pdfmark}%
}
\def\ACCSUPP@emc{%
  \special{ps:[/EMC pdfmark}%
}
%    \end{macrocode}
%    \begin{macrocode}
%</dvips>
%    \end{macrocode}
%
% \section{Test}
%
% \subsection{Catcode checks for loading}
%
%    \begin{macrocode}
%<*test1>
%    \end{macrocode}
%    \begin{macrocode}
\NeedsTeXFormat{LaTeX2e}
\documentclass{minimal}
\makeatletter
\def\RestoreCatcodes{}
\count@=0 %
\loop
  \edef\RestoreCatcodes{%
    \RestoreCatcodes
    \catcode\the\count@=\the\catcode\count@\relax
  }%
\ifnum\count@<255 %
  \advance\count@\@ne
\repeat

\def\RangeCatcodeInvalid#1#2{%
  \count@=#1\relax
  \loop
    \catcode\count@=15 %
  \ifnum\count@<#2\relax
    \advance\count@\@ne
  \repeat
}
\def\Test{%
  \RangeCatcodeInvalid{0}{47}%
  \RangeCatcodeInvalid{58}{64}%
  \RangeCatcodeInvalid{91}{96}%
  \RangeCatcodeInvalid{123}{127}%
  \catcode`\@=12 %
  \catcode`\\=0 %
  \catcode`\{=1 %
  \catcode`\}=2 %
  \catcode`\#=6 %
  \catcode`\[=12 %
  \catcode`\]=12 %
  \catcode`\%=14 %
  \catcode`\ =10 %
  \catcode13=5 %
  \RequirePackage{accsupp}[2018/01/30]\relax
  \RestoreCatcodes
}
\Test
\csname @@end\endcsname
\end
%    \end{macrocode}
%    \begin{macrocode}
%</test1>
%    \end{macrocode}
%
% \section{Installation}
%
% \subsection{Download}
%
% \paragraph{Package.} This package is available on
% CTAN\footnote{\url{http://ctan.org/pkg/accsupp}}:
% \begin{description}
% \item[\CTAN{macros/latex/contrib/oberdiek/accsupp.dtx}] The source file.
% \item[\CTAN{macros/latex/contrib/oberdiek/accsupp.pdf}] Documentation.
% \end{description}
%
%
% \paragraph{Bundle.} All the packages of the bundle `oberdiek'
% are also available in a TDS compliant ZIP archive. There
% the packages are already unpacked and the documentation files
% are generated. The files and directories obey the TDS standard.
% \begin{description}
% \item[\CTAN{install/macros/latex/contrib/oberdiek.tds.zip}]
% \end{description}
% \emph{TDS} refers to the standard ``A Directory Structure
% for \TeX\ Files'' (\CTAN{tds/tds.pdf}). Directories
% with \xfile{texmf} in their name are usually organized this way.
%
% \subsection{Bundle installation}
%
% \paragraph{Unpacking.} Unpack the \xfile{oberdiek.tds.zip} in the
% TDS tree (also known as \xfile{texmf} tree) of your choice.
% Example (linux):
% \begin{quote}
%   |unzip oberdiek.tds.zip -d ~/texmf|
% \end{quote}
%
% \paragraph{Script installation.}
% Check the directory \xfile{TDS:scripts/oberdiek/} for
% scripts that need further installation steps.
% Package \xpackage{attachfile2} comes with the Perl script
% \xfile{pdfatfi.pl} that should be installed in such a way
% that it can be called as \texttt{pdfatfi}.
% Example (linux):
% \begin{quote}
%   |chmod +x scripts/oberdiek/pdfatfi.pl|\\
%   |cp scripts/oberdiek/pdfatfi.pl /usr/local/bin/|
% \end{quote}
%
% \subsection{Package installation}
%
% \paragraph{Unpacking.} The \xfile{.dtx} file is a self-extracting
% \docstrip\ archive. The files are extracted by running the
% \xfile{.dtx} through \plainTeX:
% \begin{quote}
%   \verb|tex accsupp.dtx|
% \end{quote}
%
% \paragraph{TDS.} Now the different files must be moved into
% the different directories in your installation TDS tree
% (also known as \xfile{texmf} tree):
% \begin{quote}
% \def\t{^^A
% \begin{tabular}{@{}>{\ttfamily}l@{ $\rightarrow$ }>{\ttfamily}l@{}}
%   accsupp.sty & tex/latex/oberdiek/accsupp.sty\\
%   accsupp-pdftex.def & tex/latex/oberdiek/accsupp-pdftex.def\\
%   accsupp-luatex.def & tex/latex/oberdiek/accsupp-luatex.def\\
%   accsupp-dvipdfm.def & tex/latex/oberdiek/accsupp-dvipdfm.def\\
%   accsupp-dvips.def & tex/latex/oberdiek/accsupp-dvips.def\\
%   accsupp.pdf & doc/latex/oberdiek/accsupp.pdf\\
%   accsupp-example1.tex & doc/latex/oberdiek/accsupp-example1.tex\\
%   accsupp-example2.tex & doc/latex/oberdiek/accsupp-example2.tex\\
%   test/accsupp-test1.tex & doc/latex/oberdiek/test/accsupp-test1.tex\\
%   accsupp.dtx & source/latex/oberdiek/accsupp.dtx\\
% \end{tabular}^^A
% }^^A
% \sbox0{\t}^^A
% \ifdim\wd0>\linewidth
%   \begingroup
%     \advance\linewidth by\leftmargin
%     \advance\linewidth by\rightmargin
%   \edef\x{\endgroup
%     \def\noexpand\lw{\the\linewidth}^^A
%   }\x
%   \def\lwbox{^^A
%     \leavevmode
%     \hbox to \linewidth{^^A
%       \kern-\leftmargin\relax
%       \hss
%       \usebox0
%       \hss
%       \kern-\rightmargin\relax
%     }^^A
%   }^^A
%   \ifdim\wd0>\lw
%     \sbox0{\small\t}^^A
%     \ifdim\wd0>\linewidth
%       \ifdim\wd0>\lw
%         \sbox0{\footnotesize\t}^^A
%         \ifdim\wd0>\linewidth
%           \ifdim\wd0>\lw
%             \sbox0{\scriptsize\t}^^A
%             \ifdim\wd0>\linewidth
%               \ifdim\wd0>\lw
%                 \sbox0{\tiny\t}^^A
%                 \ifdim\wd0>\linewidth
%                   \lwbox
%                 \else
%                   \usebox0
%                 \fi
%               \else
%                 \lwbox
%               \fi
%             \else
%               \usebox0
%             \fi
%           \else
%             \lwbox
%           \fi
%         \else
%           \usebox0
%         \fi
%       \else
%         \lwbox
%       \fi
%     \else
%       \usebox0
%     \fi
%   \else
%     \lwbox
%   \fi
% \else
%   \usebox0
% \fi
% \end{quote}
% If you have a \xfile{docstrip.cfg} that configures and enables \docstrip's
% TDS installing feature, then some files can already be in the right
% place, see the documentation of \docstrip.
%
% \subsection{Refresh file name databases}
%
% If your \TeX~distribution
% (\teTeX, \mikTeX, \dots) relies on file name databases, you must refresh
% these. For example, \teTeX\ users run \verb|texhash| or
% \verb|mktexlsr|.
%
% \subsection{Some details for the interested}
%
% \paragraph{Attached source.}
%
% The PDF documentation on CTAN also includes the
% \xfile{.dtx} source file. It can be extracted by
% AcrobatReader 6 or higher. Another option is \textsf{pdftk},
% e.g. unpack the file into the current directory:
% \begin{quote}
%   \verb|pdftk accsupp.pdf unpack_files output .|
% \end{quote}
%
% \paragraph{Unpacking with \LaTeX.}
% The \xfile{.dtx} chooses its action depending on the format:
% \begin{description}
% \item[\plainTeX:] Run \docstrip\ and extract the files.
% \item[\LaTeX:] Generate the documentation.
% \end{description}
% If you insist on using \LaTeX\ for \docstrip\ (really,
% \docstrip\ does not need \LaTeX), then inform the autodetect routine
% about your intention:
% \begin{quote}
%   \verb|latex \let\install=y\input{accsupp.dtx}|
% \end{quote}
% Do not forget to quote the argument according to the demands
% of your shell.
%
% \paragraph{Generating the documentation.}
% You can use both the \xfile{.dtx} or the \xfile{.drv} to generate
% the documentation. The process can be configured by the
% configuration file \xfile{ltxdoc.cfg}. For instance, put this
% line into this file, if you want to have A4 as paper format:
% \begin{quote}
%   \verb|\PassOptionsToClass{a4paper}{article}|
% \end{quote}
% An example follows how to generate the
% documentation with pdf\LaTeX:
% \begin{quote}
%\begin{verbatim}
%pdflatex accsupp.dtx
%makeindex -s gind.ist accsupp.idx
%pdflatex accsupp.dtx
%makeindex -s gind.ist accsupp.idx
%pdflatex accsupp.dtx
%\end{verbatim}
% \end{quote}
%
% \section{Catalogue}
%
% The following XML file can be used as source for the
% \href{http://mirror.ctan.org/help/Catalogue/catalogue.html}{\TeX\ Catalogue}.
% The elements \texttt{caption} and \texttt{description} are imported
% from the original XML file from the Catalogue.
% The name of the XML file in the Catalogue is \xfile{accsupp.xml}.
%    \begin{macrocode}
%<*catalogue>
<?xml version='1.0' encoding='us-ascii'?>
<!DOCTYPE entry SYSTEM 'catalogue.dtd'>
<entry datestamp='$Date$' modifier='$Author$' id='accsupp'>
  <name>accsupp</name>
  <caption>Better accessibility support for PDF files.</caption>
  <authorref id='auth:oberdiek'/>
  <copyright owner='Heiko Oberdiek' year='2007,2010'/>
  <license type='lppl1.3'/>
  <version number='0.4'/>
  <description>
    Since PDF 1.5 portions of a page can be marked for better accessibility
    support. For example, replacement texts or expansions of abbreviations can
    be provided. This package starts with providing a minimal low-level
    interface for programmers; its status is experimental.
    <p/>
    The package is part of the <xref refid='oberdiek'>oberdiek</xref> bundle.
  </description>
  <documentation details='Package documentation'
      href='ctan:/macros/latex/contrib/oberdiek/accsupp.pdf'/>
  <ctan file='true' path='/macros/latex/contrib/oberdiek/accsupp.dtx'/>
  <miktex location='oberdiek'/>
  <texlive location='oberdiek'/>
  <install path='/macros/latex/contrib/oberdiek/oberdiek.tds.zip'/>
</entry>
%</catalogue>
%    \end{macrocode}
%
% \begin{thebibliography}{9}
% \bibitem{pdfspec}
% Adobe Systems Incorporated,
% \textit{PDF Reference}, 6th edition, 2006.
% \url{http://www.adobe.com/devnet/acrobat/pdfs/pdf_reference.pdf}
%
% \end{thebibliography}
%
% \begin{History}
%   \begin{Version}{2007/03/21 v0.1}
%   \item
%     First version.
%   \end{Version}
%   \begin{Version}{2007/11/14 v0.2}
%   \item
%     Various bug fixes.
%   \item
%     Catcode section rewritten, test added.
%   \end{Version}
%   \begin{Version}{2010/01/16 v0.3}
%   \item
%     \cs{AccSuppSetup} added.
%   \item
%     Options \xoption{space}, \xoption{spacefont}, \xoption{spacechar} added.
%   \end{Version}
%   \begin{Version}{2016/05/16 v0.4}
%   \item
%     Documentation updates.
%   \end{Version}
%   \begin{Version}{2018/01/30 v0.5}
%   \item
%     LuaTeX support added.
%   \end{Version}
% \end{History}
%
% \PrintIndex
%
% \Finale
\endinput

%        (quote the arguments according to the demands of your shell)
%
% Documentation:
%    (a) If accsupp.drv is present:
%           latex accsupp.drv
%    (b) Without accsupp.drv:
%           latex accsupp.dtx; ...
%    The class ltxdoc loads the configuration file ltxdoc.cfg
%    if available. Here you can specify further options, e.g.
%    use A4 as paper format:
%       \PassOptionsToClass{a4paper}{article}
%
%    Programm calls to get the documentation (example):
%       pdflatex accsupp.dtx
%       makeindex -s gind.ist accsupp.idx
%       pdflatex accsupp.dtx
%       makeindex -s gind.ist accsupp.idx
%       pdflatex accsupp.dtx
%
% Installation:
%    TDS:tex/latex/oberdiek/accsupp.sty
%    TDS:tex/latex/oberdiek/accsupp-pdftex.def
%    TDS:tex/latex/oberdiek/accsupp-luatex.def
%    TDS:tex/latex/oberdiek/accsupp-dvipdfm.def
%    TDS:tex/latex/oberdiek/accsupp-dvips.def
%    TDS:doc/latex/oberdiek/accsupp.pdf
%    TDS:doc/latex/oberdiek/accsupp-example1.tex
%    TDS:doc/latex/oberdiek/accsupp-example2.tex
%    TDS:doc/latex/oberdiek/test/accsupp-test1.tex
%    TDS:source/latex/oberdiek/accsupp.dtx
%
%<*ignore>
\begingroup
  \catcode123=1 %
  \catcode125=2 %
  \def\x{LaTeX2e}%
\expandafter\endgroup
\ifcase 0\ifx\install y1\fi\expandafter
         \ifx\csname processbatchFile\endcsname\relax\else1\fi
         \ifx\fmtname\x\else 1\fi\relax
\else\csname fi\endcsname
%</ignore>
%<*install>
\input docstrip.tex
\Msg{************************************************************************}
\Msg{* Installation}
\Msg{* Package: accsupp 2018/01/30 v0.5 Accessibility support by marked content (HO)}
\Msg{************************************************************************}

\keepsilent
\askforoverwritefalse

\let\MetaPrefix\relax
\preamble

This is a generated file.

Project: accsupp
Version: 2018/01/30 v0.5

Copyright (C) 2007, 2010 by
   Heiko Oberdiek <heiko.oberdiek at googlemail.com>

This work may be distributed and/or modified under the
conditions of the LaTeX Project Public License, either
version 1.3c of this license or (at your option) any later
version. This version of this license is in
   http://www.latex-project.org/lppl/lppl-1-3c.txt
and the latest version of this license is in
   http://www.latex-project.org/lppl.txt
and version 1.3 or later is part of all distributions of
LaTeX version 2005/12/01 or later.

This work has the LPPL maintenance status "maintained".

This Current Maintainer of this work is Heiko Oberdiek.

This work consists of the main source file accsupp.dtx
and the derived files
   accsupp.sty, accsupp.pdf, accsupp.ins, accsupp.drv,
   accsupp-pdftex.def, accsupp-luatex.def, accsupp-dvipdfm.def,
   accsupp-dvips.def,
   accsupp-example1.tex, accsupp-example2.tex,
   accsupp-test1.tex.

\endpreamble
\let\MetaPrefix\DoubleperCent

\generate{%
  \file{accsupp.ins}{\from{accsupp.dtx}{install}}%
  \file{accsupp.drv}{\from{accsupp.dtx}{driver}}%
  \usedir{tex/latex/oberdiek}%
  \file{accsupp.sty}{\from{accsupp.dtx}{package}}%
  \file{accsupp-pdftex.def}{\from{accsupp.dtx}{pdftex}}%
  \file{accsupp-luatex.def}{\from{accsupp.dtx}{luatex}}%
  \file{accsupp-dvipdfm.def}{\from{accsupp.dtx}{dvipdfm}}%
  \file{accsupp-dvips.def}{\from{accsupp.dtx}{dvips}}%
  \usedir{doc/latex/oberdiek}%
  \file{accsupp-example1.tex}{\from{accsupp.dtx}{example1}}%
  \file{accsupp-example2.tex}{\from{accsupp.dtx}{example2}}%
  \usedir{doc/latex/oberdiek/test}%
  \file{accsupp-test1.tex}{\from{accsupp.dtx}{test1}}%
  \nopreamble
  \nopostamble
  \usedir{source/latex/oberdiek/catalogue}%
  \file{accsupp.xml}{\from{accsupp.dtx}{catalogue}}%
}

\catcode32=13\relax% active space
\let =\space%
\Msg{************************************************************************}
\Msg{*}
\Msg{* To finish the installation you have to move the following}
\Msg{* files into a directory searched by TeX:}
\Msg{*}
\Msg{*     accsupp.sty, accsupp-pdftex.def, accsupp-dvipdfm.def,}
\Msg{*     accsupp-dvips.def}
\Msg{*}
\Msg{* To produce the documentation run the file `accsupp.drv'}
\Msg{* through LaTeX.}
\Msg{*}
\Msg{* Happy TeXing!}
\Msg{*}
\Msg{************************************************************************}

\endbatchfile
%</install>
%<*ignore>
\fi
%</ignore>
%<*driver>
\NeedsTeXFormat{LaTeX2e}
\ProvidesFile{accsupp.drv}%
  [2018/01/30 v0.5 Accessibility support by marked content (HO)]%
\documentclass{ltxdoc}
\usepackage{holtxdoc}[2011/11/22]
\begin{document}
  \DocInput{accsupp.dtx}%
\end{document}
%</driver>
% \fi
%
%
% \CharacterTable
%  {Upper-case    \A\B\C\D\E\F\G\H\I\J\K\L\M\N\O\P\Q\R\S\T\U\V\W\X\Y\Z
%   Lower-case    \a\b\c\d\e\f\g\h\i\j\k\l\m\n\o\p\q\r\s\t\u\v\w\x\y\z
%   Digits        \0\1\2\3\4\5\6\7\8\9
%   Exclamation   \!     Double quote  \"     Hash (number) \#
%   Dollar        \$     Percent       \%     Ampersand     \&
%   Acute accent  \'     Left paren    \(     Right paren   \)
%   Asterisk      \*     Plus          \+     Comma         \,
%   Minus         \-     Point         \.     Solidus       \/
%   Colon         \:     Semicolon     \;     Less than     \<
%   Equals        \=     Greater than  \>     Question mark \?
%   Commercial at \@     Left bracket  \[     Backslash     \\
%   Right bracket \]     Circumflex    \^     Underscore    \_
%   Grave accent  \`     Left brace    \{     Vertical bar  \|
%   Right brace   \}     Tilde         \~}
%
% \GetFileInfo{accsupp.drv}
%
% \title{The \xpackage{accsupp} package}
% \date{2018/01/30 v0.5}
% \author{Heiko Oberdiek\thanks
% {Please report any issues at https://github.com/ho-tex/oberdiek/issues}\\
% \xemail{heiko.oberdiek at googlemail.com}}
%
% \maketitle
%
% \begin{abstract}
% Since PDF 1.5 portions of a page can be marked for better
% accessibility support.
% For example, replacement texts or expansions of abbreviations can be
% provided. Package \xpackage{accsupp} starts with providing a minimal
% low-level interface for programmers. Status is experimental.
% \end{abstract}
%
% \tableofcontents
%
% \section{Documentation}
%
% \subsection{Macros}
%
% Section ``10.8 Accessibility Support'' of the PDF reference
% \cite{pdfspec} lists some features that can be added by
% operators for marked content.
%
% \begin{declcs}{BeginAccSupp} \M{options}
% \end{declcs}
% It puts the operator |BDC| in the page stream:
% \begin{quote}
%   |/Span|\\
%   |<<|\dots|>>| \qquad \textit{\% property dictionary}\\
%   |BDC|
% \end{quote}
% The contents of the dictionary is controlled by \meta{options}.
% See sections \ref{sec:feature} and \ref{sec:method}.
%
% \begin{declcs}{EndAccSupp} \M{options}
% \end{declcs}
% It puts the operator |EMC| in the page stream.
% The only option is \xoption{pdfliteral}, see section \ref{sec:pdftex}.
%
% \bigskip
% \noindent\textbf{Note:}
% The caller is responsible for the placement of \cs{BeginAccSupp}
% and \cs{EndAccSupp} pairs. Especially page breaks are not allowed
% in between.
%
% \subsubsection{Feature options}\label{sec:feature}
%
% The PDF reference \cite{pdfspec} describes and explains the different
% features. The names of the feature options are the same as the
% key names for the property dictionary for operator |BDC|, see
% \cs{BeginAccSupp}.
%
% \begin{description}
% \item[\xoption{ActualText}:]
%   Provides a replacement text, see examples in
%   section \ref{sec:example}.
% \item[\xoption{Alt}:]
%   Provides an alternate description.
% \item[\xoption{E}:]
%   Provides the expansion of an abbreviation or an acronym.
% \item[\xoption{Lang}:]
%   Specifies the language.
% \end{description}
%
% \subsubsection{Input methods}\label{sec:method}
%
% Except for \xoption{Lang} option \xoption{method} controls
% how the argument for \xoption{ActualText}, \xoption{Alt},
% and \xoption{E} are interpreted.
% \begin{description}
% \item[\xoption{method=plain}:]
%   The string is only expanded and written without further
%   treatment. Special characters are not protected, thus this method
%   may result in an invalid PDF file.
% \item[\xoption{mehod=escape}:]
%   The string is expanded and special characters are escaped.
%   The result is a valid PDF string.
% \item[\xoption{method=hex}:]
%   The string is given in hexadecimal notation.
%   Section \ref{sec:example:notparallel} shows an example.
% \item[\xoption{method=pdfstringdef}:]
%   If package \xpackage{hyperref} is loaded, then its \cs{pdfstringdef}
%   is used. This method is slow, but useful if the string contains
%   arbitrary \hologo{TeX} code.
% \item[\xoption{unicode}:]
%   This option is needed, if the string is given as Unicode string
%   (16 bit). Internally it adds the string prefix for Unicode.
%   In case of \xoption{method=pdfstringdef} it passes the option
%   to \cs{hypersetup}.
% \end{description}
%
% \subsection{Workaround, option \xoption{space}}
%
% PDF specification says in ``10.8.3 Replacement Text''
% \begin{quote}
% Just as alternative descriptions can be provided for images and other
% items that do not translate naturally into text (\dots), replacement
% text can be specified for content that does translate into text but
% that is represented in a nonstandard way. These nonstandard representations
% might include, for example, glyphs for ligatures or custom characters, or
% inline graphics corresponding to letters in an illuminated manustript or to
% dropped capitals.
% \end{quote}
% However, the ActualText is ignored in Acrobat Reader
% (until version 9 at least), if the content does not contain glyphs.
% Option \xoption{space} adds such an invisible glyph, a space
% character. The font name can be configured by option \xoption{spacefont},
% the default font is \texttt{phvr8r}. The character slot is given
% by option \xoption{spacechar}, default is \texttt{32}, the usual
% position of the space character.
%
% These options can also be given as package options or in
% macro \cs{AccSuppSetup} that takes a key value list as argument.
% Usually only option \xoption{space} is necessary, if the
% contents does not contain glyphs otherwise. Option \xoption{space}
% is enabled by |space| or |space=true| and disabled by |space=false|.
% It is disabled as default. The option is evaluated by \cs{BeginAccSupp}
% and ignored by \cs{EndAccSupp}.
%
% Note: Even with option \xoption{space} I get sometimes wrong texts
% when cut \& paste from AR7/Linux or AR8/Linux, e.g.
% \begin{quote}
%   |Hello| $\rightarrow$ |Helo|, |Helol|, |Hell|, \dots\\
%   |Hello World| $\rightarrow$ |Helo WorldW|, |Helo World|, \dots
% \end{quote}
% I do not know what Acrobat Reader is doing here, thus feedback and
% insights are welcome.
%
% \subsection{Driver options}
%
% Driver options are package options only.
% The special \hologo{TeX} compilers \hologo{pdfTeX}, \hologo{LuaTeX} and \hologo{XeTeX}
% are detected automatically. The default for unrecognized drivers can be
% set by defining \cs{ActualTextDriverDefault}. This can be done
% in the configuration file \xfile{accsupp.cfg}.
%
% \subsubsection{Option \xoption{pdftex}}\label{sec:pdftex}
% Package option \xoption{pdftex} is used for \hologo{pdfTeX} in PDF mode.
% Additionally \cs{BeginAccSupp} and \cs{EndAccSupp} understand
% option \xoption{pdfliteral}. It controls
% the modifier keyword for \cs{pdfliteral}:
% \begin{quote}
%   |pdfliteral=direct| $\Rightarrow$ |\pdfliteral direct{...}|
% \end{quote}
%
% \subsubsection{Option \xoption{luatex}}\label{sec:luatex}
% Package option \xoption{luatex} is used for \hologo{LuaTeX} in PDF mode.
% Additionally \cs{BeginAccSupp} and \cs{EndAccSupp} understand
% option \xoption{pdfliteral}. It controls
% the modifier keyword for \cs{pdfextension} literal:
% \begin{quote}
%   |pdfliteral=direct| $\Rightarrow$ |\pdfextension literal direct{...}|
% \end{quote}
%
% \subsubsection{Option \xoption{dvipdfm}}
%
% Package option \xoption{dvipdfm} and its aliases \xoption{dvipdfmx}
% \xoption{xetex} are used for drivers that support
% \textsf{dvipdfm} specials.
%
% \subsubsection{Option \xoption{dvips}}
%
% Package option \xoption{dvips} and its alias \xoption{dvipsone}
% write pdfmark specials in the output. Unhappily these pdfmark
% operators are ignored by \textsf{ghostscript} (latest tested
% version is 8.54). Perhaps they are recognized by commercial
% distiller applications.
%
% \subsubsection{Turning off page stream compression}
%
% For debugging it is useful to have uncompressed page stream objects.
% This can be done afterwards via \textsf{pdftk}:
% \begin{quote}
%   |pdftk file.pdf output file-uncompressed.pdf uncompress|
% \end{quote}
% Or the PDF file is generated uncompressed:
% \begin{description}
% \item[\hologo{pdfTeX}:] |\pdfcompresslevel=0|
% \item[dvipdfm:] |dvipdfm -z0 ...|
% \item[dvipdfmx:] |dvipdfmx -z0 ...|
% \item[ghostscript/ps2pdf:]
%      |ps2pdf -dCompressPages=false input.ps output.pdf|
% \end{description}
%
% \section{Example}
% \label{sec:example}
%
% \subsection{Example \cs{notparallel}}
% \label{sec:example:notparallel}
%
%    \begin{macrocode}
%<*example1>
%    \end{macrocode}
%    \begin{macrocode}
%<<END
\documentclass{article}
\usepackage{accsupp}[2007/11/14]
\usepackage{centernot}
% U+2226 NOT PARALLEL
% \mathrel{...} prevents page break in between
\newcommand*{\notparallel}{%
  \ensuremath{%
    \mathrel{%
      \BeginAccSupp{method=hex,unicode,ActualText=2226}%
      \centernot{\parallel}%
      \EndAccSupp{}%
    }%
  }%
}
\begin{document}
\begin{equation}
A\notparallel B
\end{equation}
\end{document}
%END
%    \end{macrocode}
%    \begin{macrocode}
%</example1>
%    \end{macrocode}
%
% \subsection{Example with \xoption{pdfstringdef}}
%
%    \begin{macrocode}
%<*example2>
%    \end{macrocode}
%    \begin{macrocode}
%<<END
\documentclass{article}
\usepackage[unicode]{hyperref}
\usepackage{accsupp}[2007/11/14]
\begin{document}
  \begin{equation}
    \BeginAccSupp{
      method=pdfstringdef,
      unicode,
      ActualText={%
        a\texttwosuperior +b\texttwosuperior
        =c\texttwosuperior
      }
    }
    a^2 + b^2 = c^2
    \EndAccSupp{}
  \end{equation}
\end{document}
%END
%    \end{macrocode}
%    \begin{macrocode}
%</example2>
%    \end{macrocode}
%
% \StopEventually{
% }
%
% \section{Implementation}
%
% \subsection{Package}
%
%    \begin{macrocode}
%<*package>
%    \end{macrocode}
%
%    \begin{macrocode}
\begingroup\catcode61\catcode48\catcode32=10\relax%
  \catcode13=5 % ^^M
  \endlinechar=13 %
  \catcode123=1 % {
  \catcode125=2 % }
  \catcode64=11 % @
  \def\x{\endgroup
    \expandafter\edef\csname ACCSUPP@AtEnd\endcsname{%
      \endlinechar=\the\endlinechar\relax
      \catcode13=\the\catcode13\relax
      \catcode32=\the\catcode32\relax
      \catcode35=\the\catcode35\relax
      \catcode61=\the\catcode61\relax
      \catcode64=\the\catcode64\relax
      \catcode123=\the\catcode123\relax
      \catcode125=\the\catcode125\relax
    }%
  }%
\x\catcode61\catcode48\catcode32=10\relax%
\catcode13=5 % ^^M
\endlinechar=13 %
\catcode35=6 % #
\catcode64=11 % @
\catcode123=1 % {
\catcode125=2 % }
\def\TMP@EnsureCode#1#2{%
  \edef\ACCSUPP@AtEnd{%
    \ACCSUPP@AtEnd
    \catcode#1=\the\catcode#1\relax
  }%
  \catcode#1=#2\relax
}
\TMP@EnsureCode{10}{12}% ^^J
\TMP@EnsureCode{33}{12}% !
\TMP@EnsureCode{39}{12}% '
\TMP@EnsureCode{40}{12}% (
\TMP@EnsureCode{41}{12}% )
\TMP@EnsureCode{42}{12}% *
\TMP@EnsureCode{44}{12}% ,
\TMP@EnsureCode{45}{12}% -
\TMP@EnsureCode{46}{12}% .
\TMP@EnsureCode{47}{12}% /
\TMP@EnsureCode{58}{12}% :
\TMP@EnsureCode{60}{12}% <
\TMP@EnsureCode{62}{12}% >
\TMP@EnsureCode{94}{7}% ^ (superscript)
\TMP@EnsureCode{96}{12}% `
\TMP@EnsureCode{254}{12}% ^^fe
\TMP@EnsureCode{255}{12}% ^^ff
\edef\ACCSUPP@AtEnd{\ACCSUPP@AtEnd\noexpand\endinput}
%    \end{macrocode}
%
%    Package identification.
%    \begin{macrocode}
\NeedsTeXFormat{LaTeX2e}
\ProvidesPackage{accsupp}%
  [2018/01/30 v0.5 Accessibility support by marked content (HO)]
%    \end{macrocode}
%
%    \begin{macrocode}
\RequirePackage{pdfescape}[2007/02/25]
\RequirePackage{ifpdf}
\RequirePackage{ifxetex}
\RequirePackage{kvoptions}
%    \end{macrocode}
%    \begin{macrocode}
\SetupKeyvalOptions{%
  family=ACCSUPP,%
  prefix=ACCSUPP@%
}
%    \end{macrocode}
%
% \subsection{Space setup}
%
%    \begin{macrocode}
\DeclareBoolOption{space}
\DeclareStringOption[phvr8r]{spacefont}
\DeclareStringOption[32]{spacechar}
%    \end{macrocode}
%
% \subsection{Driver detection and setup}
%
% \paragraph{Driver declarations.}
%
%    \begin{macrocode}
\def\ACCSUPP@DefineDriverKey{%
  \@dblarg\ACCSUPP@@DefineDriverKey
}
\def\ACCSUPP@@DefineDriverKey[#1]#2{%
  \define@key{ACCSUPP}{#2}[]{%
    \def\ACCSUPP@driver{#1}%
  }%
  \g@addto@macro\ACCSUPP@DisableOptions{%
    \DisableKeyvalOption{ACCSUPP}{#2}%
  }%
}
\let\ACCSUPP@DisableOptions\@empty
\ACCSUPP@DefineDriverKey{pdftex}
\ACCSUPP@DefineDriverKey{luatex}
\ACCSUPP@DefineDriverKey{dvips}
\ACCSUPP@DefineDriverKey[dvips]{dvipsone}
\ACCSUPP@DefineDriverKey{dvipdfm}
\ACCSUPP@DefineDriverKey[dvipdfm]{dvipdfmx}
\ACCSUPP@DefineDriverKey[dvipdfm]{xetex}
\let\ACCSUPP@driver\relax
\InputIfFileExists{accsupp.cfg}{}{}
\providecommand*{\ActualTextDriverDefault}{dvips}
\ifpdf
  \ifx\pdfextension\@undefined
    \def\ACCSUPP@driver{pdftex}%
  \else
    \def\ACCSUPP@driver{luatex}%
  \fi
\else
  \ifxetex
    \def\ACCSUPP@driver{dvipdfm}%
  \else
    \ifx\ACCSUPP@driver\relax
      \let\ACCSUPP@driver\ActualTextDriverDefault
    \fi
  \fi
\fi
%    \end{macrocode}
%
% \paragraph{Process options.}
%
%    \begin{macrocode}
\ProcessKeyvalOptions*
\ACCSUPP@DisableOptions
%    \end{macrocode}
%
% \paragraph{Driver validation and loading.}
%
%    \begin{macrocode}
\def\ACCSUPP@temp{pdftex}%
\ifpdf
  \ifx\pdfextension\@undefined\else
    \def\ACCSUPP@temp{luatex}%
  \fi
  \ifx\ACCSUPP@temp\ACCSUPP@driver
  \else
    \PackageWarningNoLine{accsupp}{%
      Wrong driver `\ACCSUPP@driver', using `pdftex' instead%
    }%
ed as
    \let\ACCSUPP@driver\ACCSUPP@temp
  \fi
\else
  \ifx\ACCSUPP@temp\ACCSUPP@driver
    \PackageError{accsupp}{%
      Wrong driver, pdfTeX is not running in PDF mode.\MessageBreak
      Package loading is aborted%
    }\@ehc
    \expandafter\expandafter\expandafter\ACCSUPP@AtEnd
  \fi
  \def\ACCSUPP@temp{dvipdfm}%
  \ifxetex
    \ifx\ACCSUPP@temp\ACCSUPP@driver
    \else
      \PackageWarningNoLine{accsupp}{%
        Wrong driver `\ACCSUPP@driver',\MessageBreak
        using `dvipdfm' for XeTeX instead%
      }%
      \let\ACCSUPP@driver\ACCSUPP@temp
    \fi
  \fi
\fi%
\ifx\ACCSUPP@driver\relax
  \PackageError{accsupp}{%
    Missing driver option.\MessageBreak
    Package loading is aborted%
  }\@ehc
  \expandafter\ACCSUPP@AtEnd
\fi%
\InputIfFileExists{accsupp-\ACCSUPP@driver.def}{}{%
  \PackageError{accsupp}{%
    Unsupported driver `\ACCSUPP@driver'.\MessageBreak
    Package loading is aborted%
  }\@ehc
  \ACCSUPP@AtEnd
}%
%    \end{macrocode}
%
% \subsection{Main macro}
%
%    \begin{macrocode}
\DeclareBoolOption{unicode}
\DeclareStringOption[page]{pdfliteral}
\DeclareStringOption{Lang}
\def\ACCSUPP@method{escape}
\define@key{ACCSUPP}{method}{%
  \@ifundefined{ACCSUPP@method@#1}{%
    \PackageError{accsupp}{%
      Ignoring unknown method `#1'%
    }\@ehc
  }{%
    \edef\ACCSUPP@method{#1}%
  }%
}
\let\ACCSUPP@Lang\relax
\def\ACCSUPP@temp#1{%
  \expandafter\ACCSUPP@@temp\csname ACCSUPP@#1\endcsname{#1}%
}
\def\ACCSUPP@@temp#1#2{%
  \let#1\relax
  \define@key{ACCSUPP}{#2}{%
    \def#1{##1}%
    \ifx#1\@empty
      \def#1{()}%
    \else
      \csname ACCSUPP@method@\ACCSUPP@method\endcsname#1%
    \fi
  }%
}
\ACCSUPP@temp{Alt}
\ACCSUPP@temp{ActualText}
\ACCSUPP@temp{E}
%    \end{macrocode}
%
%    \begin{macrocode}
\newcommand*{\BeginAccSupp}[1]{%
  \begingroup
    \setkeys{ACCSUPP}{#1}%
    \edef\ACCSUPP@span{%
      /Span<<%
        \ifx\ACCSUPP@Lang\relax
        \else
          /Lang\ACCSUPP@Lang
        \fi
        \ifx\ACCSUPP@Alt\relax
        \else
          /Alt\ACCSUPP@Alt
        \fi
        \ifx\ACCSUPP@ActualText\relax
        \else
          /ActualText\ACCSUPP@ActualText
        \fi
        \ifx\ACCSUPP@E\relax
        \else
          /E\ACCSUPP@E
        \fi
      >>%
    }%
    \ACCSUPP@bdc
    \ACCSUPP@space
  \endgroup
}
\newcommand*{\EndAccSupp}[1]{%
  \begingroup
    \setkeys{ACCSUPP}{#1}%
    \ACCSUPP@emc
  \endgroup
}
\def\ACCSUPP@space{%
  \ifACCSUPP@space
    \begingroup
      \@ifundefined{ACCSUPP@Font}{%
        \global\font\ACCSUPP@Font=\ACCSUPP@spacefont\relax
      }{}%
      \leavevmode
      \setbox\z@\hbox{\ACCSUPP@Font\char\ACCSUPP@spacechar}%
      \wd\z@\z@
      \ht\z@\z@
      \dp\z@\z@
      \copy\z@
    \endgroup
  \fi
}
\newcommand*{\AccSuppSetup}{%
  \setkeys{ACCSUPP}%
}
%    \end{macrocode}
%
% \subsubsection{Input methods}
%
%    \begin{macrocode}
\def\ACCSUPP@method@plain#1{%
  \csname @safe@activestrue\endcsname
  \edef#1{%
    (%
    \ifACCSUPP@unicode
      \string\376\string\377%
    \fi
    #1%
    )%
  }%
  \@onelevel@sanitize#1%
}
%    \end{macrocode}
%    \begin{macrocode}
\def\ACCSUPP@method@escape#1{%
  \EdefEscapeString#1{%
    \ifACCSUPP@unicode
      ^^fe^^ff%
    \fi
    #1%
  }%
  \edef#1{(#1)}%
}%
%    \end{macrocode}
%    \begin{macrocode}
\def\ACCSUPP@method@hex#1{%
  \edef#1{%
    <%
    \ifACCSUPP@unicode
      FEFF%
    \fi
    #1%
    >%
  }%
}
%    \end{macrocode}
%    \begin{macrocode}
\def\ACCSUPP@method@pdfstringdef#1{%
  \ifACCSUPP@unicode
    \@ifundefined{hypersetup}{}{%
      \hypersetup{unicode}%
    }%
  \fi
  \@ifundefined{pdfstringdef}{%
    \PackageError{accsupp}{%
      Method `pdfstringdef' requires package `hyperref'%
    }\@ehc
    \let\ACCSUPP@temp\@empty
  }{%
    \begingroup
      \setbox0=\hbox{%
        \pdfstringdef\ACCSUPP@temp#1%
        \global\let\ACCSUPP@temp\ACCSUPP@temp
      }%
    \endgroup
  }%
  \edef#1{(\ACCSUPP@temp)}%
}
%    \end{macrocode}
%
%    \begin{macrocode}
\ACCSUPP@AtEnd%
%</package>
%    \end{macrocode}
%
% \subsection{Drivers}
%
% \subsubsection{Driver \xoption{pdftex}}
%
%    \begin{macrocode}
%<*pdftex>
\NeedsTeXFormat{LaTeX2e}
\ProvidesFile{accsupp-pdftex.def}%
  [2018/01/30 v0.5 accsupp driver for pdfTeX (HO)]%
%    \end{macrocode}
%
%    \begin{macrocode}
\def\ACCSUPP@bdc{%
  \pdfliteral\ACCSUPP@pdfliteral{\ACCSUPP@span BDC}%
}
\def\ACCSUPP@emc{%
  \pdfliteral\ACCSUPP@pdfliteral{EMC}%
}
%    \end{macrocode}
%    \begin{macrocode}
%</pdftex>
%    \end{macrocode}
%
% \subsubsection{Driver \xoption{luatex}}
%
%    \begin{macrocode}
%<*luatex>
\NeedsTeXFormat{LaTeX2e}
\ProvidesFile{accsupp-luatex.def}%
  [2018/01/30 v0.5 accsupp driver for pdfTeX (HO)]%
%    \end{macrocode}
%
%    \begin{macrocode}
\protected\def\ACCSUPP@pdf@literal{\pdfextension literal}
\def\ACCSUPP@bdc{%
  \ACCSUPP@pdf@literal\ACCSUPP@pdfliteral{\ACCSUPP@span BDC}%
}
\def\ACCSUPP@emc{%
  \ACCSUPP@pdf@literal\ACCSUPP@pdfliteral{EMC}%
}
%    \end{macrocode}
%    \begin{macrocode}
%</luatex>
%    \end{macrocode}
%
% \subsubsection{Driver \xoption{dvipdfm}}
%
%    \begin{macrocode}
%<*dvipdfm>
\NeedsTeXFormat{LaTeX2e}
\ProvidesFile{accsupp-dvipdfm.def}%
  [2018/01/30 v0.5 accsupp driver for dvipdfm (HO)]%
%    \end{macrocode}
%
%    \begin{macrocode}
\def\ACCSUPP@bdc{%
  \special{pdf:content \ACCSUPP@span BDC}%
}
\def\ACCSUPP@emc{%
  \special{pdf:content EMC}%
}
%    \end{macrocode}
%    \begin{macrocode}
%</dvipdfm>
%    \end{macrocode}
%
% \subsubsection{Driver \xoption{dvips}}
%
%    \begin{macrocode}
%<*dvips>
\NeedsTeXFormat{LaTeX2e}
\ProvidesFile{accsupp-dvips.def}%
  [2018/01/30 v0.5 accsupp driver for dvips (HO)]%
%    \end{macrocode}
%
%    \begin{macrocode}
\def\ACCSUPP@bdc{%
  \special{ps:[\ACCSUPP@span/BDC pdfmark}%
}
\def\ACCSUPP@emc{%
  \special{ps:[/EMC pdfmark}%
}
%    \end{macrocode}
%    \begin{macrocode}
%</dvips>
%    \end{macrocode}
%
% \section{Test}
%
% \subsection{Catcode checks for loading}
%
%    \begin{macrocode}
%<*test1>
%    \end{macrocode}
%    \begin{macrocode}
\NeedsTeXFormat{LaTeX2e}
\documentclass{minimal}
\makeatletter
\def\RestoreCatcodes{}
\count@=0 %
\loop
  \edef\RestoreCatcodes{%
    \RestoreCatcodes
    \catcode\the\count@=\the\catcode\count@\relax
  }%
\ifnum\count@<255 %
  \advance\count@\@ne
\repeat

\def\RangeCatcodeInvalid#1#2{%
  \count@=#1\relax
  \loop
    \catcode\count@=15 %
  \ifnum\count@<#2\relax
    \advance\count@\@ne
  \repeat
}
\def\Test{%
  \RangeCatcodeInvalid{0}{47}%
  \RangeCatcodeInvalid{58}{64}%
  \RangeCatcodeInvalid{91}{96}%
  \RangeCatcodeInvalid{123}{127}%
  \catcode`\@=12 %
  \catcode`\\=0 %
  \catcode`\{=1 %
  \catcode`\}=2 %
  \catcode`\#=6 %
  \catcode`\[=12 %
  \catcode`\]=12 %
  \catcode`\%=14 %
  \catcode`\ =10 %
  \catcode13=5 %
  \RequirePackage{accsupp}[2018/01/30]\relax
  \RestoreCatcodes
}
\Test
\csname @@end\endcsname
\end
%    \end{macrocode}
%    \begin{macrocode}
%</test1>
%    \end{macrocode}
%
% \section{Installation}
%
% \subsection{Download}
%
% \paragraph{Package.} This package is available on
% CTAN\footnote{\url{http://ctan.org/pkg/accsupp}}:
% \begin{description}
% \item[\CTAN{macros/latex/contrib/oberdiek/accsupp.dtx}] The source file.
% \item[\CTAN{macros/latex/contrib/oberdiek/accsupp.pdf}] Documentation.
% \end{description}
%
%
% \paragraph{Bundle.} All the packages of the bundle `oberdiek'
% are also available in a TDS compliant ZIP archive. There
% the packages are already unpacked and the documentation files
% are generated. The files and directories obey the TDS standard.
% \begin{description}
% \item[\CTAN{install/macros/latex/contrib/oberdiek.tds.zip}]
% \end{description}
% \emph{TDS} refers to the standard ``A Directory Structure
% for \TeX\ Files'' (\CTAN{tds/tds.pdf}). Directories
% with \xfile{texmf} in their name are usually organized this way.
%
% \subsection{Bundle installation}
%
% \paragraph{Unpacking.} Unpack the \xfile{oberdiek.tds.zip} in the
% TDS tree (also known as \xfile{texmf} tree) of your choice.
% Example (linux):
% \begin{quote}
%   |unzip oberdiek.tds.zip -d ~/texmf|
% \end{quote}
%
% \paragraph{Script installation.}
% Check the directory \xfile{TDS:scripts/oberdiek/} for
% scripts that need further installation steps.
% Package \xpackage{attachfile2} comes with the Perl script
% \xfile{pdfatfi.pl} that should be installed in such a way
% that it can be called as \texttt{pdfatfi}.
% Example (linux):
% \begin{quote}
%   |chmod +x scripts/oberdiek/pdfatfi.pl|\\
%   |cp scripts/oberdiek/pdfatfi.pl /usr/local/bin/|
% \end{quote}
%
% \subsection{Package installation}
%
% \paragraph{Unpacking.} The \xfile{.dtx} file is a self-extracting
% \docstrip\ archive. The files are extracted by running the
% \xfile{.dtx} through \plainTeX:
% \begin{quote}
%   \verb|tex accsupp.dtx|
% \end{quote}
%
% \paragraph{TDS.} Now the different files must be moved into
% the different directories in your installation TDS tree
% (also known as \xfile{texmf} tree):
% \begin{quote}
% \def\t{^^A
% \begin{tabular}{@{}>{\ttfamily}l@{ $\rightarrow$ }>{\ttfamily}l@{}}
%   accsupp.sty & tex/latex/oberdiek/accsupp.sty\\
%   accsupp-pdftex.def & tex/latex/oberdiek/accsupp-pdftex.def\\
%   accsupp-luatex.def & tex/latex/oberdiek/accsupp-luatex.def\\
%   accsupp-dvipdfm.def & tex/latex/oberdiek/accsupp-dvipdfm.def\\
%   accsupp-dvips.def & tex/latex/oberdiek/accsupp-dvips.def\\
%   accsupp.pdf & doc/latex/oberdiek/accsupp.pdf\\
%   accsupp-example1.tex & doc/latex/oberdiek/accsupp-example1.tex\\
%   accsupp-example2.tex & doc/latex/oberdiek/accsupp-example2.tex\\
%   test/accsupp-test1.tex & doc/latex/oberdiek/test/accsupp-test1.tex\\
%   accsupp.dtx & source/latex/oberdiek/accsupp.dtx\\
% \end{tabular}^^A
% }^^A
% \sbox0{\t}^^A
% \ifdim\wd0>\linewidth
%   \begingroup
%     \advance\linewidth by\leftmargin
%     \advance\linewidth by\rightmargin
%   \edef\x{\endgroup
%     \def\noexpand\lw{\the\linewidth}^^A
%   }\x
%   \def\lwbox{^^A
%     \leavevmode
%     \hbox to \linewidth{^^A
%       \kern-\leftmargin\relax
%       \hss
%       \usebox0
%       \hss
%       \kern-\rightmargin\relax
%     }^^A
%   }^^A
%   \ifdim\wd0>\lw
%     \sbox0{\small\t}^^A
%     \ifdim\wd0>\linewidth
%       \ifdim\wd0>\lw
%         \sbox0{\footnotesize\t}^^A
%         \ifdim\wd0>\linewidth
%           \ifdim\wd0>\lw
%             \sbox0{\scriptsize\t}^^A
%             \ifdim\wd0>\linewidth
%               \ifdim\wd0>\lw
%                 \sbox0{\tiny\t}^^A
%                 \ifdim\wd0>\linewidth
%                   \lwbox
%                 \else
%                   \usebox0
%                 \fi
%               \else
%                 \lwbox
%               \fi
%             \else
%               \usebox0
%             \fi
%           \else
%             \lwbox
%           \fi
%         \else
%           \usebox0
%         \fi
%       \else
%         \lwbox
%       \fi
%     \else
%       \usebox0
%     \fi
%   \else
%     \lwbox
%   \fi
% \else
%   \usebox0
% \fi
% \end{quote}
% If you have a \xfile{docstrip.cfg} that configures and enables \docstrip's
% TDS installing feature, then some files can already be in the right
% place, see the documentation of \docstrip.
%
% \subsection{Refresh file name databases}
%
% If your \TeX~distribution
% (\teTeX, \mikTeX, \dots) relies on file name databases, you must refresh
% these. For example, \teTeX\ users run \verb|texhash| or
% \verb|mktexlsr|.
%
% \subsection{Some details for the interested}
%
% \paragraph{Attached source.}
%
% The PDF documentation on CTAN also includes the
% \xfile{.dtx} source file. It can be extracted by
% AcrobatReader 6 or higher. Another option is \textsf{pdftk},
% e.g. unpack the file into the current directory:
% \begin{quote}
%   \verb|pdftk accsupp.pdf unpack_files output .|
% \end{quote}
%
% \paragraph{Unpacking with \LaTeX.}
% The \xfile{.dtx} chooses its action depending on the format:
% \begin{description}
% \item[\plainTeX:] Run \docstrip\ and extract the files.
% \item[\LaTeX:] Generate the documentation.
% \end{description}
% If you insist on using \LaTeX\ for \docstrip\ (really,
% \docstrip\ does not need \LaTeX), then inform the autodetect routine
% about your intention:
% \begin{quote}
%   \verb|latex \let\install=y% \iffalse meta-comment
%
% File: accsupp.dtx
% Version: 2018/01/30 v0.5
% Info: Accessibility support by marked content
%
% Copyright (C) 2007, 2010 by
%    Heiko Oberdiek <heiko.oberdiek at googlemail.com>
%    2016
%    https://github.com/ho-tex/oberdiek/issues
%
% This work may be distributed and/or modified under the
% conditions of the LaTeX Project Public License, either
% version 1.3c of this license or (at your option) any later
% version. This version of this license is in
%    http://www.latex-project.org/lppl/lppl-1-3c.txt
% and the latest version of this license is in
%    http://www.latex-project.org/lppl.txt
% and version 1.3 or later is part of all distributions of
% LaTeX version 2005/12/01 or later.
%
% This work has the LPPL maintenance status "maintained".
%
% This Current Maintainer of this work is Heiko Oberdiek.
%
% This work consists of the main source file accsupp.dtx
% and the derived files
%    accsupp.sty, accsupp.pdf, accsupp.ins, accsupp.drv,
%    accsupp-pdftex.def, \ccsupp-luatex.def, accsupp-dvipdfm.def,
%    accsupp-dvips.def,
%    accsupp-example1.tex, accsupp-example2.tex,
%    accsupp-test1.tex.
%
% Distribution:
%    CTAN:macros/latex/contrib/oberdiek/accsupp.dtx
%    CTAN:macros/latex/contrib/oberdiek/accsupp.pdf
%
% Unpacking:
%    (a) If accsupp.ins is present:
%           tex accsupp.ins
%    (b) Without accsupp.ins:
%           tex accsupp.dtx
%    (c) If you insist on using LaTeX
%           latex \let\install=y\input{accsupp.dtx}
%        (quote the arguments according to the demands of your shell)
%
% Documentation:
%    (a) If accsupp.drv is present:
%           latex accsupp.drv
%    (b) Without accsupp.drv:
%           latex accsupp.dtx; ...
%    The class ltxdoc loads the configuration file ltxdoc.cfg
%    if available. Here you can specify further options, e.g.
%    use A4 as paper format:
%       \PassOptionsToClass{a4paper}{article}
%
%    Programm calls to get the documentation (example):
%       pdflatex accsupp.dtx
%       makeindex -s gind.ist accsupp.idx
%       pdflatex accsupp.dtx
%       makeindex -s gind.ist accsupp.idx
%       pdflatex accsupp.dtx
%
% Installation:
%    TDS:tex/latex/oberdiek/accsupp.sty
%    TDS:tex/latex/oberdiek/accsupp-pdftex.def
%    TDS:tex/latex/oberdiek/accsupp-luatex.def
%    TDS:tex/latex/oberdiek/accsupp-dvipdfm.def
%    TDS:tex/latex/oberdiek/accsupp-dvips.def
%    TDS:doc/latex/oberdiek/accsupp.pdf
%    TDS:doc/latex/oberdiek/accsupp-example1.tex
%    TDS:doc/latex/oberdiek/accsupp-example2.tex
%    TDS:doc/latex/oberdiek/test/accsupp-test1.tex
%    TDS:source/latex/oberdiek/accsupp.dtx
%
%<*ignore>
\begingroup
  \catcode123=1 %
  \catcode125=2 %
  \def\x{LaTeX2e}%
\expandafter\endgroup
\ifcase 0\ifx\install y1\fi\expandafter
         \ifx\csname processbatchFile\endcsname\relax\else1\fi
         \ifx\fmtname\x\else 1\fi\relax
\else\csname fi\endcsname
%</ignore>
%<*install>
\input docstrip.tex
\Msg{************************************************************************}
\Msg{* Installation}
\Msg{* Package: accsupp 2018/01/30 v0.5 Accessibility support by marked content (HO)}
\Msg{************************************************************************}

\keepsilent
\askforoverwritefalse

\let\MetaPrefix\relax
\preamble

This is a generated file.

Project: accsupp
Version: 2018/01/30 v0.5

Copyright (C) 2007, 2010 by
   Heiko Oberdiek <heiko.oberdiek at googlemail.com>

This work may be distributed and/or modified under the
conditions of the LaTeX Project Public License, either
version 1.3c of this license or (at your option) any later
version. This version of this license is in
   http://www.latex-project.org/lppl/lppl-1-3c.txt
and the latest version of this license is in
   http://www.latex-project.org/lppl.txt
and version 1.3 or later is part of all distributions of
LaTeX version 2005/12/01 or later.

This work has the LPPL maintenance status "maintained".

This Current Maintainer of this work is Heiko Oberdiek.

This work consists of the main source file accsupp.dtx
and the derived files
   accsupp.sty, accsupp.pdf, accsupp.ins, accsupp.drv,
   accsupp-pdftex.def, accsupp-luatex.def, accsupp-dvipdfm.def,
   accsupp-dvips.def,
   accsupp-example1.tex, accsupp-example2.tex,
   accsupp-test1.tex.

\endpreamble
\let\MetaPrefix\DoubleperCent

\generate{%
  \file{accsupp.ins}{\from{accsupp.dtx}{install}}%
  \file{accsupp.drv}{\from{accsupp.dtx}{driver}}%
  \usedir{tex/latex/oberdiek}%
  \file{accsupp.sty}{\from{accsupp.dtx}{package}}%
  \file{accsupp-pdftex.def}{\from{accsupp.dtx}{pdftex}}%
  \file{accsupp-luatex.def}{\from{accsupp.dtx}{luatex}}%
  \file{accsupp-dvipdfm.def}{\from{accsupp.dtx}{dvipdfm}}%
  \file{accsupp-dvips.def}{\from{accsupp.dtx}{dvips}}%
  \usedir{doc/latex/oberdiek}%
  \file{accsupp-example1.tex}{\from{accsupp.dtx}{example1}}%
  \file{accsupp-example2.tex}{\from{accsupp.dtx}{example2}}%
  \usedir{doc/latex/oberdiek/test}%
  \file{accsupp-test1.tex}{\from{accsupp.dtx}{test1}}%
  \nopreamble
  \nopostamble
  \usedir{source/latex/oberdiek/catalogue}%
  \file{accsupp.xml}{\from{accsupp.dtx}{catalogue}}%
}

\catcode32=13\relax% active space
\let =\space%
\Msg{************************************************************************}
\Msg{*}
\Msg{* To finish the installation you have to move the following}
\Msg{* files into a directory searched by TeX:}
\Msg{*}
\Msg{*     accsupp.sty, accsupp-pdftex.def, accsupp-dvipdfm.def,}
\Msg{*     accsupp-dvips.def}
\Msg{*}
\Msg{* To produce the documentation run the file `accsupp.drv'}
\Msg{* through LaTeX.}
\Msg{*}
\Msg{* Happy TeXing!}
\Msg{*}
\Msg{************************************************************************}

\endbatchfile
%</install>
%<*ignore>
\fi
%</ignore>
%<*driver>
\NeedsTeXFormat{LaTeX2e}
\ProvidesFile{accsupp.drv}%
  [2018/01/30 v0.5 Accessibility support by marked content (HO)]%
\documentclass{ltxdoc}
\usepackage{holtxdoc}[2011/11/22]
\begin{document}
  \DocInput{accsupp.dtx}%
\end{document}
%</driver>
% \fi
%
%
% \CharacterTable
%  {Upper-case    \A\B\C\D\E\F\G\H\I\J\K\L\M\N\O\P\Q\R\S\T\U\V\W\X\Y\Z
%   Lower-case    \a\b\c\d\e\f\g\h\i\j\k\l\m\n\o\p\q\r\s\t\u\v\w\x\y\z
%   Digits        \0\1\2\3\4\5\6\7\8\9
%   Exclamation   \!     Double quote  \"     Hash (number) \#
%   Dollar        \$     Percent       \%     Ampersand     \&
%   Acute accent  \'     Left paren    \(     Right paren   \)
%   Asterisk      \*     Plus          \+     Comma         \,
%   Minus         \-     Point         \.     Solidus       \/
%   Colon         \:     Semicolon     \;     Less than     \<
%   Equals        \=     Greater than  \>     Question mark \?
%   Commercial at \@     Left bracket  \[     Backslash     \\
%   Right bracket \]     Circumflex    \^     Underscore    \_
%   Grave accent  \`     Left brace    \{     Vertical bar  \|
%   Right brace   \}     Tilde         \~}
%
% \GetFileInfo{accsupp.drv}
%
% \title{The \xpackage{accsupp} package}
% \date{2018/01/30 v0.5}
% \author{Heiko Oberdiek\thanks
% {Please report any issues at https://github.com/ho-tex/oberdiek/issues}\\
% \xemail{heiko.oberdiek at googlemail.com}}
%
% \maketitle
%
% \begin{abstract}
% Since PDF 1.5 portions of a page can be marked for better
% accessibility support.
% For example, replacement texts or expansions of abbreviations can be
% provided. Package \xpackage{accsupp} starts with providing a minimal
% low-level interface for programmers. Status is experimental.
% \end{abstract}
%
% \tableofcontents
%
% \section{Documentation}
%
% \subsection{Macros}
%
% Section ``10.8 Accessibility Support'' of the PDF reference
% \cite{pdfspec} lists some features that can be added by
% operators for marked content.
%
% \begin{declcs}{BeginAccSupp} \M{options}
% \end{declcs}
% It puts the operator |BDC| in the page stream:
% \begin{quote}
%   |/Span|\\
%   |<<|\dots|>>| \qquad \textit{\% property dictionary}\\
%   |BDC|
% \end{quote}
% The contents of the dictionary is controlled by \meta{options}.
% See sections \ref{sec:feature} and \ref{sec:method}.
%
% \begin{declcs}{EndAccSupp} \M{options}
% \end{declcs}
% It puts the operator |EMC| in the page stream.
% The only option is \xoption{pdfliteral}, see section \ref{sec:pdftex}.
%
% \bigskip
% \noindent\textbf{Note:}
% The caller is responsible for the placement of \cs{BeginAccSupp}
% and \cs{EndAccSupp} pairs. Especially page breaks are not allowed
% in between.
%
% \subsubsection{Feature options}\label{sec:feature}
%
% The PDF reference \cite{pdfspec} describes and explains the different
% features. The names of the feature options are the same as the
% key names for the property dictionary for operator |BDC|, see
% \cs{BeginAccSupp}.
%
% \begin{description}
% \item[\xoption{ActualText}:]
%   Provides a replacement text, see examples in
%   section \ref{sec:example}.
% \item[\xoption{Alt}:]
%   Provides an alternate description.
% \item[\xoption{E}:]
%   Provides the expansion of an abbreviation or an acronym.
% \item[\xoption{Lang}:]
%   Specifies the language.
% \end{description}
%
% \subsubsection{Input methods}\label{sec:method}
%
% Except for \xoption{Lang} option \xoption{method} controls
% how the argument for \xoption{ActualText}, \xoption{Alt},
% and \xoption{E} are interpreted.
% \begin{description}
% \item[\xoption{method=plain}:]
%   The string is only expanded and written without further
%   treatment. Special characters are not protected, thus this method
%   may result in an invalid PDF file.
% \item[\xoption{mehod=escape}:]
%   The string is expanded and special characters are escaped.
%   The result is a valid PDF string.
% \item[\xoption{method=hex}:]
%   The string is given in hexadecimal notation.
%   Section \ref{sec:example:notparallel} shows an example.
% \item[\xoption{method=pdfstringdef}:]
%   If package \xpackage{hyperref} is loaded, then its \cs{pdfstringdef}
%   is used. This method is slow, but useful if the string contains
%   arbitrary \hologo{TeX} code.
% \item[\xoption{unicode}:]
%   This option is needed, if the string is given as Unicode string
%   (16 bit). Internally it adds the string prefix for Unicode.
%   In case of \xoption{method=pdfstringdef} it passes the option
%   to \cs{hypersetup}.
% \end{description}
%
% \subsection{Workaround, option \xoption{space}}
%
% PDF specification says in ``10.8.3 Replacement Text''
% \begin{quote}
% Just as alternative descriptions can be provided for images and other
% items that do not translate naturally into text (\dots), replacement
% text can be specified for content that does translate into text but
% that is represented in a nonstandard way. These nonstandard representations
% might include, for example, glyphs for ligatures or custom characters, or
% inline graphics corresponding to letters in an illuminated manustript or to
% dropped capitals.
% \end{quote}
% However, the ActualText is ignored in Acrobat Reader
% (until version 9 at least), if the content does not contain glyphs.
% Option \xoption{space} adds such an invisible glyph, a space
% character. The font name can be configured by option \xoption{spacefont},
% the default font is \texttt{phvr8r}. The character slot is given
% by option \xoption{spacechar}, default is \texttt{32}, the usual
% position of the space character.
%
% These options can also be given as package options or in
% macro \cs{AccSuppSetup} that takes a key value list as argument.
% Usually only option \xoption{space} is necessary, if the
% contents does not contain glyphs otherwise. Option \xoption{space}
% is enabled by |space| or |space=true| and disabled by |space=false|.
% It is disabled as default. The option is evaluated by \cs{BeginAccSupp}
% and ignored by \cs{EndAccSupp}.
%
% Note: Even with option \xoption{space} I get sometimes wrong texts
% when cut \& paste from AR7/Linux or AR8/Linux, e.g.
% \begin{quote}
%   |Hello| $\rightarrow$ |Helo|, |Helol|, |Hell|, \dots\\
%   |Hello World| $\rightarrow$ |Helo WorldW|, |Helo World|, \dots
% \end{quote}
% I do not know what Acrobat Reader is doing here, thus feedback and
% insights are welcome.
%
% \subsection{Driver options}
%
% Driver options are package options only.
% The special \hologo{TeX} compilers \hologo{pdfTeX}, \hologo{LuaTeX} and \hologo{XeTeX}
% are detected automatically. The default for unrecognized drivers can be
% set by defining \cs{ActualTextDriverDefault}. This can be done
% in the configuration file \xfile{accsupp.cfg}.
%
% \subsubsection{Option \xoption{pdftex}}\label{sec:pdftex}
% Package option \xoption{pdftex} is used for \hologo{pdfTeX} in PDF mode.
% Additionally \cs{BeginAccSupp} and \cs{EndAccSupp} understand
% option \xoption{pdfliteral}. It controls
% the modifier keyword for \cs{pdfliteral}:
% \begin{quote}
%   |pdfliteral=direct| $\Rightarrow$ |\pdfliteral direct{...}|
% \end{quote}
%
% \subsubsection{Option \xoption{luatex}}\label{sec:luatex}
% Package option \xoption{luatex} is used for \hologo{LuaTeX} in PDF mode.
% Additionally \cs{BeginAccSupp} and \cs{EndAccSupp} understand
% option \xoption{pdfliteral}. It controls
% the modifier keyword for \cs{pdfextension} literal:
% \begin{quote}
%   |pdfliteral=direct| $\Rightarrow$ |\pdfextension literal direct{...}|
% \end{quote}
%
% \subsubsection{Option \xoption{dvipdfm}}
%
% Package option \xoption{dvipdfm} and its aliases \xoption{dvipdfmx}
% \xoption{xetex} are used for drivers that support
% \textsf{dvipdfm} specials.
%
% \subsubsection{Option \xoption{dvips}}
%
% Package option \xoption{dvips} and its alias \xoption{dvipsone}
% write pdfmark specials in the output. Unhappily these pdfmark
% operators are ignored by \textsf{ghostscript} (latest tested
% version is 8.54). Perhaps they are recognized by commercial
% distiller applications.
%
% \subsubsection{Turning off page stream compression}
%
% For debugging it is useful to have uncompressed page stream objects.
% This can be done afterwards via \textsf{pdftk}:
% \begin{quote}
%   |pdftk file.pdf output file-uncompressed.pdf uncompress|
% \end{quote}
% Or the PDF file is generated uncompressed:
% \begin{description}
% \item[\hologo{pdfTeX}:] |\pdfcompresslevel=0|
% \item[dvipdfm:] |dvipdfm -z0 ...|
% \item[dvipdfmx:] |dvipdfmx -z0 ...|
% \item[ghostscript/ps2pdf:]
%      |ps2pdf -dCompressPages=false input.ps output.pdf|
% \end{description}
%
% \section{Example}
% \label{sec:example}
%
% \subsection{Example \cs{notparallel}}
% \label{sec:example:notparallel}
%
%    \begin{macrocode}
%<*example1>
%    \end{macrocode}
%    \begin{macrocode}
%<<END
\documentclass{article}
\usepackage{accsupp}[2007/11/14]
\usepackage{centernot}
% U+2226 NOT PARALLEL
% \mathrel{...} prevents page break in between
\newcommand*{\notparallel}{%
  \ensuremath{%
    \mathrel{%
      \BeginAccSupp{method=hex,unicode,ActualText=2226}%
      \centernot{\parallel}%
      \EndAccSupp{}%
    }%
  }%
}
\begin{document}
\begin{equation}
A\notparallel B
\end{equation}
\end{document}
%END
%    \end{macrocode}
%    \begin{macrocode}
%</example1>
%    \end{macrocode}
%
% \subsection{Example with \xoption{pdfstringdef}}
%
%    \begin{macrocode}
%<*example2>
%    \end{macrocode}
%    \begin{macrocode}
%<<END
\documentclass{article}
\usepackage[unicode]{hyperref}
\usepackage{accsupp}[2007/11/14]
\begin{document}
  \begin{equation}
    \BeginAccSupp{
      method=pdfstringdef,
      unicode,
      ActualText={%
        a\texttwosuperior +b\texttwosuperior
        =c\texttwosuperior
      }
    }
    a^2 + b^2 = c^2
    \EndAccSupp{}
  \end{equation}
\end{document}
%END
%    \end{macrocode}
%    \begin{macrocode}
%</example2>
%    \end{macrocode}
%
% \StopEventually{
% }
%
% \section{Implementation}
%
% \subsection{Package}
%
%    \begin{macrocode}
%<*package>
%    \end{macrocode}
%
%    \begin{macrocode}
\begingroup\catcode61\catcode48\catcode32=10\relax%
  \catcode13=5 % ^^M
  \endlinechar=13 %
  \catcode123=1 % {
  \catcode125=2 % }
  \catcode64=11 % @
  \def\x{\endgroup
    \expandafter\edef\csname ACCSUPP@AtEnd\endcsname{%
      \endlinechar=\the\endlinechar\relax
      \catcode13=\the\catcode13\relax
      \catcode32=\the\catcode32\relax
      \catcode35=\the\catcode35\relax
      \catcode61=\the\catcode61\relax
      \catcode64=\the\catcode64\relax
      \catcode123=\the\catcode123\relax
      \catcode125=\the\catcode125\relax
    }%
  }%
\x\catcode61\catcode48\catcode32=10\relax%
\catcode13=5 % ^^M
\endlinechar=13 %
\catcode35=6 % #
\catcode64=11 % @
\catcode123=1 % {
\catcode125=2 % }
\def\TMP@EnsureCode#1#2{%
  \edef\ACCSUPP@AtEnd{%
    \ACCSUPP@AtEnd
    \catcode#1=\the\catcode#1\relax
  }%
  \catcode#1=#2\relax
}
\TMP@EnsureCode{10}{12}% ^^J
\TMP@EnsureCode{33}{12}% !
\TMP@EnsureCode{39}{12}% '
\TMP@EnsureCode{40}{12}% (
\TMP@EnsureCode{41}{12}% )
\TMP@EnsureCode{42}{12}% *
\TMP@EnsureCode{44}{12}% ,
\TMP@EnsureCode{45}{12}% -
\TMP@EnsureCode{46}{12}% .
\TMP@EnsureCode{47}{12}% /
\TMP@EnsureCode{58}{12}% :
\TMP@EnsureCode{60}{12}% <
\TMP@EnsureCode{62}{12}% >
\TMP@EnsureCode{94}{7}% ^ (superscript)
\TMP@EnsureCode{96}{12}% `
\TMP@EnsureCode{254}{12}% ^^fe
\TMP@EnsureCode{255}{12}% ^^ff
\edef\ACCSUPP@AtEnd{\ACCSUPP@AtEnd\noexpand\endinput}
%    \end{macrocode}
%
%    Package identification.
%    \begin{macrocode}
\NeedsTeXFormat{LaTeX2e}
\ProvidesPackage{accsupp}%
  [2018/01/30 v0.5 Accessibility support by marked content (HO)]
%    \end{macrocode}
%
%    \begin{macrocode}
\RequirePackage{pdfescape}[2007/02/25]
\RequirePackage{ifpdf}
\RequirePackage{ifxetex}
\RequirePackage{kvoptions}
%    \end{macrocode}
%    \begin{macrocode}
\SetupKeyvalOptions{%
  family=ACCSUPP,%
  prefix=ACCSUPP@%
}
%    \end{macrocode}
%
% \subsection{Space setup}
%
%    \begin{macrocode}
\DeclareBoolOption{space}
\DeclareStringOption[phvr8r]{spacefont}
\DeclareStringOption[32]{spacechar}
%    \end{macrocode}
%
% \subsection{Driver detection and setup}
%
% \paragraph{Driver declarations.}
%
%    \begin{macrocode}
\def\ACCSUPP@DefineDriverKey{%
  \@dblarg\ACCSUPP@@DefineDriverKey
}
\def\ACCSUPP@@DefineDriverKey[#1]#2{%
  \define@key{ACCSUPP}{#2}[]{%
    \def\ACCSUPP@driver{#1}%
  }%
  \g@addto@macro\ACCSUPP@DisableOptions{%
    \DisableKeyvalOption{ACCSUPP}{#2}%
  }%
}
\let\ACCSUPP@DisableOptions\@empty
\ACCSUPP@DefineDriverKey{pdftex}
\ACCSUPP@DefineDriverKey{luatex}
\ACCSUPP@DefineDriverKey{dvips}
\ACCSUPP@DefineDriverKey[dvips]{dvipsone}
\ACCSUPP@DefineDriverKey{dvipdfm}
\ACCSUPP@DefineDriverKey[dvipdfm]{dvipdfmx}
\ACCSUPP@DefineDriverKey[dvipdfm]{xetex}
\let\ACCSUPP@driver\relax
\InputIfFileExists{accsupp.cfg}{}{}
\providecommand*{\ActualTextDriverDefault}{dvips}
\ifpdf
  \ifx\pdfextension\@undefined
    \def\ACCSUPP@driver{pdftex}%
  \else
    \def\ACCSUPP@driver{luatex}%
  \fi
\else
  \ifxetex
    \def\ACCSUPP@driver{dvipdfm}%
  \else
    \ifx\ACCSUPP@driver\relax
      \let\ACCSUPP@driver\ActualTextDriverDefault
    \fi
  \fi
\fi
%    \end{macrocode}
%
% \paragraph{Process options.}
%
%    \begin{macrocode}
\ProcessKeyvalOptions*
\ACCSUPP@DisableOptions
%    \end{macrocode}
%
% \paragraph{Driver validation and loading.}
%
%    \begin{macrocode}
\def\ACCSUPP@temp{pdftex}%
\ifpdf
  \ifx\pdfextension\@undefined\else
    \def\ACCSUPP@temp{luatex}%
  \fi
  \ifx\ACCSUPP@temp\ACCSUPP@driver
  \else
    \PackageWarningNoLine{accsupp}{%
      Wrong driver `\ACCSUPP@driver', using `pdftex' instead%
    }%
ed as
    \let\ACCSUPP@driver\ACCSUPP@temp
  \fi
\else
  \ifx\ACCSUPP@temp\ACCSUPP@driver
    \PackageError{accsupp}{%
      Wrong driver, pdfTeX is not running in PDF mode.\MessageBreak
      Package loading is aborted%
    }\@ehc
    \expandafter\expandafter\expandafter\ACCSUPP@AtEnd
  \fi
  \def\ACCSUPP@temp{dvipdfm}%
  \ifxetex
    \ifx\ACCSUPP@temp\ACCSUPP@driver
    \else
      \PackageWarningNoLine{accsupp}{%
        Wrong driver `\ACCSUPP@driver',\MessageBreak
        using `dvipdfm' for XeTeX instead%
      }%
      \let\ACCSUPP@driver\ACCSUPP@temp
    \fi
  \fi
\fi%
\ifx\ACCSUPP@driver\relax
  \PackageError{accsupp}{%
    Missing driver option.\MessageBreak
    Package loading is aborted%
  }\@ehc
  \expandafter\ACCSUPP@AtEnd
\fi%
\InputIfFileExists{accsupp-\ACCSUPP@driver.def}{}{%
  \PackageError{accsupp}{%
    Unsupported driver `\ACCSUPP@driver'.\MessageBreak
    Package loading is aborted%
  }\@ehc
  \ACCSUPP@AtEnd
}%
%    \end{macrocode}
%
% \subsection{Main macro}
%
%    \begin{macrocode}
\DeclareBoolOption{unicode}
\DeclareStringOption[page]{pdfliteral}
\DeclareStringOption{Lang}
\def\ACCSUPP@method{escape}
\define@key{ACCSUPP}{method}{%
  \@ifundefined{ACCSUPP@method@#1}{%
    \PackageError{accsupp}{%
      Ignoring unknown method `#1'%
    }\@ehc
  }{%
    \edef\ACCSUPP@method{#1}%
  }%
}
\let\ACCSUPP@Lang\relax
\def\ACCSUPP@temp#1{%
  \expandafter\ACCSUPP@@temp\csname ACCSUPP@#1\endcsname{#1}%
}
\def\ACCSUPP@@temp#1#2{%
  \let#1\relax
  \define@key{ACCSUPP}{#2}{%
    \def#1{##1}%
    \ifx#1\@empty
      \def#1{()}%
    \else
      \csname ACCSUPP@method@\ACCSUPP@method\endcsname#1%
    \fi
  }%
}
\ACCSUPP@temp{Alt}
\ACCSUPP@temp{ActualText}
\ACCSUPP@temp{E}
%    \end{macrocode}
%
%    \begin{macrocode}
\newcommand*{\BeginAccSupp}[1]{%
  \begingroup
    \setkeys{ACCSUPP}{#1}%
    \edef\ACCSUPP@span{%
      /Span<<%
        \ifx\ACCSUPP@Lang\relax
        \else
          /Lang\ACCSUPP@Lang
        \fi
        \ifx\ACCSUPP@Alt\relax
        \else
          /Alt\ACCSUPP@Alt
        \fi
        \ifx\ACCSUPP@ActualText\relax
        \else
          /ActualText\ACCSUPP@ActualText
        \fi
        \ifx\ACCSUPP@E\relax
        \else
          /E\ACCSUPP@E
        \fi
      >>%
    }%
    \ACCSUPP@bdc
    \ACCSUPP@space
  \endgroup
}
\newcommand*{\EndAccSupp}[1]{%
  \begingroup
    \setkeys{ACCSUPP}{#1}%
    \ACCSUPP@emc
  \endgroup
}
\def\ACCSUPP@space{%
  \ifACCSUPP@space
    \begingroup
      \@ifundefined{ACCSUPP@Font}{%
        \global\font\ACCSUPP@Font=\ACCSUPP@spacefont\relax
      }{}%
      \leavevmode
      \setbox\z@\hbox{\ACCSUPP@Font\char\ACCSUPP@spacechar}%
      \wd\z@\z@
      \ht\z@\z@
      \dp\z@\z@
      \copy\z@
    \endgroup
  \fi
}
\newcommand*{\AccSuppSetup}{%
  \setkeys{ACCSUPP}%
}
%    \end{macrocode}
%
% \subsubsection{Input methods}
%
%    \begin{macrocode}
\def\ACCSUPP@method@plain#1{%
  \csname @safe@activestrue\endcsname
  \edef#1{%
    (%
    \ifACCSUPP@unicode
      \string\376\string\377%
    \fi
    #1%
    )%
  }%
  \@onelevel@sanitize#1%
}
%    \end{macrocode}
%    \begin{macrocode}
\def\ACCSUPP@method@escape#1{%
  \EdefEscapeString#1{%
    \ifACCSUPP@unicode
      ^^fe^^ff%
    \fi
    #1%
  }%
  \edef#1{(#1)}%
}%
%    \end{macrocode}
%    \begin{macrocode}
\def\ACCSUPP@method@hex#1{%
  \edef#1{%
    <%
    \ifACCSUPP@unicode
      FEFF%
    \fi
    #1%
    >%
  }%
}
%    \end{macrocode}
%    \begin{macrocode}
\def\ACCSUPP@method@pdfstringdef#1{%
  \ifACCSUPP@unicode
    \@ifundefined{hypersetup}{}{%
      \hypersetup{unicode}%
    }%
  \fi
  \@ifundefined{pdfstringdef}{%
    \PackageError{accsupp}{%
      Method `pdfstringdef' requires package `hyperref'%
    }\@ehc
    \let\ACCSUPP@temp\@empty
  }{%
    \begingroup
      \setbox0=\hbox{%
        \pdfstringdef\ACCSUPP@temp#1%
        \global\let\ACCSUPP@temp\ACCSUPP@temp
      }%
    \endgroup
  }%
  \edef#1{(\ACCSUPP@temp)}%
}
%    \end{macrocode}
%
%    \begin{macrocode}
\ACCSUPP@AtEnd%
%</package>
%    \end{macrocode}
%
% \subsection{Drivers}
%
% \subsubsection{Driver \xoption{pdftex}}
%
%    \begin{macrocode}
%<*pdftex>
\NeedsTeXFormat{LaTeX2e}
\ProvidesFile{accsupp-pdftex.def}%
  [2018/01/30 v0.5 accsupp driver for pdfTeX (HO)]%
%    \end{macrocode}
%
%    \begin{macrocode}
\def\ACCSUPP@bdc{%
  \pdfliteral\ACCSUPP@pdfliteral{\ACCSUPP@span BDC}%
}
\def\ACCSUPP@emc{%
  \pdfliteral\ACCSUPP@pdfliteral{EMC}%
}
%    \end{macrocode}
%    \begin{macrocode}
%</pdftex>
%    \end{macrocode}
%
% \subsubsection{Driver \xoption{luatex}}
%
%    \begin{macrocode}
%<*luatex>
\NeedsTeXFormat{LaTeX2e}
\ProvidesFile{accsupp-luatex.def}%
  [2018/01/30 v0.5 accsupp driver for pdfTeX (HO)]%
%    \end{macrocode}
%
%    \begin{macrocode}
\protected\def\ACCSUPP@pdf@literal{\pdfextension literal}
\def\ACCSUPP@bdc{%
  \ACCSUPP@pdf@literal\ACCSUPP@pdfliteral{\ACCSUPP@span BDC}%
}
\def\ACCSUPP@emc{%
  \ACCSUPP@pdf@literal\ACCSUPP@pdfliteral{EMC}%
}
%    \end{macrocode}
%    \begin{macrocode}
%</luatex>
%    \end{macrocode}
%
% \subsubsection{Driver \xoption{dvipdfm}}
%
%    \begin{macrocode}
%<*dvipdfm>
\NeedsTeXFormat{LaTeX2e}
\ProvidesFile{accsupp-dvipdfm.def}%
  [2018/01/30 v0.5 accsupp driver for dvipdfm (HO)]%
%    \end{macrocode}
%
%    \begin{macrocode}
\def\ACCSUPP@bdc{%
  \special{pdf:content \ACCSUPP@span BDC}%
}
\def\ACCSUPP@emc{%
  \special{pdf:content EMC}%
}
%    \end{macrocode}
%    \begin{macrocode}
%</dvipdfm>
%    \end{macrocode}
%
% \subsubsection{Driver \xoption{dvips}}
%
%    \begin{macrocode}
%<*dvips>
\NeedsTeXFormat{LaTeX2e}
\ProvidesFile{accsupp-dvips.def}%
  [2018/01/30 v0.5 accsupp driver for dvips (HO)]%
%    \end{macrocode}
%
%    \begin{macrocode}
\def\ACCSUPP@bdc{%
  \special{ps:[\ACCSUPP@span/BDC pdfmark}%
}
\def\ACCSUPP@emc{%
  \special{ps:[/EMC pdfmark}%
}
%    \end{macrocode}
%    \begin{macrocode}
%</dvips>
%    \end{macrocode}
%
% \section{Test}
%
% \subsection{Catcode checks for loading}
%
%    \begin{macrocode}
%<*test1>
%    \end{macrocode}
%    \begin{macrocode}
\NeedsTeXFormat{LaTeX2e}
\documentclass{minimal}
\makeatletter
\def\RestoreCatcodes{}
\count@=0 %
\loop
  \edef\RestoreCatcodes{%
    \RestoreCatcodes
    \catcode\the\count@=\the\catcode\count@\relax
  }%
\ifnum\count@<255 %
  \advance\count@\@ne
\repeat

\def\RangeCatcodeInvalid#1#2{%
  \count@=#1\relax
  \loop
    \catcode\count@=15 %
  \ifnum\count@<#2\relax
    \advance\count@\@ne
  \repeat
}
\def\Test{%
  \RangeCatcodeInvalid{0}{47}%
  \RangeCatcodeInvalid{58}{64}%
  \RangeCatcodeInvalid{91}{96}%
  \RangeCatcodeInvalid{123}{127}%
  \catcode`\@=12 %
  \catcode`\\=0 %
  \catcode`\{=1 %
  \catcode`\}=2 %
  \catcode`\#=6 %
  \catcode`\[=12 %
  \catcode`\]=12 %
  \catcode`\%=14 %
  \catcode`\ =10 %
  \catcode13=5 %
  \RequirePackage{accsupp}[2018/01/30]\relax
  \RestoreCatcodes
}
\Test
\csname @@end\endcsname
\end
%    \end{macrocode}
%    \begin{macrocode}
%</test1>
%    \end{macrocode}
%
% \section{Installation}
%
% \subsection{Download}
%
% \paragraph{Package.} This package is available on
% CTAN\footnote{\url{http://ctan.org/pkg/accsupp}}:
% \begin{description}
% \item[\CTAN{macros/latex/contrib/oberdiek/accsupp.dtx}] The source file.
% \item[\CTAN{macros/latex/contrib/oberdiek/accsupp.pdf}] Documentation.
% \end{description}
%
%
% \paragraph{Bundle.} All the packages of the bundle `oberdiek'
% are also available in a TDS compliant ZIP archive. There
% the packages are already unpacked and the documentation files
% are generated. The files and directories obey the TDS standard.
% \begin{description}
% \item[\CTAN{install/macros/latex/contrib/oberdiek.tds.zip}]
% \end{description}
% \emph{TDS} refers to the standard ``A Directory Structure
% for \TeX\ Files'' (\CTAN{tds/tds.pdf}). Directories
% with \xfile{texmf} in their name are usually organized this way.
%
% \subsection{Bundle installation}
%
% \paragraph{Unpacking.} Unpack the \xfile{oberdiek.tds.zip} in the
% TDS tree (also known as \xfile{texmf} tree) of your choice.
% Example (linux):
% \begin{quote}
%   |unzip oberdiek.tds.zip -d ~/texmf|
% \end{quote}
%
% \paragraph{Script installation.}
% Check the directory \xfile{TDS:scripts/oberdiek/} for
% scripts that need further installation steps.
% Package \xpackage{attachfile2} comes with the Perl script
% \xfile{pdfatfi.pl} that should be installed in such a way
% that it can be called as \texttt{pdfatfi}.
% Example (linux):
% \begin{quote}
%   |chmod +x scripts/oberdiek/pdfatfi.pl|\\
%   |cp scripts/oberdiek/pdfatfi.pl /usr/local/bin/|
% \end{quote}
%
% \subsection{Package installation}
%
% \paragraph{Unpacking.} The \xfile{.dtx} file is a self-extracting
% \docstrip\ archive. The files are extracted by running the
% \xfile{.dtx} through \plainTeX:
% \begin{quote}
%   \verb|tex accsupp.dtx|
% \end{quote}
%
% \paragraph{TDS.} Now the different files must be moved into
% the different directories in your installation TDS tree
% (also known as \xfile{texmf} tree):
% \begin{quote}
% \def\t{^^A
% \begin{tabular}{@{}>{\ttfamily}l@{ $\rightarrow$ }>{\ttfamily}l@{}}
%   accsupp.sty & tex/latex/oberdiek/accsupp.sty\\
%   accsupp-pdftex.def & tex/latex/oberdiek/accsupp-pdftex.def\\
%   accsupp-luatex.def & tex/latex/oberdiek/accsupp-luatex.def\\
%   accsupp-dvipdfm.def & tex/latex/oberdiek/accsupp-dvipdfm.def\\
%   accsupp-dvips.def & tex/latex/oberdiek/accsupp-dvips.def\\
%   accsupp.pdf & doc/latex/oberdiek/accsupp.pdf\\
%   accsupp-example1.tex & doc/latex/oberdiek/accsupp-example1.tex\\
%   accsupp-example2.tex & doc/latex/oberdiek/accsupp-example2.tex\\
%   test/accsupp-test1.tex & doc/latex/oberdiek/test/accsupp-test1.tex\\
%   accsupp.dtx & source/latex/oberdiek/accsupp.dtx\\
% \end{tabular}^^A
% }^^A
% \sbox0{\t}^^A
% \ifdim\wd0>\linewidth
%   \begingroup
%     \advance\linewidth by\leftmargin
%     \advance\linewidth by\rightmargin
%   \edef\x{\endgroup
%     \def\noexpand\lw{\the\linewidth}^^A
%   }\x
%   \def\lwbox{^^A
%     \leavevmode
%     \hbox to \linewidth{^^A
%       \kern-\leftmargin\relax
%       \hss
%       \usebox0
%       \hss
%       \kern-\rightmargin\relax
%     }^^A
%   }^^A
%   \ifdim\wd0>\lw
%     \sbox0{\small\t}^^A
%     \ifdim\wd0>\linewidth
%       \ifdim\wd0>\lw
%         \sbox0{\footnotesize\t}^^A
%         \ifdim\wd0>\linewidth
%           \ifdim\wd0>\lw
%             \sbox0{\scriptsize\t}^^A
%             \ifdim\wd0>\linewidth
%               \ifdim\wd0>\lw
%                 \sbox0{\tiny\t}^^A
%                 \ifdim\wd0>\linewidth
%                   \lwbox
%                 \else
%                   \usebox0
%                 \fi
%               \else
%                 \lwbox
%               \fi
%             \else
%               \usebox0
%             \fi
%           \else
%             \lwbox
%           \fi
%         \else
%           \usebox0
%         \fi
%       \else
%         \lwbox
%       \fi
%     \else
%       \usebox0
%     \fi
%   \else
%     \lwbox
%   \fi
% \else
%   \usebox0
% \fi
% \end{quote}
% If you have a \xfile{docstrip.cfg} that configures and enables \docstrip's
% TDS installing feature, then some files can already be in the right
% place, see the documentation of \docstrip.
%
% \subsection{Refresh file name databases}
%
% If your \TeX~distribution
% (\teTeX, \mikTeX, \dots) relies on file name databases, you must refresh
% these. For example, \teTeX\ users run \verb|texhash| or
% \verb|mktexlsr|.
%
% \subsection{Some details for the interested}
%
% \paragraph{Attached source.}
%
% The PDF documentation on CTAN also includes the
% \xfile{.dtx} source file. It can be extracted by
% AcrobatReader 6 or higher. Another option is \textsf{pdftk},
% e.g. unpack the file into the current directory:
% \begin{quote}
%   \verb|pdftk accsupp.pdf unpack_files output .|
% \end{quote}
%
% \paragraph{Unpacking with \LaTeX.}
% The \xfile{.dtx} chooses its action depending on the format:
% \begin{description}
% \item[\plainTeX:] Run \docstrip\ and extract the files.
% \item[\LaTeX:] Generate the documentation.
% \end{description}
% If you insist on using \LaTeX\ for \docstrip\ (really,
% \docstrip\ does not need \LaTeX), then inform the autodetect routine
% about your intention:
% \begin{quote}
%   \verb|latex \let\install=y\input{accsupp.dtx}|
% \end{quote}
% Do not forget to quote the argument according to the demands
% of your shell.
%
% \paragraph{Generating the documentation.}
% You can use both the \xfile{.dtx} or the \xfile{.drv} to generate
% the documentation. The process can be configured by the
% configuration file \xfile{ltxdoc.cfg}. For instance, put this
% line into this file, if you want to have A4 as paper format:
% \begin{quote}
%   \verb|\PassOptionsToClass{a4paper}{article}|
% \end{quote}
% An example follows how to generate the
% documentation with pdf\LaTeX:
% \begin{quote}
%\begin{verbatim}
%pdflatex accsupp.dtx
%makeindex -s gind.ist accsupp.idx
%pdflatex accsupp.dtx
%makeindex -s gind.ist accsupp.idx
%pdflatex accsupp.dtx
%\end{verbatim}
% \end{quote}
%
% \section{Catalogue}
%
% The following XML file can be used as source for the
% \href{http://mirror.ctan.org/help/Catalogue/catalogue.html}{\TeX\ Catalogue}.
% The elements \texttt{caption} and \texttt{description} are imported
% from the original XML file from the Catalogue.
% The name of the XML file in the Catalogue is \xfile{accsupp.xml}.
%    \begin{macrocode}
%<*catalogue>
<?xml version='1.0' encoding='us-ascii'?>
<!DOCTYPE entry SYSTEM 'catalogue.dtd'>
<entry datestamp='$Date$' modifier='$Author$' id='accsupp'>
  <name>accsupp</name>
  <caption>Better accessibility support for PDF files.</caption>
  <authorref id='auth:oberdiek'/>
  <copyright owner='Heiko Oberdiek' year='2007,2010'/>
  <license type='lppl1.3'/>
  <version number='0.4'/>
  <description>
    Since PDF 1.5 portions of a page can be marked for better accessibility
    support. For example, replacement texts or expansions of abbreviations can
    be provided. This package starts with providing a minimal low-level
    interface for programmers; its status is experimental.
    <p/>
    The package is part of the <xref refid='oberdiek'>oberdiek</xref> bundle.
  </description>
  <documentation details='Package documentation'
      href='ctan:/macros/latex/contrib/oberdiek/accsupp.pdf'/>
  <ctan file='true' path='/macros/latex/contrib/oberdiek/accsupp.dtx'/>
  <miktex location='oberdiek'/>
  <texlive location='oberdiek'/>
  <install path='/macros/latex/contrib/oberdiek/oberdiek.tds.zip'/>
</entry>
%</catalogue>
%    \end{macrocode}
%
% \begin{thebibliography}{9}
% \bibitem{pdfspec}
% Adobe Systems Incorporated,
% \textit{PDF Reference}, 6th edition, 2006.
% \url{http://www.adobe.com/devnet/acrobat/pdfs/pdf_reference.pdf}
%
% \end{thebibliography}
%
% \begin{History}
%   \begin{Version}{2007/03/21 v0.1}
%   \item
%     First version.
%   \end{Version}
%   \begin{Version}{2007/11/14 v0.2}
%   \item
%     Various bug fixes.
%   \item
%     Catcode section rewritten, test added.
%   \end{Version}
%   \begin{Version}{2010/01/16 v0.3}
%   \item
%     \cs{AccSuppSetup} added.
%   \item
%     Options \xoption{space}, \xoption{spacefont}, \xoption{spacechar} added.
%   \end{Version}
%   \begin{Version}{2016/05/16 v0.4}
%   \item
%     Documentation updates.
%   \end{Version}
%   \begin{Version}{2018/01/30 v0.5}
%   \item
%     LuaTeX support added.
%   \end{Version}
% \end{History}
%
% \PrintIndex
%
% \Finale
\endinput
|
% \end{quote}
% Do not forget to quote the argument according to the demands
% of your shell.
%
% \paragraph{Generating the documentation.}
% You can use both the \xfile{.dtx} or the \xfile{.drv} to generate
% the documentation. The process can be configured by the
% configuration file \xfile{ltxdoc.cfg}. For instance, put this
% line into this file, if you want to have A4 as paper format:
% \begin{quote}
%   \verb|\PassOptionsToClass{a4paper}{article}|
% \end{quote}
% An example follows how to generate the
% documentation with pdf\LaTeX:
% \begin{quote}
%\begin{verbatim}
%pdflatex accsupp.dtx
%makeindex -s gind.ist accsupp.idx
%pdflatex accsupp.dtx
%makeindex -s gind.ist accsupp.idx
%pdflatex accsupp.dtx
%\end{verbatim}
% \end{quote}
%
% \section{Catalogue}
%
% The following XML file can be used as source for the
% \href{http://mirror.ctan.org/help/Catalogue/catalogue.html}{\TeX\ Catalogue}.
% The elements \texttt{caption} and \texttt{description} are imported
% from the original XML file from the Catalogue.
% The name of the XML file in the Catalogue is \xfile{accsupp.xml}.
%    \begin{macrocode}
%<*catalogue>
<?xml version='1.0' encoding='us-ascii'?>
<!DOCTYPE entry SYSTEM 'catalogue.dtd'>
<entry datestamp='$Date$' modifier='$Author$' id='accsupp'>
  <name>accsupp</name>
  <caption>Better accessibility support for PDF files.</caption>
  <authorref id='auth:oberdiek'/>
  <copyright owner='Heiko Oberdiek' year='2007,2010'/>
  <license type='lppl1.3'/>
  <version number='0.4'/>
  <description>
    Since PDF 1.5 portions of a page can be marked for better accessibility
    support. For example, replacement texts or expansions of abbreviations can
    be provided. This package starts with providing a minimal low-level
    interface for programmers; its status is experimental.
    <p/>
    The package is part of the <xref refid='oberdiek'>oberdiek</xref> bundle.
  </description>
  <documentation details='Package documentation'
      href='ctan:/macros/latex/contrib/oberdiek/accsupp.pdf'/>
  <ctan file='true' path='/macros/latex/contrib/oberdiek/accsupp.dtx'/>
  <miktex location='oberdiek'/>
  <texlive location='oberdiek'/>
  <install path='/macros/latex/contrib/oberdiek/oberdiek.tds.zip'/>
</entry>
%</catalogue>
%    \end{macrocode}
%
% \begin{thebibliography}{9}
% \bibitem{pdfspec}
% Adobe Systems Incorporated,
% \textit{PDF Reference}, 6th edition, 2006.
% \url{http://www.adobe.com/devnet/acrobat/pdfs/pdf_reference.pdf}
%
% \end{thebibliography}
%
% \begin{History}
%   \begin{Version}{2007/03/21 v0.1}
%   \item
%     First version.
%   \end{Version}
%   \begin{Version}{2007/11/14 v0.2}
%   \item
%     Various bug fixes.
%   \item
%     Catcode section rewritten, test added.
%   \end{Version}
%   \begin{Version}{2010/01/16 v0.3}
%   \item
%     \cs{AccSuppSetup} added.
%   \item
%     Options \xoption{space}, \xoption{spacefont}, \xoption{spacechar} added.
%   \end{Version}
%   \begin{Version}{2016/05/16 v0.4}
%   \item
%     Documentation updates.
%   \end{Version}
%   \begin{Version}{2018/01/30 v0.5}
%   \item
%     LuaTeX support added.
%   \end{Version}
% \end{History}
%
% \PrintIndex
%
% \Finale
\endinput

%        (quote the arguments according to the demands of your shell)
%
% Documentation:
%    (a) If accsupp.drv is present:
%           latex accsupp.drv
%    (b) Without accsupp.drv:
%           latex accsupp.dtx; ...
%    The class ltxdoc loads the configuration file ltxdoc.cfg
%    if available. Here you can specify further options, e.g.
%    use A4 as paper format:
%       \PassOptionsToClass{a4paper}{article}
%
%    Programm calls to get the documentation (example):
%       pdflatex accsupp.dtx
%       makeindex -s gind.ist accsupp.idx
%       pdflatex accsupp.dtx
%       makeindex -s gind.ist accsupp.idx
%       pdflatex accsupp.dtx
%
% Installation:
%    TDS:tex/latex/oberdiek/accsupp.sty
%    TDS:tex/latex/oberdiek/accsupp-pdftex.def
%    TDS:tex/latex/oberdiek/accsupp-luatex.def
%    TDS:tex/latex/oberdiek/accsupp-dvipdfm.def
%    TDS:tex/latex/oberdiek/accsupp-dvips.def
%    TDS:doc/latex/oberdiek/accsupp.pdf
%    TDS:doc/latex/oberdiek/accsupp-example1.tex
%    TDS:doc/latex/oberdiek/accsupp-example2.tex
%    TDS:doc/latex/oberdiek/test/accsupp-test1.tex
%    TDS:source/latex/oberdiek/accsupp.dtx
%
%<*ignore>
\begingroup
  \catcode123=1 %
  \catcode125=2 %
  \def\x{LaTeX2e}%
\expandafter\endgroup
\ifcase 0\ifx\install y1\fi\expandafter
         \ifx\csname processbatchFile\endcsname\relax\else1\fi
         \ifx\fmtname\x\else 1\fi\relax
\else\csname fi\endcsname
%</ignore>
%<*install>
\input docstrip.tex
\Msg{************************************************************************}
\Msg{* Installation}
\Msg{* Package: accsupp 2018/01/30 v0.5 Accessibility support by marked content (HO)}
\Msg{************************************************************************}

\keepsilent
\askforoverwritefalse

\let\MetaPrefix\relax
\preamble

This is a generated file.

Project: accsupp
Version: 2018/01/30 v0.5

Copyright (C) 2007, 2010 by
   Heiko Oberdiek <heiko.oberdiek at googlemail.com>

This work may be distributed and/or modified under the
conditions of the LaTeX Project Public License, either
version 1.3c of this license or (at your option) any later
version. This version of this license is in
   http://www.latex-project.org/lppl/lppl-1-3c.txt
and the latest version of this license is in
   http://www.latex-project.org/lppl.txt
and version 1.3 or later is part of all distributions of
LaTeX version 2005/12/01 or later.

This work has the LPPL maintenance status "maintained".

This Current Maintainer of this work is Heiko Oberdiek.

This work consists of the main source file accsupp.dtx
and the derived files
   accsupp.sty, accsupp.pdf, accsupp.ins, accsupp.drv,
   accsupp-pdftex.def, accsupp-luatex.def, accsupp-dvipdfm.def,
   accsupp-dvips.def,
   accsupp-example1.tex, accsupp-example2.tex,
   accsupp-test1.tex.

\endpreamble
\let\MetaPrefix\DoubleperCent

\generate{%
  \file{accsupp.ins}{\from{accsupp.dtx}{install}}%
  \file{accsupp.drv}{\from{accsupp.dtx}{driver}}%
  \usedir{tex/latex/oberdiek}%
  \file{accsupp.sty}{\from{accsupp.dtx}{package}}%
  \file{accsupp-pdftex.def}{\from{accsupp.dtx}{pdftex}}%
  \file{accsupp-luatex.def}{\from{accsupp.dtx}{luatex}}%
  \file{accsupp-dvipdfm.def}{\from{accsupp.dtx}{dvipdfm}}%
  \file{accsupp-dvips.def}{\from{accsupp.dtx}{dvips}}%
  \usedir{doc/latex/oberdiek}%
  \file{accsupp-example1.tex}{\from{accsupp.dtx}{example1}}%
  \file{accsupp-example2.tex}{\from{accsupp.dtx}{example2}}%
  \usedir{doc/latex/oberdiek/test}%
  \file{accsupp-test1.tex}{\from{accsupp.dtx}{test1}}%
  \nopreamble
  \nopostamble
  \usedir{source/latex/oberdiek/catalogue}%
  \file{accsupp.xml}{\from{accsupp.dtx}{catalogue}}%
}

\catcode32=13\relax% active space
\let =\space%
\Msg{************************************************************************}
\Msg{*}
\Msg{* To finish the installation you have to move the following}
\Msg{* files into a directory searched by TeX:}
\Msg{*}
\Msg{*     accsupp.sty, accsupp-pdftex.def, accsupp-dvipdfm.def,}
\Msg{*     accsupp-dvips.def}
\Msg{*}
\Msg{* To produce the documentation run the file `accsupp.drv'}
\Msg{* through LaTeX.}
\Msg{*}
\Msg{* Happy TeXing!}
\Msg{*}
\Msg{************************************************************************}

\endbatchfile
%</install>
%<*ignore>
\fi
%</ignore>
%<*driver>
\NeedsTeXFormat{LaTeX2e}
\ProvidesFile{accsupp.drv}%
  [2018/01/30 v0.5 Accessibility support by marked content (HO)]%
\documentclass{ltxdoc}
\usepackage{holtxdoc}[2011/11/22]
\begin{document}
  \DocInput{accsupp.dtx}%
\end{document}
%</driver>
% \fi
%
%
% \CharacterTable
%  {Upper-case    \A\B\C\D\E\F\G\H\I\J\K\L\M\N\O\P\Q\R\S\T\U\V\W\X\Y\Z
%   Lower-case    \a\b\c\d\e\f\g\h\i\j\k\l\m\n\o\p\q\r\s\t\u\v\w\x\y\z
%   Digits        \0\1\2\3\4\5\6\7\8\9
%   Exclamation   \!     Double quote  \"     Hash (number) \#
%   Dollar        \$     Percent       \%     Ampersand     \&
%   Acute accent  \'     Left paren    \(     Right paren   \)
%   Asterisk      \*     Plus          \+     Comma         \,
%   Minus         \-     Point         \.     Solidus       \/
%   Colon         \:     Semicolon     \;     Less than     \<
%   Equals        \=     Greater than  \>     Question mark \?
%   Commercial at \@     Left bracket  \[     Backslash     \\
%   Right bracket \]     Circumflex    \^     Underscore    \_
%   Grave accent  \`     Left brace    \{     Vertical bar  \|
%   Right brace   \}     Tilde         \~}
%
% \GetFileInfo{accsupp.drv}
%
% \title{The \xpackage{accsupp} package}
% \date{2018/01/30 v0.5}
% \author{Heiko Oberdiek\thanks
% {Please report any issues at https://github.com/ho-tex/oberdiek/issues}\\
% \xemail{heiko.oberdiek at googlemail.com}}
%
% \maketitle
%
% \begin{abstract}
% Since PDF 1.5 portions of a page can be marked for better
% accessibility support.
% For example, replacement texts or expansions of abbreviations can be
% provided. Package \xpackage{accsupp} starts with providing a minimal
% low-level interface for programmers. Status is experimental.
% \end{abstract}
%
% \tableofcontents
%
% \section{Documentation}
%
% \subsection{Macros}
%
% Section ``10.8 Accessibility Support'' of the PDF reference
% \cite{pdfspec} lists some features that can be added by
% operators for marked content.
%
% \begin{declcs}{BeginAccSupp} \M{options}
% \end{declcs}
% It puts the operator |BDC| in the page stream:
% \begin{quote}
%   |/Span|\\
%   |<<|\dots|>>| \qquad \textit{\% property dictionary}\\
%   |BDC|
% \end{quote}
% The contents of the dictionary is controlled by \meta{options}.
% See sections \ref{sec:feature} and \ref{sec:method}.
%
% \begin{declcs}{EndAccSupp} \M{options}
% \end{declcs}
% It puts the operator |EMC| in the page stream.
% The only option is \xoption{pdfliteral}, see section \ref{sec:pdftex}.
%
% \bigskip
% \noindent\textbf{Note:}
% The caller is responsible for the placement of \cs{BeginAccSupp}
% and \cs{EndAccSupp} pairs. Especially page breaks are not allowed
% in between.
%
% \subsubsection{Feature options}\label{sec:feature}
%
% The PDF reference \cite{pdfspec} describes and explains the different
% features. The names of the feature options are the same as the
% key names for the property dictionary for operator |BDC|, see
% \cs{BeginAccSupp}.
%
% \begin{description}
% \item[\xoption{ActualText}:]
%   Provides a replacement text, see examples in
%   section \ref{sec:example}.
% \item[\xoption{Alt}:]
%   Provides an alternate description.
% \item[\xoption{E}:]
%   Provides the expansion of an abbreviation or an acronym.
% \item[\xoption{Lang}:]
%   Specifies the language.
% \end{description}
%
% \subsubsection{Input methods}\label{sec:method}
%
% Except for \xoption{Lang} option \xoption{method} controls
% how the argument for \xoption{ActualText}, \xoption{Alt},
% and \xoption{E} are interpreted.
% \begin{description}
% \item[\xoption{method=plain}:]
%   The string is only expanded and written without further
%   treatment. Special characters are not protected, thus this method
%   may result in an invalid PDF file.
% \item[\xoption{mehod=escape}:]
%   The string is expanded and special characters are escaped.
%   The result is a valid PDF string.
% \item[\xoption{method=hex}:]
%   The string is given in hexadecimal notation.
%   Section \ref{sec:example:notparallel} shows an example.
% \item[\xoption{method=pdfstringdef}:]
%   If package \xpackage{hyperref} is loaded, then its \cs{pdfstringdef}
%   is used. This method is slow, but useful if the string contains
%   arbitrary \hologo{TeX} code.
% \item[\xoption{unicode}:]
%   This option is needed, if the string is given as Unicode string
%   (16 bit). Internally it adds the string prefix for Unicode.
%   In case of \xoption{method=pdfstringdef} it passes the option
%   to \cs{hypersetup}.
% \end{description}
%
% \subsection{Workaround, option \xoption{space}}
%
% PDF specification says in ``10.8.3 Replacement Text''
% \begin{quote}
% Just as alternative descriptions can be provided for images and other
% items that do not translate naturally into text (\dots), replacement
% text can be specified for content that does translate into text but
% that is represented in a nonstandard way. These nonstandard representations
% might include, for example, glyphs for ligatures or custom characters, or
% inline graphics corresponding to letters in an illuminated manustript or to
% dropped capitals.
% \end{quote}
% However, the ActualText is ignored in Acrobat Reader
% (until version 9 at least), if the content does not contain glyphs.
% Option \xoption{space} adds such an invisible glyph, a space
% character. The font name can be configured by option \xoption{spacefont},
% the default font is \texttt{phvr8r}. The character slot is given
% by option \xoption{spacechar}, default is \texttt{32}, the usual
% position of the space character.
%
% These options can also be given as package options or in
% macro \cs{AccSuppSetup} that takes a key value list as argument.
% Usually only option \xoption{space} is necessary, if the
% contents does not contain glyphs otherwise. Option \xoption{space}
% is enabled by |space| or |space=true| and disabled by |space=false|.
% It is disabled as default. The option is evaluated by \cs{BeginAccSupp}
% and ignored by \cs{EndAccSupp}.
%
% Note: Even with option \xoption{space} I get sometimes wrong texts
% when cut \& paste from AR7/Linux or AR8/Linux, e.g.
% \begin{quote}
%   |Hello| $\rightarrow$ |Helo|, |Helol|, |Hell|, \dots\\
%   |Hello World| $\rightarrow$ |Helo WorldW|, |Helo World|, \dots
% \end{quote}
% I do not know what Acrobat Reader is doing here, thus feedback and
% insights are welcome.
%
% \subsection{Driver options}
%
% Driver options are package options only.
% The special \hologo{TeX} compilers \hologo{pdfTeX}, \hologo{LuaTeX} and \hologo{XeTeX}
% are detected automatically. The default for unrecognized drivers can be
% set by defining \cs{ActualTextDriverDefault}. This can be done
% in the configuration file \xfile{accsupp.cfg}.
%
% \subsubsection{Option \xoption{pdftex}}\label{sec:pdftex}
% Package option \xoption{pdftex} is used for \hologo{pdfTeX} in PDF mode.
% Additionally \cs{BeginAccSupp} and \cs{EndAccSupp} understand
% option \xoption{pdfliteral}. It controls
% the modifier keyword for \cs{pdfliteral}:
% \begin{quote}
%   |pdfliteral=direct| $\Rightarrow$ |\pdfliteral direct{...}|
% \end{quote}
%
% \subsubsection{Option \xoption{luatex}}\label{sec:luatex}
% Package option \xoption{luatex} is used for \hologo{LuaTeX} in PDF mode.
% Additionally \cs{BeginAccSupp} and \cs{EndAccSupp} understand
% option \xoption{pdfliteral}. It controls
% the modifier keyword for \cs{pdfextension} literal:
% \begin{quote}
%   |pdfliteral=direct| $\Rightarrow$ |\pdfextension literal direct{...}|
% \end{quote}
%
% \subsubsection{Option \xoption{dvipdfm}}
%
% Package option \xoption{dvipdfm} and its aliases \xoption{dvipdfmx}
% \xoption{xetex} are used for drivers that support
% \textsf{dvipdfm} specials.
%
% \subsubsection{Option \xoption{dvips}}
%
% Package option \xoption{dvips} and its alias \xoption{dvipsone}
% write pdfmark specials in the output. Unhappily these pdfmark
% operators are ignored by \textsf{ghostscript} (latest tested
% version is 8.54). Perhaps they are recognized by commercial
% distiller applications.
%
% \subsubsection{Turning off page stream compression}
%
% For debugging it is useful to have uncompressed page stream objects.
% This can be done afterwards via \textsf{pdftk}:
% \begin{quote}
%   |pdftk file.pdf output file-uncompressed.pdf uncompress|
% \end{quote}
% Or the PDF file is generated uncompressed:
% \begin{description}
% \item[\hologo{pdfTeX}:] |\pdfcompresslevel=0|
% \item[dvipdfm:] |dvipdfm -z0 ...|
% \item[dvipdfmx:] |dvipdfmx -z0 ...|
% \item[ghostscript/ps2pdf:]
%      |ps2pdf -dCompressPages=false input.ps output.pdf|
% \end{description}
%
% \section{Example}
% \label{sec:example}
%
% \subsection{Example \cs{notparallel}}
% \label{sec:example:notparallel}
%
%    \begin{macrocode}
%<*example1>
%    \end{macrocode}
%    \begin{macrocode}
%<<END
\documentclass{article}
\usepackage{accsupp}[2007/11/14]
\usepackage{centernot}
% U+2226 NOT PARALLEL
% \mathrel{...} prevents page break in between
\newcommand*{\notparallel}{%
  \ensuremath{%
    \mathrel{%
      \BeginAccSupp{method=hex,unicode,ActualText=2226}%
      \centernot{\parallel}%
      \EndAccSupp{}%
    }%
  }%
}
\begin{document}
\begin{equation}
A\notparallel B
\end{equation}
\end{document}
%END
%    \end{macrocode}
%    \begin{macrocode}
%</example1>
%    \end{macrocode}
%
% \subsection{Example with \xoption{pdfstringdef}}
%
%    \begin{macrocode}
%<*example2>
%    \end{macrocode}
%    \begin{macrocode}
%<<END
\documentclass{article}
\usepackage[unicode]{hyperref}
\usepackage{accsupp}[2007/11/14]
\begin{document}
  \begin{equation}
    \BeginAccSupp{
      method=pdfstringdef,
      unicode,
      ActualText={%
        a\texttwosuperior +b\texttwosuperior
        =c\texttwosuperior
      }
    }
    a^2 + b^2 = c^2
    \EndAccSupp{}
  \end{equation}
\end{document}
%END
%    \end{macrocode}
%    \begin{macrocode}
%</example2>
%    \end{macrocode}
%
% \StopEventually{
% }
%
% \section{Implementation}
%
% \subsection{Package}
%
%    \begin{macrocode}
%<*package>
%    \end{macrocode}
%
%    \begin{macrocode}
\begingroup\catcode61\catcode48\catcode32=10\relax%
  \catcode13=5 % ^^M
  \endlinechar=13 %
  \catcode123=1 % {
  \catcode125=2 % }
  \catcode64=11 % @
  \def\x{\endgroup
    \expandafter\edef\csname ACCSUPP@AtEnd\endcsname{%
      \endlinechar=\the\endlinechar\relax
      \catcode13=\the\catcode13\relax
      \catcode32=\the\catcode32\relax
      \catcode35=\the\catcode35\relax
      \catcode61=\the\catcode61\relax
      \catcode64=\the\catcode64\relax
      \catcode123=\the\catcode123\relax
      \catcode125=\the\catcode125\relax
    }%
  }%
\x\catcode61\catcode48\catcode32=10\relax%
\catcode13=5 % ^^M
\endlinechar=13 %
\catcode35=6 % #
\catcode64=11 % @
\catcode123=1 % {
\catcode125=2 % }
\def\TMP@EnsureCode#1#2{%
  \edef\ACCSUPP@AtEnd{%
    \ACCSUPP@AtEnd
    \catcode#1=\the\catcode#1\relax
  }%
  \catcode#1=#2\relax
}
\TMP@EnsureCode{10}{12}% ^^J
\TMP@EnsureCode{33}{12}% !
\TMP@EnsureCode{39}{12}% '
\TMP@EnsureCode{40}{12}% (
\TMP@EnsureCode{41}{12}% )
\TMP@EnsureCode{42}{12}% *
\TMP@EnsureCode{44}{12}% ,
\TMP@EnsureCode{45}{12}% -
\TMP@EnsureCode{46}{12}% .
\TMP@EnsureCode{47}{12}% /
\TMP@EnsureCode{58}{12}% :
\TMP@EnsureCode{60}{12}% <
\TMP@EnsureCode{62}{12}% >
\TMP@EnsureCode{94}{7}% ^ (superscript)
\TMP@EnsureCode{96}{12}% `
\TMP@EnsureCode{254}{12}% ^^fe
\TMP@EnsureCode{255}{12}% ^^ff
\edef\ACCSUPP@AtEnd{\ACCSUPP@AtEnd\noexpand\endinput}
%    \end{macrocode}
%
%    Package identification.
%    \begin{macrocode}
\NeedsTeXFormat{LaTeX2e}
\ProvidesPackage{accsupp}%
  [2018/01/30 v0.5 Accessibility support by marked content (HO)]
%    \end{macrocode}
%
%    \begin{macrocode}
\RequirePackage{pdfescape}[2007/02/25]
\RequirePackage{ifpdf}
\RequirePackage{ifxetex}
\RequirePackage{kvoptions}
%    \end{macrocode}
%    \begin{macrocode}
\SetupKeyvalOptions{%
  family=ACCSUPP,%
  prefix=ACCSUPP@%
}
%    \end{macrocode}
%
% \subsection{Space setup}
%
%    \begin{macrocode}
\DeclareBoolOption{space}
\DeclareStringOption[phvr8r]{spacefont}
\DeclareStringOption[32]{spacechar}
%    \end{macrocode}
%
% \subsection{Driver detection and setup}
%
% \paragraph{Driver declarations.}
%
%    \begin{macrocode}
\def\ACCSUPP@DefineDriverKey{%
  \@dblarg\ACCSUPP@@DefineDriverKey
}
\def\ACCSUPP@@DefineDriverKey[#1]#2{%
  \define@key{ACCSUPP}{#2}[]{%
    \def\ACCSUPP@driver{#1}%
  }%
  \g@addto@macro\ACCSUPP@DisableOptions{%
    \DisableKeyvalOption{ACCSUPP}{#2}%
  }%
}
\let\ACCSUPP@DisableOptions\@empty
\ACCSUPP@DefineDriverKey{pdftex}
\ACCSUPP@DefineDriverKey{luatex}
\ACCSUPP@DefineDriverKey{dvips}
\ACCSUPP@DefineDriverKey[dvips]{dvipsone}
\ACCSUPP@DefineDriverKey{dvipdfm}
\ACCSUPP@DefineDriverKey[dvipdfm]{dvipdfmx}
\ACCSUPP@DefineDriverKey[dvipdfm]{xetex}
\let\ACCSUPP@driver\relax
\InputIfFileExists{accsupp.cfg}{}{}
\providecommand*{\ActualTextDriverDefault}{dvips}
\ifpdf
  \ifx\pdfextension\@undefined
    \def\ACCSUPP@driver{pdftex}%
  \else
    \def\ACCSUPP@driver{luatex}%
  \fi
\else
  \ifxetex
    \def\ACCSUPP@driver{dvipdfm}%
  \else
    \ifx\ACCSUPP@driver\relax
      \let\ACCSUPP@driver\ActualTextDriverDefault
    \fi
  \fi
\fi
%    \end{macrocode}
%
% \paragraph{Process options.}
%
%    \begin{macrocode}
\ProcessKeyvalOptions*
\ACCSUPP@DisableOptions
%    \end{macrocode}
%
% \paragraph{Driver validation and loading.}
%
%    \begin{macrocode}
\def\ACCSUPP@temp{pdftex}%
\ifpdf
  \ifx\pdfextension\@undefined\else
    \def\ACCSUPP@temp{luatex}%
  \fi
  \ifx\ACCSUPP@temp\ACCSUPP@driver
  \else
    \PackageWarningNoLine{accsupp}{%
      Wrong driver `\ACCSUPP@driver', using `pdftex' instead%
    }%
ed as
    \let\ACCSUPP@driver\ACCSUPP@temp
  \fi
\else
  \ifx\ACCSUPP@temp\ACCSUPP@driver
    \PackageError{accsupp}{%
      Wrong driver, pdfTeX is not running in PDF mode.\MessageBreak
      Package loading is aborted%
    }\@ehc
    \expandafter\expandafter\expandafter\ACCSUPP@AtEnd
  \fi
  \def\ACCSUPP@temp{dvipdfm}%
  \ifxetex
    \ifx\ACCSUPP@temp\ACCSUPP@driver
    \else
      \PackageWarningNoLine{accsupp}{%
        Wrong driver `\ACCSUPP@driver',\MessageBreak
        using `dvipdfm' for XeTeX instead%
      }%
      \let\ACCSUPP@driver\ACCSUPP@temp
    \fi
  \fi
\fi%
\ifx\ACCSUPP@driver\relax
  \PackageError{accsupp}{%
    Missing driver option.\MessageBreak
    Package loading is aborted%
  }\@ehc
  \expandafter\ACCSUPP@AtEnd
\fi%
\InputIfFileExists{accsupp-\ACCSUPP@driver.def}{}{%
  \PackageError{accsupp}{%
    Unsupported driver `\ACCSUPP@driver'.\MessageBreak
    Package loading is aborted%
  }\@ehc
  \ACCSUPP@AtEnd
}%
%    \end{macrocode}
%
% \subsection{Main macro}
%
%    \begin{macrocode}
\DeclareBoolOption{unicode}
\DeclareStringOption[page]{pdfliteral}
\DeclareStringOption{Lang}
\def\ACCSUPP@method{escape}
\define@key{ACCSUPP}{method}{%
  \@ifundefined{ACCSUPP@method@#1}{%
    \PackageError{accsupp}{%
      Ignoring unknown method `#1'%
    }\@ehc
  }{%
    \edef\ACCSUPP@method{#1}%
  }%
}
\let\ACCSUPP@Lang\relax
\def\ACCSUPP@temp#1{%
  \expandafter\ACCSUPP@@temp\csname ACCSUPP@#1\endcsname{#1}%
}
\def\ACCSUPP@@temp#1#2{%
  \let#1\relax
  \define@key{ACCSUPP}{#2}{%
    \def#1{##1}%
    \ifx#1\@empty
      \def#1{()}%
    \else
      \csname ACCSUPP@method@\ACCSUPP@method\endcsname#1%
    \fi
  }%
}
\ACCSUPP@temp{Alt}
\ACCSUPP@temp{ActualText}
\ACCSUPP@temp{E}
%    \end{macrocode}
%
%    \begin{macrocode}
\newcommand*{\BeginAccSupp}[1]{%
  \begingroup
    \setkeys{ACCSUPP}{#1}%
    \edef\ACCSUPP@span{%
      /Span<<%
        \ifx\ACCSUPP@Lang\relax
        \else
          /Lang\ACCSUPP@Lang
        \fi
        \ifx\ACCSUPP@Alt\relax
        \else
          /Alt\ACCSUPP@Alt
        \fi
        \ifx\ACCSUPP@ActualText\relax
        \else
          /ActualText\ACCSUPP@ActualText
        \fi
        \ifx\ACCSUPP@E\relax
        \else
          /E\ACCSUPP@E
        \fi
      >>%
    }%
    \ACCSUPP@bdc
    \ACCSUPP@space
  \endgroup
}
\newcommand*{\EndAccSupp}[1]{%
  \begingroup
    \setkeys{ACCSUPP}{#1}%
    \ACCSUPP@emc
  \endgroup
}
\def\ACCSUPP@space{%
  \ifACCSUPP@space
    \begingroup
      \@ifundefined{ACCSUPP@Font}{%
        \global\font\ACCSUPP@Font=\ACCSUPP@spacefont\relax
      }{}%
      \leavevmode
      \setbox\z@\hbox{\ACCSUPP@Font\char\ACCSUPP@spacechar}%
      \wd\z@\z@
      \ht\z@\z@
      \dp\z@\z@
      \copy\z@
    \endgroup
  \fi
}
\newcommand*{\AccSuppSetup}{%
  \setkeys{ACCSUPP}%
}
%    \end{macrocode}
%
% \subsubsection{Input methods}
%
%    \begin{macrocode}
\def\ACCSUPP@method@plain#1{%
  \csname @safe@activestrue\endcsname
  \edef#1{%
    (%
    \ifACCSUPP@unicode
      \string\376\string\377%
    \fi
    #1%
    )%
  }%
  \@onelevel@sanitize#1%
}
%    \end{macrocode}
%    \begin{macrocode}
\def\ACCSUPP@method@escape#1{%
  \EdefEscapeString#1{%
    \ifACCSUPP@unicode
      ^^fe^^ff%
    \fi
    #1%
  }%
  \edef#1{(#1)}%
}%
%    \end{macrocode}
%    \begin{macrocode}
\def\ACCSUPP@method@hex#1{%
  \edef#1{%
    <%
    \ifACCSUPP@unicode
      FEFF%
    \fi
    #1%
    >%
  }%
}
%    \end{macrocode}
%    \begin{macrocode}
\def\ACCSUPP@method@pdfstringdef#1{%
  \ifACCSUPP@unicode
    \@ifundefined{hypersetup}{}{%
      \hypersetup{unicode}%
    }%
  \fi
  \@ifundefined{pdfstringdef}{%
    \PackageError{accsupp}{%
      Method `pdfstringdef' requires package `hyperref'%
    }\@ehc
    \let\ACCSUPP@temp\@empty
  }{%
    \begingroup
      \setbox0=\hbox{%
        \pdfstringdef\ACCSUPP@temp#1%
        \global\let\ACCSUPP@temp\ACCSUPP@temp
      }%
    \endgroup
  }%
  \edef#1{(\ACCSUPP@temp)}%
}
%    \end{macrocode}
%
%    \begin{macrocode}
\ACCSUPP@AtEnd%
%</package>
%    \end{macrocode}
%
% \subsection{Drivers}
%
% \subsubsection{Driver \xoption{pdftex}}
%
%    \begin{macrocode}
%<*pdftex>
\NeedsTeXFormat{LaTeX2e}
\ProvidesFile{accsupp-pdftex.def}%
  [2018/01/30 v0.5 accsupp driver for pdfTeX (HO)]%
%    \end{macrocode}
%
%    \begin{macrocode}
\def\ACCSUPP@bdc{%
  \pdfliteral\ACCSUPP@pdfliteral{\ACCSUPP@span BDC}%
}
\def\ACCSUPP@emc{%
  \pdfliteral\ACCSUPP@pdfliteral{EMC}%
}
%    \end{macrocode}
%    \begin{macrocode}
%</pdftex>
%    \end{macrocode}
%
% \subsubsection{Driver \xoption{luatex}}
%
%    \begin{macrocode}
%<*luatex>
\NeedsTeXFormat{LaTeX2e}
\ProvidesFile{accsupp-luatex.def}%
  [2018/01/30 v0.5 accsupp driver for pdfTeX (HO)]%
%    \end{macrocode}
%
%    \begin{macrocode}
\protected\def\ACCSUPP@pdf@literal{\pdfextension literal}
\def\ACCSUPP@bdc{%
  \ACCSUPP@pdf@literal\ACCSUPP@pdfliteral{\ACCSUPP@span BDC}%
}
\def\ACCSUPP@emc{%
  \ACCSUPP@pdf@literal\ACCSUPP@pdfliteral{EMC}%
}
%    \end{macrocode}
%    \begin{macrocode}
%</luatex>
%    \end{macrocode}
%
% \subsubsection{Driver \xoption{dvipdfm}}
%
%    \begin{macrocode}
%<*dvipdfm>
\NeedsTeXFormat{LaTeX2e}
\ProvidesFile{accsupp-dvipdfm.def}%
  [2018/01/30 v0.5 accsupp driver for dvipdfm (HO)]%
%    \end{macrocode}
%
%    \begin{macrocode}
\def\ACCSUPP@bdc{%
  \special{pdf:content \ACCSUPP@span BDC}%
}
\def\ACCSUPP@emc{%
  \special{pdf:content EMC}%
}
%    \end{macrocode}
%    \begin{macrocode}
%</dvipdfm>
%    \end{macrocode}
%
% \subsubsection{Driver \xoption{dvips}}
%
%    \begin{macrocode}
%<*dvips>
\NeedsTeXFormat{LaTeX2e}
\ProvidesFile{accsupp-dvips.def}%
  [2018/01/30 v0.5 accsupp driver for dvips (HO)]%
%    \end{macrocode}
%
%    \begin{macrocode}
\def\ACCSUPP@bdc{%
  \special{ps:[\ACCSUPP@span/BDC pdfmark}%
}
\def\ACCSUPP@emc{%
  \special{ps:[/EMC pdfmark}%
}
%    \end{macrocode}
%    \begin{macrocode}
%</dvips>
%    \end{macrocode}
%
% \section{Test}
%
% \subsection{Catcode checks for loading}
%
%    \begin{macrocode}
%<*test1>
%    \end{macrocode}
%    \begin{macrocode}
\NeedsTeXFormat{LaTeX2e}
\documentclass{minimal}
\makeatletter
\def\RestoreCatcodes{}
\count@=0 %
\loop
  \edef\RestoreCatcodes{%
    \RestoreCatcodes
    \catcode\the\count@=\the\catcode\count@\relax
  }%
\ifnum\count@<255 %
  \advance\count@\@ne
\repeat

\def\RangeCatcodeInvalid#1#2{%
  \count@=#1\relax
  \loop
    \catcode\count@=15 %
  \ifnum\count@<#2\relax
    \advance\count@\@ne
  \repeat
}
\def\Test{%
  \RangeCatcodeInvalid{0}{47}%
  \RangeCatcodeInvalid{58}{64}%
  \RangeCatcodeInvalid{91}{96}%
  \RangeCatcodeInvalid{123}{127}%
  \catcode`\@=12 %
  \catcode`\\=0 %
  \catcode`\{=1 %
  \catcode`\}=2 %
  \catcode`\#=6 %
  \catcode`\[=12 %
  \catcode`\]=12 %
  \catcode`\%=14 %
  \catcode`\ =10 %
  \catcode13=5 %
  \RequirePackage{accsupp}[2018/01/30]\relax
  \RestoreCatcodes
}
\Test
\csname @@end\endcsname
\end
%    \end{macrocode}
%    \begin{macrocode}
%</test1>
%    \end{macrocode}
%
% \section{Installation}
%
% \subsection{Download}
%
% \paragraph{Package.} This package is available on
% CTAN\footnote{\url{http://ctan.org/pkg/accsupp}}:
% \begin{description}
% \item[\CTAN{macros/latex/contrib/oberdiek/accsupp.dtx}] The source file.
% \item[\CTAN{macros/latex/contrib/oberdiek/accsupp.pdf}] Documentation.
% \end{description}
%
%
% \paragraph{Bundle.} All the packages of the bundle `oberdiek'
% are also available in a TDS compliant ZIP archive. There
% the packages are already unpacked and the documentation files
% are generated. The files and directories obey the TDS standard.
% \begin{description}
% \item[\CTAN{install/macros/latex/contrib/oberdiek.tds.zip}]
% \end{description}
% \emph{TDS} refers to the standard ``A Directory Structure
% for \TeX\ Files'' (\CTAN{tds/tds.pdf}). Directories
% with \xfile{texmf} in their name are usually organized this way.
%
% \subsection{Bundle installation}
%
% \paragraph{Unpacking.} Unpack the \xfile{oberdiek.tds.zip} in the
% TDS tree (also known as \xfile{texmf} tree) of your choice.
% Example (linux):
% \begin{quote}
%   |unzip oberdiek.tds.zip -d ~/texmf|
% \end{quote}
%
% \paragraph{Script installation.}
% Check the directory \xfile{TDS:scripts/oberdiek/} for
% scripts that need further installation steps.
% Package \xpackage{attachfile2} comes with the Perl script
% \xfile{pdfatfi.pl} that should be installed in such a way
% that it can be called as \texttt{pdfatfi}.
% Example (linux):
% \begin{quote}
%   |chmod +x scripts/oberdiek/pdfatfi.pl|\\
%   |cp scripts/oberdiek/pdfatfi.pl /usr/local/bin/|
% \end{quote}
%
% \subsection{Package installation}
%
% \paragraph{Unpacking.} The \xfile{.dtx} file is a self-extracting
% \docstrip\ archive. The files are extracted by running the
% \xfile{.dtx} through \plainTeX:
% \begin{quote}
%   \verb|tex accsupp.dtx|
% \end{quote}
%
% \paragraph{TDS.} Now the different files must be moved into
% the different directories in your installation TDS tree
% (also known as \xfile{texmf} tree):
% \begin{quote}
% \def\t{^^A
% \begin{tabular}{@{}>{\ttfamily}l@{ $\rightarrow$ }>{\ttfamily}l@{}}
%   accsupp.sty & tex/latex/oberdiek/accsupp.sty\\
%   accsupp-pdftex.def & tex/latex/oberdiek/accsupp-pdftex.def\\
%   accsupp-luatex.def & tex/latex/oberdiek/accsupp-luatex.def\\
%   accsupp-dvipdfm.def & tex/latex/oberdiek/accsupp-dvipdfm.def\\
%   accsupp-dvips.def & tex/latex/oberdiek/accsupp-dvips.def\\
%   accsupp.pdf & doc/latex/oberdiek/accsupp.pdf\\
%   accsupp-example1.tex & doc/latex/oberdiek/accsupp-example1.tex\\
%   accsupp-example2.tex & doc/latex/oberdiek/accsupp-example2.tex\\
%   test/accsupp-test1.tex & doc/latex/oberdiek/test/accsupp-test1.tex\\
%   accsupp.dtx & source/latex/oberdiek/accsupp.dtx\\
% \end{tabular}^^A
% }^^A
% \sbox0{\t}^^A
% \ifdim\wd0>\linewidth
%   \begingroup
%     \advance\linewidth by\leftmargin
%     \advance\linewidth by\rightmargin
%   \edef\x{\endgroup
%     \def\noexpand\lw{\the\linewidth}^^A
%   }\x
%   \def\lwbox{^^A
%     \leavevmode
%     \hbox to \linewidth{^^A
%       \kern-\leftmargin\relax
%       \hss
%       \usebox0
%       \hss
%       \kern-\rightmargin\relax
%     }^^A
%   }^^A
%   \ifdim\wd0>\lw
%     \sbox0{\small\t}^^A
%     \ifdim\wd0>\linewidth
%       \ifdim\wd0>\lw
%         \sbox0{\footnotesize\t}^^A
%         \ifdim\wd0>\linewidth
%           \ifdim\wd0>\lw
%             \sbox0{\scriptsize\t}^^A
%             \ifdim\wd0>\linewidth
%               \ifdim\wd0>\lw
%                 \sbox0{\tiny\t}^^A
%                 \ifdim\wd0>\linewidth
%                   \lwbox
%                 \else
%                   \usebox0
%                 \fi
%               \else
%                 \lwbox
%               \fi
%             \else
%               \usebox0
%             \fi
%           \else
%             \lwbox
%           \fi
%         \else
%           \usebox0
%         \fi
%       \else
%         \lwbox
%       \fi
%     \else
%       \usebox0
%     \fi
%   \else
%     \lwbox
%   \fi
% \else
%   \usebox0
% \fi
% \end{quote}
% If you have a \xfile{docstrip.cfg} that configures and enables \docstrip's
% TDS installing feature, then some files can already be in the right
% place, see the documentation of \docstrip.
%
% \subsection{Refresh file name databases}
%
% If your \TeX~distribution
% (\teTeX, \mikTeX, \dots) relies on file name databases, you must refresh
% these. For example, \teTeX\ users run \verb|texhash| or
% \verb|mktexlsr|.
%
% \subsection{Some details for the interested}
%
% \paragraph{Attached source.}
%
% The PDF documentation on CTAN also includes the
% \xfile{.dtx} source file. It can be extracted by
% AcrobatReader 6 or higher. Another option is \textsf{pdftk},
% e.g. unpack the file into the current directory:
% \begin{quote}
%   \verb|pdftk accsupp.pdf unpack_files output .|
% \end{quote}
%
% \paragraph{Unpacking with \LaTeX.}
% The \xfile{.dtx} chooses its action depending on the format:
% \begin{description}
% \item[\plainTeX:] Run \docstrip\ and extract the files.
% \item[\LaTeX:] Generate the documentation.
% \end{description}
% If you insist on using \LaTeX\ for \docstrip\ (really,
% \docstrip\ does not need \LaTeX), then inform the autodetect routine
% about your intention:
% \begin{quote}
%   \verb|latex \let\install=y% \iffalse meta-comment
%
% File: accsupp.dtx
% Version: 2018/01/30 v0.5
% Info: Accessibility support by marked content
%
% Copyright (C) 2007, 2010 by
%    Heiko Oberdiek <heiko.oberdiek at googlemail.com>
%    2016
%    https://github.com/ho-tex/oberdiek/issues
%
% This work may be distributed and/or modified under the
% conditions of the LaTeX Project Public License, either
% version 1.3c of this license or (at your option) any later
% version. This version of this license is in
%    http://www.latex-project.org/lppl/lppl-1-3c.txt
% and the latest version of this license is in
%    http://www.latex-project.org/lppl.txt
% and version 1.3 or later is part of all distributions of
% LaTeX version 2005/12/01 or later.
%
% This work has the LPPL maintenance status "maintained".
%
% This Current Maintainer of this work is Heiko Oberdiek.
%
% This work consists of the main source file accsupp.dtx
% and the derived files
%    accsupp.sty, accsupp.pdf, accsupp.ins, accsupp.drv,
%    accsupp-pdftex.def, \ccsupp-luatex.def, accsupp-dvipdfm.def,
%    accsupp-dvips.def,
%    accsupp-example1.tex, accsupp-example2.tex,
%    accsupp-test1.tex.
%
% Distribution:
%    CTAN:macros/latex/contrib/oberdiek/accsupp.dtx
%    CTAN:macros/latex/contrib/oberdiek/accsupp.pdf
%
% Unpacking:
%    (a) If accsupp.ins is present:
%           tex accsupp.ins
%    (b) Without accsupp.ins:
%           tex accsupp.dtx
%    (c) If you insist on using LaTeX
%           latex \let\install=y% \iffalse meta-comment
%
% File: accsupp.dtx
% Version: 2018/01/30 v0.5
% Info: Accessibility support by marked content
%
% Copyright (C) 2007, 2010 by
%    Heiko Oberdiek <heiko.oberdiek at googlemail.com>
%    2016
%    https://github.com/ho-tex/oberdiek/issues
%
% This work may be distributed and/or modified under the
% conditions of the LaTeX Project Public License, either
% version 1.3c of this license or (at your option) any later
% version. This version of this license is in
%    http://www.latex-project.org/lppl/lppl-1-3c.txt
% and the latest version of this license is in
%    http://www.latex-project.org/lppl.txt
% and version 1.3 or later is part of all distributions of
% LaTeX version 2005/12/01 or later.
%
% This work has the LPPL maintenance status "maintained".
%
% This Current Maintainer of this work is Heiko Oberdiek.
%
% This work consists of the main source file accsupp.dtx
% and the derived files
%    accsupp.sty, accsupp.pdf, accsupp.ins, accsupp.drv,
%    accsupp-pdftex.def, \ccsupp-luatex.def, accsupp-dvipdfm.def,
%    accsupp-dvips.def,
%    accsupp-example1.tex, accsupp-example2.tex,
%    accsupp-test1.tex.
%
% Distribution:
%    CTAN:macros/latex/contrib/oberdiek/accsupp.dtx
%    CTAN:macros/latex/contrib/oberdiek/accsupp.pdf
%
% Unpacking:
%    (a) If accsupp.ins is present:
%           tex accsupp.ins
%    (b) Without accsupp.ins:
%           tex accsupp.dtx
%    (c) If you insist on using LaTeX
%           latex \let\install=y\input{accsupp.dtx}
%        (quote the arguments according to the demands of your shell)
%
% Documentation:
%    (a) If accsupp.drv is present:
%           latex accsupp.drv
%    (b) Without accsupp.drv:
%           latex accsupp.dtx; ...
%    The class ltxdoc loads the configuration file ltxdoc.cfg
%    if available. Here you can specify further options, e.g.
%    use A4 as paper format:
%       \PassOptionsToClass{a4paper}{article}
%
%    Programm calls to get the documentation (example):
%       pdflatex accsupp.dtx
%       makeindex -s gind.ist accsupp.idx
%       pdflatex accsupp.dtx
%       makeindex -s gind.ist accsupp.idx
%       pdflatex accsupp.dtx
%
% Installation:
%    TDS:tex/latex/oberdiek/accsupp.sty
%    TDS:tex/latex/oberdiek/accsupp-pdftex.def
%    TDS:tex/latex/oberdiek/accsupp-luatex.def
%    TDS:tex/latex/oberdiek/accsupp-dvipdfm.def
%    TDS:tex/latex/oberdiek/accsupp-dvips.def
%    TDS:doc/latex/oberdiek/accsupp.pdf
%    TDS:doc/latex/oberdiek/accsupp-example1.tex
%    TDS:doc/latex/oberdiek/accsupp-example2.tex
%    TDS:doc/latex/oberdiek/test/accsupp-test1.tex
%    TDS:source/latex/oberdiek/accsupp.dtx
%
%<*ignore>
\begingroup
  \catcode123=1 %
  \catcode125=2 %
  \def\x{LaTeX2e}%
\expandafter\endgroup
\ifcase 0\ifx\install y1\fi\expandafter
         \ifx\csname processbatchFile\endcsname\relax\else1\fi
         \ifx\fmtname\x\else 1\fi\relax
\else\csname fi\endcsname
%</ignore>
%<*install>
\input docstrip.tex
\Msg{************************************************************************}
\Msg{* Installation}
\Msg{* Package: accsupp 2018/01/30 v0.5 Accessibility support by marked content (HO)}
\Msg{************************************************************************}

\keepsilent
\askforoverwritefalse

\let\MetaPrefix\relax
\preamble

This is a generated file.

Project: accsupp
Version: 2018/01/30 v0.5

Copyright (C) 2007, 2010 by
   Heiko Oberdiek <heiko.oberdiek at googlemail.com>

This work may be distributed and/or modified under the
conditions of the LaTeX Project Public License, either
version 1.3c of this license or (at your option) any later
version. This version of this license is in
   http://www.latex-project.org/lppl/lppl-1-3c.txt
and the latest version of this license is in
   http://www.latex-project.org/lppl.txt
and version 1.3 or later is part of all distributions of
LaTeX version 2005/12/01 or later.

This work has the LPPL maintenance status "maintained".

This Current Maintainer of this work is Heiko Oberdiek.

This work consists of the main source file accsupp.dtx
and the derived files
   accsupp.sty, accsupp.pdf, accsupp.ins, accsupp.drv,
   accsupp-pdftex.def, accsupp-luatex.def, accsupp-dvipdfm.def,
   accsupp-dvips.def,
   accsupp-example1.tex, accsupp-example2.tex,
   accsupp-test1.tex.

\endpreamble
\let\MetaPrefix\DoubleperCent

\generate{%
  \file{accsupp.ins}{\from{accsupp.dtx}{install}}%
  \file{accsupp.drv}{\from{accsupp.dtx}{driver}}%
  \usedir{tex/latex/oberdiek}%
  \file{accsupp.sty}{\from{accsupp.dtx}{package}}%
  \file{accsupp-pdftex.def}{\from{accsupp.dtx}{pdftex}}%
  \file{accsupp-luatex.def}{\from{accsupp.dtx}{luatex}}%
  \file{accsupp-dvipdfm.def}{\from{accsupp.dtx}{dvipdfm}}%
  \file{accsupp-dvips.def}{\from{accsupp.dtx}{dvips}}%
  \usedir{doc/latex/oberdiek}%
  \file{accsupp-example1.tex}{\from{accsupp.dtx}{example1}}%
  \file{accsupp-example2.tex}{\from{accsupp.dtx}{example2}}%
  \usedir{doc/latex/oberdiek/test}%
  \file{accsupp-test1.tex}{\from{accsupp.dtx}{test1}}%
  \nopreamble
  \nopostamble
  \usedir{source/latex/oberdiek/catalogue}%
  \file{accsupp.xml}{\from{accsupp.dtx}{catalogue}}%
}

\catcode32=13\relax% active space
\let =\space%
\Msg{************************************************************************}
\Msg{*}
\Msg{* To finish the installation you have to move the following}
\Msg{* files into a directory searched by TeX:}
\Msg{*}
\Msg{*     accsupp.sty, accsupp-pdftex.def, accsupp-dvipdfm.def,}
\Msg{*     accsupp-dvips.def}
\Msg{*}
\Msg{* To produce the documentation run the file `accsupp.drv'}
\Msg{* through LaTeX.}
\Msg{*}
\Msg{* Happy TeXing!}
\Msg{*}
\Msg{************************************************************************}

\endbatchfile
%</install>
%<*ignore>
\fi
%</ignore>
%<*driver>
\NeedsTeXFormat{LaTeX2e}
\ProvidesFile{accsupp.drv}%
  [2018/01/30 v0.5 Accessibility support by marked content (HO)]%
\documentclass{ltxdoc}
\usepackage{holtxdoc}[2011/11/22]
\begin{document}
  \DocInput{accsupp.dtx}%
\end{document}
%</driver>
% \fi
%
%
% \CharacterTable
%  {Upper-case    \A\B\C\D\E\F\G\H\I\J\K\L\M\N\O\P\Q\R\S\T\U\V\W\X\Y\Z
%   Lower-case    \a\b\c\d\e\f\g\h\i\j\k\l\m\n\o\p\q\r\s\t\u\v\w\x\y\z
%   Digits        \0\1\2\3\4\5\6\7\8\9
%   Exclamation   \!     Double quote  \"     Hash (number) \#
%   Dollar        \$     Percent       \%     Ampersand     \&
%   Acute accent  \'     Left paren    \(     Right paren   \)
%   Asterisk      \*     Plus          \+     Comma         \,
%   Minus         \-     Point         \.     Solidus       \/
%   Colon         \:     Semicolon     \;     Less than     \<
%   Equals        \=     Greater than  \>     Question mark \?
%   Commercial at \@     Left bracket  \[     Backslash     \\
%   Right bracket \]     Circumflex    \^     Underscore    \_
%   Grave accent  \`     Left brace    \{     Vertical bar  \|
%   Right brace   \}     Tilde         \~}
%
% \GetFileInfo{accsupp.drv}
%
% \title{The \xpackage{accsupp} package}
% \date{2018/01/30 v0.5}
% \author{Heiko Oberdiek\thanks
% {Please report any issues at https://github.com/ho-tex/oberdiek/issues}\\
% \xemail{heiko.oberdiek at googlemail.com}}
%
% \maketitle
%
% \begin{abstract}
% Since PDF 1.5 portions of a page can be marked for better
% accessibility support.
% For example, replacement texts or expansions of abbreviations can be
% provided. Package \xpackage{accsupp} starts with providing a minimal
% low-level interface for programmers. Status is experimental.
% \end{abstract}
%
% \tableofcontents
%
% \section{Documentation}
%
% \subsection{Macros}
%
% Section ``10.8 Accessibility Support'' of the PDF reference
% \cite{pdfspec} lists some features that can be added by
% operators for marked content.
%
% \begin{declcs}{BeginAccSupp} \M{options}
% \end{declcs}
% It puts the operator |BDC| in the page stream:
% \begin{quote}
%   |/Span|\\
%   |<<|\dots|>>| \qquad \textit{\% property dictionary}\\
%   |BDC|
% \end{quote}
% The contents of the dictionary is controlled by \meta{options}.
% See sections \ref{sec:feature} and \ref{sec:method}.
%
% \begin{declcs}{EndAccSupp} \M{options}
% \end{declcs}
% It puts the operator |EMC| in the page stream.
% The only option is \xoption{pdfliteral}, see section \ref{sec:pdftex}.
%
% \bigskip
% \noindent\textbf{Note:}
% The caller is responsible for the placement of \cs{BeginAccSupp}
% and \cs{EndAccSupp} pairs. Especially page breaks are not allowed
% in between.
%
% \subsubsection{Feature options}\label{sec:feature}
%
% The PDF reference \cite{pdfspec} describes and explains the different
% features. The names of the feature options are the same as the
% key names for the property dictionary for operator |BDC|, see
% \cs{BeginAccSupp}.
%
% \begin{description}
% \item[\xoption{ActualText}:]
%   Provides a replacement text, see examples in
%   section \ref{sec:example}.
% \item[\xoption{Alt}:]
%   Provides an alternate description.
% \item[\xoption{E}:]
%   Provides the expansion of an abbreviation or an acronym.
% \item[\xoption{Lang}:]
%   Specifies the language.
% \end{description}
%
% \subsubsection{Input methods}\label{sec:method}
%
% Except for \xoption{Lang} option \xoption{method} controls
% how the argument for \xoption{ActualText}, \xoption{Alt},
% and \xoption{E} are interpreted.
% \begin{description}
% \item[\xoption{method=plain}:]
%   The string is only expanded and written without further
%   treatment. Special characters are not protected, thus this method
%   may result in an invalid PDF file.
% \item[\xoption{mehod=escape}:]
%   The string is expanded and special characters are escaped.
%   The result is a valid PDF string.
% \item[\xoption{method=hex}:]
%   The string is given in hexadecimal notation.
%   Section \ref{sec:example:notparallel} shows an example.
% \item[\xoption{method=pdfstringdef}:]
%   If package \xpackage{hyperref} is loaded, then its \cs{pdfstringdef}
%   is used. This method is slow, but useful if the string contains
%   arbitrary \hologo{TeX} code.
% \item[\xoption{unicode}:]
%   This option is needed, if the string is given as Unicode string
%   (16 bit). Internally it adds the string prefix for Unicode.
%   In case of \xoption{method=pdfstringdef} it passes the option
%   to \cs{hypersetup}.
% \end{description}
%
% \subsection{Workaround, option \xoption{space}}
%
% PDF specification says in ``10.8.3 Replacement Text''
% \begin{quote}
% Just as alternative descriptions can be provided for images and other
% items that do not translate naturally into text (\dots), replacement
% text can be specified for content that does translate into text but
% that is represented in a nonstandard way. These nonstandard representations
% might include, for example, glyphs for ligatures or custom characters, or
% inline graphics corresponding to letters in an illuminated manustript or to
% dropped capitals.
% \end{quote}
% However, the ActualText is ignored in Acrobat Reader
% (until version 9 at least), if the content does not contain glyphs.
% Option \xoption{space} adds such an invisible glyph, a space
% character. The font name can be configured by option \xoption{spacefont},
% the default font is \texttt{phvr8r}. The character slot is given
% by option \xoption{spacechar}, default is \texttt{32}, the usual
% position of the space character.
%
% These options can also be given as package options or in
% macro \cs{AccSuppSetup} that takes a key value list as argument.
% Usually only option \xoption{space} is necessary, if the
% contents does not contain glyphs otherwise. Option \xoption{space}
% is enabled by |space| or |space=true| and disabled by |space=false|.
% It is disabled as default. The option is evaluated by \cs{BeginAccSupp}
% and ignored by \cs{EndAccSupp}.
%
% Note: Even with option \xoption{space} I get sometimes wrong texts
% when cut \& paste from AR7/Linux or AR8/Linux, e.g.
% \begin{quote}
%   |Hello| $\rightarrow$ |Helo|, |Helol|, |Hell|, \dots\\
%   |Hello World| $\rightarrow$ |Helo WorldW|, |Helo World|, \dots
% \end{quote}
% I do not know what Acrobat Reader is doing here, thus feedback and
% insights are welcome.
%
% \subsection{Driver options}
%
% Driver options are package options only.
% The special \hologo{TeX} compilers \hologo{pdfTeX}, \hologo{LuaTeX} and \hologo{XeTeX}
% are detected automatically. The default for unrecognized drivers can be
% set by defining \cs{ActualTextDriverDefault}. This can be done
% in the configuration file \xfile{accsupp.cfg}.
%
% \subsubsection{Option \xoption{pdftex}}\label{sec:pdftex}
% Package option \xoption{pdftex} is used for \hologo{pdfTeX} in PDF mode.
% Additionally \cs{BeginAccSupp} and \cs{EndAccSupp} understand
% option \xoption{pdfliteral}. It controls
% the modifier keyword for \cs{pdfliteral}:
% \begin{quote}
%   |pdfliteral=direct| $\Rightarrow$ |\pdfliteral direct{...}|
% \end{quote}
%
% \subsubsection{Option \xoption{luatex}}\label{sec:luatex}
% Package option \xoption{luatex} is used for \hologo{LuaTeX} in PDF mode.
% Additionally \cs{BeginAccSupp} and \cs{EndAccSupp} understand
% option \xoption{pdfliteral}. It controls
% the modifier keyword for \cs{pdfextension} literal:
% \begin{quote}
%   |pdfliteral=direct| $\Rightarrow$ |\pdfextension literal direct{...}|
% \end{quote}
%
% \subsubsection{Option \xoption{dvipdfm}}
%
% Package option \xoption{dvipdfm} and its aliases \xoption{dvipdfmx}
% \xoption{xetex} are used for drivers that support
% \textsf{dvipdfm} specials.
%
% \subsubsection{Option \xoption{dvips}}
%
% Package option \xoption{dvips} and its alias \xoption{dvipsone}
% write pdfmark specials in the output. Unhappily these pdfmark
% operators are ignored by \textsf{ghostscript} (latest tested
% version is 8.54). Perhaps they are recognized by commercial
% distiller applications.
%
% \subsubsection{Turning off page stream compression}
%
% For debugging it is useful to have uncompressed page stream objects.
% This can be done afterwards via \textsf{pdftk}:
% \begin{quote}
%   |pdftk file.pdf output file-uncompressed.pdf uncompress|
% \end{quote}
% Or the PDF file is generated uncompressed:
% \begin{description}
% \item[\hologo{pdfTeX}:] |\pdfcompresslevel=0|
% \item[dvipdfm:] |dvipdfm -z0 ...|
% \item[dvipdfmx:] |dvipdfmx -z0 ...|
% \item[ghostscript/ps2pdf:]
%      |ps2pdf -dCompressPages=false input.ps output.pdf|
% \end{description}
%
% \section{Example}
% \label{sec:example}
%
% \subsection{Example \cs{notparallel}}
% \label{sec:example:notparallel}
%
%    \begin{macrocode}
%<*example1>
%    \end{macrocode}
%    \begin{macrocode}
%<<END
\documentclass{article}
\usepackage{accsupp}[2007/11/14]
\usepackage{centernot}
% U+2226 NOT PARALLEL
% \mathrel{...} prevents page break in between
\newcommand*{\notparallel}{%
  \ensuremath{%
    \mathrel{%
      \BeginAccSupp{method=hex,unicode,ActualText=2226}%
      \centernot{\parallel}%
      \EndAccSupp{}%
    }%
  }%
}
\begin{document}
\begin{equation}
A\notparallel B
\end{equation}
\end{document}
%END
%    \end{macrocode}
%    \begin{macrocode}
%</example1>
%    \end{macrocode}
%
% \subsection{Example with \xoption{pdfstringdef}}
%
%    \begin{macrocode}
%<*example2>
%    \end{macrocode}
%    \begin{macrocode}
%<<END
\documentclass{article}
\usepackage[unicode]{hyperref}
\usepackage{accsupp}[2007/11/14]
\begin{document}
  \begin{equation}
    \BeginAccSupp{
      method=pdfstringdef,
      unicode,
      ActualText={%
        a\texttwosuperior +b\texttwosuperior
        =c\texttwosuperior
      }
    }
    a^2 + b^2 = c^2
    \EndAccSupp{}
  \end{equation}
\end{document}
%END
%    \end{macrocode}
%    \begin{macrocode}
%</example2>
%    \end{macrocode}
%
% \StopEventually{
% }
%
% \section{Implementation}
%
% \subsection{Package}
%
%    \begin{macrocode}
%<*package>
%    \end{macrocode}
%
%    \begin{macrocode}
\begingroup\catcode61\catcode48\catcode32=10\relax%
  \catcode13=5 % ^^M
  \endlinechar=13 %
  \catcode123=1 % {
  \catcode125=2 % }
  \catcode64=11 % @
  \def\x{\endgroup
    \expandafter\edef\csname ACCSUPP@AtEnd\endcsname{%
      \endlinechar=\the\endlinechar\relax
      \catcode13=\the\catcode13\relax
      \catcode32=\the\catcode32\relax
      \catcode35=\the\catcode35\relax
      \catcode61=\the\catcode61\relax
      \catcode64=\the\catcode64\relax
      \catcode123=\the\catcode123\relax
      \catcode125=\the\catcode125\relax
    }%
  }%
\x\catcode61\catcode48\catcode32=10\relax%
\catcode13=5 % ^^M
\endlinechar=13 %
\catcode35=6 % #
\catcode64=11 % @
\catcode123=1 % {
\catcode125=2 % }
\def\TMP@EnsureCode#1#2{%
  \edef\ACCSUPP@AtEnd{%
    \ACCSUPP@AtEnd
    \catcode#1=\the\catcode#1\relax
  }%
  \catcode#1=#2\relax
}
\TMP@EnsureCode{10}{12}% ^^J
\TMP@EnsureCode{33}{12}% !
\TMP@EnsureCode{39}{12}% '
\TMP@EnsureCode{40}{12}% (
\TMP@EnsureCode{41}{12}% )
\TMP@EnsureCode{42}{12}% *
\TMP@EnsureCode{44}{12}% ,
\TMP@EnsureCode{45}{12}% -
\TMP@EnsureCode{46}{12}% .
\TMP@EnsureCode{47}{12}% /
\TMP@EnsureCode{58}{12}% :
\TMP@EnsureCode{60}{12}% <
\TMP@EnsureCode{62}{12}% >
\TMP@EnsureCode{94}{7}% ^ (superscript)
\TMP@EnsureCode{96}{12}% `
\TMP@EnsureCode{254}{12}% ^^fe
\TMP@EnsureCode{255}{12}% ^^ff
\edef\ACCSUPP@AtEnd{\ACCSUPP@AtEnd\noexpand\endinput}
%    \end{macrocode}
%
%    Package identification.
%    \begin{macrocode}
\NeedsTeXFormat{LaTeX2e}
\ProvidesPackage{accsupp}%
  [2018/01/30 v0.5 Accessibility support by marked content (HO)]
%    \end{macrocode}
%
%    \begin{macrocode}
\RequirePackage{pdfescape}[2007/02/25]
\RequirePackage{ifpdf}
\RequirePackage{ifxetex}
\RequirePackage{kvoptions}
%    \end{macrocode}
%    \begin{macrocode}
\SetupKeyvalOptions{%
  family=ACCSUPP,%
  prefix=ACCSUPP@%
}
%    \end{macrocode}
%
% \subsection{Space setup}
%
%    \begin{macrocode}
\DeclareBoolOption{space}
\DeclareStringOption[phvr8r]{spacefont}
\DeclareStringOption[32]{spacechar}
%    \end{macrocode}
%
% \subsection{Driver detection and setup}
%
% \paragraph{Driver declarations.}
%
%    \begin{macrocode}
\def\ACCSUPP@DefineDriverKey{%
  \@dblarg\ACCSUPP@@DefineDriverKey
}
\def\ACCSUPP@@DefineDriverKey[#1]#2{%
  \define@key{ACCSUPP}{#2}[]{%
    \def\ACCSUPP@driver{#1}%
  }%
  \g@addto@macro\ACCSUPP@DisableOptions{%
    \DisableKeyvalOption{ACCSUPP}{#2}%
  }%
}
\let\ACCSUPP@DisableOptions\@empty
\ACCSUPP@DefineDriverKey{pdftex}
\ACCSUPP@DefineDriverKey{luatex}
\ACCSUPP@DefineDriverKey{dvips}
\ACCSUPP@DefineDriverKey[dvips]{dvipsone}
\ACCSUPP@DefineDriverKey{dvipdfm}
\ACCSUPP@DefineDriverKey[dvipdfm]{dvipdfmx}
\ACCSUPP@DefineDriverKey[dvipdfm]{xetex}
\let\ACCSUPP@driver\relax
\InputIfFileExists{accsupp.cfg}{}{}
\providecommand*{\ActualTextDriverDefault}{dvips}
\ifpdf
  \ifx\pdfextension\@undefined
    \def\ACCSUPP@driver{pdftex}%
  \else
    \def\ACCSUPP@driver{luatex}%
  \fi
\else
  \ifxetex
    \def\ACCSUPP@driver{dvipdfm}%
  \else
    \ifx\ACCSUPP@driver\relax
      \let\ACCSUPP@driver\ActualTextDriverDefault
    \fi
  \fi
\fi
%    \end{macrocode}
%
% \paragraph{Process options.}
%
%    \begin{macrocode}
\ProcessKeyvalOptions*
\ACCSUPP@DisableOptions
%    \end{macrocode}
%
% \paragraph{Driver validation and loading.}
%
%    \begin{macrocode}
\def\ACCSUPP@temp{pdftex}%
\ifpdf
  \ifx\pdfextension\@undefined\else
    \def\ACCSUPP@temp{luatex}%
  \fi
  \ifx\ACCSUPP@temp\ACCSUPP@driver
  \else
    \PackageWarningNoLine{accsupp}{%
      Wrong driver `\ACCSUPP@driver', using `pdftex' instead%
    }%
ed as
    \let\ACCSUPP@driver\ACCSUPP@temp
  \fi
\else
  \ifx\ACCSUPP@temp\ACCSUPP@driver
    \PackageError{accsupp}{%
      Wrong driver, pdfTeX is not running in PDF mode.\MessageBreak
      Package loading is aborted%
    }\@ehc
    \expandafter\expandafter\expandafter\ACCSUPP@AtEnd
  \fi
  \def\ACCSUPP@temp{dvipdfm}%
  \ifxetex
    \ifx\ACCSUPP@temp\ACCSUPP@driver
    \else
      \PackageWarningNoLine{accsupp}{%
        Wrong driver `\ACCSUPP@driver',\MessageBreak
        using `dvipdfm' for XeTeX instead%
      }%
      \let\ACCSUPP@driver\ACCSUPP@temp
    \fi
  \fi
\fi%
\ifx\ACCSUPP@driver\relax
  \PackageError{accsupp}{%
    Missing driver option.\MessageBreak
    Package loading is aborted%
  }\@ehc
  \expandafter\ACCSUPP@AtEnd
\fi%
\InputIfFileExists{accsupp-\ACCSUPP@driver.def}{}{%
  \PackageError{accsupp}{%
    Unsupported driver `\ACCSUPP@driver'.\MessageBreak
    Package loading is aborted%
  }\@ehc
  \ACCSUPP@AtEnd
}%
%    \end{macrocode}
%
% \subsection{Main macro}
%
%    \begin{macrocode}
\DeclareBoolOption{unicode}
\DeclareStringOption[page]{pdfliteral}
\DeclareStringOption{Lang}
\def\ACCSUPP@method{escape}
\define@key{ACCSUPP}{method}{%
  \@ifundefined{ACCSUPP@method@#1}{%
    \PackageError{accsupp}{%
      Ignoring unknown method `#1'%
    }\@ehc
  }{%
    \edef\ACCSUPP@method{#1}%
  }%
}
\let\ACCSUPP@Lang\relax
\def\ACCSUPP@temp#1{%
  \expandafter\ACCSUPP@@temp\csname ACCSUPP@#1\endcsname{#1}%
}
\def\ACCSUPP@@temp#1#2{%
  \let#1\relax
  \define@key{ACCSUPP}{#2}{%
    \def#1{##1}%
    \ifx#1\@empty
      \def#1{()}%
    \else
      \csname ACCSUPP@method@\ACCSUPP@method\endcsname#1%
    \fi
  }%
}
\ACCSUPP@temp{Alt}
\ACCSUPP@temp{ActualText}
\ACCSUPP@temp{E}
%    \end{macrocode}
%
%    \begin{macrocode}
\newcommand*{\BeginAccSupp}[1]{%
  \begingroup
    \setkeys{ACCSUPP}{#1}%
    \edef\ACCSUPP@span{%
      /Span<<%
        \ifx\ACCSUPP@Lang\relax
        \else
          /Lang\ACCSUPP@Lang
        \fi
        \ifx\ACCSUPP@Alt\relax
        \else
          /Alt\ACCSUPP@Alt
        \fi
        \ifx\ACCSUPP@ActualText\relax
        \else
          /ActualText\ACCSUPP@ActualText
        \fi
        \ifx\ACCSUPP@E\relax
        \else
          /E\ACCSUPP@E
        \fi
      >>%
    }%
    \ACCSUPP@bdc
    \ACCSUPP@space
  \endgroup
}
\newcommand*{\EndAccSupp}[1]{%
  \begingroup
    \setkeys{ACCSUPP}{#1}%
    \ACCSUPP@emc
  \endgroup
}
\def\ACCSUPP@space{%
  \ifACCSUPP@space
    \begingroup
      \@ifundefined{ACCSUPP@Font}{%
        \global\font\ACCSUPP@Font=\ACCSUPP@spacefont\relax
      }{}%
      \leavevmode
      \setbox\z@\hbox{\ACCSUPP@Font\char\ACCSUPP@spacechar}%
      \wd\z@\z@
      \ht\z@\z@
      \dp\z@\z@
      \copy\z@
    \endgroup
  \fi
}
\newcommand*{\AccSuppSetup}{%
  \setkeys{ACCSUPP}%
}
%    \end{macrocode}
%
% \subsubsection{Input methods}
%
%    \begin{macrocode}
\def\ACCSUPP@method@plain#1{%
  \csname @safe@activestrue\endcsname
  \edef#1{%
    (%
    \ifACCSUPP@unicode
      \string\376\string\377%
    \fi
    #1%
    )%
  }%
  \@onelevel@sanitize#1%
}
%    \end{macrocode}
%    \begin{macrocode}
\def\ACCSUPP@method@escape#1{%
  \EdefEscapeString#1{%
    \ifACCSUPP@unicode
      ^^fe^^ff%
    \fi
    #1%
  }%
  \edef#1{(#1)}%
}%
%    \end{macrocode}
%    \begin{macrocode}
\def\ACCSUPP@method@hex#1{%
  \edef#1{%
    <%
    \ifACCSUPP@unicode
      FEFF%
    \fi
    #1%
    >%
  }%
}
%    \end{macrocode}
%    \begin{macrocode}
\def\ACCSUPP@method@pdfstringdef#1{%
  \ifACCSUPP@unicode
    \@ifundefined{hypersetup}{}{%
      \hypersetup{unicode}%
    }%
  \fi
  \@ifundefined{pdfstringdef}{%
    \PackageError{accsupp}{%
      Method `pdfstringdef' requires package `hyperref'%
    }\@ehc
    \let\ACCSUPP@temp\@empty
  }{%
    \begingroup
      \setbox0=\hbox{%
        \pdfstringdef\ACCSUPP@temp#1%
        \global\let\ACCSUPP@temp\ACCSUPP@temp
      }%
    \endgroup
  }%
  \edef#1{(\ACCSUPP@temp)}%
}
%    \end{macrocode}
%
%    \begin{macrocode}
\ACCSUPP@AtEnd%
%</package>
%    \end{macrocode}
%
% \subsection{Drivers}
%
% \subsubsection{Driver \xoption{pdftex}}
%
%    \begin{macrocode}
%<*pdftex>
\NeedsTeXFormat{LaTeX2e}
\ProvidesFile{accsupp-pdftex.def}%
  [2018/01/30 v0.5 accsupp driver for pdfTeX (HO)]%
%    \end{macrocode}
%
%    \begin{macrocode}
\def\ACCSUPP@bdc{%
  \pdfliteral\ACCSUPP@pdfliteral{\ACCSUPP@span BDC}%
}
\def\ACCSUPP@emc{%
  \pdfliteral\ACCSUPP@pdfliteral{EMC}%
}
%    \end{macrocode}
%    \begin{macrocode}
%</pdftex>
%    \end{macrocode}
%
% \subsubsection{Driver \xoption{luatex}}
%
%    \begin{macrocode}
%<*luatex>
\NeedsTeXFormat{LaTeX2e}
\ProvidesFile{accsupp-luatex.def}%
  [2018/01/30 v0.5 accsupp driver for pdfTeX (HO)]%
%    \end{macrocode}
%
%    \begin{macrocode}
\protected\def\ACCSUPP@pdf@literal{\pdfextension literal}
\def\ACCSUPP@bdc{%
  \ACCSUPP@pdf@literal\ACCSUPP@pdfliteral{\ACCSUPP@span BDC}%
}
\def\ACCSUPP@emc{%
  \ACCSUPP@pdf@literal\ACCSUPP@pdfliteral{EMC}%
}
%    \end{macrocode}
%    \begin{macrocode}
%</luatex>
%    \end{macrocode}
%
% \subsubsection{Driver \xoption{dvipdfm}}
%
%    \begin{macrocode}
%<*dvipdfm>
\NeedsTeXFormat{LaTeX2e}
\ProvidesFile{accsupp-dvipdfm.def}%
  [2018/01/30 v0.5 accsupp driver for dvipdfm (HO)]%
%    \end{macrocode}
%
%    \begin{macrocode}
\def\ACCSUPP@bdc{%
  \special{pdf:content \ACCSUPP@span BDC}%
}
\def\ACCSUPP@emc{%
  \special{pdf:content EMC}%
}
%    \end{macrocode}
%    \begin{macrocode}
%</dvipdfm>
%    \end{macrocode}
%
% \subsubsection{Driver \xoption{dvips}}
%
%    \begin{macrocode}
%<*dvips>
\NeedsTeXFormat{LaTeX2e}
\ProvidesFile{accsupp-dvips.def}%
  [2018/01/30 v0.5 accsupp driver for dvips (HO)]%
%    \end{macrocode}
%
%    \begin{macrocode}
\def\ACCSUPP@bdc{%
  \special{ps:[\ACCSUPP@span/BDC pdfmark}%
}
\def\ACCSUPP@emc{%
  \special{ps:[/EMC pdfmark}%
}
%    \end{macrocode}
%    \begin{macrocode}
%</dvips>
%    \end{macrocode}
%
% \section{Test}
%
% \subsection{Catcode checks for loading}
%
%    \begin{macrocode}
%<*test1>
%    \end{macrocode}
%    \begin{macrocode}
\NeedsTeXFormat{LaTeX2e}
\documentclass{minimal}
\makeatletter
\def\RestoreCatcodes{}
\count@=0 %
\loop
  \edef\RestoreCatcodes{%
    \RestoreCatcodes
    \catcode\the\count@=\the\catcode\count@\relax
  }%
\ifnum\count@<255 %
  \advance\count@\@ne
\repeat

\def\RangeCatcodeInvalid#1#2{%
  \count@=#1\relax
  \loop
    \catcode\count@=15 %
  \ifnum\count@<#2\relax
    \advance\count@\@ne
  \repeat
}
\def\Test{%
  \RangeCatcodeInvalid{0}{47}%
  \RangeCatcodeInvalid{58}{64}%
  \RangeCatcodeInvalid{91}{96}%
  \RangeCatcodeInvalid{123}{127}%
  \catcode`\@=12 %
  \catcode`\\=0 %
  \catcode`\{=1 %
  \catcode`\}=2 %
  \catcode`\#=6 %
  \catcode`\[=12 %
  \catcode`\]=12 %
  \catcode`\%=14 %
  \catcode`\ =10 %
  \catcode13=5 %
  \RequirePackage{accsupp}[2018/01/30]\relax
  \RestoreCatcodes
}
\Test
\csname @@end\endcsname
\end
%    \end{macrocode}
%    \begin{macrocode}
%</test1>
%    \end{macrocode}
%
% \section{Installation}
%
% \subsection{Download}
%
% \paragraph{Package.} This package is available on
% CTAN\footnote{\url{http://ctan.org/pkg/accsupp}}:
% \begin{description}
% \item[\CTAN{macros/latex/contrib/oberdiek/accsupp.dtx}] The source file.
% \item[\CTAN{macros/latex/contrib/oberdiek/accsupp.pdf}] Documentation.
% \end{description}
%
%
% \paragraph{Bundle.} All the packages of the bundle `oberdiek'
% are also available in a TDS compliant ZIP archive. There
% the packages are already unpacked and the documentation files
% are generated. The files and directories obey the TDS standard.
% \begin{description}
% \item[\CTAN{install/macros/latex/contrib/oberdiek.tds.zip}]
% \end{description}
% \emph{TDS} refers to the standard ``A Directory Structure
% for \TeX\ Files'' (\CTAN{tds/tds.pdf}). Directories
% with \xfile{texmf} in their name are usually organized this way.
%
% \subsection{Bundle installation}
%
% \paragraph{Unpacking.} Unpack the \xfile{oberdiek.tds.zip} in the
% TDS tree (also known as \xfile{texmf} tree) of your choice.
% Example (linux):
% \begin{quote}
%   |unzip oberdiek.tds.zip -d ~/texmf|
% \end{quote}
%
% \paragraph{Script installation.}
% Check the directory \xfile{TDS:scripts/oberdiek/} for
% scripts that need further installation steps.
% Package \xpackage{attachfile2} comes with the Perl script
% \xfile{pdfatfi.pl} that should be installed in such a way
% that it can be called as \texttt{pdfatfi}.
% Example (linux):
% \begin{quote}
%   |chmod +x scripts/oberdiek/pdfatfi.pl|\\
%   |cp scripts/oberdiek/pdfatfi.pl /usr/local/bin/|
% \end{quote}
%
% \subsection{Package installation}
%
% \paragraph{Unpacking.} The \xfile{.dtx} file is a self-extracting
% \docstrip\ archive. The files are extracted by running the
% \xfile{.dtx} through \plainTeX:
% \begin{quote}
%   \verb|tex accsupp.dtx|
% \end{quote}
%
% \paragraph{TDS.} Now the different files must be moved into
% the different directories in your installation TDS tree
% (also known as \xfile{texmf} tree):
% \begin{quote}
% \def\t{^^A
% \begin{tabular}{@{}>{\ttfamily}l@{ $\rightarrow$ }>{\ttfamily}l@{}}
%   accsupp.sty & tex/latex/oberdiek/accsupp.sty\\
%   accsupp-pdftex.def & tex/latex/oberdiek/accsupp-pdftex.def\\
%   accsupp-luatex.def & tex/latex/oberdiek/accsupp-luatex.def\\
%   accsupp-dvipdfm.def & tex/latex/oberdiek/accsupp-dvipdfm.def\\
%   accsupp-dvips.def & tex/latex/oberdiek/accsupp-dvips.def\\
%   accsupp.pdf & doc/latex/oberdiek/accsupp.pdf\\
%   accsupp-example1.tex & doc/latex/oberdiek/accsupp-example1.tex\\
%   accsupp-example2.tex & doc/latex/oberdiek/accsupp-example2.tex\\
%   test/accsupp-test1.tex & doc/latex/oberdiek/test/accsupp-test1.tex\\
%   accsupp.dtx & source/latex/oberdiek/accsupp.dtx\\
% \end{tabular}^^A
% }^^A
% \sbox0{\t}^^A
% \ifdim\wd0>\linewidth
%   \begingroup
%     \advance\linewidth by\leftmargin
%     \advance\linewidth by\rightmargin
%   \edef\x{\endgroup
%     \def\noexpand\lw{\the\linewidth}^^A
%   }\x
%   \def\lwbox{^^A
%     \leavevmode
%     \hbox to \linewidth{^^A
%       \kern-\leftmargin\relax
%       \hss
%       \usebox0
%       \hss
%       \kern-\rightmargin\relax
%     }^^A
%   }^^A
%   \ifdim\wd0>\lw
%     \sbox0{\small\t}^^A
%     \ifdim\wd0>\linewidth
%       \ifdim\wd0>\lw
%         \sbox0{\footnotesize\t}^^A
%         \ifdim\wd0>\linewidth
%           \ifdim\wd0>\lw
%             \sbox0{\scriptsize\t}^^A
%             \ifdim\wd0>\linewidth
%               \ifdim\wd0>\lw
%                 \sbox0{\tiny\t}^^A
%                 \ifdim\wd0>\linewidth
%                   \lwbox
%                 \else
%                   \usebox0
%                 \fi
%               \else
%                 \lwbox
%               \fi
%             \else
%               \usebox0
%             \fi
%           \else
%             \lwbox
%           \fi
%         \else
%           \usebox0
%         \fi
%       \else
%         \lwbox
%       \fi
%     \else
%       \usebox0
%     \fi
%   \else
%     \lwbox
%   \fi
% \else
%   \usebox0
% \fi
% \end{quote}
% If you have a \xfile{docstrip.cfg} that configures and enables \docstrip's
% TDS installing feature, then some files can already be in the right
% place, see the documentation of \docstrip.
%
% \subsection{Refresh file name databases}
%
% If your \TeX~distribution
% (\teTeX, \mikTeX, \dots) relies on file name databases, you must refresh
% these. For example, \teTeX\ users run \verb|texhash| or
% \verb|mktexlsr|.
%
% \subsection{Some details for the interested}
%
% \paragraph{Attached source.}
%
% The PDF documentation on CTAN also includes the
% \xfile{.dtx} source file. It can be extracted by
% AcrobatReader 6 or higher. Another option is \textsf{pdftk},
% e.g. unpack the file into the current directory:
% \begin{quote}
%   \verb|pdftk accsupp.pdf unpack_files output .|
% \end{quote}
%
% \paragraph{Unpacking with \LaTeX.}
% The \xfile{.dtx} chooses its action depending on the format:
% \begin{description}
% \item[\plainTeX:] Run \docstrip\ and extract the files.
% \item[\LaTeX:] Generate the documentation.
% \end{description}
% If you insist on using \LaTeX\ for \docstrip\ (really,
% \docstrip\ does not need \LaTeX), then inform the autodetect routine
% about your intention:
% \begin{quote}
%   \verb|latex \let\install=y\input{accsupp.dtx}|
% \end{quote}
% Do not forget to quote the argument according to the demands
% of your shell.
%
% \paragraph{Generating the documentation.}
% You can use both the \xfile{.dtx} or the \xfile{.drv} to generate
% the documentation. The process can be configured by the
% configuration file \xfile{ltxdoc.cfg}. For instance, put this
% line into this file, if you want to have A4 as paper format:
% \begin{quote}
%   \verb|\PassOptionsToClass{a4paper}{article}|
% \end{quote}
% An example follows how to generate the
% documentation with pdf\LaTeX:
% \begin{quote}
%\begin{verbatim}
%pdflatex accsupp.dtx
%makeindex -s gind.ist accsupp.idx
%pdflatex accsupp.dtx
%makeindex -s gind.ist accsupp.idx
%pdflatex accsupp.dtx
%\end{verbatim}
% \end{quote}
%
% \section{Catalogue}
%
% The following XML file can be used as source for the
% \href{http://mirror.ctan.org/help/Catalogue/catalogue.html}{\TeX\ Catalogue}.
% The elements \texttt{caption} and \texttt{description} are imported
% from the original XML file from the Catalogue.
% The name of the XML file in the Catalogue is \xfile{accsupp.xml}.
%    \begin{macrocode}
%<*catalogue>
<?xml version='1.0' encoding='us-ascii'?>
<!DOCTYPE entry SYSTEM 'catalogue.dtd'>
<entry datestamp='$Date$' modifier='$Author$' id='accsupp'>
  <name>accsupp</name>
  <caption>Better accessibility support for PDF files.</caption>
  <authorref id='auth:oberdiek'/>
  <copyright owner='Heiko Oberdiek' year='2007,2010'/>
  <license type='lppl1.3'/>
  <version number='0.4'/>
  <description>
    Since PDF 1.5 portions of a page can be marked for better accessibility
    support. For example, replacement texts or expansions of abbreviations can
    be provided. This package starts with providing a minimal low-level
    interface for programmers; its status is experimental.
    <p/>
    The package is part of the <xref refid='oberdiek'>oberdiek</xref> bundle.
  </description>
  <documentation details='Package documentation'
      href='ctan:/macros/latex/contrib/oberdiek/accsupp.pdf'/>
  <ctan file='true' path='/macros/latex/contrib/oberdiek/accsupp.dtx'/>
  <miktex location='oberdiek'/>
  <texlive location='oberdiek'/>
  <install path='/macros/latex/contrib/oberdiek/oberdiek.tds.zip'/>
</entry>
%</catalogue>
%    \end{macrocode}
%
% \begin{thebibliography}{9}
% \bibitem{pdfspec}
% Adobe Systems Incorporated,
% \textit{PDF Reference}, 6th edition, 2006.
% \url{http://www.adobe.com/devnet/acrobat/pdfs/pdf_reference.pdf}
%
% \end{thebibliography}
%
% \begin{History}
%   \begin{Version}{2007/03/21 v0.1}
%   \item
%     First version.
%   \end{Version}
%   \begin{Version}{2007/11/14 v0.2}
%   \item
%     Various bug fixes.
%   \item
%     Catcode section rewritten, test added.
%   \end{Version}
%   \begin{Version}{2010/01/16 v0.3}
%   \item
%     \cs{AccSuppSetup} added.
%   \item
%     Options \xoption{space}, \xoption{spacefont}, \xoption{spacechar} added.
%   \end{Version}
%   \begin{Version}{2016/05/16 v0.4}
%   \item
%     Documentation updates.
%   \end{Version}
%   \begin{Version}{2018/01/30 v0.5}
%   \item
%     LuaTeX support added.
%   \end{Version}
% \end{History}
%
% \PrintIndex
%
% \Finale
\endinput

%        (quote the arguments according to the demands of your shell)
%
% Documentation:
%    (a) If accsupp.drv is present:
%           latex accsupp.drv
%    (b) Without accsupp.drv:
%           latex accsupp.dtx; ...
%    The class ltxdoc loads the configuration file ltxdoc.cfg
%    if available. Here you can specify further options, e.g.
%    use A4 as paper format:
%       \PassOptionsToClass{a4paper}{article}
%
%    Programm calls to get the documentation (example):
%       pdflatex accsupp.dtx
%       makeindex -s gind.ist accsupp.idx
%       pdflatex accsupp.dtx
%       makeindex -s gind.ist accsupp.idx
%       pdflatex accsupp.dtx
%
% Installation:
%    TDS:tex/latex/oberdiek/accsupp.sty
%    TDS:tex/latex/oberdiek/accsupp-pdftex.def
%    TDS:tex/latex/oberdiek/accsupp-luatex.def
%    TDS:tex/latex/oberdiek/accsupp-dvipdfm.def
%    TDS:tex/latex/oberdiek/accsupp-dvips.def
%    TDS:doc/latex/oberdiek/accsupp.pdf
%    TDS:doc/latex/oberdiek/accsupp-example1.tex
%    TDS:doc/latex/oberdiek/accsupp-example2.tex
%    TDS:doc/latex/oberdiek/test/accsupp-test1.tex
%    TDS:source/latex/oberdiek/accsupp.dtx
%
%<*ignore>
\begingroup
  \catcode123=1 %
  \catcode125=2 %
  \def\x{LaTeX2e}%
\expandafter\endgroup
\ifcase 0\ifx\install y1\fi\expandafter
         \ifx\csname processbatchFile\endcsname\relax\else1\fi
         \ifx\fmtname\x\else 1\fi\relax
\else\csname fi\endcsname
%</ignore>
%<*install>
\input docstrip.tex
\Msg{************************************************************************}
\Msg{* Installation}
\Msg{* Package: accsupp 2018/01/30 v0.5 Accessibility support by marked content (HO)}
\Msg{************************************************************************}

\keepsilent
\askforoverwritefalse

\let\MetaPrefix\relax
\preamble

This is a generated file.

Project: accsupp
Version: 2018/01/30 v0.5

Copyright (C) 2007, 2010 by
   Heiko Oberdiek <heiko.oberdiek at googlemail.com>

This work may be distributed and/or modified under the
conditions of the LaTeX Project Public License, either
version 1.3c of this license or (at your option) any later
version. This version of this license is in
   http://www.latex-project.org/lppl/lppl-1-3c.txt
and the latest version of this license is in
   http://www.latex-project.org/lppl.txt
and version 1.3 or later is part of all distributions of
LaTeX version 2005/12/01 or later.

This work has the LPPL maintenance status "maintained".

This Current Maintainer of this work is Heiko Oberdiek.

This work consists of the main source file accsupp.dtx
and the derived files
   accsupp.sty, accsupp.pdf, accsupp.ins, accsupp.drv,
   accsupp-pdftex.def, accsupp-luatex.def, accsupp-dvipdfm.def,
   accsupp-dvips.def,
   accsupp-example1.tex, accsupp-example2.tex,
   accsupp-test1.tex.

\endpreamble
\let\MetaPrefix\DoubleperCent

\generate{%
  \file{accsupp.ins}{\from{accsupp.dtx}{install}}%
  \file{accsupp.drv}{\from{accsupp.dtx}{driver}}%
  \usedir{tex/latex/oberdiek}%
  \file{accsupp.sty}{\from{accsupp.dtx}{package}}%
  \file{accsupp-pdftex.def}{\from{accsupp.dtx}{pdftex}}%
  \file{accsupp-luatex.def}{\from{accsupp.dtx}{luatex}}%
  \file{accsupp-dvipdfm.def}{\from{accsupp.dtx}{dvipdfm}}%
  \file{accsupp-dvips.def}{\from{accsupp.dtx}{dvips}}%
  \usedir{doc/latex/oberdiek}%
  \file{accsupp-example1.tex}{\from{accsupp.dtx}{example1}}%
  \file{accsupp-example2.tex}{\from{accsupp.dtx}{example2}}%
  \usedir{doc/latex/oberdiek/test}%
  \file{accsupp-test1.tex}{\from{accsupp.dtx}{test1}}%
  \nopreamble
  \nopostamble
  \usedir{source/latex/oberdiek/catalogue}%
  \file{accsupp.xml}{\from{accsupp.dtx}{catalogue}}%
}

\catcode32=13\relax% active space
\let =\space%
\Msg{************************************************************************}
\Msg{*}
\Msg{* To finish the installation you have to move the following}
\Msg{* files into a directory searched by TeX:}
\Msg{*}
\Msg{*     accsupp.sty, accsupp-pdftex.def, accsupp-dvipdfm.def,}
\Msg{*     accsupp-dvips.def}
\Msg{*}
\Msg{* To produce the documentation run the file `accsupp.drv'}
\Msg{* through LaTeX.}
\Msg{*}
\Msg{* Happy TeXing!}
\Msg{*}
\Msg{************************************************************************}

\endbatchfile
%</install>
%<*ignore>
\fi
%</ignore>
%<*driver>
\NeedsTeXFormat{LaTeX2e}
\ProvidesFile{accsupp.drv}%
  [2018/01/30 v0.5 Accessibility support by marked content (HO)]%
\documentclass{ltxdoc}
\usepackage{holtxdoc}[2011/11/22]
\begin{document}
  \DocInput{accsupp.dtx}%
\end{document}
%</driver>
% \fi
%
%
% \CharacterTable
%  {Upper-case    \A\B\C\D\E\F\G\H\I\J\K\L\M\N\O\P\Q\R\S\T\U\V\W\X\Y\Z
%   Lower-case    \a\b\c\d\e\f\g\h\i\j\k\l\m\n\o\p\q\r\s\t\u\v\w\x\y\z
%   Digits        \0\1\2\3\4\5\6\7\8\9
%   Exclamation   \!     Double quote  \"     Hash (number) \#
%   Dollar        \$     Percent       \%     Ampersand     \&
%   Acute accent  \'     Left paren    \(     Right paren   \)
%   Asterisk      \*     Plus          \+     Comma         \,
%   Minus         \-     Point         \.     Solidus       \/
%   Colon         \:     Semicolon     \;     Less than     \<
%   Equals        \=     Greater than  \>     Question mark \?
%   Commercial at \@     Left bracket  \[     Backslash     \\
%   Right bracket \]     Circumflex    \^     Underscore    \_
%   Grave accent  \`     Left brace    \{     Vertical bar  \|
%   Right brace   \}     Tilde         \~}
%
% \GetFileInfo{accsupp.drv}
%
% \title{The \xpackage{accsupp} package}
% \date{2018/01/30 v0.5}
% \author{Heiko Oberdiek\thanks
% {Please report any issues at https://github.com/ho-tex/oberdiek/issues}\\
% \xemail{heiko.oberdiek at googlemail.com}}
%
% \maketitle
%
% \begin{abstract}
% Since PDF 1.5 portions of a page can be marked for better
% accessibility support.
% For example, replacement texts or expansions of abbreviations can be
% provided. Package \xpackage{accsupp} starts with providing a minimal
% low-level interface for programmers. Status is experimental.
% \end{abstract}
%
% \tableofcontents
%
% \section{Documentation}
%
% \subsection{Macros}
%
% Section ``10.8 Accessibility Support'' of the PDF reference
% \cite{pdfspec} lists some features that can be added by
% operators for marked content.
%
% \begin{declcs}{BeginAccSupp} \M{options}
% \end{declcs}
% It puts the operator |BDC| in the page stream:
% \begin{quote}
%   |/Span|\\
%   |<<|\dots|>>| \qquad \textit{\% property dictionary}\\
%   |BDC|
% \end{quote}
% The contents of the dictionary is controlled by \meta{options}.
% See sections \ref{sec:feature} and \ref{sec:method}.
%
% \begin{declcs}{EndAccSupp} \M{options}
% \end{declcs}
% It puts the operator |EMC| in the page stream.
% The only option is \xoption{pdfliteral}, see section \ref{sec:pdftex}.
%
% \bigskip
% \noindent\textbf{Note:}
% The caller is responsible for the placement of \cs{BeginAccSupp}
% and \cs{EndAccSupp} pairs. Especially page breaks are not allowed
% in between.
%
% \subsubsection{Feature options}\label{sec:feature}
%
% The PDF reference \cite{pdfspec} describes and explains the different
% features. The names of the feature options are the same as the
% key names for the property dictionary for operator |BDC|, see
% \cs{BeginAccSupp}.
%
% \begin{description}
% \item[\xoption{ActualText}:]
%   Provides a replacement text, see examples in
%   section \ref{sec:example}.
% \item[\xoption{Alt}:]
%   Provides an alternate description.
% \item[\xoption{E}:]
%   Provides the expansion of an abbreviation or an acronym.
% \item[\xoption{Lang}:]
%   Specifies the language.
% \end{description}
%
% \subsubsection{Input methods}\label{sec:method}
%
% Except for \xoption{Lang} option \xoption{method} controls
% how the argument for \xoption{ActualText}, \xoption{Alt},
% and \xoption{E} are interpreted.
% \begin{description}
% \item[\xoption{method=plain}:]
%   The string is only expanded and written without further
%   treatment. Special characters are not protected, thus this method
%   may result in an invalid PDF file.
% \item[\xoption{mehod=escape}:]
%   The string is expanded and special characters are escaped.
%   The result is a valid PDF string.
% \item[\xoption{method=hex}:]
%   The string is given in hexadecimal notation.
%   Section \ref{sec:example:notparallel} shows an example.
% \item[\xoption{method=pdfstringdef}:]
%   If package \xpackage{hyperref} is loaded, then its \cs{pdfstringdef}
%   is used. This method is slow, but useful if the string contains
%   arbitrary \hologo{TeX} code.
% \item[\xoption{unicode}:]
%   This option is needed, if the string is given as Unicode string
%   (16 bit). Internally it adds the string prefix for Unicode.
%   In case of \xoption{method=pdfstringdef} it passes the option
%   to \cs{hypersetup}.
% \end{description}
%
% \subsection{Workaround, option \xoption{space}}
%
% PDF specification says in ``10.8.3 Replacement Text''
% \begin{quote}
% Just as alternative descriptions can be provided for images and other
% items that do not translate naturally into text (\dots), replacement
% text can be specified for content that does translate into text but
% that is represented in a nonstandard way. These nonstandard representations
% might include, for example, glyphs for ligatures or custom characters, or
% inline graphics corresponding to letters in an illuminated manustript or to
% dropped capitals.
% \end{quote}
% However, the ActualText is ignored in Acrobat Reader
% (until version 9 at least), if the content does not contain glyphs.
% Option \xoption{space} adds such an invisible glyph, a space
% character. The font name can be configured by option \xoption{spacefont},
% the default font is \texttt{phvr8r}. The character slot is given
% by option \xoption{spacechar}, default is \texttt{32}, the usual
% position of the space character.
%
% These options can also be given as package options or in
% macro \cs{AccSuppSetup} that takes a key value list as argument.
% Usually only option \xoption{space} is necessary, if the
% contents does not contain glyphs otherwise. Option \xoption{space}
% is enabled by |space| or |space=true| and disabled by |space=false|.
% It is disabled as default. The option is evaluated by \cs{BeginAccSupp}
% and ignored by \cs{EndAccSupp}.
%
% Note: Even with option \xoption{space} I get sometimes wrong texts
% when cut \& paste from AR7/Linux or AR8/Linux, e.g.
% \begin{quote}
%   |Hello| $\rightarrow$ |Helo|, |Helol|, |Hell|, \dots\\
%   |Hello World| $\rightarrow$ |Helo WorldW|, |Helo World|, \dots
% \end{quote}
% I do not know what Acrobat Reader is doing here, thus feedback and
% insights are welcome.
%
% \subsection{Driver options}
%
% Driver options are package options only.
% The special \hologo{TeX} compilers \hologo{pdfTeX}, \hologo{LuaTeX} and \hologo{XeTeX}
% are detected automatically. The default for unrecognized drivers can be
% set by defining \cs{ActualTextDriverDefault}. This can be done
% in the configuration file \xfile{accsupp.cfg}.
%
% \subsubsection{Option \xoption{pdftex}}\label{sec:pdftex}
% Package option \xoption{pdftex} is used for \hologo{pdfTeX} in PDF mode.
% Additionally \cs{BeginAccSupp} and \cs{EndAccSupp} understand
% option \xoption{pdfliteral}. It controls
% the modifier keyword for \cs{pdfliteral}:
% \begin{quote}
%   |pdfliteral=direct| $\Rightarrow$ |\pdfliteral direct{...}|
% \end{quote}
%
% \subsubsection{Option \xoption{luatex}}\label{sec:luatex}
% Package option \xoption{luatex} is used for \hologo{LuaTeX} in PDF mode.
% Additionally \cs{BeginAccSupp} and \cs{EndAccSupp} understand
% option \xoption{pdfliteral}. It controls
% the modifier keyword for \cs{pdfextension} literal:
% \begin{quote}
%   |pdfliteral=direct| $\Rightarrow$ |\pdfextension literal direct{...}|
% \end{quote}
%
% \subsubsection{Option \xoption{dvipdfm}}
%
% Package option \xoption{dvipdfm} and its aliases \xoption{dvipdfmx}
% \xoption{xetex} are used for drivers that support
% \textsf{dvipdfm} specials.
%
% \subsubsection{Option \xoption{dvips}}
%
% Package option \xoption{dvips} and its alias \xoption{dvipsone}
% write pdfmark specials in the output. Unhappily these pdfmark
% operators are ignored by \textsf{ghostscript} (latest tested
% version is 8.54). Perhaps they are recognized by commercial
% distiller applications.
%
% \subsubsection{Turning off page stream compression}
%
% For debugging it is useful to have uncompressed page stream objects.
% This can be done afterwards via \textsf{pdftk}:
% \begin{quote}
%   |pdftk file.pdf output file-uncompressed.pdf uncompress|
% \end{quote}
% Or the PDF file is generated uncompressed:
% \begin{description}
% \item[\hologo{pdfTeX}:] |\pdfcompresslevel=0|
% \item[dvipdfm:] |dvipdfm -z0 ...|
% \item[dvipdfmx:] |dvipdfmx -z0 ...|
% \item[ghostscript/ps2pdf:]
%      |ps2pdf -dCompressPages=false input.ps output.pdf|
% \end{description}
%
% \section{Example}
% \label{sec:example}
%
% \subsection{Example \cs{notparallel}}
% \label{sec:example:notparallel}
%
%    \begin{macrocode}
%<*example1>
%    \end{macrocode}
%    \begin{macrocode}
%<<END
\documentclass{article}
\usepackage{accsupp}[2007/11/14]
\usepackage{centernot}
% U+2226 NOT PARALLEL
% \mathrel{...} prevents page break in between
\newcommand*{\notparallel}{%
  \ensuremath{%
    \mathrel{%
      \BeginAccSupp{method=hex,unicode,ActualText=2226}%
      \centernot{\parallel}%
      \EndAccSupp{}%
    }%
  }%
}
\begin{document}
\begin{equation}
A\notparallel B
\end{equation}
\end{document}
%END
%    \end{macrocode}
%    \begin{macrocode}
%</example1>
%    \end{macrocode}
%
% \subsection{Example with \xoption{pdfstringdef}}
%
%    \begin{macrocode}
%<*example2>
%    \end{macrocode}
%    \begin{macrocode}
%<<END
\documentclass{article}
\usepackage[unicode]{hyperref}
\usepackage{accsupp}[2007/11/14]
\begin{document}
  \begin{equation}
    \BeginAccSupp{
      method=pdfstringdef,
      unicode,
      ActualText={%
        a\texttwosuperior +b\texttwosuperior
        =c\texttwosuperior
      }
    }
    a^2 + b^2 = c^2
    \EndAccSupp{}
  \end{equation}
\end{document}
%END
%    \end{macrocode}
%    \begin{macrocode}
%</example2>
%    \end{macrocode}
%
% \StopEventually{
% }
%
% \section{Implementation}
%
% \subsection{Package}
%
%    \begin{macrocode}
%<*package>
%    \end{macrocode}
%
%    \begin{macrocode}
\begingroup\catcode61\catcode48\catcode32=10\relax%
  \catcode13=5 % ^^M
  \endlinechar=13 %
  \catcode123=1 % {
  \catcode125=2 % }
  \catcode64=11 % @
  \def\x{\endgroup
    \expandafter\edef\csname ACCSUPP@AtEnd\endcsname{%
      \endlinechar=\the\endlinechar\relax
      \catcode13=\the\catcode13\relax
      \catcode32=\the\catcode32\relax
      \catcode35=\the\catcode35\relax
      \catcode61=\the\catcode61\relax
      \catcode64=\the\catcode64\relax
      \catcode123=\the\catcode123\relax
      \catcode125=\the\catcode125\relax
    }%
  }%
\x\catcode61\catcode48\catcode32=10\relax%
\catcode13=5 % ^^M
\endlinechar=13 %
\catcode35=6 % #
\catcode64=11 % @
\catcode123=1 % {
\catcode125=2 % }
\def\TMP@EnsureCode#1#2{%
  \edef\ACCSUPP@AtEnd{%
    \ACCSUPP@AtEnd
    \catcode#1=\the\catcode#1\relax
  }%
  \catcode#1=#2\relax
}
\TMP@EnsureCode{10}{12}% ^^J
\TMP@EnsureCode{33}{12}% !
\TMP@EnsureCode{39}{12}% '
\TMP@EnsureCode{40}{12}% (
\TMP@EnsureCode{41}{12}% )
\TMP@EnsureCode{42}{12}% *
\TMP@EnsureCode{44}{12}% ,
\TMP@EnsureCode{45}{12}% -
\TMP@EnsureCode{46}{12}% .
\TMP@EnsureCode{47}{12}% /
\TMP@EnsureCode{58}{12}% :
\TMP@EnsureCode{60}{12}% <
\TMP@EnsureCode{62}{12}% >
\TMP@EnsureCode{94}{7}% ^ (superscript)
\TMP@EnsureCode{96}{12}% `
\TMP@EnsureCode{254}{12}% ^^fe
\TMP@EnsureCode{255}{12}% ^^ff
\edef\ACCSUPP@AtEnd{\ACCSUPP@AtEnd\noexpand\endinput}
%    \end{macrocode}
%
%    Package identification.
%    \begin{macrocode}
\NeedsTeXFormat{LaTeX2e}
\ProvidesPackage{accsupp}%
  [2018/01/30 v0.5 Accessibility support by marked content (HO)]
%    \end{macrocode}
%
%    \begin{macrocode}
\RequirePackage{pdfescape}[2007/02/25]
\RequirePackage{ifpdf}
\RequirePackage{ifxetex}
\RequirePackage{kvoptions}
%    \end{macrocode}
%    \begin{macrocode}
\SetupKeyvalOptions{%
  family=ACCSUPP,%
  prefix=ACCSUPP@%
}
%    \end{macrocode}
%
% \subsection{Space setup}
%
%    \begin{macrocode}
\DeclareBoolOption{space}
\DeclareStringOption[phvr8r]{spacefont}
\DeclareStringOption[32]{spacechar}
%    \end{macrocode}
%
% \subsection{Driver detection and setup}
%
% \paragraph{Driver declarations.}
%
%    \begin{macrocode}
\def\ACCSUPP@DefineDriverKey{%
  \@dblarg\ACCSUPP@@DefineDriverKey
}
\def\ACCSUPP@@DefineDriverKey[#1]#2{%
  \define@key{ACCSUPP}{#2}[]{%
    \def\ACCSUPP@driver{#1}%
  }%
  \g@addto@macro\ACCSUPP@DisableOptions{%
    \DisableKeyvalOption{ACCSUPP}{#2}%
  }%
}
\let\ACCSUPP@DisableOptions\@empty
\ACCSUPP@DefineDriverKey{pdftex}
\ACCSUPP@DefineDriverKey{luatex}
\ACCSUPP@DefineDriverKey{dvips}
\ACCSUPP@DefineDriverKey[dvips]{dvipsone}
\ACCSUPP@DefineDriverKey{dvipdfm}
\ACCSUPP@DefineDriverKey[dvipdfm]{dvipdfmx}
\ACCSUPP@DefineDriverKey[dvipdfm]{xetex}
\let\ACCSUPP@driver\relax
\InputIfFileExists{accsupp.cfg}{}{}
\providecommand*{\ActualTextDriverDefault}{dvips}
\ifpdf
  \ifx\pdfextension\@undefined
    \def\ACCSUPP@driver{pdftex}%
  \else
    \def\ACCSUPP@driver{luatex}%
  \fi
\else
  \ifxetex
    \def\ACCSUPP@driver{dvipdfm}%
  \else
    \ifx\ACCSUPP@driver\relax
      \let\ACCSUPP@driver\ActualTextDriverDefault
    \fi
  \fi
\fi
%    \end{macrocode}
%
% \paragraph{Process options.}
%
%    \begin{macrocode}
\ProcessKeyvalOptions*
\ACCSUPP@DisableOptions
%    \end{macrocode}
%
% \paragraph{Driver validation and loading.}
%
%    \begin{macrocode}
\def\ACCSUPP@temp{pdftex}%
\ifpdf
  \ifx\pdfextension\@undefined\else
    \def\ACCSUPP@temp{luatex}%
  \fi
  \ifx\ACCSUPP@temp\ACCSUPP@driver
  \else
    \PackageWarningNoLine{accsupp}{%
      Wrong driver `\ACCSUPP@driver', using `pdftex' instead%
    }%
ed as
    \let\ACCSUPP@driver\ACCSUPP@temp
  \fi
\else
  \ifx\ACCSUPP@temp\ACCSUPP@driver
    \PackageError{accsupp}{%
      Wrong driver, pdfTeX is not running in PDF mode.\MessageBreak
      Package loading is aborted%
    }\@ehc
    \expandafter\expandafter\expandafter\ACCSUPP@AtEnd
  \fi
  \def\ACCSUPP@temp{dvipdfm}%
  \ifxetex
    \ifx\ACCSUPP@temp\ACCSUPP@driver
    \else
      \PackageWarningNoLine{accsupp}{%
        Wrong driver `\ACCSUPP@driver',\MessageBreak
        using `dvipdfm' for XeTeX instead%
      }%
      \let\ACCSUPP@driver\ACCSUPP@temp
    \fi
  \fi
\fi%
\ifx\ACCSUPP@driver\relax
  \PackageError{accsupp}{%
    Missing driver option.\MessageBreak
    Package loading is aborted%
  }\@ehc
  \expandafter\ACCSUPP@AtEnd
\fi%
\InputIfFileExists{accsupp-\ACCSUPP@driver.def}{}{%
  \PackageError{accsupp}{%
    Unsupported driver `\ACCSUPP@driver'.\MessageBreak
    Package loading is aborted%
  }\@ehc
  \ACCSUPP@AtEnd
}%
%    \end{macrocode}
%
% \subsection{Main macro}
%
%    \begin{macrocode}
\DeclareBoolOption{unicode}
\DeclareStringOption[page]{pdfliteral}
\DeclareStringOption{Lang}
\def\ACCSUPP@method{escape}
\define@key{ACCSUPP}{method}{%
  \@ifundefined{ACCSUPP@method@#1}{%
    \PackageError{accsupp}{%
      Ignoring unknown method `#1'%
    }\@ehc
  }{%
    \edef\ACCSUPP@method{#1}%
  }%
}
\let\ACCSUPP@Lang\relax
\def\ACCSUPP@temp#1{%
  \expandafter\ACCSUPP@@temp\csname ACCSUPP@#1\endcsname{#1}%
}
\def\ACCSUPP@@temp#1#2{%
  \let#1\relax
  \define@key{ACCSUPP}{#2}{%
    \def#1{##1}%
    \ifx#1\@empty
      \def#1{()}%
    \else
      \csname ACCSUPP@method@\ACCSUPP@method\endcsname#1%
    \fi
  }%
}
\ACCSUPP@temp{Alt}
\ACCSUPP@temp{ActualText}
\ACCSUPP@temp{E}
%    \end{macrocode}
%
%    \begin{macrocode}
\newcommand*{\BeginAccSupp}[1]{%
  \begingroup
    \setkeys{ACCSUPP}{#1}%
    \edef\ACCSUPP@span{%
      /Span<<%
        \ifx\ACCSUPP@Lang\relax
        \else
          /Lang\ACCSUPP@Lang
        \fi
        \ifx\ACCSUPP@Alt\relax
        \else
          /Alt\ACCSUPP@Alt
        \fi
        \ifx\ACCSUPP@ActualText\relax
        \else
          /ActualText\ACCSUPP@ActualText
        \fi
        \ifx\ACCSUPP@E\relax
        \else
          /E\ACCSUPP@E
        \fi
      >>%
    }%
    \ACCSUPP@bdc
    \ACCSUPP@space
  \endgroup
}
\newcommand*{\EndAccSupp}[1]{%
  \begingroup
    \setkeys{ACCSUPP}{#1}%
    \ACCSUPP@emc
  \endgroup
}
\def\ACCSUPP@space{%
  \ifACCSUPP@space
    \begingroup
      \@ifundefined{ACCSUPP@Font}{%
        \global\font\ACCSUPP@Font=\ACCSUPP@spacefont\relax
      }{}%
      \leavevmode
      \setbox\z@\hbox{\ACCSUPP@Font\char\ACCSUPP@spacechar}%
      \wd\z@\z@
      \ht\z@\z@
      \dp\z@\z@
      \copy\z@
    \endgroup
  \fi
}
\newcommand*{\AccSuppSetup}{%
  \setkeys{ACCSUPP}%
}
%    \end{macrocode}
%
% \subsubsection{Input methods}
%
%    \begin{macrocode}
\def\ACCSUPP@method@plain#1{%
  \csname @safe@activestrue\endcsname
  \edef#1{%
    (%
    \ifACCSUPP@unicode
      \string\376\string\377%
    \fi
    #1%
    )%
  }%
  \@onelevel@sanitize#1%
}
%    \end{macrocode}
%    \begin{macrocode}
\def\ACCSUPP@method@escape#1{%
  \EdefEscapeString#1{%
    \ifACCSUPP@unicode
      ^^fe^^ff%
    \fi
    #1%
  }%
  \edef#1{(#1)}%
}%
%    \end{macrocode}
%    \begin{macrocode}
\def\ACCSUPP@method@hex#1{%
  \edef#1{%
    <%
    \ifACCSUPP@unicode
      FEFF%
    \fi
    #1%
    >%
  }%
}
%    \end{macrocode}
%    \begin{macrocode}
\def\ACCSUPP@method@pdfstringdef#1{%
  \ifACCSUPP@unicode
    \@ifundefined{hypersetup}{}{%
      \hypersetup{unicode}%
    }%
  \fi
  \@ifundefined{pdfstringdef}{%
    \PackageError{accsupp}{%
      Method `pdfstringdef' requires package `hyperref'%
    }\@ehc
    \let\ACCSUPP@temp\@empty
  }{%
    \begingroup
      \setbox0=\hbox{%
        \pdfstringdef\ACCSUPP@temp#1%
        \global\let\ACCSUPP@temp\ACCSUPP@temp
      }%
    \endgroup
  }%
  \edef#1{(\ACCSUPP@temp)}%
}
%    \end{macrocode}
%
%    \begin{macrocode}
\ACCSUPP@AtEnd%
%</package>
%    \end{macrocode}
%
% \subsection{Drivers}
%
% \subsubsection{Driver \xoption{pdftex}}
%
%    \begin{macrocode}
%<*pdftex>
\NeedsTeXFormat{LaTeX2e}
\ProvidesFile{accsupp-pdftex.def}%
  [2018/01/30 v0.5 accsupp driver for pdfTeX (HO)]%
%    \end{macrocode}
%
%    \begin{macrocode}
\def\ACCSUPP@bdc{%
  \pdfliteral\ACCSUPP@pdfliteral{\ACCSUPP@span BDC}%
}
\def\ACCSUPP@emc{%
  \pdfliteral\ACCSUPP@pdfliteral{EMC}%
}
%    \end{macrocode}
%    \begin{macrocode}
%</pdftex>
%    \end{macrocode}
%
% \subsubsection{Driver \xoption{luatex}}
%
%    \begin{macrocode}
%<*luatex>
\NeedsTeXFormat{LaTeX2e}
\ProvidesFile{accsupp-luatex.def}%
  [2018/01/30 v0.5 accsupp driver for pdfTeX (HO)]%
%    \end{macrocode}
%
%    \begin{macrocode}
\protected\def\ACCSUPP@pdf@literal{\pdfextension literal}
\def\ACCSUPP@bdc{%
  \ACCSUPP@pdf@literal\ACCSUPP@pdfliteral{\ACCSUPP@span BDC}%
}
\def\ACCSUPP@emc{%
  \ACCSUPP@pdf@literal\ACCSUPP@pdfliteral{EMC}%
}
%    \end{macrocode}
%    \begin{macrocode}
%</luatex>
%    \end{macrocode}
%
% \subsubsection{Driver \xoption{dvipdfm}}
%
%    \begin{macrocode}
%<*dvipdfm>
\NeedsTeXFormat{LaTeX2e}
\ProvidesFile{accsupp-dvipdfm.def}%
  [2018/01/30 v0.5 accsupp driver for dvipdfm (HO)]%
%    \end{macrocode}
%
%    \begin{macrocode}
\def\ACCSUPP@bdc{%
  \special{pdf:content \ACCSUPP@span BDC}%
}
\def\ACCSUPP@emc{%
  \special{pdf:content EMC}%
}
%    \end{macrocode}
%    \begin{macrocode}
%</dvipdfm>
%    \end{macrocode}
%
% \subsubsection{Driver \xoption{dvips}}
%
%    \begin{macrocode}
%<*dvips>
\NeedsTeXFormat{LaTeX2e}
\ProvidesFile{accsupp-dvips.def}%
  [2018/01/30 v0.5 accsupp driver for dvips (HO)]%
%    \end{macrocode}
%
%    \begin{macrocode}
\def\ACCSUPP@bdc{%
  \special{ps:[\ACCSUPP@span/BDC pdfmark}%
}
\def\ACCSUPP@emc{%
  \special{ps:[/EMC pdfmark}%
}
%    \end{macrocode}
%    \begin{macrocode}
%</dvips>
%    \end{macrocode}
%
% \section{Test}
%
% \subsection{Catcode checks for loading}
%
%    \begin{macrocode}
%<*test1>
%    \end{macrocode}
%    \begin{macrocode}
\NeedsTeXFormat{LaTeX2e}
\documentclass{minimal}
\makeatletter
\def\RestoreCatcodes{}
\count@=0 %
\loop
  \edef\RestoreCatcodes{%
    \RestoreCatcodes
    \catcode\the\count@=\the\catcode\count@\relax
  }%
\ifnum\count@<255 %
  \advance\count@\@ne
\repeat

\def\RangeCatcodeInvalid#1#2{%
  \count@=#1\relax
  \loop
    \catcode\count@=15 %
  \ifnum\count@<#2\relax
    \advance\count@\@ne
  \repeat
}
\def\Test{%
  \RangeCatcodeInvalid{0}{47}%
  \RangeCatcodeInvalid{58}{64}%
  \RangeCatcodeInvalid{91}{96}%
  \RangeCatcodeInvalid{123}{127}%
  \catcode`\@=12 %
  \catcode`\\=0 %
  \catcode`\{=1 %
  \catcode`\}=2 %
  \catcode`\#=6 %
  \catcode`\[=12 %
  \catcode`\]=12 %
  \catcode`\%=14 %
  \catcode`\ =10 %
  \catcode13=5 %
  \RequirePackage{accsupp}[2018/01/30]\relax
  \RestoreCatcodes
}
\Test
\csname @@end\endcsname
\end
%    \end{macrocode}
%    \begin{macrocode}
%</test1>
%    \end{macrocode}
%
% \section{Installation}
%
% \subsection{Download}
%
% \paragraph{Package.} This package is available on
% CTAN\footnote{\url{http://ctan.org/pkg/accsupp}}:
% \begin{description}
% \item[\CTAN{macros/latex/contrib/oberdiek/accsupp.dtx}] The source file.
% \item[\CTAN{macros/latex/contrib/oberdiek/accsupp.pdf}] Documentation.
% \end{description}
%
%
% \paragraph{Bundle.} All the packages of the bundle `oberdiek'
% are also available in a TDS compliant ZIP archive. There
% the packages are already unpacked and the documentation files
% are generated. The files and directories obey the TDS standard.
% \begin{description}
% \item[\CTAN{install/macros/latex/contrib/oberdiek.tds.zip}]
% \end{description}
% \emph{TDS} refers to the standard ``A Directory Structure
% for \TeX\ Files'' (\CTAN{tds/tds.pdf}). Directories
% with \xfile{texmf} in their name are usually organized this way.
%
% \subsection{Bundle installation}
%
% \paragraph{Unpacking.} Unpack the \xfile{oberdiek.tds.zip} in the
% TDS tree (also known as \xfile{texmf} tree) of your choice.
% Example (linux):
% \begin{quote}
%   |unzip oberdiek.tds.zip -d ~/texmf|
% \end{quote}
%
% \paragraph{Script installation.}
% Check the directory \xfile{TDS:scripts/oberdiek/} for
% scripts that need further installation steps.
% Package \xpackage{attachfile2} comes with the Perl script
% \xfile{pdfatfi.pl} that should be installed in such a way
% that it can be called as \texttt{pdfatfi}.
% Example (linux):
% \begin{quote}
%   |chmod +x scripts/oberdiek/pdfatfi.pl|\\
%   |cp scripts/oberdiek/pdfatfi.pl /usr/local/bin/|
% \end{quote}
%
% \subsection{Package installation}
%
% \paragraph{Unpacking.} The \xfile{.dtx} file is a self-extracting
% \docstrip\ archive. The files are extracted by running the
% \xfile{.dtx} through \plainTeX:
% \begin{quote}
%   \verb|tex accsupp.dtx|
% \end{quote}
%
% \paragraph{TDS.} Now the different files must be moved into
% the different directories in your installation TDS tree
% (also known as \xfile{texmf} tree):
% \begin{quote}
% \def\t{^^A
% \begin{tabular}{@{}>{\ttfamily}l@{ $\rightarrow$ }>{\ttfamily}l@{}}
%   accsupp.sty & tex/latex/oberdiek/accsupp.sty\\
%   accsupp-pdftex.def & tex/latex/oberdiek/accsupp-pdftex.def\\
%   accsupp-luatex.def & tex/latex/oberdiek/accsupp-luatex.def\\
%   accsupp-dvipdfm.def & tex/latex/oberdiek/accsupp-dvipdfm.def\\
%   accsupp-dvips.def & tex/latex/oberdiek/accsupp-dvips.def\\
%   accsupp.pdf & doc/latex/oberdiek/accsupp.pdf\\
%   accsupp-example1.tex & doc/latex/oberdiek/accsupp-example1.tex\\
%   accsupp-example2.tex & doc/latex/oberdiek/accsupp-example2.tex\\
%   test/accsupp-test1.tex & doc/latex/oberdiek/test/accsupp-test1.tex\\
%   accsupp.dtx & source/latex/oberdiek/accsupp.dtx\\
% \end{tabular}^^A
% }^^A
% \sbox0{\t}^^A
% \ifdim\wd0>\linewidth
%   \begingroup
%     \advance\linewidth by\leftmargin
%     \advance\linewidth by\rightmargin
%   \edef\x{\endgroup
%     \def\noexpand\lw{\the\linewidth}^^A
%   }\x
%   \def\lwbox{^^A
%     \leavevmode
%     \hbox to \linewidth{^^A
%       \kern-\leftmargin\relax
%       \hss
%       \usebox0
%       \hss
%       \kern-\rightmargin\relax
%     }^^A
%   }^^A
%   \ifdim\wd0>\lw
%     \sbox0{\small\t}^^A
%     \ifdim\wd0>\linewidth
%       \ifdim\wd0>\lw
%         \sbox0{\footnotesize\t}^^A
%         \ifdim\wd0>\linewidth
%           \ifdim\wd0>\lw
%             \sbox0{\scriptsize\t}^^A
%             \ifdim\wd0>\linewidth
%               \ifdim\wd0>\lw
%                 \sbox0{\tiny\t}^^A
%                 \ifdim\wd0>\linewidth
%                   \lwbox
%                 \else
%                   \usebox0
%                 \fi
%               \else
%                 \lwbox
%               \fi
%             \else
%               \usebox0
%             \fi
%           \else
%             \lwbox
%           \fi
%         \else
%           \usebox0
%         \fi
%       \else
%         \lwbox
%       \fi
%     \else
%       \usebox0
%     \fi
%   \else
%     \lwbox
%   \fi
% \else
%   \usebox0
% \fi
% \end{quote}
% If you have a \xfile{docstrip.cfg} that configures and enables \docstrip's
% TDS installing feature, then some files can already be in the right
% place, see the documentation of \docstrip.
%
% \subsection{Refresh file name databases}
%
% If your \TeX~distribution
% (\teTeX, \mikTeX, \dots) relies on file name databases, you must refresh
% these. For example, \teTeX\ users run \verb|texhash| or
% \verb|mktexlsr|.
%
% \subsection{Some details for the interested}
%
% \paragraph{Attached source.}
%
% The PDF documentation on CTAN also includes the
% \xfile{.dtx} source file. It can be extracted by
% AcrobatReader 6 or higher. Another option is \textsf{pdftk},
% e.g. unpack the file into the current directory:
% \begin{quote}
%   \verb|pdftk accsupp.pdf unpack_files output .|
% \end{quote}
%
% \paragraph{Unpacking with \LaTeX.}
% The \xfile{.dtx} chooses its action depending on the format:
% \begin{description}
% \item[\plainTeX:] Run \docstrip\ and extract the files.
% \item[\LaTeX:] Generate the documentation.
% \end{description}
% If you insist on using \LaTeX\ for \docstrip\ (really,
% \docstrip\ does not need \LaTeX), then inform the autodetect routine
% about your intention:
% \begin{quote}
%   \verb|latex \let\install=y% \iffalse meta-comment
%
% File: accsupp.dtx
% Version: 2018/01/30 v0.5
% Info: Accessibility support by marked content
%
% Copyright (C) 2007, 2010 by
%    Heiko Oberdiek <heiko.oberdiek at googlemail.com>
%    2016
%    https://github.com/ho-tex/oberdiek/issues
%
% This work may be distributed and/or modified under the
% conditions of the LaTeX Project Public License, either
% version 1.3c of this license or (at your option) any later
% version. This version of this license is in
%    http://www.latex-project.org/lppl/lppl-1-3c.txt
% and the latest version of this license is in
%    http://www.latex-project.org/lppl.txt
% and version 1.3 or later is part of all distributions of
% LaTeX version 2005/12/01 or later.
%
% This work has the LPPL maintenance status "maintained".
%
% This Current Maintainer of this work is Heiko Oberdiek.
%
% This work consists of the main source file accsupp.dtx
% and the derived files
%    accsupp.sty, accsupp.pdf, accsupp.ins, accsupp.drv,
%    accsupp-pdftex.def, \ccsupp-luatex.def, accsupp-dvipdfm.def,
%    accsupp-dvips.def,
%    accsupp-example1.tex, accsupp-example2.tex,
%    accsupp-test1.tex.
%
% Distribution:
%    CTAN:macros/latex/contrib/oberdiek/accsupp.dtx
%    CTAN:macros/latex/contrib/oberdiek/accsupp.pdf
%
% Unpacking:
%    (a) If accsupp.ins is present:
%           tex accsupp.ins
%    (b) Without accsupp.ins:
%           tex accsupp.dtx
%    (c) If you insist on using LaTeX
%           latex \let\install=y\input{accsupp.dtx}
%        (quote the arguments according to the demands of your shell)
%
% Documentation:
%    (a) If accsupp.drv is present:
%           latex accsupp.drv
%    (b) Without accsupp.drv:
%           latex accsupp.dtx; ...
%    The class ltxdoc loads the configuration file ltxdoc.cfg
%    if available. Here you can specify further options, e.g.
%    use A4 as paper format:
%       \PassOptionsToClass{a4paper}{article}
%
%    Programm calls to get the documentation (example):
%       pdflatex accsupp.dtx
%       makeindex -s gind.ist accsupp.idx
%       pdflatex accsupp.dtx
%       makeindex -s gind.ist accsupp.idx
%       pdflatex accsupp.dtx
%
% Installation:
%    TDS:tex/latex/oberdiek/accsupp.sty
%    TDS:tex/latex/oberdiek/accsupp-pdftex.def
%    TDS:tex/latex/oberdiek/accsupp-luatex.def
%    TDS:tex/latex/oberdiek/accsupp-dvipdfm.def
%    TDS:tex/latex/oberdiek/accsupp-dvips.def
%    TDS:doc/latex/oberdiek/accsupp.pdf
%    TDS:doc/latex/oberdiek/accsupp-example1.tex
%    TDS:doc/latex/oberdiek/accsupp-example2.tex
%    TDS:doc/latex/oberdiek/test/accsupp-test1.tex
%    TDS:source/latex/oberdiek/accsupp.dtx
%
%<*ignore>
\begingroup
  \catcode123=1 %
  \catcode125=2 %
  \def\x{LaTeX2e}%
\expandafter\endgroup
\ifcase 0\ifx\install y1\fi\expandafter
         \ifx\csname processbatchFile\endcsname\relax\else1\fi
         \ifx\fmtname\x\else 1\fi\relax
\else\csname fi\endcsname
%</ignore>
%<*install>
\input docstrip.tex
\Msg{************************************************************************}
\Msg{* Installation}
\Msg{* Package: accsupp 2018/01/30 v0.5 Accessibility support by marked content (HO)}
\Msg{************************************************************************}

\keepsilent
\askforoverwritefalse

\let\MetaPrefix\relax
\preamble

This is a generated file.

Project: accsupp
Version: 2018/01/30 v0.5

Copyright (C) 2007, 2010 by
   Heiko Oberdiek <heiko.oberdiek at googlemail.com>

This work may be distributed and/or modified under the
conditions of the LaTeX Project Public License, either
version 1.3c of this license or (at your option) any later
version. This version of this license is in
   http://www.latex-project.org/lppl/lppl-1-3c.txt
and the latest version of this license is in
   http://www.latex-project.org/lppl.txt
and version 1.3 or later is part of all distributions of
LaTeX version 2005/12/01 or later.

This work has the LPPL maintenance status "maintained".

This Current Maintainer of this work is Heiko Oberdiek.

This work consists of the main source file accsupp.dtx
and the derived files
   accsupp.sty, accsupp.pdf, accsupp.ins, accsupp.drv,
   accsupp-pdftex.def, accsupp-luatex.def, accsupp-dvipdfm.def,
   accsupp-dvips.def,
   accsupp-example1.tex, accsupp-example2.tex,
   accsupp-test1.tex.

\endpreamble
\let\MetaPrefix\DoubleperCent

\generate{%
  \file{accsupp.ins}{\from{accsupp.dtx}{install}}%
  \file{accsupp.drv}{\from{accsupp.dtx}{driver}}%
  \usedir{tex/latex/oberdiek}%
  \file{accsupp.sty}{\from{accsupp.dtx}{package}}%
  \file{accsupp-pdftex.def}{\from{accsupp.dtx}{pdftex}}%
  \file{accsupp-luatex.def}{\from{accsupp.dtx}{luatex}}%
  \file{accsupp-dvipdfm.def}{\from{accsupp.dtx}{dvipdfm}}%
  \file{accsupp-dvips.def}{\from{accsupp.dtx}{dvips}}%
  \usedir{doc/latex/oberdiek}%
  \file{accsupp-example1.tex}{\from{accsupp.dtx}{example1}}%
  \file{accsupp-example2.tex}{\from{accsupp.dtx}{example2}}%
  \usedir{doc/latex/oberdiek/test}%
  \file{accsupp-test1.tex}{\from{accsupp.dtx}{test1}}%
  \nopreamble
  \nopostamble
  \usedir{source/latex/oberdiek/catalogue}%
  \file{accsupp.xml}{\from{accsupp.dtx}{catalogue}}%
}

\catcode32=13\relax% active space
\let =\space%
\Msg{************************************************************************}
\Msg{*}
\Msg{* To finish the installation you have to move the following}
\Msg{* files into a directory searched by TeX:}
\Msg{*}
\Msg{*     accsupp.sty, accsupp-pdftex.def, accsupp-dvipdfm.def,}
\Msg{*     accsupp-dvips.def}
\Msg{*}
\Msg{* To produce the documentation run the file `accsupp.drv'}
\Msg{* through LaTeX.}
\Msg{*}
\Msg{* Happy TeXing!}
\Msg{*}
\Msg{************************************************************************}

\endbatchfile
%</install>
%<*ignore>
\fi
%</ignore>
%<*driver>
\NeedsTeXFormat{LaTeX2e}
\ProvidesFile{accsupp.drv}%
  [2018/01/30 v0.5 Accessibility support by marked content (HO)]%
\documentclass{ltxdoc}
\usepackage{holtxdoc}[2011/11/22]
\begin{document}
  \DocInput{accsupp.dtx}%
\end{document}
%</driver>
% \fi
%
%
% \CharacterTable
%  {Upper-case    \A\B\C\D\E\F\G\H\I\J\K\L\M\N\O\P\Q\R\S\T\U\V\W\X\Y\Z
%   Lower-case    \a\b\c\d\e\f\g\h\i\j\k\l\m\n\o\p\q\r\s\t\u\v\w\x\y\z
%   Digits        \0\1\2\3\4\5\6\7\8\9
%   Exclamation   \!     Double quote  \"     Hash (number) \#
%   Dollar        \$     Percent       \%     Ampersand     \&
%   Acute accent  \'     Left paren    \(     Right paren   \)
%   Asterisk      \*     Plus          \+     Comma         \,
%   Minus         \-     Point         \.     Solidus       \/
%   Colon         \:     Semicolon     \;     Less than     \<
%   Equals        \=     Greater than  \>     Question mark \?
%   Commercial at \@     Left bracket  \[     Backslash     \\
%   Right bracket \]     Circumflex    \^     Underscore    \_
%   Grave accent  \`     Left brace    \{     Vertical bar  \|
%   Right brace   \}     Tilde         \~}
%
% \GetFileInfo{accsupp.drv}
%
% \title{The \xpackage{accsupp} package}
% \date{2018/01/30 v0.5}
% \author{Heiko Oberdiek\thanks
% {Please report any issues at https://github.com/ho-tex/oberdiek/issues}\\
% \xemail{heiko.oberdiek at googlemail.com}}
%
% \maketitle
%
% \begin{abstract}
% Since PDF 1.5 portions of a page can be marked for better
% accessibility support.
% For example, replacement texts or expansions of abbreviations can be
% provided. Package \xpackage{accsupp} starts with providing a minimal
% low-level interface for programmers. Status is experimental.
% \end{abstract}
%
% \tableofcontents
%
% \section{Documentation}
%
% \subsection{Macros}
%
% Section ``10.8 Accessibility Support'' of the PDF reference
% \cite{pdfspec} lists some features that can be added by
% operators for marked content.
%
% \begin{declcs}{BeginAccSupp} \M{options}
% \end{declcs}
% It puts the operator |BDC| in the page stream:
% \begin{quote}
%   |/Span|\\
%   |<<|\dots|>>| \qquad \textit{\% property dictionary}\\
%   |BDC|
% \end{quote}
% The contents of the dictionary is controlled by \meta{options}.
% See sections \ref{sec:feature} and \ref{sec:method}.
%
% \begin{declcs}{EndAccSupp} \M{options}
% \end{declcs}
% It puts the operator |EMC| in the page stream.
% The only option is \xoption{pdfliteral}, see section \ref{sec:pdftex}.
%
% \bigskip
% \noindent\textbf{Note:}
% The caller is responsible for the placement of \cs{BeginAccSupp}
% and \cs{EndAccSupp} pairs. Especially page breaks are not allowed
% in between.
%
% \subsubsection{Feature options}\label{sec:feature}
%
% The PDF reference \cite{pdfspec} describes and explains the different
% features. The names of the feature options are the same as the
% key names for the property dictionary for operator |BDC|, see
% \cs{BeginAccSupp}.
%
% \begin{description}
% \item[\xoption{ActualText}:]
%   Provides a replacement text, see examples in
%   section \ref{sec:example}.
% \item[\xoption{Alt}:]
%   Provides an alternate description.
% \item[\xoption{E}:]
%   Provides the expansion of an abbreviation or an acronym.
% \item[\xoption{Lang}:]
%   Specifies the language.
% \end{description}
%
% \subsubsection{Input methods}\label{sec:method}
%
% Except for \xoption{Lang} option \xoption{method} controls
% how the argument for \xoption{ActualText}, \xoption{Alt},
% and \xoption{E} are interpreted.
% \begin{description}
% \item[\xoption{method=plain}:]
%   The string is only expanded and written without further
%   treatment. Special characters are not protected, thus this method
%   may result in an invalid PDF file.
% \item[\xoption{mehod=escape}:]
%   The string is expanded and special characters are escaped.
%   The result is a valid PDF string.
% \item[\xoption{method=hex}:]
%   The string is given in hexadecimal notation.
%   Section \ref{sec:example:notparallel} shows an example.
% \item[\xoption{method=pdfstringdef}:]
%   If package \xpackage{hyperref} is loaded, then its \cs{pdfstringdef}
%   is used. This method is slow, but useful if the string contains
%   arbitrary \hologo{TeX} code.
% \item[\xoption{unicode}:]
%   This option is needed, if the string is given as Unicode string
%   (16 bit). Internally it adds the string prefix for Unicode.
%   In case of \xoption{method=pdfstringdef} it passes the option
%   to \cs{hypersetup}.
% \end{description}
%
% \subsection{Workaround, option \xoption{space}}
%
% PDF specification says in ``10.8.3 Replacement Text''
% \begin{quote}
% Just as alternative descriptions can be provided for images and other
% items that do not translate naturally into text (\dots), replacement
% text can be specified for content that does translate into text but
% that is represented in a nonstandard way. These nonstandard representations
% might include, for example, glyphs for ligatures or custom characters, or
% inline graphics corresponding to letters in an illuminated manustript or to
% dropped capitals.
% \end{quote}
% However, the ActualText is ignored in Acrobat Reader
% (until version 9 at least), if the content does not contain glyphs.
% Option \xoption{space} adds such an invisible glyph, a space
% character. The font name can be configured by option \xoption{spacefont},
% the default font is \texttt{phvr8r}. The character slot is given
% by option \xoption{spacechar}, default is \texttt{32}, the usual
% position of the space character.
%
% These options can also be given as package options or in
% macro \cs{AccSuppSetup} that takes a key value list as argument.
% Usually only option \xoption{space} is necessary, if the
% contents does not contain glyphs otherwise. Option \xoption{space}
% is enabled by |space| or |space=true| and disabled by |space=false|.
% It is disabled as default. The option is evaluated by \cs{BeginAccSupp}
% and ignored by \cs{EndAccSupp}.
%
% Note: Even with option \xoption{space} I get sometimes wrong texts
% when cut \& paste from AR7/Linux or AR8/Linux, e.g.
% \begin{quote}
%   |Hello| $\rightarrow$ |Helo|, |Helol|, |Hell|, \dots\\
%   |Hello World| $\rightarrow$ |Helo WorldW|, |Helo World|, \dots
% \end{quote}
% I do not know what Acrobat Reader is doing here, thus feedback and
% insights are welcome.
%
% \subsection{Driver options}
%
% Driver options are package options only.
% The special \hologo{TeX} compilers \hologo{pdfTeX}, \hologo{LuaTeX} and \hologo{XeTeX}
% are detected automatically. The default for unrecognized drivers can be
% set by defining \cs{ActualTextDriverDefault}. This can be done
% in the configuration file \xfile{accsupp.cfg}.
%
% \subsubsection{Option \xoption{pdftex}}\label{sec:pdftex}
% Package option \xoption{pdftex} is used for \hologo{pdfTeX} in PDF mode.
% Additionally \cs{BeginAccSupp} and \cs{EndAccSupp} understand
% option \xoption{pdfliteral}. It controls
% the modifier keyword for \cs{pdfliteral}:
% \begin{quote}
%   |pdfliteral=direct| $\Rightarrow$ |\pdfliteral direct{...}|
% \end{quote}
%
% \subsubsection{Option \xoption{luatex}}\label{sec:luatex}
% Package option \xoption{luatex} is used for \hologo{LuaTeX} in PDF mode.
% Additionally \cs{BeginAccSupp} and \cs{EndAccSupp} understand
% option \xoption{pdfliteral}. It controls
% the modifier keyword for \cs{pdfextension} literal:
% \begin{quote}
%   |pdfliteral=direct| $\Rightarrow$ |\pdfextension literal direct{...}|
% \end{quote}
%
% \subsubsection{Option \xoption{dvipdfm}}
%
% Package option \xoption{dvipdfm} and its aliases \xoption{dvipdfmx}
% \xoption{xetex} are used for drivers that support
% \textsf{dvipdfm} specials.
%
% \subsubsection{Option \xoption{dvips}}
%
% Package option \xoption{dvips} and its alias \xoption{dvipsone}
% write pdfmark specials in the output. Unhappily these pdfmark
% operators are ignored by \textsf{ghostscript} (latest tested
% version is 8.54). Perhaps they are recognized by commercial
% distiller applications.
%
% \subsubsection{Turning off page stream compression}
%
% For debugging it is useful to have uncompressed page stream objects.
% This can be done afterwards via \textsf{pdftk}:
% \begin{quote}
%   |pdftk file.pdf output file-uncompressed.pdf uncompress|
% \end{quote}
% Or the PDF file is generated uncompressed:
% \begin{description}
% \item[\hologo{pdfTeX}:] |\pdfcompresslevel=0|
% \item[dvipdfm:] |dvipdfm -z0 ...|
% \item[dvipdfmx:] |dvipdfmx -z0 ...|
% \item[ghostscript/ps2pdf:]
%      |ps2pdf -dCompressPages=false input.ps output.pdf|
% \end{description}
%
% \section{Example}
% \label{sec:example}
%
% \subsection{Example \cs{notparallel}}
% \label{sec:example:notparallel}
%
%    \begin{macrocode}
%<*example1>
%    \end{macrocode}
%    \begin{macrocode}
%<<END
\documentclass{article}
\usepackage{accsupp}[2007/11/14]
\usepackage{centernot}
% U+2226 NOT PARALLEL
% \mathrel{...} prevents page break in between
\newcommand*{\notparallel}{%
  \ensuremath{%
    \mathrel{%
      \BeginAccSupp{method=hex,unicode,ActualText=2226}%
      \centernot{\parallel}%
      \EndAccSupp{}%
    }%
  }%
}
\begin{document}
\begin{equation}
A\notparallel B
\end{equation}
\end{document}
%END
%    \end{macrocode}
%    \begin{macrocode}
%</example1>
%    \end{macrocode}
%
% \subsection{Example with \xoption{pdfstringdef}}
%
%    \begin{macrocode}
%<*example2>
%    \end{macrocode}
%    \begin{macrocode}
%<<END
\documentclass{article}
\usepackage[unicode]{hyperref}
\usepackage{accsupp}[2007/11/14]
\begin{document}
  \begin{equation}
    \BeginAccSupp{
      method=pdfstringdef,
      unicode,
      ActualText={%
        a\texttwosuperior +b\texttwosuperior
        =c\texttwosuperior
      }
    }
    a^2 + b^2 = c^2
    \EndAccSupp{}
  \end{equation}
\end{document}
%END
%    \end{macrocode}
%    \begin{macrocode}
%</example2>
%    \end{macrocode}
%
% \StopEventually{
% }
%
% \section{Implementation}
%
% \subsection{Package}
%
%    \begin{macrocode}
%<*package>
%    \end{macrocode}
%
%    \begin{macrocode}
\begingroup\catcode61\catcode48\catcode32=10\relax%
  \catcode13=5 % ^^M
  \endlinechar=13 %
  \catcode123=1 % {
  \catcode125=2 % }
  \catcode64=11 % @
  \def\x{\endgroup
    \expandafter\edef\csname ACCSUPP@AtEnd\endcsname{%
      \endlinechar=\the\endlinechar\relax
      \catcode13=\the\catcode13\relax
      \catcode32=\the\catcode32\relax
      \catcode35=\the\catcode35\relax
      \catcode61=\the\catcode61\relax
      \catcode64=\the\catcode64\relax
      \catcode123=\the\catcode123\relax
      \catcode125=\the\catcode125\relax
    }%
  }%
\x\catcode61\catcode48\catcode32=10\relax%
\catcode13=5 % ^^M
\endlinechar=13 %
\catcode35=6 % #
\catcode64=11 % @
\catcode123=1 % {
\catcode125=2 % }
\def\TMP@EnsureCode#1#2{%
  \edef\ACCSUPP@AtEnd{%
    \ACCSUPP@AtEnd
    \catcode#1=\the\catcode#1\relax
  }%
  \catcode#1=#2\relax
}
\TMP@EnsureCode{10}{12}% ^^J
\TMP@EnsureCode{33}{12}% !
\TMP@EnsureCode{39}{12}% '
\TMP@EnsureCode{40}{12}% (
\TMP@EnsureCode{41}{12}% )
\TMP@EnsureCode{42}{12}% *
\TMP@EnsureCode{44}{12}% ,
\TMP@EnsureCode{45}{12}% -
\TMP@EnsureCode{46}{12}% .
\TMP@EnsureCode{47}{12}% /
\TMP@EnsureCode{58}{12}% :
\TMP@EnsureCode{60}{12}% <
\TMP@EnsureCode{62}{12}% >
\TMP@EnsureCode{94}{7}% ^ (superscript)
\TMP@EnsureCode{96}{12}% `
\TMP@EnsureCode{254}{12}% ^^fe
\TMP@EnsureCode{255}{12}% ^^ff
\edef\ACCSUPP@AtEnd{\ACCSUPP@AtEnd\noexpand\endinput}
%    \end{macrocode}
%
%    Package identification.
%    \begin{macrocode}
\NeedsTeXFormat{LaTeX2e}
\ProvidesPackage{accsupp}%
  [2018/01/30 v0.5 Accessibility support by marked content (HO)]
%    \end{macrocode}
%
%    \begin{macrocode}
\RequirePackage{pdfescape}[2007/02/25]
\RequirePackage{ifpdf}
\RequirePackage{ifxetex}
\RequirePackage{kvoptions}
%    \end{macrocode}
%    \begin{macrocode}
\SetupKeyvalOptions{%
  family=ACCSUPP,%
  prefix=ACCSUPP@%
}
%    \end{macrocode}
%
% \subsection{Space setup}
%
%    \begin{macrocode}
\DeclareBoolOption{space}
\DeclareStringOption[phvr8r]{spacefont}
\DeclareStringOption[32]{spacechar}
%    \end{macrocode}
%
% \subsection{Driver detection and setup}
%
% \paragraph{Driver declarations.}
%
%    \begin{macrocode}
\def\ACCSUPP@DefineDriverKey{%
  \@dblarg\ACCSUPP@@DefineDriverKey
}
\def\ACCSUPP@@DefineDriverKey[#1]#2{%
  \define@key{ACCSUPP}{#2}[]{%
    \def\ACCSUPP@driver{#1}%
  }%
  \g@addto@macro\ACCSUPP@DisableOptions{%
    \DisableKeyvalOption{ACCSUPP}{#2}%
  }%
}
\let\ACCSUPP@DisableOptions\@empty
\ACCSUPP@DefineDriverKey{pdftex}
\ACCSUPP@DefineDriverKey{luatex}
\ACCSUPP@DefineDriverKey{dvips}
\ACCSUPP@DefineDriverKey[dvips]{dvipsone}
\ACCSUPP@DefineDriverKey{dvipdfm}
\ACCSUPP@DefineDriverKey[dvipdfm]{dvipdfmx}
\ACCSUPP@DefineDriverKey[dvipdfm]{xetex}
\let\ACCSUPP@driver\relax
\InputIfFileExists{accsupp.cfg}{}{}
\providecommand*{\ActualTextDriverDefault}{dvips}
\ifpdf
  \ifx\pdfextension\@undefined
    \def\ACCSUPP@driver{pdftex}%
  \else
    \def\ACCSUPP@driver{luatex}%
  \fi
\else
  \ifxetex
    \def\ACCSUPP@driver{dvipdfm}%
  \else
    \ifx\ACCSUPP@driver\relax
      \let\ACCSUPP@driver\ActualTextDriverDefault
    \fi
  \fi
\fi
%    \end{macrocode}
%
% \paragraph{Process options.}
%
%    \begin{macrocode}
\ProcessKeyvalOptions*
\ACCSUPP@DisableOptions
%    \end{macrocode}
%
% \paragraph{Driver validation and loading.}
%
%    \begin{macrocode}
\def\ACCSUPP@temp{pdftex}%
\ifpdf
  \ifx\pdfextension\@undefined\else
    \def\ACCSUPP@temp{luatex}%
  \fi
  \ifx\ACCSUPP@temp\ACCSUPP@driver
  \else
    \PackageWarningNoLine{accsupp}{%
      Wrong driver `\ACCSUPP@driver', using `pdftex' instead%
    }%
ed as
    \let\ACCSUPP@driver\ACCSUPP@temp
  \fi
\else
  \ifx\ACCSUPP@temp\ACCSUPP@driver
    \PackageError{accsupp}{%
      Wrong driver, pdfTeX is not running in PDF mode.\MessageBreak
      Package loading is aborted%
    }\@ehc
    \expandafter\expandafter\expandafter\ACCSUPP@AtEnd
  \fi
  \def\ACCSUPP@temp{dvipdfm}%
  \ifxetex
    \ifx\ACCSUPP@temp\ACCSUPP@driver
    \else
      \PackageWarningNoLine{accsupp}{%
        Wrong driver `\ACCSUPP@driver',\MessageBreak
        using `dvipdfm' for XeTeX instead%
      }%
      \let\ACCSUPP@driver\ACCSUPP@temp
    \fi
  \fi
\fi%
\ifx\ACCSUPP@driver\relax
  \PackageError{accsupp}{%
    Missing driver option.\MessageBreak
    Package loading is aborted%
  }\@ehc
  \expandafter\ACCSUPP@AtEnd
\fi%
\InputIfFileExists{accsupp-\ACCSUPP@driver.def}{}{%
  \PackageError{accsupp}{%
    Unsupported driver `\ACCSUPP@driver'.\MessageBreak
    Package loading is aborted%
  }\@ehc
  \ACCSUPP@AtEnd
}%
%    \end{macrocode}
%
% \subsection{Main macro}
%
%    \begin{macrocode}
\DeclareBoolOption{unicode}
\DeclareStringOption[page]{pdfliteral}
\DeclareStringOption{Lang}
\def\ACCSUPP@method{escape}
\define@key{ACCSUPP}{method}{%
  \@ifundefined{ACCSUPP@method@#1}{%
    \PackageError{accsupp}{%
      Ignoring unknown method `#1'%
    }\@ehc
  }{%
    \edef\ACCSUPP@method{#1}%
  }%
}
\let\ACCSUPP@Lang\relax
\def\ACCSUPP@temp#1{%
  \expandafter\ACCSUPP@@temp\csname ACCSUPP@#1\endcsname{#1}%
}
\def\ACCSUPP@@temp#1#2{%
  \let#1\relax
  \define@key{ACCSUPP}{#2}{%
    \def#1{##1}%
    \ifx#1\@empty
      \def#1{()}%
    \else
      \csname ACCSUPP@method@\ACCSUPP@method\endcsname#1%
    \fi
  }%
}
\ACCSUPP@temp{Alt}
\ACCSUPP@temp{ActualText}
\ACCSUPP@temp{E}
%    \end{macrocode}
%
%    \begin{macrocode}
\newcommand*{\BeginAccSupp}[1]{%
  \begingroup
    \setkeys{ACCSUPP}{#1}%
    \edef\ACCSUPP@span{%
      /Span<<%
        \ifx\ACCSUPP@Lang\relax
        \else
          /Lang\ACCSUPP@Lang
        \fi
        \ifx\ACCSUPP@Alt\relax
        \else
          /Alt\ACCSUPP@Alt
        \fi
        \ifx\ACCSUPP@ActualText\relax
        \else
          /ActualText\ACCSUPP@ActualText
        \fi
        \ifx\ACCSUPP@E\relax
        \else
          /E\ACCSUPP@E
        \fi
      >>%
    }%
    \ACCSUPP@bdc
    \ACCSUPP@space
  \endgroup
}
\newcommand*{\EndAccSupp}[1]{%
  \begingroup
    \setkeys{ACCSUPP}{#1}%
    \ACCSUPP@emc
  \endgroup
}
\def\ACCSUPP@space{%
  \ifACCSUPP@space
    \begingroup
      \@ifundefined{ACCSUPP@Font}{%
        \global\font\ACCSUPP@Font=\ACCSUPP@spacefont\relax
      }{}%
      \leavevmode
      \setbox\z@\hbox{\ACCSUPP@Font\char\ACCSUPP@spacechar}%
      \wd\z@\z@
      \ht\z@\z@
      \dp\z@\z@
      \copy\z@
    \endgroup
  \fi
}
\newcommand*{\AccSuppSetup}{%
  \setkeys{ACCSUPP}%
}
%    \end{macrocode}
%
% \subsubsection{Input methods}
%
%    \begin{macrocode}
\def\ACCSUPP@method@plain#1{%
  \csname @safe@activestrue\endcsname
  \edef#1{%
    (%
    \ifACCSUPP@unicode
      \string\376\string\377%
    \fi
    #1%
    )%
  }%
  \@onelevel@sanitize#1%
}
%    \end{macrocode}
%    \begin{macrocode}
\def\ACCSUPP@method@escape#1{%
  \EdefEscapeString#1{%
    \ifACCSUPP@unicode
      ^^fe^^ff%
    \fi
    #1%
  }%
  \edef#1{(#1)}%
}%
%    \end{macrocode}
%    \begin{macrocode}
\def\ACCSUPP@method@hex#1{%
  \edef#1{%
    <%
    \ifACCSUPP@unicode
      FEFF%
    \fi
    #1%
    >%
  }%
}
%    \end{macrocode}
%    \begin{macrocode}
\def\ACCSUPP@method@pdfstringdef#1{%
  \ifACCSUPP@unicode
    \@ifundefined{hypersetup}{}{%
      \hypersetup{unicode}%
    }%
  \fi
  \@ifundefined{pdfstringdef}{%
    \PackageError{accsupp}{%
      Method `pdfstringdef' requires package `hyperref'%
    }\@ehc
    \let\ACCSUPP@temp\@empty
  }{%
    \begingroup
      \setbox0=\hbox{%
        \pdfstringdef\ACCSUPP@temp#1%
        \global\let\ACCSUPP@temp\ACCSUPP@temp
      }%
    \endgroup
  }%
  \edef#1{(\ACCSUPP@temp)}%
}
%    \end{macrocode}
%
%    \begin{macrocode}
\ACCSUPP@AtEnd%
%</package>
%    \end{macrocode}
%
% \subsection{Drivers}
%
% \subsubsection{Driver \xoption{pdftex}}
%
%    \begin{macrocode}
%<*pdftex>
\NeedsTeXFormat{LaTeX2e}
\ProvidesFile{accsupp-pdftex.def}%
  [2018/01/30 v0.5 accsupp driver for pdfTeX (HO)]%
%    \end{macrocode}
%
%    \begin{macrocode}
\def\ACCSUPP@bdc{%
  \pdfliteral\ACCSUPP@pdfliteral{\ACCSUPP@span BDC}%
}
\def\ACCSUPP@emc{%
  \pdfliteral\ACCSUPP@pdfliteral{EMC}%
}
%    \end{macrocode}
%    \begin{macrocode}
%</pdftex>
%    \end{macrocode}
%
% \subsubsection{Driver \xoption{luatex}}
%
%    \begin{macrocode}
%<*luatex>
\NeedsTeXFormat{LaTeX2e}
\ProvidesFile{accsupp-luatex.def}%
  [2018/01/30 v0.5 accsupp driver for pdfTeX (HO)]%
%    \end{macrocode}
%
%    \begin{macrocode}
\protected\def\ACCSUPP@pdf@literal{\pdfextension literal}
\def\ACCSUPP@bdc{%
  \ACCSUPP@pdf@literal\ACCSUPP@pdfliteral{\ACCSUPP@span BDC}%
}
\def\ACCSUPP@emc{%
  \ACCSUPP@pdf@literal\ACCSUPP@pdfliteral{EMC}%
}
%    \end{macrocode}
%    \begin{macrocode}
%</luatex>
%    \end{macrocode}
%
% \subsubsection{Driver \xoption{dvipdfm}}
%
%    \begin{macrocode}
%<*dvipdfm>
\NeedsTeXFormat{LaTeX2e}
\ProvidesFile{accsupp-dvipdfm.def}%
  [2018/01/30 v0.5 accsupp driver for dvipdfm (HO)]%
%    \end{macrocode}
%
%    \begin{macrocode}
\def\ACCSUPP@bdc{%
  \special{pdf:content \ACCSUPP@span BDC}%
}
\def\ACCSUPP@emc{%
  \special{pdf:content EMC}%
}
%    \end{macrocode}
%    \begin{macrocode}
%</dvipdfm>
%    \end{macrocode}
%
% \subsubsection{Driver \xoption{dvips}}
%
%    \begin{macrocode}
%<*dvips>
\NeedsTeXFormat{LaTeX2e}
\ProvidesFile{accsupp-dvips.def}%
  [2018/01/30 v0.5 accsupp driver for dvips (HO)]%
%    \end{macrocode}
%
%    \begin{macrocode}
\def\ACCSUPP@bdc{%
  \special{ps:[\ACCSUPP@span/BDC pdfmark}%
}
\def\ACCSUPP@emc{%
  \special{ps:[/EMC pdfmark}%
}
%    \end{macrocode}
%    \begin{macrocode}
%</dvips>
%    \end{macrocode}
%
% \section{Test}
%
% \subsection{Catcode checks for loading}
%
%    \begin{macrocode}
%<*test1>
%    \end{macrocode}
%    \begin{macrocode}
\NeedsTeXFormat{LaTeX2e}
\documentclass{minimal}
\makeatletter
\def\RestoreCatcodes{}
\count@=0 %
\loop
  \edef\RestoreCatcodes{%
    \RestoreCatcodes
    \catcode\the\count@=\the\catcode\count@\relax
  }%
\ifnum\count@<255 %
  \advance\count@\@ne
\repeat

\def\RangeCatcodeInvalid#1#2{%
  \count@=#1\relax
  \loop
    \catcode\count@=15 %
  \ifnum\count@<#2\relax
    \advance\count@\@ne
  \repeat
}
\def\Test{%
  \RangeCatcodeInvalid{0}{47}%
  \RangeCatcodeInvalid{58}{64}%
  \RangeCatcodeInvalid{91}{96}%
  \RangeCatcodeInvalid{123}{127}%
  \catcode`\@=12 %
  \catcode`\\=0 %
  \catcode`\{=1 %
  \catcode`\}=2 %
  \catcode`\#=6 %
  \catcode`\[=12 %
  \catcode`\]=12 %
  \catcode`\%=14 %
  \catcode`\ =10 %
  \catcode13=5 %
  \RequirePackage{accsupp}[2018/01/30]\relax
  \RestoreCatcodes
}
\Test
\csname @@end\endcsname
\end
%    \end{macrocode}
%    \begin{macrocode}
%</test1>
%    \end{macrocode}
%
% \section{Installation}
%
% \subsection{Download}
%
% \paragraph{Package.} This package is available on
% CTAN\footnote{\url{http://ctan.org/pkg/accsupp}}:
% \begin{description}
% \item[\CTAN{macros/latex/contrib/oberdiek/accsupp.dtx}] The source file.
% \item[\CTAN{macros/latex/contrib/oberdiek/accsupp.pdf}] Documentation.
% \end{description}
%
%
% \paragraph{Bundle.} All the packages of the bundle `oberdiek'
% are also available in a TDS compliant ZIP archive. There
% the packages are already unpacked and the documentation files
% are generated. The files and directories obey the TDS standard.
% \begin{description}
% \item[\CTAN{install/macros/latex/contrib/oberdiek.tds.zip}]
% \end{description}
% \emph{TDS} refers to the standard ``A Directory Structure
% for \TeX\ Files'' (\CTAN{tds/tds.pdf}). Directories
% with \xfile{texmf} in their name are usually organized this way.
%
% \subsection{Bundle installation}
%
% \paragraph{Unpacking.} Unpack the \xfile{oberdiek.tds.zip} in the
% TDS tree (also known as \xfile{texmf} tree) of your choice.
% Example (linux):
% \begin{quote}
%   |unzip oberdiek.tds.zip -d ~/texmf|
% \end{quote}
%
% \paragraph{Script installation.}
% Check the directory \xfile{TDS:scripts/oberdiek/} for
% scripts that need further installation steps.
% Package \xpackage{attachfile2} comes with the Perl script
% \xfile{pdfatfi.pl} that should be installed in such a way
% that it can be called as \texttt{pdfatfi}.
% Example (linux):
% \begin{quote}
%   |chmod +x scripts/oberdiek/pdfatfi.pl|\\
%   |cp scripts/oberdiek/pdfatfi.pl /usr/local/bin/|
% \end{quote}
%
% \subsection{Package installation}
%
% \paragraph{Unpacking.} The \xfile{.dtx} file is a self-extracting
% \docstrip\ archive. The files are extracted by running the
% \xfile{.dtx} through \plainTeX:
% \begin{quote}
%   \verb|tex accsupp.dtx|
% \end{quote}
%
% \paragraph{TDS.} Now the different files must be moved into
% the different directories in your installation TDS tree
% (also known as \xfile{texmf} tree):
% \begin{quote}
% \def\t{^^A
% \begin{tabular}{@{}>{\ttfamily}l@{ $\rightarrow$ }>{\ttfamily}l@{}}
%   accsupp.sty & tex/latex/oberdiek/accsupp.sty\\
%   accsupp-pdftex.def & tex/latex/oberdiek/accsupp-pdftex.def\\
%   accsupp-luatex.def & tex/latex/oberdiek/accsupp-luatex.def\\
%   accsupp-dvipdfm.def & tex/latex/oberdiek/accsupp-dvipdfm.def\\
%   accsupp-dvips.def & tex/latex/oberdiek/accsupp-dvips.def\\
%   accsupp.pdf & doc/latex/oberdiek/accsupp.pdf\\
%   accsupp-example1.tex & doc/latex/oberdiek/accsupp-example1.tex\\
%   accsupp-example2.tex & doc/latex/oberdiek/accsupp-example2.tex\\
%   test/accsupp-test1.tex & doc/latex/oberdiek/test/accsupp-test1.tex\\
%   accsupp.dtx & source/latex/oberdiek/accsupp.dtx\\
% \end{tabular}^^A
% }^^A
% \sbox0{\t}^^A
% \ifdim\wd0>\linewidth
%   \begingroup
%     \advance\linewidth by\leftmargin
%     \advance\linewidth by\rightmargin
%   \edef\x{\endgroup
%     \def\noexpand\lw{\the\linewidth}^^A
%   }\x
%   \def\lwbox{^^A
%     \leavevmode
%     \hbox to \linewidth{^^A
%       \kern-\leftmargin\relax
%       \hss
%       \usebox0
%       \hss
%       \kern-\rightmargin\relax
%     }^^A
%   }^^A
%   \ifdim\wd0>\lw
%     \sbox0{\small\t}^^A
%     \ifdim\wd0>\linewidth
%       \ifdim\wd0>\lw
%         \sbox0{\footnotesize\t}^^A
%         \ifdim\wd0>\linewidth
%           \ifdim\wd0>\lw
%             \sbox0{\scriptsize\t}^^A
%             \ifdim\wd0>\linewidth
%               \ifdim\wd0>\lw
%                 \sbox0{\tiny\t}^^A
%                 \ifdim\wd0>\linewidth
%                   \lwbox
%                 \else
%                   \usebox0
%                 \fi
%               \else
%                 \lwbox
%               \fi
%             \else
%               \usebox0
%             \fi
%           \else
%             \lwbox
%           \fi
%         \else
%           \usebox0
%         \fi
%       \else
%         \lwbox
%       \fi
%     \else
%       \usebox0
%     \fi
%   \else
%     \lwbox
%   \fi
% \else
%   \usebox0
% \fi
% \end{quote}
% If you have a \xfile{docstrip.cfg} that configures and enables \docstrip's
% TDS installing feature, then some files can already be in the right
% place, see the documentation of \docstrip.
%
% \subsection{Refresh file name databases}
%
% If your \TeX~distribution
% (\teTeX, \mikTeX, \dots) relies on file name databases, you must refresh
% these. For example, \teTeX\ users run \verb|texhash| or
% \verb|mktexlsr|.
%
% \subsection{Some details for the interested}
%
% \paragraph{Attached source.}
%
% The PDF documentation on CTAN also includes the
% \xfile{.dtx} source file. It can be extracted by
% AcrobatReader 6 or higher. Another option is \textsf{pdftk},
% e.g. unpack the file into the current directory:
% \begin{quote}
%   \verb|pdftk accsupp.pdf unpack_files output .|
% \end{quote}
%
% \paragraph{Unpacking with \LaTeX.}
% The \xfile{.dtx} chooses its action depending on the format:
% \begin{description}
% \item[\plainTeX:] Run \docstrip\ and extract the files.
% \item[\LaTeX:] Generate the documentation.
% \end{description}
% If you insist on using \LaTeX\ for \docstrip\ (really,
% \docstrip\ does not need \LaTeX), then inform the autodetect routine
% about your intention:
% \begin{quote}
%   \verb|latex \let\install=y\input{accsupp.dtx}|
% \end{quote}
% Do not forget to quote the argument according to the demands
% of your shell.
%
% \paragraph{Generating the documentation.}
% You can use both the \xfile{.dtx} or the \xfile{.drv} to generate
% the documentation. The process can be configured by the
% configuration file \xfile{ltxdoc.cfg}. For instance, put this
% line into this file, if you want to have A4 as paper format:
% \begin{quote}
%   \verb|\PassOptionsToClass{a4paper}{article}|
% \end{quote}
% An example follows how to generate the
% documentation with pdf\LaTeX:
% \begin{quote}
%\begin{verbatim}
%pdflatex accsupp.dtx
%makeindex -s gind.ist accsupp.idx
%pdflatex accsupp.dtx
%makeindex -s gind.ist accsupp.idx
%pdflatex accsupp.dtx
%\end{verbatim}
% \end{quote}
%
% \section{Catalogue}
%
% The following XML file can be used as source for the
% \href{http://mirror.ctan.org/help/Catalogue/catalogue.html}{\TeX\ Catalogue}.
% The elements \texttt{caption} and \texttt{description} are imported
% from the original XML file from the Catalogue.
% The name of the XML file in the Catalogue is \xfile{accsupp.xml}.
%    \begin{macrocode}
%<*catalogue>
<?xml version='1.0' encoding='us-ascii'?>
<!DOCTYPE entry SYSTEM 'catalogue.dtd'>
<entry datestamp='$Date$' modifier='$Author$' id='accsupp'>
  <name>accsupp</name>
  <caption>Better accessibility support for PDF files.</caption>
  <authorref id='auth:oberdiek'/>
  <copyright owner='Heiko Oberdiek' year='2007,2010'/>
  <license type='lppl1.3'/>
  <version number='0.4'/>
  <description>
    Since PDF 1.5 portions of a page can be marked for better accessibility
    support. For example, replacement texts or expansions of abbreviations can
    be provided. This package starts with providing a minimal low-level
    interface for programmers; its status is experimental.
    <p/>
    The package is part of the <xref refid='oberdiek'>oberdiek</xref> bundle.
  </description>
  <documentation details='Package documentation'
      href='ctan:/macros/latex/contrib/oberdiek/accsupp.pdf'/>
  <ctan file='true' path='/macros/latex/contrib/oberdiek/accsupp.dtx'/>
  <miktex location='oberdiek'/>
  <texlive location='oberdiek'/>
  <install path='/macros/latex/contrib/oberdiek/oberdiek.tds.zip'/>
</entry>
%</catalogue>
%    \end{macrocode}
%
% \begin{thebibliography}{9}
% \bibitem{pdfspec}
% Adobe Systems Incorporated,
% \textit{PDF Reference}, 6th edition, 2006.
% \url{http://www.adobe.com/devnet/acrobat/pdfs/pdf_reference.pdf}
%
% \end{thebibliography}
%
% \begin{History}
%   \begin{Version}{2007/03/21 v0.1}
%   \item
%     First version.
%   \end{Version}
%   \begin{Version}{2007/11/14 v0.2}
%   \item
%     Various bug fixes.
%   \item
%     Catcode section rewritten, test added.
%   \end{Version}
%   \begin{Version}{2010/01/16 v0.3}
%   \item
%     \cs{AccSuppSetup} added.
%   \item
%     Options \xoption{space}, \xoption{spacefont}, \xoption{spacechar} added.
%   \end{Version}
%   \begin{Version}{2016/05/16 v0.4}
%   \item
%     Documentation updates.
%   \end{Version}
%   \begin{Version}{2018/01/30 v0.5}
%   \item
%     LuaTeX support added.
%   \end{Version}
% \end{History}
%
% \PrintIndex
%
% \Finale
\endinput
|
% \end{quote}
% Do not forget to quote the argument according to the demands
% of your shell.
%
% \paragraph{Generating the documentation.}
% You can use both the \xfile{.dtx} or the \xfile{.drv} to generate
% the documentation. The process can be configured by the
% configuration file \xfile{ltxdoc.cfg}. For instance, put this
% line into this file, if you want to have A4 as paper format:
% \begin{quote}
%   \verb|\PassOptionsToClass{a4paper}{article}|
% \end{quote}
% An example follows how to generate the
% documentation with pdf\LaTeX:
% \begin{quote}
%\begin{verbatim}
%pdflatex accsupp.dtx
%makeindex -s gind.ist accsupp.idx
%pdflatex accsupp.dtx
%makeindex -s gind.ist accsupp.idx
%pdflatex accsupp.dtx
%\end{verbatim}
% \end{quote}
%
% \section{Catalogue}
%
% The following XML file can be used as source for the
% \href{http://mirror.ctan.org/help/Catalogue/catalogue.html}{\TeX\ Catalogue}.
% The elements \texttt{caption} and \texttt{description} are imported
% from the original XML file from the Catalogue.
% The name of the XML file in the Catalogue is \xfile{accsupp.xml}.
%    \begin{macrocode}
%<*catalogue>
<?xml version='1.0' encoding='us-ascii'?>
<!DOCTYPE entry SYSTEM 'catalogue.dtd'>
<entry datestamp='$Date$' modifier='$Author$' id='accsupp'>
  <name>accsupp</name>
  <caption>Better accessibility support for PDF files.</caption>
  <authorref id='auth:oberdiek'/>
  <copyright owner='Heiko Oberdiek' year='2007,2010'/>
  <license type='lppl1.3'/>
  <version number='0.4'/>
  <description>
    Since PDF 1.5 portions of a page can be marked for better accessibility
    support. For example, replacement texts or expansions of abbreviations can
    be provided. This package starts with providing a minimal low-level
    interface for programmers; its status is experimental.
    <p/>
    The package is part of the <xref refid='oberdiek'>oberdiek</xref> bundle.
  </description>
  <documentation details='Package documentation'
      href='ctan:/macros/latex/contrib/oberdiek/accsupp.pdf'/>
  <ctan file='true' path='/macros/latex/contrib/oberdiek/accsupp.dtx'/>
  <miktex location='oberdiek'/>
  <texlive location='oberdiek'/>
  <install path='/macros/latex/contrib/oberdiek/oberdiek.tds.zip'/>
</entry>
%</catalogue>
%    \end{macrocode}
%
% \begin{thebibliography}{9}
% \bibitem{pdfspec}
% Adobe Systems Incorporated,
% \textit{PDF Reference}, 6th edition, 2006.
% \url{http://www.adobe.com/devnet/acrobat/pdfs/pdf_reference.pdf}
%
% \end{thebibliography}
%
% \begin{History}
%   \begin{Version}{2007/03/21 v0.1}
%   \item
%     First version.
%   \end{Version}
%   \begin{Version}{2007/11/14 v0.2}
%   \item
%     Various bug fixes.
%   \item
%     Catcode section rewritten, test added.
%   \end{Version}
%   \begin{Version}{2010/01/16 v0.3}
%   \item
%     \cs{AccSuppSetup} added.
%   \item
%     Options \xoption{space}, \xoption{spacefont}, \xoption{spacechar} added.
%   \end{Version}
%   \begin{Version}{2016/05/16 v0.4}
%   \item
%     Documentation updates.
%   \end{Version}
%   \begin{Version}{2018/01/30 v0.5}
%   \item
%     LuaTeX support added.
%   \end{Version}
% \end{History}
%
% \PrintIndex
%
% \Finale
\endinput
|
% \end{quote}
% Do not forget to quote the argument according to the demands
% of your shell.
%
% \paragraph{Generating the documentation.}
% You can use both the \xfile{.dtx} or the \xfile{.drv} to generate
% the documentation. The process can be configured by the
% configuration file \xfile{ltxdoc.cfg}. For instance, put this
% line into this file, if you want to have A4 as paper format:
% \begin{quote}
%   \verb|\PassOptionsToClass{a4paper}{article}|
% \end{quote}
% An example follows how to generate the
% documentation with pdf\LaTeX:
% \begin{quote}
%\begin{verbatim}
%pdflatex accsupp.dtx
%makeindex -s gind.ist accsupp.idx
%pdflatex accsupp.dtx
%makeindex -s gind.ist accsupp.idx
%pdflatex accsupp.dtx
%\end{verbatim}
% \end{quote}
%
% \section{Catalogue}
%
% The following XML file can be used as source for the
% \href{http://mirror.ctan.org/help/Catalogue/catalogue.html}{\TeX\ Catalogue}.
% The elements \texttt{caption} and \texttt{description} are imported
% from the original XML file from the Catalogue.
% The name of the XML file in the Catalogue is \xfile{accsupp.xml}.
%    \begin{macrocode}
%<*catalogue>
<?xml version='1.0' encoding='us-ascii'?>
<!DOCTYPE entry SYSTEM 'catalogue.dtd'>
<entry datestamp='$Date$' modifier='$Author$' id='accsupp'>
  <name>accsupp</name>
  <caption>Better accessibility support for PDF files.</caption>
  <authorref id='auth:oberdiek'/>
  <copyright owner='Heiko Oberdiek' year='2007,2010'/>
  <license type='lppl1.3'/>
  <version number='0.4'/>
  <description>
    Since PDF 1.5 portions of a page can be marked for better accessibility
    support. For example, replacement texts or expansions of abbreviations can
    be provided. This package starts with providing a minimal low-level
    interface for programmers; its status is experimental.
    <p/>
    The package is part of the <xref refid='oberdiek'>oberdiek</xref> bundle.
  </description>
  <documentation details='Package documentation'
      href='ctan:/macros/latex/contrib/oberdiek/accsupp.pdf'/>
  <ctan file='true' path='/macros/latex/contrib/oberdiek/accsupp.dtx'/>
  <miktex location='oberdiek'/>
  <texlive location='oberdiek'/>
  <install path='/macros/latex/contrib/oberdiek/oberdiek.tds.zip'/>
</entry>
%</catalogue>
%    \end{macrocode}
%
% \begin{thebibliography}{9}
% \bibitem{pdfspec}
% Adobe Systems Incorporated,
% \textit{PDF Reference}, 6th edition, 2006.
% \url{http://www.adobe.com/devnet/acrobat/pdfs/pdf_reference.pdf}
%
% \end{thebibliography}
%
% \begin{History}
%   \begin{Version}{2007/03/21 v0.1}
%   \item
%     First version.
%   \end{Version}
%   \begin{Version}{2007/11/14 v0.2}
%   \item
%     Various bug fixes.
%   \item
%     Catcode section rewritten, test added.
%   \end{Version}
%   \begin{Version}{2010/01/16 v0.3}
%   \item
%     \cs{AccSuppSetup} added.
%   \item
%     Options \xoption{space}, \xoption{spacefont}, \xoption{spacechar} added.
%   \end{Version}
%   \begin{Version}{2016/05/16 v0.4}
%   \item
%     Documentation updates.
%   \end{Version}
%   \begin{Version}{2018/01/30 v0.5}
%   \item
%     LuaTeX support added.
%   \end{Version}
% \end{History}
%
% \PrintIndex
%
% \Finale
\endinput

%        (quote the arguments according to the demands of your shell)
%
% Documentation:
%    (a) If accsupp.drv is present:
%           latex accsupp.drv
%    (b) Without accsupp.drv:
%           latex accsupp.dtx; ...
%    The class ltxdoc loads the configuration file ltxdoc.cfg
%    if available. Here you can specify further options, e.g.
%    use A4 as paper format:
%       \PassOptionsToClass{a4paper}{article}
%
%    Programm calls to get the documentation (example):
%       pdflatex accsupp.dtx
%       makeindex -s gind.ist accsupp.idx
%       pdflatex accsupp.dtx
%       makeindex -s gind.ist accsupp.idx
%       pdflatex accsupp.dtx
%
% Installation:
%    TDS:tex/latex/oberdiek/accsupp.sty
%    TDS:tex/latex/oberdiek/accsupp-pdftex.def
%    TDS:tex/latex/oberdiek/accsupp-dvipdfm.def
%    TDS:tex/latex/oberdiek/accsupp-dvips.def
%    TDS:doc/latex/oberdiek/accsupp.pdf
%    TDS:doc/latex/oberdiek/accsupp-example1.tex
%    TDS:doc/latex/oberdiek/accsupp-example2.tex
%    TDS:doc/latex/oberdiek/test/accsupp-test1.tex
%    TDS:source/latex/oberdiek/accsupp.dtx
%
%<*ignore>
\begingroup
  \catcode123=1 %
  \catcode125=2 %
  \def\x{LaTeX2e}%
\expandafter\endgroup
\ifcase 0\ifx\install y1\fi\expandafter
         \ifx\csname processbatchFile\endcsname\relax\else1\fi
         \ifx\fmtname\x\else 1\fi\relax
\else\csname fi\endcsname
%</ignore>
%<*install>
\input docstrip.tex
\Msg{************************************************************************}
\Msg{* Installation}
\Msg{* Package: accsupp 2016/05/16 v0.4 Accessibility support by marked content (HO)}
\Msg{************************************************************************}

\keepsilent
\askforoverwritefalse

\let\MetaPrefix\relax
\preamble

This is a generated file.

Project: accsupp
Version: 2016/05/16 v0.4

Copyright (C) 2007, 2010 by
   Heiko Oberdiek <heiko.oberdiek at googlemail.com>

This work may be distributed and/or modified under the
conditions of the LaTeX Project Public License, either
version 1.3c of this license or (at your option) any later
version. This version of this license is in
   http://www.latex-project.org/lppl/lppl-1-3c.txt
and the latest version of this license is in
   http://www.latex-project.org/lppl.txt
and version 1.3 or later is part of all distributions of
LaTeX version 2005/12/01 or later.

This work has the LPPL maintenance status "maintained".

This Current Maintainer of this work is Heiko Oberdiek.

This work consists of the main source file accsupp.dtx
and the derived files
   accsupp.sty, accsupp.pdf, accsupp.ins, accsupp.drv,
   accsupp-pdftex.def, accsupp-dvipdfm.def, accsupp-dvips.def,
   accsupp-example1.tex, accsupp-example2.tex,
   accsupp-test1.tex.

\endpreamble
\let\MetaPrefix\DoubleperCent

\generate{%
  \file{accsupp.ins}{\from{accsupp.dtx}{install}}%
  \file{accsupp.drv}{\from{accsupp.dtx}{driver}}%
  \usedir{tex/latex/oberdiek}%
  \file{accsupp.sty}{\from{accsupp.dtx}{package}}%
  \file{accsupp-pdftex.def}{\from{accsupp.dtx}{pdftex}}%
  \file{accsupp-dvipdfm.def}{\from{accsupp.dtx}{dvipdfm}}%
  \file{accsupp-dvips.def}{\from{accsupp.dtx}{dvips}}%
  \usedir{doc/latex/oberdiek}%
  \file{accsupp-example1.tex}{\from{accsupp.dtx}{example1}}%
  \file{accsupp-example2.tex}{\from{accsupp.dtx}{example2}}%
  \usedir{doc/latex/oberdiek/test}%
  \file{accsupp-test1.tex}{\from{accsupp.dtx}{test1}}%
  \nopreamble
  \nopostamble
  \usedir{source/latex/oberdiek/catalogue}%
  \file{accsupp.xml}{\from{accsupp.dtx}{catalogue}}%
}

\catcode32=13\relax% active space
\let =\space%
\Msg{************************************************************************}
\Msg{*}
\Msg{* To finish the installation you have to move the following}
\Msg{* files into a directory searched by TeX:}
\Msg{*}
\Msg{*     accsupp.sty, accsupp-pdftex.def, accsupp-dvipdfm.def,}
\Msg{*     accsupp-dvips.def}
\Msg{*}
\Msg{* To produce the documentation run the file `accsupp.drv'}
\Msg{* through LaTeX.}
\Msg{*}
\Msg{* Happy TeXing!}
\Msg{*}
\Msg{************************************************************************}

\endbatchfile
%</install>
%<*ignore>
\fi
%</ignore>
%<*driver>
\NeedsTeXFormat{LaTeX2e}
\ProvidesFile{accsupp.drv}%
  [2016/05/16 v0.4 Accessibility support by marked content (HO)]%
\documentclass{ltxdoc}
\usepackage{holtxdoc}[2011/11/22]
\begin{document}
  \DocInput{accsupp.dtx}%
\end{document}
%</driver>
% \fi
%
%
% \CharacterTable
%  {Upper-case    \A\B\C\D\E\F\G\H\I\J\K\L\M\N\O\P\Q\R\S\T\U\V\W\X\Y\Z
%   Lower-case    \a\b\c\d\e\f\g\h\i\j\k\l\m\n\o\p\q\r\s\t\u\v\w\x\y\z
%   Digits        \0\1\2\3\4\5\6\7\8\9
%   Exclamation   \!     Double quote  \"     Hash (number) \#
%   Dollar        \$     Percent       \%     Ampersand     \&
%   Acute accent  \'     Left paren    \(     Right paren   \)
%   Asterisk      \*     Plus          \+     Comma         \,
%   Minus         \-     Point         \.     Solidus       \/
%   Colon         \:     Semicolon     \;     Less than     \<
%   Equals        \=     Greater than  \>     Question mark \?
%   Commercial at \@     Left bracket  \[     Backslash     \\
%   Right bracket \]     Circumflex    \^     Underscore    \_
%   Grave accent  \`     Left brace    \{     Vertical bar  \|
%   Right brace   \}     Tilde         \~}
%
% \GetFileInfo{accsupp.drv}
%
% \title{The \xpackage{accsupp} package}
% \date{2016/05/16 v0.4}
% \author{Heiko Oberdiek\thanks
% {Please report any issues at https://github.com/ho-tex/oberdiek/issues}\\
% \xemail{heiko.oberdiek at googlemail.com}}
%
% \maketitle
%
% \begin{abstract}
% Since PDF 1.5 portions of a page can be marked for better
% accessibility support.
% For example, replacement texts or expansions of abbreviations can be
% provided. Package \xpackage{accsupp} starts with providing a minimal
% low-level interface for programmers. Status is experimental.
% \end{abstract}
%
% \tableofcontents
%
% \section{Documentation}
%
% \subsection{Macros}
%
% Section ``10.8 Accessibility Support'' of the PDF reference
% \cite{pdfspec} lists some features that can be added by
% operators for marked content.
%
% \begin{declcs}{BeginAccSupp} \M{options}
% \end{declcs}
% It puts the operator |BDC| in the page stream:
% \begin{quote}
%   |/Span|\\
%   |<<|\dots|>>| \qquad \textit{\% property dictionary}\\
%   |BDC|
% \end{quote}
% The contents of the dictionary is controlled by \meta{options}.
% See sections \ref{sec:feature} and \ref{sec:method}.
%
% \begin{declcs}{EndAccSupp} \M{options}
% \end{declcs}
% It puts the operator |EMC| in the page stream.
% The only option is \xoption{pdfliteral}, see section \ref{sec:pdftex}.
%
% \bigskip
% \noindent\textbf{Note:}
% The caller is responsible for the placement of \cs{BeginAccSupp}
% and \cs{EndAccSupp} pairs. Especially page breaks are not allowed
% in between.
%
% \subsubsection{Feature options}\label{sec:feature}
%
% The PDF reference \cite{pdfspec} describes and explains the different
% features. The names of the feature options are the same as the
% key names for the property dictionary for operator |BDC|, see
% \cs{BeginAccSupp}.
%
% \begin{description}
% \item[\xoption{ActualText}:]
%   Provides a replacement text, see examples in
%   section \ref{sec:example}.
% \item[\xoption{Alt}:]
%   Provides an alternate description.
% \item[\xoption{E}:]
%   Provides the expansion of an abbreviation or an acronym.
% \item[\xoption{Lang}:]
%   Specifies the language.
% \end{description}
%
% \subsubsection{Input methods}\label{sec:method}
%
% Except for \xoption{Lang} option \xoption{method} controls
% how the argument for \xoption{ActualText}, \xoption{Alt},
% and \xoption{E} are interpreted.
% \begin{description}
% \item[\xoption{method=plain}:]
%   The string is only expanded and written without further
%   treatment. Special characters are not protected, thus this method
%   may result in an invalid PDF file.
% \item[\xoption{mehod=escape}:]
%   The string is expanded and special characters are escaped.
%   The result is a valid PDF string.
% \item[\xoption{method=hex}:]
%   The string is given in hexadecimal notation.
%   Section \ref{sec:example:notparallel} shows an example.
% \item[\xoption{method=pdfstringdef}:]
%   If package \xpackage{hyperref} is loaded, then its \cs{pdfstringdef}
%   is used. This method is slow, but useful if the string contains
%   arbitrary \hologo{TeX} code.
% \item[\xoption{unicode}:]
%   This option is needed, if the string is given as Unicode string
%   (16 bit). Internally it adds the string prefix for Unicode.
%   In case of \xoption{method=pdfstringdef} it passes the option
%   to \cs{hypersetup}.
% \end{description}
%
% \subsection{Workaround, option \xoption{space}}
%
% PDF specification says in ``10.8.3 Replacement Text''
% \begin{quote}
% Just as alternative descriptions can be provided for images and other
% items that do not translate naturally into text (\dots), replacement
% text can be specified for content that does translate into text but
% that is represented in a nonstandard way. These nonstandard representations
% might include, for example, glyphs for ligatures or custom characters, or
% inline graphics corresponding to letters in an illuminated manustript or to
% dropped capitals.
% \end{quote}
% However, the ActualText is ignored in Acrobat Reader
% (until version 9 at least), if the content does not contain glyphs.
% Option \xoption{space} adds such an invisible glyph, a space
% character. The font name can be configured by option \xoption{spacefont},
% the default font is \texttt{phvr8r}. The character slot is given
% by option \xoption{spacechar}, default is \texttt{32}, the usual
% position of the space character.
%
% These options can also be given as package options or in
% macro \cs{AccSuppSetup} that takes a key value list as argument.
% Usually only option \xoption{space} is necessary, if the
% contents does not contain glyphs otherwise. Option \xoption{space}
% is enabled by |space| or |space=true| and disabled by |space=false|.
% It is disabled as default. The option is evaluated by \cs{BeginAccSupp}
% and ignored by \cs{EndAccSupp}.
%
% Note: Even with option \xoption{space} I get sometimes wrong texts
% when cut \& paste from AR7/Linux or AR8/Linux, e.g.
% \begin{quote}
%   |Hello| $\rightarrow$ |Helo|, |Helol|, |Hell|, \dots\\
%   |Hello World| $\rightarrow$ |Helo WorldW|, |Helo World|, \dots
% \end{quote}
% I do not know what Acrobat Reader is doing here, thus feedback and
% insights are welcome.
%
% \subsection{Driver options}
%
% Driver options are package options only.
% The special \hologo{TeX} compilers \hologo{pdfTeX} and \hologo{XeTeX}
% are detected automatically. The default for unrecognized drivers can be
% set by defining \cs{ActualTextDriverDefault}. This can be done
% in the configuration file \xfile{accsupp.cfg}.
%
% \subsubsection{Option \xoption{pdftex}}\label{sec:pdftex}
% Package option \xoption{pdftex} is used for \hologo{pdfTeX} in PDF mode.
% Additionally \cs{BeginAccSupp} and \cs{EndAccSupp} understand
% option \xoption{pdfliteral}. It controls
% the modifier keyword for \cs{pdfliteral}:
% \begin{quote}
%   |pdfliteral=direct| $\Rightarrow$ |\pdfliteral direct{...}|
% \end{quote}
%
% \subsubsection{Option \xoption{dvipdfm}}
%
% Package option \xoption{dvipdfm} and its aliases \xoption{dvipdfmx}
% \xoption{xetex} are used for drivers that support
% \textsf{dvipdfm} specials.
%
% \subsubsection{Option \xoption{dvips}}
%
% Package option \xoption{dvips} and its alias \xoption{dvipsone}
% write pdfmark specials in the output. Unhappily these pdfmark
% operators are ignored by \textsf{ghostscript} (latest tested
% version is 8.54). Perhaps they are recognized by commercial
% distiller applications.
%
% \subsubsection{Turning off page stream compression}
%
% For debugging it is useful to have uncompressed page stream objects.
% This can be done afterwards via \textsf{pdftk}:
% \begin{quote}
%   |pdftk file.pdf output file-uncompressed.pdf uncompress|
% \end{quote}
% Or the PDF file is generated uncompressed:
% \begin{description}
% \item[\hologo{pdfTeX}:] |\pdfcompresslevel=0|
% \item[dvipdfm:] |dvipdfm -z0 ...|
% \item[dvipdfmx:] |dvipdfmx -z0 ...|
% \item[ghostscript/ps2pdf:]
%      |ps2pdf -dCompressPages=false input.ps output.pdf|
% \end{description}
%
% \section{Example}
% \label{sec:example}
%
% \subsection{Example \cs{notparallel}}
% \label{sec:example:notparallel}
%
%    \begin{macrocode}
%<*example1>
%    \end{macrocode}
%    \begin{macrocode}
%<<END
\documentclass{article}
\usepackage{accsupp}[2007/11/14]
\usepackage{centernot}
% U+2226 NOT PARALLEL
% \mathrel{...} prevents page break in between
\newcommand*{\notparallel}{%
  \ensuremath{%
    \mathrel{%
      \BeginAccSupp{method=hex,unicode,ActualText=2226}%
      \centernot{\parallel}%
      \EndAccSupp{}%
    }%
  }%
}
\begin{document}
\begin{equation}
A\notparallel B
\end{equation}
\end{document}
%END
%    \end{macrocode}
%    \begin{macrocode}
%</example1>
%    \end{macrocode}
%
% \subsection{Example with \xoption{pdfstringdef}}
%
%    \begin{macrocode}
%<*example2>
%    \end{macrocode}
%    \begin{macrocode}
%<<END
\documentclass{article}
\usepackage[unicode]{hyperref}
\usepackage{accsupp}[2007/11/14]
\begin{document}
  \begin{equation}
    \BeginAccSupp{
      method=pdfstringdef,
      unicode,
      ActualText={%
        a\texttwosuperior +b\texttwosuperior
        =c\texttwosuperior
      }
    }
    a^2 + b^2 = c^2
    \EndAccSupp{}
  \end{equation}
\end{document}
%END
%    \end{macrocode}
%    \begin{macrocode}
%</example2>
%    \end{macrocode}
%
% \StopEventually{
% }
%
% \section{Implementation}
%
% \subsection{Package}
%
%    \begin{macrocode}
%<*package>
%    \end{macrocode}
%
%    \begin{macrocode}
\begingroup\catcode61\catcode48\catcode32=10\relax%
  \catcode13=5 % ^^M
  \endlinechar=13 %
  \catcode123=1 % {
  \catcode125=2 % }
  \catcode64=11 % @
  \def\x{\endgroup
    \expandafter\edef\csname ACCSUPP@AtEnd\endcsname{%
      \endlinechar=\the\endlinechar\relax
      \catcode13=\the\catcode13\relax
      \catcode32=\the\catcode32\relax
      \catcode35=\the\catcode35\relax
      \catcode61=\the\catcode61\relax
      \catcode64=\the\catcode64\relax
      \catcode123=\the\catcode123\relax
      \catcode125=\the\catcode125\relax
    }%
  }%
\x\catcode61\catcode48\catcode32=10\relax%
\catcode13=5 % ^^M
\endlinechar=13 %
\catcode35=6 % #
\catcode64=11 % @
\catcode123=1 % {
\catcode125=2 % }
\def\TMP@EnsureCode#1#2{%
  \edef\ACCSUPP@AtEnd{%
    \ACCSUPP@AtEnd
    \catcode#1=\the\catcode#1\relax
  }%
  \catcode#1=#2\relax
}
\TMP@EnsureCode{10}{12}% ^^J
\TMP@EnsureCode{33}{12}% !
\TMP@EnsureCode{39}{12}% '
\TMP@EnsureCode{40}{12}% (
\TMP@EnsureCode{41}{12}% )
\TMP@EnsureCode{42}{12}% *
\TMP@EnsureCode{44}{12}% ,
\TMP@EnsureCode{45}{12}% -
\TMP@EnsureCode{46}{12}% .
\TMP@EnsureCode{47}{12}% /
\TMP@EnsureCode{58}{12}% :
\TMP@EnsureCode{60}{12}% <
\TMP@EnsureCode{62}{12}% >
\TMP@EnsureCode{94}{7}% ^ (superscript)
\TMP@EnsureCode{96}{12}% `
\TMP@EnsureCode{254}{12}% ^^fe
\TMP@EnsureCode{255}{12}% ^^ff
\edef\ACCSUPP@AtEnd{\ACCSUPP@AtEnd\noexpand\endinput}
%    \end{macrocode}
%
%    Package identification.
%    \begin{macrocode}
\NeedsTeXFormat{LaTeX2e}
\ProvidesPackage{accsupp}%
  [2016/05/16 v0.4 Accessibility support by marked content (HO)]
%    \end{macrocode}
%
%    \begin{macrocode}
\RequirePackage{pdfescape}[2007/02/25]
\RequirePackage{ifpdf}
\RequirePackage{ifxetex}
\RequirePackage{kvoptions}
%    \end{macrocode}
%    \begin{macrocode}
\SetupKeyvalOptions{%
  family=ACCSUPP,%
  prefix=ACCSUPP@%
}
%    \end{macrocode}
%
% \subsection{Space setup}
%
%    \begin{macrocode}
\DeclareBoolOption{space}
\DeclareStringOption[phvr8r]{spacefont}
\DeclareStringOption[32]{spacechar}
%    \end{macrocode}
%
% \subsection{Driver detection and setup}
%
% \paragraph{Driver declarations.}
%
%    \begin{macrocode}
\def\ACCSUPP@DefineDriverKey{%
  \@dblarg\ACCSUPP@@DefineDriverKey
}
\def\ACCSUPP@@DefineDriverKey[#1]#2{%
  \define@key{ACCSUPP}{#2}[]{%
    \def\ACCSUPP@driver{#1}%
  }%
  \g@addto@macro\ACCSUPP@DisableOptions{%
    \DisableKeyvalOption{ACCSUPP}{#2}%
  }%
}
\let\ACCSUPP@DisableOptions\@empty
\ACCSUPP@DefineDriverKey{pdftex}
\ACCSUPP@DefineDriverKey{dvips}
\ACCSUPP@DefineDriverKey[dvips]{dvipsone}
\ACCSUPP@DefineDriverKey{dvipdfm}
\ACCSUPP@DefineDriverKey[dvipdfm]{dvipdfmx}
\ACCSUPP@DefineDriverKey[dvipdfm]{xetex}
\let\ACCSUPP@driver\relax
\InputIfFileExists{accsupp.cfg}{}{}
\providecommand*{\ActualTextDriverDefault}{dvips}
\ifpdf
  \def\ACCSUPP@driver{pdftex}%
\else
  \ifxetex
    \def\ACCSUPP@driver{dvipdfm}%
  \else
    \ifx\ACCSUPP@driver\relax
      \let\ACCSUPP@driver\ActualTextDriverDefault
    \fi
  \fi
\fi
%    \end{macrocode}
%
% \paragraph{Process options.}
%
%    \begin{macrocode}
\ProcessKeyvalOptions*
\ACCSUPP@DisableOptions
%    \end{macrocode}
%
% \paragraph{Driver validation and loading.}
%
%    \begin{macrocode}
\def\ACCSUPP@temp{pdftex}%
\ifpdf
  \ifx\ACCSUPP@temp\ACCSUPP@driver
  \else
    \PackageWarningNoLine{accsupp}{%
      Wrong driver `\ACCSUPP@driver', using `pdftex' instead%
    }%
    \let\ACCSUPP@driver\ACCSUPP@temp
  \fi
\else
  \ifx\ACCSUPP@temp\ACCSUPP@driver
    \PackageError{accsupp}{%
      Wrong driver, pdfTeX is not running in PDF mode.\MessageBreak
      Package loading is aborted%
    }\@ehc
    \expandafter\expandafter\expandafter\ACCSUPP@AtEnd
  \fi
  \def\ACCSUPP@temp{dvipdfm}%
  \ifxetex
    \ifx\ACCSUPP@temp\ACCSUPP@driver
    \else
      \PackageWarningNoLine{accsupp}{%
        Wrong driver `\ACCSUPP@driver',\MessageBreak
        using `dvipdfm' for XeTeX instead%
      }%
      \let\ACCSUPP@driver\ACCSUPP@temp
    \fi
  \fi
\fi%
\ifx\ACCSUPP@driver\relax
  \PackageError{accsupp}{%
    Missing driver option.\MessageBreak
    Package loading is aborted%
  }\@ehc
  \expandafter\ACCSUPP@AtEnd
\fi%
\InputIfFileExists{accsupp-\ACCSUPP@driver.def}{}{%
  \PackageError{accsupp}{%
    Unsupported driver `\ACCSUPP@driver'.\MessageBreak
    Package loading is aborted%
  }\@ehc
  \ACCSUPP@AtEnd
}%
%    \end{macrocode}
%
% \subsection{Main macro}
%
%    \begin{macrocode}
\DeclareBoolOption{unicode}
\DeclareStringOption[page]{pdfliteral}
\DeclareStringOption{Lang}
\def\ACCSUPP@method{escape}
\define@key{ACCSUPP}{method}{%
  \@ifundefined{ACCSUPP@method@#1}{%
    \PackageError{accsupp}{%
      Ignoring unknown method `#1'%
    }\@ehc
  }{%
    \edef\ACCSUPP@method{#1}%
  }%
}
\let\ACCSUPP@Lang\relax
\def\ACCSUPP@temp#1{%
  \expandafter\ACCSUPP@@temp\csname ACCSUPP@#1\endcsname{#1}%
}
\def\ACCSUPP@@temp#1#2{%
  \let#1\relax
  \define@key{ACCSUPP}{#2}{%
    \def#1{##1}%
    \ifx#1\@empty
      \def#1{()}%
    \else
      \csname ACCSUPP@method@\ACCSUPP@method\endcsname#1%
    \fi
  }%
}
\ACCSUPP@temp{Alt}
\ACCSUPP@temp{ActualText}
\ACCSUPP@temp{E}
%    \end{macrocode}
%
%    \begin{macrocode}
\newcommand*{\BeginAccSupp}[1]{%
  \begingroup
    \setkeys{ACCSUPP}{#1}%
    \edef\ACCSUPP@span{%
      /Span<<%
        \ifx\ACCSUPP@Lang\relax
        \else
          /Lang\ACCSUPP@Lang
        \fi
        \ifx\ACCSUPP@Alt\relax
        \else
          /Alt\ACCSUPP@Alt
        \fi
        \ifx\ACCSUPP@ActualText\relax
        \else
          /ActualText\ACCSUPP@ActualText
        \fi
        \ifx\ACCSUPP@E\relax
        \else
          /E\ACCSUPP@E
        \fi
      >>%
    }%
    \ACCSUPP@bdc
    \ACCSUPP@space
  \endgroup
}
\newcommand*{\EndAccSupp}[1]{%
  \begingroup
    \setkeys{ACCSUPP}{#1}%
    \ACCSUPP@emc
  \endgroup
}
\def\ACCSUPP@space{%
  \ifACCSUPP@space
    \begingroup
      \@ifundefined{ACCSUPP@Font}{%
        \global\font\ACCSUPP@Font=\ACCSUPP@spacefont\relax
      }{}%
      \leavevmode
      \setbox\z@\hbox{\ACCSUPP@Font\char\ACCSUPP@spacechar}%
      \wd\z@\z@
      \ht\z@\z@
      \dp\z@\z@
      \copy\z@
    \endgroup
  \fi
}
\newcommand*{\AccSuppSetup}{%
  \setkeys{ACCSUPP}%
}
%    \end{macrocode}
%
% \subsubsection{Input methods}
%
%    \begin{macrocode}
\def\ACCSUPP@method@plain#1{%
  \csname @safe@activestrue\endcsname
  \edef#1{%
    (%
    \ifACCSUPP@unicode
      \string\376\string\377%
    \fi
    #1%
    )%
  }%
  \@onelevel@sanitize#1%
}
%    \end{macrocode}
%    \begin{macrocode}
\def\ACCSUPP@method@escape#1{%
  \EdefEscapeString#1{%
    \ifACCSUPP@unicode
      ^^fe^^ff%
    \fi
    #1%
  }%
  \edef#1{(#1)}%
}%
%    \end{macrocode}
%    \begin{macrocode}
\def\ACCSUPP@method@hex#1{%
  \edef#1{%
    <%
    \ifACCSUPP@unicode
      FEFF%
    \fi
    #1%
    >%
  }%
}
%    \end{macrocode}
%    \begin{macrocode}
\def\ACCSUPP@method@pdfstringdef#1{%
  \ifACCSUPP@unicode
    \@ifundefined{hypersetup}{}{%
      \hypersetup{unicode}%
    }%
  \fi
  \@ifundefined{pdfstringdef}{%
    \PackageError{accsupp}{%
      Method `pdfstringdef' requires package `hyperref'%
    }\@ehc
    \let\ACCSUPP@temp\@empty
  }{%
    \begingroup
      \setbox0=\hbox{%
        \pdfstringdef\ACCSUPP@temp#1%
        \global\let\ACCSUPP@temp\ACCSUPP@temp
      }%
    \endgroup
  }%
  \edef#1{(\ACCSUPP@temp)}%
}
%    \end{macrocode}
%
%    \begin{macrocode}
\ACCSUPP@AtEnd%
%</package>
%    \end{macrocode}
%
% \subsection{Drivers}
%
% \subsubsection{Driver \xoption{pdftex}}
%
%    \begin{macrocode}
%<*pdftex>
\NeedsTeXFormat{LaTeX2e}
\ProvidesFile{accsupp-pdftex.def}%
  [2016/05/16 v0.4 accsupp driver for pdfTeX (HO)]%
%    \end{macrocode}
%
%    \begin{macrocode}
\def\ACCSUPP@bdc{%
  \pdfliteral\ACCSUPP@pdfliteral{\ACCSUPP@span BDC}%
}
\def\ACCSUPP@emc{%
  \pdfliteral\ACCSUPP@pdfliteral{EMC}%
}
%    \end{macrocode}
%    \begin{macrocode}
%</pdftex>
%    \end{macrocode}
%
% \subsubsection{Driver \xoption{dvipdfm}}
%
%    \begin{macrocode}
%<*dvipdfm>
\NeedsTeXFormat{LaTeX2e}
\ProvidesFile{accsupp-dvipdfm.def}%
  [2016/05/16 v0.4 accsupp driver for dvipdfm (HO)]%
%    \end{macrocode}
%
%    \begin{macrocode}
\def\ACCSUPP@bdc{%
  \special{pdf:content \ACCSUPP@span BDC}%
}
\def\ACCSUPP@emc{%
  \special{pdf:content EMC}%
}
%    \end{macrocode}
%    \begin{macrocode}
%</dvipdfm>
%    \end{macrocode}
%
% \subsubsection{Driver \xoption{dvips}}
%
%    \begin{macrocode}
%<*dvips>
\NeedsTeXFormat{LaTeX2e}
\ProvidesFile{accsupp-dvips.def}%
  [2016/05/16 v0.4 accsupp driver for dvips (HO)]%
%    \end{macrocode}
%
%    \begin{macrocode}
\def\ACCSUPP@bdc{%
  \special{ps:[\ACCSUPP@span/BDC pdfmark}%
}
\def\ACCSUPP@emc{%
  \special{ps:[/EMC pdfmark}%
}
%    \end{macrocode}
%    \begin{macrocode}
%</dvips>
%    \end{macrocode}
%
% \section{Test}
%
% \subsection{Catcode checks for loading}
%
%    \begin{macrocode}
%<*test1>
%    \end{macrocode}
%    \begin{macrocode}
\NeedsTeXFormat{LaTeX2e}
\documentclass{minimal}
\makeatletter
\def\RestoreCatcodes{}
\count@=0 %
\loop
  \edef\RestoreCatcodes{%
    \RestoreCatcodes
    \catcode\the\count@=\the\catcode\count@\relax
  }%
\ifnum\count@<255 %
  \advance\count@\@ne
\repeat

\def\RangeCatcodeInvalid#1#2{%
  \count@=#1\relax
  \loop
    \catcode\count@=15 %
  \ifnum\count@<#2\relax
    \advance\count@\@ne
  \repeat
}
\def\Test{%
  \RangeCatcodeInvalid{0}{47}%
  \RangeCatcodeInvalid{58}{64}%
  \RangeCatcodeInvalid{91}{96}%
  \RangeCatcodeInvalid{123}{127}%
  \catcode`\@=12 %
  \catcode`\\=0 %
  \catcode`\{=1 %
  \catcode`\}=2 %
  \catcode`\#=6 %
  \catcode`\[=12 %
  \catcode`\]=12 %
  \catcode`\%=14 %
  \catcode`\ =10 %
  \catcode13=5 %
  \RequirePackage{accsupp}[2016/05/16]\relax
  \RestoreCatcodes
}
\Test
\csname @@end\endcsname
\end
%    \end{macrocode}
%    \begin{macrocode}
%</test1>
%    \end{macrocode}
%
% \section{Installation}
%
% \subsection{Download}
%
% \paragraph{Package.} This package is available on
% CTAN\footnote{\url{http://ctan.org/pkg/accsupp}}:
% \begin{description}
% \item[\CTAN{macros/latex/contrib/oberdiek/accsupp.dtx}] The source file.
% \item[\CTAN{macros/latex/contrib/oberdiek/accsupp.pdf}] Documentation.
% \end{description}
%
%
% \paragraph{Bundle.} All the packages of the bundle `oberdiek'
% are also available in a TDS compliant ZIP archive. There
% the packages are already unpacked and the documentation files
% are generated. The files and directories obey the TDS standard.
% \begin{description}
% \item[\CTAN{install/macros/latex/contrib/oberdiek.tds.zip}]
% \end{description}
% \emph{TDS} refers to the standard ``A Directory Structure
% for \TeX\ Files'' (\CTAN{tds/tds.pdf}). Directories
% with \xfile{texmf} in their name are usually organized this way.
%
% \subsection{Bundle installation}
%
% \paragraph{Unpacking.} Unpack the \xfile{oberdiek.tds.zip} in the
% TDS tree (also known as \xfile{texmf} tree) of your choice.
% Example (linux):
% \begin{quote}
%   |unzip oberdiek.tds.zip -d ~/texmf|
% \end{quote}
%
% \paragraph{Script installation.}
% Check the directory \xfile{TDS:scripts/oberdiek/} for
% scripts that need further installation steps.
% Package \xpackage{attachfile2} comes with the Perl script
% \xfile{pdfatfi.pl} that should be installed in such a way
% that it can be called as \texttt{pdfatfi}.
% Example (linux):
% \begin{quote}
%   |chmod +x scripts/oberdiek/pdfatfi.pl|\\
%   |cp scripts/oberdiek/pdfatfi.pl /usr/local/bin/|
% \end{quote}
%
% \subsection{Package installation}
%
% \paragraph{Unpacking.} The \xfile{.dtx} file is a self-extracting
% \docstrip\ archive. The files are extracted by running the
% \xfile{.dtx} through \plainTeX:
% \begin{quote}
%   \verb|tex accsupp.dtx|
% \end{quote}
%
% \paragraph{TDS.} Now the different files must be moved into
% the different directories in your installation TDS tree
% (also known as \xfile{texmf} tree):
% \begin{quote}
% \def\t{^^A
% \begin{tabular}{@{}>{\ttfamily}l@{ $\rightarrow$ }>{\ttfamily}l@{}}
%   accsupp.sty & tex/latex/oberdiek/accsupp.sty\\
%   accsupp-pdftex.def & tex/latex/oberdiek/accsupp-pdftex.def\\
%   accsupp-dvipdfm.def & tex/latex/oberdiek/accsupp-dvipdfm.def\\
%   accsupp-dvips.def & tex/latex/oberdiek/accsupp-dvips.def\\
%   accsupp.pdf & doc/latex/oberdiek/accsupp.pdf\\
%   accsupp-example1.tex & doc/latex/oberdiek/accsupp-example1.tex\\
%   accsupp-example2.tex & doc/latex/oberdiek/accsupp-example2.tex\\
%   test/accsupp-test1.tex & doc/latex/oberdiek/test/accsupp-test1.tex\\
%   accsupp.dtx & source/latex/oberdiek/accsupp.dtx\\
% \end{tabular}^^A
% }^^A
% \sbox0{\t}^^A
% \ifdim\wd0>\linewidth
%   \begingroup
%     \advance\linewidth by\leftmargin
%     \advance\linewidth by\rightmargin
%   \edef\x{\endgroup
%     \def\noexpand\lw{\the\linewidth}^^A
%   }\x
%   \def\lwbox{^^A
%     \leavevmode
%     \hbox to \linewidth{^^A
%       \kern-\leftmargin\relax
%       \hss
%       \usebox0
%       \hss
%       \kern-\rightmargin\relax
%     }^^A
%   }^^A
%   \ifdim\wd0>\lw
%     \sbox0{\small\t}^^A
%     \ifdim\wd0>\linewidth
%       \ifdim\wd0>\lw
%         \sbox0{\footnotesize\t}^^A
%         \ifdim\wd0>\linewidth
%           \ifdim\wd0>\lw
%             \sbox0{\scriptsize\t}^^A
%             \ifdim\wd0>\linewidth
%               \ifdim\wd0>\lw
%                 \sbox0{\tiny\t}^^A
%                 \ifdim\wd0>\linewidth
%                   \lwbox
%                 \else
%                   \usebox0
%                 \fi
%               \else
%                 \lwbox
%               \fi
%             \else
%               \usebox0
%             \fi
%           \else
%             \lwbox
%           \fi
%         \else
%           \usebox0
%         \fi
%       \else
%         \lwbox
%       \fi
%     \else
%       \usebox0
%     \fi
%   \else
%     \lwbox
%   \fi
% \else
%   \usebox0
% \fi
% \end{quote}
% If you have a \xfile{docstrip.cfg} that configures and enables \docstrip's
% TDS installing feature, then some files can already be in the right
% place, see the documentation of \docstrip.
%
% \subsection{Refresh file name databases}
%
% If your \TeX~distribution
% (\teTeX, \mikTeX, \dots) relies on file name databases, you must refresh
% these. For example, \teTeX\ users run \verb|texhash| or
% \verb|mktexlsr|.
%
% \subsection{Some details for the interested}
%
% \paragraph{Attached source.}
%
% The PDF documentation on CTAN also includes the
% \xfile{.dtx} source file. It can be extracted by
% AcrobatReader 6 or higher. Another option is \textsf{pdftk},
% e.g. unpack the file into the current directory:
% \begin{quote}
%   \verb|pdftk accsupp.pdf unpack_files output .|
% \end{quote}
%
% \paragraph{Unpacking with \LaTeX.}
% The \xfile{.dtx} chooses its action depending on the format:
% \begin{description}
% \item[\plainTeX:] Run \docstrip\ and extract the files.
% \item[\LaTeX:] Generate the documentation.
% \end{description}
% If you insist on using \LaTeX\ for \docstrip\ (really,
% \docstrip\ does not need \LaTeX), then inform the autodetect routine
% about your intention:
% \begin{quote}
%   \verb|latex \let\install=y% \iffalse meta-comment
%
% File: accsupp.dtx
% Version: 2018/01/30 v0.5
% Info: Accessibility support by marked content
%
% Copyright (C) 2007, 2010 by
%    Heiko Oberdiek <heiko.oberdiek at googlemail.com>
%    2016
%    https://github.com/ho-tex/oberdiek/issues
%
% This work may be distributed and/or modified under the
% conditions of the LaTeX Project Public License, either
% version 1.3c of this license or (at your option) any later
% version. This version of this license is in
%    http://www.latex-project.org/lppl/lppl-1-3c.txt
% and the latest version of this license is in
%    http://www.latex-project.org/lppl.txt
% and version 1.3 or later is part of all distributions of
% LaTeX version 2005/12/01 or later.
%
% This work has the LPPL maintenance status "maintained".
%
% This Current Maintainer of this work is Heiko Oberdiek.
%
% This work consists of the main source file accsupp.dtx
% and the derived files
%    accsupp.sty, accsupp.pdf, accsupp.ins, accsupp.drv,
%    accsupp-pdftex.def, \ccsupp-luatex.def, accsupp-dvipdfm.def,
%    accsupp-dvips.def,
%    accsupp-example1.tex, accsupp-example2.tex,
%    accsupp-test1.tex.
%
% Distribution:
%    CTAN:macros/latex/contrib/oberdiek/accsupp.dtx
%    CTAN:macros/latex/contrib/oberdiek/accsupp.pdf
%
% Unpacking:
%    (a) If accsupp.ins is present:
%           tex accsupp.ins
%    (b) Without accsupp.ins:
%           tex accsupp.dtx
%    (c) If you insist on using LaTeX
%           latex \let\install=y% \iffalse meta-comment
%
% File: accsupp.dtx
% Version: 2018/01/30 v0.5
% Info: Accessibility support by marked content
%
% Copyright (C) 2007, 2010 by
%    Heiko Oberdiek <heiko.oberdiek at googlemail.com>
%    2016
%    https://github.com/ho-tex/oberdiek/issues
%
% This work may be distributed and/or modified under the
% conditions of the LaTeX Project Public License, either
% version 1.3c of this license or (at your option) any later
% version. This version of this license is in
%    http://www.latex-project.org/lppl/lppl-1-3c.txt
% and the latest version of this license is in
%    http://www.latex-project.org/lppl.txt
% and version 1.3 or later is part of all distributions of
% LaTeX version 2005/12/01 or later.
%
% This work has the LPPL maintenance status "maintained".
%
% This Current Maintainer of this work is Heiko Oberdiek.
%
% This work consists of the main source file accsupp.dtx
% and the derived files
%    accsupp.sty, accsupp.pdf, accsupp.ins, accsupp.drv,
%    accsupp-pdftex.def, \ccsupp-luatex.def, accsupp-dvipdfm.def,
%    accsupp-dvips.def,
%    accsupp-example1.tex, accsupp-example2.tex,
%    accsupp-test1.tex.
%
% Distribution:
%    CTAN:macros/latex/contrib/oberdiek/accsupp.dtx
%    CTAN:macros/latex/contrib/oberdiek/accsupp.pdf
%
% Unpacking:
%    (a) If accsupp.ins is present:
%           tex accsupp.ins
%    (b) Without accsupp.ins:
%           tex accsupp.dtx
%    (c) If you insist on using LaTeX
%           latex \let\install=y% \iffalse meta-comment
%
% File: accsupp.dtx
% Version: 2018/01/30 v0.5
% Info: Accessibility support by marked content
%
% Copyright (C) 2007, 2010 by
%    Heiko Oberdiek <heiko.oberdiek at googlemail.com>
%    2016
%    https://github.com/ho-tex/oberdiek/issues
%
% This work may be distributed and/or modified under the
% conditions of the LaTeX Project Public License, either
% version 1.3c of this license or (at your option) any later
% version. This version of this license is in
%    http://www.latex-project.org/lppl/lppl-1-3c.txt
% and the latest version of this license is in
%    http://www.latex-project.org/lppl.txt
% and version 1.3 or later is part of all distributions of
% LaTeX version 2005/12/01 or later.
%
% This work has the LPPL maintenance status "maintained".
%
% This Current Maintainer of this work is Heiko Oberdiek.
%
% This work consists of the main source file accsupp.dtx
% and the derived files
%    accsupp.sty, accsupp.pdf, accsupp.ins, accsupp.drv,
%    accsupp-pdftex.def, \ccsupp-luatex.def, accsupp-dvipdfm.def,
%    accsupp-dvips.def,
%    accsupp-example1.tex, accsupp-example2.tex,
%    accsupp-test1.tex.
%
% Distribution:
%    CTAN:macros/latex/contrib/oberdiek/accsupp.dtx
%    CTAN:macros/latex/contrib/oberdiek/accsupp.pdf
%
% Unpacking:
%    (a) If accsupp.ins is present:
%           tex accsupp.ins
%    (b) Without accsupp.ins:
%           tex accsupp.dtx
%    (c) If you insist on using LaTeX
%           latex \let\install=y\input{accsupp.dtx}
%        (quote the arguments according to the demands of your shell)
%
% Documentation:
%    (a) If accsupp.drv is present:
%           latex accsupp.drv
%    (b) Without accsupp.drv:
%           latex accsupp.dtx; ...
%    The class ltxdoc loads the configuration file ltxdoc.cfg
%    if available. Here you can specify further options, e.g.
%    use A4 as paper format:
%       \PassOptionsToClass{a4paper}{article}
%
%    Programm calls to get the documentation (example):
%       pdflatex accsupp.dtx
%       makeindex -s gind.ist accsupp.idx
%       pdflatex accsupp.dtx
%       makeindex -s gind.ist accsupp.idx
%       pdflatex accsupp.dtx
%
% Installation:
%    TDS:tex/latex/oberdiek/accsupp.sty
%    TDS:tex/latex/oberdiek/accsupp-pdftex.def
%    TDS:tex/latex/oberdiek/accsupp-luatex.def
%    TDS:tex/latex/oberdiek/accsupp-dvipdfm.def
%    TDS:tex/latex/oberdiek/accsupp-dvips.def
%    TDS:doc/latex/oberdiek/accsupp.pdf
%    TDS:doc/latex/oberdiek/accsupp-example1.tex
%    TDS:doc/latex/oberdiek/accsupp-example2.tex
%    TDS:doc/latex/oberdiek/test/accsupp-test1.tex
%    TDS:source/latex/oberdiek/accsupp.dtx
%
%<*ignore>
\begingroup
  \catcode123=1 %
  \catcode125=2 %
  \def\x{LaTeX2e}%
\expandafter\endgroup
\ifcase 0\ifx\install y1\fi\expandafter
         \ifx\csname processbatchFile\endcsname\relax\else1\fi
         \ifx\fmtname\x\else 1\fi\relax
\else\csname fi\endcsname
%</ignore>
%<*install>
\input docstrip.tex
\Msg{************************************************************************}
\Msg{* Installation}
\Msg{* Package: accsupp 2018/01/30 v0.5 Accessibility support by marked content (HO)}
\Msg{************************************************************************}

\keepsilent
\askforoverwritefalse

\let\MetaPrefix\relax
\preamble

This is a generated file.

Project: accsupp
Version: 2018/01/30 v0.5

Copyright (C) 2007, 2010 by
   Heiko Oberdiek <heiko.oberdiek at googlemail.com>

This work may be distributed and/or modified under the
conditions of the LaTeX Project Public License, either
version 1.3c of this license or (at your option) any later
version. This version of this license is in
   http://www.latex-project.org/lppl/lppl-1-3c.txt
and the latest version of this license is in
   http://www.latex-project.org/lppl.txt
and version 1.3 or later is part of all distributions of
LaTeX version 2005/12/01 or later.

This work has the LPPL maintenance status "maintained".

This Current Maintainer of this work is Heiko Oberdiek.

This work consists of the main source file accsupp.dtx
and the derived files
   accsupp.sty, accsupp.pdf, accsupp.ins, accsupp.drv,
   accsupp-pdftex.def, accsupp-luatex.def, accsupp-dvipdfm.def,
   accsupp-dvips.def,
   accsupp-example1.tex, accsupp-example2.tex,
   accsupp-test1.tex.

\endpreamble
\let\MetaPrefix\DoubleperCent

\generate{%
  \file{accsupp.ins}{\from{accsupp.dtx}{install}}%
  \file{accsupp.drv}{\from{accsupp.dtx}{driver}}%
  \usedir{tex/latex/oberdiek}%
  \file{accsupp.sty}{\from{accsupp.dtx}{package}}%
  \file{accsupp-pdftex.def}{\from{accsupp.dtx}{pdftex}}%
  \file{accsupp-luatex.def}{\from{accsupp.dtx}{luatex}}%
  \file{accsupp-dvipdfm.def}{\from{accsupp.dtx}{dvipdfm}}%
  \file{accsupp-dvips.def}{\from{accsupp.dtx}{dvips}}%
  \usedir{doc/latex/oberdiek}%
  \file{accsupp-example1.tex}{\from{accsupp.dtx}{example1}}%
  \file{accsupp-example2.tex}{\from{accsupp.dtx}{example2}}%
  \usedir{doc/latex/oberdiek/test}%
  \file{accsupp-test1.tex}{\from{accsupp.dtx}{test1}}%
  \nopreamble
  \nopostamble
  \usedir{source/latex/oberdiek/catalogue}%
  \file{accsupp.xml}{\from{accsupp.dtx}{catalogue}}%
}

\catcode32=13\relax% active space
\let =\space%
\Msg{************************************************************************}
\Msg{*}
\Msg{* To finish the installation you have to move the following}
\Msg{* files into a directory searched by TeX:}
\Msg{*}
\Msg{*     accsupp.sty, accsupp-pdftex.def, accsupp-dvipdfm.def,}
\Msg{*     accsupp-dvips.def}
\Msg{*}
\Msg{* To produce the documentation run the file `accsupp.drv'}
\Msg{* through LaTeX.}
\Msg{*}
\Msg{* Happy TeXing!}
\Msg{*}
\Msg{************************************************************************}

\endbatchfile
%</install>
%<*ignore>
\fi
%</ignore>
%<*driver>
\NeedsTeXFormat{LaTeX2e}
\ProvidesFile{accsupp.drv}%
  [2018/01/30 v0.5 Accessibility support by marked content (HO)]%
\documentclass{ltxdoc}
\usepackage{holtxdoc}[2011/11/22]
\begin{document}
  \DocInput{accsupp.dtx}%
\end{document}
%</driver>
% \fi
%
%
% \CharacterTable
%  {Upper-case    \A\B\C\D\E\F\G\H\I\J\K\L\M\N\O\P\Q\R\S\T\U\V\W\X\Y\Z
%   Lower-case    \a\b\c\d\e\f\g\h\i\j\k\l\m\n\o\p\q\r\s\t\u\v\w\x\y\z
%   Digits        \0\1\2\3\4\5\6\7\8\9
%   Exclamation   \!     Double quote  \"     Hash (number) \#
%   Dollar        \$     Percent       \%     Ampersand     \&
%   Acute accent  \'     Left paren    \(     Right paren   \)
%   Asterisk      \*     Plus          \+     Comma         \,
%   Minus         \-     Point         \.     Solidus       \/
%   Colon         \:     Semicolon     \;     Less than     \<
%   Equals        \=     Greater than  \>     Question mark \?
%   Commercial at \@     Left bracket  \[     Backslash     \\
%   Right bracket \]     Circumflex    \^     Underscore    \_
%   Grave accent  \`     Left brace    \{     Vertical bar  \|
%   Right brace   \}     Tilde         \~}
%
% \GetFileInfo{accsupp.drv}
%
% \title{The \xpackage{accsupp} package}
% \date{2018/01/30 v0.5}
% \author{Heiko Oberdiek\thanks
% {Please report any issues at https://github.com/ho-tex/oberdiek/issues}\\
% \xemail{heiko.oberdiek at googlemail.com}}
%
% \maketitle
%
% \begin{abstract}
% Since PDF 1.5 portions of a page can be marked for better
% accessibility support.
% For example, replacement texts or expansions of abbreviations can be
% provided. Package \xpackage{accsupp} starts with providing a minimal
% low-level interface for programmers. Status is experimental.
% \end{abstract}
%
% \tableofcontents
%
% \section{Documentation}
%
% \subsection{Macros}
%
% Section ``10.8 Accessibility Support'' of the PDF reference
% \cite{pdfspec} lists some features that can be added by
% operators for marked content.
%
% \begin{declcs}{BeginAccSupp} \M{options}
% \end{declcs}
% It puts the operator |BDC| in the page stream:
% \begin{quote}
%   |/Span|\\
%   |<<|\dots|>>| \qquad \textit{\% property dictionary}\\
%   |BDC|
% \end{quote}
% The contents of the dictionary is controlled by \meta{options}.
% See sections \ref{sec:feature} and \ref{sec:method}.
%
% \begin{declcs}{EndAccSupp} \M{options}
% \end{declcs}
% It puts the operator |EMC| in the page stream.
% The only option is \xoption{pdfliteral}, see section \ref{sec:pdftex}.
%
% \bigskip
% \noindent\textbf{Note:}
% The caller is responsible for the placement of \cs{BeginAccSupp}
% and \cs{EndAccSupp} pairs. Especially page breaks are not allowed
% in between.
%
% \subsubsection{Feature options}\label{sec:feature}
%
% The PDF reference \cite{pdfspec} describes and explains the different
% features. The names of the feature options are the same as the
% key names for the property dictionary for operator |BDC|, see
% \cs{BeginAccSupp}.
%
% \begin{description}
% \item[\xoption{ActualText}:]
%   Provides a replacement text, see examples in
%   section \ref{sec:example}.
% \item[\xoption{Alt}:]
%   Provides an alternate description.
% \item[\xoption{E}:]
%   Provides the expansion of an abbreviation or an acronym.
% \item[\xoption{Lang}:]
%   Specifies the language.
% \end{description}
%
% \subsubsection{Input methods}\label{sec:method}
%
% Except for \xoption{Lang} option \xoption{method} controls
% how the argument for \xoption{ActualText}, \xoption{Alt},
% and \xoption{E} are interpreted.
% \begin{description}
% \item[\xoption{method=plain}:]
%   The string is only expanded and written without further
%   treatment. Special characters are not protected, thus this method
%   may result in an invalid PDF file.
% \item[\xoption{mehod=escape}:]
%   The string is expanded and special characters are escaped.
%   The result is a valid PDF string.
% \item[\xoption{method=hex}:]
%   The string is given in hexadecimal notation.
%   Section \ref{sec:example:notparallel} shows an example.
% \item[\xoption{method=pdfstringdef}:]
%   If package \xpackage{hyperref} is loaded, then its \cs{pdfstringdef}
%   is used. This method is slow, but useful if the string contains
%   arbitrary \hologo{TeX} code.
% \item[\xoption{unicode}:]
%   This option is needed, if the string is given as Unicode string
%   (16 bit). Internally it adds the string prefix for Unicode.
%   In case of \xoption{method=pdfstringdef} it passes the option
%   to \cs{hypersetup}.
% \end{description}
%
% \subsection{Workaround, option \xoption{space}}
%
% PDF specification says in ``10.8.3 Replacement Text''
% \begin{quote}
% Just as alternative descriptions can be provided for images and other
% items that do not translate naturally into text (\dots), replacement
% text can be specified for content that does translate into text but
% that is represented in a nonstandard way. These nonstandard representations
% might include, for example, glyphs for ligatures or custom characters, or
% inline graphics corresponding to letters in an illuminated manustript or to
% dropped capitals.
% \end{quote}
% However, the ActualText is ignored in Acrobat Reader
% (until version 9 at least), if the content does not contain glyphs.
% Option \xoption{space} adds such an invisible glyph, a space
% character. The font name can be configured by option \xoption{spacefont},
% the default font is \texttt{phvr8r}. The character slot is given
% by option \xoption{spacechar}, default is \texttt{32}, the usual
% position of the space character.
%
% These options can also be given as package options or in
% macro \cs{AccSuppSetup} that takes a key value list as argument.
% Usually only option \xoption{space} is necessary, if the
% contents does not contain glyphs otherwise. Option \xoption{space}
% is enabled by |space| or |space=true| and disabled by |space=false|.
% It is disabled as default. The option is evaluated by \cs{BeginAccSupp}
% and ignored by \cs{EndAccSupp}.
%
% Note: Even with option \xoption{space} I get sometimes wrong texts
% when cut \& paste from AR7/Linux or AR8/Linux, e.g.
% \begin{quote}
%   |Hello| $\rightarrow$ |Helo|, |Helol|, |Hell|, \dots\\
%   |Hello World| $\rightarrow$ |Helo WorldW|, |Helo World|, \dots
% \end{quote}
% I do not know what Acrobat Reader is doing here, thus feedback and
% insights are welcome.
%
% \subsection{Driver options}
%
% Driver options are package options only.
% The special \hologo{TeX} compilers \hologo{pdfTeX}, \hologo{LuaTeX} and \hologo{XeTeX}
% are detected automatically. The default for unrecognized drivers can be
% set by defining \cs{ActualTextDriverDefault}. This can be done
% in the configuration file \xfile{accsupp.cfg}.
%
% \subsubsection{Option \xoption{pdftex}}\label{sec:pdftex}
% Package option \xoption{pdftex} is used for \hologo{pdfTeX} in PDF mode.
% Additionally \cs{BeginAccSupp} and \cs{EndAccSupp} understand
% option \xoption{pdfliteral}. It controls
% the modifier keyword for \cs{pdfliteral}:
% \begin{quote}
%   |pdfliteral=direct| $\Rightarrow$ |\pdfliteral direct{...}|
% \end{quote}
%
% \subsubsection{Option \xoption{luatex}}\label{sec:luatex}
% Package option \xoption{luatex} is used for \hologo{LuaTeX} in PDF mode.
% Additionally \cs{BeginAccSupp} and \cs{EndAccSupp} understand
% option \xoption{pdfliteral}. It controls
% the modifier keyword for \cs{pdfextension} literal:
% \begin{quote}
%   |pdfliteral=direct| $\Rightarrow$ |\pdfextension literal direct{...}|
% \end{quote}
%
% \subsubsection{Option \xoption{dvipdfm}}
%
% Package option \xoption{dvipdfm} and its aliases \xoption{dvipdfmx}
% \xoption{xetex} are used for drivers that support
% \textsf{dvipdfm} specials.
%
% \subsubsection{Option \xoption{dvips}}
%
% Package option \xoption{dvips} and its alias \xoption{dvipsone}
% write pdfmark specials in the output. Unhappily these pdfmark
% operators are ignored by \textsf{ghostscript} (latest tested
% version is 8.54). Perhaps they are recognized by commercial
% distiller applications.
%
% \subsubsection{Turning off page stream compression}
%
% For debugging it is useful to have uncompressed page stream objects.
% This can be done afterwards via \textsf{pdftk}:
% \begin{quote}
%   |pdftk file.pdf output file-uncompressed.pdf uncompress|
% \end{quote}
% Or the PDF file is generated uncompressed:
% \begin{description}
% \item[\hologo{pdfTeX}:] |\pdfcompresslevel=0|
% \item[dvipdfm:] |dvipdfm -z0 ...|
% \item[dvipdfmx:] |dvipdfmx -z0 ...|
% \item[ghostscript/ps2pdf:]
%      |ps2pdf -dCompressPages=false input.ps output.pdf|
% \end{description}
%
% \section{Example}
% \label{sec:example}
%
% \subsection{Example \cs{notparallel}}
% \label{sec:example:notparallel}
%
%    \begin{macrocode}
%<*example1>
%    \end{macrocode}
%    \begin{macrocode}
%<<END
\documentclass{article}
\usepackage{accsupp}[2007/11/14]
\usepackage{centernot}
% U+2226 NOT PARALLEL
% \mathrel{...} prevents page break in between
\newcommand*{\notparallel}{%
  \ensuremath{%
    \mathrel{%
      \BeginAccSupp{method=hex,unicode,ActualText=2226}%
      \centernot{\parallel}%
      \EndAccSupp{}%
    }%
  }%
}
\begin{document}
\begin{equation}
A\notparallel B
\end{equation}
\end{document}
%END
%    \end{macrocode}
%    \begin{macrocode}
%</example1>
%    \end{macrocode}
%
% \subsection{Example with \xoption{pdfstringdef}}
%
%    \begin{macrocode}
%<*example2>
%    \end{macrocode}
%    \begin{macrocode}
%<<END
\documentclass{article}
\usepackage[unicode]{hyperref}
\usepackage{accsupp}[2007/11/14]
\begin{document}
  \begin{equation}
    \BeginAccSupp{
      method=pdfstringdef,
      unicode,
      ActualText={%
        a\texttwosuperior +b\texttwosuperior
        =c\texttwosuperior
      }
    }
    a^2 + b^2 = c^2
    \EndAccSupp{}
  \end{equation}
\end{document}
%END
%    \end{macrocode}
%    \begin{macrocode}
%</example2>
%    \end{macrocode}
%
% \StopEventually{
% }
%
% \section{Implementation}
%
% \subsection{Package}
%
%    \begin{macrocode}
%<*package>
%    \end{macrocode}
%
%    \begin{macrocode}
\begingroup\catcode61\catcode48\catcode32=10\relax%
  \catcode13=5 % ^^M
  \endlinechar=13 %
  \catcode123=1 % {
  \catcode125=2 % }
  \catcode64=11 % @
  \def\x{\endgroup
    \expandafter\edef\csname ACCSUPP@AtEnd\endcsname{%
      \endlinechar=\the\endlinechar\relax
      \catcode13=\the\catcode13\relax
      \catcode32=\the\catcode32\relax
      \catcode35=\the\catcode35\relax
      \catcode61=\the\catcode61\relax
      \catcode64=\the\catcode64\relax
      \catcode123=\the\catcode123\relax
      \catcode125=\the\catcode125\relax
    }%
  }%
\x\catcode61\catcode48\catcode32=10\relax%
\catcode13=5 % ^^M
\endlinechar=13 %
\catcode35=6 % #
\catcode64=11 % @
\catcode123=1 % {
\catcode125=2 % }
\def\TMP@EnsureCode#1#2{%
  \edef\ACCSUPP@AtEnd{%
    \ACCSUPP@AtEnd
    \catcode#1=\the\catcode#1\relax
  }%
  \catcode#1=#2\relax
}
\TMP@EnsureCode{10}{12}% ^^J
\TMP@EnsureCode{33}{12}% !
\TMP@EnsureCode{39}{12}% '
\TMP@EnsureCode{40}{12}% (
\TMP@EnsureCode{41}{12}% )
\TMP@EnsureCode{42}{12}% *
\TMP@EnsureCode{44}{12}% ,
\TMP@EnsureCode{45}{12}% -
\TMP@EnsureCode{46}{12}% .
\TMP@EnsureCode{47}{12}% /
\TMP@EnsureCode{58}{12}% :
\TMP@EnsureCode{60}{12}% <
\TMP@EnsureCode{62}{12}% >
\TMP@EnsureCode{94}{7}% ^ (superscript)
\TMP@EnsureCode{96}{12}% `
\TMP@EnsureCode{254}{12}% ^^fe
\TMP@EnsureCode{255}{12}% ^^ff
\edef\ACCSUPP@AtEnd{\ACCSUPP@AtEnd\noexpand\endinput}
%    \end{macrocode}
%
%    Package identification.
%    \begin{macrocode}
\NeedsTeXFormat{LaTeX2e}
\ProvidesPackage{accsupp}%
  [2018/01/30 v0.5 Accessibility support by marked content (HO)]
%    \end{macrocode}
%
%    \begin{macrocode}
\RequirePackage{pdfescape}[2007/02/25]
\RequirePackage{ifpdf}
\RequirePackage{ifxetex}
\RequirePackage{kvoptions}
%    \end{macrocode}
%    \begin{macrocode}
\SetupKeyvalOptions{%
  family=ACCSUPP,%
  prefix=ACCSUPP@%
}
%    \end{macrocode}
%
% \subsection{Space setup}
%
%    \begin{macrocode}
\DeclareBoolOption{space}
\DeclareStringOption[phvr8r]{spacefont}
\DeclareStringOption[32]{spacechar}
%    \end{macrocode}
%
% \subsection{Driver detection and setup}
%
% \paragraph{Driver declarations.}
%
%    \begin{macrocode}
\def\ACCSUPP@DefineDriverKey{%
  \@dblarg\ACCSUPP@@DefineDriverKey
}
\def\ACCSUPP@@DefineDriverKey[#1]#2{%
  \define@key{ACCSUPP}{#2}[]{%
    \def\ACCSUPP@driver{#1}%
  }%
  \g@addto@macro\ACCSUPP@DisableOptions{%
    \DisableKeyvalOption{ACCSUPP}{#2}%
  }%
}
\let\ACCSUPP@DisableOptions\@empty
\ACCSUPP@DefineDriverKey{pdftex}
\ACCSUPP@DefineDriverKey{luatex}
\ACCSUPP@DefineDriverKey{dvips}
\ACCSUPP@DefineDriverKey[dvips]{dvipsone}
\ACCSUPP@DefineDriverKey{dvipdfm}
\ACCSUPP@DefineDriverKey[dvipdfm]{dvipdfmx}
\ACCSUPP@DefineDriverKey[dvipdfm]{xetex}
\let\ACCSUPP@driver\relax
\InputIfFileExists{accsupp.cfg}{}{}
\providecommand*{\ActualTextDriverDefault}{dvips}
\ifpdf
  \ifx\pdfextension\@undefined
    \def\ACCSUPP@driver{pdftex}%
  \else
    \def\ACCSUPP@driver{luatex}%
  \fi
\else
  \ifxetex
    \def\ACCSUPP@driver{dvipdfm}%
  \else
    \ifx\ACCSUPP@driver\relax
      \let\ACCSUPP@driver\ActualTextDriverDefault
    \fi
  \fi
\fi
%    \end{macrocode}
%
% \paragraph{Process options.}
%
%    \begin{macrocode}
\ProcessKeyvalOptions*
\ACCSUPP@DisableOptions
%    \end{macrocode}
%
% \paragraph{Driver validation and loading.}
%
%    \begin{macrocode}
\def\ACCSUPP@temp{pdftex}%
\ifpdf
  \ifx\pdfextension\@undefined\else
    \def\ACCSUPP@temp{luatex}%
  \fi
  \ifx\ACCSUPP@temp\ACCSUPP@driver
  \else
    \PackageWarningNoLine{accsupp}{%
      Wrong driver `\ACCSUPP@driver', using `pdftex' instead%
    }%
ed as
    \let\ACCSUPP@driver\ACCSUPP@temp
  \fi
\else
  \ifx\ACCSUPP@temp\ACCSUPP@driver
    \PackageError{accsupp}{%
      Wrong driver, pdfTeX is not running in PDF mode.\MessageBreak
      Package loading is aborted%
    }\@ehc
    \expandafter\expandafter\expandafter\ACCSUPP@AtEnd
  \fi
  \def\ACCSUPP@temp{dvipdfm}%
  \ifxetex
    \ifx\ACCSUPP@temp\ACCSUPP@driver
    \else
      \PackageWarningNoLine{accsupp}{%
        Wrong driver `\ACCSUPP@driver',\MessageBreak
        using `dvipdfm' for XeTeX instead%
      }%
      \let\ACCSUPP@driver\ACCSUPP@temp
    \fi
  \fi
\fi%
\ifx\ACCSUPP@driver\relax
  \PackageError{accsupp}{%
    Missing driver option.\MessageBreak
    Package loading is aborted%
  }\@ehc
  \expandafter\ACCSUPP@AtEnd
\fi%
\InputIfFileExists{accsupp-\ACCSUPP@driver.def}{}{%
  \PackageError{accsupp}{%
    Unsupported driver `\ACCSUPP@driver'.\MessageBreak
    Package loading is aborted%
  }\@ehc
  \ACCSUPP@AtEnd
}%
%    \end{macrocode}
%
% \subsection{Main macro}
%
%    \begin{macrocode}
\DeclareBoolOption{unicode}
\DeclareStringOption[page]{pdfliteral}
\DeclareStringOption{Lang}
\def\ACCSUPP@method{escape}
\define@key{ACCSUPP}{method}{%
  \@ifundefined{ACCSUPP@method@#1}{%
    \PackageError{accsupp}{%
      Ignoring unknown method `#1'%
    }\@ehc
  }{%
    \edef\ACCSUPP@method{#1}%
  }%
}
\let\ACCSUPP@Lang\relax
\def\ACCSUPP@temp#1{%
  \expandafter\ACCSUPP@@temp\csname ACCSUPP@#1\endcsname{#1}%
}
\def\ACCSUPP@@temp#1#2{%
  \let#1\relax
  \define@key{ACCSUPP}{#2}{%
    \def#1{##1}%
    \ifx#1\@empty
      \def#1{()}%
    \else
      \csname ACCSUPP@method@\ACCSUPP@method\endcsname#1%
    \fi
  }%
}
\ACCSUPP@temp{Alt}
\ACCSUPP@temp{ActualText}
\ACCSUPP@temp{E}
%    \end{macrocode}
%
%    \begin{macrocode}
\newcommand*{\BeginAccSupp}[1]{%
  \begingroup
    \setkeys{ACCSUPP}{#1}%
    \edef\ACCSUPP@span{%
      /Span<<%
        \ifx\ACCSUPP@Lang\relax
        \else
          /Lang\ACCSUPP@Lang
        \fi
        \ifx\ACCSUPP@Alt\relax
        \else
          /Alt\ACCSUPP@Alt
        \fi
        \ifx\ACCSUPP@ActualText\relax
        \else
          /ActualText\ACCSUPP@ActualText
        \fi
        \ifx\ACCSUPP@E\relax
        \else
          /E\ACCSUPP@E
        \fi
      >>%
    }%
    \ACCSUPP@bdc
    \ACCSUPP@space
  \endgroup
}
\newcommand*{\EndAccSupp}[1]{%
  \begingroup
    \setkeys{ACCSUPP}{#1}%
    \ACCSUPP@emc
  \endgroup
}
\def\ACCSUPP@space{%
  \ifACCSUPP@space
    \begingroup
      \@ifundefined{ACCSUPP@Font}{%
        \global\font\ACCSUPP@Font=\ACCSUPP@spacefont\relax
      }{}%
      \leavevmode
      \setbox\z@\hbox{\ACCSUPP@Font\char\ACCSUPP@spacechar}%
      \wd\z@\z@
      \ht\z@\z@
      \dp\z@\z@
      \copy\z@
    \endgroup
  \fi
}
\newcommand*{\AccSuppSetup}{%
  \setkeys{ACCSUPP}%
}
%    \end{macrocode}
%
% \subsubsection{Input methods}
%
%    \begin{macrocode}
\def\ACCSUPP@method@plain#1{%
  \csname @safe@activestrue\endcsname
  \edef#1{%
    (%
    \ifACCSUPP@unicode
      \string\376\string\377%
    \fi
    #1%
    )%
  }%
  \@onelevel@sanitize#1%
}
%    \end{macrocode}
%    \begin{macrocode}
\def\ACCSUPP@method@escape#1{%
  \EdefEscapeString#1{%
    \ifACCSUPP@unicode
      ^^fe^^ff%
    \fi
    #1%
  }%
  \edef#1{(#1)}%
}%
%    \end{macrocode}
%    \begin{macrocode}
\def\ACCSUPP@method@hex#1{%
  \edef#1{%
    <%
    \ifACCSUPP@unicode
      FEFF%
    \fi
    #1%
    >%
  }%
}
%    \end{macrocode}
%    \begin{macrocode}
\def\ACCSUPP@method@pdfstringdef#1{%
  \ifACCSUPP@unicode
    \@ifundefined{hypersetup}{}{%
      \hypersetup{unicode}%
    }%
  \fi
  \@ifundefined{pdfstringdef}{%
    \PackageError{accsupp}{%
      Method `pdfstringdef' requires package `hyperref'%
    }\@ehc
    \let\ACCSUPP@temp\@empty
  }{%
    \begingroup
      \setbox0=\hbox{%
        \pdfstringdef\ACCSUPP@temp#1%
        \global\let\ACCSUPP@temp\ACCSUPP@temp
      }%
    \endgroup
  }%
  \edef#1{(\ACCSUPP@temp)}%
}
%    \end{macrocode}
%
%    \begin{macrocode}
\ACCSUPP@AtEnd%
%</package>
%    \end{macrocode}
%
% \subsection{Drivers}
%
% \subsubsection{Driver \xoption{pdftex}}
%
%    \begin{macrocode}
%<*pdftex>
\NeedsTeXFormat{LaTeX2e}
\ProvidesFile{accsupp-pdftex.def}%
  [2018/01/30 v0.5 accsupp driver for pdfTeX (HO)]%
%    \end{macrocode}
%
%    \begin{macrocode}
\def\ACCSUPP@bdc{%
  \pdfliteral\ACCSUPP@pdfliteral{\ACCSUPP@span BDC}%
}
\def\ACCSUPP@emc{%
  \pdfliteral\ACCSUPP@pdfliteral{EMC}%
}
%    \end{macrocode}
%    \begin{macrocode}
%</pdftex>
%    \end{macrocode}
%
% \subsubsection{Driver \xoption{luatex}}
%
%    \begin{macrocode}
%<*luatex>
\NeedsTeXFormat{LaTeX2e}
\ProvidesFile{accsupp-luatex.def}%
  [2018/01/30 v0.5 accsupp driver for pdfTeX (HO)]%
%    \end{macrocode}
%
%    \begin{macrocode}
\protected\def\ACCSUPP@pdf@literal{\pdfextension literal}
\def\ACCSUPP@bdc{%
  \ACCSUPP@pdf@literal\ACCSUPP@pdfliteral{\ACCSUPP@span BDC}%
}
\def\ACCSUPP@emc{%
  \ACCSUPP@pdf@literal\ACCSUPP@pdfliteral{EMC}%
}
%    \end{macrocode}
%    \begin{macrocode}
%</luatex>
%    \end{macrocode}
%
% \subsubsection{Driver \xoption{dvipdfm}}
%
%    \begin{macrocode}
%<*dvipdfm>
\NeedsTeXFormat{LaTeX2e}
\ProvidesFile{accsupp-dvipdfm.def}%
  [2018/01/30 v0.5 accsupp driver for dvipdfm (HO)]%
%    \end{macrocode}
%
%    \begin{macrocode}
\def\ACCSUPP@bdc{%
  \special{pdf:content \ACCSUPP@span BDC}%
}
\def\ACCSUPP@emc{%
  \special{pdf:content EMC}%
}
%    \end{macrocode}
%    \begin{macrocode}
%</dvipdfm>
%    \end{macrocode}
%
% \subsubsection{Driver \xoption{dvips}}
%
%    \begin{macrocode}
%<*dvips>
\NeedsTeXFormat{LaTeX2e}
\ProvidesFile{accsupp-dvips.def}%
  [2018/01/30 v0.5 accsupp driver for dvips (HO)]%
%    \end{macrocode}
%
%    \begin{macrocode}
\def\ACCSUPP@bdc{%
  \special{ps:[\ACCSUPP@span/BDC pdfmark}%
}
\def\ACCSUPP@emc{%
  \special{ps:[/EMC pdfmark}%
}
%    \end{macrocode}
%    \begin{macrocode}
%</dvips>
%    \end{macrocode}
%
% \section{Test}
%
% \subsection{Catcode checks for loading}
%
%    \begin{macrocode}
%<*test1>
%    \end{macrocode}
%    \begin{macrocode}
\NeedsTeXFormat{LaTeX2e}
\documentclass{minimal}
\makeatletter
\def\RestoreCatcodes{}
\count@=0 %
\loop
  \edef\RestoreCatcodes{%
    \RestoreCatcodes
    \catcode\the\count@=\the\catcode\count@\relax
  }%
\ifnum\count@<255 %
  \advance\count@\@ne
\repeat

\def\RangeCatcodeInvalid#1#2{%
  \count@=#1\relax
  \loop
    \catcode\count@=15 %
  \ifnum\count@<#2\relax
    \advance\count@\@ne
  \repeat
}
\def\Test{%
  \RangeCatcodeInvalid{0}{47}%
  \RangeCatcodeInvalid{58}{64}%
  \RangeCatcodeInvalid{91}{96}%
  \RangeCatcodeInvalid{123}{127}%
  \catcode`\@=12 %
  \catcode`\\=0 %
  \catcode`\{=1 %
  \catcode`\}=2 %
  \catcode`\#=6 %
  \catcode`\[=12 %
  \catcode`\]=12 %
  \catcode`\%=14 %
  \catcode`\ =10 %
  \catcode13=5 %
  \RequirePackage{accsupp}[2018/01/30]\relax
  \RestoreCatcodes
}
\Test
\csname @@end\endcsname
\end
%    \end{macrocode}
%    \begin{macrocode}
%</test1>
%    \end{macrocode}
%
% \section{Installation}
%
% \subsection{Download}
%
% \paragraph{Package.} This package is available on
% CTAN\footnote{\url{http://ctan.org/pkg/accsupp}}:
% \begin{description}
% \item[\CTAN{macros/latex/contrib/oberdiek/accsupp.dtx}] The source file.
% \item[\CTAN{macros/latex/contrib/oberdiek/accsupp.pdf}] Documentation.
% \end{description}
%
%
% \paragraph{Bundle.} All the packages of the bundle `oberdiek'
% are also available in a TDS compliant ZIP archive. There
% the packages are already unpacked and the documentation files
% are generated. The files and directories obey the TDS standard.
% \begin{description}
% \item[\CTAN{install/macros/latex/contrib/oberdiek.tds.zip}]
% \end{description}
% \emph{TDS} refers to the standard ``A Directory Structure
% for \TeX\ Files'' (\CTAN{tds/tds.pdf}). Directories
% with \xfile{texmf} in their name are usually organized this way.
%
% \subsection{Bundle installation}
%
% \paragraph{Unpacking.} Unpack the \xfile{oberdiek.tds.zip} in the
% TDS tree (also known as \xfile{texmf} tree) of your choice.
% Example (linux):
% \begin{quote}
%   |unzip oberdiek.tds.zip -d ~/texmf|
% \end{quote}
%
% \paragraph{Script installation.}
% Check the directory \xfile{TDS:scripts/oberdiek/} for
% scripts that need further installation steps.
% Package \xpackage{attachfile2} comes with the Perl script
% \xfile{pdfatfi.pl} that should be installed in such a way
% that it can be called as \texttt{pdfatfi}.
% Example (linux):
% \begin{quote}
%   |chmod +x scripts/oberdiek/pdfatfi.pl|\\
%   |cp scripts/oberdiek/pdfatfi.pl /usr/local/bin/|
% \end{quote}
%
% \subsection{Package installation}
%
% \paragraph{Unpacking.} The \xfile{.dtx} file is a self-extracting
% \docstrip\ archive. The files are extracted by running the
% \xfile{.dtx} through \plainTeX:
% \begin{quote}
%   \verb|tex accsupp.dtx|
% \end{quote}
%
% \paragraph{TDS.} Now the different files must be moved into
% the different directories in your installation TDS tree
% (also known as \xfile{texmf} tree):
% \begin{quote}
% \def\t{^^A
% \begin{tabular}{@{}>{\ttfamily}l@{ $\rightarrow$ }>{\ttfamily}l@{}}
%   accsupp.sty & tex/latex/oberdiek/accsupp.sty\\
%   accsupp-pdftex.def & tex/latex/oberdiek/accsupp-pdftex.def\\
%   accsupp-luatex.def & tex/latex/oberdiek/accsupp-luatex.def\\
%   accsupp-dvipdfm.def & tex/latex/oberdiek/accsupp-dvipdfm.def\\
%   accsupp-dvips.def & tex/latex/oberdiek/accsupp-dvips.def\\
%   accsupp.pdf & doc/latex/oberdiek/accsupp.pdf\\
%   accsupp-example1.tex & doc/latex/oberdiek/accsupp-example1.tex\\
%   accsupp-example2.tex & doc/latex/oberdiek/accsupp-example2.tex\\
%   test/accsupp-test1.tex & doc/latex/oberdiek/test/accsupp-test1.tex\\
%   accsupp.dtx & source/latex/oberdiek/accsupp.dtx\\
% \end{tabular}^^A
% }^^A
% \sbox0{\t}^^A
% \ifdim\wd0>\linewidth
%   \begingroup
%     \advance\linewidth by\leftmargin
%     \advance\linewidth by\rightmargin
%   \edef\x{\endgroup
%     \def\noexpand\lw{\the\linewidth}^^A
%   }\x
%   \def\lwbox{^^A
%     \leavevmode
%     \hbox to \linewidth{^^A
%       \kern-\leftmargin\relax
%       \hss
%       \usebox0
%       \hss
%       \kern-\rightmargin\relax
%     }^^A
%   }^^A
%   \ifdim\wd0>\lw
%     \sbox0{\small\t}^^A
%     \ifdim\wd0>\linewidth
%       \ifdim\wd0>\lw
%         \sbox0{\footnotesize\t}^^A
%         \ifdim\wd0>\linewidth
%           \ifdim\wd0>\lw
%             \sbox0{\scriptsize\t}^^A
%             \ifdim\wd0>\linewidth
%               \ifdim\wd0>\lw
%                 \sbox0{\tiny\t}^^A
%                 \ifdim\wd0>\linewidth
%                   \lwbox
%                 \else
%                   \usebox0
%                 \fi
%               \else
%                 \lwbox
%               \fi
%             \else
%               \usebox0
%             \fi
%           \else
%             \lwbox
%           \fi
%         \else
%           \usebox0
%         \fi
%       \else
%         \lwbox
%       \fi
%     \else
%       \usebox0
%     \fi
%   \else
%     \lwbox
%   \fi
% \else
%   \usebox0
% \fi
% \end{quote}
% If you have a \xfile{docstrip.cfg} that configures and enables \docstrip's
% TDS installing feature, then some files can already be in the right
% place, see the documentation of \docstrip.
%
% \subsection{Refresh file name databases}
%
% If your \TeX~distribution
% (\teTeX, \mikTeX, \dots) relies on file name databases, you must refresh
% these. For example, \teTeX\ users run \verb|texhash| or
% \verb|mktexlsr|.
%
% \subsection{Some details for the interested}
%
% \paragraph{Attached source.}
%
% The PDF documentation on CTAN also includes the
% \xfile{.dtx} source file. It can be extracted by
% AcrobatReader 6 or higher. Another option is \textsf{pdftk},
% e.g. unpack the file into the current directory:
% \begin{quote}
%   \verb|pdftk accsupp.pdf unpack_files output .|
% \end{quote}
%
% \paragraph{Unpacking with \LaTeX.}
% The \xfile{.dtx} chooses its action depending on the format:
% \begin{description}
% \item[\plainTeX:] Run \docstrip\ and extract the files.
% \item[\LaTeX:] Generate the documentation.
% \end{description}
% If you insist on using \LaTeX\ for \docstrip\ (really,
% \docstrip\ does not need \LaTeX), then inform the autodetect routine
% about your intention:
% \begin{quote}
%   \verb|latex \let\install=y\input{accsupp.dtx}|
% \end{quote}
% Do not forget to quote the argument according to the demands
% of your shell.
%
% \paragraph{Generating the documentation.}
% You can use both the \xfile{.dtx} or the \xfile{.drv} to generate
% the documentation. The process can be configured by the
% configuration file \xfile{ltxdoc.cfg}. For instance, put this
% line into this file, if you want to have A4 as paper format:
% \begin{quote}
%   \verb|\PassOptionsToClass{a4paper}{article}|
% \end{quote}
% An example follows how to generate the
% documentation with pdf\LaTeX:
% \begin{quote}
%\begin{verbatim}
%pdflatex accsupp.dtx
%makeindex -s gind.ist accsupp.idx
%pdflatex accsupp.dtx
%makeindex -s gind.ist accsupp.idx
%pdflatex accsupp.dtx
%\end{verbatim}
% \end{quote}
%
% \section{Catalogue}
%
% The following XML file can be used as source for the
% \href{http://mirror.ctan.org/help/Catalogue/catalogue.html}{\TeX\ Catalogue}.
% The elements \texttt{caption} and \texttt{description} are imported
% from the original XML file from the Catalogue.
% The name of the XML file in the Catalogue is \xfile{accsupp.xml}.
%    \begin{macrocode}
%<*catalogue>
<?xml version='1.0' encoding='us-ascii'?>
<!DOCTYPE entry SYSTEM 'catalogue.dtd'>
<entry datestamp='$Date$' modifier='$Author$' id='accsupp'>
  <name>accsupp</name>
  <caption>Better accessibility support for PDF files.</caption>
  <authorref id='auth:oberdiek'/>
  <copyright owner='Heiko Oberdiek' year='2007,2010'/>
  <license type='lppl1.3'/>
  <version number='0.4'/>
  <description>
    Since PDF 1.5 portions of a page can be marked for better accessibility
    support. For example, replacement texts or expansions of abbreviations can
    be provided. This package starts with providing a minimal low-level
    interface for programmers; its status is experimental.
    <p/>
    The package is part of the <xref refid='oberdiek'>oberdiek</xref> bundle.
  </description>
  <documentation details='Package documentation'
      href='ctan:/macros/latex/contrib/oberdiek/accsupp.pdf'/>
  <ctan file='true' path='/macros/latex/contrib/oberdiek/accsupp.dtx'/>
  <miktex location='oberdiek'/>
  <texlive location='oberdiek'/>
  <install path='/macros/latex/contrib/oberdiek/oberdiek.tds.zip'/>
</entry>
%</catalogue>
%    \end{macrocode}
%
% \begin{thebibliography}{9}
% \bibitem{pdfspec}
% Adobe Systems Incorporated,
% \textit{PDF Reference}, 6th edition, 2006.
% \url{http://www.adobe.com/devnet/acrobat/pdfs/pdf_reference.pdf}
%
% \end{thebibliography}
%
% \begin{History}
%   \begin{Version}{2007/03/21 v0.1}
%   \item
%     First version.
%   \end{Version}
%   \begin{Version}{2007/11/14 v0.2}
%   \item
%     Various bug fixes.
%   \item
%     Catcode section rewritten, test added.
%   \end{Version}
%   \begin{Version}{2010/01/16 v0.3}
%   \item
%     \cs{AccSuppSetup} added.
%   \item
%     Options \xoption{space}, \xoption{spacefont}, \xoption{spacechar} added.
%   \end{Version}
%   \begin{Version}{2016/05/16 v0.4}
%   \item
%     Documentation updates.
%   \end{Version}
%   \begin{Version}{2018/01/30 v0.5}
%   \item
%     LuaTeX support added.
%   \end{Version}
% \end{History}
%
% \PrintIndex
%
% \Finale
\endinput

%        (quote the arguments according to the demands of your shell)
%
% Documentation:
%    (a) If accsupp.drv is present:
%           latex accsupp.drv
%    (b) Without accsupp.drv:
%           latex accsupp.dtx; ...
%    The class ltxdoc loads the configuration file ltxdoc.cfg
%    if available. Here you can specify further options, e.g.
%    use A4 as paper format:
%       \PassOptionsToClass{a4paper}{article}
%
%    Programm calls to get the documentation (example):
%       pdflatex accsupp.dtx
%       makeindex -s gind.ist accsupp.idx
%       pdflatex accsupp.dtx
%       makeindex -s gind.ist accsupp.idx
%       pdflatex accsupp.dtx
%
% Installation:
%    TDS:tex/latex/oberdiek/accsupp.sty
%    TDS:tex/latex/oberdiek/accsupp-pdftex.def
%    TDS:tex/latex/oberdiek/accsupp-luatex.def
%    TDS:tex/latex/oberdiek/accsupp-dvipdfm.def
%    TDS:tex/latex/oberdiek/accsupp-dvips.def
%    TDS:doc/latex/oberdiek/accsupp.pdf
%    TDS:doc/latex/oberdiek/accsupp-example1.tex
%    TDS:doc/latex/oberdiek/accsupp-example2.tex
%    TDS:doc/latex/oberdiek/test/accsupp-test1.tex
%    TDS:source/latex/oberdiek/accsupp.dtx
%
%<*ignore>
\begingroup
  \catcode123=1 %
  \catcode125=2 %
  \def\x{LaTeX2e}%
\expandafter\endgroup
\ifcase 0\ifx\install y1\fi\expandafter
         \ifx\csname processbatchFile\endcsname\relax\else1\fi
         \ifx\fmtname\x\else 1\fi\relax
\else\csname fi\endcsname
%</ignore>
%<*install>
\input docstrip.tex
\Msg{************************************************************************}
\Msg{* Installation}
\Msg{* Package: accsupp 2018/01/30 v0.5 Accessibility support by marked content (HO)}
\Msg{************************************************************************}

\keepsilent
\askforoverwritefalse

\let\MetaPrefix\relax
\preamble

This is a generated file.

Project: accsupp
Version: 2018/01/30 v0.5

Copyright (C) 2007, 2010 by
   Heiko Oberdiek <heiko.oberdiek at googlemail.com>

This work may be distributed and/or modified under the
conditions of the LaTeX Project Public License, either
version 1.3c of this license or (at your option) any later
version. This version of this license is in
   http://www.latex-project.org/lppl/lppl-1-3c.txt
and the latest version of this license is in
   http://www.latex-project.org/lppl.txt
and version 1.3 or later is part of all distributions of
LaTeX version 2005/12/01 or later.

This work has the LPPL maintenance status "maintained".

This Current Maintainer of this work is Heiko Oberdiek.

This work consists of the main source file accsupp.dtx
and the derived files
   accsupp.sty, accsupp.pdf, accsupp.ins, accsupp.drv,
   accsupp-pdftex.def, accsupp-luatex.def, accsupp-dvipdfm.def,
   accsupp-dvips.def,
   accsupp-example1.tex, accsupp-example2.tex,
   accsupp-test1.tex.

\endpreamble
\let\MetaPrefix\DoubleperCent

\generate{%
  \file{accsupp.ins}{\from{accsupp.dtx}{install}}%
  \file{accsupp.drv}{\from{accsupp.dtx}{driver}}%
  \usedir{tex/latex/oberdiek}%
  \file{accsupp.sty}{\from{accsupp.dtx}{package}}%
  \file{accsupp-pdftex.def}{\from{accsupp.dtx}{pdftex}}%
  \file{accsupp-luatex.def}{\from{accsupp.dtx}{luatex}}%
  \file{accsupp-dvipdfm.def}{\from{accsupp.dtx}{dvipdfm}}%
  \file{accsupp-dvips.def}{\from{accsupp.dtx}{dvips}}%
  \usedir{doc/latex/oberdiek}%
  \file{accsupp-example1.tex}{\from{accsupp.dtx}{example1}}%
  \file{accsupp-example2.tex}{\from{accsupp.dtx}{example2}}%
  \usedir{doc/latex/oberdiek/test}%
  \file{accsupp-test1.tex}{\from{accsupp.dtx}{test1}}%
  \nopreamble
  \nopostamble
  \usedir{source/latex/oberdiek/catalogue}%
  \file{accsupp.xml}{\from{accsupp.dtx}{catalogue}}%
}

\catcode32=13\relax% active space
\let =\space%
\Msg{************************************************************************}
\Msg{*}
\Msg{* To finish the installation you have to move the following}
\Msg{* files into a directory searched by TeX:}
\Msg{*}
\Msg{*     accsupp.sty, accsupp-pdftex.def, accsupp-dvipdfm.def,}
\Msg{*     accsupp-dvips.def}
\Msg{*}
\Msg{* To produce the documentation run the file `accsupp.drv'}
\Msg{* through LaTeX.}
\Msg{*}
\Msg{* Happy TeXing!}
\Msg{*}
\Msg{************************************************************************}

\endbatchfile
%</install>
%<*ignore>
\fi
%</ignore>
%<*driver>
\NeedsTeXFormat{LaTeX2e}
\ProvidesFile{accsupp.drv}%
  [2018/01/30 v0.5 Accessibility support by marked content (HO)]%
\documentclass{ltxdoc}
\usepackage{holtxdoc}[2011/11/22]
\begin{document}
  \DocInput{accsupp.dtx}%
\end{document}
%</driver>
% \fi
%
%
% \CharacterTable
%  {Upper-case    \A\B\C\D\E\F\G\H\I\J\K\L\M\N\O\P\Q\R\S\T\U\V\W\X\Y\Z
%   Lower-case    \a\b\c\d\e\f\g\h\i\j\k\l\m\n\o\p\q\r\s\t\u\v\w\x\y\z
%   Digits        \0\1\2\3\4\5\6\7\8\9
%   Exclamation   \!     Double quote  \"     Hash (number) \#
%   Dollar        \$     Percent       \%     Ampersand     \&
%   Acute accent  \'     Left paren    \(     Right paren   \)
%   Asterisk      \*     Plus          \+     Comma         \,
%   Minus         \-     Point         \.     Solidus       \/
%   Colon         \:     Semicolon     \;     Less than     \<
%   Equals        \=     Greater than  \>     Question mark \?
%   Commercial at \@     Left bracket  \[     Backslash     \\
%   Right bracket \]     Circumflex    \^     Underscore    \_
%   Grave accent  \`     Left brace    \{     Vertical bar  \|
%   Right brace   \}     Tilde         \~}
%
% \GetFileInfo{accsupp.drv}
%
% \title{The \xpackage{accsupp} package}
% \date{2018/01/30 v0.5}
% \author{Heiko Oberdiek\thanks
% {Please report any issues at https://github.com/ho-tex/oberdiek/issues}\\
% \xemail{heiko.oberdiek at googlemail.com}}
%
% \maketitle
%
% \begin{abstract}
% Since PDF 1.5 portions of a page can be marked for better
% accessibility support.
% For example, replacement texts or expansions of abbreviations can be
% provided. Package \xpackage{accsupp} starts with providing a minimal
% low-level interface for programmers. Status is experimental.
% \end{abstract}
%
% \tableofcontents
%
% \section{Documentation}
%
% \subsection{Macros}
%
% Section ``10.8 Accessibility Support'' of the PDF reference
% \cite{pdfspec} lists some features that can be added by
% operators for marked content.
%
% \begin{declcs}{BeginAccSupp} \M{options}
% \end{declcs}
% It puts the operator |BDC| in the page stream:
% \begin{quote}
%   |/Span|\\
%   |<<|\dots|>>| \qquad \textit{\% property dictionary}\\
%   |BDC|
% \end{quote}
% The contents of the dictionary is controlled by \meta{options}.
% See sections \ref{sec:feature} and \ref{sec:method}.
%
% \begin{declcs}{EndAccSupp} \M{options}
% \end{declcs}
% It puts the operator |EMC| in the page stream.
% The only option is \xoption{pdfliteral}, see section \ref{sec:pdftex}.
%
% \bigskip
% \noindent\textbf{Note:}
% The caller is responsible for the placement of \cs{BeginAccSupp}
% and \cs{EndAccSupp} pairs. Especially page breaks are not allowed
% in between.
%
% \subsubsection{Feature options}\label{sec:feature}
%
% The PDF reference \cite{pdfspec} describes and explains the different
% features. The names of the feature options are the same as the
% key names for the property dictionary for operator |BDC|, see
% \cs{BeginAccSupp}.
%
% \begin{description}
% \item[\xoption{ActualText}:]
%   Provides a replacement text, see examples in
%   section \ref{sec:example}.
% \item[\xoption{Alt}:]
%   Provides an alternate description.
% \item[\xoption{E}:]
%   Provides the expansion of an abbreviation or an acronym.
% \item[\xoption{Lang}:]
%   Specifies the language.
% \end{description}
%
% \subsubsection{Input methods}\label{sec:method}
%
% Except for \xoption{Lang} option \xoption{method} controls
% how the argument for \xoption{ActualText}, \xoption{Alt},
% and \xoption{E} are interpreted.
% \begin{description}
% \item[\xoption{method=plain}:]
%   The string is only expanded and written without further
%   treatment. Special characters are not protected, thus this method
%   may result in an invalid PDF file.
% \item[\xoption{mehod=escape}:]
%   The string is expanded and special characters are escaped.
%   The result is a valid PDF string.
% \item[\xoption{method=hex}:]
%   The string is given in hexadecimal notation.
%   Section \ref{sec:example:notparallel} shows an example.
% \item[\xoption{method=pdfstringdef}:]
%   If package \xpackage{hyperref} is loaded, then its \cs{pdfstringdef}
%   is used. This method is slow, but useful if the string contains
%   arbitrary \hologo{TeX} code.
% \item[\xoption{unicode}:]
%   This option is needed, if the string is given as Unicode string
%   (16 bit). Internally it adds the string prefix for Unicode.
%   In case of \xoption{method=pdfstringdef} it passes the option
%   to \cs{hypersetup}.
% \end{description}
%
% \subsection{Workaround, option \xoption{space}}
%
% PDF specification says in ``10.8.3 Replacement Text''
% \begin{quote}
% Just as alternative descriptions can be provided for images and other
% items that do not translate naturally into text (\dots), replacement
% text can be specified for content that does translate into text but
% that is represented in a nonstandard way. These nonstandard representations
% might include, for example, glyphs for ligatures or custom characters, or
% inline graphics corresponding to letters in an illuminated manustript or to
% dropped capitals.
% \end{quote}
% However, the ActualText is ignored in Acrobat Reader
% (until version 9 at least), if the content does not contain glyphs.
% Option \xoption{space} adds such an invisible glyph, a space
% character. The font name can be configured by option \xoption{spacefont},
% the default font is \texttt{phvr8r}. The character slot is given
% by option \xoption{spacechar}, default is \texttt{32}, the usual
% position of the space character.
%
% These options can also be given as package options or in
% macro \cs{AccSuppSetup} that takes a key value list as argument.
% Usually only option \xoption{space} is necessary, if the
% contents does not contain glyphs otherwise. Option \xoption{space}
% is enabled by |space| or |space=true| and disabled by |space=false|.
% It is disabled as default. The option is evaluated by \cs{BeginAccSupp}
% and ignored by \cs{EndAccSupp}.
%
% Note: Even with option \xoption{space} I get sometimes wrong texts
% when cut \& paste from AR7/Linux or AR8/Linux, e.g.
% \begin{quote}
%   |Hello| $\rightarrow$ |Helo|, |Helol|, |Hell|, \dots\\
%   |Hello World| $\rightarrow$ |Helo WorldW|, |Helo World|, \dots
% \end{quote}
% I do not know what Acrobat Reader is doing here, thus feedback and
% insights are welcome.
%
% \subsection{Driver options}
%
% Driver options are package options only.
% The special \hologo{TeX} compilers \hologo{pdfTeX}, \hologo{LuaTeX} and \hologo{XeTeX}
% are detected automatically. The default for unrecognized drivers can be
% set by defining \cs{ActualTextDriverDefault}. This can be done
% in the configuration file \xfile{accsupp.cfg}.
%
% \subsubsection{Option \xoption{pdftex}}\label{sec:pdftex}
% Package option \xoption{pdftex} is used for \hologo{pdfTeX} in PDF mode.
% Additionally \cs{BeginAccSupp} and \cs{EndAccSupp} understand
% option \xoption{pdfliteral}. It controls
% the modifier keyword for \cs{pdfliteral}:
% \begin{quote}
%   |pdfliteral=direct| $\Rightarrow$ |\pdfliteral direct{...}|
% \end{quote}
%
% \subsubsection{Option \xoption{luatex}}\label{sec:luatex}
% Package option \xoption{luatex} is used for \hologo{LuaTeX} in PDF mode.
% Additionally \cs{BeginAccSupp} and \cs{EndAccSupp} understand
% option \xoption{pdfliteral}. It controls
% the modifier keyword for \cs{pdfextension} literal:
% \begin{quote}
%   |pdfliteral=direct| $\Rightarrow$ |\pdfextension literal direct{...}|
% \end{quote}
%
% \subsubsection{Option \xoption{dvipdfm}}
%
% Package option \xoption{dvipdfm} and its aliases \xoption{dvipdfmx}
% \xoption{xetex} are used for drivers that support
% \textsf{dvipdfm} specials.
%
% \subsubsection{Option \xoption{dvips}}
%
% Package option \xoption{dvips} and its alias \xoption{dvipsone}
% write pdfmark specials in the output. Unhappily these pdfmark
% operators are ignored by \textsf{ghostscript} (latest tested
% version is 8.54). Perhaps they are recognized by commercial
% distiller applications.
%
% \subsubsection{Turning off page stream compression}
%
% For debugging it is useful to have uncompressed page stream objects.
% This can be done afterwards via \textsf{pdftk}:
% \begin{quote}
%   |pdftk file.pdf output file-uncompressed.pdf uncompress|
% \end{quote}
% Or the PDF file is generated uncompressed:
% \begin{description}
% \item[\hologo{pdfTeX}:] |\pdfcompresslevel=0|
% \item[dvipdfm:] |dvipdfm -z0 ...|
% \item[dvipdfmx:] |dvipdfmx -z0 ...|
% \item[ghostscript/ps2pdf:]
%      |ps2pdf -dCompressPages=false input.ps output.pdf|
% \end{description}
%
% \section{Example}
% \label{sec:example}
%
% \subsection{Example \cs{notparallel}}
% \label{sec:example:notparallel}
%
%    \begin{macrocode}
%<*example1>
%    \end{macrocode}
%    \begin{macrocode}
%<<END
\documentclass{article}
\usepackage{accsupp}[2007/11/14]
\usepackage{centernot}
% U+2226 NOT PARALLEL
% \mathrel{...} prevents page break in between
\newcommand*{\notparallel}{%
  \ensuremath{%
    \mathrel{%
      \BeginAccSupp{method=hex,unicode,ActualText=2226}%
      \centernot{\parallel}%
      \EndAccSupp{}%
    }%
  }%
}
\begin{document}
\begin{equation}
A\notparallel B
\end{equation}
\end{document}
%END
%    \end{macrocode}
%    \begin{macrocode}
%</example1>
%    \end{macrocode}
%
% \subsection{Example with \xoption{pdfstringdef}}
%
%    \begin{macrocode}
%<*example2>
%    \end{macrocode}
%    \begin{macrocode}
%<<END
\documentclass{article}
\usepackage[unicode]{hyperref}
\usepackage{accsupp}[2007/11/14]
\begin{document}
  \begin{equation}
    \BeginAccSupp{
      method=pdfstringdef,
      unicode,
      ActualText={%
        a\texttwosuperior +b\texttwosuperior
        =c\texttwosuperior
      }
    }
    a^2 + b^2 = c^2
    \EndAccSupp{}
  \end{equation}
\end{document}
%END
%    \end{macrocode}
%    \begin{macrocode}
%</example2>
%    \end{macrocode}
%
% \StopEventually{
% }
%
% \section{Implementation}
%
% \subsection{Package}
%
%    \begin{macrocode}
%<*package>
%    \end{macrocode}
%
%    \begin{macrocode}
\begingroup\catcode61\catcode48\catcode32=10\relax%
  \catcode13=5 % ^^M
  \endlinechar=13 %
  \catcode123=1 % {
  \catcode125=2 % }
  \catcode64=11 % @
  \def\x{\endgroup
    \expandafter\edef\csname ACCSUPP@AtEnd\endcsname{%
      \endlinechar=\the\endlinechar\relax
      \catcode13=\the\catcode13\relax
      \catcode32=\the\catcode32\relax
      \catcode35=\the\catcode35\relax
      \catcode61=\the\catcode61\relax
      \catcode64=\the\catcode64\relax
      \catcode123=\the\catcode123\relax
      \catcode125=\the\catcode125\relax
    }%
  }%
\x\catcode61\catcode48\catcode32=10\relax%
\catcode13=5 % ^^M
\endlinechar=13 %
\catcode35=6 % #
\catcode64=11 % @
\catcode123=1 % {
\catcode125=2 % }
\def\TMP@EnsureCode#1#2{%
  \edef\ACCSUPP@AtEnd{%
    \ACCSUPP@AtEnd
    \catcode#1=\the\catcode#1\relax
  }%
  \catcode#1=#2\relax
}
\TMP@EnsureCode{10}{12}% ^^J
\TMP@EnsureCode{33}{12}% !
\TMP@EnsureCode{39}{12}% '
\TMP@EnsureCode{40}{12}% (
\TMP@EnsureCode{41}{12}% )
\TMP@EnsureCode{42}{12}% *
\TMP@EnsureCode{44}{12}% ,
\TMP@EnsureCode{45}{12}% -
\TMP@EnsureCode{46}{12}% .
\TMP@EnsureCode{47}{12}% /
\TMP@EnsureCode{58}{12}% :
\TMP@EnsureCode{60}{12}% <
\TMP@EnsureCode{62}{12}% >
\TMP@EnsureCode{94}{7}% ^ (superscript)
\TMP@EnsureCode{96}{12}% `
\TMP@EnsureCode{254}{12}% ^^fe
\TMP@EnsureCode{255}{12}% ^^ff
\edef\ACCSUPP@AtEnd{\ACCSUPP@AtEnd\noexpand\endinput}
%    \end{macrocode}
%
%    Package identification.
%    \begin{macrocode}
\NeedsTeXFormat{LaTeX2e}
\ProvidesPackage{accsupp}%
  [2018/01/30 v0.5 Accessibility support by marked content (HO)]
%    \end{macrocode}
%
%    \begin{macrocode}
\RequirePackage{pdfescape}[2007/02/25]
\RequirePackage{ifpdf}
\RequirePackage{ifxetex}
\RequirePackage{kvoptions}
%    \end{macrocode}
%    \begin{macrocode}
\SetupKeyvalOptions{%
  family=ACCSUPP,%
  prefix=ACCSUPP@%
}
%    \end{macrocode}
%
% \subsection{Space setup}
%
%    \begin{macrocode}
\DeclareBoolOption{space}
\DeclareStringOption[phvr8r]{spacefont}
\DeclareStringOption[32]{spacechar}
%    \end{macrocode}
%
% \subsection{Driver detection and setup}
%
% \paragraph{Driver declarations.}
%
%    \begin{macrocode}
\def\ACCSUPP@DefineDriverKey{%
  \@dblarg\ACCSUPP@@DefineDriverKey
}
\def\ACCSUPP@@DefineDriverKey[#1]#2{%
  \define@key{ACCSUPP}{#2}[]{%
    \def\ACCSUPP@driver{#1}%
  }%
  \g@addto@macro\ACCSUPP@DisableOptions{%
    \DisableKeyvalOption{ACCSUPP}{#2}%
  }%
}
\let\ACCSUPP@DisableOptions\@empty
\ACCSUPP@DefineDriverKey{pdftex}
\ACCSUPP@DefineDriverKey{luatex}
\ACCSUPP@DefineDriverKey{dvips}
\ACCSUPP@DefineDriverKey[dvips]{dvipsone}
\ACCSUPP@DefineDriverKey{dvipdfm}
\ACCSUPP@DefineDriverKey[dvipdfm]{dvipdfmx}
\ACCSUPP@DefineDriverKey[dvipdfm]{xetex}
\let\ACCSUPP@driver\relax
\InputIfFileExists{accsupp.cfg}{}{}
\providecommand*{\ActualTextDriverDefault}{dvips}
\ifpdf
  \ifx\pdfextension\@undefined
    \def\ACCSUPP@driver{pdftex}%
  \else
    \def\ACCSUPP@driver{luatex}%
  \fi
\else
  \ifxetex
    \def\ACCSUPP@driver{dvipdfm}%
  \else
    \ifx\ACCSUPP@driver\relax
      \let\ACCSUPP@driver\ActualTextDriverDefault
    \fi
  \fi
\fi
%    \end{macrocode}
%
% \paragraph{Process options.}
%
%    \begin{macrocode}
\ProcessKeyvalOptions*
\ACCSUPP@DisableOptions
%    \end{macrocode}
%
% \paragraph{Driver validation and loading.}
%
%    \begin{macrocode}
\def\ACCSUPP@temp{pdftex}%
\ifpdf
  \ifx\pdfextension\@undefined\else
    \def\ACCSUPP@temp{luatex}%
  \fi
  \ifx\ACCSUPP@temp\ACCSUPP@driver
  \else
    \PackageWarningNoLine{accsupp}{%
      Wrong driver `\ACCSUPP@driver', using `pdftex' instead%
    }%
ed as
    \let\ACCSUPP@driver\ACCSUPP@temp
  \fi
\else
  \ifx\ACCSUPP@temp\ACCSUPP@driver
    \PackageError{accsupp}{%
      Wrong driver, pdfTeX is not running in PDF mode.\MessageBreak
      Package loading is aborted%
    }\@ehc
    \expandafter\expandafter\expandafter\ACCSUPP@AtEnd
  \fi
  \def\ACCSUPP@temp{dvipdfm}%
  \ifxetex
    \ifx\ACCSUPP@temp\ACCSUPP@driver
    \else
      \PackageWarningNoLine{accsupp}{%
        Wrong driver `\ACCSUPP@driver',\MessageBreak
        using `dvipdfm' for XeTeX instead%
      }%
      \let\ACCSUPP@driver\ACCSUPP@temp
    \fi
  \fi
\fi%
\ifx\ACCSUPP@driver\relax
  \PackageError{accsupp}{%
    Missing driver option.\MessageBreak
    Package loading is aborted%
  }\@ehc
  \expandafter\ACCSUPP@AtEnd
\fi%
\InputIfFileExists{accsupp-\ACCSUPP@driver.def}{}{%
  \PackageError{accsupp}{%
    Unsupported driver `\ACCSUPP@driver'.\MessageBreak
    Package loading is aborted%
  }\@ehc
  \ACCSUPP@AtEnd
}%
%    \end{macrocode}
%
% \subsection{Main macro}
%
%    \begin{macrocode}
\DeclareBoolOption{unicode}
\DeclareStringOption[page]{pdfliteral}
\DeclareStringOption{Lang}
\def\ACCSUPP@method{escape}
\define@key{ACCSUPP}{method}{%
  \@ifundefined{ACCSUPP@method@#1}{%
    \PackageError{accsupp}{%
      Ignoring unknown method `#1'%
    }\@ehc
  }{%
    \edef\ACCSUPP@method{#1}%
  }%
}
\let\ACCSUPP@Lang\relax
\def\ACCSUPP@temp#1{%
  \expandafter\ACCSUPP@@temp\csname ACCSUPP@#1\endcsname{#1}%
}
\def\ACCSUPP@@temp#1#2{%
  \let#1\relax
  \define@key{ACCSUPP}{#2}{%
    \def#1{##1}%
    \ifx#1\@empty
      \def#1{()}%
    \else
      \csname ACCSUPP@method@\ACCSUPP@method\endcsname#1%
    \fi
  }%
}
\ACCSUPP@temp{Alt}
\ACCSUPP@temp{ActualText}
\ACCSUPP@temp{E}
%    \end{macrocode}
%
%    \begin{macrocode}
\newcommand*{\BeginAccSupp}[1]{%
  \begingroup
    \setkeys{ACCSUPP}{#1}%
    \edef\ACCSUPP@span{%
      /Span<<%
        \ifx\ACCSUPP@Lang\relax
        \else
          /Lang\ACCSUPP@Lang
        \fi
        \ifx\ACCSUPP@Alt\relax
        \else
          /Alt\ACCSUPP@Alt
        \fi
        \ifx\ACCSUPP@ActualText\relax
        \else
          /ActualText\ACCSUPP@ActualText
        \fi
        \ifx\ACCSUPP@E\relax
        \else
          /E\ACCSUPP@E
        \fi
      >>%
    }%
    \ACCSUPP@bdc
    \ACCSUPP@space
  \endgroup
}
\newcommand*{\EndAccSupp}[1]{%
  \begingroup
    \setkeys{ACCSUPP}{#1}%
    \ACCSUPP@emc
  \endgroup
}
\def\ACCSUPP@space{%
  \ifACCSUPP@space
    \begingroup
      \@ifundefined{ACCSUPP@Font}{%
        \global\font\ACCSUPP@Font=\ACCSUPP@spacefont\relax
      }{}%
      \leavevmode
      \setbox\z@\hbox{\ACCSUPP@Font\char\ACCSUPP@spacechar}%
      \wd\z@\z@
      \ht\z@\z@
      \dp\z@\z@
      \copy\z@
    \endgroup
  \fi
}
\newcommand*{\AccSuppSetup}{%
  \setkeys{ACCSUPP}%
}
%    \end{macrocode}
%
% \subsubsection{Input methods}
%
%    \begin{macrocode}
\def\ACCSUPP@method@plain#1{%
  \csname @safe@activestrue\endcsname
  \edef#1{%
    (%
    \ifACCSUPP@unicode
      \string\376\string\377%
    \fi
    #1%
    )%
  }%
  \@onelevel@sanitize#1%
}
%    \end{macrocode}
%    \begin{macrocode}
\def\ACCSUPP@method@escape#1{%
  \EdefEscapeString#1{%
    \ifACCSUPP@unicode
      ^^fe^^ff%
    \fi
    #1%
  }%
  \edef#1{(#1)}%
}%
%    \end{macrocode}
%    \begin{macrocode}
\def\ACCSUPP@method@hex#1{%
  \edef#1{%
    <%
    \ifACCSUPP@unicode
      FEFF%
    \fi
    #1%
    >%
  }%
}
%    \end{macrocode}
%    \begin{macrocode}
\def\ACCSUPP@method@pdfstringdef#1{%
  \ifACCSUPP@unicode
    \@ifundefined{hypersetup}{}{%
      \hypersetup{unicode}%
    }%
  \fi
  \@ifundefined{pdfstringdef}{%
    \PackageError{accsupp}{%
      Method `pdfstringdef' requires package `hyperref'%
    }\@ehc
    \let\ACCSUPP@temp\@empty
  }{%
    \begingroup
      \setbox0=\hbox{%
        \pdfstringdef\ACCSUPP@temp#1%
        \global\let\ACCSUPP@temp\ACCSUPP@temp
      }%
    \endgroup
  }%
  \edef#1{(\ACCSUPP@temp)}%
}
%    \end{macrocode}
%
%    \begin{macrocode}
\ACCSUPP@AtEnd%
%</package>
%    \end{macrocode}
%
% \subsection{Drivers}
%
% \subsubsection{Driver \xoption{pdftex}}
%
%    \begin{macrocode}
%<*pdftex>
\NeedsTeXFormat{LaTeX2e}
\ProvidesFile{accsupp-pdftex.def}%
  [2018/01/30 v0.5 accsupp driver for pdfTeX (HO)]%
%    \end{macrocode}
%
%    \begin{macrocode}
\def\ACCSUPP@bdc{%
  \pdfliteral\ACCSUPP@pdfliteral{\ACCSUPP@span BDC}%
}
\def\ACCSUPP@emc{%
  \pdfliteral\ACCSUPP@pdfliteral{EMC}%
}
%    \end{macrocode}
%    \begin{macrocode}
%</pdftex>
%    \end{macrocode}
%
% \subsubsection{Driver \xoption{luatex}}
%
%    \begin{macrocode}
%<*luatex>
\NeedsTeXFormat{LaTeX2e}
\ProvidesFile{accsupp-luatex.def}%
  [2018/01/30 v0.5 accsupp driver for pdfTeX (HO)]%
%    \end{macrocode}
%
%    \begin{macrocode}
\protected\def\ACCSUPP@pdf@literal{\pdfextension literal}
\def\ACCSUPP@bdc{%
  \ACCSUPP@pdf@literal\ACCSUPP@pdfliteral{\ACCSUPP@span BDC}%
}
\def\ACCSUPP@emc{%
  \ACCSUPP@pdf@literal\ACCSUPP@pdfliteral{EMC}%
}
%    \end{macrocode}
%    \begin{macrocode}
%</luatex>
%    \end{macrocode}
%
% \subsubsection{Driver \xoption{dvipdfm}}
%
%    \begin{macrocode}
%<*dvipdfm>
\NeedsTeXFormat{LaTeX2e}
\ProvidesFile{accsupp-dvipdfm.def}%
  [2018/01/30 v0.5 accsupp driver for dvipdfm (HO)]%
%    \end{macrocode}
%
%    \begin{macrocode}
\def\ACCSUPP@bdc{%
  \special{pdf:content \ACCSUPP@span BDC}%
}
\def\ACCSUPP@emc{%
  \special{pdf:content EMC}%
}
%    \end{macrocode}
%    \begin{macrocode}
%</dvipdfm>
%    \end{macrocode}
%
% \subsubsection{Driver \xoption{dvips}}
%
%    \begin{macrocode}
%<*dvips>
\NeedsTeXFormat{LaTeX2e}
\ProvidesFile{accsupp-dvips.def}%
  [2018/01/30 v0.5 accsupp driver for dvips (HO)]%
%    \end{macrocode}
%
%    \begin{macrocode}
\def\ACCSUPP@bdc{%
  \special{ps:[\ACCSUPP@span/BDC pdfmark}%
}
\def\ACCSUPP@emc{%
  \special{ps:[/EMC pdfmark}%
}
%    \end{macrocode}
%    \begin{macrocode}
%</dvips>
%    \end{macrocode}
%
% \section{Test}
%
% \subsection{Catcode checks for loading}
%
%    \begin{macrocode}
%<*test1>
%    \end{macrocode}
%    \begin{macrocode}
\NeedsTeXFormat{LaTeX2e}
\documentclass{minimal}
\makeatletter
\def\RestoreCatcodes{}
\count@=0 %
\loop
  \edef\RestoreCatcodes{%
    \RestoreCatcodes
    \catcode\the\count@=\the\catcode\count@\relax
  }%
\ifnum\count@<255 %
  \advance\count@\@ne
\repeat

\def\RangeCatcodeInvalid#1#2{%
  \count@=#1\relax
  \loop
    \catcode\count@=15 %
  \ifnum\count@<#2\relax
    \advance\count@\@ne
  \repeat
}
\def\Test{%
  \RangeCatcodeInvalid{0}{47}%
  \RangeCatcodeInvalid{58}{64}%
  \RangeCatcodeInvalid{91}{96}%
  \RangeCatcodeInvalid{123}{127}%
  \catcode`\@=12 %
  \catcode`\\=0 %
  \catcode`\{=1 %
  \catcode`\}=2 %
  \catcode`\#=6 %
  \catcode`\[=12 %
  \catcode`\]=12 %
  \catcode`\%=14 %
  \catcode`\ =10 %
  \catcode13=5 %
  \RequirePackage{accsupp}[2018/01/30]\relax
  \RestoreCatcodes
}
\Test
\csname @@end\endcsname
\end
%    \end{macrocode}
%    \begin{macrocode}
%</test1>
%    \end{macrocode}
%
% \section{Installation}
%
% \subsection{Download}
%
% \paragraph{Package.} This package is available on
% CTAN\footnote{\url{http://ctan.org/pkg/accsupp}}:
% \begin{description}
% \item[\CTAN{macros/latex/contrib/oberdiek/accsupp.dtx}] The source file.
% \item[\CTAN{macros/latex/contrib/oberdiek/accsupp.pdf}] Documentation.
% \end{description}
%
%
% \paragraph{Bundle.} All the packages of the bundle `oberdiek'
% are also available in a TDS compliant ZIP archive. There
% the packages are already unpacked and the documentation files
% are generated. The files and directories obey the TDS standard.
% \begin{description}
% \item[\CTAN{install/macros/latex/contrib/oberdiek.tds.zip}]
% \end{description}
% \emph{TDS} refers to the standard ``A Directory Structure
% for \TeX\ Files'' (\CTAN{tds/tds.pdf}). Directories
% with \xfile{texmf} in their name are usually organized this way.
%
% \subsection{Bundle installation}
%
% \paragraph{Unpacking.} Unpack the \xfile{oberdiek.tds.zip} in the
% TDS tree (also known as \xfile{texmf} tree) of your choice.
% Example (linux):
% \begin{quote}
%   |unzip oberdiek.tds.zip -d ~/texmf|
% \end{quote}
%
% \paragraph{Script installation.}
% Check the directory \xfile{TDS:scripts/oberdiek/} for
% scripts that need further installation steps.
% Package \xpackage{attachfile2} comes with the Perl script
% \xfile{pdfatfi.pl} that should be installed in such a way
% that it can be called as \texttt{pdfatfi}.
% Example (linux):
% \begin{quote}
%   |chmod +x scripts/oberdiek/pdfatfi.pl|\\
%   |cp scripts/oberdiek/pdfatfi.pl /usr/local/bin/|
% \end{quote}
%
% \subsection{Package installation}
%
% \paragraph{Unpacking.} The \xfile{.dtx} file is a self-extracting
% \docstrip\ archive. The files are extracted by running the
% \xfile{.dtx} through \plainTeX:
% \begin{quote}
%   \verb|tex accsupp.dtx|
% \end{quote}
%
% \paragraph{TDS.} Now the different files must be moved into
% the different directories in your installation TDS tree
% (also known as \xfile{texmf} tree):
% \begin{quote}
% \def\t{^^A
% \begin{tabular}{@{}>{\ttfamily}l@{ $\rightarrow$ }>{\ttfamily}l@{}}
%   accsupp.sty & tex/latex/oberdiek/accsupp.sty\\
%   accsupp-pdftex.def & tex/latex/oberdiek/accsupp-pdftex.def\\
%   accsupp-luatex.def & tex/latex/oberdiek/accsupp-luatex.def\\
%   accsupp-dvipdfm.def & tex/latex/oberdiek/accsupp-dvipdfm.def\\
%   accsupp-dvips.def & tex/latex/oberdiek/accsupp-dvips.def\\
%   accsupp.pdf & doc/latex/oberdiek/accsupp.pdf\\
%   accsupp-example1.tex & doc/latex/oberdiek/accsupp-example1.tex\\
%   accsupp-example2.tex & doc/latex/oberdiek/accsupp-example2.tex\\
%   test/accsupp-test1.tex & doc/latex/oberdiek/test/accsupp-test1.tex\\
%   accsupp.dtx & source/latex/oberdiek/accsupp.dtx\\
% \end{tabular}^^A
% }^^A
% \sbox0{\t}^^A
% \ifdim\wd0>\linewidth
%   \begingroup
%     \advance\linewidth by\leftmargin
%     \advance\linewidth by\rightmargin
%   \edef\x{\endgroup
%     \def\noexpand\lw{\the\linewidth}^^A
%   }\x
%   \def\lwbox{^^A
%     \leavevmode
%     \hbox to \linewidth{^^A
%       \kern-\leftmargin\relax
%       \hss
%       \usebox0
%       \hss
%       \kern-\rightmargin\relax
%     }^^A
%   }^^A
%   \ifdim\wd0>\lw
%     \sbox0{\small\t}^^A
%     \ifdim\wd0>\linewidth
%       \ifdim\wd0>\lw
%         \sbox0{\footnotesize\t}^^A
%         \ifdim\wd0>\linewidth
%           \ifdim\wd0>\lw
%             \sbox0{\scriptsize\t}^^A
%             \ifdim\wd0>\linewidth
%               \ifdim\wd0>\lw
%                 \sbox0{\tiny\t}^^A
%                 \ifdim\wd0>\linewidth
%                   \lwbox
%                 \else
%                   \usebox0
%                 \fi
%               \else
%                 \lwbox
%               \fi
%             \else
%               \usebox0
%             \fi
%           \else
%             \lwbox
%           \fi
%         \else
%           \usebox0
%         \fi
%       \else
%         \lwbox
%       \fi
%     \else
%       \usebox0
%     \fi
%   \else
%     \lwbox
%   \fi
% \else
%   \usebox0
% \fi
% \end{quote}
% If you have a \xfile{docstrip.cfg} that configures and enables \docstrip's
% TDS installing feature, then some files can already be in the right
% place, see the documentation of \docstrip.
%
% \subsection{Refresh file name databases}
%
% If your \TeX~distribution
% (\teTeX, \mikTeX, \dots) relies on file name databases, you must refresh
% these. For example, \teTeX\ users run \verb|texhash| or
% \verb|mktexlsr|.
%
% \subsection{Some details for the interested}
%
% \paragraph{Attached source.}
%
% The PDF documentation on CTAN also includes the
% \xfile{.dtx} source file. It can be extracted by
% AcrobatReader 6 or higher. Another option is \textsf{pdftk},
% e.g. unpack the file into the current directory:
% \begin{quote}
%   \verb|pdftk accsupp.pdf unpack_files output .|
% \end{quote}
%
% \paragraph{Unpacking with \LaTeX.}
% The \xfile{.dtx} chooses its action depending on the format:
% \begin{description}
% \item[\plainTeX:] Run \docstrip\ and extract the files.
% \item[\LaTeX:] Generate the documentation.
% \end{description}
% If you insist on using \LaTeX\ for \docstrip\ (really,
% \docstrip\ does not need \LaTeX), then inform the autodetect routine
% about your intention:
% \begin{quote}
%   \verb|latex \let\install=y% \iffalse meta-comment
%
% File: accsupp.dtx
% Version: 2018/01/30 v0.5
% Info: Accessibility support by marked content
%
% Copyright (C) 2007, 2010 by
%    Heiko Oberdiek <heiko.oberdiek at googlemail.com>
%    2016
%    https://github.com/ho-tex/oberdiek/issues
%
% This work may be distributed and/or modified under the
% conditions of the LaTeX Project Public License, either
% version 1.3c of this license or (at your option) any later
% version. This version of this license is in
%    http://www.latex-project.org/lppl/lppl-1-3c.txt
% and the latest version of this license is in
%    http://www.latex-project.org/lppl.txt
% and version 1.3 or later is part of all distributions of
% LaTeX version 2005/12/01 or later.
%
% This work has the LPPL maintenance status "maintained".
%
% This Current Maintainer of this work is Heiko Oberdiek.
%
% This work consists of the main source file accsupp.dtx
% and the derived files
%    accsupp.sty, accsupp.pdf, accsupp.ins, accsupp.drv,
%    accsupp-pdftex.def, \ccsupp-luatex.def, accsupp-dvipdfm.def,
%    accsupp-dvips.def,
%    accsupp-example1.tex, accsupp-example2.tex,
%    accsupp-test1.tex.
%
% Distribution:
%    CTAN:macros/latex/contrib/oberdiek/accsupp.dtx
%    CTAN:macros/latex/contrib/oberdiek/accsupp.pdf
%
% Unpacking:
%    (a) If accsupp.ins is present:
%           tex accsupp.ins
%    (b) Without accsupp.ins:
%           tex accsupp.dtx
%    (c) If you insist on using LaTeX
%           latex \let\install=y\input{accsupp.dtx}
%        (quote the arguments according to the demands of your shell)
%
% Documentation:
%    (a) If accsupp.drv is present:
%           latex accsupp.drv
%    (b) Without accsupp.drv:
%           latex accsupp.dtx; ...
%    The class ltxdoc loads the configuration file ltxdoc.cfg
%    if available. Here you can specify further options, e.g.
%    use A4 as paper format:
%       \PassOptionsToClass{a4paper}{article}
%
%    Programm calls to get the documentation (example):
%       pdflatex accsupp.dtx
%       makeindex -s gind.ist accsupp.idx
%       pdflatex accsupp.dtx
%       makeindex -s gind.ist accsupp.idx
%       pdflatex accsupp.dtx
%
% Installation:
%    TDS:tex/latex/oberdiek/accsupp.sty
%    TDS:tex/latex/oberdiek/accsupp-pdftex.def
%    TDS:tex/latex/oberdiek/accsupp-luatex.def
%    TDS:tex/latex/oberdiek/accsupp-dvipdfm.def
%    TDS:tex/latex/oberdiek/accsupp-dvips.def
%    TDS:doc/latex/oberdiek/accsupp.pdf
%    TDS:doc/latex/oberdiek/accsupp-example1.tex
%    TDS:doc/latex/oberdiek/accsupp-example2.tex
%    TDS:doc/latex/oberdiek/test/accsupp-test1.tex
%    TDS:source/latex/oberdiek/accsupp.dtx
%
%<*ignore>
\begingroup
  \catcode123=1 %
  \catcode125=2 %
  \def\x{LaTeX2e}%
\expandafter\endgroup
\ifcase 0\ifx\install y1\fi\expandafter
         \ifx\csname processbatchFile\endcsname\relax\else1\fi
         \ifx\fmtname\x\else 1\fi\relax
\else\csname fi\endcsname
%</ignore>
%<*install>
\input docstrip.tex
\Msg{************************************************************************}
\Msg{* Installation}
\Msg{* Package: accsupp 2018/01/30 v0.5 Accessibility support by marked content (HO)}
\Msg{************************************************************************}

\keepsilent
\askforoverwritefalse

\let\MetaPrefix\relax
\preamble

This is a generated file.

Project: accsupp
Version: 2018/01/30 v0.5

Copyright (C) 2007, 2010 by
   Heiko Oberdiek <heiko.oberdiek at googlemail.com>

This work may be distributed and/or modified under the
conditions of the LaTeX Project Public License, either
version 1.3c of this license or (at your option) any later
version. This version of this license is in
   http://www.latex-project.org/lppl/lppl-1-3c.txt
and the latest version of this license is in
   http://www.latex-project.org/lppl.txt
and version 1.3 or later is part of all distributions of
LaTeX version 2005/12/01 or later.

This work has the LPPL maintenance status "maintained".

This Current Maintainer of this work is Heiko Oberdiek.

This work consists of the main source file accsupp.dtx
and the derived files
   accsupp.sty, accsupp.pdf, accsupp.ins, accsupp.drv,
   accsupp-pdftex.def, accsupp-luatex.def, accsupp-dvipdfm.def,
   accsupp-dvips.def,
   accsupp-example1.tex, accsupp-example2.tex,
   accsupp-test1.tex.

\endpreamble
\let\MetaPrefix\DoubleperCent

\generate{%
  \file{accsupp.ins}{\from{accsupp.dtx}{install}}%
  \file{accsupp.drv}{\from{accsupp.dtx}{driver}}%
  \usedir{tex/latex/oberdiek}%
  \file{accsupp.sty}{\from{accsupp.dtx}{package}}%
  \file{accsupp-pdftex.def}{\from{accsupp.dtx}{pdftex}}%
  \file{accsupp-luatex.def}{\from{accsupp.dtx}{luatex}}%
  \file{accsupp-dvipdfm.def}{\from{accsupp.dtx}{dvipdfm}}%
  \file{accsupp-dvips.def}{\from{accsupp.dtx}{dvips}}%
  \usedir{doc/latex/oberdiek}%
  \file{accsupp-example1.tex}{\from{accsupp.dtx}{example1}}%
  \file{accsupp-example2.tex}{\from{accsupp.dtx}{example2}}%
  \usedir{doc/latex/oberdiek/test}%
  \file{accsupp-test1.tex}{\from{accsupp.dtx}{test1}}%
  \nopreamble
  \nopostamble
  \usedir{source/latex/oberdiek/catalogue}%
  \file{accsupp.xml}{\from{accsupp.dtx}{catalogue}}%
}

\catcode32=13\relax% active space
\let =\space%
\Msg{************************************************************************}
\Msg{*}
\Msg{* To finish the installation you have to move the following}
\Msg{* files into a directory searched by TeX:}
\Msg{*}
\Msg{*     accsupp.sty, accsupp-pdftex.def, accsupp-dvipdfm.def,}
\Msg{*     accsupp-dvips.def}
\Msg{*}
\Msg{* To produce the documentation run the file `accsupp.drv'}
\Msg{* through LaTeX.}
\Msg{*}
\Msg{* Happy TeXing!}
\Msg{*}
\Msg{************************************************************************}

\endbatchfile
%</install>
%<*ignore>
\fi
%</ignore>
%<*driver>
\NeedsTeXFormat{LaTeX2e}
\ProvidesFile{accsupp.drv}%
  [2018/01/30 v0.5 Accessibility support by marked content (HO)]%
\documentclass{ltxdoc}
\usepackage{holtxdoc}[2011/11/22]
\begin{document}
  \DocInput{accsupp.dtx}%
\end{document}
%</driver>
% \fi
%
%
% \CharacterTable
%  {Upper-case    \A\B\C\D\E\F\G\H\I\J\K\L\M\N\O\P\Q\R\S\T\U\V\W\X\Y\Z
%   Lower-case    \a\b\c\d\e\f\g\h\i\j\k\l\m\n\o\p\q\r\s\t\u\v\w\x\y\z
%   Digits        \0\1\2\3\4\5\6\7\8\9
%   Exclamation   \!     Double quote  \"     Hash (number) \#
%   Dollar        \$     Percent       \%     Ampersand     \&
%   Acute accent  \'     Left paren    \(     Right paren   \)
%   Asterisk      \*     Plus          \+     Comma         \,
%   Minus         \-     Point         \.     Solidus       \/
%   Colon         \:     Semicolon     \;     Less than     \<
%   Equals        \=     Greater than  \>     Question mark \?
%   Commercial at \@     Left bracket  \[     Backslash     \\
%   Right bracket \]     Circumflex    \^     Underscore    \_
%   Grave accent  \`     Left brace    \{     Vertical bar  \|
%   Right brace   \}     Tilde         \~}
%
% \GetFileInfo{accsupp.drv}
%
% \title{The \xpackage{accsupp} package}
% \date{2018/01/30 v0.5}
% \author{Heiko Oberdiek\thanks
% {Please report any issues at https://github.com/ho-tex/oberdiek/issues}\\
% \xemail{heiko.oberdiek at googlemail.com}}
%
% \maketitle
%
% \begin{abstract}
% Since PDF 1.5 portions of a page can be marked for better
% accessibility support.
% For example, replacement texts or expansions of abbreviations can be
% provided. Package \xpackage{accsupp} starts with providing a minimal
% low-level interface for programmers. Status is experimental.
% \end{abstract}
%
% \tableofcontents
%
% \section{Documentation}
%
% \subsection{Macros}
%
% Section ``10.8 Accessibility Support'' of the PDF reference
% \cite{pdfspec} lists some features that can be added by
% operators for marked content.
%
% \begin{declcs}{BeginAccSupp} \M{options}
% \end{declcs}
% It puts the operator |BDC| in the page stream:
% \begin{quote}
%   |/Span|\\
%   |<<|\dots|>>| \qquad \textit{\% property dictionary}\\
%   |BDC|
% \end{quote}
% The contents of the dictionary is controlled by \meta{options}.
% See sections \ref{sec:feature} and \ref{sec:method}.
%
% \begin{declcs}{EndAccSupp} \M{options}
% \end{declcs}
% It puts the operator |EMC| in the page stream.
% The only option is \xoption{pdfliteral}, see section \ref{sec:pdftex}.
%
% \bigskip
% \noindent\textbf{Note:}
% The caller is responsible for the placement of \cs{BeginAccSupp}
% and \cs{EndAccSupp} pairs. Especially page breaks are not allowed
% in between.
%
% \subsubsection{Feature options}\label{sec:feature}
%
% The PDF reference \cite{pdfspec} describes and explains the different
% features. The names of the feature options are the same as the
% key names for the property dictionary for operator |BDC|, see
% \cs{BeginAccSupp}.
%
% \begin{description}
% \item[\xoption{ActualText}:]
%   Provides a replacement text, see examples in
%   section \ref{sec:example}.
% \item[\xoption{Alt}:]
%   Provides an alternate description.
% \item[\xoption{E}:]
%   Provides the expansion of an abbreviation or an acronym.
% \item[\xoption{Lang}:]
%   Specifies the language.
% \end{description}
%
% \subsubsection{Input methods}\label{sec:method}
%
% Except for \xoption{Lang} option \xoption{method} controls
% how the argument for \xoption{ActualText}, \xoption{Alt},
% and \xoption{E} are interpreted.
% \begin{description}
% \item[\xoption{method=plain}:]
%   The string is only expanded and written without further
%   treatment. Special characters are not protected, thus this method
%   may result in an invalid PDF file.
% \item[\xoption{mehod=escape}:]
%   The string is expanded and special characters are escaped.
%   The result is a valid PDF string.
% \item[\xoption{method=hex}:]
%   The string is given in hexadecimal notation.
%   Section \ref{sec:example:notparallel} shows an example.
% \item[\xoption{method=pdfstringdef}:]
%   If package \xpackage{hyperref} is loaded, then its \cs{pdfstringdef}
%   is used. This method is slow, but useful if the string contains
%   arbitrary \hologo{TeX} code.
% \item[\xoption{unicode}:]
%   This option is needed, if the string is given as Unicode string
%   (16 bit). Internally it adds the string prefix for Unicode.
%   In case of \xoption{method=pdfstringdef} it passes the option
%   to \cs{hypersetup}.
% \end{description}
%
% \subsection{Workaround, option \xoption{space}}
%
% PDF specification says in ``10.8.3 Replacement Text''
% \begin{quote}
% Just as alternative descriptions can be provided for images and other
% items that do not translate naturally into text (\dots), replacement
% text can be specified for content that does translate into text but
% that is represented in a nonstandard way. These nonstandard representations
% might include, for example, glyphs for ligatures or custom characters, or
% inline graphics corresponding to letters in an illuminated manustript or to
% dropped capitals.
% \end{quote}
% However, the ActualText is ignored in Acrobat Reader
% (until version 9 at least), if the content does not contain glyphs.
% Option \xoption{space} adds such an invisible glyph, a space
% character. The font name can be configured by option \xoption{spacefont},
% the default font is \texttt{phvr8r}. The character slot is given
% by option \xoption{spacechar}, default is \texttt{32}, the usual
% position of the space character.
%
% These options can also be given as package options or in
% macro \cs{AccSuppSetup} that takes a key value list as argument.
% Usually only option \xoption{space} is necessary, if the
% contents does not contain glyphs otherwise. Option \xoption{space}
% is enabled by |space| or |space=true| and disabled by |space=false|.
% It is disabled as default. The option is evaluated by \cs{BeginAccSupp}
% and ignored by \cs{EndAccSupp}.
%
% Note: Even with option \xoption{space} I get sometimes wrong texts
% when cut \& paste from AR7/Linux or AR8/Linux, e.g.
% \begin{quote}
%   |Hello| $\rightarrow$ |Helo|, |Helol|, |Hell|, \dots\\
%   |Hello World| $\rightarrow$ |Helo WorldW|, |Helo World|, \dots
% \end{quote}
% I do not know what Acrobat Reader is doing here, thus feedback and
% insights are welcome.
%
% \subsection{Driver options}
%
% Driver options are package options only.
% The special \hologo{TeX} compilers \hologo{pdfTeX}, \hologo{LuaTeX} and \hologo{XeTeX}
% are detected automatically. The default for unrecognized drivers can be
% set by defining \cs{ActualTextDriverDefault}. This can be done
% in the configuration file \xfile{accsupp.cfg}.
%
% \subsubsection{Option \xoption{pdftex}}\label{sec:pdftex}
% Package option \xoption{pdftex} is used for \hologo{pdfTeX} in PDF mode.
% Additionally \cs{BeginAccSupp} and \cs{EndAccSupp} understand
% option \xoption{pdfliteral}. It controls
% the modifier keyword for \cs{pdfliteral}:
% \begin{quote}
%   |pdfliteral=direct| $\Rightarrow$ |\pdfliteral direct{...}|
% \end{quote}
%
% \subsubsection{Option \xoption{luatex}}\label{sec:luatex}
% Package option \xoption{luatex} is used for \hologo{LuaTeX} in PDF mode.
% Additionally \cs{BeginAccSupp} and \cs{EndAccSupp} understand
% option \xoption{pdfliteral}. It controls
% the modifier keyword for \cs{pdfextension} literal:
% \begin{quote}
%   |pdfliteral=direct| $\Rightarrow$ |\pdfextension literal direct{...}|
% \end{quote}
%
% \subsubsection{Option \xoption{dvipdfm}}
%
% Package option \xoption{dvipdfm} and its aliases \xoption{dvipdfmx}
% \xoption{xetex} are used for drivers that support
% \textsf{dvipdfm} specials.
%
% \subsubsection{Option \xoption{dvips}}
%
% Package option \xoption{dvips} and its alias \xoption{dvipsone}
% write pdfmark specials in the output. Unhappily these pdfmark
% operators are ignored by \textsf{ghostscript} (latest tested
% version is 8.54). Perhaps they are recognized by commercial
% distiller applications.
%
% \subsubsection{Turning off page stream compression}
%
% For debugging it is useful to have uncompressed page stream objects.
% This can be done afterwards via \textsf{pdftk}:
% \begin{quote}
%   |pdftk file.pdf output file-uncompressed.pdf uncompress|
% \end{quote}
% Or the PDF file is generated uncompressed:
% \begin{description}
% \item[\hologo{pdfTeX}:] |\pdfcompresslevel=0|
% \item[dvipdfm:] |dvipdfm -z0 ...|
% \item[dvipdfmx:] |dvipdfmx -z0 ...|
% \item[ghostscript/ps2pdf:]
%      |ps2pdf -dCompressPages=false input.ps output.pdf|
% \end{description}
%
% \section{Example}
% \label{sec:example}
%
% \subsection{Example \cs{notparallel}}
% \label{sec:example:notparallel}
%
%    \begin{macrocode}
%<*example1>
%    \end{macrocode}
%    \begin{macrocode}
%<<END
\documentclass{article}
\usepackage{accsupp}[2007/11/14]
\usepackage{centernot}
% U+2226 NOT PARALLEL
% \mathrel{...} prevents page break in between
\newcommand*{\notparallel}{%
  \ensuremath{%
    \mathrel{%
      \BeginAccSupp{method=hex,unicode,ActualText=2226}%
      \centernot{\parallel}%
      \EndAccSupp{}%
    }%
  }%
}
\begin{document}
\begin{equation}
A\notparallel B
\end{equation}
\end{document}
%END
%    \end{macrocode}
%    \begin{macrocode}
%</example1>
%    \end{macrocode}
%
% \subsection{Example with \xoption{pdfstringdef}}
%
%    \begin{macrocode}
%<*example2>
%    \end{macrocode}
%    \begin{macrocode}
%<<END
\documentclass{article}
\usepackage[unicode]{hyperref}
\usepackage{accsupp}[2007/11/14]
\begin{document}
  \begin{equation}
    \BeginAccSupp{
      method=pdfstringdef,
      unicode,
      ActualText={%
        a\texttwosuperior +b\texttwosuperior
        =c\texttwosuperior
      }
    }
    a^2 + b^2 = c^2
    \EndAccSupp{}
  \end{equation}
\end{document}
%END
%    \end{macrocode}
%    \begin{macrocode}
%</example2>
%    \end{macrocode}
%
% \StopEventually{
% }
%
% \section{Implementation}
%
% \subsection{Package}
%
%    \begin{macrocode}
%<*package>
%    \end{macrocode}
%
%    \begin{macrocode}
\begingroup\catcode61\catcode48\catcode32=10\relax%
  \catcode13=5 % ^^M
  \endlinechar=13 %
  \catcode123=1 % {
  \catcode125=2 % }
  \catcode64=11 % @
  \def\x{\endgroup
    \expandafter\edef\csname ACCSUPP@AtEnd\endcsname{%
      \endlinechar=\the\endlinechar\relax
      \catcode13=\the\catcode13\relax
      \catcode32=\the\catcode32\relax
      \catcode35=\the\catcode35\relax
      \catcode61=\the\catcode61\relax
      \catcode64=\the\catcode64\relax
      \catcode123=\the\catcode123\relax
      \catcode125=\the\catcode125\relax
    }%
  }%
\x\catcode61\catcode48\catcode32=10\relax%
\catcode13=5 % ^^M
\endlinechar=13 %
\catcode35=6 % #
\catcode64=11 % @
\catcode123=1 % {
\catcode125=2 % }
\def\TMP@EnsureCode#1#2{%
  \edef\ACCSUPP@AtEnd{%
    \ACCSUPP@AtEnd
    \catcode#1=\the\catcode#1\relax
  }%
  \catcode#1=#2\relax
}
\TMP@EnsureCode{10}{12}% ^^J
\TMP@EnsureCode{33}{12}% !
\TMP@EnsureCode{39}{12}% '
\TMP@EnsureCode{40}{12}% (
\TMP@EnsureCode{41}{12}% )
\TMP@EnsureCode{42}{12}% *
\TMP@EnsureCode{44}{12}% ,
\TMP@EnsureCode{45}{12}% -
\TMP@EnsureCode{46}{12}% .
\TMP@EnsureCode{47}{12}% /
\TMP@EnsureCode{58}{12}% :
\TMP@EnsureCode{60}{12}% <
\TMP@EnsureCode{62}{12}% >
\TMP@EnsureCode{94}{7}% ^ (superscript)
\TMP@EnsureCode{96}{12}% `
\TMP@EnsureCode{254}{12}% ^^fe
\TMP@EnsureCode{255}{12}% ^^ff
\edef\ACCSUPP@AtEnd{\ACCSUPP@AtEnd\noexpand\endinput}
%    \end{macrocode}
%
%    Package identification.
%    \begin{macrocode}
\NeedsTeXFormat{LaTeX2e}
\ProvidesPackage{accsupp}%
  [2018/01/30 v0.5 Accessibility support by marked content (HO)]
%    \end{macrocode}
%
%    \begin{macrocode}
\RequirePackage{pdfescape}[2007/02/25]
\RequirePackage{ifpdf}
\RequirePackage{ifxetex}
\RequirePackage{kvoptions}
%    \end{macrocode}
%    \begin{macrocode}
\SetupKeyvalOptions{%
  family=ACCSUPP,%
  prefix=ACCSUPP@%
}
%    \end{macrocode}
%
% \subsection{Space setup}
%
%    \begin{macrocode}
\DeclareBoolOption{space}
\DeclareStringOption[phvr8r]{spacefont}
\DeclareStringOption[32]{spacechar}
%    \end{macrocode}
%
% \subsection{Driver detection and setup}
%
% \paragraph{Driver declarations.}
%
%    \begin{macrocode}
\def\ACCSUPP@DefineDriverKey{%
  \@dblarg\ACCSUPP@@DefineDriverKey
}
\def\ACCSUPP@@DefineDriverKey[#1]#2{%
  \define@key{ACCSUPP}{#2}[]{%
    \def\ACCSUPP@driver{#1}%
  }%
  \g@addto@macro\ACCSUPP@DisableOptions{%
    \DisableKeyvalOption{ACCSUPP}{#2}%
  }%
}
\let\ACCSUPP@DisableOptions\@empty
\ACCSUPP@DefineDriverKey{pdftex}
\ACCSUPP@DefineDriverKey{luatex}
\ACCSUPP@DefineDriverKey{dvips}
\ACCSUPP@DefineDriverKey[dvips]{dvipsone}
\ACCSUPP@DefineDriverKey{dvipdfm}
\ACCSUPP@DefineDriverKey[dvipdfm]{dvipdfmx}
\ACCSUPP@DefineDriverKey[dvipdfm]{xetex}
\let\ACCSUPP@driver\relax
\InputIfFileExists{accsupp.cfg}{}{}
\providecommand*{\ActualTextDriverDefault}{dvips}
\ifpdf
  \ifx\pdfextension\@undefined
    \def\ACCSUPP@driver{pdftex}%
  \else
    \def\ACCSUPP@driver{luatex}%
  \fi
\else
  \ifxetex
    \def\ACCSUPP@driver{dvipdfm}%
  \else
    \ifx\ACCSUPP@driver\relax
      \let\ACCSUPP@driver\ActualTextDriverDefault
    \fi
  \fi
\fi
%    \end{macrocode}
%
% \paragraph{Process options.}
%
%    \begin{macrocode}
\ProcessKeyvalOptions*
\ACCSUPP@DisableOptions
%    \end{macrocode}
%
% \paragraph{Driver validation and loading.}
%
%    \begin{macrocode}
\def\ACCSUPP@temp{pdftex}%
\ifpdf
  \ifx\pdfextension\@undefined\else
    \def\ACCSUPP@temp{luatex}%
  \fi
  \ifx\ACCSUPP@temp\ACCSUPP@driver
  \else
    \PackageWarningNoLine{accsupp}{%
      Wrong driver `\ACCSUPP@driver', using `pdftex' instead%
    }%
ed as
    \let\ACCSUPP@driver\ACCSUPP@temp
  \fi
\else
  \ifx\ACCSUPP@temp\ACCSUPP@driver
    \PackageError{accsupp}{%
      Wrong driver, pdfTeX is not running in PDF mode.\MessageBreak
      Package loading is aborted%
    }\@ehc
    \expandafter\expandafter\expandafter\ACCSUPP@AtEnd
  \fi
  \def\ACCSUPP@temp{dvipdfm}%
  \ifxetex
    \ifx\ACCSUPP@temp\ACCSUPP@driver
    \else
      \PackageWarningNoLine{accsupp}{%
        Wrong driver `\ACCSUPP@driver',\MessageBreak
        using `dvipdfm' for XeTeX instead%
      }%
      \let\ACCSUPP@driver\ACCSUPP@temp
    \fi
  \fi
\fi%
\ifx\ACCSUPP@driver\relax
  \PackageError{accsupp}{%
    Missing driver option.\MessageBreak
    Package loading is aborted%
  }\@ehc
  \expandafter\ACCSUPP@AtEnd
\fi%
\InputIfFileExists{accsupp-\ACCSUPP@driver.def}{}{%
  \PackageError{accsupp}{%
    Unsupported driver `\ACCSUPP@driver'.\MessageBreak
    Package loading is aborted%
  }\@ehc
  \ACCSUPP@AtEnd
}%
%    \end{macrocode}
%
% \subsection{Main macro}
%
%    \begin{macrocode}
\DeclareBoolOption{unicode}
\DeclareStringOption[page]{pdfliteral}
\DeclareStringOption{Lang}
\def\ACCSUPP@method{escape}
\define@key{ACCSUPP}{method}{%
  \@ifundefined{ACCSUPP@method@#1}{%
    \PackageError{accsupp}{%
      Ignoring unknown method `#1'%
    }\@ehc
  }{%
    \edef\ACCSUPP@method{#1}%
  }%
}
\let\ACCSUPP@Lang\relax
\def\ACCSUPP@temp#1{%
  \expandafter\ACCSUPP@@temp\csname ACCSUPP@#1\endcsname{#1}%
}
\def\ACCSUPP@@temp#1#2{%
  \let#1\relax
  \define@key{ACCSUPP}{#2}{%
    \def#1{##1}%
    \ifx#1\@empty
      \def#1{()}%
    \else
      \csname ACCSUPP@method@\ACCSUPP@method\endcsname#1%
    \fi
  }%
}
\ACCSUPP@temp{Alt}
\ACCSUPP@temp{ActualText}
\ACCSUPP@temp{E}
%    \end{macrocode}
%
%    \begin{macrocode}
\newcommand*{\BeginAccSupp}[1]{%
  \begingroup
    \setkeys{ACCSUPP}{#1}%
    \edef\ACCSUPP@span{%
      /Span<<%
        \ifx\ACCSUPP@Lang\relax
        \else
          /Lang\ACCSUPP@Lang
        \fi
        \ifx\ACCSUPP@Alt\relax
        \else
          /Alt\ACCSUPP@Alt
        \fi
        \ifx\ACCSUPP@ActualText\relax
        \else
          /ActualText\ACCSUPP@ActualText
        \fi
        \ifx\ACCSUPP@E\relax
        \else
          /E\ACCSUPP@E
        \fi
      >>%
    }%
    \ACCSUPP@bdc
    \ACCSUPP@space
  \endgroup
}
\newcommand*{\EndAccSupp}[1]{%
  \begingroup
    \setkeys{ACCSUPP}{#1}%
    \ACCSUPP@emc
  \endgroup
}
\def\ACCSUPP@space{%
  \ifACCSUPP@space
    \begingroup
      \@ifundefined{ACCSUPP@Font}{%
        \global\font\ACCSUPP@Font=\ACCSUPP@spacefont\relax
      }{}%
      \leavevmode
      \setbox\z@\hbox{\ACCSUPP@Font\char\ACCSUPP@spacechar}%
      \wd\z@\z@
      \ht\z@\z@
      \dp\z@\z@
      \copy\z@
    \endgroup
  \fi
}
\newcommand*{\AccSuppSetup}{%
  \setkeys{ACCSUPP}%
}
%    \end{macrocode}
%
% \subsubsection{Input methods}
%
%    \begin{macrocode}
\def\ACCSUPP@method@plain#1{%
  \csname @safe@activestrue\endcsname
  \edef#1{%
    (%
    \ifACCSUPP@unicode
      \string\376\string\377%
    \fi
    #1%
    )%
  }%
  \@onelevel@sanitize#1%
}
%    \end{macrocode}
%    \begin{macrocode}
\def\ACCSUPP@method@escape#1{%
  \EdefEscapeString#1{%
    \ifACCSUPP@unicode
      ^^fe^^ff%
    \fi
    #1%
  }%
  \edef#1{(#1)}%
}%
%    \end{macrocode}
%    \begin{macrocode}
\def\ACCSUPP@method@hex#1{%
  \edef#1{%
    <%
    \ifACCSUPP@unicode
      FEFF%
    \fi
    #1%
    >%
  }%
}
%    \end{macrocode}
%    \begin{macrocode}
\def\ACCSUPP@method@pdfstringdef#1{%
  \ifACCSUPP@unicode
    \@ifundefined{hypersetup}{}{%
      \hypersetup{unicode}%
    }%
  \fi
  \@ifundefined{pdfstringdef}{%
    \PackageError{accsupp}{%
      Method `pdfstringdef' requires package `hyperref'%
    }\@ehc
    \let\ACCSUPP@temp\@empty
  }{%
    \begingroup
      \setbox0=\hbox{%
        \pdfstringdef\ACCSUPP@temp#1%
        \global\let\ACCSUPP@temp\ACCSUPP@temp
      }%
    \endgroup
  }%
  \edef#1{(\ACCSUPP@temp)}%
}
%    \end{macrocode}
%
%    \begin{macrocode}
\ACCSUPP@AtEnd%
%</package>
%    \end{macrocode}
%
% \subsection{Drivers}
%
% \subsubsection{Driver \xoption{pdftex}}
%
%    \begin{macrocode}
%<*pdftex>
\NeedsTeXFormat{LaTeX2e}
\ProvidesFile{accsupp-pdftex.def}%
  [2018/01/30 v0.5 accsupp driver for pdfTeX (HO)]%
%    \end{macrocode}
%
%    \begin{macrocode}
\def\ACCSUPP@bdc{%
  \pdfliteral\ACCSUPP@pdfliteral{\ACCSUPP@span BDC}%
}
\def\ACCSUPP@emc{%
  \pdfliteral\ACCSUPP@pdfliteral{EMC}%
}
%    \end{macrocode}
%    \begin{macrocode}
%</pdftex>
%    \end{macrocode}
%
% \subsubsection{Driver \xoption{luatex}}
%
%    \begin{macrocode}
%<*luatex>
\NeedsTeXFormat{LaTeX2e}
\ProvidesFile{accsupp-luatex.def}%
  [2018/01/30 v0.5 accsupp driver for pdfTeX (HO)]%
%    \end{macrocode}
%
%    \begin{macrocode}
\protected\def\ACCSUPP@pdf@literal{\pdfextension literal}
\def\ACCSUPP@bdc{%
  \ACCSUPP@pdf@literal\ACCSUPP@pdfliteral{\ACCSUPP@span BDC}%
}
\def\ACCSUPP@emc{%
  \ACCSUPP@pdf@literal\ACCSUPP@pdfliteral{EMC}%
}
%    \end{macrocode}
%    \begin{macrocode}
%</luatex>
%    \end{macrocode}
%
% \subsubsection{Driver \xoption{dvipdfm}}
%
%    \begin{macrocode}
%<*dvipdfm>
\NeedsTeXFormat{LaTeX2e}
\ProvidesFile{accsupp-dvipdfm.def}%
  [2018/01/30 v0.5 accsupp driver for dvipdfm (HO)]%
%    \end{macrocode}
%
%    \begin{macrocode}
\def\ACCSUPP@bdc{%
  \special{pdf:content \ACCSUPP@span BDC}%
}
\def\ACCSUPP@emc{%
  \special{pdf:content EMC}%
}
%    \end{macrocode}
%    \begin{macrocode}
%</dvipdfm>
%    \end{macrocode}
%
% \subsubsection{Driver \xoption{dvips}}
%
%    \begin{macrocode}
%<*dvips>
\NeedsTeXFormat{LaTeX2e}
\ProvidesFile{accsupp-dvips.def}%
  [2018/01/30 v0.5 accsupp driver for dvips (HO)]%
%    \end{macrocode}
%
%    \begin{macrocode}
\def\ACCSUPP@bdc{%
  \special{ps:[\ACCSUPP@span/BDC pdfmark}%
}
\def\ACCSUPP@emc{%
  \special{ps:[/EMC pdfmark}%
}
%    \end{macrocode}
%    \begin{macrocode}
%</dvips>
%    \end{macrocode}
%
% \section{Test}
%
% \subsection{Catcode checks for loading}
%
%    \begin{macrocode}
%<*test1>
%    \end{macrocode}
%    \begin{macrocode}
\NeedsTeXFormat{LaTeX2e}
\documentclass{minimal}
\makeatletter
\def\RestoreCatcodes{}
\count@=0 %
\loop
  \edef\RestoreCatcodes{%
    \RestoreCatcodes
    \catcode\the\count@=\the\catcode\count@\relax
  }%
\ifnum\count@<255 %
  \advance\count@\@ne
\repeat

\def\RangeCatcodeInvalid#1#2{%
  \count@=#1\relax
  \loop
    \catcode\count@=15 %
  \ifnum\count@<#2\relax
    \advance\count@\@ne
  \repeat
}
\def\Test{%
  \RangeCatcodeInvalid{0}{47}%
  \RangeCatcodeInvalid{58}{64}%
  \RangeCatcodeInvalid{91}{96}%
  \RangeCatcodeInvalid{123}{127}%
  \catcode`\@=12 %
  \catcode`\\=0 %
  \catcode`\{=1 %
  \catcode`\}=2 %
  \catcode`\#=6 %
  \catcode`\[=12 %
  \catcode`\]=12 %
  \catcode`\%=14 %
  \catcode`\ =10 %
  \catcode13=5 %
  \RequirePackage{accsupp}[2018/01/30]\relax
  \RestoreCatcodes
}
\Test
\csname @@end\endcsname
\end
%    \end{macrocode}
%    \begin{macrocode}
%</test1>
%    \end{macrocode}
%
% \section{Installation}
%
% \subsection{Download}
%
% \paragraph{Package.} This package is available on
% CTAN\footnote{\url{http://ctan.org/pkg/accsupp}}:
% \begin{description}
% \item[\CTAN{macros/latex/contrib/oberdiek/accsupp.dtx}] The source file.
% \item[\CTAN{macros/latex/contrib/oberdiek/accsupp.pdf}] Documentation.
% \end{description}
%
%
% \paragraph{Bundle.} All the packages of the bundle `oberdiek'
% are also available in a TDS compliant ZIP archive. There
% the packages are already unpacked and the documentation files
% are generated. The files and directories obey the TDS standard.
% \begin{description}
% \item[\CTAN{install/macros/latex/contrib/oberdiek.tds.zip}]
% \end{description}
% \emph{TDS} refers to the standard ``A Directory Structure
% for \TeX\ Files'' (\CTAN{tds/tds.pdf}). Directories
% with \xfile{texmf} in their name are usually organized this way.
%
% \subsection{Bundle installation}
%
% \paragraph{Unpacking.} Unpack the \xfile{oberdiek.tds.zip} in the
% TDS tree (also known as \xfile{texmf} tree) of your choice.
% Example (linux):
% \begin{quote}
%   |unzip oberdiek.tds.zip -d ~/texmf|
% \end{quote}
%
% \paragraph{Script installation.}
% Check the directory \xfile{TDS:scripts/oberdiek/} for
% scripts that need further installation steps.
% Package \xpackage{attachfile2} comes with the Perl script
% \xfile{pdfatfi.pl} that should be installed in such a way
% that it can be called as \texttt{pdfatfi}.
% Example (linux):
% \begin{quote}
%   |chmod +x scripts/oberdiek/pdfatfi.pl|\\
%   |cp scripts/oberdiek/pdfatfi.pl /usr/local/bin/|
% \end{quote}
%
% \subsection{Package installation}
%
% \paragraph{Unpacking.} The \xfile{.dtx} file is a self-extracting
% \docstrip\ archive. The files are extracted by running the
% \xfile{.dtx} through \plainTeX:
% \begin{quote}
%   \verb|tex accsupp.dtx|
% \end{quote}
%
% \paragraph{TDS.} Now the different files must be moved into
% the different directories in your installation TDS tree
% (also known as \xfile{texmf} tree):
% \begin{quote}
% \def\t{^^A
% \begin{tabular}{@{}>{\ttfamily}l@{ $\rightarrow$ }>{\ttfamily}l@{}}
%   accsupp.sty & tex/latex/oberdiek/accsupp.sty\\
%   accsupp-pdftex.def & tex/latex/oberdiek/accsupp-pdftex.def\\
%   accsupp-luatex.def & tex/latex/oberdiek/accsupp-luatex.def\\
%   accsupp-dvipdfm.def & tex/latex/oberdiek/accsupp-dvipdfm.def\\
%   accsupp-dvips.def & tex/latex/oberdiek/accsupp-dvips.def\\
%   accsupp.pdf & doc/latex/oberdiek/accsupp.pdf\\
%   accsupp-example1.tex & doc/latex/oberdiek/accsupp-example1.tex\\
%   accsupp-example2.tex & doc/latex/oberdiek/accsupp-example2.tex\\
%   test/accsupp-test1.tex & doc/latex/oberdiek/test/accsupp-test1.tex\\
%   accsupp.dtx & source/latex/oberdiek/accsupp.dtx\\
% \end{tabular}^^A
% }^^A
% \sbox0{\t}^^A
% \ifdim\wd0>\linewidth
%   \begingroup
%     \advance\linewidth by\leftmargin
%     \advance\linewidth by\rightmargin
%   \edef\x{\endgroup
%     \def\noexpand\lw{\the\linewidth}^^A
%   }\x
%   \def\lwbox{^^A
%     \leavevmode
%     \hbox to \linewidth{^^A
%       \kern-\leftmargin\relax
%       \hss
%       \usebox0
%       \hss
%       \kern-\rightmargin\relax
%     }^^A
%   }^^A
%   \ifdim\wd0>\lw
%     \sbox0{\small\t}^^A
%     \ifdim\wd0>\linewidth
%       \ifdim\wd0>\lw
%         \sbox0{\footnotesize\t}^^A
%         \ifdim\wd0>\linewidth
%           \ifdim\wd0>\lw
%             \sbox0{\scriptsize\t}^^A
%             \ifdim\wd0>\linewidth
%               \ifdim\wd0>\lw
%                 \sbox0{\tiny\t}^^A
%                 \ifdim\wd0>\linewidth
%                   \lwbox
%                 \else
%                   \usebox0
%                 \fi
%               \else
%                 \lwbox
%               \fi
%             \else
%               \usebox0
%             \fi
%           \else
%             \lwbox
%           \fi
%         \else
%           \usebox0
%         \fi
%       \else
%         \lwbox
%       \fi
%     \else
%       \usebox0
%     \fi
%   \else
%     \lwbox
%   \fi
% \else
%   \usebox0
% \fi
% \end{quote}
% If you have a \xfile{docstrip.cfg} that configures and enables \docstrip's
% TDS installing feature, then some files can already be in the right
% place, see the documentation of \docstrip.
%
% \subsection{Refresh file name databases}
%
% If your \TeX~distribution
% (\teTeX, \mikTeX, \dots) relies on file name databases, you must refresh
% these. For example, \teTeX\ users run \verb|texhash| or
% \verb|mktexlsr|.
%
% \subsection{Some details for the interested}
%
% \paragraph{Attached source.}
%
% The PDF documentation on CTAN also includes the
% \xfile{.dtx} source file. It can be extracted by
% AcrobatReader 6 or higher. Another option is \textsf{pdftk},
% e.g. unpack the file into the current directory:
% \begin{quote}
%   \verb|pdftk accsupp.pdf unpack_files output .|
% \end{quote}
%
% \paragraph{Unpacking with \LaTeX.}
% The \xfile{.dtx} chooses its action depending on the format:
% \begin{description}
% \item[\plainTeX:] Run \docstrip\ and extract the files.
% \item[\LaTeX:] Generate the documentation.
% \end{description}
% If you insist on using \LaTeX\ for \docstrip\ (really,
% \docstrip\ does not need \LaTeX), then inform the autodetect routine
% about your intention:
% \begin{quote}
%   \verb|latex \let\install=y\input{accsupp.dtx}|
% \end{quote}
% Do not forget to quote the argument according to the demands
% of your shell.
%
% \paragraph{Generating the documentation.}
% You can use both the \xfile{.dtx} or the \xfile{.drv} to generate
% the documentation. The process can be configured by the
% configuration file \xfile{ltxdoc.cfg}. For instance, put this
% line into this file, if you want to have A4 as paper format:
% \begin{quote}
%   \verb|\PassOptionsToClass{a4paper}{article}|
% \end{quote}
% An example follows how to generate the
% documentation with pdf\LaTeX:
% \begin{quote}
%\begin{verbatim}
%pdflatex accsupp.dtx
%makeindex -s gind.ist accsupp.idx
%pdflatex accsupp.dtx
%makeindex -s gind.ist accsupp.idx
%pdflatex accsupp.dtx
%\end{verbatim}
% \end{quote}
%
% \section{Catalogue}
%
% The following XML file can be used as source for the
% \href{http://mirror.ctan.org/help/Catalogue/catalogue.html}{\TeX\ Catalogue}.
% The elements \texttt{caption} and \texttt{description} are imported
% from the original XML file from the Catalogue.
% The name of the XML file in the Catalogue is \xfile{accsupp.xml}.
%    \begin{macrocode}
%<*catalogue>
<?xml version='1.0' encoding='us-ascii'?>
<!DOCTYPE entry SYSTEM 'catalogue.dtd'>
<entry datestamp='$Date$' modifier='$Author$' id='accsupp'>
  <name>accsupp</name>
  <caption>Better accessibility support for PDF files.</caption>
  <authorref id='auth:oberdiek'/>
  <copyright owner='Heiko Oberdiek' year='2007,2010'/>
  <license type='lppl1.3'/>
  <version number='0.4'/>
  <description>
    Since PDF 1.5 portions of a page can be marked for better accessibility
    support. For example, replacement texts or expansions of abbreviations can
    be provided. This package starts with providing a minimal low-level
    interface for programmers; its status is experimental.
    <p/>
    The package is part of the <xref refid='oberdiek'>oberdiek</xref> bundle.
  </description>
  <documentation details='Package documentation'
      href='ctan:/macros/latex/contrib/oberdiek/accsupp.pdf'/>
  <ctan file='true' path='/macros/latex/contrib/oberdiek/accsupp.dtx'/>
  <miktex location='oberdiek'/>
  <texlive location='oberdiek'/>
  <install path='/macros/latex/contrib/oberdiek/oberdiek.tds.zip'/>
</entry>
%</catalogue>
%    \end{macrocode}
%
% \begin{thebibliography}{9}
% \bibitem{pdfspec}
% Adobe Systems Incorporated,
% \textit{PDF Reference}, 6th edition, 2006.
% \url{http://www.adobe.com/devnet/acrobat/pdfs/pdf_reference.pdf}
%
% \end{thebibliography}
%
% \begin{History}
%   \begin{Version}{2007/03/21 v0.1}
%   \item
%     First version.
%   \end{Version}
%   \begin{Version}{2007/11/14 v0.2}
%   \item
%     Various bug fixes.
%   \item
%     Catcode section rewritten, test added.
%   \end{Version}
%   \begin{Version}{2010/01/16 v0.3}
%   \item
%     \cs{AccSuppSetup} added.
%   \item
%     Options \xoption{space}, \xoption{spacefont}, \xoption{spacechar} added.
%   \end{Version}
%   \begin{Version}{2016/05/16 v0.4}
%   \item
%     Documentation updates.
%   \end{Version}
%   \begin{Version}{2018/01/30 v0.5}
%   \item
%     LuaTeX support added.
%   \end{Version}
% \end{History}
%
% \PrintIndex
%
% \Finale
\endinput
|
% \end{quote}
% Do not forget to quote the argument according to the demands
% of your shell.
%
% \paragraph{Generating the documentation.}
% You can use both the \xfile{.dtx} or the \xfile{.drv} to generate
% the documentation. The process can be configured by the
% configuration file \xfile{ltxdoc.cfg}. For instance, put this
% line into this file, if you want to have A4 as paper format:
% \begin{quote}
%   \verb|\PassOptionsToClass{a4paper}{article}|
% \end{quote}
% An example follows how to generate the
% documentation with pdf\LaTeX:
% \begin{quote}
%\begin{verbatim}
%pdflatex accsupp.dtx
%makeindex -s gind.ist accsupp.idx
%pdflatex accsupp.dtx
%makeindex -s gind.ist accsupp.idx
%pdflatex accsupp.dtx
%\end{verbatim}
% \end{quote}
%
% \section{Catalogue}
%
% The following XML file can be used as source for the
% \href{http://mirror.ctan.org/help/Catalogue/catalogue.html}{\TeX\ Catalogue}.
% The elements \texttt{caption} and \texttt{description} are imported
% from the original XML file from the Catalogue.
% The name of the XML file in the Catalogue is \xfile{accsupp.xml}.
%    \begin{macrocode}
%<*catalogue>
<?xml version='1.0' encoding='us-ascii'?>
<!DOCTYPE entry SYSTEM 'catalogue.dtd'>
<entry datestamp='$Date$' modifier='$Author$' id='accsupp'>
  <name>accsupp</name>
  <caption>Better accessibility support for PDF files.</caption>
  <authorref id='auth:oberdiek'/>
  <copyright owner='Heiko Oberdiek' year='2007,2010'/>
  <license type='lppl1.3'/>
  <version number='0.4'/>
  <description>
    Since PDF 1.5 portions of a page can be marked for better accessibility
    support. For example, replacement texts or expansions of abbreviations can
    be provided. This package starts with providing a minimal low-level
    interface for programmers; its status is experimental.
    <p/>
    The package is part of the <xref refid='oberdiek'>oberdiek</xref> bundle.
  </description>
  <documentation details='Package documentation'
      href='ctan:/macros/latex/contrib/oberdiek/accsupp.pdf'/>
  <ctan file='true' path='/macros/latex/contrib/oberdiek/accsupp.dtx'/>
  <miktex location='oberdiek'/>
  <texlive location='oberdiek'/>
  <install path='/macros/latex/contrib/oberdiek/oberdiek.tds.zip'/>
</entry>
%</catalogue>
%    \end{macrocode}
%
% \begin{thebibliography}{9}
% \bibitem{pdfspec}
% Adobe Systems Incorporated,
% \textit{PDF Reference}, 6th edition, 2006.
% \url{http://www.adobe.com/devnet/acrobat/pdfs/pdf_reference.pdf}
%
% \end{thebibliography}
%
% \begin{History}
%   \begin{Version}{2007/03/21 v0.1}
%   \item
%     First version.
%   \end{Version}
%   \begin{Version}{2007/11/14 v0.2}
%   \item
%     Various bug fixes.
%   \item
%     Catcode section rewritten, test added.
%   \end{Version}
%   \begin{Version}{2010/01/16 v0.3}
%   \item
%     \cs{AccSuppSetup} added.
%   \item
%     Options \xoption{space}, \xoption{spacefont}, \xoption{spacechar} added.
%   \end{Version}
%   \begin{Version}{2016/05/16 v0.4}
%   \item
%     Documentation updates.
%   \end{Version}
%   \begin{Version}{2018/01/30 v0.5}
%   \item
%     LuaTeX support added.
%   \end{Version}
% \end{History}
%
% \PrintIndex
%
% \Finale
\endinput

%        (quote the arguments according to the demands of your shell)
%
% Documentation:
%    (a) If accsupp.drv is present:
%           latex accsupp.drv
%    (b) Without accsupp.drv:
%           latex accsupp.dtx; ...
%    The class ltxdoc loads the configuration file ltxdoc.cfg
%    if available. Here you can specify further options, e.g.
%    use A4 as paper format:
%       \PassOptionsToClass{a4paper}{article}
%
%    Programm calls to get the documentation (example):
%       pdflatex accsupp.dtx
%       makeindex -s gind.ist accsupp.idx
%       pdflatex accsupp.dtx
%       makeindex -s gind.ist accsupp.idx
%       pdflatex accsupp.dtx
%
% Installation:
%    TDS:tex/latex/oberdiek/accsupp.sty
%    TDS:tex/latex/oberdiek/accsupp-pdftex.def
%    TDS:tex/latex/oberdiek/accsupp-luatex.def
%    TDS:tex/latex/oberdiek/accsupp-dvipdfm.def
%    TDS:tex/latex/oberdiek/accsupp-dvips.def
%    TDS:doc/latex/oberdiek/accsupp.pdf
%    TDS:doc/latex/oberdiek/accsupp-example1.tex
%    TDS:doc/latex/oberdiek/accsupp-example2.tex
%    TDS:doc/latex/oberdiek/test/accsupp-test1.tex
%    TDS:source/latex/oberdiek/accsupp.dtx
%
%<*ignore>
\begingroup
  \catcode123=1 %
  \catcode125=2 %
  \def\x{LaTeX2e}%
\expandafter\endgroup
\ifcase 0\ifx\install y1\fi\expandafter
         \ifx\csname processbatchFile\endcsname\relax\else1\fi
         \ifx\fmtname\x\else 1\fi\relax
\else\csname fi\endcsname
%</ignore>
%<*install>
\input docstrip.tex
\Msg{************************************************************************}
\Msg{* Installation}
\Msg{* Package: accsupp 2018/01/30 v0.5 Accessibility support by marked content (HO)}
\Msg{************************************************************************}

\keepsilent
\askforoverwritefalse

\let\MetaPrefix\relax
\preamble

This is a generated file.

Project: accsupp
Version: 2018/01/30 v0.5

Copyright (C) 2007, 2010 by
   Heiko Oberdiek <heiko.oberdiek at googlemail.com>

This work may be distributed and/or modified under the
conditions of the LaTeX Project Public License, either
version 1.3c of this license or (at your option) any later
version. This version of this license is in
   http://www.latex-project.org/lppl/lppl-1-3c.txt
and the latest version of this license is in
   http://www.latex-project.org/lppl.txt
and version 1.3 or later is part of all distributions of
LaTeX version 2005/12/01 or later.

This work has the LPPL maintenance status "maintained".

This Current Maintainer of this work is Heiko Oberdiek.

This work consists of the main source file accsupp.dtx
and the derived files
   accsupp.sty, accsupp.pdf, accsupp.ins, accsupp.drv,
   accsupp-pdftex.def, accsupp-luatex.def, accsupp-dvipdfm.def,
   accsupp-dvips.def,
   accsupp-example1.tex, accsupp-example2.tex,
   accsupp-test1.tex.

\endpreamble
\let\MetaPrefix\DoubleperCent

\generate{%
  \file{accsupp.ins}{\from{accsupp.dtx}{install}}%
  \file{accsupp.drv}{\from{accsupp.dtx}{driver}}%
  \usedir{tex/latex/oberdiek}%
  \file{accsupp.sty}{\from{accsupp.dtx}{package}}%
  \file{accsupp-pdftex.def}{\from{accsupp.dtx}{pdftex}}%
  \file{accsupp-luatex.def}{\from{accsupp.dtx}{luatex}}%
  \file{accsupp-dvipdfm.def}{\from{accsupp.dtx}{dvipdfm}}%
  \file{accsupp-dvips.def}{\from{accsupp.dtx}{dvips}}%
  \usedir{doc/latex/oberdiek}%
  \file{accsupp-example1.tex}{\from{accsupp.dtx}{example1}}%
  \file{accsupp-example2.tex}{\from{accsupp.dtx}{example2}}%
  \usedir{doc/latex/oberdiek/test}%
  \file{accsupp-test1.tex}{\from{accsupp.dtx}{test1}}%
  \nopreamble
  \nopostamble
  \usedir{source/latex/oberdiek/catalogue}%
  \file{accsupp.xml}{\from{accsupp.dtx}{catalogue}}%
}

\catcode32=13\relax% active space
\let =\space%
\Msg{************************************************************************}
\Msg{*}
\Msg{* To finish the installation you have to move the following}
\Msg{* files into a directory searched by TeX:}
\Msg{*}
\Msg{*     accsupp.sty, accsupp-pdftex.def, accsupp-dvipdfm.def,}
\Msg{*     accsupp-dvips.def}
\Msg{*}
\Msg{* To produce the documentation run the file `accsupp.drv'}
\Msg{* through LaTeX.}
\Msg{*}
\Msg{* Happy TeXing!}
\Msg{*}
\Msg{************************************************************************}

\endbatchfile
%</install>
%<*ignore>
\fi
%</ignore>
%<*driver>
\NeedsTeXFormat{LaTeX2e}
\ProvidesFile{accsupp.drv}%
  [2018/01/30 v0.5 Accessibility support by marked content (HO)]%
\documentclass{ltxdoc}
\usepackage{holtxdoc}[2011/11/22]
\begin{document}
  \DocInput{accsupp.dtx}%
\end{document}
%</driver>
% \fi
%
%
% \CharacterTable
%  {Upper-case    \A\B\C\D\E\F\G\H\I\J\K\L\M\N\O\P\Q\R\S\T\U\V\W\X\Y\Z
%   Lower-case    \a\b\c\d\e\f\g\h\i\j\k\l\m\n\o\p\q\r\s\t\u\v\w\x\y\z
%   Digits        \0\1\2\3\4\5\6\7\8\9
%   Exclamation   \!     Double quote  \"     Hash (number) \#
%   Dollar        \$     Percent       \%     Ampersand     \&
%   Acute accent  \'     Left paren    \(     Right paren   \)
%   Asterisk      \*     Plus          \+     Comma         \,
%   Minus         \-     Point         \.     Solidus       \/
%   Colon         \:     Semicolon     \;     Less than     \<
%   Equals        \=     Greater than  \>     Question mark \?
%   Commercial at \@     Left bracket  \[     Backslash     \\
%   Right bracket \]     Circumflex    \^     Underscore    \_
%   Grave accent  \`     Left brace    \{     Vertical bar  \|
%   Right brace   \}     Tilde         \~}
%
% \GetFileInfo{accsupp.drv}
%
% \title{The \xpackage{accsupp} package}
% \date{2018/01/30 v0.5}
% \author{Heiko Oberdiek\thanks
% {Please report any issues at https://github.com/ho-tex/oberdiek/issues}\\
% \xemail{heiko.oberdiek at googlemail.com}}
%
% \maketitle
%
% \begin{abstract}
% Since PDF 1.5 portions of a page can be marked for better
% accessibility support.
% For example, replacement texts or expansions of abbreviations can be
% provided. Package \xpackage{accsupp} starts with providing a minimal
% low-level interface for programmers. Status is experimental.
% \end{abstract}
%
% \tableofcontents
%
% \section{Documentation}
%
% \subsection{Macros}
%
% Section ``10.8 Accessibility Support'' of the PDF reference
% \cite{pdfspec} lists some features that can be added by
% operators for marked content.
%
% \begin{declcs}{BeginAccSupp} \M{options}
% \end{declcs}
% It puts the operator |BDC| in the page stream:
% \begin{quote}
%   |/Span|\\
%   |<<|\dots|>>| \qquad \textit{\% property dictionary}\\
%   |BDC|
% \end{quote}
% The contents of the dictionary is controlled by \meta{options}.
% See sections \ref{sec:feature} and \ref{sec:method}.
%
% \begin{declcs}{EndAccSupp} \M{options}
% \end{declcs}
% It puts the operator |EMC| in the page stream.
% The only option is \xoption{pdfliteral}, see section \ref{sec:pdftex}.
%
% \bigskip
% \noindent\textbf{Note:}
% The caller is responsible for the placement of \cs{BeginAccSupp}
% and \cs{EndAccSupp} pairs. Especially page breaks are not allowed
% in between.
%
% \subsubsection{Feature options}\label{sec:feature}
%
% The PDF reference \cite{pdfspec} describes and explains the different
% features. The names of the feature options are the same as the
% key names for the property dictionary for operator |BDC|, see
% \cs{BeginAccSupp}.
%
% \begin{description}
% \item[\xoption{ActualText}:]
%   Provides a replacement text, see examples in
%   section \ref{sec:example}.
% \item[\xoption{Alt}:]
%   Provides an alternate description.
% \item[\xoption{E}:]
%   Provides the expansion of an abbreviation or an acronym.
% \item[\xoption{Lang}:]
%   Specifies the language.
% \end{description}
%
% \subsubsection{Input methods}\label{sec:method}
%
% Except for \xoption{Lang} option \xoption{method} controls
% how the argument for \xoption{ActualText}, \xoption{Alt},
% and \xoption{E} are interpreted.
% \begin{description}
% \item[\xoption{method=plain}:]
%   The string is only expanded and written without further
%   treatment. Special characters are not protected, thus this method
%   may result in an invalid PDF file.
% \item[\xoption{mehod=escape}:]
%   The string is expanded and special characters are escaped.
%   The result is a valid PDF string.
% \item[\xoption{method=hex}:]
%   The string is given in hexadecimal notation.
%   Section \ref{sec:example:notparallel} shows an example.
% \item[\xoption{method=pdfstringdef}:]
%   If package \xpackage{hyperref} is loaded, then its \cs{pdfstringdef}
%   is used. This method is slow, but useful if the string contains
%   arbitrary \hologo{TeX} code.
% \item[\xoption{unicode}:]
%   This option is needed, if the string is given as Unicode string
%   (16 bit). Internally it adds the string prefix for Unicode.
%   In case of \xoption{method=pdfstringdef} it passes the option
%   to \cs{hypersetup}.
% \end{description}
%
% \subsection{Workaround, option \xoption{space}}
%
% PDF specification says in ``10.8.3 Replacement Text''
% \begin{quote}
% Just as alternative descriptions can be provided for images and other
% items that do not translate naturally into text (\dots), replacement
% text can be specified for content that does translate into text but
% that is represented in a nonstandard way. These nonstandard representations
% might include, for example, glyphs for ligatures or custom characters, or
% inline graphics corresponding to letters in an illuminated manustript or to
% dropped capitals.
% \end{quote}
% However, the ActualText is ignored in Acrobat Reader
% (until version 9 at least), if the content does not contain glyphs.
% Option \xoption{space} adds such an invisible glyph, a space
% character. The font name can be configured by option \xoption{spacefont},
% the default font is \texttt{phvr8r}. The character slot is given
% by option \xoption{spacechar}, default is \texttt{32}, the usual
% position of the space character.
%
% These options can also be given as package options or in
% macro \cs{AccSuppSetup} that takes a key value list as argument.
% Usually only option \xoption{space} is necessary, if the
% contents does not contain glyphs otherwise. Option \xoption{space}
% is enabled by |space| or |space=true| and disabled by |space=false|.
% It is disabled as default. The option is evaluated by \cs{BeginAccSupp}
% and ignored by \cs{EndAccSupp}.
%
% Note: Even with option \xoption{space} I get sometimes wrong texts
% when cut \& paste from AR7/Linux or AR8/Linux, e.g.
% \begin{quote}
%   |Hello| $\rightarrow$ |Helo|, |Helol|, |Hell|, \dots\\
%   |Hello World| $\rightarrow$ |Helo WorldW|, |Helo World|, \dots
% \end{quote}
% I do not know what Acrobat Reader is doing here, thus feedback and
% insights are welcome.
%
% \subsection{Driver options}
%
% Driver options are package options only.
% The special \hologo{TeX} compilers \hologo{pdfTeX}, \hologo{LuaTeX} and \hologo{XeTeX}
% are detected automatically. The default for unrecognized drivers can be
% set by defining \cs{ActualTextDriverDefault}. This can be done
% in the configuration file \xfile{accsupp.cfg}.
%
% \subsubsection{Option \xoption{pdftex}}\label{sec:pdftex}
% Package option \xoption{pdftex} is used for \hologo{pdfTeX} in PDF mode.
% Additionally \cs{BeginAccSupp} and \cs{EndAccSupp} understand
% option \xoption{pdfliteral}. It controls
% the modifier keyword for \cs{pdfliteral}:
% \begin{quote}
%   |pdfliteral=direct| $\Rightarrow$ |\pdfliteral direct{...}|
% \end{quote}
%
% \subsubsection{Option \xoption{luatex}}\label{sec:luatex}
% Package option \xoption{luatex} is used for \hologo{LuaTeX} in PDF mode.
% Additionally \cs{BeginAccSupp} and \cs{EndAccSupp} understand
% option \xoption{pdfliteral}. It controls
% the modifier keyword for \cs{pdfextension} literal:
% \begin{quote}
%   |pdfliteral=direct| $\Rightarrow$ |\pdfextension literal direct{...}|
% \end{quote}
%
% \subsubsection{Option \xoption{dvipdfm}}
%
% Package option \xoption{dvipdfm} and its aliases \xoption{dvipdfmx}
% \xoption{xetex} are used for drivers that support
% \textsf{dvipdfm} specials.
%
% \subsubsection{Option \xoption{dvips}}
%
% Package option \xoption{dvips} and its alias \xoption{dvipsone}
% write pdfmark specials in the output. Unhappily these pdfmark
% operators are ignored by \textsf{ghostscript} (latest tested
% version is 8.54). Perhaps they are recognized by commercial
% distiller applications.
%
% \subsubsection{Turning off page stream compression}
%
% For debugging it is useful to have uncompressed page stream objects.
% This can be done afterwards via \textsf{pdftk}:
% \begin{quote}
%   |pdftk file.pdf output file-uncompressed.pdf uncompress|
% \end{quote}
% Or the PDF file is generated uncompressed:
% \begin{description}
% \item[\hologo{pdfTeX}:] |\pdfcompresslevel=0|
% \item[dvipdfm:] |dvipdfm -z0 ...|
% \item[dvipdfmx:] |dvipdfmx -z0 ...|
% \item[ghostscript/ps2pdf:]
%      |ps2pdf -dCompressPages=false input.ps output.pdf|
% \end{description}
%
% \section{Example}
% \label{sec:example}
%
% \subsection{Example \cs{notparallel}}
% \label{sec:example:notparallel}
%
%    \begin{macrocode}
%<*example1>
%    \end{macrocode}
%    \begin{macrocode}
%<<END
\documentclass{article}
\usepackage{accsupp}[2007/11/14]
\usepackage{centernot}
% U+2226 NOT PARALLEL
% \mathrel{...} prevents page break in between
\newcommand*{\notparallel}{%
  \ensuremath{%
    \mathrel{%
      \BeginAccSupp{method=hex,unicode,ActualText=2226}%
      \centernot{\parallel}%
      \EndAccSupp{}%
    }%
  }%
}
\begin{document}
\begin{equation}
A\notparallel B
\end{equation}
\end{document}
%END
%    \end{macrocode}
%    \begin{macrocode}
%</example1>
%    \end{macrocode}
%
% \subsection{Example with \xoption{pdfstringdef}}
%
%    \begin{macrocode}
%<*example2>
%    \end{macrocode}
%    \begin{macrocode}
%<<END
\documentclass{article}
\usepackage[unicode]{hyperref}
\usepackage{accsupp}[2007/11/14]
\begin{document}
  \begin{equation}
    \BeginAccSupp{
      method=pdfstringdef,
      unicode,
      ActualText={%
        a\texttwosuperior +b\texttwosuperior
        =c\texttwosuperior
      }
    }
    a^2 + b^2 = c^2
    \EndAccSupp{}
  \end{equation}
\end{document}
%END
%    \end{macrocode}
%    \begin{macrocode}
%</example2>
%    \end{macrocode}
%
% \StopEventually{
% }
%
% \section{Implementation}
%
% \subsection{Package}
%
%    \begin{macrocode}
%<*package>
%    \end{macrocode}
%
%    \begin{macrocode}
\begingroup\catcode61\catcode48\catcode32=10\relax%
  \catcode13=5 % ^^M
  \endlinechar=13 %
  \catcode123=1 % {
  \catcode125=2 % }
  \catcode64=11 % @
  \def\x{\endgroup
    \expandafter\edef\csname ACCSUPP@AtEnd\endcsname{%
      \endlinechar=\the\endlinechar\relax
      \catcode13=\the\catcode13\relax
      \catcode32=\the\catcode32\relax
      \catcode35=\the\catcode35\relax
      \catcode61=\the\catcode61\relax
      \catcode64=\the\catcode64\relax
      \catcode123=\the\catcode123\relax
      \catcode125=\the\catcode125\relax
    }%
  }%
\x\catcode61\catcode48\catcode32=10\relax%
\catcode13=5 % ^^M
\endlinechar=13 %
\catcode35=6 % #
\catcode64=11 % @
\catcode123=1 % {
\catcode125=2 % }
\def\TMP@EnsureCode#1#2{%
  \edef\ACCSUPP@AtEnd{%
    \ACCSUPP@AtEnd
    \catcode#1=\the\catcode#1\relax
  }%
  \catcode#1=#2\relax
}
\TMP@EnsureCode{10}{12}% ^^J
\TMP@EnsureCode{33}{12}% !
\TMP@EnsureCode{39}{12}% '
\TMP@EnsureCode{40}{12}% (
\TMP@EnsureCode{41}{12}% )
\TMP@EnsureCode{42}{12}% *
\TMP@EnsureCode{44}{12}% ,
\TMP@EnsureCode{45}{12}% -
\TMP@EnsureCode{46}{12}% .
\TMP@EnsureCode{47}{12}% /
\TMP@EnsureCode{58}{12}% :
\TMP@EnsureCode{60}{12}% <
\TMP@EnsureCode{62}{12}% >
\TMP@EnsureCode{94}{7}% ^ (superscript)
\TMP@EnsureCode{96}{12}% `
\TMP@EnsureCode{254}{12}% ^^fe
\TMP@EnsureCode{255}{12}% ^^ff
\edef\ACCSUPP@AtEnd{\ACCSUPP@AtEnd\noexpand\endinput}
%    \end{macrocode}
%
%    Package identification.
%    \begin{macrocode}
\NeedsTeXFormat{LaTeX2e}
\ProvidesPackage{accsupp}%
  [2018/01/30 v0.5 Accessibility support by marked content (HO)]
%    \end{macrocode}
%
%    \begin{macrocode}
\RequirePackage{pdfescape}[2007/02/25]
\RequirePackage{ifpdf}
\RequirePackage{ifxetex}
\RequirePackage{kvoptions}
%    \end{macrocode}
%    \begin{macrocode}
\SetupKeyvalOptions{%
  family=ACCSUPP,%
  prefix=ACCSUPP@%
}
%    \end{macrocode}
%
% \subsection{Space setup}
%
%    \begin{macrocode}
\DeclareBoolOption{space}
\DeclareStringOption[phvr8r]{spacefont}
\DeclareStringOption[32]{spacechar}
%    \end{macrocode}
%
% \subsection{Driver detection and setup}
%
% \paragraph{Driver declarations.}
%
%    \begin{macrocode}
\def\ACCSUPP@DefineDriverKey{%
  \@dblarg\ACCSUPP@@DefineDriverKey
}
\def\ACCSUPP@@DefineDriverKey[#1]#2{%
  \define@key{ACCSUPP}{#2}[]{%
    \def\ACCSUPP@driver{#1}%
  }%
  \g@addto@macro\ACCSUPP@DisableOptions{%
    \DisableKeyvalOption{ACCSUPP}{#2}%
  }%
}
\let\ACCSUPP@DisableOptions\@empty
\ACCSUPP@DefineDriverKey{pdftex}
\ACCSUPP@DefineDriverKey{luatex}
\ACCSUPP@DefineDriverKey{dvips}
\ACCSUPP@DefineDriverKey[dvips]{dvipsone}
\ACCSUPP@DefineDriverKey{dvipdfm}
\ACCSUPP@DefineDriverKey[dvipdfm]{dvipdfmx}
\ACCSUPP@DefineDriverKey[dvipdfm]{xetex}
\let\ACCSUPP@driver\relax
\InputIfFileExists{accsupp.cfg}{}{}
\providecommand*{\ActualTextDriverDefault}{dvips}
\ifpdf
  \ifx\pdfextension\@undefined
    \def\ACCSUPP@driver{pdftex}%
  \else
    \def\ACCSUPP@driver{luatex}%
  \fi
\else
  \ifxetex
    \def\ACCSUPP@driver{dvipdfm}%
  \else
    \ifx\ACCSUPP@driver\relax
      \let\ACCSUPP@driver\ActualTextDriverDefault
    \fi
  \fi
\fi
%    \end{macrocode}
%
% \paragraph{Process options.}
%
%    \begin{macrocode}
\ProcessKeyvalOptions*
\ACCSUPP@DisableOptions
%    \end{macrocode}
%
% \paragraph{Driver validation and loading.}
%
%    \begin{macrocode}
\def\ACCSUPP@temp{pdftex}%
\ifpdf
  \ifx\pdfextension\@undefined\else
    \def\ACCSUPP@temp{luatex}%
  \fi
  \ifx\ACCSUPP@temp\ACCSUPP@driver
  \else
    \PackageWarningNoLine{accsupp}{%
      Wrong driver `\ACCSUPP@driver', using `pdftex' instead%
    }%
ed as
    \let\ACCSUPP@driver\ACCSUPP@temp
  \fi
\else
  \ifx\ACCSUPP@temp\ACCSUPP@driver
    \PackageError{accsupp}{%
      Wrong driver, pdfTeX is not running in PDF mode.\MessageBreak
      Package loading is aborted%
    }\@ehc
    \expandafter\expandafter\expandafter\ACCSUPP@AtEnd
  \fi
  \def\ACCSUPP@temp{dvipdfm}%
  \ifxetex
    \ifx\ACCSUPP@temp\ACCSUPP@driver
    \else
      \PackageWarningNoLine{accsupp}{%
        Wrong driver `\ACCSUPP@driver',\MessageBreak
        using `dvipdfm' for XeTeX instead%
      }%
      \let\ACCSUPP@driver\ACCSUPP@temp
    \fi
  \fi
\fi%
\ifx\ACCSUPP@driver\relax
  \PackageError{accsupp}{%
    Missing driver option.\MessageBreak
    Package loading is aborted%
  }\@ehc
  \expandafter\ACCSUPP@AtEnd
\fi%
\InputIfFileExists{accsupp-\ACCSUPP@driver.def}{}{%
  \PackageError{accsupp}{%
    Unsupported driver `\ACCSUPP@driver'.\MessageBreak
    Package loading is aborted%
  }\@ehc
  \ACCSUPP@AtEnd
}%
%    \end{macrocode}
%
% \subsection{Main macro}
%
%    \begin{macrocode}
\DeclareBoolOption{unicode}
\DeclareStringOption[page]{pdfliteral}
\DeclareStringOption{Lang}
\def\ACCSUPP@method{escape}
\define@key{ACCSUPP}{method}{%
  \@ifundefined{ACCSUPP@method@#1}{%
    \PackageError{accsupp}{%
      Ignoring unknown method `#1'%
    }\@ehc
  }{%
    \edef\ACCSUPP@method{#1}%
  }%
}
\let\ACCSUPP@Lang\relax
\def\ACCSUPP@temp#1{%
  \expandafter\ACCSUPP@@temp\csname ACCSUPP@#1\endcsname{#1}%
}
\def\ACCSUPP@@temp#1#2{%
  \let#1\relax
  \define@key{ACCSUPP}{#2}{%
    \def#1{##1}%
    \ifx#1\@empty
      \def#1{()}%
    \else
      \csname ACCSUPP@method@\ACCSUPP@method\endcsname#1%
    \fi
  }%
}
\ACCSUPP@temp{Alt}
\ACCSUPP@temp{ActualText}
\ACCSUPP@temp{E}
%    \end{macrocode}
%
%    \begin{macrocode}
\newcommand*{\BeginAccSupp}[1]{%
  \begingroup
    \setkeys{ACCSUPP}{#1}%
    \edef\ACCSUPP@span{%
      /Span<<%
        \ifx\ACCSUPP@Lang\relax
        \else
          /Lang\ACCSUPP@Lang
        \fi
        \ifx\ACCSUPP@Alt\relax
        \else
          /Alt\ACCSUPP@Alt
        \fi
        \ifx\ACCSUPP@ActualText\relax
        \else
          /ActualText\ACCSUPP@ActualText
        \fi
        \ifx\ACCSUPP@E\relax
        \else
          /E\ACCSUPP@E
        \fi
      >>%
    }%
    \ACCSUPP@bdc
    \ACCSUPP@space
  \endgroup
}
\newcommand*{\EndAccSupp}[1]{%
  \begingroup
    \setkeys{ACCSUPP}{#1}%
    \ACCSUPP@emc
  \endgroup
}
\def\ACCSUPP@space{%
  \ifACCSUPP@space
    \begingroup
      \@ifundefined{ACCSUPP@Font}{%
        \global\font\ACCSUPP@Font=\ACCSUPP@spacefont\relax
      }{}%
      \leavevmode
      \setbox\z@\hbox{\ACCSUPP@Font\char\ACCSUPP@spacechar}%
      \wd\z@\z@
      \ht\z@\z@
      \dp\z@\z@
      \copy\z@
    \endgroup
  \fi
}
\newcommand*{\AccSuppSetup}{%
  \setkeys{ACCSUPP}%
}
%    \end{macrocode}
%
% \subsubsection{Input methods}
%
%    \begin{macrocode}
\def\ACCSUPP@method@plain#1{%
  \csname @safe@activestrue\endcsname
  \edef#1{%
    (%
    \ifACCSUPP@unicode
      \string\376\string\377%
    \fi
    #1%
    )%
  }%
  \@onelevel@sanitize#1%
}
%    \end{macrocode}
%    \begin{macrocode}
\def\ACCSUPP@method@escape#1{%
  \EdefEscapeString#1{%
    \ifACCSUPP@unicode
      ^^fe^^ff%
    \fi
    #1%
  }%
  \edef#1{(#1)}%
}%
%    \end{macrocode}
%    \begin{macrocode}
\def\ACCSUPP@method@hex#1{%
  \edef#1{%
    <%
    \ifACCSUPP@unicode
      FEFF%
    \fi
    #1%
    >%
  }%
}
%    \end{macrocode}
%    \begin{macrocode}
\def\ACCSUPP@method@pdfstringdef#1{%
  \ifACCSUPP@unicode
    \@ifundefined{hypersetup}{}{%
      \hypersetup{unicode}%
    }%
  \fi
  \@ifundefined{pdfstringdef}{%
    \PackageError{accsupp}{%
      Method `pdfstringdef' requires package `hyperref'%
    }\@ehc
    \let\ACCSUPP@temp\@empty
  }{%
    \begingroup
      \setbox0=\hbox{%
        \pdfstringdef\ACCSUPP@temp#1%
        \global\let\ACCSUPP@temp\ACCSUPP@temp
      }%
    \endgroup
  }%
  \edef#1{(\ACCSUPP@temp)}%
}
%    \end{macrocode}
%
%    \begin{macrocode}
\ACCSUPP@AtEnd%
%</package>
%    \end{macrocode}
%
% \subsection{Drivers}
%
% \subsubsection{Driver \xoption{pdftex}}
%
%    \begin{macrocode}
%<*pdftex>
\NeedsTeXFormat{LaTeX2e}
\ProvidesFile{accsupp-pdftex.def}%
  [2018/01/30 v0.5 accsupp driver for pdfTeX (HO)]%
%    \end{macrocode}
%
%    \begin{macrocode}
\def\ACCSUPP@bdc{%
  \pdfliteral\ACCSUPP@pdfliteral{\ACCSUPP@span BDC}%
}
\def\ACCSUPP@emc{%
  \pdfliteral\ACCSUPP@pdfliteral{EMC}%
}
%    \end{macrocode}
%    \begin{macrocode}
%</pdftex>
%    \end{macrocode}
%
% \subsubsection{Driver \xoption{luatex}}
%
%    \begin{macrocode}
%<*luatex>
\NeedsTeXFormat{LaTeX2e}
\ProvidesFile{accsupp-luatex.def}%
  [2018/01/30 v0.5 accsupp driver for pdfTeX (HO)]%
%    \end{macrocode}
%
%    \begin{macrocode}
\protected\def\ACCSUPP@pdf@literal{\pdfextension literal}
\def\ACCSUPP@bdc{%
  \ACCSUPP@pdf@literal\ACCSUPP@pdfliteral{\ACCSUPP@span BDC}%
}
\def\ACCSUPP@emc{%
  \ACCSUPP@pdf@literal\ACCSUPP@pdfliteral{EMC}%
}
%    \end{macrocode}
%    \begin{macrocode}
%</luatex>
%    \end{macrocode}
%
% \subsubsection{Driver \xoption{dvipdfm}}
%
%    \begin{macrocode}
%<*dvipdfm>
\NeedsTeXFormat{LaTeX2e}
\ProvidesFile{accsupp-dvipdfm.def}%
  [2018/01/30 v0.5 accsupp driver for dvipdfm (HO)]%
%    \end{macrocode}
%
%    \begin{macrocode}
\def\ACCSUPP@bdc{%
  \special{pdf:content \ACCSUPP@span BDC}%
}
\def\ACCSUPP@emc{%
  \special{pdf:content EMC}%
}
%    \end{macrocode}
%    \begin{macrocode}
%</dvipdfm>
%    \end{macrocode}
%
% \subsubsection{Driver \xoption{dvips}}
%
%    \begin{macrocode}
%<*dvips>
\NeedsTeXFormat{LaTeX2e}
\ProvidesFile{accsupp-dvips.def}%
  [2018/01/30 v0.5 accsupp driver for dvips (HO)]%
%    \end{macrocode}
%
%    \begin{macrocode}
\def\ACCSUPP@bdc{%
  \special{ps:[\ACCSUPP@span/BDC pdfmark}%
}
\def\ACCSUPP@emc{%
  \special{ps:[/EMC pdfmark}%
}
%    \end{macrocode}
%    \begin{macrocode}
%</dvips>
%    \end{macrocode}
%
% \section{Test}
%
% \subsection{Catcode checks for loading}
%
%    \begin{macrocode}
%<*test1>
%    \end{macrocode}
%    \begin{macrocode}
\NeedsTeXFormat{LaTeX2e}
\documentclass{minimal}
\makeatletter
\def\RestoreCatcodes{}
\count@=0 %
\loop
  \edef\RestoreCatcodes{%
    \RestoreCatcodes
    \catcode\the\count@=\the\catcode\count@\relax
  }%
\ifnum\count@<255 %
  \advance\count@\@ne
\repeat

\def\RangeCatcodeInvalid#1#2{%
  \count@=#1\relax
  \loop
    \catcode\count@=15 %
  \ifnum\count@<#2\relax
    \advance\count@\@ne
  \repeat
}
\def\Test{%
  \RangeCatcodeInvalid{0}{47}%
  \RangeCatcodeInvalid{58}{64}%
  \RangeCatcodeInvalid{91}{96}%
  \RangeCatcodeInvalid{123}{127}%
  \catcode`\@=12 %
  \catcode`\\=0 %
  \catcode`\{=1 %
  \catcode`\}=2 %
  \catcode`\#=6 %
  \catcode`\[=12 %
  \catcode`\]=12 %
  \catcode`\%=14 %
  \catcode`\ =10 %
  \catcode13=5 %
  \RequirePackage{accsupp}[2018/01/30]\relax
  \RestoreCatcodes
}
\Test
\csname @@end\endcsname
\end
%    \end{macrocode}
%    \begin{macrocode}
%</test1>
%    \end{macrocode}
%
% \section{Installation}
%
% \subsection{Download}
%
% \paragraph{Package.} This package is available on
% CTAN\footnote{\url{http://ctan.org/pkg/accsupp}}:
% \begin{description}
% \item[\CTAN{macros/latex/contrib/oberdiek/accsupp.dtx}] The source file.
% \item[\CTAN{macros/latex/contrib/oberdiek/accsupp.pdf}] Documentation.
% \end{description}
%
%
% \paragraph{Bundle.} All the packages of the bundle `oberdiek'
% are also available in a TDS compliant ZIP archive. There
% the packages are already unpacked and the documentation files
% are generated. The files and directories obey the TDS standard.
% \begin{description}
% \item[\CTAN{install/macros/latex/contrib/oberdiek.tds.zip}]
% \end{description}
% \emph{TDS} refers to the standard ``A Directory Structure
% for \TeX\ Files'' (\CTAN{tds/tds.pdf}). Directories
% with \xfile{texmf} in their name are usually organized this way.
%
% \subsection{Bundle installation}
%
% \paragraph{Unpacking.} Unpack the \xfile{oberdiek.tds.zip} in the
% TDS tree (also known as \xfile{texmf} tree) of your choice.
% Example (linux):
% \begin{quote}
%   |unzip oberdiek.tds.zip -d ~/texmf|
% \end{quote}
%
% \paragraph{Script installation.}
% Check the directory \xfile{TDS:scripts/oberdiek/} for
% scripts that need further installation steps.
% Package \xpackage{attachfile2} comes with the Perl script
% \xfile{pdfatfi.pl} that should be installed in such a way
% that it can be called as \texttt{pdfatfi}.
% Example (linux):
% \begin{quote}
%   |chmod +x scripts/oberdiek/pdfatfi.pl|\\
%   |cp scripts/oberdiek/pdfatfi.pl /usr/local/bin/|
% \end{quote}
%
% \subsection{Package installation}
%
% \paragraph{Unpacking.} The \xfile{.dtx} file is a self-extracting
% \docstrip\ archive. The files are extracted by running the
% \xfile{.dtx} through \plainTeX:
% \begin{quote}
%   \verb|tex accsupp.dtx|
% \end{quote}
%
% \paragraph{TDS.} Now the different files must be moved into
% the different directories in your installation TDS tree
% (also known as \xfile{texmf} tree):
% \begin{quote}
% \def\t{^^A
% \begin{tabular}{@{}>{\ttfamily}l@{ $\rightarrow$ }>{\ttfamily}l@{}}
%   accsupp.sty & tex/latex/oberdiek/accsupp.sty\\
%   accsupp-pdftex.def & tex/latex/oberdiek/accsupp-pdftex.def\\
%   accsupp-luatex.def & tex/latex/oberdiek/accsupp-luatex.def\\
%   accsupp-dvipdfm.def & tex/latex/oberdiek/accsupp-dvipdfm.def\\
%   accsupp-dvips.def & tex/latex/oberdiek/accsupp-dvips.def\\
%   accsupp.pdf & doc/latex/oberdiek/accsupp.pdf\\
%   accsupp-example1.tex & doc/latex/oberdiek/accsupp-example1.tex\\
%   accsupp-example2.tex & doc/latex/oberdiek/accsupp-example2.tex\\
%   test/accsupp-test1.tex & doc/latex/oberdiek/test/accsupp-test1.tex\\
%   accsupp.dtx & source/latex/oberdiek/accsupp.dtx\\
% \end{tabular}^^A
% }^^A
% \sbox0{\t}^^A
% \ifdim\wd0>\linewidth
%   \begingroup
%     \advance\linewidth by\leftmargin
%     \advance\linewidth by\rightmargin
%   \edef\x{\endgroup
%     \def\noexpand\lw{\the\linewidth}^^A
%   }\x
%   \def\lwbox{^^A
%     \leavevmode
%     \hbox to \linewidth{^^A
%       \kern-\leftmargin\relax
%       \hss
%       \usebox0
%       \hss
%       \kern-\rightmargin\relax
%     }^^A
%   }^^A
%   \ifdim\wd0>\lw
%     \sbox0{\small\t}^^A
%     \ifdim\wd0>\linewidth
%       \ifdim\wd0>\lw
%         \sbox0{\footnotesize\t}^^A
%         \ifdim\wd0>\linewidth
%           \ifdim\wd0>\lw
%             \sbox0{\scriptsize\t}^^A
%             \ifdim\wd0>\linewidth
%               \ifdim\wd0>\lw
%                 \sbox0{\tiny\t}^^A
%                 \ifdim\wd0>\linewidth
%                   \lwbox
%                 \else
%                   \usebox0
%                 \fi
%               \else
%                 \lwbox
%               \fi
%             \else
%               \usebox0
%             \fi
%           \else
%             \lwbox
%           \fi
%         \else
%           \usebox0
%         \fi
%       \else
%         \lwbox
%       \fi
%     \else
%       \usebox0
%     \fi
%   \else
%     \lwbox
%   \fi
% \else
%   \usebox0
% \fi
% \end{quote}
% If you have a \xfile{docstrip.cfg} that configures and enables \docstrip's
% TDS installing feature, then some files can already be in the right
% place, see the documentation of \docstrip.
%
% \subsection{Refresh file name databases}
%
% If your \TeX~distribution
% (\teTeX, \mikTeX, \dots) relies on file name databases, you must refresh
% these. For example, \teTeX\ users run \verb|texhash| or
% \verb|mktexlsr|.
%
% \subsection{Some details for the interested}
%
% \paragraph{Attached source.}
%
% The PDF documentation on CTAN also includes the
% \xfile{.dtx} source file. It can be extracted by
% AcrobatReader 6 or higher. Another option is \textsf{pdftk},
% e.g. unpack the file into the current directory:
% \begin{quote}
%   \verb|pdftk accsupp.pdf unpack_files output .|
% \end{quote}
%
% \paragraph{Unpacking with \LaTeX.}
% The \xfile{.dtx} chooses its action depending on the format:
% \begin{description}
% \item[\plainTeX:] Run \docstrip\ and extract the files.
% \item[\LaTeX:] Generate the documentation.
% \end{description}
% If you insist on using \LaTeX\ for \docstrip\ (really,
% \docstrip\ does not need \LaTeX), then inform the autodetect routine
% about your intention:
% \begin{quote}
%   \verb|latex \let\install=y% \iffalse meta-comment
%
% File: accsupp.dtx
% Version: 2018/01/30 v0.5
% Info: Accessibility support by marked content
%
% Copyright (C) 2007, 2010 by
%    Heiko Oberdiek <heiko.oberdiek at googlemail.com>
%    2016
%    https://github.com/ho-tex/oberdiek/issues
%
% This work may be distributed and/or modified under the
% conditions of the LaTeX Project Public License, either
% version 1.3c of this license or (at your option) any later
% version. This version of this license is in
%    http://www.latex-project.org/lppl/lppl-1-3c.txt
% and the latest version of this license is in
%    http://www.latex-project.org/lppl.txt
% and version 1.3 or later is part of all distributions of
% LaTeX version 2005/12/01 or later.
%
% This work has the LPPL maintenance status "maintained".
%
% This Current Maintainer of this work is Heiko Oberdiek.
%
% This work consists of the main source file accsupp.dtx
% and the derived files
%    accsupp.sty, accsupp.pdf, accsupp.ins, accsupp.drv,
%    accsupp-pdftex.def, \ccsupp-luatex.def, accsupp-dvipdfm.def,
%    accsupp-dvips.def,
%    accsupp-example1.tex, accsupp-example2.tex,
%    accsupp-test1.tex.
%
% Distribution:
%    CTAN:macros/latex/contrib/oberdiek/accsupp.dtx
%    CTAN:macros/latex/contrib/oberdiek/accsupp.pdf
%
% Unpacking:
%    (a) If accsupp.ins is present:
%           tex accsupp.ins
%    (b) Without accsupp.ins:
%           tex accsupp.dtx
%    (c) If you insist on using LaTeX
%           latex \let\install=y% \iffalse meta-comment
%
% File: accsupp.dtx
% Version: 2018/01/30 v0.5
% Info: Accessibility support by marked content
%
% Copyright (C) 2007, 2010 by
%    Heiko Oberdiek <heiko.oberdiek at googlemail.com>
%    2016
%    https://github.com/ho-tex/oberdiek/issues
%
% This work may be distributed and/or modified under the
% conditions of the LaTeX Project Public License, either
% version 1.3c of this license or (at your option) any later
% version. This version of this license is in
%    http://www.latex-project.org/lppl/lppl-1-3c.txt
% and the latest version of this license is in
%    http://www.latex-project.org/lppl.txt
% and version 1.3 or later is part of all distributions of
% LaTeX version 2005/12/01 or later.
%
% This work has the LPPL maintenance status "maintained".
%
% This Current Maintainer of this work is Heiko Oberdiek.
%
% This work consists of the main source file accsupp.dtx
% and the derived files
%    accsupp.sty, accsupp.pdf, accsupp.ins, accsupp.drv,
%    accsupp-pdftex.def, \ccsupp-luatex.def, accsupp-dvipdfm.def,
%    accsupp-dvips.def,
%    accsupp-example1.tex, accsupp-example2.tex,
%    accsupp-test1.tex.
%
% Distribution:
%    CTAN:macros/latex/contrib/oberdiek/accsupp.dtx
%    CTAN:macros/latex/contrib/oberdiek/accsupp.pdf
%
% Unpacking:
%    (a) If accsupp.ins is present:
%           tex accsupp.ins
%    (b) Without accsupp.ins:
%           tex accsupp.dtx
%    (c) If you insist on using LaTeX
%           latex \let\install=y\input{accsupp.dtx}
%        (quote the arguments according to the demands of your shell)
%
% Documentation:
%    (a) If accsupp.drv is present:
%           latex accsupp.drv
%    (b) Without accsupp.drv:
%           latex accsupp.dtx; ...
%    The class ltxdoc loads the configuration file ltxdoc.cfg
%    if available. Here you can specify further options, e.g.
%    use A4 as paper format:
%       \PassOptionsToClass{a4paper}{article}
%
%    Programm calls to get the documentation (example):
%       pdflatex accsupp.dtx
%       makeindex -s gind.ist accsupp.idx
%       pdflatex accsupp.dtx
%       makeindex -s gind.ist accsupp.idx
%       pdflatex accsupp.dtx
%
% Installation:
%    TDS:tex/latex/oberdiek/accsupp.sty
%    TDS:tex/latex/oberdiek/accsupp-pdftex.def
%    TDS:tex/latex/oberdiek/accsupp-luatex.def
%    TDS:tex/latex/oberdiek/accsupp-dvipdfm.def
%    TDS:tex/latex/oberdiek/accsupp-dvips.def
%    TDS:doc/latex/oberdiek/accsupp.pdf
%    TDS:doc/latex/oberdiek/accsupp-example1.tex
%    TDS:doc/latex/oberdiek/accsupp-example2.tex
%    TDS:doc/latex/oberdiek/test/accsupp-test1.tex
%    TDS:source/latex/oberdiek/accsupp.dtx
%
%<*ignore>
\begingroup
  \catcode123=1 %
  \catcode125=2 %
  \def\x{LaTeX2e}%
\expandafter\endgroup
\ifcase 0\ifx\install y1\fi\expandafter
         \ifx\csname processbatchFile\endcsname\relax\else1\fi
         \ifx\fmtname\x\else 1\fi\relax
\else\csname fi\endcsname
%</ignore>
%<*install>
\input docstrip.tex
\Msg{************************************************************************}
\Msg{* Installation}
\Msg{* Package: accsupp 2018/01/30 v0.5 Accessibility support by marked content (HO)}
\Msg{************************************************************************}

\keepsilent
\askforoverwritefalse

\let\MetaPrefix\relax
\preamble

This is a generated file.

Project: accsupp
Version: 2018/01/30 v0.5

Copyright (C) 2007, 2010 by
   Heiko Oberdiek <heiko.oberdiek at googlemail.com>

This work may be distributed and/or modified under the
conditions of the LaTeX Project Public License, either
version 1.3c of this license or (at your option) any later
version. This version of this license is in
   http://www.latex-project.org/lppl/lppl-1-3c.txt
and the latest version of this license is in
   http://www.latex-project.org/lppl.txt
and version 1.3 or later is part of all distributions of
LaTeX version 2005/12/01 or later.

This work has the LPPL maintenance status "maintained".

This Current Maintainer of this work is Heiko Oberdiek.

This work consists of the main source file accsupp.dtx
and the derived files
   accsupp.sty, accsupp.pdf, accsupp.ins, accsupp.drv,
   accsupp-pdftex.def, accsupp-luatex.def, accsupp-dvipdfm.def,
   accsupp-dvips.def,
   accsupp-example1.tex, accsupp-example2.tex,
   accsupp-test1.tex.

\endpreamble
\let\MetaPrefix\DoubleperCent

\generate{%
  \file{accsupp.ins}{\from{accsupp.dtx}{install}}%
  \file{accsupp.drv}{\from{accsupp.dtx}{driver}}%
  \usedir{tex/latex/oberdiek}%
  \file{accsupp.sty}{\from{accsupp.dtx}{package}}%
  \file{accsupp-pdftex.def}{\from{accsupp.dtx}{pdftex}}%
  \file{accsupp-luatex.def}{\from{accsupp.dtx}{luatex}}%
  \file{accsupp-dvipdfm.def}{\from{accsupp.dtx}{dvipdfm}}%
  \file{accsupp-dvips.def}{\from{accsupp.dtx}{dvips}}%
  \usedir{doc/latex/oberdiek}%
  \file{accsupp-example1.tex}{\from{accsupp.dtx}{example1}}%
  \file{accsupp-example2.tex}{\from{accsupp.dtx}{example2}}%
  \usedir{doc/latex/oberdiek/test}%
  \file{accsupp-test1.tex}{\from{accsupp.dtx}{test1}}%
  \nopreamble
  \nopostamble
  \usedir{source/latex/oberdiek/catalogue}%
  \file{accsupp.xml}{\from{accsupp.dtx}{catalogue}}%
}

\catcode32=13\relax% active space
\let =\space%
\Msg{************************************************************************}
\Msg{*}
\Msg{* To finish the installation you have to move the following}
\Msg{* files into a directory searched by TeX:}
\Msg{*}
\Msg{*     accsupp.sty, accsupp-pdftex.def, accsupp-dvipdfm.def,}
\Msg{*     accsupp-dvips.def}
\Msg{*}
\Msg{* To produce the documentation run the file `accsupp.drv'}
\Msg{* through LaTeX.}
\Msg{*}
\Msg{* Happy TeXing!}
\Msg{*}
\Msg{************************************************************************}

\endbatchfile
%</install>
%<*ignore>
\fi
%</ignore>
%<*driver>
\NeedsTeXFormat{LaTeX2e}
\ProvidesFile{accsupp.drv}%
  [2018/01/30 v0.5 Accessibility support by marked content (HO)]%
\documentclass{ltxdoc}
\usepackage{holtxdoc}[2011/11/22]
\begin{document}
  \DocInput{accsupp.dtx}%
\end{document}
%</driver>
% \fi
%
%
% \CharacterTable
%  {Upper-case    \A\B\C\D\E\F\G\H\I\J\K\L\M\N\O\P\Q\R\S\T\U\V\W\X\Y\Z
%   Lower-case    \a\b\c\d\e\f\g\h\i\j\k\l\m\n\o\p\q\r\s\t\u\v\w\x\y\z
%   Digits        \0\1\2\3\4\5\6\7\8\9
%   Exclamation   \!     Double quote  \"     Hash (number) \#
%   Dollar        \$     Percent       \%     Ampersand     \&
%   Acute accent  \'     Left paren    \(     Right paren   \)
%   Asterisk      \*     Plus          \+     Comma         \,
%   Minus         \-     Point         \.     Solidus       \/
%   Colon         \:     Semicolon     \;     Less than     \<
%   Equals        \=     Greater than  \>     Question mark \?
%   Commercial at \@     Left bracket  \[     Backslash     \\
%   Right bracket \]     Circumflex    \^     Underscore    \_
%   Grave accent  \`     Left brace    \{     Vertical bar  \|
%   Right brace   \}     Tilde         \~}
%
% \GetFileInfo{accsupp.drv}
%
% \title{The \xpackage{accsupp} package}
% \date{2018/01/30 v0.5}
% \author{Heiko Oberdiek\thanks
% {Please report any issues at https://github.com/ho-tex/oberdiek/issues}\\
% \xemail{heiko.oberdiek at googlemail.com}}
%
% \maketitle
%
% \begin{abstract}
% Since PDF 1.5 portions of a page can be marked for better
% accessibility support.
% For example, replacement texts or expansions of abbreviations can be
% provided. Package \xpackage{accsupp} starts with providing a minimal
% low-level interface for programmers. Status is experimental.
% \end{abstract}
%
% \tableofcontents
%
% \section{Documentation}
%
% \subsection{Macros}
%
% Section ``10.8 Accessibility Support'' of the PDF reference
% \cite{pdfspec} lists some features that can be added by
% operators for marked content.
%
% \begin{declcs}{BeginAccSupp} \M{options}
% \end{declcs}
% It puts the operator |BDC| in the page stream:
% \begin{quote}
%   |/Span|\\
%   |<<|\dots|>>| \qquad \textit{\% property dictionary}\\
%   |BDC|
% \end{quote}
% The contents of the dictionary is controlled by \meta{options}.
% See sections \ref{sec:feature} and \ref{sec:method}.
%
% \begin{declcs}{EndAccSupp} \M{options}
% \end{declcs}
% It puts the operator |EMC| in the page stream.
% The only option is \xoption{pdfliteral}, see section \ref{sec:pdftex}.
%
% \bigskip
% \noindent\textbf{Note:}
% The caller is responsible for the placement of \cs{BeginAccSupp}
% and \cs{EndAccSupp} pairs. Especially page breaks are not allowed
% in between.
%
% \subsubsection{Feature options}\label{sec:feature}
%
% The PDF reference \cite{pdfspec} describes and explains the different
% features. The names of the feature options are the same as the
% key names for the property dictionary for operator |BDC|, see
% \cs{BeginAccSupp}.
%
% \begin{description}
% \item[\xoption{ActualText}:]
%   Provides a replacement text, see examples in
%   section \ref{sec:example}.
% \item[\xoption{Alt}:]
%   Provides an alternate description.
% \item[\xoption{E}:]
%   Provides the expansion of an abbreviation or an acronym.
% \item[\xoption{Lang}:]
%   Specifies the language.
% \end{description}
%
% \subsubsection{Input methods}\label{sec:method}
%
% Except for \xoption{Lang} option \xoption{method} controls
% how the argument for \xoption{ActualText}, \xoption{Alt},
% and \xoption{E} are interpreted.
% \begin{description}
% \item[\xoption{method=plain}:]
%   The string is only expanded and written without further
%   treatment. Special characters are not protected, thus this method
%   may result in an invalid PDF file.
% \item[\xoption{mehod=escape}:]
%   The string is expanded and special characters are escaped.
%   The result is a valid PDF string.
% \item[\xoption{method=hex}:]
%   The string is given in hexadecimal notation.
%   Section \ref{sec:example:notparallel} shows an example.
% \item[\xoption{method=pdfstringdef}:]
%   If package \xpackage{hyperref} is loaded, then its \cs{pdfstringdef}
%   is used. This method is slow, but useful if the string contains
%   arbitrary \hologo{TeX} code.
% \item[\xoption{unicode}:]
%   This option is needed, if the string is given as Unicode string
%   (16 bit). Internally it adds the string prefix for Unicode.
%   In case of \xoption{method=pdfstringdef} it passes the option
%   to \cs{hypersetup}.
% \end{description}
%
% \subsection{Workaround, option \xoption{space}}
%
% PDF specification says in ``10.8.3 Replacement Text''
% \begin{quote}
% Just as alternative descriptions can be provided for images and other
% items that do not translate naturally into text (\dots), replacement
% text can be specified for content that does translate into text but
% that is represented in a nonstandard way. These nonstandard representations
% might include, for example, glyphs for ligatures or custom characters, or
% inline graphics corresponding to letters in an illuminated manustript or to
% dropped capitals.
% \end{quote}
% However, the ActualText is ignored in Acrobat Reader
% (until version 9 at least), if the content does not contain glyphs.
% Option \xoption{space} adds such an invisible glyph, a space
% character. The font name can be configured by option \xoption{spacefont},
% the default font is \texttt{phvr8r}. The character slot is given
% by option \xoption{spacechar}, default is \texttt{32}, the usual
% position of the space character.
%
% These options can also be given as package options or in
% macro \cs{AccSuppSetup} that takes a key value list as argument.
% Usually only option \xoption{space} is necessary, if the
% contents does not contain glyphs otherwise. Option \xoption{space}
% is enabled by |space| or |space=true| and disabled by |space=false|.
% It is disabled as default. The option is evaluated by \cs{BeginAccSupp}
% and ignored by \cs{EndAccSupp}.
%
% Note: Even with option \xoption{space} I get sometimes wrong texts
% when cut \& paste from AR7/Linux or AR8/Linux, e.g.
% \begin{quote}
%   |Hello| $\rightarrow$ |Helo|, |Helol|, |Hell|, \dots\\
%   |Hello World| $\rightarrow$ |Helo WorldW|, |Helo World|, \dots
% \end{quote}
% I do not know what Acrobat Reader is doing here, thus feedback and
% insights are welcome.
%
% \subsection{Driver options}
%
% Driver options are package options only.
% The special \hologo{TeX} compilers \hologo{pdfTeX}, \hologo{LuaTeX} and \hologo{XeTeX}
% are detected automatically. The default for unrecognized drivers can be
% set by defining \cs{ActualTextDriverDefault}. This can be done
% in the configuration file \xfile{accsupp.cfg}.
%
% \subsubsection{Option \xoption{pdftex}}\label{sec:pdftex}
% Package option \xoption{pdftex} is used for \hologo{pdfTeX} in PDF mode.
% Additionally \cs{BeginAccSupp} and \cs{EndAccSupp} understand
% option \xoption{pdfliteral}. It controls
% the modifier keyword for \cs{pdfliteral}:
% \begin{quote}
%   |pdfliteral=direct| $\Rightarrow$ |\pdfliteral direct{...}|
% \end{quote}
%
% \subsubsection{Option \xoption{luatex}}\label{sec:luatex}
% Package option \xoption{luatex} is used for \hologo{LuaTeX} in PDF mode.
% Additionally \cs{BeginAccSupp} and \cs{EndAccSupp} understand
% option \xoption{pdfliteral}. It controls
% the modifier keyword for \cs{pdfextension} literal:
% \begin{quote}
%   |pdfliteral=direct| $\Rightarrow$ |\pdfextension literal direct{...}|
% \end{quote}
%
% \subsubsection{Option \xoption{dvipdfm}}
%
% Package option \xoption{dvipdfm} and its aliases \xoption{dvipdfmx}
% \xoption{xetex} are used for drivers that support
% \textsf{dvipdfm} specials.
%
% \subsubsection{Option \xoption{dvips}}
%
% Package option \xoption{dvips} and its alias \xoption{dvipsone}
% write pdfmark specials in the output. Unhappily these pdfmark
% operators are ignored by \textsf{ghostscript} (latest tested
% version is 8.54). Perhaps they are recognized by commercial
% distiller applications.
%
% \subsubsection{Turning off page stream compression}
%
% For debugging it is useful to have uncompressed page stream objects.
% This can be done afterwards via \textsf{pdftk}:
% \begin{quote}
%   |pdftk file.pdf output file-uncompressed.pdf uncompress|
% \end{quote}
% Or the PDF file is generated uncompressed:
% \begin{description}
% \item[\hologo{pdfTeX}:] |\pdfcompresslevel=0|
% \item[dvipdfm:] |dvipdfm -z0 ...|
% \item[dvipdfmx:] |dvipdfmx -z0 ...|
% \item[ghostscript/ps2pdf:]
%      |ps2pdf -dCompressPages=false input.ps output.pdf|
% \end{description}
%
% \section{Example}
% \label{sec:example}
%
% \subsection{Example \cs{notparallel}}
% \label{sec:example:notparallel}
%
%    \begin{macrocode}
%<*example1>
%    \end{macrocode}
%    \begin{macrocode}
%<<END
\documentclass{article}
\usepackage{accsupp}[2007/11/14]
\usepackage{centernot}
% U+2226 NOT PARALLEL
% \mathrel{...} prevents page break in between
\newcommand*{\notparallel}{%
  \ensuremath{%
    \mathrel{%
      \BeginAccSupp{method=hex,unicode,ActualText=2226}%
      \centernot{\parallel}%
      \EndAccSupp{}%
    }%
  }%
}
\begin{document}
\begin{equation}
A\notparallel B
\end{equation}
\end{document}
%END
%    \end{macrocode}
%    \begin{macrocode}
%</example1>
%    \end{macrocode}
%
% \subsection{Example with \xoption{pdfstringdef}}
%
%    \begin{macrocode}
%<*example2>
%    \end{macrocode}
%    \begin{macrocode}
%<<END
\documentclass{article}
\usepackage[unicode]{hyperref}
\usepackage{accsupp}[2007/11/14]
\begin{document}
  \begin{equation}
    \BeginAccSupp{
      method=pdfstringdef,
      unicode,
      ActualText={%
        a\texttwosuperior +b\texttwosuperior
        =c\texttwosuperior
      }
    }
    a^2 + b^2 = c^2
    \EndAccSupp{}
  \end{equation}
\end{document}
%END
%    \end{macrocode}
%    \begin{macrocode}
%</example2>
%    \end{macrocode}
%
% \StopEventually{
% }
%
% \section{Implementation}
%
% \subsection{Package}
%
%    \begin{macrocode}
%<*package>
%    \end{macrocode}
%
%    \begin{macrocode}
\begingroup\catcode61\catcode48\catcode32=10\relax%
  \catcode13=5 % ^^M
  \endlinechar=13 %
  \catcode123=1 % {
  \catcode125=2 % }
  \catcode64=11 % @
  \def\x{\endgroup
    \expandafter\edef\csname ACCSUPP@AtEnd\endcsname{%
      \endlinechar=\the\endlinechar\relax
      \catcode13=\the\catcode13\relax
      \catcode32=\the\catcode32\relax
      \catcode35=\the\catcode35\relax
      \catcode61=\the\catcode61\relax
      \catcode64=\the\catcode64\relax
      \catcode123=\the\catcode123\relax
      \catcode125=\the\catcode125\relax
    }%
  }%
\x\catcode61\catcode48\catcode32=10\relax%
\catcode13=5 % ^^M
\endlinechar=13 %
\catcode35=6 % #
\catcode64=11 % @
\catcode123=1 % {
\catcode125=2 % }
\def\TMP@EnsureCode#1#2{%
  \edef\ACCSUPP@AtEnd{%
    \ACCSUPP@AtEnd
    \catcode#1=\the\catcode#1\relax
  }%
  \catcode#1=#2\relax
}
\TMP@EnsureCode{10}{12}% ^^J
\TMP@EnsureCode{33}{12}% !
\TMP@EnsureCode{39}{12}% '
\TMP@EnsureCode{40}{12}% (
\TMP@EnsureCode{41}{12}% )
\TMP@EnsureCode{42}{12}% *
\TMP@EnsureCode{44}{12}% ,
\TMP@EnsureCode{45}{12}% -
\TMP@EnsureCode{46}{12}% .
\TMP@EnsureCode{47}{12}% /
\TMP@EnsureCode{58}{12}% :
\TMP@EnsureCode{60}{12}% <
\TMP@EnsureCode{62}{12}% >
\TMP@EnsureCode{94}{7}% ^ (superscript)
\TMP@EnsureCode{96}{12}% `
\TMP@EnsureCode{254}{12}% ^^fe
\TMP@EnsureCode{255}{12}% ^^ff
\edef\ACCSUPP@AtEnd{\ACCSUPP@AtEnd\noexpand\endinput}
%    \end{macrocode}
%
%    Package identification.
%    \begin{macrocode}
\NeedsTeXFormat{LaTeX2e}
\ProvidesPackage{accsupp}%
  [2018/01/30 v0.5 Accessibility support by marked content (HO)]
%    \end{macrocode}
%
%    \begin{macrocode}
\RequirePackage{pdfescape}[2007/02/25]
\RequirePackage{ifpdf}
\RequirePackage{ifxetex}
\RequirePackage{kvoptions}
%    \end{macrocode}
%    \begin{macrocode}
\SetupKeyvalOptions{%
  family=ACCSUPP,%
  prefix=ACCSUPP@%
}
%    \end{macrocode}
%
% \subsection{Space setup}
%
%    \begin{macrocode}
\DeclareBoolOption{space}
\DeclareStringOption[phvr8r]{spacefont}
\DeclareStringOption[32]{spacechar}
%    \end{macrocode}
%
% \subsection{Driver detection and setup}
%
% \paragraph{Driver declarations.}
%
%    \begin{macrocode}
\def\ACCSUPP@DefineDriverKey{%
  \@dblarg\ACCSUPP@@DefineDriverKey
}
\def\ACCSUPP@@DefineDriverKey[#1]#2{%
  \define@key{ACCSUPP}{#2}[]{%
    \def\ACCSUPP@driver{#1}%
  }%
  \g@addto@macro\ACCSUPP@DisableOptions{%
    \DisableKeyvalOption{ACCSUPP}{#2}%
  }%
}
\let\ACCSUPP@DisableOptions\@empty
\ACCSUPP@DefineDriverKey{pdftex}
\ACCSUPP@DefineDriverKey{luatex}
\ACCSUPP@DefineDriverKey{dvips}
\ACCSUPP@DefineDriverKey[dvips]{dvipsone}
\ACCSUPP@DefineDriverKey{dvipdfm}
\ACCSUPP@DefineDriverKey[dvipdfm]{dvipdfmx}
\ACCSUPP@DefineDriverKey[dvipdfm]{xetex}
\let\ACCSUPP@driver\relax
\InputIfFileExists{accsupp.cfg}{}{}
\providecommand*{\ActualTextDriverDefault}{dvips}
\ifpdf
  \ifx\pdfextension\@undefined
    \def\ACCSUPP@driver{pdftex}%
  \else
    \def\ACCSUPP@driver{luatex}%
  \fi
\else
  \ifxetex
    \def\ACCSUPP@driver{dvipdfm}%
  \else
    \ifx\ACCSUPP@driver\relax
      \let\ACCSUPP@driver\ActualTextDriverDefault
    \fi
  \fi
\fi
%    \end{macrocode}
%
% \paragraph{Process options.}
%
%    \begin{macrocode}
\ProcessKeyvalOptions*
\ACCSUPP@DisableOptions
%    \end{macrocode}
%
% \paragraph{Driver validation and loading.}
%
%    \begin{macrocode}
\def\ACCSUPP@temp{pdftex}%
\ifpdf
  \ifx\pdfextension\@undefined\else
    \def\ACCSUPP@temp{luatex}%
  \fi
  \ifx\ACCSUPP@temp\ACCSUPP@driver
  \else
    \PackageWarningNoLine{accsupp}{%
      Wrong driver `\ACCSUPP@driver', using `pdftex' instead%
    }%
ed as
    \let\ACCSUPP@driver\ACCSUPP@temp
  \fi
\else
  \ifx\ACCSUPP@temp\ACCSUPP@driver
    \PackageError{accsupp}{%
      Wrong driver, pdfTeX is not running in PDF mode.\MessageBreak
      Package loading is aborted%
    }\@ehc
    \expandafter\expandafter\expandafter\ACCSUPP@AtEnd
  \fi
  \def\ACCSUPP@temp{dvipdfm}%
  \ifxetex
    \ifx\ACCSUPP@temp\ACCSUPP@driver
    \else
      \PackageWarningNoLine{accsupp}{%
        Wrong driver `\ACCSUPP@driver',\MessageBreak
        using `dvipdfm' for XeTeX instead%
      }%
      \let\ACCSUPP@driver\ACCSUPP@temp
    \fi
  \fi
\fi%
\ifx\ACCSUPP@driver\relax
  \PackageError{accsupp}{%
    Missing driver option.\MessageBreak
    Package loading is aborted%
  }\@ehc
  \expandafter\ACCSUPP@AtEnd
\fi%
\InputIfFileExists{accsupp-\ACCSUPP@driver.def}{}{%
  \PackageError{accsupp}{%
    Unsupported driver `\ACCSUPP@driver'.\MessageBreak
    Package loading is aborted%
  }\@ehc
  \ACCSUPP@AtEnd
}%
%    \end{macrocode}
%
% \subsection{Main macro}
%
%    \begin{macrocode}
\DeclareBoolOption{unicode}
\DeclareStringOption[page]{pdfliteral}
\DeclareStringOption{Lang}
\def\ACCSUPP@method{escape}
\define@key{ACCSUPP}{method}{%
  \@ifundefined{ACCSUPP@method@#1}{%
    \PackageError{accsupp}{%
      Ignoring unknown method `#1'%
    }\@ehc
  }{%
    \edef\ACCSUPP@method{#1}%
  }%
}
\let\ACCSUPP@Lang\relax
\def\ACCSUPP@temp#1{%
  \expandafter\ACCSUPP@@temp\csname ACCSUPP@#1\endcsname{#1}%
}
\def\ACCSUPP@@temp#1#2{%
  \let#1\relax
  \define@key{ACCSUPP}{#2}{%
    \def#1{##1}%
    \ifx#1\@empty
      \def#1{()}%
    \else
      \csname ACCSUPP@method@\ACCSUPP@method\endcsname#1%
    \fi
  }%
}
\ACCSUPP@temp{Alt}
\ACCSUPP@temp{ActualText}
\ACCSUPP@temp{E}
%    \end{macrocode}
%
%    \begin{macrocode}
\newcommand*{\BeginAccSupp}[1]{%
  \begingroup
    \setkeys{ACCSUPP}{#1}%
    \edef\ACCSUPP@span{%
      /Span<<%
        \ifx\ACCSUPP@Lang\relax
        \else
          /Lang\ACCSUPP@Lang
        \fi
        \ifx\ACCSUPP@Alt\relax
        \else
          /Alt\ACCSUPP@Alt
        \fi
        \ifx\ACCSUPP@ActualText\relax
        \else
          /ActualText\ACCSUPP@ActualText
        \fi
        \ifx\ACCSUPP@E\relax
        \else
          /E\ACCSUPP@E
        \fi
      >>%
    }%
    \ACCSUPP@bdc
    \ACCSUPP@space
  \endgroup
}
\newcommand*{\EndAccSupp}[1]{%
  \begingroup
    \setkeys{ACCSUPP}{#1}%
    \ACCSUPP@emc
  \endgroup
}
\def\ACCSUPP@space{%
  \ifACCSUPP@space
    \begingroup
      \@ifundefined{ACCSUPP@Font}{%
        \global\font\ACCSUPP@Font=\ACCSUPP@spacefont\relax
      }{}%
      \leavevmode
      \setbox\z@\hbox{\ACCSUPP@Font\char\ACCSUPP@spacechar}%
      \wd\z@\z@
      \ht\z@\z@
      \dp\z@\z@
      \copy\z@
    \endgroup
  \fi
}
\newcommand*{\AccSuppSetup}{%
  \setkeys{ACCSUPP}%
}
%    \end{macrocode}
%
% \subsubsection{Input methods}
%
%    \begin{macrocode}
\def\ACCSUPP@method@plain#1{%
  \csname @safe@activestrue\endcsname
  \edef#1{%
    (%
    \ifACCSUPP@unicode
      \string\376\string\377%
    \fi
    #1%
    )%
  }%
  \@onelevel@sanitize#1%
}
%    \end{macrocode}
%    \begin{macrocode}
\def\ACCSUPP@method@escape#1{%
  \EdefEscapeString#1{%
    \ifACCSUPP@unicode
      ^^fe^^ff%
    \fi
    #1%
  }%
  \edef#1{(#1)}%
}%
%    \end{macrocode}
%    \begin{macrocode}
\def\ACCSUPP@method@hex#1{%
  \edef#1{%
    <%
    \ifACCSUPP@unicode
      FEFF%
    \fi
    #1%
    >%
  }%
}
%    \end{macrocode}
%    \begin{macrocode}
\def\ACCSUPP@method@pdfstringdef#1{%
  \ifACCSUPP@unicode
    \@ifundefined{hypersetup}{}{%
      \hypersetup{unicode}%
    }%
  \fi
  \@ifundefined{pdfstringdef}{%
    \PackageError{accsupp}{%
      Method `pdfstringdef' requires package `hyperref'%
    }\@ehc
    \let\ACCSUPP@temp\@empty
  }{%
    \begingroup
      \setbox0=\hbox{%
        \pdfstringdef\ACCSUPP@temp#1%
        \global\let\ACCSUPP@temp\ACCSUPP@temp
      }%
    \endgroup
  }%
  \edef#1{(\ACCSUPP@temp)}%
}
%    \end{macrocode}
%
%    \begin{macrocode}
\ACCSUPP@AtEnd%
%</package>
%    \end{macrocode}
%
% \subsection{Drivers}
%
% \subsubsection{Driver \xoption{pdftex}}
%
%    \begin{macrocode}
%<*pdftex>
\NeedsTeXFormat{LaTeX2e}
\ProvidesFile{accsupp-pdftex.def}%
  [2018/01/30 v0.5 accsupp driver for pdfTeX (HO)]%
%    \end{macrocode}
%
%    \begin{macrocode}
\def\ACCSUPP@bdc{%
  \pdfliteral\ACCSUPP@pdfliteral{\ACCSUPP@span BDC}%
}
\def\ACCSUPP@emc{%
  \pdfliteral\ACCSUPP@pdfliteral{EMC}%
}
%    \end{macrocode}
%    \begin{macrocode}
%</pdftex>
%    \end{macrocode}
%
% \subsubsection{Driver \xoption{luatex}}
%
%    \begin{macrocode}
%<*luatex>
\NeedsTeXFormat{LaTeX2e}
\ProvidesFile{accsupp-luatex.def}%
  [2018/01/30 v0.5 accsupp driver for pdfTeX (HO)]%
%    \end{macrocode}
%
%    \begin{macrocode}
\protected\def\ACCSUPP@pdf@literal{\pdfextension literal}
\def\ACCSUPP@bdc{%
  \ACCSUPP@pdf@literal\ACCSUPP@pdfliteral{\ACCSUPP@span BDC}%
}
\def\ACCSUPP@emc{%
  \ACCSUPP@pdf@literal\ACCSUPP@pdfliteral{EMC}%
}
%    \end{macrocode}
%    \begin{macrocode}
%</luatex>
%    \end{macrocode}
%
% \subsubsection{Driver \xoption{dvipdfm}}
%
%    \begin{macrocode}
%<*dvipdfm>
\NeedsTeXFormat{LaTeX2e}
\ProvidesFile{accsupp-dvipdfm.def}%
  [2018/01/30 v0.5 accsupp driver for dvipdfm (HO)]%
%    \end{macrocode}
%
%    \begin{macrocode}
\def\ACCSUPP@bdc{%
  \special{pdf:content \ACCSUPP@span BDC}%
}
\def\ACCSUPP@emc{%
  \special{pdf:content EMC}%
}
%    \end{macrocode}
%    \begin{macrocode}
%</dvipdfm>
%    \end{macrocode}
%
% \subsubsection{Driver \xoption{dvips}}
%
%    \begin{macrocode}
%<*dvips>
\NeedsTeXFormat{LaTeX2e}
\ProvidesFile{accsupp-dvips.def}%
  [2018/01/30 v0.5 accsupp driver for dvips (HO)]%
%    \end{macrocode}
%
%    \begin{macrocode}
\def\ACCSUPP@bdc{%
  \special{ps:[\ACCSUPP@span/BDC pdfmark}%
}
\def\ACCSUPP@emc{%
  \special{ps:[/EMC pdfmark}%
}
%    \end{macrocode}
%    \begin{macrocode}
%</dvips>
%    \end{macrocode}
%
% \section{Test}
%
% \subsection{Catcode checks for loading}
%
%    \begin{macrocode}
%<*test1>
%    \end{macrocode}
%    \begin{macrocode}
\NeedsTeXFormat{LaTeX2e}
\documentclass{minimal}
\makeatletter
\def\RestoreCatcodes{}
\count@=0 %
\loop
  \edef\RestoreCatcodes{%
    \RestoreCatcodes
    \catcode\the\count@=\the\catcode\count@\relax
  }%
\ifnum\count@<255 %
  \advance\count@\@ne
\repeat

\def\RangeCatcodeInvalid#1#2{%
  \count@=#1\relax
  \loop
    \catcode\count@=15 %
  \ifnum\count@<#2\relax
    \advance\count@\@ne
  \repeat
}
\def\Test{%
  \RangeCatcodeInvalid{0}{47}%
  \RangeCatcodeInvalid{58}{64}%
  \RangeCatcodeInvalid{91}{96}%
  \RangeCatcodeInvalid{123}{127}%
  \catcode`\@=12 %
  \catcode`\\=0 %
  \catcode`\{=1 %
  \catcode`\}=2 %
  \catcode`\#=6 %
  \catcode`\[=12 %
  \catcode`\]=12 %
  \catcode`\%=14 %
  \catcode`\ =10 %
  \catcode13=5 %
  \RequirePackage{accsupp}[2018/01/30]\relax
  \RestoreCatcodes
}
\Test
\csname @@end\endcsname
\end
%    \end{macrocode}
%    \begin{macrocode}
%</test1>
%    \end{macrocode}
%
% \section{Installation}
%
% \subsection{Download}
%
% \paragraph{Package.} This package is available on
% CTAN\footnote{\url{http://ctan.org/pkg/accsupp}}:
% \begin{description}
% \item[\CTAN{macros/latex/contrib/oberdiek/accsupp.dtx}] The source file.
% \item[\CTAN{macros/latex/contrib/oberdiek/accsupp.pdf}] Documentation.
% \end{description}
%
%
% \paragraph{Bundle.} All the packages of the bundle `oberdiek'
% are also available in a TDS compliant ZIP archive. There
% the packages are already unpacked and the documentation files
% are generated. The files and directories obey the TDS standard.
% \begin{description}
% \item[\CTAN{install/macros/latex/contrib/oberdiek.tds.zip}]
% \end{description}
% \emph{TDS} refers to the standard ``A Directory Structure
% for \TeX\ Files'' (\CTAN{tds/tds.pdf}). Directories
% with \xfile{texmf} in their name are usually organized this way.
%
% \subsection{Bundle installation}
%
% \paragraph{Unpacking.} Unpack the \xfile{oberdiek.tds.zip} in the
% TDS tree (also known as \xfile{texmf} tree) of your choice.
% Example (linux):
% \begin{quote}
%   |unzip oberdiek.tds.zip -d ~/texmf|
% \end{quote}
%
% \paragraph{Script installation.}
% Check the directory \xfile{TDS:scripts/oberdiek/} for
% scripts that need further installation steps.
% Package \xpackage{attachfile2} comes with the Perl script
% \xfile{pdfatfi.pl} that should be installed in such a way
% that it can be called as \texttt{pdfatfi}.
% Example (linux):
% \begin{quote}
%   |chmod +x scripts/oberdiek/pdfatfi.pl|\\
%   |cp scripts/oberdiek/pdfatfi.pl /usr/local/bin/|
% \end{quote}
%
% \subsection{Package installation}
%
% \paragraph{Unpacking.} The \xfile{.dtx} file is a self-extracting
% \docstrip\ archive. The files are extracted by running the
% \xfile{.dtx} through \plainTeX:
% \begin{quote}
%   \verb|tex accsupp.dtx|
% \end{quote}
%
% \paragraph{TDS.} Now the different files must be moved into
% the different directories in your installation TDS tree
% (also known as \xfile{texmf} tree):
% \begin{quote}
% \def\t{^^A
% \begin{tabular}{@{}>{\ttfamily}l@{ $\rightarrow$ }>{\ttfamily}l@{}}
%   accsupp.sty & tex/latex/oberdiek/accsupp.sty\\
%   accsupp-pdftex.def & tex/latex/oberdiek/accsupp-pdftex.def\\
%   accsupp-luatex.def & tex/latex/oberdiek/accsupp-luatex.def\\
%   accsupp-dvipdfm.def & tex/latex/oberdiek/accsupp-dvipdfm.def\\
%   accsupp-dvips.def & tex/latex/oberdiek/accsupp-dvips.def\\
%   accsupp.pdf & doc/latex/oberdiek/accsupp.pdf\\
%   accsupp-example1.tex & doc/latex/oberdiek/accsupp-example1.tex\\
%   accsupp-example2.tex & doc/latex/oberdiek/accsupp-example2.tex\\
%   test/accsupp-test1.tex & doc/latex/oberdiek/test/accsupp-test1.tex\\
%   accsupp.dtx & source/latex/oberdiek/accsupp.dtx\\
% \end{tabular}^^A
% }^^A
% \sbox0{\t}^^A
% \ifdim\wd0>\linewidth
%   \begingroup
%     \advance\linewidth by\leftmargin
%     \advance\linewidth by\rightmargin
%   \edef\x{\endgroup
%     \def\noexpand\lw{\the\linewidth}^^A
%   }\x
%   \def\lwbox{^^A
%     \leavevmode
%     \hbox to \linewidth{^^A
%       \kern-\leftmargin\relax
%       \hss
%       \usebox0
%       \hss
%       \kern-\rightmargin\relax
%     }^^A
%   }^^A
%   \ifdim\wd0>\lw
%     \sbox0{\small\t}^^A
%     \ifdim\wd0>\linewidth
%       \ifdim\wd0>\lw
%         \sbox0{\footnotesize\t}^^A
%         \ifdim\wd0>\linewidth
%           \ifdim\wd0>\lw
%             \sbox0{\scriptsize\t}^^A
%             \ifdim\wd0>\linewidth
%               \ifdim\wd0>\lw
%                 \sbox0{\tiny\t}^^A
%                 \ifdim\wd0>\linewidth
%                   \lwbox
%                 \else
%                   \usebox0
%                 \fi
%               \else
%                 \lwbox
%               \fi
%             \else
%               \usebox0
%             \fi
%           \else
%             \lwbox
%           \fi
%         \else
%           \usebox0
%         \fi
%       \else
%         \lwbox
%       \fi
%     \else
%       \usebox0
%     \fi
%   \else
%     \lwbox
%   \fi
% \else
%   \usebox0
% \fi
% \end{quote}
% If you have a \xfile{docstrip.cfg} that configures and enables \docstrip's
% TDS installing feature, then some files can already be in the right
% place, see the documentation of \docstrip.
%
% \subsection{Refresh file name databases}
%
% If your \TeX~distribution
% (\teTeX, \mikTeX, \dots) relies on file name databases, you must refresh
% these. For example, \teTeX\ users run \verb|texhash| or
% \verb|mktexlsr|.
%
% \subsection{Some details for the interested}
%
% \paragraph{Attached source.}
%
% The PDF documentation on CTAN also includes the
% \xfile{.dtx} source file. It can be extracted by
% AcrobatReader 6 or higher. Another option is \textsf{pdftk},
% e.g. unpack the file into the current directory:
% \begin{quote}
%   \verb|pdftk accsupp.pdf unpack_files output .|
% \end{quote}
%
% \paragraph{Unpacking with \LaTeX.}
% The \xfile{.dtx} chooses its action depending on the format:
% \begin{description}
% \item[\plainTeX:] Run \docstrip\ and extract the files.
% \item[\LaTeX:] Generate the documentation.
% \end{description}
% If you insist on using \LaTeX\ for \docstrip\ (really,
% \docstrip\ does not need \LaTeX), then inform the autodetect routine
% about your intention:
% \begin{quote}
%   \verb|latex \let\install=y\input{accsupp.dtx}|
% \end{quote}
% Do not forget to quote the argument according to the demands
% of your shell.
%
% \paragraph{Generating the documentation.}
% You can use both the \xfile{.dtx} or the \xfile{.drv} to generate
% the documentation. The process can be configured by the
% configuration file \xfile{ltxdoc.cfg}. For instance, put this
% line into this file, if you want to have A4 as paper format:
% \begin{quote}
%   \verb|\PassOptionsToClass{a4paper}{article}|
% \end{quote}
% An example follows how to generate the
% documentation with pdf\LaTeX:
% \begin{quote}
%\begin{verbatim}
%pdflatex accsupp.dtx
%makeindex -s gind.ist accsupp.idx
%pdflatex accsupp.dtx
%makeindex -s gind.ist accsupp.idx
%pdflatex accsupp.dtx
%\end{verbatim}
% \end{quote}
%
% \section{Catalogue}
%
% The following XML file can be used as source for the
% \href{http://mirror.ctan.org/help/Catalogue/catalogue.html}{\TeX\ Catalogue}.
% The elements \texttt{caption} and \texttt{description} are imported
% from the original XML file from the Catalogue.
% The name of the XML file in the Catalogue is \xfile{accsupp.xml}.
%    \begin{macrocode}
%<*catalogue>
<?xml version='1.0' encoding='us-ascii'?>
<!DOCTYPE entry SYSTEM 'catalogue.dtd'>
<entry datestamp='$Date$' modifier='$Author$' id='accsupp'>
  <name>accsupp</name>
  <caption>Better accessibility support for PDF files.</caption>
  <authorref id='auth:oberdiek'/>
  <copyright owner='Heiko Oberdiek' year='2007,2010'/>
  <license type='lppl1.3'/>
  <version number='0.4'/>
  <description>
    Since PDF 1.5 portions of a page can be marked for better accessibility
    support. For example, replacement texts or expansions of abbreviations can
    be provided. This package starts with providing a minimal low-level
    interface for programmers; its status is experimental.
    <p/>
    The package is part of the <xref refid='oberdiek'>oberdiek</xref> bundle.
  </description>
  <documentation details='Package documentation'
      href='ctan:/macros/latex/contrib/oberdiek/accsupp.pdf'/>
  <ctan file='true' path='/macros/latex/contrib/oberdiek/accsupp.dtx'/>
  <miktex location='oberdiek'/>
  <texlive location='oberdiek'/>
  <install path='/macros/latex/contrib/oberdiek/oberdiek.tds.zip'/>
</entry>
%</catalogue>
%    \end{macrocode}
%
% \begin{thebibliography}{9}
% \bibitem{pdfspec}
% Adobe Systems Incorporated,
% \textit{PDF Reference}, 6th edition, 2006.
% \url{http://www.adobe.com/devnet/acrobat/pdfs/pdf_reference.pdf}
%
% \end{thebibliography}
%
% \begin{History}
%   \begin{Version}{2007/03/21 v0.1}
%   \item
%     First version.
%   \end{Version}
%   \begin{Version}{2007/11/14 v0.2}
%   \item
%     Various bug fixes.
%   \item
%     Catcode section rewritten, test added.
%   \end{Version}
%   \begin{Version}{2010/01/16 v0.3}
%   \item
%     \cs{AccSuppSetup} added.
%   \item
%     Options \xoption{space}, \xoption{spacefont}, \xoption{spacechar} added.
%   \end{Version}
%   \begin{Version}{2016/05/16 v0.4}
%   \item
%     Documentation updates.
%   \end{Version}
%   \begin{Version}{2018/01/30 v0.5}
%   \item
%     LuaTeX support added.
%   \end{Version}
% \end{History}
%
% \PrintIndex
%
% \Finale
\endinput

%        (quote the arguments according to the demands of your shell)
%
% Documentation:
%    (a) If accsupp.drv is present:
%           latex accsupp.drv
%    (b) Without accsupp.drv:
%           latex accsupp.dtx; ...
%    The class ltxdoc loads the configuration file ltxdoc.cfg
%    if available. Here you can specify further options, e.g.
%    use A4 as paper format:
%       \PassOptionsToClass{a4paper}{article}
%
%    Programm calls to get the documentation (example):
%       pdflatex accsupp.dtx
%       makeindex -s gind.ist accsupp.idx
%       pdflatex accsupp.dtx
%       makeindex -s gind.ist accsupp.idx
%       pdflatex accsupp.dtx
%
% Installation:
%    TDS:tex/latex/oberdiek/accsupp.sty
%    TDS:tex/latex/oberdiek/accsupp-pdftex.def
%    TDS:tex/latex/oberdiek/accsupp-luatex.def
%    TDS:tex/latex/oberdiek/accsupp-dvipdfm.def
%    TDS:tex/latex/oberdiek/accsupp-dvips.def
%    TDS:doc/latex/oberdiek/accsupp.pdf
%    TDS:doc/latex/oberdiek/accsupp-example1.tex
%    TDS:doc/latex/oberdiek/accsupp-example2.tex
%    TDS:doc/latex/oberdiek/test/accsupp-test1.tex
%    TDS:source/latex/oberdiek/accsupp.dtx
%
%<*ignore>
\begingroup
  \catcode123=1 %
  \catcode125=2 %
  \def\x{LaTeX2e}%
\expandafter\endgroup
\ifcase 0\ifx\install y1\fi\expandafter
         \ifx\csname processbatchFile\endcsname\relax\else1\fi
         \ifx\fmtname\x\else 1\fi\relax
\else\csname fi\endcsname
%</ignore>
%<*install>
\input docstrip.tex
\Msg{************************************************************************}
\Msg{* Installation}
\Msg{* Package: accsupp 2018/01/30 v0.5 Accessibility support by marked content (HO)}
\Msg{************************************************************************}

\keepsilent
\askforoverwritefalse

\let\MetaPrefix\relax
\preamble

This is a generated file.

Project: accsupp
Version: 2018/01/30 v0.5

Copyright (C) 2007, 2010 by
   Heiko Oberdiek <heiko.oberdiek at googlemail.com>

This work may be distributed and/or modified under the
conditions of the LaTeX Project Public License, either
version 1.3c of this license or (at your option) any later
version. This version of this license is in
   http://www.latex-project.org/lppl/lppl-1-3c.txt
and the latest version of this license is in
   http://www.latex-project.org/lppl.txt
and version 1.3 or later is part of all distributions of
LaTeX version 2005/12/01 or later.

This work has the LPPL maintenance status "maintained".

This Current Maintainer of this work is Heiko Oberdiek.

This work consists of the main source file accsupp.dtx
and the derived files
   accsupp.sty, accsupp.pdf, accsupp.ins, accsupp.drv,
   accsupp-pdftex.def, accsupp-luatex.def, accsupp-dvipdfm.def,
   accsupp-dvips.def,
   accsupp-example1.tex, accsupp-example2.tex,
   accsupp-test1.tex.

\endpreamble
\let\MetaPrefix\DoubleperCent

\generate{%
  \file{accsupp.ins}{\from{accsupp.dtx}{install}}%
  \file{accsupp.drv}{\from{accsupp.dtx}{driver}}%
  \usedir{tex/latex/oberdiek}%
  \file{accsupp.sty}{\from{accsupp.dtx}{package}}%
  \file{accsupp-pdftex.def}{\from{accsupp.dtx}{pdftex}}%
  \file{accsupp-luatex.def}{\from{accsupp.dtx}{luatex}}%
  \file{accsupp-dvipdfm.def}{\from{accsupp.dtx}{dvipdfm}}%
  \file{accsupp-dvips.def}{\from{accsupp.dtx}{dvips}}%
  \usedir{doc/latex/oberdiek}%
  \file{accsupp-example1.tex}{\from{accsupp.dtx}{example1}}%
  \file{accsupp-example2.tex}{\from{accsupp.dtx}{example2}}%
  \usedir{doc/latex/oberdiek/test}%
  \file{accsupp-test1.tex}{\from{accsupp.dtx}{test1}}%
  \nopreamble
  \nopostamble
  \usedir{source/latex/oberdiek/catalogue}%
  \file{accsupp.xml}{\from{accsupp.dtx}{catalogue}}%
}

\catcode32=13\relax% active space
\let =\space%
\Msg{************************************************************************}
\Msg{*}
\Msg{* To finish the installation you have to move the following}
\Msg{* files into a directory searched by TeX:}
\Msg{*}
\Msg{*     accsupp.sty, accsupp-pdftex.def, accsupp-dvipdfm.def,}
\Msg{*     accsupp-dvips.def}
\Msg{*}
\Msg{* To produce the documentation run the file `accsupp.drv'}
\Msg{* through LaTeX.}
\Msg{*}
\Msg{* Happy TeXing!}
\Msg{*}
\Msg{************************************************************************}

\endbatchfile
%</install>
%<*ignore>
\fi
%</ignore>
%<*driver>
\NeedsTeXFormat{LaTeX2e}
\ProvidesFile{accsupp.drv}%
  [2018/01/30 v0.5 Accessibility support by marked content (HO)]%
\documentclass{ltxdoc}
\usepackage{holtxdoc}[2011/11/22]
\begin{document}
  \DocInput{accsupp.dtx}%
\end{document}
%</driver>
% \fi
%
%
% \CharacterTable
%  {Upper-case    \A\B\C\D\E\F\G\H\I\J\K\L\M\N\O\P\Q\R\S\T\U\V\W\X\Y\Z
%   Lower-case    \a\b\c\d\e\f\g\h\i\j\k\l\m\n\o\p\q\r\s\t\u\v\w\x\y\z
%   Digits        \0\1\2\3\4\5\6\7\8\9
%   Exclamation   \!     Double quote  \"     Hash (number) \#
%   Dollar        \$     Percent       \%     Ampersand     \&
%   Acute accent  \'     Left paren    \(     Right paren   \)
%   Asterisk      \*     Plus          \+     Comma         \,
%   Minus         \-     Point         \.     Solidus       \/
%   Colon         \:     Semicolon     \;     Less than     \<
%   Equals        \=     Greater than  \>     Question mark \?
%   Commercial at \@     Left bracket  \[     Backslash     \\
%   Right bracket \]     Circumflex    \^     Underscore    \_
%   Grave accent  \`     Left brace    \{     Vertical bar  \|
%   Right brace   \}     Tilde         \~}
%
% \GetFileInfo{accsupp.drv}
%
% \title{The \xpackage{accsupp} package}
% \date{2018/01/30 v0.5}
% \author{Heiko Oberdiek\thanks
% {Please report any issues at https://github.com/ho-tex/oberdiek/issues}\\
% \xemail{heiko.oberdiek at googlemail.com}}
%
% \maketitle
%
% \begin{abstract}
% Since PDF 1.5 portions of a page can be marked for better
% accessibility support.
% For example, replacement texts or expansions of abbreviations can be
% provided. Package \xpackage{accsupp} starts with providing a minimal
% low-level interface for programmers. Status is experimental.
% \end{abstract}
%
% \tableofcontents
%
% \section{Documentation}
%
% \subsection{Macros}
%
% Section ``10.8 Accessibility Support'' of the PDF reference
% \cite{pdfspec} lists some features that can be added by
% operators for marked content.
%
% \begin{declcs}{BeginAccSupp} \M{options}
% \end{declcs}
% It puts the operator |BDC| in the page stream:
% \begin{quote}
%   |/Span|\\
%   |<<|\dots|>>| \qquad \textit{\% property dictionary}\\
%   |BDC|
% \end{quote}
% The contents of the dictionary is controlled by \meta{options}.
% See sections \ref{sec:feature} and \ref{sec:method}.
%
% \begin{declcs}{EndAccSupp} \M{options}
% \end{declcs}
% It puts the operator |EMC| in the page stream.
% The only option is \xoption{pdfliteral}, see section \ref{sec:pdftex}.
%
% \bigskip
% \noindent\textbf{Note:}
% The caller is responsible for the placement of \cs{BeginAccSupp}
% and \cs{EndAccSupp} pairs. Especially page breaks are not allowed
% in between.
%
% \subsubsection{Feature options}\label{sec:feature}
%
% The PDF reference \cite{pdfspec} describes and explains the different
% features. The names of the feature options are the same as the
% key names for the property dictionary for operator |BDC|, see
% \cs{BeginAccSupp}.
%
% \begin{description}
% \item[\xoption{ActualText}:]
%   Provides a replacement text, see examples in
%   section \ref{sec:example}.
% \item[\xoption{Alt}:]
%   Provides an alternate description.
% \item[\xoption{E}:]
%   Provides the expansion of an abbreviation or an acronym.
% \item[\xoption{Lang}:]
%   Specifies the language.
% \end{description}
%
% \subsubsection{Input methods}\label{sec:method}
%
% Except for \xoption{Lang} option \xoption{method} controls
% how the argument for \xoption{ActualText}, \xoption{Alt},
% and \xoption{E} are interpreted.
% \begin{description}
% \item[\xoption{method=plain}:]
%   The string is only expanded and written without further
%   treatment. Special characters are not protected, thus this method
%   may result in an invalid PDF file.
% \item[\xoption{mehod=escape}:]
%   The string is expanded and special characters are escaped.
%   The result is a valid PDF string.
% \item[\xoption{method=hex}:]
%   The string is given in hexadecimal notation.
%   Section \ref{sec:example:notparallel} shows an example.
% \item[\xoption{method=pdfstringdef}:]
%   If package \xpackage{hyperref} is loaded, then its \cs{pdfstringdef}
%   is used. This method is slow, but useful if the string contains
%   arbitrary \hologo{TeX} code.
% \item[\xoption{unicode}:]
%   This option is needed, if the string is given as Unicode string
%   (16 bit). Internally it adds the string prefix for Unicode.
%   In case of \xoption{method=pdfstringdef} it passes the option
%   to \cs{hypersetup}.
% \end{description}
%
% \subsection{Workaround, option \xoption{space}}
%
% PDF specification says in ``10.8.3 Replacement Text''
% \begin{quote}
% Just as alternative descriptions can be provided for images and other
% items that do not translate naturally into text (\dots), replacement
% text can be specified for content that does translate into text but
% that is represented in a nonstandard way. These nonstandard representations
% might include, for example, glyphs for ligatures or custom characters, or
% inline graphics corresponding to letters in an illuminated manustript or to
% dropped capitals.
% \end{quote}
% However, the ActualText is ignored in Acrobat Reader
% (until version 9 at least), if the content does not contain glyphs.
% Option \xoption{space} adds such an invisible glyph, a space
% character. The font name can be configured by option \xoption{spacefont},
% the default font is \texttt{phvr8r}. The character slot is given
% by option \xoption{spacechar}, default is \texttt{32}, the usual
% position of the space character.
%
% These options can also be given as package options or in
% macro \cs{AccSuppSetup} that takes a key value list as argument.
% Usually only option \xoption{space} is necessary, if the
% contents does not contain glyphs otherwise. Option \xoption{space}
% is enabled by |space| or |space=true| and disabled by |space=false|.
% It is disabled as default. The option is evaluated by \cs{BeginAccSupp}
% and ignored by \cs{EndAccSupp}.
%
% Note: Even with option \xoption{space} I get sometimes wrong texts
% when cut \& paste from AR7/Linux or AR8/Linux, e.g.
% \begin{quote}
%   |Hello| $\rightarrow$ |Helo|, |Helol|, |Hell|, \dots\\
%   |Hello World| $\rightarrow$ |Helo WorldW|, |Helo World|, \dots
% \end{quote}
% I do not know what Acrobat Reader is doing here, thus feedback and
% insights are welcome.
%
% \subsection{Driver options}
%
% Driver options are package options only.
% The special \hologo{TeX} compilers \hologo{pdfTeX}, \hologo{LuaTeX} and \hologo{XeTeX}
% are detected automatically. The default for unrecognized drivers can be
% set by defining \cs{ActualTextDriverDefault}. This can be done
% in the configuration file \xfile{accsupp.cfg}.
%
% \subsubsection{Option \xoption{pdftex}}\label{sec:pdftex}
% Package option \xoption{pdftex} is used for \hologo{pdfTeX} in PDF mode.
% Additionally \cs{BeginAccSupp} and \cs{EndAccSupp} understand
% option \xoption{pdfliteral}. It controls
% the modifier keyword for \cs{pdfliteral}:
% \begin{quote}
%   |pdfliteral=direct| $\Rightarrow$ |\pdfliteral direct{...}|
% \end{quote}
%
% \subsubsection{Option \xoption{luatex}}\label{sec:luatex}
% Package option \xoption{luatex} is used for \hologo{LuaTeX} in PDF mode.
% Additionally \cs{BeginAccSupp} and \cs{EndAccSupp} understand
% option \xoption{pdfliteral}. It controls
% the modifier keyword for \cs{pdfextension} literal:
% \begin{quote}
%   |pdfliteral=direct| $\Rightarrow$ |\pdfextension literal direct{...}|
% \end{quote}
%
% \subsubsection{Option \xoption{dvipdfm}}
%
% Package option \xoption{dvipdfm} and its aliases \xoption{dvipdfmx}
% \xoption{xetex} are used for drivers that support
% \textsf{dvipdfm} specials.
%
% \subsubsection{Option \xoption{dvips}}
%
% Package option \xoption{dvips} and its alias \xoption{dvipsone}
% write pdfmark specials in the output. Unhappily these pdfmark
% operators are ignored by \textsf{ghostscript} (latest tested
% version is 8.54). Perhaps they are recognized by commercial
% distiller applications.
%
% \subsubsection{Turning off page stream compression}
%
% For debugging it is useful to have uncompressed page stream objects.
% This can be done afterwards via \textsf{pdftk}:
% \begin{quote}
%   |pdftk file.pdf output file-uncompressed.pdf uncompress|
% \end{quote}
% Or the PDF file is generated uncompressed:
% \begin{description}
% \item[\hologo{pdfTeX}:] |\pdfcompresslevel=0|
% \item[dvipdfm:] |dvipdfm -z0 ...|
% \item[dvipdfmx:] |dvipdfmx -z0 ...|
% \item[ghostscript/ps2pdf:]
%      |ps2pdf -dCompressPages=false input.ps output.pdf|
% \end{description}
%
% \section{Example}
% \label{sec:example}
%
% \subsection{Example \cs{notparallel}}
% \label{sec:example:notparallel}
%
%    \begin{macrocode}
%<*example1>
%    \end{macrocode}
%    \begin{macrocode}
%<<END
\documentclass{article}
\usepackage{accsupp}[2007/11/14]
\usepackage{centernot}
% U+2226 NOT PARALLEL
% \mathrel{...} prevents page break in between
\newcommand*{\notparallel}{%
  \ensuremath{%
    \mathrel{%
      \BeginAccSupp{method=hex,unicode,ActualText=2226}%
      \centernot{\parallel}%
      \EndAccSupp{}%
    }%
  }%
}
\begin{document}
\begin{equation}
A\notparallel B
\end{equation}
\end{document}
%END
%    \end{macrocode}
%    \begin{macrocode}
%</example1>
%    \end{macrocode}
%
% \subsection{Example with \xoption{pdfstringdef}}
%
%    \begin{macrocode}
%<*example2>
%    \end{macrocode}
%    \begin{macrocode}
%<<END
\documentclass{article}
\usepackage[unicode]{hyperref}
\usepackage{accsupp}[2007/11/14]
\begin{document}
  \begin{equation}
    \BeginAccSupp{
      method=pdfstringdef,
      unicode,
      ActualText={%
        a\texttwosuperior +b\texttwosuperior
        =c\texttwosuperior
      }
    }
    a^2 + b^2 = c^2
    \EndAccSupp{}
  \end{equation}
\end{document}
%END
%    \end{macrocode}
%    \begin{macrocode}
%</example2>
%    \end{macrocode}
%
% \StopEventually{
% }
%
% \section{Implementation}
%
% \subsection{Package}
%
%    \begin{macrocode}
%<*package>
%    \end{macrocode}
%
%    \begin{macrocode}
\begingroup\catcode61\catcode48\catcode32=10\relax%
  \catcode13=5 % ^^M
  \endlinechar=13 %
  \catcode123=1 % {
  \catcode125=2 % }
  \catcode64=11 % @
  \def\x{\endgroup
    \expandafter\edef\csname ACCSUPP@AtEnd\endcsname{%
      \endlinechar=\the\endlinechar\relax
      \catcode13=\the\catcode13\relax
      \catcode32=\the\catcode32\relax
      \catcode35=\the\catcode35\relax
      \catcode61=\the\catcode61\relax
      \catcode64=\the\catcode64\relax
      \catcode123=\the\catcode123\relax
      \catcode125=\the\catcode125\relax
    }%
  }%
\x\catcode61\catcode48\catcode32=10\relax%
\catcode13=5 % ^^M
\endlinechar=13 %
\catcode35=6 % #
\catcode64=11 % @
\catcode123=1 % {
\catcode125=2 % }
\def\TMP@EnsureCode#1#2{%
  \edef\ACCSUPP@AtEnd{%
    \ACCSUPP@AtEnd
    \catcode#1=\the\catcode#1\relax
  }%
  \catcode#1=#2\relax
}
\TMP@EnsureCode{10}{12}% ^^J
\TMP@EnsureCode{33}{12}% !
\TMP@EnsureCode{39}{12}% '
\TMP@EnsureCode{40}{12}% (
\TMP@EnsureCode{41}{12}% )
\TMP@EnsureCode{42}{12}% *
\TMP@EnsureCode{44}{12}% ,
\TMP@EnsureCode{45}{12}% -
\TMP@EnsureCode{46}{12}% .
\TMP@EnsureCode{47}{12}% /
\TMP@EnsureCode{58}{12}% :
\TMP@EnsureCode{60}{12}% <
\TMP@EnsureCode{62}{12}% >
\TMP@EnsureCode{94}{7}% ^ (superscript)
\TMP@EnsureCode{96}{12}% `
\TMP@EnsureCode{254}{12}% ^^fe
\TMP@EnsureCode{255}{12}% ^^ff
\edef\ACCSUPP@AtEnd{\ACCSUPP@AtEnd\noexpand\endinput}
%    \end{macrocode}
%
%    Package identification.
%    \begin{macrocode}
\NeedsTeXFormat{LaTeX2e}
\ProvidesPackage{accsupp}%
  [2018/01/30 v0.5 Accessibility support by marked content (HO)]
%    \end{macrocode}
%
%    \begin{macrocode}
\RequirePackage{pdfescape}[2007/02/25]
\RequirePackage{ifpdf}
\RequirePackage{ifxetex}
\RequirePackage{kvoptions}
%    \end{macrocode}
%    \begin{macrocode}
\SetupKeyvalOptions{%
  family=ACCSUPP,%
  prefix=ACCSUPP@%
}
%    \end{macrocode}
%
% \subsection{Space setup}
%
%    \begin{macrocode}
\DeclareBoolOption{space}
\DeclareStringOption[phvr8r]{spacefont}
\DeclareStringOption[32]{spacechar}
%    \end{macrocode}
%
% \subsection{Driver detection and setup}
%
% \paragraph{Driver declarations.}
%
%    \begin{macrocode}
\def\ACCSUPP@DefineDriverKey{%
  \@dblarg\ACCSUPP@@DefineDriverKey
}
\def\ACCSUPP@@DefineDriverKey[#1]#2{%
  \define@key{ACCSUPP}{#2}[]{%
    \def\ACCSUPP@driver{#1}%
  }%
  \g@addto@macro\ACCSUPP@DisableOptions{%
    \DisableKeyvalOption{ACCSUPP}{#2}%
  }%
}
\let\ACCSUPP@DisableOptions\@empty
\ACCSUPP@DefineDriverKey{pdftex}
\ACCSUPP@DefineDriverKey{luatex}
\ACCSUPP@DefineDriverKey{dvips}
\ACCSUPP@DefineDriverKey[dvips]{dvipsone}
\ACCSUPP@DefineDriverKey{dvipdfm}
\ACCSUPP@DefineDriverKey[dvipdfm]{dvipdfmx}
\ACCSUPP@DefineDriverKey[dvipdfm]{xetex}
\let\ACCSUPP@driver\relax
\InputIfFileExists{accsupp.cfg}{}{}
\providecommand*{\ActualTextDriverDefault}{dvips}
\ifpdf
  \ifx\pdfextension\@undefined
    \def\ACCSUPP@driver{pdftex}%
  \else
    \def\ACCSUPP@driver{luatex}%
  \fi
\else
  \ifxetex
    \def\ACCSUPP@driver{dvipdfm}%
  \else
    \ifx\ACCSUPP@driver\relax
      \let\ACCSUPP@driver\ActualTextDriverDefault
    \fi
  \fi
\fi
%    \end{macrocode}
%
% \paragraph{Process options.}
%
%    \begin{macrocode}
\ProcessKeyvalOptions*
\ACCSUPP@DisableOptions
%    \end{macrocode}
%
% \paragraph{Driver validation and loading.}
%
%    \begin{macrocode}
\def\ACCSUPP@temp{pdftex}%
\ifpdf
  \ifx\pdfextension\@undefined\else
    \def\ACCSUPP@temp{luatex}%
  \fi
  \ifx\ACCSUPP@temp\ACCSUPP@driver
  \else
    \PackageWarningNoLine{accsupp}{%
      Wrong driver `\ACCSUPP@driver', using `pdftex' instead%
    }%
ed as
    \let\ACCSUPP@driver\ACCSUPP@temp
  \fi
\else
  \ifx\ACCSUPP@temp\ACCSUPP@driver
    \PackageError{accsupp}{%
      Wrong driver, pdfTeX is not running in PDF mode.\MessageBreak
      Package loading is aborted%
    }\@ehc
    \expandafter\expandafter\expandafter\ACCSUPP@AtEnd
  \fi
  \def\ACCSUPP@temp{dvipdfm}%
  \ifxetex
    \ifx\ACCSUPP@temp\ACCSUPP@driver
    \else
      \PackageWarningNoLine{accsupp}{%
        Wrong driver `\ACCSUPP@driver',\MessageBreak
        using `dvipdfm' for XeTeX instead%
      }%
      \let\ACCSUPP@driver\ACCSUPP@temp
    \fi
  \fi
\fi%
\ifx\ACCSUPP@driver\relax
  \PackageError{accsupp}{%
    Missing driver option.\MessageBreak
    Package loading is aborted%
  }\@ehc
  \expandafter\ACCSUPP@AtEnd
\fi%
\InputIfFileExists{accsupp-\ACCSUPP@driver.def}{}{%
  \PackageError{accsupp}{%
    Unsupported driver `\ACCSUPP@driver'.\MessageBreak
    Package loading is aborted%
  }\@ehc
  \ACCSUPP@AtEnd
}%
%    \end{macrocode}
%
% \subsection{Main macro}
%
%    \begin{macrocode}
\DeclareBoolOption{unicode}
\DeclareStringOption[page]{pdfliteral}
\DeclareStringOption{Lang}
\def\ACCSUPP@method{escape}
\define@key{ACCSUPP}{method}{%
  \@ifundefined{ACCSUPP@method@#1}{%
    \PackageError{accsupp}{%
      Ignoring unknown method `#1'%
    }\@ehc
  }{%
    \edef\ACCSUPP@method{#1}%
  }%
}
\let\ACCSUPP@Lang\relax
\def\ACCSUPP@temp#1{%
  \expandafter\ACCSUPP@@temp\csname ACCSUPP@#1\endcsname{#1}%
}
\def\ACCSUPP@@temp#1#2{%
  \let#1\relax
  \define@key{ACCSUPP}{#2}{%
    \def#1{##1}%
    \ifx#1\@empty
      \def#1{()}%
    \else
      \csname ACCSUPP@method@\ACCSUPP@method\endcsname#1%
    \fi
  }%
}
\ACCSUPP@temp{Alt}
\ACCSUPP@temp{ActualText}
\ACCSUPP@temp{E}
%    \end{macrocode}
%
%    \begin{macrocode}
\newcommand*{\BeginAccSupp}[1]{%
  \begingroup
    \setkeys{ACCSUPP}{#1}%
    \edef\ACCSUPP@span{%
      /Span<<%
        \ifx\ACCSUPP@Lang\relax
        \else
          /Lang\ACCSUPP@Lang
        \fi
        \ifx\ACCSUPP@Alt\relax
        \else
          /Alt\ACCSUPP@Alt
        \fi
        \ifx\ACCSUPP@ActualText\relax
        \else
          /ActualText\ACCSUPP@ActualText
        \fi
        \ifx\ACCSUPP@E\relax
        \else
          /E\ACCSUPP@E
        \fi
      >>%
    }%
    \ACCSUPP@bdc
    \ACCSUPP@space
  \endgroup
}
\newcommand*{\EndAccSupp}[1]{%
  \begingroup
    \setkeys{ACCSUPP}{#1}%
    \ACCSUPP@emc
  \endgroup
}
\def\ACCSUPP@space{%
  \ifACCSUPP@space
    \begingroup
      \@ifundefined{ACCSUPP@Font}{%
        \global\font\ACCSUPP@Font=\ACCSUPP@spacefont\relax
      }{}%
      \leavevmode
      \setbox\z@\hbox{\ACCSUPP@Font\char\ACCSUPP@spacechar}%
      \wd\z@\z@
      \ht\z@\z@
      \dp\z@\z@
      \copy\z@
    \endgroup
  \fi
}
\newcommand*{\AccSuppSetup}{%
  \setkeys{ACCSUPP}%
}
%    \end{macrocode}
%
% \subsubsection{Input methods}
%
%    \begin{macrocode}
\def\ACCSUPP@method@plain#1{%
  \csname @safe@activestrue\endcsname
  \edef#1{%
    (%
    \ifACCSUPP@unicode
      \string\376\string\377%
    \fi
    #1%
    )%
  }%
  \@onelevel@sanitize#1%
}
%    \end{macrocode}
%    \begin{macrocode}
\def\ACCSUPP@method@escape#1{%
  \EdefEscapeString#1{%
    \ifACCSUPP@unicode
      ^^fe^^ff%
    \fi
    #1%
  }%
  \edef#1{(#1)}%
}%
%    \end{macrocode}
%    \begin{macrocode}
\def\ACCSUPP@method@hex#1{%
  \edef#1{%
    <%
    \ifACCSUPP@unicode
      FEFF%
    \fi
    #1%
    >%
  }%
}
%    \end{macrocode}
%    \begin{macrocode}
\def\ACCSUPP@method@pdfstringdef#1{%
  \ifACCSUPP@unicode
    \@ifundefined{hypersetup}{}{%
      \hypersetup{unicode}%
    }%
  \fi
  \@ifundefined{pdfstringdef}{%
    \PackageError{accsupp}{%
      Method `pdfstringdef' requires package `hyperref'%
    }\@ehc
    \let\ACCSUPP@temp\@empty
  }{%
    \begingroup
      \setbox0=\hbox{%
        \pdfstringdef\ACCSUPP@temp#1%
        \global\let\ACCSUPP@temp\ACCSUPP@temp
      }%
    \endgroup
  }%
  \edef#1{(\ACCSUPP@temp)}%
}
%    \end{macrocode}
%
%    \begin{macrocode}
\ACCSUPP@AtEnd%
%</package>
%    \end{macrocode}
%
% \subsection{Drivers}
%
% \subsubsection{Driver \xoption{pdftex}}
%
%    \begin{macrocode}
%<*pdftex>
\NeedsTeXFormat{LaTeX2e}
\ProvidesFile{accsupp-pdftex.def}%
  [2018/01/30 v0.5 accsupp driver for pdfTeX (HO)]%
%    \end{macrocode}
%
%    \begin{macrocode}
\def\ACCSUPP@bdc{%
  \pdfliteral\ACCSUPP@pdfliteral{\ACCSUPP@span BDC}%
}
\def\ACCSUPP@emc{%
  \pdfliteral\ACCSUPP@pdfliteral{EMC}%
}
%    \end{macrocode}
%    \begin{macrocode}
%</pdftex>
%    \end{macrocode}
%
% \subsubsection{Driver \xoption{luatex}}
%
%    \begin{macrocode}
%<*luatex>
\NeedsTeXFormat{LaTeX2e}
\ProvidesFile{accsupp-luatex.def}%
  [2018/01/30 v0.5 accsupp driver for pdfTeX (HO)]%
%    \end{macrocode}
%
%    \begin{macrocode}
\protected\def\ACCSUPP@pdf@literal{\pdfextension literal}
\def\ACCSUPP@bdc{%
  \ACCSUPP@pdf@literal\ACCSUPP@pdfliteral{\ACCSUPP@span BDC}%
}
\def\ACCSUPP@emc{%
  \ACCSUPP@pdf@literal\ACCSUPP@pdfliteral{EMC}%
}
%    \end{macrocode}
%    \begin{macrocode}
%</luatex>
%    \end{macrocode}
%
% \subsubsection{Driver \xoption{dvipdfm}}
%
%    \begin{macrocode}
%<*dvipdfm>
\NeedsTeXFormat{LaTeX2e}
\ProvidesFile{accsupp-dvipdfm.def}%
  [2018/01/30 v0.5 accsupp driver for dvipdfm (HO)]%
%    \end{macrocode}
%
%    \begin{macrocode}
\def\ACCSUPP@bdc{%
  \special{pdf:content \ACCSUPP@span BDC}%
}
\def\ACCSUPP@emc{%
  \special{pdf:content EMC}%
}
%    \end{macrocode}
%    \begin{macrocode}
%</dvipdfm>
%    \end{macrocode}
%
% \subsubsection{Driver \xoption{dvips}}
%
%    \begin{macrocode}
%<*dvips>
\NeedsTeXFormat{LaTeX2e}
\ProvidesFile{accsupp-dvips.def}%
  [2018/01/30 v0.5 accsupp driver for dvips (HO)]%
%    \end{macrocode}
%
%    \begin{macrocode}
\def\ACCSUPP@bdc{%
  \special{ps:[\ACCSUPP@span/BDC pdfmark}%
}
\def\ACCSUPP@emc{%
  \special{ps:[/EMC pdfmark}%
}
%    \end{macrocode}
%    \begin{macrocode}
%</dvips>
%    \end{macrocode}
%
% \section{Test}
%
% \subsection{Catcode checks for loading}
%
%    \begin{macrocode}
%<*test1>
%    \end{macrocode}
%    \begin{macrocode}
\NeedsTeXFormat{LaTeX2e}
\documentclass{minimal}
\makeatletter
\def\RestoreCatcodes{}
\count@=0 %
\loop
  \edef\RestoreCatcodes{%
    \RestoreCatcodes
    \catcode\the\count@=\the\catcode\count@\relax
  }%
\ifnum\count@<255 %
  \advance\count@\@ne
\repeat

\def\RangeCatcodeInvalid#1#2{%
  \count@=#1\relax
  \loop
    \catcode\count@=15 %
  \ifnum\count@<#2\relax
    \advance\count@\@ne
  \repeat
}
\def\Test{%
  \RangeCatcodeInvalid{0}{47}%
  \RangeCatcodeInvalid{58}{64}%
  \RangeCatcodeInvalid{91}{96}%
  \RangeCatcodeInvalid{123}{127}%
  \catcode`\@=12 %
  \catcode`\\=0 %
  \catcode`\{=1 %
  \catcode`\}=2 %
  \catcode`\#=6 %
  \catcode`\[=12 %
  \catcode`\]=12 %
  \catcode`\%=14 %
  \catcode`\ =10 %
  \catcode13=5 %
  \RequirePackage{accsupp}[2018/01/30]\relax
  \RestoreCatcodes
}
\Test
\csname @@end\endcsname
\end
%    \end{macrocode}
%    \begin{macrocode}
%</test1>
%    \end{macrocode}
%
% \section{Installation}
%
% \subsection{Download}
%
% \paragraph{Package.} This package is available on
% CTAN\footnote{\url{http://ctan.org/pkg/accsupp}}:
% \begin{description}
% \item[\CTAN{macros/latex/contrib/oberdiek/accsupp.dtx}] The source file.
% \item[\CTAN{macros/latex/contrib/oberdiek/accsupp.pdf}] Documentation.
% \end{description}
%
%
% \paragraph{Bundle.} All the packages of the bundle `oberdiek'
% are also available in a TDS compliant ZIP archive. There
% the packages are already unpacked and the documentation files
% are generated. The files and directories obey the TDS standard.
% \begin{description}
% \item[\CTAN{install/macros/latex/contrib/oberdiek.tds.zip}]
% \end{description}
% \emph{TDS} refers to the standard ``A Directory Structure
% for \TeX\ Files'' (\CTAN{tds/tds.pdf}). Directories
% with \xfile{texmf} in their name are usually organized this way.
%
% \subsection{Bundle installation}
%
% \paragraph{Unpacking.} Unpack the \xfile{oberdiek.tds.zip} in the
% TDS tree (also known as \xfile{texmf} tree) of your choice.
% Example (linux):
% \begin{quote}
%   |unzip oberdiek.tds.zip -d ~/texmf|
% \end{quote}
%
% \paragraph{Script installation.}
% Check the directory \xfile{TDS:scripts/oberdiek/} for
% scripts that need further installation steps.
% Package \xpackage{attachfile2} comes with the Perl script
% \xfile{pdfatfi.pl} that should be installed in such a way
% that it can be called as \texttt{pdfatfi}.
% Example (linux):
% \begin{quote}
%   |chmod +x scripts/oberdiek/pdfatfi.pl|\\
%   |cp scripts/oberdiek/pdfatfi.pl /usr/local/bin/|
% \end{quote}
%
% \subsection{Package installation}
%
% \paragraph{Unpacking.} The \xfile{.dtx} file is a self-extracting
% \docstrip\ archive. The files are extracted by running the
% \xfile{.dtx} through \plainTeX:
% \begin{quote}
%   \verb|tex accsupp.dtx|
% \end{quote}
%
% \paragraph{TDS.} Now the different files must be moved into
% the different directories in your installation TDS tree
% (also known as \xfile{texmf} tree):
% \begin{quote}
% \def\t{^^A
% \begin{tabular}{@{}>{\ttfamily}l@{ $\rightarrow$ }>{\ttfamily}l@{}}
%   accsupp.sty & tex/latex/oberdiek/accsupp.sty\\
%   accsupp-pdftex.def & tex/latex/oberdiek/accsupp-pdftex.def\\
%   accsupp-luatex.def & tex/latex/oberdiek/accsupp-luatex.def\\
%   accsupp-dvipdfm.def & tex/latex/oberdiek/accsupp-dvipdfm.def\\
%   accsupp-dvips.def & tex/latex/oberdiek/accsupp-dvips.def\\
%   accsupp.pdf & doc/latex/oberdiek/accsupp.pdf\\
%   accsupp-example1.tex & doc/latex/oberdiek/accsupp-example1.tex\\
%   accsupp-example2.tex & doc/latex/oberdiek/accsupp-example2.tex\\
%   test/accsupp-test1.tex & doc/latex/oberdiek/test/accsupp-test1.tex\\
%   accsupp.dtx & source/latex/oberdiek/accsupp.dtx\\
% \end{tabular}^^A
% }^^A
% \sbox0{\t}^^A
% \ifdim\wd0>\linewidth
%   \begingroup
%     \advance\linewidth by\leftmargin
%     \advance\linewidth by\rightmargin
%   \edef\x{\endgroup
%     \def\noexpand\lw{\the\linewidth}^^A
%   }\x
%   \def\lwbox{^^A
%     \leavevmode
%     \hbox to \linewidth{^^A
%       \kern-\leftmargin\relax
%       \hss
%       \usebox0
%       \hss
%       \kern-\rightmargin\relax
%     }^^A
%   }^^A
%   \ifdim\wd0>\lw
%     \sbox0{\small\t}^^A
%     \ifdim\wd0>\linewidth
%       \ifdim\wd0>\lw
%         \sbox0{\footnotesize\t}^^A
%         \ifdim\wd0>\linewidth
%           \ifdim\wd0>\lw
%             \sbox0{\scriptsize\t}^^A
%             \ifdim\wd0>\linewidth
%               \ifdim\wd0>\lw
%                 \sbox0{\tiny\t}^^A
%                 \ifdim\wd0>\linewidth
%                   \lwbox
%                 \else
%                   \usebox0
%                 \fi
%               \else
%                 \lwbox
%               \fi
%             \else
%               \usebox0
%             \fi
%           \else
%             \lwbox
%           \fi
%         \else
%           \usebox0
%         \fi
%       \else
%         \lwbox
%       \fi
%     \else
%       \usebox0
%     \fi
%   \else
%     \lwbox
%   \fi
% \else
%   \usebox0
% \fi
% \end{quote}
% If you have a \xfile{docstrip.cfg} that configures and enables \docstrip's
% TDS installing feature, then some files can already be in the right
% place, see the documentation of \docstrip.
%
% \subsection{Refresh file name databases}
%
% If your \TeX~distribution
% (\teTeX, \mikTeX, \dots) relies on file name databases, you must refresh
% these. For example, \teTeX\ users run \verb|texhash| or
% \verb|mktexlsr|.
%
% \subsection{Some details for the interested}
%
% \paragraph{Attached source.}
%
% The PDF documentation on CTAN also includes the
% \xfile{.dtx} source file. It can be extracted by
% AcrobatReader 6 or higher. Another option is \textsf{pdftk},
% e.g. unpack the file into the current directory:
% \begin{quote}
%   \verb|pdftk accsupp.pdf unpack_files output .|
% \end{quote}
%
% \paragraph{Unpacking with \LaTeX.}
% The \xfile{.dtx} chooses its action depending on the format:
% \begin{description}
% \item[\plainTeX:] Run \docstrip\ and extract the files.
% \item[\LaTeX:] Generate the documentation.
% \end{description}
% If you insist on using \LaTeX\ for \docstrip\ (really,
% \docstrip\ does not need \LaTeX), then inform the autodetect routine
% about your intention:
% \begin{quote}
%   \verb|latex \let\install=y% \iffalse meta-comment
%
% File: accsupp.dtx
% Version: 2018/01/30 v0.5
% Info: Accessibility support by marked content
%
% Copyright (C) 2007, 2010 by
%    Heiko Oberdiek <heiko.oberdiek at googlemail.com>
%    2016
%    https://github.com/ho-tex/oberdiek/issues
%
% This work may be distributed and/or modified under the
% conditions of the LaTeX Project Public License, either
% version 1.3c of this license or (at your option) any later
% version. This version of this license is in
%    http://www.latex-project.org/lppl/lppl-1-3c.txt
% and the latest version of this license is in
%    http://www.latex-project.org/lppl.txt
% and version 1.3 or later is part of all distributions of
% LaTeX version 2005/12/01 or later.
%
% This work has the LPPL maintenance status "maintained".
%
% This Current Maintainer of this work is Heiko Oberdiek.
%
% This work consists of the main source file accsupp.dtx
% and the derived files
%    accsupp.sty, accsupp.pdf, accsupp.ins, accsupp.drv,
%    accsupp-pdftex.def, \ccsupp-luatex.def, accsupp-dvipdfm.def,
%    accsupp-dvips.def,
%    accsupp-example1.tex, accsupp-example2.tex,
%    accsupp-test1.tex.
%
% Distribution:
%    CTAN:macros/latex/contrib/oberdiek/accsupp.dtx
%    CTAN:macros/latex/contrib/oberdiek/accsupp.pdf
%
% Unpacking:
%    (a) If accsupp.ins is present:
%           tex accsupp.ins
%    (b) Without accsupp.ins:
%           tex accsupp.dtx
%    (c) If you insist on using LaTeX
%           latex \let\install=y\input{accsupp.dtx}
%        (quote the arguments according to the demands of your shell)
%
% Documentation:
%    (a) If accsupp.drv is present:
%           latex accsupp.drv
%    (b) Without accsupp.drv:
%           latex accsupp.dtx; ...
%    The class ltxdoc loads the configuration file ltxdoc.cfg
%    if available. Here you can specify further options, e.g.
%    use A4 as paper format:
%       \PassOptionsToClass{a4paper}{article}
%
%    Programm calls to get the documentation (example):
%       pdflatex accsupp.dtx
%       makeindex -s gind.ist accsupp.idx
%       pdflatex accsupp.dtx
%       makeindex -s gind.ist accsupp.idx
%       pdflatex accsupp.dtx
%
% Installation:
%    TDS:tex/latex/oberdiek/accsupp.sty
%    TDS:tex/latex/oberdiek/accsupp-pdftex.def
%    TDS:tex/latex/oberdiek/accsupp-luatex.def
%    TDS:tex/latex/oberdiek/accsupp-dvipdfm.def
%    TDS:tex/latex/oberdiek/accsupp-dvips.def
%    TDS:doc/latex/oberdiek/accsupp.pdf
%    TDS:doc/latex/oberdiek/accsupp-example1.tex
%    TDS:doc/latex/oberdiek/accsupp-example2.tex
%    TDS:doc/latex/oberdiek/test/accsupp-test1.tex
%    TDS:source/latex/oberdiek/accsupp.dtx
%
%<*ignore>
\begingroup
  \catcode123=1 %
  \catcode125=2 %
  \def\x{LaTeX2e}%
\expandafter\endgroup
\ifcase 0\ifx\install y1\fi\expandafter
         \ifx\csname processbatchFile\endcsname\relax\else1\fi
         \ifx\fmtname\x\else 1\fi\relax
\else\csname fi\endcsname
%</ignore>
%<*install>
\input docstrip.tex
\Msg{************************************************************************}
\Msg{* Installation}
\Msg{* Package: accsupp 2018/01/30 v0.5 Accessibility support by marked content (HO)}
\Msg{************************************************************************}

\keepsilent
\askforoverwritefalse

\let\MetaPrefix\relax
\preamble

This is a generated file.

Project: accsupp
Version: 2018/01/30 v0.5

Copyright (C) 2007, 2010 by
   Heiko Oberdiek <heiko.oberdiek at googlemail.com>

This work may be distributed and/or modified under the
conditions of the LaTeX Project Public License, either
version 1.3c of this license or (at your option) any later
version. This version of this license is in
   http://www.latex-project.org/lppl/lppl-1-3c.txt
and the latest version of this license is in
   http://www.latex-project.org/lppl.txt
and version 1.3 or later is part of all distributions of
LaTeX version 2005/12/01 or later.

This work has the LPPL maintenance status "maintained".

This Current Maintainer of this work is Heiko Oberdiek.

This work consists of the main source file accsupp.dtx
and the derived files
   accsupp.sty, accsupp.pdf, accsupp.ins, accsupp.drv,
   accsupp-pdftex.def, accsupp-luatex.def, accsupp-dvipdfm.def,
   accsupp-dvips.def,
   accsupp-example1.tex, accsupp-example2.tex,
   accsupp-test1.tex.

\endpreamble
\let\MetaPrefix\DoubleperCent

\generate{%
  \file{accsupp.ins}{\from{accsupp.dtx}{install}}%
  \file{accsupp.drv}{\from{accsupp.dtx}{driver}}%
  \usedir{tex/latex/oberdiek}%
  \file{accsupp.sty}{\from{accsupp.dtx}{package}}%
  \file{accsupp-pdftex.def}{\from{accsupp.dtx}{pdftex}}%
  \file{accsupp-luatex.def}{\from{accsupp.dtx}{luatex}}%
  \file{accsupp-dvipdfm.def}{\from{accsupp.dtx}{dvipdfm}}%
  \file{accsupp-dvips.def}{\from{accsupp.dtx}{dvips}}%
  \usedir{doc/latex/oberdiek}%
  \file{accsupp-example1.tex}{\from{accsupp.dtx}{example1}}%
  \file{accsupp-example2.tex}{\from{accsupp.dtx}{example2}}%
  \usedir{doc/latex/oberdiek/test}%
  \file{accsupp-test1.tex}{\from{accsupp.dtx}{test1}}%
  \nopreamble
  \nopostamble
  \usedir{source/latex/oberdiek/catalogue}%
  \file{accsupp.xml}{\from{accsupp.dtx}{catalogue}}%
}

\catcode32=13\relax% active space
\let =\space%
\Msg{************************************************************************}
\Msg{*}
\Msg{* To finish the installation you have to move the following}
\Msg{* files into a directory searched by TeX:}
\Msg{*}
\Msg{*     accsupp.sty, accsupp-pdftex.def, accsupp-dvipdfm.def,}
\Msg{*     accsupp-dvips.def}
\Msg{*}
\Msg{* To produce the documentation run the file `accsupp.drv'}
\Msg{* through LaTeX.}
\Msg{*}
\Msg{* Happy TeXing!}
\Msg{*}
\Msg{************************************************************************}

\endbatchfile
%</install>
%<*ignore>
\fi
%</ignore>
%<*driver>
\NeedsTeXFormat{LaTeX2e}
\ProvidesFile{accsupp.drv}%
  [2018/01/30 v0.5 Accessibility support by marked content (HO)]%
\documentclass{ltxdoc}
\usepackage{holtxdoc}[2011/11/22]
\begin{document}
  \DocInput{accsupp.dtx}%
\end{document}
%</driver>
% \fi
%
%
% \CharacterTable
%  {Upper-case    \A\B\C\D\E\F\G\H\I\J\K\L\M\N\O\P\Q\R\S\T\U\V\W\X\Y\Z
%   Lower-case    \a\b\c\d\e\f\g\h\i\j\k\l\m\n\o\p\q\r\s\t\u\v\w\x\y\z
%   Digits        \0\1\2\3\4\5\6\7\8\9
%   Exclamation   \!     Double quote  \"     Hash (number) \#
%   Dollar        \$     Percent       \%     Ampersand     \&
%   Acute accent  \'     Left paren    \(     Right paren   \)
%   Asterisk      \*     Plus          \+     Comma         \,
%   Minus         \-     Point         \.     Solidus       \/
%   Colon         \:     Semicolon     \;     Less than     \<
%   Equals        \=     Greater than  \>     Question mark \?
%   Commercial at \@     Left bracket  \[     Backslash     \\
%   Right bracket \]     Circumflex    \^     Underscore    \_
%   Grave accent  \`     Left brace    \{     Vertical bar  \|
%   Right brace   \}     Tilde         \~}
%
% \GetFileInfo{accsupp.drv}
%
% \title{The \xpackage{accsupp} package}
% \date{2018/01/30 v0.5}
% \author{Heiko Oberdiek\thanks
% {Please report any issues at https://github.com/ho-tex/oberdiek/issues}\\
% \xemail{heiko.oberdiek at googlemail.com}}
%
% \maketitle
%
% \begin{abstract}
% Since PDF 1.5 portions of a page can be marked for better
% accessibility support.
% For example, replacement texts or expansions of abbreviations can be
% provided. Package \xpackage{accsupp} starts with providing a minimal
% low-level interface for programmers. Status is experimental.
% \end{abstract}
%
% \tableofcontents
%
% \section{Documentation}
%
% \subsection{Macros}
%
% Section ``10.8 Accessibility Support'' of the PDF reference
% \cite{pdfspec} lists some features that can be added by
% operators for marked content.
%
% \begin{declcs}{BeginAccSupp} \M{options}
% \end{declcs}
% It puts the operator |BDC| in the page stream:
% \begin{quote}
%   |/Span|\\
%   |<<|\dots|>>| \qquad \textit{\% property dictionary}\\
%   |BDC|
% \end{quote}
% The contents of the dictionary is controlled by \meta{options}.
% See sections \ref{sec:feature} and \ref{sec:method}.
%
% \begin{declcs}{EndAccSupp} \M{options}
% \end{declcs}
% It puts the operator |EMC| in the page stream.
% The only option is \xoption{pdfliteral}, see section \ref{sec:pdftex}.
%
% \bigskip
% \noindent\textbf{Note:}
% The caller is responsible for the placement of \cs{BeginAccSupp}
% and \cs{EndAccSupp} pairs. Especially page breaks are not allowed
% in between.
%
% \subsubsection{Feature options}\label{sec:feature}
%
% The PDF reference \cite{pdfspec} describes and explains the different
% features. The names of the feature options are the same as the
% key names for the property dictionary for operator |BDC|, see
% \cs{BeginAccSupp}.
%
% \begin{description}
% \item[\xoption{ActualText}:]
%   Provides a replacement text, see examples in
%   section \ref{sec:example}.
% \item[\xoption{Alt}:]
%   Provides an alternate description.
% \item[\xoption{E}:]
%   Provides the expansion of an abbreviation or an acronym.
% \item[\xoption{Lang}:]
%   Specifies the language.
% \end{description}
%
% \subsubsection{Input methods}\label{sec:method}
%
% Except for \xoption{Lang} option \xoption{method} controls
% how the argument for \xoption{ActualText}, \xoption{Alt},
% and \xoption{E} are interpreted.
% \begin{description}
% \item[\xoption{method=plain}:]
%   The string is only expanded and written without further
%   treatment. Special characters are not protected, thus this method
%   may result in an invalid PDF file.
% \item[\xoption{mehod=escape}:]
%   The string is expanded and special characters are escaped.
%   The result is a valid PDF string.
% \item[\xoption{method=hex}:]
%   The string is given in hexadecimal notation.
%   Section \ref{sec:example:notparallel} shows an example.
% \item[\xoption{method=pdfstringdef}:]
%   If package \xpackage{hyperref} is loaded, then its \cs{pdfstringdef}
%   is used. This method is slow, but useful if the string contains
%   arbitrary \hologo{TeX} code.
% \item[\xoption{unicode}:]
%   This option is needed, if the string is given as Unicode string
%   (16 bit). Internally it adds the string prefix for Unicode.
%   In case of \xoption{method=pdfstringdef} it passes the option
%   to \cs{hypersetup}.
% \end{description}
%
% \subsection{Workaround, option \xoption{space}}
%
% PDF specification says in ``10.8.3 Replacement Text''
% \begin{quote}
% Just as alternative descriptions can be provided for images and other
% items that do not translate naturally into text (\dots), replacement
% text can be specified for content that does translate into text but
% that is represented in a nonstandard way. These nonstandard representations
% might include, for example, glyphs for ligatures or custom characters, or
% inline graphics corresponding to letters in an illuminated manustript or to
% dropped capitals.
% \end{quote}
% However, the ActualText is ignored in Acrobat Reader
% (until version 9 at least), if the content does not contain glyphs.
% Option \xoption{space} adds such an invisible glyph, a space
% character. The font name can be configured by option \xoption{spacefont},
% the default font is \texttt{phvr8r}. The character slot is given
% by option \xoption{spacechar}, default is \texttt{32}, the usual
% position of the space character.
%
% These options can also be given as package options or in
% macro \cs{AccSuppSetup} that takes a key value list as argument.
% Usually only option \xoption{space} is necessary, if the
% contents does not contain glyphs otherwise. Option \xoption{space}
% is enabled by |space| or |space=true| and disabled by |space=false|.
% It is disabled as default. The option is evaluated by \cs{BeginAccSupp}
% and ignored by \cs{EndAccSupp}.
%
% Note: Even with option \xoption{space} I get sometimes wrong texts
% when cut \& paste from AR7/Linux or AR8/Linux, e.g.
% \begin{quote}
%   |Hello| $\rightarrow$ |Helo|, |Helol|, |Hell|, \dots\\
%   |Hello World| $\rightarrow$ |Helo WorldW|, |Helo World|, \dots
% \end{quote}
% I do not know what Acrobat Reader is doing here, thus feedback and
% insights are welcome.
%
% \subsection{Driver options}
%
% Driver options are package options only.
% The special \hologo{TeX} compilers \hologo{pdfTeX}, \hologo{LuaTeX} and \hologo{XeTeX}
% are detected automatically. The default for unrecognized drivers can be
% set by defining \cs{ActualTextDriverDefault}. This can be done
% in the configuration file \xfile{accsupp.cfg}.
%
% \subsubsection{Option \xoption{pdftex}}\label{sec:pdftex}
% Package option \xoption{pdftex} is used for \hologo{pdfTeX} in PDF mode.
% Additionally \cs{BeginAccSupp} and \cs{EndAccSupp} understand
% option \xoption{pdfliteral}. It controls
% the modifier keyword for \cs{pdfliteral}:
% \begin{quote}
%   |pdfliteral=direct| $\Rightarrow$ |\pdfliteral direct{...}|
% \end{quote}
%
% \subsubsection{Option \xoption{luatex}}\label{sec:luatex}
% Package option \xoption{luatex} is used for \hologo{LuaTeX} in PDF mode.
% Additionally \cs{BeginAccSupp} and \cs{EndAccSupp} understand
% option \xoption{pdfliteral}. It controls
% the modifier keyword for \cs{pdfextension} literal:
% \begin{quote}
%   |pdfliteral=direct| $\Rightarrow$ |\pdfextension literal direct{...}|
% \end{quote}
%
% \subsubsection{Option \xoption{dvipdfm}}
%
% Package option \xoption{dvipdfm} and its aliases \xoption{dvipdfmx}
% \xoption{xetex} are used for drivers that support
% \textsf{dvipdfm} specials.
%
% \subsubsection{Option \xoption{dvips}}
%
% Package option \xoption{dvips} and its alias \xoption{dvipsone}
% write pdfmark specials in the output. Unhappily these pdfmark
% operators are ignored by \textsf{ghostscript} (latest tested
% version is 8.54). Perhaps they are recognized by commercial
% distiller applications.
%
% \subsubsection{Turning off page stream compression}
%
% For debugging it is useful to have uncompressed page stream objects.
% This can be done afterwards via \textsf{pdftk}:
% \begin{quote}
%   |pdftk file.pdf output file-uncompressed.pdf uncompress|
% \end{quote}
% Or the PDF file is generated uncompressed:
% \begin{description}
% \item[\hologo{pdfTeX}:] |\pdfcompresslevel=0|
% \item[dvipdfm:] |dvipdfm -z0 ...|
% \item[dvipdfmx:] |dvipdfmx -z0 ...|
% \item[ghostscript/ps2pdf:]
%      |ps2pdf -dCompressPages=false input.ps output.pdf|
% \end{description}
%
% \section{Example}
% \label{sec:example}
%
% \subsection{Example \cs{notparallel}}
% \label{sec:example:notparallel}
%
%    \begin{macrocode}
%<*example1>
%    \end{macrocode}
%    \begin{macrocode}
%<<END
\documentclass{article}
\usepackage{accsupp}[2007/11/14]
\usepackage{centernot}
% U+2226 NOT PARALLEL
% \mathrel{...} prevents page break in between
\newcommand*{\notparallel}{%
  \ensuremath{%
    \mathrel{%
      \BeginAccSupp{method=hex,unicode,ActualText=2226}%
      \centernot{\parallel}%
      \EndAccSupp{}%
    }%
  }%
}
\begin{document}
\begin{equation}
A\notparallel B
\end{equation}
\end{document}
%END
%    \end{macrocode}
%    \begin{macrocode}
%</example1>
%    \end{macrocode}
%
% \subsection{Example with \xoption{pdfstringdef}}
%
%    \begin{macrocode}
%<*example2>
%    \end{macrocode}
%    \begin{macrocode}
%<<END
\documentclass{article}
\usepackage[unicode]{hyperref}
\usepackage{accsupp}[2007/11/14]
\begin{document}
  \begin{equation}
    \BeginAccSupp{
      method=pdfstringdef,
      unicode,
      ActualText={%
        a\texttwosuperior +b\texttwosuperior
        =c\texttwosuperior
      }
    }
    a^2 + b^2 = c^2
    \EndAccSupp{}
  \end{equation}
\end{document}
%END
%    \end{macrocode}
%    \begin{macrocode}
%</example2>
%    \end{macrocode}
%
% \StopEventually{
% }
%
% \section{Implementation}
%
% \subsection{Package}
%
%    \begin{macrocode}
%<*package>
%    \end{macrocode}
%
%    \begin{macrocode}
\begingroup\catcode61\catcode48\catcode32=10\relax%
  \catcode13=5 % ^^M
  \endlinechar=13 %
  \catcode123=1 % {
  \catcode125=2 % }
  \catcode64=11 % @
  \def\x{\endgroup
    \expandafter\edef\csname ACCSUPP@AtEnd\endcsname{%
      \endlinechar=\the\endlinechar\relax
      \catcode13=\the\catcode13\relax
      \catcode32=\the\catcode32\relax
      \catcode35=\the\catcode35\relax
      \catcode61=\the\catcode61\relax
      \catcode64=\the\catcode64\relax
      \catcode123=\the\catcode123\relax
      \catcode125=\the\catcode125\relax
    }%
  }%
\x\catcode61\catcode48\catcode32=10\relax%
\catcode13=5 % ^^M
\endlinechar=13 %
\catcode35=6 % #
\catcode64=11 % @
\catcode123=1 % {
\catcode125=2 % }
\def\TMP@EnsureCode#1#2{%
  \edef\ACCSUPP@AtEnd{%
    \ACCSUPP@AtEnd
    \catcode#1=\the\catcode#1\relax
  }%
  \catcode#1=#2\relax
}
\TMP@EnsureCode{10}{12}% ^^J
\TMP@EnsureCode{33}{12}% !
\TMP@EnsureCode{39}{12}% '
\TMP@EnsureCode{40}{12}% (
\TMP@EnsureCode{41}{12}% )
\TMP@EnsureCode{42}{12}% *
\TMP@EnsureCode{44}{12}% ,
\TMP@EnsureCode{45}{12}% -
\TMP@EnsureCode{46}{12}% .
\TMP@EnsureCode{47}{12}% /
\TMP@EnsureCode{58}{12}% :
\TMP@EnsureCode{60}{12}% <
\TMP@EnsureCode{62}{12}% >
\TMP@EnsureCode{94}{7}% ^ (superscript)
\TMP@EnsureCode{96}{12}% `
\TMP@EnsureCode{254}{12}% ^^fe
\TMP@EnsureCode{255}{12}% ^^ff
\edef\ACCSUPP@AtEnd{\ACCSUPP@AtEnd\noexpand\endinput}
%    \end{macrocode}
%
%    Package identification.
%    \begin{macrocode}
\NeedsTeXFormat{LaTeX2e}
\ProvidesPackage{accsupp}%
  [2018/01/30 v0.5 Accessibility support by marked content (HO)]
%    \end{macrocode}
%
%    \begin{macrocode}
\RequirePackage{pdfescape}[2007/02/25]
\RequirePackage{ifpdf}
\RequirePackage{ifxetex}
\RequirePackage{kvoptions}
%    \end{macrocode}
%    \begin{macrocode}
\SetupKeyvalOptions{%
  family=ACCSUPP,%
  prefix=ACCSUPP@%
}
%    \end{macrocode}
%
% \subsection{Space setup}
%
%    \begin{macrocode}
\DeclareBoolOption{space}
\DeclareStringOption[phvr8r]{spacefont}
\DeclareStringOption[32]{spacechar}
%    \end{macrocode}
%
% \subsection{Driver detection and setup}
%
% \paragraph{Driver declarations.}
%
%    \begin{macrocode}
\def\ACCSUPP@DefineDriverKey{%
  \@dblarg\ACCSUPP@@DefineDriverKey
}
\def\ACCSUPP@@DefineDriverKey[#1]#2{%
  \define@key{ACCSUPP}{#2}[]{%
    \def\ACCSUPP@driver{#1}%
  }%
  \g@addto@macro\ACCSUPP@DisableOptions{%
    \DisableKeyvalOption{ACCSUPP}{#2}%
  }%
}
\let\ACCSUPP@DisableOptions\@empty
\ACCSUPP@DefineDriverKey{pdftex}
\ACCSUPP@DefineDriverKey{luatex}
\ACCSUPP@DefineDriverKey{dvips}
\ACCSUPP@DefineDriverKey[dvips]{dvipsone}
\ACCSUPP@DefineDriverKey{dvipdfm}
\ACCSUPP@DefineDriverKey[dvipdfm]{dvipdfmx}
\ACCSUPP@DefineDriverKey[dvipdfm]{xetex}
\let\ACCSUPP@driver\relax
\InputIfFileExists{accsupp.cfg}{}{}
\providecommand*{\ActualTextDriverDefault}{dvips}
\ifpdf
  \ifx\pdfextension\@undefined
    \def\ACCSUPP@driver{pdftex}%
  \else
    \def\ACCSUPP@driver{luatex}%
  \fi
\else
  \ifxetex
    \def\ACCSUPP@driver{dvipdfm}%
  \else
    \ifx\ACCSUPP@driver\relax
      \let\ACCSUPP@driver\ActualTextDriverDefault
    \fi
  \fi
\fi
%    \end{macrocode}
%
% \paragraph{Process options.}
%
%    \begin{macrocode}
\ProcessKeyvalOptions*
\ACCSUPP@DisableOptions
%    \end{macrocode}
%
% \paragraph{Driver validation and loading.}
%
%    \begin{macrocode}
\def\ACCSUPP@temp{pdftex}%
\ifpdf
  \ifx\pdfextension\@undefined\else
    \def\ACCSUPP@temp{luatex}%
  \fi
  \ifx\ACCSUPP@temp\ACCSUPP@driver
  \else
    \PackageWarningNoLine{accsupp}{%
      Wrong driver `\ACCSUPP@driver', using `pdftex' instead%
    }%
ed as
    \let\ACCSUPP@driver\ACCSUPP@temp
  \fi
\else
  \ifx\ACCSUPP@temp\ACCSUPP@driver
    \PackageError{accsupp}{%
      Wrong driver, pdfTeX is not running in PDF mode.\MessageBreak
      Package loading is aborted%
    }\@ehc
    \expandafter\expandafter\expandafter\ACCSUPP@AtEnd
  \fi
  \def\ACCSUPP@temp{dvipdfm}%
  \ifxetex
    \ifx\ACCSUPP@temp\ACCSUPP@driver
    \else
      \PackageWarningNoLine{accsupp}{%
        Wrong driver `\ACCSUPP@driver',\MessageBreak
        using `dvipdfm' for XeTeX instead%
      }%
      \let\ACCSUPP@driver\ACCSUPP@temp
    \fi
  \fi
\fi%
\ifx\ACCSUPP@driver\relax
  \PackageError{accsupp}{%
    Missing driver option.\MessageBreak
    Package loading is aborted%
  }\@ehc
  \expandafter\ACCSUPP@AtEnd
\fi%
\InputIfFileExists{accsupp-\ACCSUPP@driver.def}{}{%
  \PackageError{accsupp}{%
    Unsupported driver `\ACCSUPP@driver'.\MessageBreak
    Package loading is aborted%
  }\@ehc
  \ACCSUPP@AtEnd
}%
%    \end{macrocode}
%
% \subsection{Main macro}
%
%    \begin{macrocode}
\DeclareBoolOption{unicode}
\DeclareStringOption[page]{pdfliteral}
\DeclareStringOption{Lang}
\def\ACCSUPP@method{escape}
\define@key{ACCSUPP}{method}{%
  \@ifundefined{ACCSUPP@method@#1}{%
    \PackageError{accsupp}{%
      Ignoring unknown method `#1'%
    }\@ehc
  }{%
    \edef\ACCSUPP@method{#1}%
  }%
}
\let\ACCSUPP@Lang\relax
\def\ACCSUPP@temp#1{%
  \expandafter\ACCSUPP@@temp\csname ACCSUPP@#1\endcsname{#1}%
}
\def\ACCSUPP@@temp#1#2{%
  \let#1\relax
  \define@key{ACCSUPP}{#2}{%
    \def#1{##1}%
    \ifx#1\@empty
      \def#1{()}%
    \else
      \csname ACCSUPP@method@\ACCSUPP@method\endcsname#1%
    \fi
  }%
}
\ACCSUPP@temp{Alt}
\ACCSUPP@temp{ActualText}
\ACCSUPP@temp{E}
%    \end{macrocode}
%
%    \begin{macrocode}
\newcommand*{\BeginAccSupp}[1]{%
  \begingroup
    \setkeys{ACCSUPP}{#1}%
    \edef\ACCSUPP@span{%
      /Span<<%
        \ifx\ACCSUPP@Lang\relax
        \else
          /Lang\ACCSUPP@Lang
        \fi
        \ifx\ACCSUPP@Alt\relax
        \else
          /Alt\ACCSUPP@Alt
        \fi
        \ifx\ACCSUPP@ActualText\relax
        \else
          /ActualText\ACCSUPP@ActualText
        \fi
        \ifx\ACCSUPP@E\relax
        \else
          /E\ACCSUPP@E
        \fi
      >>%
    }%
    \ACCSUPP@bdc
    \ACCSUPP@space
  \endgroup
}
\newcommand*{\EndAccSupp}[1]{%
  \begingroup
    \setkeys{ACCSUPP}{#1}%
    \ACCSUPP@emc
  \endgroup
}
\def\ACCSUPP@space{%
  \ifACCSUPP@space
    \begingroup
      \@ifundefined{ACCSUPP@Font}{%
        \global\font\ACCSUPP@Font=\ACCSUPP@spacefont\relax
      }{}%
      \leavevmode
      \setbox\z@\hbox{\ACCSUPP@Font\char\ACCSUPP@spacechar}%
      \wd\z@\z@
      \ht\z@\z@
      \dp\z@\z@
      \copy\z@
    \endgroup
  \fi
}
\newcommand*{\AccSuppSetup}{%
  \setkeys{ACCSUPP}%
}
%    \end{macrocode}
%
% \subsubsection{Input methods}
%
%    \begin{macrocode}
\def\ACCSUPP@method@plain#1{%
  \csname @safe@activestrue\endcsname
  \edef#1{%
    (%
    \ifACCSUPP@unicode
      \string\376\string\377%
    \fi
    #1%
    )%
  }%
  \@onelevel@sanitize#1%
}
%    \end{macrocode}
%    \begin{macrocode}
\def\ACCSUPP@method@escape#1{%
  \EdefEscapeString#1{%
    \ifACCSUPP@unicode
      ^^fe^^ff%
    \fi
    #1%
  }%
  \edef#1{(#1)}%
}%
%    \end{macrocode}
%    \begin{macrocode}
\def\ACCSUPP@method@hex#1{%
  \edef#1{%
    <%
    \ifACCSUPP@unicode
      FEFF%
    \fi
    #1%
    >%
  }%
}
%    \end{macrocode}
%    \begin{macrocode}
\def\ACCSUPP@method@pdfstringdef#1{%
  \ifACCSUPP@unicode
    \@ifundefined{hypersetup}{}{%
      \hypersetup{unicode}%
    }%
  \fi
  \@ifundefined{pdfstringdef}{%
    \PackageError{accsupp}{%
      Method `pdfstringdef' requires package `hyperref'%
    }\@ehc
    \let\ACCSUPP@temp\@empty
  }{%
    \begingroup
      \setbox0=\hbox{%
        \pdfstringdef\ACCSUPP@temp#1%
        \global\let\ACCSUPP@temp\ACCSUPP@temp
      }%
    \endgroup
  }%
  \edef#1{(\ACCSUPP@temp)}%
}
%    \end{macrocode}
%
%    \begin{macrocode}
\ACCSUPP@AtEnd%
%</package>
%    \end{macrocode}
%
% \subsection{Drivers}
%
% \subsubsection{Driver \xoption{pdftex}}
%
%    \begin{macrocode}
%<*pdftex>
\NeedsTeXFormat{LaTeX2e}
\ProvidesFile{accsupp-pdftex.def}%
  [2018/01/30 v0.5 accsupp driver for pdfTeX (HO)]%
%    \end{macrocode}
%
%    \begin{macrocode}
\def\ACCSUPP@bdc{%
  \pdfliteral\ACCSUPP@pdfliteral{\ACCSUPP@span BDC}%
}
\def\ACCSUPP@emc{%
  \pdfliteral\ACCSUPP@pdfliteral{EMC}%
}
%    \end{macrocode}
%    \begin{macrocode}
%</pdftex>
%    \end{macrocode}
%
% \subsubsection{Driver \xoption{luatex}}
%
%    \begin{macrocode}
%<*luatex>
\NeedsTeXFormat{LaTeX2e}
\ProvidesFile{accsupp-luatex.def}%
  [2018/01/30 v0.5 accsupp driver for pdfTeX (HO)]%
%    \end{macrocode}
%
%    \begin{macrocode}
\protected\def\ACCSUPP@pdf@literal{\pdfextension literal}
\def\ACCSUPP@bdc{%
  \ACCSUPP@pdf@literal\ACCSUPP@pdfliteral{\ACCSUPP@span BDC}%
}
\def\ACCSUPP@emc{%
  \ACCSUPP@pdf@literal\ACCSUPP@pdfliteral{EMC}%
}
%    \end{macrocode}
%    \begin{macrocode}
%</luatex>
%    \end{macrocode}
%
% \subsubsection{Driver \xoption{dvipdfm}}
%
%    \begin{macrocode}
%<*dvipdfm>
\NeedsTeXFormat{LaTeX2e}
\ProvidesFile{accsupp-dvipdfm.def}%
  [2018/01/30 v0.5 accsupp driver for dvipdfm (HO)]%
%    \end{macrocode}
%
%    \begin{macrocode}
\def\ACCSUPP@bdc{%
  \special{pdf:content \ACCSUPP@span BDC}%
}
\def\ACCSUPP@emc{%
  \special{pdf:content EMC}%
}
%    \end{macrocode}
%    \begin{macrocode}
%</dvipdfm>
%    \end{macrocode}
%
% \subsubsection{Driver \xoption{dvips}}
%
%    \begin{macrocode}
%<*dvips>
\NeedsTeXFormat{LaTeX2e}
\ProvidesFile{accsupp-dvips.def}%
  [2018/01/30 v0.5 accsupp driver for dvips (HO)]%
%    \end{macrocode}
%
%    \begin{macrocode}
\def\ACCSUPP@bdc{%
  \special{ps:[\ACCSUPP@span/BDC pdfmark}%
}
\def\ACCSUPP@emc{%
  \special{ps:[/EMC pdfmark}%
}
%    \end{macrocode}
%    \begin{macrocode}
%</dvips>
%    \end{macrocode}
%
% \section{Test}
%
% \subsection{Catcode checks for loading}
%
%    \begin{macrocode}
%<*test1>
%    \end{macrocode}
%    \begin{macrocode}
\NeedsTeXFormat{LaTeX2e}
\documentclass{minimal}
\makeatletter
\def\RestoreCatcodes{}
\count@=0 %
\loop
  \edef\RestoreCatcodes{%
    \RestoreCatcodes
    \catcode\the\count@=\the\catcode\count@\relax
  }%
\ifnum\count@<255 %
  \advance\count@\@ne
\repeat

\def\RangeCatcodeInvalid#1#2{%
  \count@=#1\relax
  \loop
    \catcode\count@=15 %
  \ifnum\count@<#2\relax
    \advance\count@\@ne
  \repeat
}
\def\Test{%
  \RangeCatcodeInvalid{0}{47}%
  \RangeCatcodeInvalid{58}{64}%
  \RangeCatcodeInvalid{91}{96}%
  \RangeCatcodeInvalid{123}{127}%
  \catcode`\@=12 %
  \catcode`\\=0 %
  \catcode`\{=1 %
  \catcode`\}=2 %
  \catcode`\#=6 %
  \catcode`\[=12 %
  \catcode`\]=12 %
  \catcode`\%=14 %
  \catcode`\ =10 %
  \catcode13=5 %
  \RequirePackage{accsupp}[2018/01/30]\relax
  \RestoreCatcodes
}
\Test
\csname @@end\endcsname
\end
%    \end{macrocode}
%    \begin{macrocode}
%</test1>
%    \end{macrocode}
%
% \section{Installation}
%
% \subsection{Download}
%
% \paragraph{Package.} This package is available on
% CTAN\footnote{\url{http://ctan.org/pkg/accsupp}}:
% \begin{description}
% \item[\CTAN{macros/latex/contrib/oberdiek/accsupp.dtx}] The source file.
% \item[\CTAN{macros/latex/contrib/oberdiek/accsupp.pdf}] Documentation.
% \end{description}
%
%
% \paragraph{Bundle.} All the packages of the bundle `oberdiek'
% are also available in a TDS compliant ZIP archive. There
% the packages are already unpacked and the documentation files
% are generated. The files and directories obey the TDS standard.
% \begin{description}
% \item[\CTAN{install/macros/latex/contrib/oberdiek.tds.zip}]
% \end{description}
% \emph{TDS} refers to the standard ``A Directory Structure
% for \TeX\ Files'' (\CTAN{tds/tds.pdf}). Directories
% with \xfile{texmf} in their name are usually organized this way.
%
% \subsection{Bundle installation}
%
% \paragraph{Unpacking.} Unpack the \xfile{oberdiek.tds.zip} in the
% TDS tree (also known as \xfile{texmf} tree) of your choice.
% Example (linux):
% \begin{quote}
%   |unzip oberdiek.tds.zip -d ~/texmf|
% \end{quote}
%
% \paragraph{Script installation.}
% Check the directory \xfile{TDS:scripts/oberdiek/} for
% scripts that need further installation steps.
% Package \xpackage{attachfile2} comes with the Perl script
% \xfile{pdfatfi.pl} that should be installed in such a way
% that it can be called as \texttt{pdfatfi}.
% Example (linux):
% \begin{quote}
%   |chmod +x scripts/oberdiek/pdfatfi.pl|\\
%   |cp scripts/oberdiek/pdfatfi.pl /usr/local/bin/|
% \end{quote}
%
% \subsection{Package installation}
%
% \paragraph{Unpacking.} The \xfile{.dtx} file is a self-extracting
% \docstrip\ archive. The files are extracted by running the
% \xfile{.dtx} through \plainTeX:
% \begin{quote}
%   \verb|tex accsupp.dtx|
% \end{quote}
%
% \paragraph{TDS.} Now the different files must be moved into
% the different directories in your installation TDS tree
% (also known as \xfile{texmf} tree):
% \begin{quote}
% \def\t{^^A
% \begin{tabular}{@{}>{\ttfamily}l@{ $\rightarrow$ }>{\ttfamily}l@{}}
%   accsupp.sty & tex/latex/oberdiek/accsupp.sty\\
%   accsupp-pdftex.def & tex/latex/oberdiek/accsupp-pdftex.def\\
%   accsupp-luatex.def & tex/latex/oberdiek/accsupp-luatex.def\\
%   accsupp-dvipdfm.def & tex/latex/oberdiek/accsupp-dvipdfm.def\\
%   accsupp-dvips.def & tex/latex/oberdiek/accsupp-dvips.def\\
%   accsupp.pdf & doc/latex/oberdiek/accsupp.pdf\\
%   accsupp-example1.tex & doc/latex/oberdiek/accsupp-example1.tex\\
%   accsupp-example2.tex & doc/latex/oberdiek/accsupp-example2.tex\\
%   test/accsupp-test1.tex & doc/latex/oberdiek/test/accsupp-test1.tex\\
%   accsupp.dtx & source/latex/oberdiek/accsupp.dtx\\
% \end{tabular}^^A
% }^^A
% \sbox0{\t}^^A
% \ifdim\wd0>\linewidth
%   \begingroup
%     \advance\linewidth by\leftmargin
%     \advance\linewidth by\rightmargin
%   \edef\x{\endgroup
%     \def\noexpand\lw{\the\linewidth}^^A
%   }\x
%   \def\lwbox{^^A
%     \leavevmode
%     \hbox to \linewidth{^^A
%       \kern-\leftmargin\relax
%       \hss
%       \usebox0
%       \hss
%       \kern-\rightmargin\relax
%     }^^A
%   }^^A
%   \ifdim\wd0>\lw
%     \sbox0{\small\t}^^A
%     \ifdim\wd0>\linewidth
%       \ifdim\wd0>\lw
%         \sbox0{\footnotesize\t}^^A
%         \ifdim\wd0>\linewidth
%           \ifdim\wd0>\lw
%             \sbox0{\scriptsize\t}^^A
%             \ifdim\wd0>\linewidth
%               \ifdim\wd0>\lw
%                 \sbox0{\tiny\t}^^A
%                 \ifdim\wd0>\linewidth
%                   \lwbox
%                 \else
%                   \usebox0
%                 \fi
%               \else
%                 \lwbox
%               \fi
%             \else
%               \usebox0
%             \fi
%           \else
%             \lwbox
%           \fi
%         \else
%           \usebox0
%         \fi
%       \else
%         \lwbox
%       \fi
%     \else
%       \usebox0
%     \fi
%   \else
%     \lwbox
%   \fi
% \else
%   \usebox0
% \fi
% \end{quote}
% If you have a \xfile{docstrip.cfg} that configures and enables \docstrip's
% TDS installing feature, then some files can already be in the right
% place, see the documentation of \docstrip.
%
% \subsection{Refresh file name databases}
%
% If your \TeX~distribution
% (\teTeX, \mikTeX, \dots) relies on file name databases, you must refresh
% these. For example, \teTeX\ users run \verb|texhash| or
% \verb|mktexlsr|.
%
% \subsection{Some details for the interested}
%
% \paragraph{Attached source.}
%
% The PDF documentation on CTAN also includes the
% \xfile{.dtx} source file. It can be extracted by
% AcrobatReader 6 or higher. Another option is \textsf{pdftk},
% e.g. unpack the file into the current directory:
% \begin{quote}
%   \verb|pdftk accsupp.pdf unpack_files output .|
% \end{quote}
%
% \paragraph{Unpacking with \LaTeX.}
% The \xfile{.dtx} chooses its action depending on the format:
% \begin{description}
% \item[\plainTeX:] Run \docstrip\ and extract the files.
% \item[\LaTeX:] Generate the documentation.
% \end{description}
% If you insist on using \LaTeX\ for \docstrip\ (really,
% \docstrip\ does not need \LaTeX), then inform the autodetect routine
% about your intention:
% \begin{quote}
%   \verb|latex \let\install=y\input{accsupp.dtx}|
% \end{quote}
% Do not forget to quote the argument according to the demands
% of your shell.
%
% \paragraph{Generating the documentation.}
% You can use both the \xfile{.dtx} or the \xfile{.drv} to generate
% the documentation. The process can be configured by the
% configuration file \xfile{ltxdoc.cfg}. For instance, put this
% line into this file, if you want to have A4 as paper format:
% \begin{quote}
%   \verb|\PassOptionsToClass{a4paper}{article}|
% \end{quote}
% An example follows how to generate the
% documentation with pdf\LaTeX:
% \begin{quote}
%\begin{verbatim}
%pdflatex accsupp.dtx
%makeindex -s gind.ist accsupp.idx
%pdflatex accsupp.dtx
%makeindex -s gind.ist accsupp.idx
%pdflatex accsupp.dtx
%\end{verbatim}
% \end{quote}
%
% \section{Catalogue}
%
% The following XML file can be used as source for the
% \href{http://mirror.ctan.org/help/Catalogue/catalogue.html}{\TeX\ Catalogue}.
% The elements \texttt{caption} and \texttt{description} are imported
% from the original XML file from the Catalogue.
% The name of the XML file in the Catalogue is \xfile{accsupp.xml}.
%    \begin{macrocode}
%<*catalogue>
<?xml version='1.0' encoding='us-ascii'?>
<!DOCTYPE entry SYSTEM 'catalogue.dtd'>
<entry datestamp='$Date$' modifier='$Author$' id='accsupp'>
  <name>accsupp</name>
  <caption>Better accessibility support for PDF files.</caption>
  <authorref id='auth:oberdiek'/>
  <copyright owner='Heiko Oberdiek' year='2007,2010'/>
  <license type='lppl1.3'/>
  <version number='0.4'/>
  <description>
    Since PDF 1.5 portions of a page can be marked for better accessibility
    support. For example, replacement texts or expansions of abbreviations can
    be provided. This package starts with providing a minimal low-level
    interface for programmers; its status is experimental.
    <p/>
    The package is part of the <xref refid='oberdiek'>oberdiek</xref> bundle.
  </description>
  <documentation details='Package documentation'
      href='ctan:/macros/latex/contrib/oberdiek/accsupp.pdf'/>
  <ctan file='true' path='/macros/latex/contrib/oberdiek/accsupp.dtx'/>
  <miktex location='oberdiek'/>
  <texlive location='oberdiek'/>
  <install path='/macros/latex/contrib/oberdiek/oberdiek.tds.zip'/>
</entry>
%</catalogue>
%    \end{macrocode}
%
% \begin{thebibliography}{9}
% \bibitem{pdfspec}
% Adobe Systems Incorporated,
% \textit{PDF Reference}, 6th edition, 2006.
% \url{http://www.adobe.com/devnet/acrobat/pdfs/pdf_reference.pdf}
%
% \end{thebibliography}
%
% \begin{History}
%   \begin{Version}{2007/03/21 v0.1}
%   \item
%     First version.
%   \end{Version}
%   \begin{Version}{2007/11/14 v0.2}
%   \item
%     Various bug fixes.
%   \item
%     Catcode section rewritten, test added.
%   \end{Version}
%   \begin{Version}{2010/01/16 v0.3}
%   \item
%     \cs{AccSuppSetup} added.
%   \item
%     Options \xoption{space}, \xoption{spacefont}, \xoption{spacechar} added.
%   \end{Version}
%   \begin{Version}{2016/05/16 v0.4}
%   \item
%     Documentation updates.
%   \end{Version}
%   \begin{Version}{2018/01/30 v0.5}
%   \item
%     LuaTeX support added.
%   \end{Version}
% \end{History}
%
% \PrintIndex
%
% \Finale
\endinput
|
% \end{quote}
% Do not forget to quote the argument according to the demands
% of your shell.
%
% \paragraph{Generating the documentation.}
% You can use both the \xfile{.dtx} or the \xfile{.drv} to generate
% the documentation. The process can be configured by the
% configuration file \xfile{ltxdoc.cfg}. For instance, put this
% line into this file, if you want to have A4 as paper format:
% \begin{quote}
%   \verb|\PassOptionsToClass{a4paper}{article}|
% \end{quote}
% An example follows how to generate the
% documentation with pdf\LaTeX:
% \begin{quote}
%\begin{verbatim}
%pdflatex accsupp.dtx
%makeindex -s gind.ist accsupp.idx
%pdflatex accsupp.dtx
%makeindex -s gind.ist accsupp.idx
%pdflatex accsupp.dtx
%\end{verbatim}
% \end{quote}
%
% \section{Catalogue}
%
% The following XML file can be used as source for the
% \href{http://mirror.ctan.org/help/Catalogue/catalogue.html}{\TeX\ Catalogue}.
% The elements \texttt{caption} and \texttt{description} are imported
% from the original XML file from the Catalogue.
% The name of the XML file in the Catalogue is \xfile{accsupp.xml}.
%    \begin{macrocode}
%<*catalogue>
<?xml version='1.0' encoding='us-ascii'?>
<!DOCTYPE entry SYSTEM 'catalogue.dtd'>
<entry datestamp='$Date$' modifier='$Author$' id='accsupp'>
  <name>accsupp</name>
  <caption>Better accessibility support for PDF files.</caption>
  <authorref id='auth:oberdiek'/>
  <copyright owner='Heiko Oberdiek' year='2007,2010'/>
  <license type='lppl1.3'/>
  <version number='0.4'/>
  <description>
    Since PDF 1.5 portions of a page can be marked for better accessibility
    support. For example, replacement texts or expansions of abbreviations can
    be provided. This package starts with providing a minimal low-level
    interface for programmers; its status is experimental.
    <p/>
    The package is part of the <xref refid='oberdiek'>oberdiek</xref> bundle.
  </description>
  <documentation details='Package documentation'
      href='ctan:/macros/latex/contrib/oberdiek/accsupp.pdf'/>
  <ctan file='true' path='/macros/latex/contrib/oberdiek/accsupp.dtx'/>
  <miktex location='oberdiek'/>
  <texlive location='oberdiek'/>
  <install path='/macros/latex/contrib/oberdiek/oberdiek.tds.zip'/>
</entry>
%</catalogue>
%    \end{macrocode}
%
% \begin{thebibliography}{9}
% \bibitem{pdfspec}
% Adobe Systems Incorporated,
% \textit{PDF Reference}, 6th edition, 2006.
% \url{http://www.adobe.com/devnet/acrobat/pdfs/pdf_reference.pdf}
%
% \end{thebibliography}
%
% \begin{History}
%   \begin{Version}{2007/03/21 v0.1}
%   \item
%     First version.
%   \end{Version}
%   \begin{Version}{2007/11/14 v0.2}
%   \item
%     Various bug fixes.
%   \item
%     Catcode section rewritten, test added.
%   \end{Version}
%   \begin{Version}{2010/01/16 v0.3}
%   \item
%     \cs{AccSuppSetup} added.
%   \item
%     Options \xoption{space}, \xoption{spacefont}, \xoption{spacechar} added.
%   \end{Version}
%   \begin{Version}{2016/05/16 v0.4}
%   \item
%     Documentation updates.
%   \end{Version}
%   \begin{Version}{2018/01/30 v0.5}
%   \item
%     LuaTeX support added.
%   \end{Version}
% \end{History}
%
% \PrintIndex
%
% \Finale
\endinput
|
% \end{quote}
% Do not forget to quote the argument according to the demands
% of your shell.
%
% \paragraph{Generating the documentation.}
% You can use both the \xfile{.dtx} or the \xfile{.drv} to generate
% the documentation. The process can be configured by the
% configuration file \xfile{ltxdoc.cfg}. For instance, put this
% line into this file, if you want to have A4 as paper format:
% \begin{quote}
%   \verb|\PassOptionsToClass{a4paper}{article}|
% \end{quote}
% An example follows how to generate the
% documentation with pdf\LaTeX:
% \begin{quote}
%\begin{verbatim}
%pdflatex accsupp.dtx
%makeindex -s gind.ist accsupp.idx
%pdflatex accsupp.dtx
%makeindex -s gind.ist accsupp.idx
%pdflatex accsupp.dtx
%\end{verbatim}
% \end{quote}
%
% \section{Catalogue}
%
% The following XML file can be used as source for the
% \href{http://mirror.ctan.org/help/Catalogue/catalogue.html}{\TeX\ Catalogue}.
% The elements \texttt{caption} and \texttt{description} are imported
% from the original XML file from the Catalogue.
% The name of the XML file in the Catalogue is \xfile{accsupp.xml}.
%    \begin{macrocode}
%<*catalogue>
<?xml version='1.0' encoding='us-ascii'?>
<!DOCTYPE entry SYSTEM 'catalogue.dtd'>
<entry datestamp='$Date$' modifier='$Author$' id='accsupp'>
  <name>accsupp</name>
  <caption>Better accessibility support for PDF files.</caption>
  <authorref id='auth:oberdiek'/>
  <copyright owner='Heiko Oberdiek' year='2007,2010'/>
  <license type='lppl1.3'/>
  <version number='0.4'/>
  <description>
    Since PDF 1.5 portions of a page can be marked for better accessibility
    support. For example, replacement texts or expansions of abbreviations can
    be provided. This package starts with providing a minimal low-level
    interface for programmers; its status is experimental.
    <p/>
    The package is part of the <xref refid='oberdiek'>oberdiek</xref> bundle.
  </description>
  <documentation details='Package documentation'
      href='ctan:/macros/latex/contrib/oberdiek/accsupp.pdf'/>
  <ctan file='true' path='/macros/latex/contrib/oberdiek/accsupp.dtx'/>
  <miktex location='oberdiek'/>
  <texlive location='oberdiek'/>
  <install path='/macros/latex/contrib/oberdiek/oberdiek.tds.zip'/>
</entry>
%</catalogue>
%    \end{macrocode}
%
% \begin{thebibliography}{9}
% \bibitem{pdfspec}
% Adobe Systems Incorporated,
% \textit{PDF Reference}, 6th edition, 2006.
% \url{http://www.adobe.com/devnet/acrobat/pdfs/pdf_reference.pdf}
%
% \end{thebibliography}
%
% \begin{History}
%   \begin{Version}{2007/03/21 v0.1}
%   \item
%     First version.
%   \end{Version}
%   \begin{Version}{2007/11/14 v0.2}
%   \item
%     Various bug fixes.
%   \item
%     Catcode section rewritten, test added.
%   \end{Version}
%   \begin{Version}{2010/01/16 v0.3}
%   \item
%     \cs{AccSuppSetup} added.
%   \item
%     Options \xoption{space}, \xoption{spacefont}, \xoption{spacechar} added.
%   \end{Version}
%   \begin{Version}{2016/05/16 v0.4}
%   \item
%     Documentation updates.
%   \end{Version}
%   \begin{Version}{2018/01/30 v0.5}
%   \item
%     LuaTeX support added.
%   \end{Version}
% \end{History}
%
% \PrintIndex
%
% \Finale
\endinput
|
% \end{quote}
% Do not forget to quote the argument according to the demands
% of your shell.
%
% \paragraph{Generating the documentation.}
% You can use both the \xfile{.dtx} or the \xfile{.drv} to generate
% the documentation. The process can be configured by the
% configuration file \xfile{ltxdoc.cfg}. For instance, put this
% line into this file, if you want to have A4 as paper format:
% \begin{quote}
%   \verb|\PassOptionsToClass{a4paper}{article}|
% \end{quote}
% An example follows how to generate the
% documentation with pdf\LaTeX:
% \begin{quote}
%\begin{verbatim}
%pdflatex accsupp.dtx
%makeindex -s gind.ist accsupp.idx
%pdflatex accsupp.dtx
%makeindex -s gind.ist accsupp.idx
%pdflatex accsupp.dtx
%\end{verbatim}
% \end{quote}
%
% \section{Catalogue}
%
% The following XML file can be used as source for the
% \href{http://mirror.ctan.org/help/Catalogue/catalogue.html}{\TeX\ Catalogue}.
% The elements \texttt{caption} and \texttt{description} are imported
% from the original XML file from the Catalogue.
% The name of the XML file in the Catalogue is \xfile{accsupp.xml}.
%    \begin{macrocode}
%<*catalogue>
<?xml version='1.0' encoding='us-ascii'?>
<!DOCTYPE entry SYSTEM 'catalogue.dtd'>
<entry datestamp='$Date$' modifier='$Author$' id='accsupp'>
  <name>accsupp</name>
  <caption>Better accessibility support for PDF files.</caption>
  <authorref id='auth:oberdiek'/>
  <copyright owner='Heiko Oberdiek' year='2007,2010'/>
  <license type='lppl1.3'/>
  <version number='0.4'/>
  <description>
    Since PDF 1.5 portions of a page can be marked for better accessibility
    support. For example, replacement texts or expansions of abbreviations can
    be provided. This package starts with providing a minimal low-level
    interface for programmers; its status is experimental.
    <p/>
    The package is part of the <xref refid='oberdiek'>oberdiek</xref> bundle.
  </description>
  <documentation details='Package documentation'
      href='ctan:/macros/latex/contrib/oberdiek/accsupp.pdf'/>
  <ctan file='true' path='/macros/latex/contrib/oberdiek/accsupp.dtx'/>
  <miktex location='oberdiek'/>
  <texlive location='oberdiek'/>
  <install path='/macros/latex/contrib/oberdiek/oberdiek.tds.zip'/>
</entry>
%</catalogue>
%    \end{macrocode}
%
% \begin{thebibliography}{9}
% \bibitem{pdfspec}
% Adobe Systems Incorporated,
% \textit{PDF Reference}, 6th edition, 2006.
% \url{http://www.adobe.com/devnet/acrobat/pdfs/pdf_reference.pdf}
%
% \end{thebibliography}
%
% \begin{History}
%   \begin{Version}{2007/03/21 v0.1}
%   \item
%     First version.
%   \end{Version}
%   \begin{Version}{2007/11/14 v0.2}
%   \item
%     Various bug fixes.
%   \item
%     Catcode section rewritten, test added.
%   \end{Version}
%   \begin{Version}{2010/01/16 v0.3}
%   \item
%     \cs{AccSuppSetup} added.
%   \item
%     Options \xoption{space}, \xoption{spacefont}, \xoption{spacechar} added.
%   \end{Version}
%   \begin{Version}{2016/05/16 v0.4}
%   \item
%     Documentation updates.
%   \end{Version}
% \end{History}
%
% \PrintIndex
%
% \Finale
\endinput
