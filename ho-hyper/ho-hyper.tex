\NeedsTeXFormat{LaTeX2e}

\newcommand*{\Title}{Overview}
\newcommand*{\CTANdir}{macros/latex/contrib/ho-hyper/}
\newcommand*{\CTANroot}{ftp://ftp.ctan.org/tex-archive/}
\newcommand*{\Subject}{CTAN:\CTANdir}
\newcommand*{\Author}{Heiko Oberdiek}
\newcommand*{\Email}{ho-tex at tug.org}
\newcommand*{\Date}{2018/11/30}

% Copyright (C) 2006-2018 by
%    Heiko Oberdiek <heiko.oberdiek at googlemail.com>
%
% This work may be distributed and/or modified under the
% conditions of the LaTeX Project Public License, either
% version 1.3 of this license or (at your option) any later
% version. The latest version of this license is in
%    http://www.latex-project.org/lppl.txt
% and version 1.3 or later is part of all distributions of
% LaTeX version 2003/12/01 or later.
%
% This work has the LPPL maintenance status "maintained".
%
% This Current Maintainer of this work is Heiko Oberdiek.
%
% This work consists of the overview "oberdiek.pdf", its source
% "oberdiek.tex", and the installation script "oberdiek.ins"
% for the projects in CTAN:macros/latex/contrib/oberdiek/.
%
\documentclass[a4paper,12pt]{article}

\usepackage{ifluatex}
\ifluatex
  \usepackage{fontspec}[2011/09/18]%
  \usepackage{unicode-math}[2011/09/19]%
  \setmathfont{latinmodern-math.otf}%
\fi

\usepackage[
  ignorehead,
  top=1in,
]{geometry}
\usepackage{longtable}
\usepackage{color}
\usepackage[ngerman,english]{babel}
\usepackage{hologo}
\usepackage{biblatex}% for internals in .toc files

\definecolor{link}{rgb}{1,0,0}% red
\definecolor{file}{rgb}{0,0,1}% blue
\definecolor{url}{cmyk}{0.1,1,0,0.1}

\definecolor{file}{rgb}{1,0,0}% red
\definecolor{url}{rgb}{0,0,1}% blue
\definecolor{link}{rgb}{0,0.75,0}%

\usepackage[
  colorlinks,
]{hyperref}[2006/02/12]
\hypersetup{
  pdftitle={CTAN:\CTANdir},
  pdfsubject={Package Overview},
  pdfauthor={\Author\ <\Email>},
  bookmarksnumbered,
  bookmarksopen,
  bookmarksopenlevel=2,
  bookmarksdepth=2,
  filecolor=file,
  urlcolor=url,
  linkcolor=link,
}
\usepackage{bookmark}
\usepackage{hypdestopt}
\setcounter{tocdepth}{1}
\setcounter{secnumdepth}{1}

\title{%
  \href{\CTANroot\CTANdir}{CTAN:\CTANdir}%
}
\author{%
  \Author\\
  \textless
  \href{mailto:\Email}{\texttt{\Email}}%
  \textgreater
}
\date{\Date}

\providecommand*{\pdfTeX}{pdf\TeX}
\providecommand*{\plainTeX}{\mbox{plain-\TeX}}
\providecommand*{\iniTeX}{\mbox{ini-\TeX}}
\providecommand*{\VTeX}{V\TeX}
\providecommand*{\eTeX}{$\csname m@th\endcsname\varepsilon$-\TeX}
\providecommand*{\LuaTeX}{%
  L\textsc{ua}\TeX
}
\newcommand*{\xpackage}[1]{\textsf{#1}}
\newcommand*{\xmodule}[1]{\textsf{#1}}
\newcommand*{\xfile}[1]{\texttt{#1}}
\newcommand*{\xext}[1]{\texttt{.#1}}
\newcommand*{\xoption}[1]{\textsf{#1}}
\newcommand*{\cs}[1]{\texttt{\textbackslash#1}}
\newcommand*{\meta}[1]{%
  \ensuremath\langle
  \textit{#1}%
  \ensuremath\rangle
}

\makeatletter
\g@addto@macro\abstract{\noindent\ignorespaces}

\newcommand*{\tocinclude}[1]{%
  \setcounter{tocdepth}{3}%
  \begingroup
    \makeatletter
    \def\@prj{#1}%
    \let\contentsline\foreign@contentsline
    \input{\@prj.toc}%
  \endgroup
}
\def\foreign@contentsline#1#2#3#4{%
  \ifx\\#4\\%
    \csname l@#1\endcsname{#2}{#3}%
  \else
    \ifHy@linktocpage
      \csname l@#1\endcsname{{#2}}{%
        \hyper@linkfile{#3}{\@prj.pdf}{#4}%
      }%
    \else
      \csname l@#1\endcsname{%
        \hyper@linkfile{#2}{\@prj.pdf}{#4}%
      }{#3}%
    \fi
  \fi
}%

\newenvironment{overview}{%
  \setlength{\tabcolsep}{0.8\tabcolsep}%
  \setlength{\LTleft}{0pt}%
  \longtable{@{}llll@{}}
}{%
  \endlongtable
}
\newcommand*{\entry}[4]{%
  \href{file:#1.pdf}{%
    \bfseries\xpackage{#1}%
  }%
  & #2%
  & v#3%
  & \href{\CTANroot\CTANdir #1.pdf}{[pdf]} %
    \href{\CTANroot\CTANdir #1.dtx}{[dtx]}
  \\*%
  \hyperref[{#1}]{\small (contents)}%
  &
  \multicolumn{2}{l}{%
    #4%
  }%
  \\%
}
\newcommand*{\entrysep}{1.5ex}

\newcommand*{\pkgsectformat}[1]{%
  \texorpdfstring{%
    \textcolor{link}{The} %
    \xpackage{#1} %
    \textcolor{link}{package}%
  }{#1}%
}

\makeatother

\begin{document}
\maketitle

\section{Overview}
\begin{overview}
\entry{hycolor}{2011/01/30}{1.7}{Color options for hyperref/bookmark}%
[\entrysep]
\entry{hypbmsec}{2007/04/11}{2.4}{Bookmarks in sectioning commands}%
[\entrysep]
\entry{hypcap}{2011/02/16}{1.11}{Adjusting the anchors of captions}%
[\entrysep]
\entry{hypdestopt}{2011/05/13}{2.3}{Hyperref destination optimizer}%
[\entrysep]
\entry{hypdoc}{2011/08/19}{1.11}{Hyper extensions for doc.sty}%
[\entrysep]
\entry{hypgotoe}{2007/10/30}{0.1}{Links to embedded files}%
\end{overview}

\section{Packages}
\hypersetup{bookmarksnumbered=false}
\subsection{\pkgsectformat{hycolor}}
\label{hycolor}
\begin{abstract}
Package \xpackage{hycolor} implements the color option stuff that
is used by packages \xpackage{hyperref} and \xpackage{bookmark}.
It is not intended as package for the user.
\end{abstract}
\tocinclude{hycolor}

\newpage
\subsection{\pkgsectformat{hypbmsec}}
\label{hypbmsec}
\begin{abstract}
This package expands the syntax of the sectioning commands. If the
argument of the sectioning commands isn't usable as outline entry,
a replacement for the bookmarks can be given.
\end{abstract}
\tocinclude{hypbmsec}

\newpage
\subsection{\pkgsectformat{hypcap}}
\label{hypcap}
\begin{abstract}
This package tries a solution of the problem with
hyperref, that links to floats points below the
caption and not at the beginning of the float.
Therefore this package divides the task into two
part, the link setting with \cs{capstart} or
automatically at the beginning of a float and
the rest in the \cs{caption} command.
\end{abstract}
\tocinclude{hypcap}

\newpage
\subsection{\pkgsectformat{hypdestopt}}
\label{hypdestopt}
\begin{abstract}
Package \xpackage{hypdestopt} supports \xpackage{hyperref}'s
\xoption{pdftex} driver. It removes unnecessary destinations
and shortens the destination names or uses numbered destinations
to get smaller PDF files.
\end{abstract}
\tocinclude{hypdestopt}

\newpage
\subsection{\pkgsectformat{hypdoc}}
\label{hypdoc}
\begin{abstract}
This package adds hyper features to the package
\xpackage{doc} that is used in the documentation
system of \LaTeXe. Bookmarks are added and references
are linked as far as possible.
\end{abstract}
\tocinclude{hypdoc}

\newpage
\subsection{\pkgsectformat{hypgotoe}}
\label{hypgotoe}
\begin{abstract}
Experimental package for links to embedded files.
\end{abstract}
\tocinclude{hypgotoe}

\end{document}
