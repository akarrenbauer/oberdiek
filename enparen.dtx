% \iffalse meta-comment
%
% File: enparen.dtx
% Version: 2012/01/07 v1.0
% Info: Parentheses nesting
%
% Copyright (C) 2012 by
%    Heiko Oberdiek <heiko.oberdiek at googlemail.com>
%
% This work may be distributed and/or modified under the
% conditions of the LaTeX Project Public License, either
% version 1.3c of this license or (at your option) any later
% version. This version of this license is in
%    http://www.latex-project.org/lppl/lppl-1-3c.txt
% and the latest version of this license is in
%    http://www.latex-project.org/lppl.txt
% and version 1.3 or later is part of all distributions of
% LaTeX version 2005/12/01 or later.
%
% This work has the LPPL maintenance status "maintained".
%
% This Current Maintainer of this work is Heiko Oberdiek.
%
% This work consists of the main source file enparen.dtx
% and the derived files
%    enparen.sty, enparen.pdf, enparen.ins, enparen.drv.
%
% Distribution:
%    CTAN:macros/latex/contrib/oberdiek/enparen.dtx
%    CTAN:macros/latex/contrib/oberdiek/enparen.pdf
%
% Unpacking:
%    (a) If enparen.ins is present:
%           tex enparen.ins
%    (b) Without enparen.ins:
%           tex enparen.dtx
%    (c) If you insist on using LaTeX
%           latex \let\install=y% \iffalse meta-comment
%
% File: enparen.dtx
% Version: 2012/01/07 v1.0
% Info: Parentheses nesting
%
% Copyright (C) 2012 by
%    Heiko Oberdiek <heiko.oberdiek at googlemail.com>
%
% This work may be distributed and/or modified under the
% conditions of the LaTeX Project Public License, either
% version 1.3c of this license or (at your option) any later
% version. This version of this license is in
%    http://www.latex-project.org/lppl/lppl-1-3c.txt
% and the latest version of this license is in
%    http://www.latex-project.org/lppl.txt
% and version 1.3 or later is part of all distributions of
% LaTeX version 2005/12/01 or later.
%
% This work has the LPPL maintenance status "maintained".
%
% This Current Maintainer of this work is Heiko Oberdiek.
%
% This work consists of the main source file enparen.dtx
% and the derived files
%    enparen.sty, enparen.pdf, enparen.ins, enparen.drv.
%
% Distribution:
%    CTAN:macros/latex/contrib/oberdiek/enparen.dtx
%    CTAN:macros/latex/contrib/oberdiek/enparen.pdf
%
% Unpacking:
%    (a) If enparen.ins is present:
%           tex enparen.ins
%    (b) Without enparen.ins:
%           tex enparen.dtx
%    (c) If you insist on using LaTeX
%           latex \let\install=y% \iffalse meta-comment
%
% File: enparen.dtx
% Version: 2012/01/07 v1.0
% Info: Parentheses nesting
%
% Copyright (C) 2012 by
%    Heiko Oberdiek <heiko.oberdiek at googlemail.com>
%
% This work may be distributed and/or modified under the
% conditions of the LaTeX Project Public License, either
% version 1.3c of this license or (at your option) any later
% version. This version of this license is in
%    http://www.latex-project.org/lppl/lppl-1-3c.txt
% and the latest version of this license is in
%    http://www.latex-project.org/lppl.txt
% and version 1.3 or later is part of all distributions of
% LaTeX version 2005/12/01 or later.
%
% This work has the LPPL maintenance status "maintained".
%
% This Current Maintainer of this work is Heiko Oberdiek.
%
% This work consists of the main source file enparen.dtx
% and the derived files
%    enparen.sty, enparen.pdf, enparen.ins, enparen.drv.
%
% Distribution:
%    CTAN:macros/latex/contrib/oberdiek/enparen.dtx
%    CTAN:macros/latex/contrib/oberdiek/enparen.pdf
%
% Unpacking:
%    (a) If enparen.ins is present:
%           tex enparen.ins
%    (b) Without enparen.ins:
%           tex enparen.dtx
%    (c) If you insist on using LaTeX
%           latex \let\install=y% \iffalse meta-comment
%
% File: enparen.dtx
% Version: 2012/01/07 v1.0
% Info: Parentheses nesting
%
% Copyright (C) 2012 by
%    Heiko Oberdiek <heiko.oberdiek at googlemail.com>
%
% This work may be distributed and/or modified under the
% conditions of the LaTeX Project Public License, either
% version 1.3c of this license or (at your option) any later
% version. This version of this license is in
%    http://www.latex-project.org/lppl/lppl-1-3c.txt
% and the latest version of this license is in
%    http://www.latex-project.org/lppl.txt
% and version 1.3 or later is part of all distributions of
% LaTeX version 2005/12/01 or later.
%
% This work has the LPPL maintenance status "maintained".
%
% This Current Maintainer of this work is Heiko Oberdiek.
%
% This work consists of the main source file enparen.dtx
% and the derived files
%    enparen.sty, enparen.pdf, enparen.ins, enparen.drv.
%
% Distribution:
%    CTAN:macros/latex/contrib/oberdiek/enparen.dtx
%    CTAN:macros/latex/contrib/oberdiek/enparen.pdf
%
% Unpacking:
%    (a) If enparen.ins is present:
%           tex enparen.ins
%    (b) Without enparen.ins:
%           tex enparen.dtx
%    (c) If you insist on using LaTeX
%           latex \let\install=y\input{enparen.dtx}
%        (quote the arguments according to the demands of your shell)
%
% Documentation:
%    (a) If enparen.drv is present:
%           latex enparen.drv
%    (b) Without enparen.drv:
%           latex enparen.dtx; ...
%    The class ltxdoc loads the configuration file ltxdoc.cfg
%    if available. Here you can specify further options, e.g.
%    use A4 as paper format:
%       \PassOptionsToClass{a4paper}{article}
%
%    Programm calls to get the documentation (example):
%       pdflatex enparen.dtx
%       makeindex -s gind.ist enparen.idx
%       pdflatex enparen.dtx
%       makeindex -s gind.ist enparen.idx
%       pdflatex enparen.dtx
%
% Installation:
%    TDS:tex/latex/oberdiek/enparen.sty
%    TDS:doc/latex/oberdiek/enparen.pdf
%    TDS:source/latex/oberdiek/enparen.dtx
%
%<*ignore>
\begingroup
  \catcode123=1 %
  \catcode125=2 %
  \def\x{LaTeX2e}%
\expandafter\endgroup
\ifcase 0\ifx\install y1\fi\expandafter
         \ifx\csname processbatchFile\endcsname\relax\else1\fi
         \ifx\fmtname\x\else 1\fi\relax
\else\csname fi\endcsname
%</ignore>
%<*install>
\input docstrip.tex
\Msg{************************************************************************}
\Msg{* Installation}
\Msg{* Package: enparen 2012/01/07 v1.0 Parentheses nesting (HO)}
\Msg{************************************************************************}

\keepsilent
\askforoverwritefalse

\let\MetaPrefix\relax
\preamble

This is a generated file.

Project: enparen
Version: 2012/01/07 v1.0

Copyright (C) 2012 by
   Heiko Oberdiek <heiko.oberdiek at googlemail.com>

This work may be distributed and/or modified under the
conditions of the LaTeX Project Public License, either
version 1.3c of this license or (at your option) any later
version. This version of this license is in
   http://www.latex-project.org/lppl/lppl-1-3c.txt
and the latest version of this license is in
   http://www.latex-project.org/lppl.txt
and version 1.3 or later is part of all distributions of
LaTeX version 2005/12/01 or later.

This work has the LPPL maintenance status "maintained".

This Current Maintainer of this work is Heiko Oberdiek.

This work consists of the main source file enparen.dtx
and the derived files
   enparen.sty, enparen.pdf, enparen.ins, enparen.drv.

\endpreamble
\let\MetaPrefix\DoubleperCent

\generate{%
  \file{enparen.ins}{\from{enparen.dtx}{install}}%
  \file{enparen.drv}{\from{enparen.dtx}{driver}}%
  \usedir{tex/latex/oberdiek}%
  \file{enparen.sty}{\from{enparen.dtx}{package}}%
}

\catcode32=13\relax% active space
\let =\space%
\Msg{************************************************************************}
\Msg{*}
\Msg{* To finish the installation you have to move the following}
\Msg{* file into a directory searched by TeX:}
\Msg{*}
\Msg{*     enparen.sty}
\Msg{*}
\Msg{* To produce the documentation run the file `enparen.drv'}
\Msg{* through LaTeX.}
\Msg{*}
\Msg{* Happy TeXing!}
\Msg{*}
\Msg{************************************************************************}

\endbatchfile
%</install>
%<*ignore>
\fi
%</ignore>
%<*driver>
\NeedsTeXFormat{LaTeX2e}
\ProvidesFile{enparen.drv}%
  [2012/01/07 v1.0 Parentheses nesting (HO)]%
\documentclass{ltxdoc}
\usepackage{holtxdoc}[2011/11/22]
\usepackage{enparen}[2012/01/07]
\begin{document}
  \DocInput{enparen.dtx}%
\end{document}
%</driver>
% \fi
%
% \CheckSum{339}
%
% \CharacterTable
%  {Upper-case    \A\B\C\D\E\F\G\H\I\J\K\L\M\N\O\P\Q\R\S\T\U\V\W\X\Y\Z
%   Lower-case    \a\b\c\d\e\f\g\h\i\j\k\l\m\n\o\p\q\r\s\t\u\v\w\x\y\z
%   Digits        \0\1\2\3\4\5\6\7\8\9
%   Exclamation   \!     Double quote  \"     Hash (number) \#
%   Dollar        \$     Percent       \%     Ampersand     \&
%   Acute accent  \'     Left paren    \(     Right paren   \)
%   Asterisk      \*     Plus          \+     Comma         \,
%   Minus         \-     Point         \.     Solidus       \/
%   Colon         \:     Semicolon     \;     Less than     \<
%   Equals        \=     Greater than  \>     Question mark \?
%   Commercial at \@     Left bracket  \[     Backslash     \\
%   Right bracket \]     Circumflex    \^     Underscore    \_
%   Grave accent  \`     Left brace    \{     Vertical bar  \|
%   Right brace   \}     Tilde         \~}
%
% \GetFileInfo{enparen.drv}
%
% \title{The \xpackage{enparen} package}
% \date{2012/01/07 v1.0}
% \author{Heiko Oberdiek\\\xemail{heiko.oberdiek at googlemail.com}}
%
% \maketitle
%
% \begin{abstract}
% The package defines macros to set parentheses that automatically
% change the symbols from inner to outer fences.
% \end{abstract}
%
% \tableofcontents
%
% \section{Documentation}
%
% The \hologo{LaTeX} package provides macros for automatically changed
% parentheses symbols depending on the fence order.
% The innermost parentheses are always using the same
% parentheses. The symbols changes for the outer fences.
%
% Example:
% \begin{quote}
%   \enparen{a \enparen{b} \enparen{c \enparen{d}} \enparen{e}}
% \end{quote}
% is generated by
%\begin{quote}
%\begin{verbatim}
%\documentclass{article}
%\usepackage{enparen}
%\begin{document}
%  \enparen{a \enparen{b} \enparen{c \enparen{d}} \enparen{e}}
%\end{document}
%\end{verbatim}
%\end{quote}
%
% The package is the result of a newsgroup thread. Dan gives
% the following specification \cite{dan}:
% \begin{quote}
% ``On the other hand, the rules for fences are usually the reverse:
% innermost fences are always |()|, next outer are |[]|, etc. This means
% the opening fence has to wait until all the fences between it and
% the matching close have been detected before it can decide whether
% to be |(| or |[| or |{|.''
% \end{quote}
%
% The fence level counting starts from innermost parentheses with one.
% For the next outer fences the level is increased by one.
% The example above with level indexes:
% \begin{quote}
%   \def\1{\ensuremath{_{1}}}
%   \def\2{\ensuremath{_{2}}}
%   \def\3{\ensuremath{_{3}}}
%   \enparen{\3a \enparen{\1b}\1 \enparen{\2c \enparen{\1d}\1}\2 \enparen{e}}\3
% \end{quote}
% The correct level is only known at the closing symbol. Therefore the
% correct value is remembered in the main \xext{aux} file and used
% in the second \hologo{LaTeX} run.
%
% \subsection{User macros}
%
% \begin{declcs}{enparen} \M{text}
% \end{declcs}
% The macro \cs{enparen} puts its argument \meta{text} in parentheses.
% If the macro is nested, the used fence symbols change for the
% outer fences.
%
% \begin{declcs}{enparenLeft}\\
%   \cs{enparenRight}
% \end{declcs}
% Instead of |\enparen{text}| the left and right symbol can be
% used separately, but in pairs:
% \begin{quote}
%   |\enparenLeft text\enparenRight|
% \end{quote}
% \cs{enparenLeft} and \cs{enparenRight} may be used at different
% group levels, but they must be properly nested.
%
% \begin{declcs}{enparenSetSymbols} \M{level} \M{opening symbol} \M{closing symbol}
% \end{declcs}
% Macro \cs{enparenSetSymbols} configures the \meta{opening symbol} and
% \meta{closing symbol} for the \meta{level}. The \meta{level}
% is a number and the counting starts with one. Level zero is used,
% if the correct level is not known (e.g. in the first \hologo{LaTeX} run).
% The package defines the following sets:
%\begin{quote}
%\begin{verbatim}
%\enparenSetSymbols{0}{(}{)}
%\enparenSetSymbols{1}{(}{)}
%\enparenSetSymbols{2}{[}{]}
%\enparenSetSymbols{3}{\{}{\}}
%\end{verbatim}
%\end{quote}
% Example for changing the third and adding a fourth level:
%\begin{quote}
%\begin{verbatim}
%\enparenSetSymbols{3}{\ensuremath{\langle}}{\ensuremath{\rangle}}
%\enparenSetSymbols{4}{\{}{\}}
%\end{verbatim}
%\end{quote}
%
% \begin{declcs}{enparenUnsetSymbols} \M{level}
% \end{declcs}
% The symbols for level \meta{level} are removed.
% Example scenario: Only two nesting levels must be used,
% the package defines more, then the third level can be
% disabled by |\enparenUnsetSymbols{3}| and the user
% gets warnings if parentheses at level 3 are needed.
%
% \subsection{Contexts}
%
% \begin{declcs}{enparenBeginContext} \M{name}\\
%   \cs{enparenEndContext} \M{name}
% \end{declcs}
% If the current text is interrupted by footnotes, floats
% with captions, then the parentheses inside the text of footnotes,
% captions, \dots should restart from scratch. This can be achieved
% by embedding the text inside macros \cs{enparenBeginContext}
% and \cs{enparenEndContext}. These macros must be properly nested.
% The \meta{name} for the begin and end macro must be the same.
% It is a help for debugging problems, because the warning messages
% show the context name. But it is not necessary that the
% begin\slash end pairs have different names. Example:
%\begin{quote}
%\begin{verbatim}
%\enparenLeft text before table ...
%\begin{table}
%  \caption{Table caption}
%  \enparenBegin{Context}{table}
%  Other text \enparen{foobar}.
%  \enparenEnd{Context}{table}
%\end{table}
%text after table ...
%\enparenRight
%\end{verbatim}
%\end{quote}
% The parentheses inside the table environment and context `table'
% are not nested inside other parentheses: \enparen{foobar}.
% In case of captions and footnotes the contexts are automatically
% added, see next section about options.
%
% \subsection{Options}
%
% \begin{declcs}{enparenSetup} \M{key value list}
% \end{declcs}
% Some options (currently all) can also be set after the
% package is loaded. They can be set in the argument
% \meta{key value list} of \cs{enparenSetup}. Options are disabled
% after they are used the last time. Currently all options are boolean
% options and are disabled in |\begin{document}|.
%
% \begin{description}
% \item[\xoption{caption}:] The caption text is put in a
%   context |caption|.
% \item[\xoption{footnote}:] The footnote text is put in a
%   context |footnote|.
% \end{description}
%
% Example for disabling the two options at different places:
%\begin{quote}
%\begin{verbatim}
%\usepackage[caption=false]{enparen}
%\enparenSetup{footnote=false}
%\end{verbatim}
%\end{quote}
%
% \subsection{Notes}
%
% \begin{description}
% \item[Implicite kerning:]
%   Unexpandable stuff might affect the implicite kerning.
%   The package cannot avoid this, because it need to define
%   and redefine macros at the occurence of each symbol.
%   This is done before the opening and after the closing
%   symbol, thus that the implicite kerning inside
%   is not affected.
% \end{description}
%
% \StopEventually{
% }
%
% \section{Implementation}
%
%    \begin{macrocode}
%<*package>
\NeedsTeXFormat{LaTeX2e}
\ProvidesPackage{enparen}
  [2012/01/07 v1.0 Parentheses nesting (HO)]%
%    \end{macrocode}
%
% \subsection{Resources}
%
%    \begin{macrocode}
\RequirePackage{ltxcmds}[2011/11/09]
%    \end{macrocode}
%
%    \begin{macrocode}
\ltx@IfUndefined{numexpr}{%
  \PackageError{enparen}{%
    Missing e-TeX's \ltx@backslashchar numexpr.\MessageBreak
    The package will continue with emergency definitions%
  }\@ehc
  \def\enparenLeft{(}%
  \def\enparenRight{)}%
  \long\def\enparen#1{\enparenLeft#1\enparenRight}%
  \let\enparenSetup\ltx@gobble
  \let\enparenSetSymbols\ltx@gobblethree
  \let\enparenUnsetSymbols\ltx@gobble
  \endinput
}{}
%    \end{macrocode}
%
%    \begin{macrocode}
\RequirePackage{protecteddef}[2011/01/31]
\RequirePackage{atveryend}[2011/06/30]
\RequirePackage{uniquecounter}[2011/01/30]
\RequirePackage{zref-base}[2011/03/18]
\RequirePackage{kvoptions}[2011/06/30]
\RequirePackage{kvsetkeys}[2011/10/18]
%    \end{macrocode}
%
%    \begin{macro}{\zref@wrapper@mainaux}
%    \begin{macrocode}
\providecommand{\zref@wrapper@mainaux}[1]{%
  \ifx\@auxout\@mainaux
    #1%
  \else
    \begingroup
      \let\@auxout\@mainaux
      #1%
    \endgroup
  \fi
}
%    \end{macrocode}
%    \end{macro}
%
%    \begin{macrocode}
\UniqueCounterNew{enparen}
%    \end{macrocode}
%
%    \begin{macrocode}
\zref@newprop{enparen}[0]{}
%    \end{macrocode}
%
% \subsection{Contexts}
%
%    \begin{macro}{\enparenContextDefault}
%    \begin{macrocode}
\def\enparenContextDefault{main}
%    \end{macrocode}
%    \end{macro}
%
%    \begin{macro}{\enparen@ctx}
%    \begin{macrocode}
\let\enparen@ctx\ltx@empty
%    \end{macrocode}
%    \end{macro}
%
%    \begin{macro}{\enparen@stack}
%    \begin{macrocode}
\let\enparen@stack\ltx@empty
%    \end{macrocode}
%    \end{macro}
%
% \subsubsection{Stack for contexts}
%
%    \begin{macro}{\enparen@CtxStack}
%    \begin{macrocode}
\global\let\enparen@CtxStack\ltx@empty
%    \end{macrocode}
%    \end{macro}
%
%    \begin{macro}{\enparen@CtxStackPush}
%    \begin{macrocode}
\def\enparen@CtxStackPush#1{%
  \xdef\enparen@CtxStack{%
    {\enparen@ctx}{\enparen@stack}%
    \enparen@CtxStack
  }%
  \xdef\enparen@ctx{#1}%
  \global\let\enparen@stack\ltx@empty
}
%    \end{macrocode}
%    \end{macro}
%
%    \begin{macro}{\enparen@CtxStackPop}
%    \begin{macrocode}
\def\enparen@CtxStackPop{%
  \ifx\enparen@CtxStack\ltx@empty
    \PackageWarning{enparen}{%
      Pop request for empty context stack%
    }%
    \global\let\enparen@ctx\enparenContextDefault
    \global\let\enparen@stack\ltx@empty
  \else
    \xdef\enparen@ctx{%
      \expandafter\ltx@car\enparen@CtxStack\@nil
    }%
    \xdef\enparen@stack{%
      \expandafter\ltx@carsecond\enparen@CtxStack\@nil
    }%
    \xdef\enparen@CtxStack{%
      \expandafter\ltx@cdrtwo\enparen@CtxStack\@nil
    }%
  \fi
}
%    \end{macrocode}
%    \end{macro}
%
% \subsubsection{Context user macros}
%
%    \begin{macro}{\enparenBeginContext}
%    \begin{macrocode}
\ProtectedDef*{\enparenBeginContext}[1]{%
  \enparen@CtxStackPush{#1}%
}
%    \end{macrocode}
%    \end{macro}
%    \begin{macro}{\enparenEndContext}
%    \begin{macrocode}
\ProtectedDef*{\enparenEndContext}[1]{%
  \edef\enparen@temp{#1}%
  \ifx\enparen@temp\enparen@ctx
  \else
    \PackageWarning{enparen}{%
      Context mismatch in end request.\MessageBreak
      `#1' should be ended, but current context\MessageBreak
      is `\enparen@ctx'%
    }%
  \fi
  \enparenCheckEmptyStack
  \enparen@CtxStackPop
}
%    \end{macrocode}
%    \end{macro}
%
%    \begin{macro}{\enparenCheckEmptyStack}
%    \begin{macrocode}
\ProtectedDef*{\enparenCheckEmptyStack}[0]{%
  \ifx\enparen@stack\ltx@empty
  \else
    \PackageWarning{enparen}{%
      Ending non-empty stack `\enparen@ctx':\MessageBreak
      \enparen@PrintStack\MessageBreak
    }%
  \fi
}
%    \end{macrocode}
%    \end{macro}
%
%    \begin{macro}{\enparen@PrintStack}
%    \begin{macrocode}
\def\enparen@PrintStack{%
  \expandafter\enparen@PrintStackAux
  \enparen@stack\ltx@empty\ltx@empty
}
%    \end{macrocode}
%    \end{macro}
%    \begin{macro}{\enparen@PrintStackAux}
%    \begin{macrocode}
\def\enparen@PrintStackAux#1#2{%
  \ifx\ltx@empty#1%
  \else
    {#1:#2}%
    \expandafter\enparen@PrintStackAux
  \fi
}
%    \end{macrocode}
%    \end{macro}
%
% \subsection{Symbols}
%
%    \begin{macro}{\enparenSetSymbols}
%    \begin{macrocode}
\ProtectedDef*{\enparenSetSymbols}[3]{%
  \expandafter
  \def\csname enparen@symbol\the\numexpr#1L\endcsname{#2}%
  \expandafter
  \def\csname enparen@symbol\the\numexpr#1R\endcsname{#3}%
}
%    \end{macrocode}
%    \end{macro}
%    \begin{macro}{\enparenUnsetSymbols}
%    \begin{macrocode}
\ProtectedDef*{\enparenUnsetSymbols}[1]{%
  \expandafter
  \let\csname enparen@symbol\the\numexpr#1L\endcsname\ltx@undefined
  \expandafter
  \let\csname enparen@symbol\the\numexpr#1R\endcsname\ltx@undefined
}
%    \end{macrocode}
%    \end{macro}
%
%    \begin{macrocode}
\enparenSetSymbols{0}{(}{)}
\enparenSetSymbols{1}{(}{)}
\enparenSetSymbols{2}{[}{]}
\enparenSetSymbols{3}{\{}{\}}
\enparenSetSymbols{4}{\ensuremath{\langle}}{\ensuremath{\rangle}}
%    \end{macrocode}
%
% \subsection{Main user macros}
%
%    \begin{macro}{\enparen}
%    \begin{macrocode}
\ProtectedDef{\enparen}[1]{%
  \enparenLeft#1\enparenRight
}
%    \end{macrocode}
%    \end{macro}
%
%    \begin{macro}{\enparenLeft}
%    \begin{macrocode}
\ProtectedDef*{\enparenLeft}[0]{%
  \UniqueCounterCall{enparen}\enparen@Left
}
%    \end{macrocode}
%    \end{macro}
%    \begin{macro}{\enparen@Left}
%    \begin{macrocode}
\def\enparen@Left#1{%
  \xdef\enparen@stack{%
    {#1}{1}%
    \expandafter\enparen@Inc\expandafter2\expandafter!%
    \enparen@stack\ltx@empty\ltx@empty
  }%
  \edef\enparen@tmp{\zref@extract{enparen#1}{enparen}}%
  \ltx@IfUndefined{enparen@symbol\enparen@tmp L}{%
    \PackageWarning{enparen}{%
      Undefined symbols for level \enparen@tmp
    }%
    \csname enparen@symbol0L\endcsname
  }{%
    \csname enparen@symbol\enparen@tmp L\endcsname
  }%
}
%    \end{macrocode}
%    \end{macro}
%    \begin{macro}{\enparen@Inc}
%    \begin{macrocode}
\def\enparen@Inc#1!#2#3{%
  \ifx\ltx@empty#2%
  \else
    \ifnum#3<#1 %
      {#2}{#1}%
      \expandafter\enparen@Inc
      \the\numexpr#1+1\expandafter\expandafter\expandafter!%
    \else
      {#2}{#3}%
    \fi
  \fi
}
%    \end{macrocode}
%    \end{macro}
%
%    \begin{macro}{\enparenRight}
%    \begin{macrocode}
\ProtectedDef*{\enparenRight}[0]{%
  \ifx\enparen@stack\ltx@empty
    \PackageWarning{enparen}{%
      Missing left symbol for right symbol%
    }%
    \csname enparen@symbol0R\endcsname
  \else
    \expandafter\enparen@Right\enparen@stack\@nil
  \fi
}
%    \end{macrocode}
%    \end{macro}
%
%    \begin{macro}{\enparen@Right}
%    \begin{macrocode}
\def\enparen@Right#1#2#3\@nil{%
  \ltx@IfUndefined{%
    enparen@symbol%
    \zref@extract{enparen#1}{enparen}%
    R%
  }{%
    \csname enparen@symbol0R\endcsname
  }{%
    \csname
      enparen@symbol%
      \zref@extract{enparen#1}{enparen}%
       R%
    \endcsname
  }%
  \zref@wrapper@mainaux{%
    \zref@setcurrent{enparen}{#2}%
    \zref@labelbyprops{enparen#1}{enparen}%
  }%
  \xdef\enparen@stack{#3}%
}
%    \end{macrocode}
%    \end{macro}
%
% \subsection{Options}
%
%    \begin{macrocode}
\SetupKeyvalOptions{%
  family=enparen,%
  prefix=enparen@,%
}
%    \end{macrocode}
%
%    \begin{macro}{\enparenSetup}
%    \begin{macrocode}
\ProtectedDef*{\enparenSetup}[0]{%
  \kvsetkeys{enparen}%
}
%    \end{macrocode}
%    \end{macro}
%
%    \begin{macrocode}
\DeclareBoolOption[true]{footnote}
\DeclareBoolOption[true]{caption}
\ProcessKeyvalOptions*
%    \end{macrocode}
%
% \subsection{Context settings}
%
%    \begin{macro}{\enparen@AtBegin}
%    \begin{macrocode}
\def\enparen@AtBegin{%
  \ifenparen@footnote
    \let\enparen@org@makefntext\@makefntext
    \long\def\@makefntext##1{%
      \enparen@org@makefntext{%
        \enparenBeginContext{footnote}%
        ##1%
        \enparenEndContext{footnote}%
      }%
    }%
  \fi
  \enparen@Disable{footnote}%
  \ifenparen@caption
    \let\enparen@org@makecaption\@makecaption
    \long\def\@makecaption##1##2{%
      \enparen@org@makecaption{##1}{%
        \enparenBeginContext{caption}%
        ##2%
        \enparenEndContext{caption}%
      }%
    }%
  \fi
  \enparen@Disable{caption}%
}
%    \end{macrocode}
%    \end{macro}
%
%    \begin{macro}{\enparen@Disable}
%    \begin{macrocode}
\def\enparen@Disable#1{%
  \DisableKeyvalOption[%
    action=warning,%
    package=enparen,%
  ]{enparen}{#1}%
}
%    \end{macrocode}
%    \end{macro}
%
%    \begin{macrocode}
\AtBeginDocument{\enparen@AtBegin}
%    \end{macrocode}
%
% \subsection{At end of document}
%
%    \begin{macro}{\enparen@AtEnd}
%    \begin{macrocode}
\def\enparen@AtEnd{%
  \enparenCheckEmptyStack
  \ifx\enparen@CtxStack\ltx@empty
  \else
    \PackageWarningNoLine{enparen}{%
      Context stack is not empty at end of document.\MessageBreak
      Current stack and contents of context stack:\MessageBreak
      [\enparen@ctx]:[\enparen@PrintStack]%
      \expandafter
      \enparen@PrintContextStack\enparen@CtxStack\relax\relax
    }%
  \fi
}
%    \end{macrocode}
%    \end{macro}
%    \begin{macro}{\enparen@PrintContextStack}
%    \begin{macrocode}
\def\enparen@PrintContextStack#1#2{%
  \ifx\relax#1\ltx@empty
  \else
    \MessageBreak
    [#1]:[\enparen@PrintStackAux#2\ltx@empty\ltx@empty]% hash-ok
    \expandafter\enparen@PrintContextStack
  \fi
}
%    \end{macrocode}
%    \end{macro}
%    \begin{macrocode}
\AtVeryEndDocument{\enparen@AtEnd}
%    \end{macrocode}
%
%    \begin{macrocode}
%</package>
%    \end{macrocode}
%
% \section{Installation}
%
% \subsection{Download}
%
% \paragraph{Package.} This package is available on
% CTAN\footnote{\url{ftp://ftp.ctan.org/tex-archive/}}:
% \begin{description}
% \item[\CTAN{macros/latex/contrib/oberdiek/enparen.dtx}] The source file.
% \item[\CTAN{macros/latex/contrib/oberdiek/enparen.pdf}] Documentation.
% \end{description}
%
%
% \paragraph{Bundle.} All the packages of the bundle `oberdiek'
% are also available in a TDS compliant ZIP archive. There
% the packages are already unpacked and the documentation files
% are generated. The files and directories obey the TDS standard.
% \begin{description}
% \item[\CTAN{install/macros/latex/contrib/oberdiek.tds.zip}]
% \end{description}
% \emph{TDS} refers to the standard ``A Directory Structure
% for \TeX\ Files'' (\CTAN{tds/tds.pdf}). Directories
% with \xfile{texmf} in their name are usually organized this way.
%
% \subsection{Bundle installation}
%
% \paragraph{Unpacking.} Unpack the \xfile{oberdiek.tds.zip} in the
% TDS tree (also known as \xfile{texmf} tree) of your choice.
% Example (linux):
% \begin{quote}
%   |unzip oberdiek.tds.zip -d ~/texmf|
% \end{quote}
%
% \paragraph{Script installation.}
% Check the directory \xfile{TDS:scripts/oberdiek/} for
% scripts that need further installation steps.
% Package \xpackage{attachfile2} comes with the Perl script
% \xfile{pdfatfi.pl} that should be installed in such a way
% that it can be called as \texttt{pdfatfi}.
% Example (linux):
% \begin{quote}
%   |chmod +x scripts/oberdiek/pdfatfi.pl|\\
%   |cp scripts/oberdiek/pdfatfi.pl /usr/local/bin/|
% \end{quote}
%
% \subsection{Package installation}
%
% \paragraph{Unpacking.} The \xfile{.dtx} file is a self-extracting
% \docstrip\ archive. The files are extracted by running the
% \xfile{.dtx} through \plainTeX:
% \begin{quote}
%   \verb|tex enparen.dtx|
% \end{quote}
%
% \paragraph{TDS.} Now the different files must be moved into
% the different directories in your installation TDS tree
% (also known as \xfile{texmf} tree):
% \begin{quote}
% \def\t{^^A
% \begin{tabular}{@{}>{\ttfamily}l@{ $\rightarrow$ }>{\ttfamily}l@{}}
%   enparen.sty & tex/latex/oberdiek/enparen.sty\\
%   enparen.pdf & doc/latex/oberdiek/enparen.pdf\\
%   enparen.dtx & source/latex/oberdiek/enparen.dtx\\
% \end{tabular}^^A
% }^^A
% \sbox0{\t}^^A
% \ifdim\wd0>\linewidth
%   \begingroup
%     \advance\linewidth by\leftmargin
%     \advance\linewidth by\rightmargin
%   \edef\x{\endgroup
%     \def\noexpand\lw{\the\linewidth}^^A
%   }\x
%   \def\lwbox{^^A
%     \leavevmode
%     \hbox to \linewidth{^^A
%       \kern-\leftmargin\relax
%       \hss
%       \usebox0
%       \hss
%       \kern-\rightmargin\relax
%     }^^A
%   }^^A
%   \ifdim\wd0>\lw
%     \sbox0{\small\t}^^A
%     \ifdim\wd0>\linewidth
%       \ifdim\wd0>\lw
%         \sbox0{\footnotesize\t}^^A
%         \ifdim\wd0>\linewidth
%           \ifdim\wd0>\lw
%             \sbox0{\scriptsize\t}^^A
%             \ifdim\wd0>\linewidth
%               \ifdim\wd0>\lw
%                 \sbox0{\tiny\t}^^A
%                 \ifdim\wd0>\linewidth
%                   \lwbox
%                 \else
%                   \usebox0
%                 \fi
%               \else
%                 \lwbox
%               \fi
%             \else
%               \usebox0
%             \fi
%           \else
%             \lwbox
%           \fi
%         \else
%           \usebox0
%         \fi
%       \else
%         \lwbox
%       \fi
%     \else
%       \usebox0
%     \fi
%   \else
%     \lwbox
%   \fi
% \else
%   \usebox0
% \fi
% \end{quote}
% If you have a \xfile{docstrip.cfg} that configures and enables \docstrip's
% TDS installing feature, then some files can already be in the right
% place, see the documentation of \docstrip.
%
% \subsection{Refresh file name databases}
%
% If your \TeX~distribution
% (\teTeX, \mikTeX, \dots) relies on file name databases, you must refresh
% these. For example, \teTeX\ users run \verb|texhash| or
% \verb|mktexlsr|.
%
% \subsection{Some details for the interested}
%
% \paragraph{Attached source.}
%
% The PDF documentation on CTAN also includes the
% \xfile{.dtx} source file. It can be extracted by
% AcrobatReader 6 or higher. Another option is \textsf{pdftk},
% e.g. unpack the file into the current directory:
% \begin{quote}
%   \verb|pdftk enparen.pdf unpack_files output .|
% \end{quote}
%
% \paragraph{Unpacking with \LaTeX.}
% The \xfile{.dtx} chooses its action depending on the format:
% \begin{description}
% \item[\plainTeX:] Run \docstrip\ and extract the files.
% \item[\LaTeX:] Generate the documentation.
% \end{description}
% If you insist on using \LaTeX\ for \docstrip\ (really,
% \docstrip\ does not need \LaTeX), then inform the autodetect routine
% about your intention:
% \begin{quote}
%   \verb|latex \let\install=y\input{enparen.dtx}|
% \end{quote}
% Do not forget to quote the argument according to the demands
% of your shell.
%
% \paragraph{Generating the documentation.}
% You can use both the \xfile{.dtx} or the \xfile{.drv} to generate
% the documentation. The process can be configured by the
% configuration file \xfile{ltxdoc.cfg}. For instance, put this
% line into this file, if you want to have A4 as paper format:
% \begin{quote}
%   \verb|\PassOptionsToClass{a4paper}{article}|
% \end{quote}
% An example follows how to generate the
% documentation with pdf\LaTeX:
% \begin{quote}
%\begin{verbatim}
%pdflatex enparen.dtx
%makeindex -s gind.ist enparen.idx
%pdflatex enparen.dtx
%makeindex -s gind.ist enparen.idx
%pdflatex enparen.dtx
%\end{verbatim}
% \end{quote}
%
% \begin{thebibliography}{9}
% \raggedright
% \bibitem{dan}
%   Dan Luecking: \textit{Re: bracket order};
%   newsgroup \xnewsgroup{comp.text.tex};
%   message id \texttt{\small$\langle$9b07c9c8-ff92-4cbf-b3a9-84efecfeb506@l24g2000yqm.googlegroups.com$\rangle$}
%   2012-01-05.
%   \url{http://groups.google.com/group/comp.text.tex/msg/8774519da31c2352}
% \end{thebibliography}
%
% \begin{History}
%   \begin{Version}{2012/01/07 v1.0}
%   \item
%     First version.
%   \end{Version}
% \end{History}
%
% \PrintIndex
%
% \Finale
\endinput

%        (quote the arguments according to the demands of your shell)
%
% Documentation:
%    (a) If enparen.drv is present:
%           latex enparen.drv
%    (b) Without enparen.drv:
%           latex enparen.dtx; ...
%    The class ltxdoc loads the configuration file ltxdoc.cfg
%    if available. Here you can specify further options, e.g.
%    use A4 as paper format:
%       \PassOptionsToClass{a4paper}{article}
%
%    Programm calls to get the documentation (example):
%       pdflatex enparen.dtx
%       makeindex -s gind.ist enparen.idx
%       pdflatex enparen.dtx
%       makeindex -s gind.ist enparen.idx
%       pdflatex enparen.dtx
%
% Installation:
%    TDS:tex/latex/oberdiek/enparen.sty
%    TDS:doc/latex/oberdiek/enparen.pdf
%    TDS:source/latex/oberdiek/enparen.dtx
%
%<*ignore>
\begingroup
  \catcode123=1 %
  \catcode125=2 %
  \def\x{LaTeX2e}%
\expandafter\endgroup
\ifcase 0\ifx\install y1\fi\expandafter
         \ifx\csname processbatchFile\endcsname\relax\else1\fi
         \ifx\fmtname\x\else 1\fi\relax
\else\csname fi\endcsname
%</ignore>
%<*install>
\input docstrip.tex
\Msg{************************************************************************}
\Msg{* Installation}
\Msg{* Package: enparen 2012/01/07 v1.0 Parentheses nesting (HO)}
\Msg{************************************************************************}

\keepsilent
\askforoverwritefalse

\let\MetaPrefix\relax
\preamble

This is a generated file.

Project: enparen
Version: 2012/01/07 v1.0

Copyright (C) 2012 by
   Heiko Oberdiek <heiko.oberdiek at googlemail.com>

This work may be distributed and/or modified under the
conditions of the LaTeX Project Public License, either
version 1.3c of this license or (at your option) any later
version. This version of this license is in
   http://www.latex-project.org/lppl/lppl-1-3c.txt
and the latest version of this license is in
   http://www.latex-project.org/lppl.txt
and version 1.3 or later is part of all distributions of
LaTeX version 2005/12/01 or later.

This work has the LPPL maintenance status "maintained".

This Current Maintainer of this work is Heiko Oberdiek.

This work consists of the main source file enparen.dtx
and the derived files
   enparen.sty, enparen.pdf, enparen.ins, enparen.drv.

\endpreamble
\let\MetaPrefix\DoubleperCent

\generate{%
  \file{enparen.ins}{\from{enparen.dtx}{install}}%
  \file{enparen.drv}{\from{enparen.dtx}{driver}}%
  \usedir{tex/latex/oberdiek}%
  \file{enparen.sty}{\from{enparen.dtx}{package}}%
}

\catcode32=13\relax% active space
\let =\space%
\Msg{************************************************************************}
\Msg{*}
\Msg{* To finish the installation you have to move the following}
\Msg{* file into a directory searched by TeX:}
\Msg{*}
\Msg{*     enparen.sty}
\Msg{*}
\Msg{* To produce the documentation run the file `enparen.drv'}
\Msg{* through LaTeX.}
\Msg{*}
\Msg{* Happy TeXing!}
\Msg{*}
\Msg{************************************************************************}

\endbatchfile
%</install>
%<*ignore>
\fi
%</ignore>
%<*driver>
\NeedsTeXFormat{LaTeX2e}
\ProvidesFile{enparen.drv}%
  [2012/01/07 v1.0 Parentheses nesting (HO)]%
\documentclass{ltxdoc}
\usepackage{holtxdoc}[2011/11/22]
\usepackage{enparen}[2012/01/07]
\begin{document}
  \DocInput{enparen.dtx}%
\end{document}
%</driver>
% \fi
%
% \CheckSum{339}
%
% \CharacterTable
%  {Upper-case    \A\B\C\D\E\F\G\H\I\J\K\L\M\N\O\P\Q\R\S\T\U\V\W\X\Y\Z
%   Lower-case    \a\b\c\d\e\f\g\h\i\j\k\l\m\n\o\p\q\r\s\t\u\v\w\x\y\z
%   Digits        \0\1\2\3\4\5\6\7\8\9
%   Exclamation   \!     Double quote  \"     Hash (number) \#
%   Dollar        \$     Percent       \%     Ampersand     \&
%   Acute accent  \'     Left paren    \(     Right paren   \)
%   Asterisk      \*     Plus          \+     Comma         \,
%   Minus         \-     Point         \.     Solidus       \/
%   Colon         \:     Semicolon     \;     Less than     \<
%   Equals        \=     Greater than  \>     Question mark \?
%   Commercial at \@     Left bracket  \[     Backslash     \\
%   Right bracket \]     Circumflex    \^     Underscore    \_
%   Grave accent  \`     Left brace    \{     Vertical bar  \|
%   Right brace   \}     Tilde         \~}
%
% \GetFileInfo{enparen.drv}
%
% \title{The \xpackage{enparen} package}
% \date{2012/01/07 v1.0}
% \author{Heiko Oberdiek\\\xemail{heiko.oberdiek at googlemail.com}}
%
% \maketitle
%
% \begin{abstract}
% The package defines macros to set parentheses that automatically
% change the symbols from inner to outer fences.
% \end{abstract}
%
% \tableofcontents
%
% \section{Documentation}
%
% The \hologo{LaTeX} package provides macros for automatically changed
% parentheses symbols depending on the fence order.
% The innermost parentheses are always using the same
% parentheses. The symbols changes for the outer fences.
%
% Example:
% \begin{quote}
%   \enparen{a \enparen{b} \enparen{c \enparen{d}} \enparen{e}}
% \end{quote}
% is generated by
%\begin{quote}
%\begin{verbatim}
%\documentclass{article}
%\usepackage{enparen}
%\begin{document}
%  \enparen{a \enparen{b} \enparen{c \enparen{d}} \enparen{e}}
%\end{document}
%\end{verbatim}
%\end{quote}
%
% The package is the result of a newsgroup thread. Dan gives
% the following specification \cite{dan}:
% \begin{quote}
% ``On the other hand, the rules for fences are usually the reverse:
% innermost fences are always |()|, next outer are |[]|, etc. This means
% the opening fence has to wait until all the fences between it and
% the matching close have been detected before it can decide whether
% to be |(| or |[| or |{|.''
% \end{quote}
%
% The fence level counting starts from innermost parentheses with one.
% For the next outer fences the level is increased by one.
% The example above with level indexes:
% \begin{quote}
%   \def\1{\ensuremath{_{1}}}
%   \def\2{\ensuremath{_{2}}}
%   \def\3{\ensuremath{_{3}}}
%   \enparen{\3a \enparen{\1b}\1 \enparen{\2c \enparen{\1d}\1}\2 \enparen{e}}\3
% \end{quote}
% The correct level is only known at the closing symbol. Therefore the
% correct value is remembered in the main \xext{aux} file and used
% in the second \hologo{LaTeX} run.
%
% \subsection{User macros}
%
% \begin{declcs}{enparen} \M{text}
% \end{declcs}
% The macro \cs{enparen} puts its argument \meta{text} in parentheses.
% If the macro is nested, the used fence symbols change for the
% outer fences.
%
% \begin{declcs}{enparenLeft}\\
%   \cs{enparenRight}
% \end{declcs}
% Instead of |\enparen{text}| the left and right symbol can be
% used separately, but in pairs:
% \begin{quote}
%   |\enparenLeft text\enparenRight|
% \end{quote}
% \cs{enparenLeft} and \cs{enparenRight} may be used at different
% group levels, but they must be properly nested.
%
% \begin{declcs}{enparenSetSymbols} \M{level} \M{opening symbol} \M{closing symbol}
% \end{declcs}
% Macro \cs{enparenSetSymbols} configures the \meta{opening symbol} and
% \meta{closing symbol} for the \meta{level}. The \meta{level}
% is a number and the counting starts with one. Level zero is used,
% if the correct level is not known (e.g. in the first \hologo{LaTeX} run).
% The package defines the following sets:
%\begin{quote}
%\begin{verbatim}
%\enparenSetSymbols{0}{(}{)}
%\enparenSetSymbols{1}{(}{)}
%\enparenSetSymbols{2}{[}{]}
%\enparenSetSymbols{3}{\{}{\}}
%\end{verbatim}
%\end{quote}
% Example for changing the third and adding a fourth level:
%\begin{quote}
%\begin{verbatim}
%\enparenSetSymbols{3}{\ensuremath{\langle}}{\ensuremath{\rangle}}
%\enparenSetSymbols{4}{\{}{\}}
%\end{verbatim}
%\end{quote}
%
% \begin{declcs}{enparenUnsetSymbols} \M{level}
% \end{declcs}
% The symbols for level \meta{level} are removed.
% Example scenario: Only two nesting levels must be used,
% the package defines more, then the third level can be
% disabled by |\enparenUnsetSymbols{3}| and the user
% gets warnings if parentheses at level 3 are needed.
%
% \subsection{Contexts}
%
% \begin{declcs}{enparenBeginContext} \M{name}\\
%   \cs{enparenEndContext} \M{name}
% \end{declcs}
% If the current text is interrupted by footnotes, floats
% with captions, then the parentheses inside the text of footnotes,
% captions, \dots should restart from scratch. This can be achieved
% by embedding the text inside macros \cs{enparenBeginContext}
% and \cs{enparenEndContext}. These macros must be properly nested.
% The \meta{name} for the begin and end macro must be the same.
% It is a help for debugging problems, because the warning messages
% show the context name. But it is not necessary that the
% begin\slash end pairs have different names. Example:
%\begin{quote}
%\begin{verbatim}
%\enparenLeft text before table ...
%\begin{table}
%  \caption{Table caption}
%  \enparenBegin{Context}{table}
%  Other text \enparen{foobar}.
%  \enparenEnd{Context}{table}
%\end{table}
%text after table ...
%\enparenRight
%\end{verbatim}
%\end{quote}
% The parentheses inside the table environment and context `table'
% are not nested inside other parentheses: \enparen{foobar}.
% In case of captions and footnotes the contexts are automatically
% added, see next section about options.
%
% \subsection{Options}
%
% \begin{declcs}{enparenSetup} \M{key value list}
% \end{declcs}
% Some options (currently all) can also be set after the
% package is loaded. They can be set in the argument
% \meta{key value list} of \cs{enparenSetup}. Options are disabled
% after they are used the last time. Currently all options are boolean
% options and are disabled in |\begin{document}|.
%
% \begin{description}
% \item[\xoption{caption}:] The caption text is put in a
%   context |caption|.
% \item[\xoption{footnote}:] The footnote text is put in a
%   context |footnote|.
% \end{description}
%
% Example for disabling the two options at different places:
%\begin{quote}
%\begin{verbatim}
%\usepackage[caption=false]{enparen}
%\enparenSetup{footnote=false}
%\end{verbatim}
%\end{quote}
%
% \subsection{Notes}
%
% \begin{description}
% \item[Implicite kerning:]
%   Unexpandable stuff might affect the implicite kerning.
%   The package cannot avoid this, because it need to define
%   and redefine macros at the occurence of each symbol.
%   This is done before the opening and after the closing
%   symbol, thus that the implicite kerning inside
%   is not affected.
% \end{description}
%
% \StopEventually{
% }
%
% \section{Implementation}
%
%    \begin{macrocode}
%<*package>
\NeedsTeXFormat{LaTeX2e}
\ProvidesPackage{enparen}
  [2012/01/07 v1.0 Parentheses nesting (HO)]%
%    \end{macrocode}
%
% \subsection{Resources}
%
%    \begin{macrocode}
\RequirePackage{ltxcmds}[2011/11/09]
%    \end{macrocode}
%
%    \begin{macrocode}
\ltx@IfUndefined{numexpr}{%
  \PackageError{enparen}{%
    Missing e-TeX's \ltx@backslashchar numexpr.\MessageBreak
    The package will continue with emergency definitions%
  }\@ehc
  \def\enparenLeft{(}%
  \def\enparenRight{)}%
  \long\def\enparen#1{\enparenLeft#1\enparenRight}%
  \let\enparenSetup\ltx@gobble
  \let\enparenSetSymbols\ltx@gobblethree
  \let\enparenUnsetSymbols\ltx@gobble
  \endinput
}{}
%    \end{macrocode}
%
%    \begin{macrocode}
\RequirePackage{protecteddef}[2011/01/31]
\RequirePackage{atveryend}[2011/06/30]
\RequirePackage{uniquecounter}[2011/01/30]
\RequirePackage{zref-base}[2011/03/18]
\RequirePackage{kvoptions}[2011/06/30]
\RequirePackage{kvsetkeys}[2011/10/18]
%    \end{macrocode}
%
%    \begin{macro}{\zref@wrapper@mainaux}
%    \begin{macrocode}
\providecommand{\zref@wrapper@mainaux}[1]{%
  \ifx\@auxout\@mainaux
    #1%
  \else
    \begingroup
      \let\@auxout\@mainaux
      #1%
    \endgroup
  \fi
}
%    \end{macrocode}
%    \end{macro}
%
%    \begin{macrocode}
\UniqueCounterNew{enparen}
%    \end{macrocode}
%
%    \begin{macrocode}
\zref@newprop{enparen}[0]{}
%    \end{macrocode}
%
% \subsection{Contexts}
%
%    \begin{macro}{\enparenContextDefault}
%    \begin{macrocode}
\def\enparenContextDefault{main}
%    \end{macrocode}
%    \end{macro}
%
%    \begin{macro}{\enparen@ctx}
%    \begin{macrocode}
\let\enparen@ctx\ltx@empty
%    \end{macrocode}
%    \end{macro}
%
%    \begin{macro}{\enparen@stack}
%    \begin{macrocode}
\let\enparen@stack\ltx@empty
%    \end{macrocode}
%    \end{macro}
%
% \subsubsection{Stack for contexts}
%
%    \begin{macro}{\enparen@CtxStack}
%    \begin{macrocode}
\global\let\enparen@CtxStack\ltx@empty
%    \end{macrocode}
%    \end{macro}
%
%    \begin{macro}{\enparen@CtxStackPush}
%    \begin{macrocode}
\def\enparen@CtxStackPush#1{%
  \xdef\enparen@CtxStack{%
    {\enparen@ctx}{\enparen@stack}%
    \enparen@CtxStack
  }%
  \xdef\enparen@ctx{#1}%
  \global\let\enparen@stack\ltx@empty
}
%    \end{macrocode}
%    \end{macro}
%
%    \begin{macro}{\enparen@CtxStackPop}
%    \begin{macrocode}
\def\enparen@CtxStackPop{%
  \ifx\enparen@CtxStack\ltx@empty
    \PackageWarning{enparen}{%
      Pop request for empty context stack%
    }%
    \global\let\enparen@ctx\enparenContextDefault
    \global\let\enparen@stack\ltx@empty
  \else
    \xdef\enparen@ctx{%
      \expandafter\ltx@car\enparen@CtxStack\@nil
    }%
    \xdef\enparen@stack{%
      \expandafter\ltx@carsecond\enparen@CtxStack\@nil
    }%
    \xdef\enparen@CtxStack{%
      \expandafter\ltx@cdrtwo\enparen@CtxStack\@nil
    }%
  \fi
}
%    \end{macrocode}
%    \end{macro}
%
% \subsubsection{Context user macros}
%
%    \begin{macro}{\enparenBeginContext}
%    \begin{macrocode}
\ProtectedDef*{\enparenBeginContext}[1]{%
  \enparen@CtxStackPush{#1}%
}
%    \end{macrocode}
%    \end{macro}
%    \begin{macro}{\enparenEndContext}
%    \begin{macrocode}
\ProtectedDef*{\enparenEndContext}[1]{%
  \edef\enparen@temp{#1}%
  \ifx\enparen@temp\enparen@ctx
  \else
    \PackageWarning{enparen}{%
      Context mismatch in end request.\MessageBreak
      `#1' should be ended, but current context\MessageBreak
      is `\enparen@ctx'%
    }%
  \fi
  \enparenCheckEmptyStack
  \enparen@CtxStackPop
}
%    \end{macrocode}
%    \end{macro}
%
%    \begin{macro}{\enparenCheckEmptyStack}
%    \begin{macrocode}
\ProtectedDef*{\enparenCheckEmptyStack}[0]{%
  \ifx\enparen@stack\ltx@empty
  \else
    \PackageWarning{enparen}{%
      Ending non-empty stack `\enparen@ctx':\MessageBreak
      \enparen@PrintStack\MessageBreak
    }%
  \fi
}
%    \end{macrocode}
%    \end{macro}
%
%    \begin{macro}{\enparen@PrintStack}
%    \begin{macrocode}
\def\enparen@PrintStack{%
  \expandafter\enparen@PrintStackAux
  \enparen@stack\ltx@empty\ltx@empty
}
%    \end{macrocode}
%    \end{macro}
%    \begin{macro}{\enparen@PrintStackAux}
%    \begin{macrocode}
\def\enparen@PrintStackAux#1#2{%
  \ifx\ltx@empty#1%
  \else
    {#1:#2}%
    \expandafter\enparen@PrintStackAux
  \fi
}
%    \end{macrocode}
%    \end{macro}
%
% \subsection{Symbols}
%
%    \begin{macro}{\enparenSetSymbols}
%    \begin{macrocode}
\ProtectedDef*{\enparenSetSymbols}[3]{%
  \expandafter
  \def\csname enparen@symbol\the\numexpr#1L\endcsname{#2}%
  \expandafter
  \def\csname enparen@symbol\the\numexpr#1R\endcsname{#3}%
}
%    \end{macrocode}
%    \end{macro}
%    \begin{macro}{\enparenUnsetSymbols}
%    \begin{macrocode}
\ProtectedDef*{\enparenUnsetSymbols}[1]{%
  \expandafter
  \let\csname enparen@symbol\the\numexpr#1L\endcsname\ltx@undefined
  \expandafter
  \let\csname enparen@symbol\the\numexpr#1R\endcsname\ltx@undefined
}
%    \end{macrocode}
%    \end{macro}
%
%    \begin{macrocode}
\enparenSetSymbols{0}{(}{)}
\enparenSetSymbols{1}{(}{)}
\enparenSetSymbols{2}{[}{]}
\enparenSetSymbols{3}{\{}{\}}
\enparenSetSymbols{4}{\ensuremath{\langle}}{\ensuremath{\rangle}}
%    \end{macrocode}
%
% \subsection{Main user macros}
%
%    \begin{macro}{\enparen}
%    \begin{macrocode}
\ProtectedDef{\enparen}[1]{%
  \enparenLeft#1\enparenRight
}
%    \end{macrocode}
%    \end{macro}
%
%    \begin{macro}{\enparenLeft}
%    \begin{macrocode}
\ProtectedDef*{\enparenLeft}[0]{%
  \UniqueCounterCall{enparen}\enparen@Left
}
%    \end{macrocode}
%    \end{macro}
%    \begin{macro}{\enparen@Left}
%    \begin{macrocode}
\def\enparen@Left#1{%
  \xdef\enparen@stack{%
    {#1}{1}%
    \expandafter\enparen@Inc\expandafter2\expandafter!%
    \enparen@stack\ltx@empty\ltx@empty
  }%
  \edef\enparen@tmp{\zref@extract{enparen#1}{enparen}}%
  \ltx@IfUndefined{enparen@symbol\enparen@tmp L}{%
    \PackageWarning{enparen}{%
      Undefined symbols for level \enparen@tmp
    }%
    \csname enparen@symbol0L\endcsname
  }{%
    \csname enparen@symbol\enparen@tmp L\endcsname
  }%
}
%    \end{macrocode}
%    \end{macro}
%    \begin{macro}{\enparen@Inc}
%    \begin{macrocode}
\def\enparen@Inc#1!#2#3{%
  \ifx\ltx@empty#2%
  \else
    \ifnum#3<#1 %
      {#2}{#1}%
      \expandafter\enparen@Inc
      \the\numexpr#1+1\expandafter\expandafter\expandafter!%
    \else
      {#2}{#3}%
    \fi
  \fi
}
%    \end{macrocode}
%    \end{macro}
%
%    \begin{macro}{\enparenRight}
%    \begin{macrocode}
\ProtectedDef*{\enparenRight}[0]{%
  \ifx\enparen@stack\ltx@empty
    \PackageWarning{enparen}{%
      Missing left symbol for right symbol%
    }%
    \csname enparen@symbol0R\endcsname
  \else
    \expandafter\enparen@Right\enparen@stack\@nil
  \fi
}
%    \end{macrocode}
%    \end{macro}
%
%    \begin{macro}{\enparen@Right}
%    \begin{macrocode}
\def\enparen@Right#1#2#3\@nil{%
  \ltx@IfUndefined{%
    enparen@symbol%
    \zref@extract{enparen#1}{enparen}%
    R%
  }{%
    \csname enparen@symbol0R\endcsname
  }{%
    \csname
      enparen@symbol%
      \zref@extract{enparen#1}{enparen}%
       R%
    \endcsname
  }%
  \zref@wrapper@mainaux{%
    \zref@setcurrent{enparen}{#2}%
    \zref@labelbyprops{enparen#1}{enparen}%
  }%
  \xdef\enparen@stack{#3}%
}
%    \end{macrocode}
%    \end{macro}
%
% \subsection{Options}
%
%    \begin{macrocode}
\SetupKeyvalOptions{%
  family=enparen,%
  prefix=enparen@,%
}
%    \end{macrocode}
%
%    \begin{macro}{\enparenSetup}
%    \begin{macrocode}
\ProtectedDef*{\enparenSetup}[0]{%
  \kvsetkeys{enparen}%
}
%    \end{macrocode}
%    \end{macro}
%
%    \begin{macrocode}
\DeclareBoolOption[true]{footnote}
\DeclareBoolOption[true]{caption}
\ProcessKeyvalOptions*
%    \end{macrocode}
%
% \subsection{Context settings}
%
%    \begin{macro}{\enparen@AtBegin}
%    \begin{macrocode}
\def\enparen@AtBegin{%
  \ifenparen@footnote
    \let\enparen@org@makefntext\@makefntext
    \long\def\@makefntext##1{%
      \enparen@org@makefntext{%
        \enparenBeginContext{footnote}%
        ##1%
        \enparenEndContext{footnote}%
      }%
    }%
  \fi
  \enparen@Disable{footnote}%
  \ifenparen@caption
    \let\enparen@org@makecaption\@makecaption
    \long\def\@makecaption##1##2{%
      \enparen@org@makecaption{##1}{%
        \enparenBeginContext{caption}%
        ##2%
        \enparenEndContext{caption}%
      }%
    }%
  \fi
  \enparen@Disable{caption}%
}
%    \end{macrocode}
%    \end{macro}
%
%    \begin{macro}{\enparen@Disable}
%    \begin{macrocode}
\def\enparen@Disable#1{%
  \DisableKeyvalOption[%
    action=warning,%
    package=enparen,%
  ]{enparen}{#1}%
}
%    \end{macrocode}
%    \end{macro}
%
%    \begin{macrocode}
\AtBeginDocument{\enparen@AtBegin}
%    \end{macrocode}
%
% \subsection{At end of document}
%
%    \begin{macro}{\enparen@AtEnd}
%    \begin{macrocode}
\def\enparen@AtEnd{%
  \enparenCheckEmptyStack
  \ifx\enparen@CtxStack\ltx@empty
  \else
    \PackageWarningNoLine{enparen}{%
      Context stack is not empty at end of document.\MessageBreak
      Current stack and contents of context stack:\MessageBreak
      [\enparen@ctx]:[\enparen@PrintStack]%
      \expandafter
      \enparen@PrintContextStack\enparen@CtxStack\relax\relax
    }%
  \fi
}
%    \end{macrocode}
%    \end{macro}
%    \begin{macro}{\enparen@PrintContextStack}
%    \begin{macrocode}
\def\enparen@PrintContextStack#1#2{%
  \ifx\relax#1\ltx@empty
  \else
    \MessageBreak
    [#1]:[\enparen@PrintStackAux#2\ltx@empty\ltx@empty]% hash-ok
    \expandafter\enparen@PrintContextStack
  \fi
}
%    \end{macrocode}
%    \end{macro}
%    \begin{macrocode}
\AtVeryEndDocument{\enparen@AtEnd}
%    \end{macrocode}
%
%    \begin{macrocode}
%</package>
%    \end{macrocode}
%
% \section{Installation}
%
% \subsection{Download}
%
% \paragraph{Package.} This package is available on
% CTAN\footnote{\url{ftp://ftp.ctan.org/tex-archive/}}:
% \begin{description}
% \item[\CTAN{macros/latex/contrib/oberdiek/enparen.dtx}] The source file.
% \item[\CTAN{macros/latex/contrib/oberdiek/enparen.pdf}] Documentation.
% \end{description}
%
%
% \paragraph{Bundle.} All the packages of the bundle `oberdiek'
% are also available in a TDS compliant ZIP archive. There
% the packages are already unpacked and the documentation files
% are generated. The files and directories obey the TDS standard.
% \begin{description}
% \item[\CTAN{install/macros/latex/contrib/oberdiek.tds.zip}]
% \end{description}
% \emph{TDS} refers to the standard ``A Directory Structure
% for \TeX\ Files'' (\CTAN{tds/tds.pdf}). Directories
% with \xfile{texmf} in their name are usually organized this way.
%
% \subsection{Bundle installation}
%
% \paragraph{Unpacking.} Unpack the \xfile{oberdiek.tds.zip} in the
% TDS tree (also known as \xfile{texmf} tree) of your choice.
% Example (linux):
% \begin{quote}
%   |unzip oberdiek.tds.zip -d ~/texmf|
% \end{quote}
%
% \paragraph{Script installation.}
% Check the directory \xfile{TDS:scripts/oberdiek/} for
% scripts that need further installation steps.
% Package \xpackage{attachfile2} comes with the Perl script
% \xfile{pdfatfi.pl} that should be installed in such a way
% that it can be called as \texttt{pdfatfi}.
% Example (linux):
% \begin{quote}
%   |chmod +x scripts/oberdiek/pdfatfi.pl|\\
%   |cp scripts/oberdiek/pdfatfi.pl /usr/local/bin/|
% \end{quote}
%
% \subsection{Package installation}
%
% \paragraph{Unpacking.} The \xfile{.dtx} file is a self-extracting
% \docstrip\ archive. The files are extracted by running the
% \xfile{.dtx} through \plainTeX:
% \begin{quote}
%   \verb|tex enparen.dtx|
% \end{quote}
%
% \paragraph{TDS.} Now the different files must be moved into
% the different directories in your installation TDS tree
% (also known as \xfile{texmf} tree):
% \begin{quote}
% \def\t{^^A
% \begin{tabular}{@{}>{\ttfamily}l@{ $\rightarrow$ }>{\ttfamily}l@{}}
%   enparen.sty & tex/latex/oberdiek/enparen.sty\\
%   enparen.pdf & doc/latex/oberdiek/enparen.pdf\\
%   enparen.dtx & source/latex/oberdiek/enparen.dtx\\
% \end{tabular}^^A
% }^^A
% \sbox0{\t}^^A
% \ifdim\wd0>\linewidth
%   \begingroup
%     \advance\linewidth by\leftmargin
%     \advance\linewidth by\rightmargin
%   \edef\x{\endgroup
%     \def\noexpand\lw{\the\linewidth}^^A
%   }\x
%   \def\lwbox{^^A
%     \leavevmode
%     \hbox to \linewidth{^^A
%       \kern-\leftmargin\relax
%       \hss
%       \usebox0
%       \hss
%       \kern-\rightmargin\relax
%     }^^A
%   }^^A
%   \ifdim\wd0>\lw
%     \sbox0{\small\t}^^A
%     \ifdim\wd0>\linewidth
%       \ifdim\wd0>\lw
%         \sbox0{\footnotesize\t}^^A
%         \ifdim\wd0>\linewidth
%           \ifdim\wd0>\lw
%             \sbox0{\scriptsize\t}^^A
%             \ifdim\wd0>\linewidth
%               \ifdim\wd0>\lw
%                 \sbox0{\tiny\t}^^A
%                 \ifdim\wd0>\linewidth
%                   \lwbox
%                 \else
%                   \usebox0
%                 \fi
%               \else
%                 \lwbox
%               \fi
%             \else
%               \usebox0
%             \fi
%           \else
%             \lwbox
%           \fi
%         \else
%           \usebox0
%         \fi
%       \else
%         \lwbox
%       \fi
%     \else
%       \usebox0
%     \fi
%   \else
%     \lwbox
%   \fi
% \else
%   \usebox0
% \fi
% \end{quote}
% If you have a \xfile{docstrip.cfg} that configures and enables \docstrip's
% TDS installing feature, then some files can already be in the right
% place, see the documentation of \docstrip.
%
% \subsection{Refresh file name databases}
%
% If your \TeX~distribution
% (\teTeX, \mikTeX, \dots) relies on file name databases, you must refresh
% these. For example, \teTeX\ users run \verb|texhash| or
% \verb|mktexlsr|.
%
% \subsection{Some details for the interested}
%
% \paragraph{Attached source.}
%
% The PDF documentation on CTAN also includes the
% \xfile{.dtx} source file. It can be extracted by
% AcrobatReader 6 or higher. Another option is \textsf{pdftk},
% e.g. unpack the file into the current directory:
% \begin{quote}
%   \verb|pdftk enparen.pdf unpack_files output .|
% \end{quote}
%
% \paragraph{Unpacking with \LaTeX.}
% The \xfile{.dtx} chooses its action depending on the format:
% \begin{description}
% \item[\plainTeX:] Run \docstrip\ and extract the files.
% \item[\LaTeX:] Generate the documentation.
% \end{description}
% If you insist on using \LaTeX\ for \docstrip\ (really,
% \docstrip\ does not need \LaTeX), then inform the autodetect routine
% about your intention:
% \begin{quote}
%   \verb|latex \let\install=y% \iffalse meta-comment
%
% File: enparen.dtx
% Version: 2012/01/07 v1.0
% Info: Parentheses nesting
%
% Copyright (C) 2012 by
%    Heiko Oberdiek <heiko.oberdiek at googlemail.com>
%
% This work may be distributed and/or modified under the
% conditions of the LaTeX Project Public License, either
% version 1.3c of this license or (at your option) any later
% version. This version of this license is in
%    http://www.latex-project.org/lppl/lppl-1-3c.txt
% and the latest version of this license is in
%    http://www.latex-project.org/lppl.txt
% and version 1.3 or later is part of all distributions of
% LaTeX version 2005/12/01 or later.
%
% This work has the LPPL maintenance status "maintained".
%
% This Current Maintainer of this work is Heiko Oberdiek.
%
% This work consists of the main source file enparen.dtx
% and the derived files
%    enparen.sty, enparen.pdf, enparen.ins, enparen.drv.
%
% Distribution:
%    CTAN:macros/latex/contrib/oberdiek/enparen.dtx
%    CTAN:macros/latex/contrib/oberdiek/enparen.pdf
%
% Unpacking:
%    (a) If enparen.ins is present:
%           tex enparen.ins
%    (b) Without enparen.ins:
%           tex enparen.dtx
%    (c) If you insist on using LaTeX
%           latex \let\install=y\input{enparen.dtx}
%        (quote the arguments according to the demands of your shell)
%
% Documentation:
%    (a) If enparen.drv is present:
%           latex enparen.drv
%    (b) Without enparen.drv:
%           latex enparen.dtx; ...
%    The class ltxdoc loads the configuration file ltxdoc.cfg
%    if available. Here you can specify further options, e.g.
%    use A4 as paper format:
%       \PassOptionsToClass{a4paper}{article}
%
%    Programm calls to get the documentation (example):
%       pdflatex enparen.dtx
%       makeindex -s gind.ist enparen.idx
%       pdflatex enparen.dtx
%       makeindex -s gind.ist enparen.idx
%       pdflatex enparen.dtx
%
% Installation:
%    TDS:tex/latex/oberdiek/enparen.sty
%    TDS:doc/latex/oberdiek/enparen.pdf
%    TDS:source/latex/oberdiek/enparen.dtx
%
%<*ignore>
\begingroup
  \catcode123=1 %
  \catcode125=2 %
  \def\x{LaTeX2e}%
\expandafter\endgroup
\ifcase 0\ifx\install y1\fi\expandafter
         \ifx\csname processbatchFile\endcsname\relax\else1\fi
         \ifx\fmtname\x\else 1\fi\relax
\else\csname fi\endcsname
%</ignore>
%<*install>
\input docstrip.tex
\Msg{************************************************************************}
\Msg{* Installation}
\Msg{* Package: enparen 2012/01/07 v1.0 Parentheses nesting (HO)}
\Msg{************************************************************************}

\keepsilent
\askforoverwritefalse

\let\MetaPrefix\relax
\preamble

This is a generated file.

Project: enparen
Version: 2012/01/07 v1.0

Copyright (C) 2012 by
   Heiko Oberdiek <heiko.oberdiek at googlemail.com>

This work may be distributed and/or modified under the
conditions of the LaTeX Project Public License, either
version 1.3c of this license or (at your option) any later
version. This version of this license is in
   http://www.latex-project.org/lppl/lppl-1-3c.txt
and the latest version of this license is in
   http://www.latex-project.org/lppl.txt
and version 1.3 or later is part of all distributions of
LaTeX version 2005/12/01 or later.

This work has the LPPL maintenance status "maintained".

This Current Maintainer of this work is Heiko Oberdiek.

This work consists of the main source file enparen.dtx
and the derived files
   enparen.sty, enparen.pdf, enparen.ins, enparen.drv.

\endpreamble
\let\MetaPrefix\DoubleperCent

\generate{%
  \file{enparen.ins}{\from{enparen.dtx}{install}}%
  \file{enparen.drv}{\from{enparen.dtx}{driver}}%
  \usedir{tex/latex/oberdiek}%
  \file{enparen.sty}{\from{enparen.dtx}{package}}%
}

\catcode32=13\relax% active space
\let =\space%
\Msg{************************************************************************}
\Msg{*}
\Msg{* To finish the installation you have to move the following}
\Msg{* file into a directory searched by TeX:}
\Msg{*}
\Msg{*     enparen.sty}
\Msg{*}
\Msg{* To produce the documentation run the file `enparen.drv'}
\Msg{* through LaTeX.}
\Msg{*}
\Msg{* Happy TeXing!}
\Msg{*}
\Msg{************************************************************************}

\endbatchfile
%</install>
%<*ignore>
\fi
%</ignore>
%<*driver>
\NeedsTeXFormat{LaTeX2e}
\ProvidesFile{enparen.drv}%
  [2012/01/07 v1.0 Parentheses nesting (HO)]%
\documentclass{ltxdoc}
\usepackage{holtxdoc}[2011/11/22]
\usepackage{enparen}[2012/01/07]
\begin{document}
  \DocInput{enparen.dtx}%
\end{document}
%</driver>
% \fi
%
% \CheckSum{339}
%
% \CharacterTable
%  {Upper-case    \A\B\C\D\E\F\G\H\I\J\K\L\M\N\O\P\Q\R\S\T\U\V\W\X\Y\Z
%   Lower-case    \a\b\c\d\e\f\g\h\i\j\k\l\m\n\o\p\q\r\s\t\u\v\w\x\y\z
%   Digits        \0\1\2\3\4\5\6\7\8\9
%   Exclamation   \!     Double quote  \"     Hash (number) \#
%   Dollar        \$     Percent       \%     Ampersand     \&
%   Acute accent  \'     Left paren    \(     Right paren   \)
%   Asterisk      \*     Plus          \+     Comma         \,
%   Minus         \-     Point         \.     Solidus       \/
%   Colon         \:     Semicolon     \;     Less than     \<
%   Equals        \=     Greater than  \>     Question mark \?
%   Commercial at \@     Left bracket  \[     Backslash     \\
%   Right bracket \]     Circumflex    \^     Underscore    \_
%   Grave accent  \`     Left brace    \{     Vertical bar  \|
%   Right brace   \}     Tilde         \~}
%
% \GetFileInfo{enparen.drv}
%
% \title{The \xpackage{enparen} package}
% \date{2012/01/07 v1.0}
% \author{Heiko Oberdiek\\\xemail{heiko.oberdiek at googlemail.com}}
%
% \maketitle
%
% \begin{abstract}
% The package defines macros to set parentheses that automatically
% change the symbols from inner to outer fences.
% \end{abstract}
%
% \tableofcontents
%
% \section{Documentation}
%
% The \hologo{LaTeX} package provides macros for automatically changed
% parentheses symbols depending on the fence order.
% The innermost parentheses are always using the same
% parentheses. The symbols changes for the outer fences.
%
% Example:
% \begin{quote}
%   \enparen{a \enparen{b} \enparen{c \enparen{d}} \enparen{e}}
% \end{quote}
% is generated by
%\begin{quote}
%\begin{verbatim}
%\documentclass{article}
%\usepackage{enparen}
%\begin{document}
%  \enparen{a \enparen{b} \enparen{c \enparen{d}} \enparen{e}}
%\end{document}
%\end{verbatim}
%\end{quote}
%
% The package is the result of a newsgroup thread. Dan gives
% the following specification \cite{dan}:
% \begin{quote}
% ``On the other hand, the rules for fences are usually the reverse:
% innermost fences are always |()|, next outer are |[]|, etc. This means
% the opening fence has to wait until all the fences between it and
% the matching close have been detected before it can decide whether
% to be |(| or |[| or |{|.''
% \end{quote}
%
% The fence level counting starts from innermost parentheses with one.
% For the next outer fences the level is increased by one.
% The example above with level indexes:
% \begin{quote}
%   \def\1{\ensuremath{_{1}}}
%   \def\2{\ensuremath{_{2}}}
%   \def\3{\ensuremath{_{3}}}
%   \enparen{\3a \enparen{\1b}\1 \enparen{\2c \enparen{\1d}\1}\2 \enparen{e}}\3
% \end{quote}
% The correct level is only known at the closing symbol. Therefore the
% correct value is remembered in the main \xext{aux} file and used
% in the second \hologo{LaTeX} run.
%
% \subsection{User macros}
%
% \begin{declcs}{enparen} \M{text}
% \end{declcs}
% The macro \cs{enparen} puts its argument \meta{text} in parentheses.
% If the macro is nested, the used fence symbols change for the
% outer fences.
%
% \begin{declcs}{enparenLeft}\\
%   \cs{enparenRight}
% \end{declcs}
% Instead of |\enparen{text}| the left and right symbol can be
% used separately, but in pairs:
% \begin{quote}
%   |\enparenLeft text\enparenRight|
% \end{quote}
% \cs{enparenLeft} and \cs{enparenRight} may be used at different
% group levels, but they must be properly nested.
%
% \begin{declcs}{enparenSetSymbols} \M{level} \M{opening symbol} \M{closing symbol}
% \end{declcs}
% Macro \cs{enparenSetSymbols} configures the \meta{opening symbol} and
% \meta{closing symbol} for the \meta{level}. The \meta{level}
% is a number and the counting starts with one. Level zero is used,
% if the correct level is not known (e.g. in the first \hologo{LaTeX} run).
% The package defines the following sets:
%\begin{quote}
%\begin{verbatim}
%\enparenSetSymbols{0}{(}{)}
%\enparenSetSymbols{1}{(}{)}
%\enparenSetSymbols{2}{[}{]}
%\enparenSetSymbols{3}{\{}{\}}
%\end{verbatim}
%\end{quote}
% Example for changing the third and adding a fourth level:
%\begin{quote}
%\begin{verbatim}
%\enparenSetSymbols{3}{\ensuremath{\langle}}{\ensuremath{\rangle}}
%\enparenSetSymbols{4}{\{}{\}}
%\end{verbatim}
%\end{quote}
%
% \begin{declcs}{enparenUnsetSymbols} \M{level}
% \end{declcs}
% The symbols for level \meta{level} are removed.
% Example scenario: Only two nesting levels must be used,
% the package defines more, then the third level can be
% disabled by |\enparenUnsetSymbols{3}| and the user
% gets warnings if parentheses at level 3 are needed.
%
% \subsection{Contexts}
%
% \begin{declcs}{enparenBeginContext} \M{name}\\
%   \cs{enparenEndContext} \M{name}
% \end{declcs}
% If the current text is interrupted by footnotes, floats
% with captions, then the parentheses inside the text of footnotes,
% captions, \dots should restart from scratch. This can be achieved
% by embedding the text inside macros \cs{enparenBeginContext}
% and \cs{enparenEndContext}. These macros must be properly nested.
% The \meta{name} for the begin and end macro must be the same.
% It is a help for debugging problems, because the warning messages
% show the context name. But it is not necessary that the
% begin\slash end pairs have different names. Example:
%\begin{quote}
%\begin{verbatim}
%\enparenLeft text before table ...
%\begin{table}
%  \caption{Table caption}
%  \enparenBegin{Context}{table}
%  Other text \enparen{foobar}.
%  \enparenEnd{Context}{table}
%\end{table}
%text after table ...
%\enparenRight
%\end{verbatim}
%\end{quote}
% The parentheses inside the table environment and context `table'
% are not nested inside other parentheses: \enparen{foobar}.
% In case of captions and footnotes the contexts are automatically
% added, see next section about options.
%
% \subsection{Options}
%
% \begin{declcs}{enparenSetup} \M{key value list}
% \end{declcs}
% Some options (currently all) can also be set after the
% package is loaded. They can be set in the argument
% \meta{key value list} of \cs{enparenSetup}. Options are disabled
% after they are used the last time. Currently all options are boolean
% options and are disabled in |\begin{document}|.
%
% \begin{description}
% \item[\xoption{caption}:] The caption text is put in a
%   context |caption|.
% \item[\xoption{footnote}:] The footnote text is put in a
%   context |footnote|.
% \end{description}
%
% Example for disabling the two options at different places:
%\begin{quote}
%\begin{verbatim}
%\usepackage[caption=false]{enparen}
%\enparenSetup{footnote=false}
%\end{verbatim}
%\end{quote}
%
% \subsection{Notes}
%
% \begin{description}
% \item[Implicite kerning:]
%   Unexpandable stuff might affect the implicite kerning.
%   The package cannot avoid this, because it need to define
%   and redefine macros at the occurence of each symbol.
%   This is done before the opening and after the closing
%   symbol, thus that the implicite kerning inside
%   is not affected.
% \end{description}
%
% \StopEventually{
% }
%
% \section{Implementation}
%
%    \begin{macrocode}
%<*package>
\NeedsTeXFormat{LaTeX2e}
\ProvidesPackage{enparen}
  [2012/01/07 v1.0 Parentheses nesting (HO)]%
%    \end{macrocode}
%
% \subsection{Resources}
%
%    \begin{macrocode}
\RequirePackage{ltxcmds}[2011/11/09]
%    \end{macrocode}
%
%    \begin{macrocode}
\ltx@IfUndefined{numexpr}{%
  \PackageError{enparen}{%
    Missing e-TeX's \ltx@backslashchar numexpr.\MessageBreak
    The package will continue with emergency definitions%
  }\@ehc
  \def\enparenLeft{(}%
  \def\enparenRight{)}%
  \long\def\enparen#1{\enparenLeft#1\enparenRight}%
  \let\enparenSetup\ltx@gobble
  \let\enparenSetSymbols\ltx@gobblethree
  \let\enparenUnsetSymbols\ltx@gobble
  \endinput
}{}
%    \end{macrocode}
%
%    \begin{macrocode}
\RequirePackage{protecteddef}[2011/01/31]
\RequirePackage{atveryend}[2011/06/30]
\RequirePackage{uniquecounter}[2011/01/30]
\RequirePackage{zref-base}[2011/03/18]
\RequirePackage{kvoptions}[2011/06/30]
\RequirePackage{kvsetkeys}[2011/10/18]
%    \end{macrocode}
%
%    \begin{macro}{\zref@wrapper@mainaux}
%    \begin{macrocode}
\providecommand{\zref@wrapper@mainaux}[1]{%
  \ifx\@auxout\@mainaux
    #1%
  \else
    \begingroup
      \let\@auxout\@mainaux
      #1%
    \endgroup
  \fi
}
%    \end{macrocode}
%    \end{macro}
%
%    \begin{macrocode}
\UniqueCounterNew{enparen}
%    \end{macrocode}
%
%    \begin{macrocode}
\zref@newprop{enparen}[0]{}
%    \end{macrocode}
%
% \subsection{Contexts}
%
%    \begin{macro}{\enparenContextDefault}
%    \begin{macrocode}
\def\enparenContextDefault{main}
%    \end{macrocode}
%    \end{macro}
%
%    \begin{macro}{\enparen@ctx}
%    \begin{macrocode}
\let\enparen@ctx\ltx@empty
%    \end{macrocode}
%    \end{macro}
%
%    \begin{macro}{\enparen@stack}
%    \begin{macrocode}
\let\enparen@stack\ltx@empty
%    \end{macrocode}
%    \end{macro}
%
% \subsubsection{Stack for contexts}
%
%    \begin{macro}{\enparen@CtxStack}
%    \begin{macrocode}
\global\let\enparen@CtxStack\ltx@empty
%    \end{macrocode}
%    \end{macro}
%
%    \begin{macro}{\enparen@CtxStackPush}
%    \begin{macrocode}
\def\enparen@CtxStackPush#1{%
  \xdef\enparen@CtxStack{%
    {\enparen@ctx}{\enparen@stack}%
    \enparen@CtxStack
  }%
  \xdef\enparen@ctx{#1}%
  \global\let\enparen@stack\ltx@empty
}
%    \end{macrocode}
%    \end{macro}
%
%    \begin{macro}{\enparen@CtxStackPop}
%    \begin{macrocode}
\def\enparen@CtxStackPop{%
  \ifx\enparen@CtxStack\ltx@empty
    \PackageWarning{enparen}{%
      Pop request for empty context stack%
    }%
    \global\let\enparen@ctx\enparenContextDefault
    \global\let\enparen@stack\ltx@empty
  \else
    \xdef\enparen@ctx{%
      \expandafter\ltx@car\enparen@CtxStack\@nil
    }%
    \xdef\enparen@stack{%
      \expandafter\ltx@carsecond\enparen@CtxStack\@nil
    }%
    \xdef\enparen@CtxStack{%
      \expandafter\ltx@cdrtwo\enparen@CtxStack\@nil
    }%
  \fi
}
%    \end{macrocode}
%    \end{macro}
%
% \subsubsection{Context user macros}
%
%    \begin{macro}{\enparenBeginContext}
%    \begin{macrocode}
\ProtectedDef*{\enparenBeginContext}[1]{%
  \enparen@CtxStackPush{#1}%
}
%    \end{macrocode}
%    \end{macro}
%    \begin{macro}{\enparenEndContext}
%    \begin{macrocode}
\ProtectedDef*{\enparenEndContext}[1]{%
  \edef\enparen@temp{#1}%
  \ifx\enparen@temp\enparen@ctx
  \else
    \PackageWarning{enparen}{%
      Context mismatch in end request.\MessageBreak
      `#1' should be ended, but current context\MessageBreak
      is `\enparen@ctx'%
    }%
  \fi
  \enparenCheckEmptyStack
  \enparen@CtxStackPop
}
%    \end{macrocode}
%    \end{macro}
%
%    \begin{macro}{\enparenCheckEmptyStack}
%    \begin{macrocode}
\ProtectedDef*{\enparenCheckEmptyStack}[0]{%
  \ifx\enparen@stack\ltx@empty
  \else
    \PackageWarning{enparen}{%
      Ending non-empty stack `\enparen@ctx':\MessageBreak
      \enparen@PrintStack\MessageBreak
    }%
  \fi
}
%    \end{macrocode}
%    \end{macro}
%
%    \begin{macro}{\enparen@PrintStack}
%    \begin{macrocode}
\def\enparen@PrintStack{%
  \expandafter\enparen@PrintStackAux
  \enparen@stack\ltx@empty\ltx@empty
}
%    \end{macrocode}
%    \end{macro}
%    \begin{macro}{\enparen@PrintStackAux}
%    \begin{macrocode}
\def\enparen@PrintStackAux#1#2{%
  \ifx\ltx@empty#1%
  \else
    {#1:#2}%
    \expandafter\enparen@PrintStackAux
  \fi
}
%    \end{macrocode}
%    \end{macro}
%
% \subsection{Symbols}
%
%    \begin{macro}{\enparenSetSymbols}
%    \begin{macrocode}
\ProtectedDef*{\enparenSetSymbols}[3]{%
  \expandafter
  \def\csname enparen@symbol\the\numexpr#1L\endcsname{#2}%
  \expandafter
  \def\csname enparen@symbol\the\numexpr#1R\endcsname{#3}%
}
%    \end{macrocode}
%    \end{macro}
%    \begin{macro}{\enparenUnsetSymbols}
%    \begin{macrocode}
\ProtectedDef*{\enparenUnsetSymbols}[1]{%
  \expandafter
  \let\csname enparen@symbol\the\numexpr#1L\endcsname\ltx@undefined
  \expandafter
  \let\csname enparen@symbol\the\numexpr#1R\endcsname\ltx@undefined
}
%    \end{macrocode}
%    \end{macro}
%
%    \begin{macrocode}
\enparenSetSymbols{0}{(}{)}
\enparenSetSymbols{1}{(}{)}
\enparenSetSymbols{2}{[}{]}
\enparenSetSymbols{3}{\{}{\}}
\enparenSetSymbols{4}{\ensuremath{\langle}}{\ensuremath{\rangle}}
%    \end{macrocode}
%
% \subsection{Main user macros}
%
%    \begin{macro}{\enparen}
%    \begin{macrocode}
\ProtectedDef{\enparen}[1]{%
  \enparenLeft#1\enparenRight
}
%    \end{macrocode}
%    \end{macro}
%
%    \begin{macro}{\enparenLeft}
%    \begin{macrocode}
\ProtectedDef*{\enparenLeft}[0]{%
  \UniqueCounterCall{enparen}\enparen@Left
}
%    \end{macrocode}
%    \end{macro}
%    \begin{macro}{\enparen@Left}
%    \begin{macrocode}
\def\enparen@Left#1{%
  \xdef\enparen@stack{%
    {#1}{1}%
    \expandafter\enparen@Inc\expandafter2\expandafter!%
    \enparen@stack\ltx@empty\ltx@empty
  }%
  \edef\enparen@tmp{\zref@extract{enparen#1}{enparen}}%
  \ltx@IfUndefined{enparen@symbol\enparen@tmp L}{%
    \PackageWarning{enparen}{%
      Undefined symbols for level \enparen@tmp
    }%
    \csname enparen@symbol0L\endcsname
  }{%
    \csname enparen@symbol\enparen@tmp L\endcsname
  }%
}
%    \end{macrocode}
%    \end{macro}
%    \begin{macro}{\enparen@Inc}
%    \begin{macrocode}
\def\enparen@Inc#1!#2#3{%
  \ifx\ltx@empty#2%
  \else
    \ifnum#3<#1 %
      {#2}{#1}%
      \expandafter\enparen@Inc
      \the\numexpr#1+1\expandafter\expandafter\expandafter!%
    \else
      {#2}{#3}%
    \fi
  \fi
}
%    \end{macrocode}
%    \end{macro}
%
%    \begin{macro}{\enparenRight}
%    \begin{macrocode}
\ProtectedDef*{\enparenRight}[0]{%
  \ifx\enparen@stack\ltx@empty
    \PackageWarning{enparen}{%
      Missing left symbol for right symbol%
    }%
    \csname enparen@symbol0R\endcsname
  \else
    \expandafter\enparen@Right\enparen@stack\@nil
  \fi
}
%    \end{macrocode}
%    \end{macro}
%
%    \begin{macro}{\enparen@Right}
%    \begin{macrocode}
\def\enparen@Right#1#2#3\@nil{%
  \ltx@IfUndefined{%
    enparen@symbol%
    \zref@extract{enparen#1}{enparen}%
    R%
  }{%
    \csname enparen@symbol0R\endcsname
  }{%
    \csname
      enparen@symbol%
      \zref@extract{enparen#1}{enparen}%
       R%
    \endcsname
  }%
  \zref@wrapper@mainaux{%
    \zref@setcurrent{enparen}{#2}%
    \zref@labelbyprops{enparen#1}{enparen}%
  }%
  \xdef\enparen@stack{#3}%
}
%    \end{macrocode}
%    \end{macro}
%
% \subsection{Options}
%
%    \begin{macrocode}
\SetupKeyvalOptions{%
  family=enparen,%
  prefix=enparen@,%
}
%    \end{macrocode}
%
%    \begin{macro}{\enparenSetup}
%    \begin{macrocode}
\ProtectedDef*{\enparenSetup}[0]{%
  \kvsetkeys{enparen}%
}
%    \end{macrocode}
%    \end{macro}
%
%    \begin{macrocode}
\DeclareBoolOption[true]{footnote}
\DeclareBoolOption[true]{caption}
\ProcessKeyvalOptions*
%    \end{macrocode}
%
% \subsection{Context settings}
%
%    \begin{macro}{\enparen@AtBegin}
%    \begin{macrocode}
\def\enparen@AtBegin{%
  \ifenparen@footnote
    \let\enparen@org@makefntext\@makefntext
    \long\def\@makefntext##1{%
      \enparen@org@makefntext{%
        \enparenBeginContext{footnote}%
        ##1%
        \enparenEndContext{footnote}%
      }%
    }%
  \fi
  \enparen@Disable{footnote}%
  \ifenparen@caption
    \let\enparen@org@makecaption\@makecaption
    \long\def\@makecaption##1##2{%
      \enparen@org@makecaption{##1}{%
        \enparenBeginContext{caption}%
        ##2%
        \enparenEndContext{caption}%
      }%
    }%
  \fi
  \enparen@Disable{caption}%
}
%    \end{macrocode}
%    \end{macro}
%
%    \begin{macro}{\enparen@Disable}
%    \begin{macrocode}
\def\enparen@Disable#1{%
  \DisableKeyvalOption[%
    action=warning,%
    package=enparen,%
  ]{enparen}{#1}%
}
%    \end{macrocode}
%    \end{macro}
%
%    \begin{macrocode}
\AtBeginDocument{\enparen@AtBegin}
%    \end{macrocode}
%
% \subsection{At end of document}
%
%    \begin{macro}{\enparen@AtEnd}
%    \begin{macrocode}
\def\enparen@AtEnd{%
  \enparenCheckEmptyStack
  \ifx\enparen@CtxStack\ltx@empty
  \else
    \PackageWarningNoLine{enparen}{%
      Context stack is not empty at end of document.\MessageBreak
      Current stack and contents of context stack:\MessageBreak
      [\enparen@ctx]:[\enparen@PrintStack]%
      \expandafter
      \enparen@PrintContextStack\enparen@CtxStack\relax\relax
    }%
  \fi
}
%    \end{macrocode}
%    \end{macro}
%    \begin{macro}{\enparen@PrintContextStack}
%    \begin{macrocode}
\def\enparen@PrintContextStack#1#2{%
  \ifx\relax#1\ltx@empty
  \else
    \MessageBreak
    [#1]:[\enparen@PrintStackAux#2\ltx@empty\ltx@empty]% hash-ok
    \expandafter\enparen@PrintContextStack
  \fi
}
%    \end{macrocode}
%    \end{macro}
%    \begin{macrocode}
\AtVeryEndDocument{\enparen@AtEnd}
%    \end{macrocode}
%
%    \begin{macrocode}
%</package>
%    \end{macrocode}
%
% \section{Installation}
%
% \subsection{Download}
%
% \paragraph{Package.} This package is available on
% CTAN\footnote{\url{ftp://ftp.ctan.org/tex-archive/}}:
% \begin{description}
% \item[\CTAN{macros/latex/contrib/oberdiek/enparen.dtx}] The source file.
% \item[\CTAN{macros/latex/contrib/oberdiek/enparen.pdf}] Documentation.
% \end{description}
%
%
% \paragraph{Bundle.} All the packages of the bundle `oberdiek'
% are also available in a TDS compliant ZIP archive. There
% the packages are already unpacked and the documentation files
% are generated. The files and directories obey the TDS standard.
% \begin{description}
% \item[\CTAN{install/macros/latex/contrib/oberdiek.tds.zip}]
% \end{description}
% \emph{TDS} refers to the standard ``A Directory Structure
% for \TeX\ Files'' (\CTAN{tds/tds.pdf}). Directories
% with \xfile{texmf} in their name are usually organized this way.
%
% \subsection{Bundle installation}
%
% \paragraph{Unpacking.} Unpack the \xfile{oberdiek.tds.zip} in the
% TDS tree (also known as \xfile{texmf} tree) of your choice.
% Example (linux):
% \begin{quote}
%   |unzip oberdiek.tds.zip -d ~/texmf|
% \end{quote}
%
% \paragraph{Script installation.}
% Check the directory \xfile{TDS:scripts/oberdiek/} for
% scripts that need further installation steps.
% Package \xpackage{attachfile2} comes with the Perl script
% \xfile{pdfatfi.pl} that should be installed in such a way
% that it can be called as \texttt{pdfatfi}.
% Example (linux):
% \begin{quote}
%   |chmod +x scripts/oberdiek/pdfatfi.pl|\\
%   |cp scripts/oberdiek/pdfatfi.pl /usr/local/bin/|
% \end{quote}
%
% \subsection{Package installation}
%
% \paragraph{Unpacking.} The \xfile{.dtx} file is a self-extracting
% \docstrip\ archive. The files are extracted by running the
% \xfile{.dtx} through \plainTeX:
% \begin{quote}
%   \verb|tex enparen.dtx|
% \end{quote}
%
% \paragraph{TDS.} Now the different files must be moved into
% the different directories in your installation TDS tree
% (also known as \xfile{texmf} tree):
% \begin{quote}
% \def\t{^^A
% \begin{tabular}{@{}>{\ttfamily}l@{ $\rightarrow$ }>{\ttfamily}l@{}}
%   enparen.sty & tex/latex/oberdiek/enparen.sty\\
%   enparen.pdf & doc/latex/oberdiek/enparen.pdf\\
%   enparen.dtx & source/latex/oberdiek/enparen.dtx\\
% \end{tabular}^^A
% }^^A
% \sbox0{\t}^^A
% \ifdim\wd0>\linewidth
%   \begingroup
%     \advance\linewidth by\leftmargin
%     \advance\linewidth by\rightmargin
%   \edef\x{\endgroup
%     \def\noexpand\lw{\the\linewidth}^^A
%   }\x
%   \def\lwbox{^^A
%     \leavevmode
%     \hbox to \linewidth{^^A
%       \kern-\leftmargin\relax
%       \hss
%       \usebox0
%       \hss
%       \kern-\rightmargin\relax
%     }^^A
%   }^^A
%   \ifdim\wd0>\lw
%     \sbox0{\small\t}^^A
%     \ifdim\wd0>\linewidth
%       \ifdim\wd0>\lw
%         \sbox0{\footnotesize\t}^^A
%         \ifdim\wd0>\linewidth
%           \ifdim\wd0>\lw
%             \sbox0{\scriptsize\t}^^A
%             \ifdim\wd0>\linewidth
%               \ifdim\wd0>\lw
%                 \sbox0{\tiny\t}^^A
%                 \ifdim\wd0>\linewidth
%                   \lwbox
%                 \else
%                   \usebox0
%                 \fi
%               \else
%                 \lwbox
%               \fi
%             \else
%               \usebox0
%             \fi
%           \else
%             \lwbox
%           \fi
%         \else
%           \usebox0
%         \fi
%       \else
%         \lwbox
%       \fi
%     \else
%       \usebox0
%     \fi
%   \else
%     \lwbox
%   \fi
% \else
%   \usebox0
% \fi
% \end{quote}
% If you have a \xfile{docstrip.cfg} that configures and enables \docstrip's
% TDS installing feature, then some files can already be in the right
% place, see the documentation of \docstrip.
%
% \subsection{Refresh file name databases}
%
% If your \TeX~distribution
% (\teTeX, \mikTeX, \dots) relies on file name databases, you must refresh
% these. For example, \teTeX\ users run \verb|texhash| or
% \verb|mktexlsr|.
%
% \subsection{Some details for the interested}
%
% \paragraph{Attached source.}
%
% The PDF documentation on CTAN also includes the
% \xfile{.dtx} source file. It can be extracted by
% AcrobatReader 6 or higher. Another option is \textsf{pdftk},
% e.g. unpack the file into the current directory:
% \begin{quote}
%   \verb|pdftk enparen.pdf unpack_files output .|
% \end{quote}
%
% \paragraph{Unpacking with \LaTeX.}
% The \xfile{.dtx} chooses its action depending on the format:
% \begin{description}
% \item[\plainTeX:] Run \docstrip\ and extract the files.
% \item[\LaTeX:] Generate the documentation.
% \end{description}
% If you insist on using \LaTeX\ for \docstrip\ (really,
% \docstrip\ does not need \LaTeX), then inform the autodetect routine
% about your intention:
% \begin{quote}
%   \verb|latex \let\install=y\input{enparen.dtx}|
% \end{quote}
% Do not forget to quote the argument according to the demands
% of your shell.
%
% \paragraph{Generating the documentation.}
% You can use both the \xfile{.dtx} or the \xfile{.drv} to generate
% the documentation. The process can be configured by the
% configuration file \xfile{ltxdoc.cfg}. For instance, put this
% line into this file, if you want to have A4 as paper format:
% \begin{quote}
%   \verb|\PassOptionsToClass{a4paper}{article}|
% \end{quote}
% An example follows how to generate the
% documentation with pdf\LaTeX:
% \begin{quote}
%\begin{verbatim}
%pdflatex enparen.dtx
%makeindex -s gind.ist enparen.idx
%pdflatex enparen.dtx
%makeindex -s gind.ist enparen.idx
%pdflatex enparen.dtx
%\end{verbatim}
% \end{quote}
%
% \begin{thebibliography}{9}
% \raggedright
% \bibitem{dan}
%   Dan Luecking: \textit{Re: bracket order};
%   newsgroup \xnewsgroup{comp.text.tex};
%   message id \texttt{\small$\langle$9b07c9c8-ff92-4cbf-b3a9-84efecfeb506@l24g2000yqm.googlegroups.com$\rangle$}
%   2012-01-05.
%   \url{http://groups.google.com/group/comp.text.tex/msg/8774519da31c2352}
% \end{thebibliography}
%
% \begin{History}
%   \begin{Version}{2012/01/07 v1.0}
%   \item
%     First version.
%   \end{Version}
% \end{History}
%
% \PrintIndex
%
% \Finale
\endinput
|
% \end{quote}
% Do not forget to quote the argument according to the demands
% of your shell.
%
% \paragraph{Generating the documentation.}
% You can use both the \xfile{.dtx} or the \xfile{.drv} to generate
% the documentation. The process can be configured by the
% configuration file \xfile{ltxdoc.cfg}. For instance, put this
% line into this file, if you want to have A4 as paper format:
% \begin{quote}
%   \verb|\PassOptionsToClass{a4paper}{article}|
% \end{quote}
% An example follows how to generate the
% documentation with pdf\LaTeX:
% \begin{quote}
%\begin{verbatim}
%pdflatex enparen.dtx
%makeindex -s gind.ist enparen.idx
%pdflatex enparen.dtx
%makeindex -s gind.ist enparen.idx
%pdflatex enparen.dtx
%\end{verbatim}
% \end{quote}
%
% \begin{thebibliography}{9}
% \raggedright
% \bibitem{dan}
%   Dan Luecking: \textit{Re: bracket order};
%   newsgroup \xnewsgroup{comp.text.tex};
%   message id \texttt{\small$\langle$9b07c9c8-ff92-4cbf-b3a9-84efecfeb506@l24g2000yqm.googlegroups.com$\rangle$}
%   2012-01-05.
%   \url{http://groups.google.com/group/comp.text.tex/msg/8774519da31c2352}
% \end{thebibliography}
%
% \begin{History}
%   \begin{Version}{2012/01/07 v1.0}
%   \item
%     First version.
%   \end{Version}
% \end{History}
%
% \PrintIndex
%
% \Finale
\endinput

%        (quote the arguments according to the demands of your shell)
%
% Documentation:
%    (a) If enparen.drv is present:
%           latex enparen.drv
%    (b) Without enparen.drv:
%           latex enparen.dtx; ...
%    The class ltxdoc loads the configuration file ltxdoc.cfg
%    if available. Here you can specify further options, e.g.
%    use A4 as paper format:
%       \PassOptionsToClass{a4paper}{article}
%
%    Programm calls to get the documentation (example):
%       pdflatex enparen.dtx
%       makeindex -s gind.ist enparen.idx
%       pdflatex enparen.dtx
%       makeindex -s gind.ist enparen.idx
%       pdflatex enparen.dtx
%
% Installation:
%    TDS:tex/latex/oberdiek/enparen.sty
%    TDS:doc/latex/oberdiek/enparen.pdf
%    TDS:source/latex/oberdiek/enparen.dtx
%
%<*ignore>
\begingroup
  \catcode123=1 %
  \catcode125=2 %
  \def\x{LaTeX2e}%
\expandafter\endgroup
\ifcase 0\ifx\install y1\fi\expandafter
         \ifx\csname processbatchFile\endcsname\relax\else1\fi
         \ifx\fmtname\x\else 1\fi\relax
\else\csname fi\endcsname
%</ignore>
%<*install>
\input docstrip.tex
\Msg{************************************************************************}
\Msg{* Installation}
\Msg{* Package: enparen 2012/01/07 v1.0 Parentheses nesting (HO)}
\Msg{************************************************************************}

\keepsilent
\askforoverwritefalse

\let\MetaPrefix\relax
\preamble

This is a generated file.

Project: enparen
Version: 2012/01/07 v1.0

Copyright (C) 2012 by
   Heiko Oberdiek <heiko.oberdiek at googlemail.com>

This work may be distributed and/or modified under the
conditions of the LaTeX Project Public License, either
version 1.3c of this license or (at your option) any later
version. This version of this license is in
   http://www.latex-project.org/lppl/lppl-1-3c.txt
and the latest version of this license is in
   http://www.latex-project.org/lppl.txt
and version 1.3 or later is part of all distributions of
LaTeX version 2005/12/01 or later.

This work has the LPPL maintenance status "maintained".

This Current Maintainer of this work is Heiko Oberdiek.

This work consists of the main source file enparen.dtx
and the derived files
   enparen.sty, enparen.pdf, enparen.ins, enparen.drv.

\endpreamble
\let\MetaPrefix\DoubleperCent

\generate{%
  \file{enparen.ins}{\from{enparen.dtx}{install}}%
  \file{enparen.drv}{\from{enparen.dtx}{driver}}%
  \usedir{tex/latex/oberdiek}%
  \file{enparen.sty}{\from{enparen.dtx}{package}}%
}

\catcode32=13\relax% active space
\let =\space%
\Msg{************************************************************************}
\Msg{*}
\Msg{* To finish the installation you have to move the following}
\Msg{* file into a directory searched by TeX:}
\Msg{*}
\Msg{*     enparen.sty}
\Msg{*}
\Msg{* To produce the documentation run the file `enparen.drv'}
\Msg{* through LaTeX.}
\Msg{*}
\Msg{* Happy TeXing!}
\Msg{*}
\Msg{************************************************************************}

\endbatchfile
%</install>
%<*ignore>
\fi
%</ignore>
%<*driver>
\NeedsTeXFormat{LaTeX2e}
\ProvidesFile{enparen.drv}%
  [2012/01/07 v1.0 Parentheses nesting (HO)]%
\documentclass{ltxdoc}
\usepackage{holtxdoc}[2011/11/22]
\usepackage{enparen}[2012/01/07]
\begin{document}
  \DocInput{enparen.dtx}%
\end{document}
%</driver>
% \fi
%
% \CheckSum{339}
%
% \CharacterTable
%  {Upper-case    \A\B\C\D\E\F\G\H\I\J\K\L\M\N\O\P\Q\R\S\T\U\V\W\X\Y\Z
%   Lower-case    \a\b\c\d\e\f\g\h\i\j\k\l\m\n\o\p\q\r\s\t\u\v\w\x\y\z
%   Digits        \0\1\2\3\4\5\6\7\8\9
%   Exclamation   \!     Double quote  \"     Hash (number) \#
%   Dollar        \$     Percent       \%     Ampersand     \&
%   Acute accent  \'     Left paren    \(     Right paren   \)
%   Asterisk      \*     Plus          \+     Comma         \,
%   Minus         \-     Point         \.     Solidus       \/
%   Colon         \:     Semicolon     \;     Less than     \<
%   Equals        \=     Greater than  \>     Question mark \?
%   Commercial at \@     Left bracket  \[     Backslash     \\
%   Right bracket \]     Circumflex    \^     Underscore    \_
%   Grave accent  \`     Left brace    \{     Vertical bar  \|
%   Right brace   \}     Tilde         \~}
%
% \GetFileInfo{enparen.drv}
%
% \title{The \xpackage{enparen} package}
% \date{2012/01/07 v1.0}
% \author{Heiko Oberdiek\\\xemail{heiko.oberdiek at googlemail.com}}
%
% \maketitle
%
% \begin{abstract}
% The package defines macros to set parentheses that automatically
% change the symbols from inner to outer fences.
% \end{abstract}
%
% \tableofcontents
%
% \section{Documentation}
%
% The \hologo{LaTeX} package provides macros for automatically changed
% parentheses symbols depending on the fence order.
% The innermost parentheses are always using the same
% parentheses. The symbols changes for the outer fences.
%
% Example:
% \begin{quote}
%   \enparen{a \enparen{b} \enparen{c \enparen{d}} \enparen{e}}
% \end{quote}
% is generated by
%\begin{quote}
%\begin{verbatim}
%\documentclass{article}
%\usepackage{enparen}
%\begin{document}
%  \enparen{a \enparen{b} \enparen{c \enparen{d}} \enparen{e}}
%\end{document}
%\end{verbatim}
%\end{quote}
%
% The package is the result of a newsgroup thread. Dan gives
% the following specification \cite{dan}:
% \begin{quote}
% ``On the other hand, the rules for fences are usually the reverse:
% innermost fences are always |()|, next outer are |[]|, etc. This means
% the opening fence has to wait until all the fences between it and
% the matching close have been detected before it can decide whether
% to be |(| or |[| or |{|.''
% \end{quote}
%
% The fence level counting starts from innermost parentheses with one.
% For the next outer fences the level is increased by one.
% The example above with level indexes:
% \begin{quote}
%   \def\1{\ensuremath{_{1}}}
%   \def\2{\ensuremath{_{2}}}
%   \def\3{\ensuremath{_{3}}}
%   \enparen{\3a \enparen{\1b}\1 \enparen{\2c \enparen{\1d}\1}\2 \enparen{e}}\3
% \end{quote}
% The correct level is only known at the closing symbol. Therefore the
% correct value is remembered in the main \xext{aux} file and used
% in the second \hologo{LaTeX} run.
%
% \subsection{User macros}
%
% \begin{declcs}{enparen} \M{text}
% \end{declcs}
% The macro \cs{enparen} puts its argument \meta{text} in parentheses.
% If the macro is nested, the used fence symbols change for the
% outer fences.
%
% \begin{declcs}{enparenLeft}\\
%   \cs{enparenRight}
% \end{declcs}
% Instead of |\enparen{text}| the left and right symbol can be
% used separately, but in pairs:
% \begin{quote}
%   |\enparenLeft text\enparenRight|
% \end{quote}
% \cs{enparenLeft} and \cs{enparenRight} may be used at different
% group levels, but they must be properly nested.
%
% \begin{declcs}{enparenSetSymbols} \M{level} \M{opening symbol} \M{closing symbol}
% \end{declcs}
% Macro \cs{enparenSetSymbols} configures the \meta{opening symbol} and
% \meta{closing symbol} for the \meta{level}. The \meta{level}
% is a number and the counting starts with one. Level zero is used,
% if the correct level is not known (e.g. in the first \hologo{LaTeX} run).
% The package defines the following sets:
%\begin{quote}
%\begin{verbatim}
%\enparenSetSymbols{0}{(}{)}
%\enparenSetSymbols{1}{(}{)}
%\enparenSetSymbols{2}{[}{]}
%\enparenSetSymbols{3}{\{}{\}}
%\end{verbatim}
%\end{quote}
% Example for changing the third and adding a fourth level:
%\begin{quote}
%\begin{verbatim}
%\enparenSetSymbols{3}{\ensuremath{\langle}}{\ensuremath{\rangle}}
%\enparenSetSymbols{4}{\{}{\}}
%\end{verbatim}
%\end{quote}
%
% \begin{declcs}{enparenUnsetSymbols} \M{level}
% \end{declcs}
% The symbols for level \meta{level} are removed.
% Example scenario: Only two nesting levels must be used,
% the package defines more, then the third level can be
% disabled by |\enparenUnsetSymbols{3}| and the user
% gets warnings if parentheses at level 3 are needed.
%
% \subsection{Contexts}
%
% \begin{declcs}{enparenBeginContext} \M{name}\\
%   \cs{enparenEndContext} \M{name}
% \end{declcs}
% If the current text is interrupted by footnotes, floats
% with captions, then the parentheses inside the text of footnotes,
% captions, \dots should restart from scratch. This can be achieved
% by embedding the text inside macros \cs{enparenBeginContext}
% and \cs{enparenEndContext}. These macros must be properly nested.
% The \meta{name} for the begin and end macro must be the same.
% It is a help for debugging problems, because the warning messages
% show the context name. But it is not necessary that the
% begin\slash end pairs have different names. Example:
%\begin{quote}
%\begin{verbatim}
%\enparenLeft text before table ...
%\begin{table}
%  \caption{Table caption}
%  \enparenBegin{Context}{table}
%  Other text \enparen{foobar}.
%  \enparenEnd{Context}{table}
%\end{table}
%text after table ...
%\enparenRight
%\end{verbatim}
%\end{quote}
% The parentheses inside the table environment and context `table'
% are not nested inside other parentheses: \enparen{foobar}.
% In case of captions and footnotes the contexts are automatically
% added, see next section about options.
%
% \subsection{Options}
%
% \begin{declcs}{enparenSetup} \M{key value list}
% \end{declcs}
% Some options (currently all) can also be set after the
% package is loaded. They can be set in the argument
% \meta{key value list} of \cs{enparenSetup}. Options are disabled
% after they are used the last time. Currently all options are boolean
% options and are disabled in |\begin{document}|.
%
% \begin{description}
% \item[\xoption{caption}:] The caption text is put in a
%   context |caption|.
% \item[\xoption{footnote}:] The footnote text is put in a
%   context |footnote|.
% \end{description}
%
% Example for disabling the two options at different places:
%\begin{quote}
%\begin{verbatim}
%\usepackage[caption=false]{enparen}
%\enparenSetup{footnote=false}
%\end{verbatim}
%\end{quote}
%
% \subsection{Notes}
%
% \begin{description}
% \item[Implicite kerning:]
%   Unexpandable stuff might affect the implicite kerning.
%   The package cannot avoid this, because it need to define
%   and redefine macros at the occurence of each symbol.
%   This is done before the opening and after the closing
%   symbol, thus that the implicite kerning inside
%   is not affected.
% \end{description}
%
% \StopEventually{
% }
%
% \section{Implementation}
%
%    \begin{macrocode}
%<*package>
\NeedsTeXFormat{LaTeX2e}
\ProvidesPackage{enparen}
  [2012/01/07 v1.0 Parentheses nesting (HO)]%
%    \end{macrocode}
%
% \subsection{Resources}
%
%    \begin{macrocode}
\RequirePackage{ltxcmds}[2011/11/09]
%    \end{macrocode}
%
%    \begin{macrocode}
\ltx@IfUndefined{numexpr}{%
  \PackageError{enparen}{%
    Missing e-TeX's \ltx@backslashchar numexpr.\MessageBreak
    The package will continue with emergency definitions%
  }\@ehc
  \def\enparenLeft{(}%
  \def\enparenRight{)}%
  \long\def\enparen#1{\enparenLeft#1\enparenRight}%
  \let\enparenSetup\ltx@gobble
  \let\enparenSetSymbols\ltx@gobblethree
  \let\enparenUnsetSymbols\ltx@gobble
  \endinput
}{}
%    \end{macrocode}
%
%    \begin{macrocode}
\RequirePackage{protecteddef}[2011/01/31]
\RequirePackage{atveryend}[2011/06/30]
\RequirePackage{uniquecounter}[2011/01/30]
\RequirePackage{zref-base}[2011/03/18]
\RequirePackage{kvoptions}[2011/06/30]
\RequirePackage{kvsetkeys}[2011/10/18]
%    \end{macrocode}
%
%    \begin{macro}{\zref@wrapper@mainaux}
%    \begin{macrocode}
\providecommand{\zref@wrapper@mainaux}[1]{%
  \ifx\@auxout\@mainaux
    #1%
  \else
    \begingroup
      \let\@auxout\@mainaux
      #1%
    \endgroup
  \fi
}
%    \end{macrocode}
%    \end{macro}
%
%    \begin{macrocode}
\UniqueCounterNew{enparen}
%    \end{macrocode}
%
%    \begin{macrocode}
\zref@newprop{enparen}[0]{}
%    \end{macrocode}
%
% \subsection{Contexts}
%
%    \begin{macro}{\enparenContextDefault}
%    \begin{macrocode}
\def\enparenContextDefault{main}
%    \end{macrocode}
%    \end{macro}
%
%    \begin{macro}{\enparen@ctx}
%    \begin{macrocode}
\let\enparen@ctx\ltx@empty
%    \end{macrocode}
%    \end{macro}
%
%    \begin{macro}{\enparen@stack}
%    \begin{macrocode}
\let\enparen@stack\ltx@empty
%    \end{macrocode}
%    \end{macro}
%
% \subsubsection{Stack for contexts}
%
%    \begin{macro}{\enparen@CtxStack}
%    \begin{macrocode}
\global\let\enparen@CtxStack\ltx@empty
%    \end{macrocode}
%    \end{macro}
%
%    \begin{macro}{\enparen@CtxStackPush}
%    \begin{macrocode}
\def\enparen@CtxStackPush#1{%
  \xdef\enparen@CtxStack{%
    {\enparen@ctx}{\enparen@stack}%
    \enparen@CtxStack
  }%
  \xdef\enparen@ctx{#1}%
  \global\let\enparen@stack\ltx@empty
}
%    \end{macrocode}
%    \end{macro}
%
%    \begin{macro}{\enparen@CtxStackPop}
%    \begin{macrocode}
\def\enparen@CtxStackPop{%
  \ifx\enparen@CtxStack\ltx@empty
    \PackageWarning{enparen}{%
      Pop request for empty context stack%
    }%
    \global\let\enparen@ctx\enparenContextDefault
    \global\let\enparen@stack\ltx@empty
  \else
    \xdef\enparen@ctx{%
      \expandafter\ltx@car\enparen@CtxStack\@nil
    }%
    \xdef\enparen@stack{%
      \expandafter\ltx@carsecond\enparen@CtxStack\@nil
    }%
    \xdef\enparen@CtxStack{%
      \expandafter\ltx@cdrtwo\enparen@CtxStack\@nil
    }%
  \fi
}
%    \end{macrocode}
%    \end{macro}
%
% \subsubsection{Context user macros}
%
%    \begin{macro}{\enparenBeginContext}
%    \begin{macrocode}
\ProtectedDef*{\enparenBeginContext}[1]{%
  \enparen@CtxStackPush{#1}%
}
%    \end{macrocode}
%    \end{macro}
%    \begin{macro}{\enparenEndContext}
%    \begin{macrocode}
\ProtectedDef*{\enparenEndContext}[1]{%
  \edef\enparen@temp{#1}%
  \ifx\enparen@temp\enparen@ctx
  \else
    \PackageWarning{enparen}{%
      Context mismatch in end request.\MessageBreak
      `#1' should be ended, but current context\MessageBreak
      is `\enparen@ctx'%
    }%
  \fi
  \enparenCheckEmptyStack
  \enparen@CtxStackPop
}
%    \end{macrocode}
%    \end{macro}
%
%    \begin{macro}{\enparenCheckEmptyStack}
%    \begin{macrocode}
\ProtectedDef*{\enparenCheckEmptyStack}[0]{%
  \ifx\enparen@stack\ltx@empty
  \else
    \PackageWarning{enparen}{%
      Ending non-empty stack `\enparen@ctx':\MessageBreak
      \enparen@PrintStack\MessageBreak
    }%
  \fi
}
%    \end{macrocode}
%    \end{macro}
%
%    \begin{macro}{\enparen@PrintStack}
%    \begin{macrocode}
\def\enparen@PrintStack{%
  \expandafter\enparen@PrintStackAux
  \enparen@stack\ltx@empty\ltx@empty
}
%    \end{macrocode}
%    \end{macro}
%    \begin{macro}{\enparen@PrintStackAux}
%    \begin{macrocode}
\def\enparen@PrintStackAux#1#2{%
  \ifx\ltx@empty#1%
  \else
    {#1:#2}%
    \expandafter\enparen@PrintStackAux
  \fi
}
%    \end{macrocode}
%    \end{macro}
%
% \subsection{Symbols}
%
%    \begin{macro}{\enparenSetSymbols}
%    \begin{macrocode}
\ProtectedDef*{\enparenSetSymbols}[3]{%
  \expandafter
  \def\csname enparen@symbol\the\numexpr#1L\endcsname{#2}%
  \expandafter
  \def\csname enparen@symbol\the\numexpr#1R\endcsname{#3}%
}
%    \end{macrocode}
%    \end{macro}
%    \begin{macro}{\enparenUnsetSymbols}
%    \begin{macrocode}
\ProtectedDef*{\enparenUnsetSymbols}[1]{%
  \expandafter
  \let\csname enparen@symbol\the\numexpr#1L\endcsname\ltx@undefined
  \expandafter
  \let\csname enparen@symbol\the\numexpr#1R\endcsname\ltx@undefined
}
%    \end{macrocode}
%    \end{macro}
%
%    \begin{macrocode}
\enparenSetSymbols{0}{(}{)}
\enparenSetSymbols{1}{(}{)}
\enparenSetSymbols{2}{[}{]}
\enparenSetSymbols{3}{\{}{\}}
\enparenSetSymbols{4}{\ensuremath{\langle}}{\ensuremath{\rangle}}
%    \end{macrocode}
%
% \subsection{Main user macros}
%
%    \begin{macro}{\enparen}
%    \begin{macrocode}
\ProtectedDef{\enparen}[1]{%
  \enparenLeft#1\enparenRight
}
%    \end{macrocode}
%    \end{macro}
%
%    \begin{macro}{\enparenLeft}
%    \begin{macrocode}
\ProtectedDef*{\enparenLeft}[0]{%
  \UniqueCounterCall{enparen}\enparen@Left
}
%    \end{macrocode}
%    \end{macro}
%    \begin{macro}{\enparen@Left}
%    \begin{macrocode}
\def\enparen@Left#1{%
  \xdef\enparen@stack{%
    {#1}{1}%
    \expandafter\enparen@Inc\expandafter2\expandafter!%
    \enparen@stack\ltx@empty\ltx@empty
  }%
  \edef\enparen@tmp{\zref@extract{enparen#1}{enparen}}%
  \ltx@IfUndefined{enparen@symbol\enparen@tmp L}{%
    \PackageWarning{enparen}{%
      Undefined symbols for level \enparen@tmp
    }%
    \csname enparen@symbol0L\endcsname
  }{%
    \csname enparen@symbol\enparen@tmp L\endcsname
  }%
}
%    \end{macrocode}
%    \end{macro}
%    \begin{macro}{\enparen@Inc}
%    \begin{macrocode}
\def\enparen@Inc#1!#2#3{%
  \ifx\ltx@empty#2%
  \else
    \ifnum#3<#1 %
      {#2}{#1}%
      \expandafter\enparen@Inc
      \the\numexpr#1+1\expandafter\expandafter\expandafter!%
    \else
      {#2}{#3}%
    \fi
  \fi
}
%    \end{macrocode}
%    \end{macro}
%
%    \begin{macro}{\enparenRight}
%    \begin{macrocode}
\ProtectedDef*{\enparenRight}[0]{%
  \ifx\enparen@stack\ltx@empty
    \PackageWarning{enparen}{%
      Missing left symbol for right symbol%
    }%
    \csname enparen@symbol0R\endcsname
  \else
    \expandafter\enparen@Right\enparen@stack\@nil
  \fi
}
%    \end{macrocode}
%    \end{macro}
%
%    \begin{macro}{\enparen@Right}
%    \begin{macrocode}
\def\enparen@Right#1#2#3\@nil{%
  \ltx@IfUndefined{%
    enparen@symbol%
    \zref@extract{enparen#1}{enparen}%
    R%
  }{%
    \csname enparen@symbol0R\endcsname
  }{%
    \csname
      enparen@symbol%
      \zref@extract{enparen#1}{enparen}%
       R%
    \endcsname
  }%
  \zref@wrapper@mainaux{%
    \zref@setcurrent{enparen}{#2}%
    \zref@labelbyprops{enparen#1}{enparen}%
  }%
  \xdef\enparen@stack{#3}%
}
%    \end{macrocode}
%    \end{macro}
%
% \subsection{Options}
%
%    \begin{macrocode}
\SetupKeyvalOptions{%
  family=enparen,%
  prefix=enparen@,%
}
%    \end{macrocode}
%
%    \begin{macro}{\enparenSetup}
%    \begin{macrocode}
\ProtectedDef*{\enparenSetup}[0]{%
  \kvsetkeys{enparen}%
}
%    \end{macrocode}
%    \end{macro}
%
%    \begin{macrocode}
\DeclareBoolOption[true]{footnote}
\DeclareBoolOption[true]{caption}
\ProcessKeyvalOptions*
%    \end{macrocode}
%
% \subsection{Context settings}
%
%    \begin{macro}{\enparen@AtBegin}
%    \begin{macrocode}
\def\enparen@AtBegin{%
  \ifenparen@footnote
    \let\enparen@org@makefntext\@makefntext
    \long\def\@makefntext##1{%
      \enparen@org@makefntext{%
        \enparenBeginContext{footnote}%
        ##1%
        \enparenEndContext{footnote}%
      }%
    }%
  \fi
  \enparen@Disable{footnote}%
  \ifenparen@caption
    \let\enparen@org@makecaption\@makecaption
    \long\def\@makecaption##1##2{%
      \enparen@org@makecaption{##1}{%
        \enparenBeginContext{caption}%
        ##2%
        \enparenEndContext{caption}%
      }%
    }%
  \fi
  \enparen@Disable{caption}%
}
%    \end{macrocode}
%    \end{macro}
%
%    \begin{macro}{\enparen@Disable}
%    \begin{macrocode}
\def\enparen@Disable#1{%
  \DisableKeyvalOption[%
    action=warning,%
    package=enparen,%
  ]{enparen}{#1}%
}
%    \end{macrocode}
%    \end{macro}
%
%    \begin{macrocode}
\AtBeginDocument{\enparen@AtBegin}
%    \end{macrocode}
%
% \subsection{At end of document}
%
%    \begin{macro}{\enparen@AtEnd}
%    \begin{macrocode}
\def\enparen@AtEnd{%
  \enparenCheckEmptyStack
  \ifx\enparen@CtxStack\ltx@empty
  \else
    \PackageWarningNoLine{enparen}{%
      Context stack is not empty at end of document.\MessageBreak
      Current stack and contents of context stack:\MessageBreak
      [\enparen@ctx]:[\enparen@PrintStack]%
      \expandafter
      \enparen@PrintContextStack\enparen@CtxStack\relax\relax
    }%
  \fi
}
%    \end{macrocode}
%    \end{macro}
%    \begin{macro}{\enparen@PrintContextStack}
%    \begin{macrocode}
\def\enparen@PrintContextStack#1#2{%
  \ifx\relax#1\ltx@empty
  \else
    \MessageBreak
    [#1]:[\enparen@PrintStackAux#2\ltx@empty\ltx@empty]% hash-ok
    \expandafter\enparen@PrintContextStack
  \fi
}
%    \end{macrocode}
%    \end{macro}
%    \begin{macrocode}
\AtVeryEndDocument{\enparen@AtEnd}
%    \end{macrocode}
%
%    \begin{macrocode}
%</package>
%    \end{macrocode}
%
% \section{Installation}
%
% \subsection{Download}
%
% \paragraph{Package.} This package is available on
% CTAN\footnote{\url{ftp://ftp.ctan.org/tex-archive/}}:
% \begin{description}
% \item[\CTAN{macros/latex/contrib/oberdiek/enparen.dtx}] The source file.
% \item[\CTAN{macros/latex/contrib/oberdiek/enparen.pdf}] Documentation.
% \end{description}
%
%
% \paragraph{Bundle.} All the packages of the bundle `oberdiek'
% are also available in a TDS compliant ZIP archive. There
% the packages are already unpacked and the documentation files
% are generated. The files and directories obey the TDS standard.
% \begin{description}
% \item[\CTAN{install/macros/latex/contrib/oberdiek.tds.zip}]
% \end{description}
% \emph{TDS} refers to the standard ``A Directory Structure
% for \TeX\ Files'' (\CTAN{tds/tds.pdf}). Directories
% with \xfile{texmf} in their name are usually organized this way.
%
% \subsection{Bundle installation}
%
% \paragraph{Unpacking.} Unpack the \xfile{oberdiek.tds.zip} in the
% TDS tree (also known as \xfile{texmf} tree) of your choice.
% Example (linux):
% \begin{quote}
%   |unzip oberdiek.tds.zip -d ~/texmf|
% \end{quote}
%
% \paragraph{Script installation.}
% Check the directory \xfile{TDS:scripts/oberdiek/} for
% scripts that need further installation steps.
% Package \xpackage{attachfile2} comes with the Perl script
% \xfile{pdfatfi.pl} that should be installed in such a way
% that it can be called as \texttt{pdfatfi}.
% Example (linux):
% \begin{quote}
%   |chmod +x scripts/oberdiek/pdfatfi.pl|\\
%   |cp scripts/oberdiek/pdfatfi.pl /usr/local/bin/|
% \end{quote}
%
% \subsection{Package installation}
%
% \paragraph{Unpacking.} The \xfile{.dtx} file is a self-extracting
% \docstrip\ archive. The files are extracted by running the
% \xfile{.dtx} through \plainTeX:
% \begin{quote}
%   \verb|tex enparen.dtx|
% \end{quote}
%
% \paragraph{TDS.} Now the different files must be moved into
% the different directories in your installation TDS tree
% (also known as \xfile{texmf} tree):
% \begin{quote}
% \def\t{^^A
% \begin{tabular}{@{}>{\ttfamily}l@{ $\rightarrow$ }>{\ttfamily}l@{}}
%   enparen.sty & tex/latex/oberdiek/enparen.sty\\
%   enparen.pdf & doc/latex/oberdiek/enparen.pdf\\
%   enparen.dtx & source/latex/oberdiek/enparen.dtx\\
% \end{tabular}^^A
% }^^A
% \sbox0{\t}^^A
% \ifdim\wd0>\linewidth
%   \begingroup
%     \advance\linewidth by\leftmargin
%     \advance\linewidth by\rightmargin
%   \edef\x{\endgroup
%     \def\noexpand\lw{\the\linewidth}^^A
%   }\x
%   \def\lwbox{^^A
%     \leavevmode
%     \hbox to \linewidth{^^A
%       \kern-\leftmargin\relax
%       \hss
%       \usebox0
%       \hss
%       \kern-\rightmargin\relax
%     }^^A
%   }^^A
%   \ifdim\wd0>\lw
%     \sbox0{\small\t}^^A
%     \ifdim\wd0>\linewidth
%       \ifdim\wd0>\lw
%         \sbox0{\footnotesize\t}^^A
%         \ifdim\wd0>\linewidth
%           \ifdim\wd0>\lw
%             \sbox0{\scriptsize\t}^^A
%             \ifdim\wd0>\linewidth
%               \ifdim\wd0>\lw
%                 \sbox0{\tiny\t}^^A
%                 \ifdim\wd0>\linewidth
%                   \lwbox
%                 \else
%                   \usebox0
%                 \fi
%               \else
%                 \lwbox
%               \fi
%             \else
%               \usebox0
%             \fi
%           \else
%             \lwbox
%           \fi
%         \else
%           \usebox0
%         \fi
%       \else
%         \lwbox
%       \fi
%     \else
%       \usebox0
%     \fi
%   \else
%     \lwbox
%   \fi
% \else
%   \usebox0
% \fi
% \end{quote}
% If you have a \xfile{docstrip.cfg} that configures and enables \docstrip's
% TDS installing feature, then some files can already be in the right
% place, see the documentation of \docstrip.
%
% \subsection{Refresh file name databases}
%
% If your \TeX~distribution
% (\teTeX, \mikTeX, \dots) relies on file name databases, you must refresh
% these. For example, \teTeX\ users run \verb|texhash| or
% \verb|mktexlsr|.
%
% \subsection{Some details for the interested}
%
% \paragraph{Attached source.}
%
% The PDF documentation on CTAN also includes the
% \xfile{.dtx} source file. It can be extracted by
% AcrobatReader 6 or higher. Another option is \textsf{pdftk},
% e.g. unpack the file into the current directory:
% \begin{quote}
%   \verb|pdftk enparen.pdf unpack_files output .|
% \end{quote}
%
% \paragraph{Unpacking with \LaTeX.}
% The \xfile{.dtx} chooses its action depending on the format:
% \begin{description}
% \item[\plainTeX:] Run \docstrip\ and extract the files.
% \item[\LaTeX:] Generate the documentation.
% \end{description}
% If you insist on using \LaTeX\ for \docstrip\ (really,
% \docstrip\ does not need \LaTeX), then inform the autodetect routine
% about your intention:
% \begin{quote}
%   \verb|latex \let\install=y% \iffalse meta-comment
%
% File: enparen.dtx
% Version: 2012/01/07 v1.0
% Info: Parentheses nesting
%
% Copyright (C) 2012 by
%    Heiko Oberdiek <heiko.oberdiek at googlemail.com>
%
% This work may be distributed and/or modified under the
% conditions of the LaTeX Project Public License, either
% version 1.3c of this license or (at your option) any later
% version. This version of this license is in
%    http://www.latex-project.org/lppl/lppl-1-3c.txt
% and the latest version of this license is in
%    http://www.latex-project.org/lppl.txt
% and version 1.3 or later is part of all distributions of
% LaTeX version 2005/12/01 or later.
%
% This work has the LPPL maintenance status "maintained".
%
% This Current Maintainer of this work is Heiko Oberdiek.
%
% This work consists of the main source file enparen.dtx
% and the derived files
%    enparen.sty, enparen.pdf, enparen.ins, enparen.drv.
%
% Distribution:
%    CTAN:macros/latex/contrib/oberdiek/enparen.dtx
%    CTAN:macros/latex/contrib/oberdiek/enparen.pdf
%
% Unpacking:
%    (a) If enparen.ins is present:
%           tex enparen.ins
%    (b) Without enparen.ins:
%           tex enparen.dtx
%    (c) If you insist on using LaTeX
%           latex \let\install=y% \iffalse meta-comment
%
% File: enparen.dtx
% Version: 2012/01/07 v1.0
% Info: Parentheses nesting
%
% Copyright (C) 2012 by
%    Heiko Oberdiek <heiko.oberdiek at googlemail.com>
%
% This work may be distributed and/or modified under the
% conditions of the LaTeX Project Public License, either
% version 1.3c of this license or (at your option) any later
% version. This version of this license is in
%    http://www.latex-project.org/lppl/lppl-1-3c.txt
% and the latest version of this license is in
%    http://www.latex-project.org/lppl.txt
% and version 1.3 or later is part of all distributions of
% LaTeX version 2005/12/01 or later.
%
% This work has the LPPL maintenance status "maintained".
%
% This Current Maintainer of this work is Heiko Oberdiek.
%
% This work consists of the main source file enparen.dtx
% and the derived files
%    enparen.sty, enparen.pdf, enparen.ins, enparen.drv.
%
% Distribution:
%    CTAN:macros/latex/contrib/oberdiek/enparen.dtx
%    CTAN:macros/latex/contrib/oberdiek/enparen.pdf
%
% Unpacking:
%    (a) If enparen.ins is present:
%           tex enparen.ins
%    (b) Without enparen.ins:
%           tex enparen.dtx
%    (c) If you insist on using LaTeX
%           latex \let\install=y\input{enparen.dtx}
%        (quote the arguments according to the demands of your shell)
%
% Documentation:
%    (a) If enparen.drv is present:
%           latex enparen.drv
%    (b) Without enparen.drv:
%           latex enparen.dtx; ...
%    The class ltxdoc loads the configuration file ltxdoc.cfg
%    if available. Here you can specify further options, e.g.
%    use A4 as paper format:
%       \PassOptionsToClass{a4paper}{article}
%
%    Programm calls to get the documentation (example):
%       pdflatex enparen.dtx
%       makeindex -s gind.ist enparen.idx
%       pdflatex enparen.dtx
%       makeindex -s gind.ist enparen.idx
%       pdflatex enparen.dtx
%
% Installation:
%    TDS:tex/latex/oberdiek/enparen.sty
%    TDS:doc/latex/oberdiek/enparen.pdf
%    TDS:source/latex/oberdiek/enparen.dtx
%
%<*ignore>
\begingroup
  \catcode123=1 %
  \catcode125=2 %
  \def\x{LaTeX2e}%
\expandafter\endgroup
\ifcase 0\ifx\install y1\fi\expandafter
         \ifx\csname processbatchFile\endcsname\relax\else1\fi
         \ifx\fmtname\x\else 1\fi\relax
\else\csname fi\endcsname
%</ignore>
%<*install>
\input docstrip.tex
\Msg{************************************************************************}
\Msg{* Installation}
\Msg{* Package: enparen 2012/01/07 v1.0 Parentheses nesting (HO)}
\Msg{************************************************************************}

\keepsilent
\askforoverwritefalse

\let\MetaPrefix\relax
\preamble

This is a generated file.

Project: enparen
Version: 2012/01/07 v1.0

Copyright (C) 2012 by
   Heiko Oberdiek <heiko.oberdiek at googlemail.com>

This work may be distributed and/or modified under the
conditions of the LaTeX Project Public License, either
version 1.3c of this license or (at your option) any later
version. This version of this license is in
   http://www.latex-project.org/lppl/lppl-1-3c.txt
and the latest version of this license is in
   http://www.latex-project.org/lppl.txt
and version 1.3 or later is part of all distributions of
LaTeX version 2005/12/01 or later.

This work has the LPPL maintenance status "maintained".

This Current Maintainer of this work is Heiko Oberdiek.

This work consists of the main source file enparen.dtx
and the derived files
   enparen.sty, enparen.pdf, enparen.ins, enparen.drv.

\endpreamble
\let\MetaPrefix\DoubleperCent

\generate{%
  \file{enparen.ins}{\from{enparen.dtx}{install}}%
  \file{enparen.drv}{\from{enparen.dtx}{driver}}%
  \usedir{tex/latex/oberdiek}%
  \file{enparen.sty}{\from{enparen.dtx}{package}}%
}

\catcode32=13\relax% active space
\let =\space%
\Msg{************************************************************************}
\Msg{*}
\Msg{* To finish the installation you have to move the following}
\Msg{* file into a directory searched by TeX:}
\Msg{*}
\Msg{*     enparen.sty}
\Msg{*}
\Msg{* To produce the documentation run the file `enparen.drv'}
\Msg{* through LaTeX.}
\Msg{*}
\Msg{* Happy TeXing!}
\Msg{*}
\Msg{************************************************************************}

\endbatchfile
%</install>
%<*ignore>
\fi
%</ignore>
%<*driver>
\NeedsTeXFormat{LaTeX2e}
\ProvidesFile{enparen.drv}%
  [2012/01/07 v1.0 Parentheses nesting (HO)]%
\documentclass{ltxdoc}
\usepackage{holtxdoc}[2011/11/22]
\usepackage{enparen}[2012/01/07]
\begin{document}
  \DocInput{enparen.dtx}%
\end{document}
%</driver>
% \fi
%
% \CheckSum{339}
%
% \CharacterTable
%  {Upper-case    \A\B\C\D\E\F\G\H\I\J\K\L\M\N\O\P\Q\R\S\T\U\V\W\X\Y\Z
%   Lower-case    \a\b\c\d\e\f\g\h\i\j\k\l\m\n\o\p\q\r\s\t\u\v\w\x\y\z
%   Digits        \0\1\2\3\4\5\6\7\8\9
%   Exclamation   \!     Double quote  \"     Hash (number) \#
%   Dollar        \$     Percent       \%     Ampersand     \&
%   Acute accent  \'     Left paren    \(     Right paren   \)
%   Asterisk      \*     Plus          \+     Comma         \,
%   Minus         \-     Point         \.     Solidus       \/
%   Colon         \:     Semicolon     \;     Less than     \<
%   Equals        \=     Greater than  \>     Question mark \?
%   Commercial at \@     Left bracket  \[     Backslash     \\
%   Right bracket \]     Circumflex    \^     Underscore    \_
%   Grave accent  \`     Left brace    \{     Vertical bar  \|
%   Right brace   \}     Tilde         \~}
%
% \GetFileInfo{enparen.drv}
%
% \title{The \xpackage{enparen} package}
% \date{2012/01/07 v1.0}
% \author{Heiko Oberdiek\\\xemail{heiko.oberdiek at googlemail.com}}
%
% \maketitle
%
% \begin{abstract}
% The package defines macros to set parentheses that automatically
% change the symbols from inner to outer fences.
% \end{abstract}
%
% \tableofcontents
%
% \section{Documentation}
%
% The \hologo{LaTeX} package provides macros for automatically changed
% parentheses symbols depending on the fence order.
% The innermost parentheses are always using the same
% parentheses. The symbols changes for the outer fences.
%
% Example:
% \begin{quote}
%   \enparen{a \enparen{b} \enparen{c \enparen{d}} \enparen{e}}
% \end{quote}
% is generated by
%\begin{quote}
%\begin{verbatim}
%\documentclass{article}
%\usepackage{enparen}
%\begin{document}
%  \enparen{a \enparen{b} \enparen{c \enparen{d}} \enparen{e}}
%\end{document}
%\end{verbatim}
%\end{quote}
%
% The package is the result of a newsgroup thread. Dan gives
% the following specification \cite{dan}:
% \begin{quote}
% ``On the other hand, the rules for fences are usually the reverse:
% innermost fences are always |()|, next outer are |[]|, etc. This means
% the opening fence has to wait until all the fences between it and
% the matching close have been detected before it can decide whether
% to be |(| or |[| or |{|.''
% \end{quote}
%
% The fence level counting starts from innermost parentheses with one.
% For the next outer fences the level is increased by one.
% The example above with level indexes:
% \begin{quote}
%   \def\1{\ensuremath{_{1}}}
%   \def\2{\ensuremath{_{2}}}
%   \def\3{\ensuremath{_{3}}}
%   \enparen{\3a \enparen{\1b}\1 \enparen{\2c \enparen{\1d}\1}\2 \enparen{e}}\3
% \end{quote}
% The correct level is only known at the closing symbol. Therefore the
% correct value is remembered in the main \xext{aux} file and used
% in the second \hologo{LaTeX} run.
%
% \subsection{User macros}
%
% \begin{declcs}{enparen} \M{text}
% \end{declcs}
% The macro \cs{enparen} puts its argument \meta{text} in parentheses.
% If the macro is nested, the used fence symbols change for the
% outer fences.
%
% \begin{declcs}{enparenLeft}\\
%   \cs{enparenRight}
% \end{declcs}
% Instead of |\enparen{text}| the left and right symbol can be
% used separately, but in pairs:
% \begin{quote}
%   |\enparenLeft text\enparenRight|
% \end{quote}
% \cs{enparenLeft} and \cs{enparenRight} may be used at different
% group levels, but they must be properly nested.
%
% \begin{declcs}{enparenSetSymbols} \M{level} \M{opening symbol} \M{closing symbol}
% \end{declcs}
% Macro \cs{enparenSetSymbols} configures the \meta{opening symbol} and
% \meta{closing symbol} for the \meta{level}. The \meta{level}
% is a number and the counting starts with one. Level zero is used,
% if the correct level is not known (e.g. in the first \hologo{LaTeX} run).
% The package defines the following sets:
%\begin{quote}
%\begin{verbatim}
%\enparenSetSymbols{0}{(}{)}
%\enparenSetSymbols{1}{(}{)}
%\enparenSetSymbols{2}{[}{]}
%\enparenSetSymbols{3}{\{}{\}}
%\end{verbatim}
%\end{quote}
% Example for changing the third and adding a fourth level:
%\begin{quote}
%\begin{verbatim}
%\enparenSetSymbols{3}{\ensuremath{\langle}}{\ensuremath{\rangle}}
%\enparenSetSymbols{4}{\{}{\}}
%\end{verbatim}
%\end{quote}
%
% \begin{declcs}{enparenUnsetSymbols} \M{level}
% \end{declcs}
% The symbols for level \meta{level} are removed.
% Example scenario: Only two nesting levels must be used,
% the package defines more, then the third level can be
% disabled by |\enparenUnsetSymbols{3}| and the user
% gets warnings if parentheses at level 3 are needed.
%
% \subsection{Contexts}
%
% \begin{declcs}{enparenBeginContext} \M{name}\\
%   \cs{enparenEndContext} \M{name}
% \end{declcs}
% If the current text is interrupted by footnotes, floats
% with captions, then the parentheses inside the text of footnotes,
% captions, \dots should restart from scratch. This can be achieved
% by embedding the text inside macros \cs{enparenBeginContext}
% and \cs{enparenEndContext}. These macros must be properly nested.
% The \meta{name} for the begin and end macro must be the same.
% It is a help for debugging problems, because the warning messages
% show the context name. But it is not necessary that the
% begin\slash end pairs have different names. Example:
%\begin{quote}
%\begin{verbatim}
%\enparenLeft text before table ...
%\begin{table}
%  \caption{Table caption}
%  \enparenBegin{Context}{table}
%  Other text \enparen{foobar}.
%  \enparenEnd{Context}{table}
%\end{table}
%text after table ...
%\enparenRight
%\end{verbatim}
%\end{quote}
% The parentheses inside the table environment and context `table'
% are not nested inside other parentheses: \enparen{foobar}.
% In case of captions and footnotes the contexts are automatically
% added, see next section about options.
%
% \subsection{Options}
%
% \begin{declcs}{enparenSetup} \M{key value list}
% \end{declcs}
% Some options (currently all) can also be set after the
% package is loaded. They can be set in the argument
% \meta{key value list} of \cs{enparenSetup}. Options are disabled
% after they are used the last time. Currently all options are boolean
% options and are disabled in |\begin{document}|.
%
% \begin{description}
% \item[\xoption{caption}:] The caption text is put in a
%   context |caption|.
% \item[\xoption{footnote}:] The footnote text is put in a
%   context |footnote|.
% \end{description}
%
% Example for disabling the two options at different places:
%\begin{quote}
%\begin{verbatim}
%\usepackage[caption=false]{enparen}
%\enparenSetup{footnote=false}
%\end{verbatim}
%\end{quote}
%
% \subsection{Notes}
%
% \begin{description}
% \item[Implicite kerning:]
%   Unexpandable stuff might affect the implicite kerning.
%   The package cannot avoid this, because it need to define
%   and redefine macros at the occurence of each symbol.
%   This is done before the opening and after the closing
%   symbol, thus that the implicite kerning inside
%   is not affected.
% \end{description}
%
% \StopEventually{
% }
%
% \section{Implementation}
%
%    \begin{macrocode}
%<*package>
\NeedsTeXFormat{LaTeX2e}
\ProvidesPackage{enparen}
  [2012/01/07 v1.0 Parentheses nesting (HO)]%
%    \end{macrocode}
%
% \subsection{Resources}
%
%    \begin{macrocode}
\RequirePackage{ltxcmds}[2011/11/09]
%    \end{macrocode}
%
%    \begin{macrocode}
\ltx@IfUndefined{numexpr}{%
  \PackageError{enparen}{%
    Missing e-TeX's \ltx@backslashchar numexpr.\MessageBreak
    The package will continue with emergency definitions%
  }\@ehc
  \def\enparenLeft{(}%
  \def\enparenRight{)}%
  \long\def\enparen#1{\enparenLeft#1\enparenRight}%
  \let\enparenSetup\ltx@gobble
  \let\enparenSetSymbols\ltx@gobblethree
  \let\enparenUnsetSymbols\ltx@gobble
  \endinput
}{}
%    \end{macrocode}
%
%    \begin{macrocode}
\RequirePackage{protecteddef}[2011/01/31]
\RequirePackage{atveryend}[2011/06/30]
\RequirePackage{uniquecounter}[2011/01/30]
\RequirePackage{zref-base}[2011/03/18]
\RequirePackage{kvoptions}[2011/06/30]
\RequirePackage{kvsetkeys}[2011/10/18]
%    \end{macrocode}
%
%    \begin{macro}{\zref@wrapper@mainaux}
%    \begin{macrocode}
\providecommand{\zref@wrapper@mainaux}[1]{%
  \ifx\@auxout\@mainaux
    #1%
  \else
    \begingroup
      \let\@auxout\@mainaux
      #1%
    \endgroup
  \fi
}
%    \end{macrocode}
%    \end{macro}
%
%    \begin{macrocode}
\UniqueCounterNew{enparen}
%    \end{macrocode}
%
%    \begin{macrocode}
\zref@newprop{enparen}[0]{}
%    \end{macrocode}
%
% \subsection{Contexts}
%
%    \begin{macro}{\enparenContextDefault}
%    \begin{macrocode}
\def\enparenContextDefault{main}
%    \end{macrocode}
%    \end{macro}
%
%    \begin{macro}{\enparen@ctx}
%    \begin{macrocode}
\let\enparen@ctx\ltx@empty
%    \end{macrocode}
%    \end{macro}
%
%    \begin{macro}{\enparen@stack}
%    \begin{macrocode}
\let\enparen@stack\ltx@empty
%    \end{macrocode}
%    \end{macro}
%
% \subsubsection{Stack for contexts}
%
%    \begin{macro}{\enparen@CtxStack}
%    \begin{macrocode}
\global\let\enparen@CtxStack\ltx@empty
%    \end{macrocode}
%    \end{macro}
%
%    \begin{macro}{\enparen@CtxStackPush}
%    \begin{macrocode}
\def\enparen@CtxStackPush#1{%
  \xdef\enparen@CtxStack{%
    {\enparen@ctx}{\enparen@stack}%
    \enparen@CtxStack
  }%
  \xdef\enparen@ctx{#1}%
  \global\let\enparen@stack\ltx@empty
}
%    \end{macrocode}
%    \end{macro}
%
%    \begin{macro}{\enparen@CtxStackPop}
%    \begin{macrocode}
\def\enparen@CtxStackPop{%
  \ifx\enparen@CtxStack\ltx@empty
    \PackageWarning{enparen}{%
      Pop request for empty context stack%
    }%
    \global\let\enparen@ctx\enparenContextDefault
    \global\let\enparen@stack\ltx@empty
  \else
    \xdef\enparen@ctx{%
      \expandafter\ltx@car\enparen@CtxStack\@nil
    }%
    \xdef\enparen@stack{%
      \expandafter\ltx@carsecond\enparen@CtxStack\@nil
    }%
    \xdef\enparen@CtxStack{%
      \expandafter\ltx@cdrtwo\enparen@CtxStack\@nil
    }%
  \fi
}
%    \end{macrocode}
%    \end{macro}
%
% \subsubsection{Context user macros}
%
%    \begin{macro}{\enparenBeginContext}
%    \begin{macrocode}
\ProtectedDef*{\enparenBeginContext}[1]{%
  \enparen@CtxStackPush{#1}%
}
%    \end{macrocode}
%    \end{macro}
%    \begin{macro}{\enparenEndContext}
%    \begin{macrocode}
\ProtectedDef*{\enparenEndContext}[1]{%
  \edef\enparen@temp{#1}%
  \ifx\enparen@temp\enparen@ctx
  \else
    \PackageWarning{enparen}{%
      Context mismatch in end request.\MessageBreak
      `#1' should be ended, but current context\MessageBreak
      is `\enparen@ctx'%
    }%
  \fi
  \enparenCheckEmptyStack
  \enparen@CtxStackPop
}
%    \end{macrocode}
%    \end{macro}
%
%    \begin{macro}{\enparenCheckEmptyStack}
%    \begin{macrocode}
\ProtectedDef*{\enparenCheckEmptyStack}[0]{%
  \ifx\enparen@stack\ltx@empty
  \else
    \PackageWarning{enparen}{%
      Ending non-empty stack `\enparen@ctx':\MessageBreak
      \enparen@PrintStack\MessageBreak
    }%
  \fi
}
%    \end{macrocode}
%    \end{macro}
%
%    \begin{macro}{\enparen@PrintStack}
%    \begin{macrocode}
\def\enparen@PrintStack{%
  \expandafter\enparen@PrintStackAux
  \enparen@stack\ltx@empty\ltx@empty
}
%    \end{macrocode}
%    \end{macro}
%    \begin{macro}{\enparen@PrintStackAux}
%    \begin{macrocode}
\def\enparen@PrintStackAux#1#2{%
  \ifx\ltx@empty#1%
  \else
    {#1:#2}%
    \expandafter\enparen@PrintStackAux
  \fi
}
%    \end{macrocode}
%    \end{macro}
%
% \subsection{Symbols}
%
%    \begin{macro}{\enparenSetSymbols}
%    \begin{macrocode}
\ProtectedDef*{\enparenSetSymbols}[3]{%
  \expandafter
  \def\csname enparen@symbol\the\numexpr#1L\endcsname{#2}%
  \expandafter
  \def\csname enparen@symbol\the\numexpr#1R\endcsname{#3}%
}
%    \end{macrocode}
%    \end{macro}
%    \begin{macro}{\enparenUnsetSymbols}
%    \begin{macrocode}
\ProtectedDef*{\enparenUnsetSymbols}[1]{%
  \expandafter
  \let\csname enparen@symbol\the\numexpr#1L\endcsname\ltx@undefined
  \expandafter
  \let\csname enparen@symbol\the\numexpr#1R\endcsname\ltx@undefined
}
%    \end{macrocode}
%    \end{macro}
%
%    \begin{macrocode}
\enparenSetSymbols{0}{(}{)}
\enparenSetSymbols{1}{(}{)}
\enparenSetSymbols{2}{[}{]}
\enparenSetSymbols{3}{\{}{\}}
\enparenSetSymbols{4}{\ensuremath{\langle}}{\ensuremath{\rangle}}
%    \end{macrocode}
%
% \subsection{Main user macros}
%
%    \begin{macro}{\enparen}
%    \begin{macrocode}
\ProtectedDef{\enparen}[1]{%
  \enparenLeft#1\enparenRight
}
%    \end{macrocode}
%    \end{macro}
%
%    \begin{macro}{\enparenLeft}
%    \begin{macrocode}
\ProtectedDef*{\enparenLeft}[0]{%
  \UniqueCounterCall{enparen}\enparen@Left
}
%    \end{macrocode}
%    \end{macro}
%    \begin{macro}{\enparen@Left}
%    \begin{macrocode}
\def\enparen@Left#1{%
  \xdef\enparen@stack{%
    {#1}{1}%
    \expandafter\enparen@Inc\expandafter2\expandafter!%
    \enparen@stack\ltx@empty\ltx@empty
  }%
  \edef\enparen@tmp{\zref@extract{enparen#1}{enparen}}%
  \ltx@IfUndefined{enparen@symbol\enparen@tmp L}{%
    \PackageWarning{enparen}{%
      Undefined symbols for level \enparen@tmp
    }%
    \csname enparen@symbol0L\endcsname
  }{%
    \csname enparen@symbol\enparen@tmp L\endcsname
  }%
}
%    \end{macrocode}
%    \end{macro}
%    \begin{macro}{\enparen@Inc}
%    \begin{macrocode}
\def\enparen@Inc#1!#2#3{%
  \ifx\ltx@empty#2%
  \else
    \ifnum#3<#1 %
      {#2}{#1}%
      \expandafter\enparen@Inc
      \the\numexpr#1+1\expandafter\expandafter\expandafter!%
    \else
      {#2}{#3}%
    \fi
  \fi
}
%    \end{macrocode}
%    \end{macro}
%
%    \begin{macro}{\enparenRight}
%    \begin{macrocode}
\ProtectedDef*{\enparenRight}[0]{%
  \ifx\enparen@stack\ltx@empty
    \PackageWarning{enparen}{%
      Missing left symbol for right symbol%
    }%
    \csname enparen@symbol0R\endcsname
  \else
    \expandafter\enparen@Right\enparen@stack\@nil
  \fi
}
%    \end{macrocode}
%    \end{macro}
%
%    \begin{macro}{\enparen@Right}
%    \begin{macrocode}
\def\enparen@Right#1#2#3\@nil{%
  \ltx@IfUndefined{%
    enparen@symbol%
    \zref@extract{enparen#1}{enparen}%
    R%
  }{%
    \csname enparen@symbol0R\endcsname
  }{%
    \csname
      enparen@symbol%
      \zref@extract{enparen#1}{enparen}%
       R%
    \endcsname
  }%
  \zref@wrapper@mainaux{%
    \zref@setcurrent{enparen}{#2}%
    \zref@labelbyprops{enparen#1}{enparen}%
  }%
  \xdef\enparen@stack{#3}%
}
%    \end{macrocode}
%    \end{macro}
%
% \subsection{Options}
%
%    \begin{macrocode}
\SetupKeyvalOptions{%
  family=enparen,%
  prefix=enparen@,%
}
%    \end{macrocode}
%
%    \begin{macro}{\enparenSetup}
%    \begin{macrocode}
\ProtectedDef*{\enparenSetup}[0]{%
  \kvsetkeys{enparen}%
}
%    \end{macrocode}
%    \end{macro}
%
%    \begin{macrocode}
\DeclareBoolOption[true]{footnote}
\DeclareBoolOption[true]{caption}
\ProcessKeyvalOptions*
%    \end{macrocode}
%
% \subsection{Context settings}
%
%    \begin{macro}{\enparen@AtBegin}
%    \begin{macrocode}
\def\enparen@AtBegin{%
  \ifenparen@footnote
    \let\enparen@org@makefntext\@makefntext
    \long\def\@makefntext##1{%
      \enparen@org@makefntext{%
        \enparenBeginContext{footnote}%
        ##1%
        \enparenEndContext{footnote}%
      }%
    }%
  \fi
  \enparen@Disable{footnote}%
  \ifenparen@caption
    \let\enparen@org@makecaption\@makecaption
    \long\def\@makecaption##1##2{%
      \enparen@org@makecaption{##1}{%
        \enparenBeginContext{caption}%
        ##2%
        \enparenEndContext{caption}%
      }%
    }%
  \fi
  \enparen@Disable{caption}%
}
%    \end{macrocode}
%    \end{macro}
%
%    \begin{macro}{\enparen@Disable}
%    \begin{macrocode}
\def\enparen@Disable#1{%
  \DisableKeyvalOption[%
    action=warning,%
    package=enparen,%
  ]{enparen}{#1}%
}
%    \end{macrocode}
%    \end{macro}
%
%    \begin{macrocode}
\AtBeginDocument{\enparen@AtBegin}
%    \end{macrocode}
%
% \subsection{At end of document}
%
%    \begin{macro}{\enparen@AtEnd}
%    \begin{macrocode}
\def\enparen@AtEnd{%
  \enparenCheckEmptyStack
  \ifx\enparen@CtxStack\ltx@empty
  \else
    \PackageWarningNoLine{enparen}{%
      Context stack is not empty at end of document.\MessageBreak
      Current stack and contents of context stack:\MessageBreak
      [\enparen@ctx]:[\enparen@PrintStack]%
      \expandafter
      \enparen@PrintContextStack\enparen@CtxStack\relax\relax
    }%
  \fi
}
%    \end{macrocode}
%    \end{macro}
%    \begin{macro}{\enparen@PrintContextStack}
%    \begin{macrocode}
\def\enparen@PrintContextStack#1#2{%
  \ifx\relax#1\ltx@empty
  \else
    \MessageBreak
    [#1]:[\enparen@PrintStackAux#2\ltx@empty\ltx@empty]% hash-ok
    \expandafter\enparen@PrintContextStack
  \fi
}
%    \end{macrocode}
%    \end{macro}
%    \begin{macrocode}
\AtVeryEndDocument{\enparen@AtEnd}
%    \end{macrocode}
%
%    \begin{macrocode}
%</package>
%    \end{macrocode}
%
% \section{Installation}
%
% \subsection{Download}
%
% \paragraph{Package.} This package is available on
% CTAN\footnote{\url{ftp://ftp.ctan.org/tex-archive/}}:
% \begin{description}
% \item[\CTAN{macros/latex/contrib/oberdiek/enparen.dtx}] The source file.
% \item[\CTAN{macros/latex/contrib/oberdiek/enparen.pdf}] Documentation.
% \end{description}
%
%
% \paragraph{Bundle.} All the packages of the bundle `oberdiek'
% are also available in a TDS compliant ZIP archive. There
% the packages are already unpacked and the documentation files
% are generated. The files and directories obey the TDS standard.
% \begin{description}
% \item[\CTAN{install/macros/latex/contrib/oberdiek.tds.zip}]
% \end{description}
% \emph{TDS} refers to the standard ``A Directory Structure
% for \TeX\ Files'' (\CTAN{tds/tds.pdf}). Directories
% with \xfile{texmf} in their name are usually organized this way.
%
% \subsection{Bundle installation}
%
% \paragraph{Unpacking.} Unpack the \xfile{oberdiek.tds.zip} in the
% TDS tree (also known as \xfile{texmf} tree) of your choice.
% Example (linux):
% \begin{quote}
%   |unzip oberdiek.tds.zip -d ~/texmf|
% \end{quote}
%
% \paragraph{Script installation.}
% Check the directory \xfile{TDS:scripts/oberdiek/} for
% scripts that need further installation steps.
% Package \xpackage{attachfile2} comes with the Perl script
% \xfile{pdfatfi.pl} that should be installed in such a way
% that it can be called as \texttt{pdfatfi}.
% Example (linux):
% \begin{quote}
%   |chmod +x scripts/oberdiek/pdfatfi.pl|\\
%   |cp scripts/oberdiek/pdfatfi.pl /usr/local/bin/|
% \end{quote}
%
% \subsection{Package installation}
%
% \paragraph{Unpacking.} The \xfile{.dtx} file is a self-extracting
% \docstrip\ archive. The files are extracted by running the
% \xfile{.dtx} through \plainTeX:
% \begin{quote}
%   \verb|tex enparen.dtx|
% \end{quote}
%
% \paragraph{TDS.} Now the different files must be moved into
% the different directories in your installation TDS tree
% (also known as \xfile{texmf} tree):
% \begin{quote}
% \def\t{^^A
% \begin{tabular}{@{}>{\ttfamily}l@{ $\rightarrow$ }>{\ttfamily}l@{}}
%   enparen.sty & tex/latex/oberdiek/enparen.sty\\
%   enparen.pdf & doc/latex/oberdiek/enparen.pdf\\
%   enparen.dtx & source/latex/oberdiek/enparen.dtx\\
% \end{tabular}^^A
% }^^A
% \sbox0{\t}^^A
% \ifdim\wd0>\linewidth
%   \begingroup
%     \advance\linewidth by\leftmargin
%     \advance\linewidth by\rightmargin
%   \edef\x{\endgroup
%     \def\noexpand\lw{\the\linewidth}^^A
%   }\x
%   \def\lwbox{^^A
%     \leavevmode
%     \hbox to \linewidth{^^A
%       \kern-\leftmargin\relax
%       \hss
%       \usebox0
%       \hss
%       \kern-\rightmargin\relax
%     }^^A
%   }^^A
%   \ifdim\wd0>\lw
%     \sbox0{\small\t}^^A
%     \ifdim\wd0>\linewidth
%       \ifdim\wd0>\lw
%         \sbox0{\footnotesize\t}^^A
%         \ifdim\wd0>\linewidth
%           \ifdim\wd0>\lw
%             \sbox0{\scriptsize\t}^^A
%             \ifdim\wd0>\linewidth
%               \ifdim\wd0>\lw
%                 \sbox0{\tiny\t}^^A
%                 \ifdim\wd0>\linewidth
%                   \lwbox
%                 \else
%                   \usebox0
%                 \fi
%               \else
%                 \lwbox
%               \fi
%             \else
%               \usebox0
%             \fi
%           \else
%             \lwbox
%           \fi
%         \else
%           \usebox0
%         \fi
%       \else
%         \lwbox
%       \fi
%     \else
%       \usebox0
%     \fi
%   \else
%     \lwbox
%   \fi
% \else
%   \usebox0
% \fi
% \end{quote}
% If you have a \xfile{docstrip.cfg} that configures and enables \docstrip's
% TDS installing feature, then some files can already be in the right
% place, see the documentation of \docstrip.
%
% \subsection{Refresh file name databases}
%
% If your \TeX~distribution
% (\teTeX, \mikTeX, \dots) relies on file name databases, you must refresh
% these. For example, \teTeX\ users run \verb|texhash| or
% \verb|mktexlsr|.
%
% \subsection{Some details for the interested}
%
% \paragraph{Attached source.}
%
% The PDF documentation on CTAN also includes the
% \xfile{.dtx} source file. It can be extracted by
% AcrobatReader 6 or higher. Another option is \textsf{pdftk},
% e.g. unpack the file into the current directory:
% \begin{quote}
%   \verb|pdftk enparen.pdf unpack_files output .|
% \end{quote}
%
% \paragraph{Unpacking with \LaTeX.}
% The \xfile{.dtx} chooses its action depending on the format:
% \begin{description}
% \item[\plainTeX:] Run \docstrip\ and extract the files.
% \item[\LaTeX:] Generate the documentation.
% \end{description}
% If you insist on using \LaTeX\ for \docstrip\ (really,
% \docstrip\ does not need \LaTeX), then inform the autodetect routine
% about your intention:
% \begin{quote}
%   \verb|latex \let\install=y\input{enparen.dtx}|
% \end{quote}
% Do not forget to quote the argument according to the demands
% of your shell.
%
% \paragraph{Generating the documentation.}
% You can use both the \xfile{.dtx} or the \xfile{.drv} to generate
% the documentation. The process can be configured by the
% configuration file \xfile{ltxdoc.cfg}. For instance, put this
% line into this file, if you want to have A4 as paper format:
% \begin{quote}
%   \verb|\PassOptionsToClass{a4paper}{article}|
% \end{quote}
% An example follows how to generate the
% documentation with pdf\LaTeX:
% \begin{quote}
%\begin{verbatim}
%pdflatex enparen.dtx
%makeindex -s gind.ist enparen.idx
%pdflatex enparen.dtx
%makeindex -s gind.ist enparen.idx
%pdflatex enparen.dtx
%\end{verbatim}
% \end{quote}
%
% \begin{thebibliography}{9}
% \raggedright
% \bibitem{dan}
%   Dan Luecking: \textit{Re: bracket order};
%   newsgroup \xnewsgroup{comp.text.tex};
%   message id \texttt{\small$\langle$9b07c9c8-ff92-4cbf-b3a9-84efecfeb506@l24g2000yqm.googlegroups.com$\rangle$}
%   2012-01-05.
%   \url{http://groups.google.com/group/comp.text.tex/msg/8774519da31c2352}
% \end{thebibliography}
%
% \begin{History}
%   \begin{Version}{2012/01/07 v1.0}
%   \item
%     First version.
%   \end{Version}
% \end{History}
%
% \PrintIndex
%
% \Finale
\endinput

%        (quote the arguments according to the demands of your shell)
%
% Documentation:
%    (a) If enparen.drv is present:
%           latex enparen.drv
%    (b) Without enparen.drv:
%           latex enparen.dtx; ...
%    The class ltxdoc loads the configuration file ltxdoc.cfg
%    if available. Here you can specify further options, e.g.
%    use A4 as paper format:
%       \PassOptionsToClass{a4paper}{article}
%
%    Programm calls to get the documentation (example):
%       pdflatex enparen.dtx
%       makeindex -s gind.ist enparen.idx
%       pdflatex enparen.dtx
%       makeindex -s gind.ist enparen.idx
%       pdflatex enparen.dtx
%
% Installation:
%    TDS:tex/latex/oberdiek/enparen.sty
%    TDS:doc/latex/oberdiek/enparen.pdf
%    TDS:source/latex/oberdiek/enparen.dtx
%
%<*ignore>
\begingroup
  \catcode123=1 %
  \catcode125=2 %
  \def\x{LaTeX2e}%
\expandafter\endgroup
\ifcase 0\ifx\install y1\fi\expandafter
         \ifx\csname processbatchFile\endcsname\relax\else1\fi
         \ifx\fmtname\x\else 1\fi\relax
\else\csname fi\endcsname
%</ignore>
%<*install>
\input docstrip.tex
\Msg{************************************************************************}
\Msg{* Installation}
\Msg{* Package: enparen 2012/01/07 v1.0 Parentheses nesting (HO)}
\Msg{************************************************************************}

\keepsilent
\askforoverwritefalse

\let\MetaPrefix\relax
\preamble

This is a generated file.

Project: enparen
Version: 2012/01/07 v1.0

Copyright (C) 2012 by
   Heiko Oberdiek <heiko.oberdiek at googlemail.com>

This work may be distributed and/or modified under the
conditions of the LaTeX Project Public License, either
version 1.3c of this license or (at your option) any later
version. This version of this license is in
   http://www.latex-project.org/lppl/lppl-1-3c.txt
and the latest version of this license is in
   http://www.latex-project.org/lppl.txt
and version 1.3 or later is part of all distributions of
LaTeX version 2005/12/01 or later.

This work has the LPPL maintenance status "maintained".

This Current Maintainer of this work is Heiko Oberdiek.

This work consists of the main source file enparen.dtx
and the derived files
   enparen.sty, enparen.pdf, enparen.ins, enparen.drv.

\endpreamble
\let\MetaPrefix\DoubleperCent

\generate{%
  \file{enparen.ins}{\from{enparen.dtx}{install}}%
  \file{enparen.drv}{\from{enparen.dtx}{driver}}%
  \usedir{tex/latex/oberdiek}%
  \file{enparen.sty}{\from{enparen.dtx}{package}}%
}

\catcode32=13\relax% active space
\let =\space%
\Msg{************************************************************************}
\Msg{*}
\Msg{* To finish the installation you have to move the following}
\Msg{* file into a directory searched by TeX:}
\Msg{*}
\Msg{*     enparen.sty}
\Msg{*}
\Msg{* To produce the documentation run the file `enparen.drv'}
\Msg{* through LaTeX.}
\Msg{*}
\Msg{* Happy TeXing!}
\Msg{*}
\Msg{************************************************************************}

\endbatchfile
%</install>
%<*ignore>
\fi
%</ignore>
%<*driver>
\NeedsTeXFormat{LaTeX2e}
\ProvidesFile{enparen.drv}%
  [2012/01/07 v1.0 Parentheses nesting (HO)]%
\documentclass{ltxdoc}
\usepackage{holtxdoc}[2011/11/22]
\usepackage{enparen}[2012/01/07]
\begin{document}
  \DocInput{enparen.dtx}%
\end{document}
%</driver>
% \fi
%
% \CheckSum{339}
%
% \CharacterTable
%  {Upper-case    \A\B\C\D\E\F\G\H\I\J\K\L\M\N\O\P\Q\R\S\T\U\V\W\X\Y\Z
%   Lower-case    \a\b\c\d\e\f\g\h\i\j\k\l\m\n\o\p\q\r\s\t\u\v\w\x\y\z
%   Digits        \0\1\2\3\4\5\6\7\8\9
%   Exclamation   \!     Double quote  \"     Hash (number) \#
%   Dollar        \$     Percent       \%     Ampersand     \&
%   Acute accent  \'     Left paren    \(     Right paren   \)
%   Asterisk      \*     Plus          \+     Comma         \,
%   Minus         \-     Point         \.     Solidus       \/
%   Colon         \:     Semicolon     \;     Less than     \<
%   Equals        \=     Greater than  \>     Question mark \?
%   Commercial at \@     Left bracket  \[     Backslash     \\
%   Right bracket \]     Circumflex    \^     Underscore    \_
%   Grave accent  \`     Left brace    \{     Vertical bar  \|
%   Right brace   \}     Tilde         \~}
%
% \GetFileInfo{enparen.drv}
%
% \title{The \xpackage{enparen} package}
% \date{2012/01/07 v1.0}
% \author{Heiko Oberdiek\\\xemail{heiko.oberdiek at googlemail.com}}
%
% \maketitle
%
% \begin{abstract}
% The package defines macros to set parentheses that automatically
% change the symbols from inner to outer fences.
% \end{abstract}
%
% \tableofcontents
%
% \section{Documentation}
%
% The \hologo{LaTeX} package provides macros for automatically changed
% parentheses symbols depending on the fence order.
% The innermost parentheses are always using the same
% parentheses. The symbols changes for the outer fences.
%
% Example:
% \begin{quote}
%   \enparen{a \enparen{b} \enparen{c \enparen{d}} \enparen{e}}
% \end{quote}
% is generated by
%\begin{quote}
%\begin{verbatim}
%\documentclass{article}
%\usepackage{enparen}
%\begin{document}
%  \enparen{a \enparen{b} \enparen{c \enparen{d}} \enparen{e}}
%\end{document}
%\end{verbatim}
%\end{quote}
%
% The package is the result of a newsgroup thread. Dan gives
% the following specification \cite{dan}:
% \begin{quote}
% ``On the other hand, the rules for fences are usually the reverse:
% innermost fences are always |()|, next outer are |[]|, etc. This means
% the opening fence has to wait until all the fences between it and
% the matching close have been detected before it can decide whether
% to be |(| or |[| or |{|.''
% \end{quote}
%
% The fence level counting starts from innermost parentheses with one.
% For the next outer fences the level is increased by one.
% The example above with level indexes:
% \begin{quote}
%   \def\1{\ensuremath{_{1}}}
%   \def\2{\ensuremath{_{2}}}
%   \def\3{\ensuremath{_{3}}}
%   \enparen{\3a \enparen{\1b}\1 \enparen{\2c \enparen{\1d}\1}\2 \enparen{e}}\3
% \end{quote}
% The correct level is only known at the closing symbol. Therefore the
% correct value is remembered in the main \xext{aux} file and used
% in the second \hologo{LaTeX} run.
%
% \subsection{User macros}
%
% \begin{declcs}{enparen} \M{text}
% \end{declcs}
% The macro \cs{enparen} puts its argument \meta{text} in parentheses.
% If the macro is nested, the used fence symbols change for the
% outer fences.
%
% \begin{declcs}{enparenLeft}\\
%   \cs{enparenRight}
% \end{declcs}
% Instead of |\enparen{text}| the left and right symbol can be
% used separately, but in pairs:
% \begin{quote}
%   |\enparenLeft text\enparenRight|
% \end{quote}
% \cs{enparenLeft} and \cs{enparenRight} may be used at different
% group levels, but they must be properly nested.
%
% \begin{declcs}{enparenSetSymbols} \M{level} \M{opening symbol} \M{closing symbol}
% \end{declcs}
% Macro \cs{enparenSetSymbols} configures the \meta{opening symbol} and
% \meta{closing symbol} for the \meta{level}. The \meta{level}
% is a number and the counting starts with one. Level zero is used,
% if the correct level is not known (e.g. in the first \hologo{LaTeX} run).
% The package defines the following sets:
%\begin{quote}
%\begin{verbatim}
%\enparenSetSymbols{0}{(}{)}
%\enparenSetSymbols{1}{(}{)}
%\enparenSetSymbols{2}{[}{]}
%\enparenSetSymbols{3}{\{}{\}}
%\end{verbatim}
%\end{quote}
% Example for changing the third and adding a fourth level:
%\begin{quote}
%\begin{verbatim}
%\enparenSetSymbols{3}{\ensuremath{\langle}}{\ensuremath{\rangle}}
%\enparenSetSymbols{4}{\{}{\}}
%\end{verbatim}
%\end{quote}
%
% \begin{declcs}{enparenUnsetSymbols} \M{level}
% \end{declcs}
% The symbols for level \meta{level} are removed.
% Example scenario: Only two nesting levels must be used,
% the package defines more, then the third level can be
% disabled by |\enparenUnsetSymbols{3}| and the user
% gets warnings if parentheses at level 3 are needed.
%
% \subsection{Contexts}
%
% \begin{declcs}{enparenBeginContext} \M{name}\\
%   \cs{enparenEndContext} \M{name}
% \end{declcs}
% If the current text is interrupted by footnotes, floats
% with captions, then the parentheses inside the text of footnotes,
% captions, \dots should restart from scratch. This can be achieved
% by embedding the text inside macros \cs{enparenBeginContext}
% and \cs{enparenEndContext}. These macros must be properly nested.
% The \meta{name} for the begin and end macro must be the same.
% It is a help for debugging problems, because the warning messages
% show the context name. But it is not necessary that the
% begin\slash end pairs have different names. Example:
%\begin{quote}
%\begin{verbatim}
%\enparenLeft text before table ...
%\begin{table}
%  \caption{Table caption}
%  \enparenBegin{Context}{table}
%  Other text \enparen{foobar}.
%  \enparenEnd{Context}{table}
%\end{table}
%text after table ...
%\enparenRight
%\end{verbatim}
%\end{quote}
% The parentheses inside the table environment and context `table'
% are not nested inside other parentheses: \enparen{foobar}.
% In case of captions and footnotes the contexts are automatically
% added, see next section about options.
%
% \subsection{Options}
%
% \begin{declcs}{enparenSetup} \M{key value list}
% \end{declcs}
% Some options (currently all) can also be set after the
% package is loaded. They can be set in the argument
% \meta{key value list} of \cs{enparenSetup}. Options are disabled
% after they are used the last time. Currently all options are boolean
% options and are disabled in |\begin{document}|.
%
% \begin{description}
% \item[\xoption{caption}:] The caption text is put in a
%   context |caption|.
% \item[\xoption{footnote}:] The footnote text is put in a
%   context |footnote|.
% \end{description}
%
% Example for disabling the two options at different places:
%\begin{quote}
%\begin{verbatim}
%\usepackage[caption=false]{enparen}
%\enparenSetup{footnote=false}
%\end{verbatim}
%\end{quote}
%
% \subsection{Notes}
%
% \begin{description}
% \item[Implicite kerning:]
%   Unexpandable stuff might affect the implicite kerning.
%   The package cannot avoid this, because it need to define
%   and redefine macros at the occurence of each symbol.
%   This is done before the opening and after the closing
%   symbol, thus that the implicite kerning inside
%   is not affected.
% \end{description}
%
% \StopEventually{
% }
%
% \section{Implementation}
%
%    \begin{macrocode}
%<*package>
\NeedsTeXFormat{LaTeX2e}
\ProvidesPackage{enparen}
  [2012/01/07 v1.0 Parentheses nesting (HO)]%
%    \end{macrocode}
%
% \subsection{Resources}
%
%    \begin{macrocode}
\RequirePackage{ltxcmds}[2011/11/09]
%    \end{macrocode}
%
%    \begin{macrocode}
\ltx@IfUndefined{numexpr}{%
  \PackageError{enparen}{%
    Missing e-TeX's \ltx@backslashchar numexpr.\MessageBreak
    The package will continue with emergency definitions%
  }\@ehc
  \def\enparenLeft{(}%
  \def\enparenRight{)}%
  \long\def\enparen#1{\enparenLeft#1\enparenRight}%
  \let\enparenSetup\ltx@gobble
  \let\enparenSetSymbols\ltx@gobblethree
  \let\enparenUnsetSymbols\ltx@gobble
  \endinput
}{}
%    \end{macrocode}
%
%    \begin{macrocode}
\RequirePackage{protecteddef}[2011/01/31]
\RequirePackage{atveryend}[2011/06/30]
\RequirePackage{uniquecounter}[2011/01/30]
\RequirePackage{zref-base}[2011/03/18]
\RequirePackage{kvoptions}[2011/06/30]
\RequirePackage{kvsetkeys}[2011/10/18]
%    \end{macrocode}
%
%    \begin{macro}{\zref@wrapper@mainaux}
%    \begin{macrocode}
\providecommand{\zref@wrapper@mainaux}[1]{%
  \ifx\@auxout\@mainaux
    #1%
  \else
    \begingroup
      \let\@auxout\@mainaux
      #1%
    \endgroup
  \fi
}
%    \end{macrocode}
%    \end{macro}
%
%    \begin{macrocode}
\UniqueCounterNew{enparen}
%    \end{macrocode}
%
%    \begin{macrocode}
\zref@newprop{enparen}[0]{}
%    \end{macrocode}
%
% \subsection{Contexts}
%
%    \begin{macro}{\enparenContextDefault}
%    \begin{macrocode}
\def\enparenContextDefault{main}
%    \end{macrocode}
%    \end{macro}
%
%    \begin{macro}{\enparen@ctx}
%    \begin{macrocode}
\let\enparen@ctx\ltx@empty
%    \end{macrocode}
%    \end{macro}
%
%    \begin{macro}{\enparen@stack}
%    \begin{macrocode}
\let\enparen@stack\ltx@empty
%    \end{macrocode}
%    \end{macro}
%
% \subsubsection{Stack for contexts}
%
%    \begin{macro}{\enparen@CtxStack}
%    \begin{macrocode}
\global\let\enparen@CtxStack\ltx@empty
%    \end{macrocode}
%    \end{macro}
%
%    \begin{macro}{\enparen@CtxStackPush}
%    \begin{macrocode}
\def\enparen@CtxStackPush#1{%
  \xdef\enparen@CtxStack{%
    {\enparen@ctx}{\enparen@stack}%
    \enparen@CtxStack
  }%
  \xdef\enparen@ctx{#1}%
  \global\let\enparen@stack\ltx@empty
}
%    \end{macrocode}
%    \end{macro}
%
%    \begin{macro}{\enparen@CtxStackPop}
%    \begin{macrocode}
\def\enparen@CtxStackPop{%
  \ifx\enparen@CtxStack\ltx@empty
    \PackageWarning{enparen}{%
      Pop request for empty context stack%
    }%
    \global\let\enparen@ctx\enparenContextDefault
    \global\let\enparen@stack\ltx@empty
  \else
    \xdef\enparen@ctx{%
      \expandafter\ltx@car\enparen@CtxStack\@nil
    }%
    \xdef\enparen@stack{%
      \expandafter\ltx@carsecond\enparen@CtxStack\@nil
    }%
    \xdef\enparen@CtxStack{%
      \expandafter\ltx@cdrtwo\enparen@CtxStack\@nil
    }%
  \fi
}
%    \end{macrocode}
%    \end{macro}
%
% \subsubsection{Context user macros}
%
%    \begin{macro}{\enparenBeginContext}
%    \begin{macrocode}
\ProtectedDef*{\enparenBeginContext}[1]{%
  \enparen@CtxStackPush{#1}%
}
%    \end{macrocode}
%    \end{macro}
%    \begin{macro}{\enparenEndContext}
%    \begin{macrocode}
\ProtectedDef*{\enparenEndContext}[1]{%
  \edef\enparen@temp{#1}%
  \ifx\enparen@temp\enparen@ctx
  \else
    \PackageWarning{enparen}{%
      Context mismatch in end request.\MessageBreak
      `#1' should be ended, but current context\MessageBreak
      is `\enparen@ctx'%
    }%
  \fi
  \enparenCheckEmptyStack
  \enparen@CtxStackPop
}
%    \end{macrocode}
%    \end{macro}
%
%    \begin{macro}{\enparenCheckEmptyStack}
%    \begin{macrocode}
\ProtectedDef*{\enparenCheckEmptyStack}[0]{%
  \ifx\enparen@stack\ltx@empty
  \else
    \PackageWarning{enparen}{%
      Ending non-empty stack `\enparen@ctx':\MessageBreak
      \enparen@PrintStack\MessageBreak
    }%
  \fi
}
%    \end{macrocode}
%    \end{macro}
%
%    \begin{macro}{\enparen@PrintStack}
%    \begin{macrocode}
\def\enparen@PrintStack{%
  \expandafter\enparen@PrintStackAux
  \enparen@stack\ltx@empty\ltx@empty
}
%    \end{macrocode}
%    \end{macro}
%    \begin{macro}{\enparen@PrintStackAux}
%    \begin{macrocode}
\def\enparen@PrintStackAux#1#2{%
  \ifx\ltx@empty#1%
  \else
    {#1:#2}%
    \expandafter\enparen@PrintStackAux
  \fi
}
%    \end{macrocode}
%    \end{macro}
%
% \subsection{Symbols}
%
%    \begin{macro}{\enparenSetSymbols}
%    \begin{macrocode}
\ProtectedDef*{\enparenSetSymbols}[3]{%
  \expandafter
  \def\csname enparen@symbol\the\numexpr#1L\endcsname{#2}%
  \expandafter
  \def\csname enparen@symbol\the\numexpr#1R\endcsname{#3}%
}
%    \end{macrocode}
%    \end{macro}
%    \begin{macro}{\enparenUnsetSymbols}
%    \begin{macrocode}
\ProtectedDef*{\enparenUnsetSymbols}[1]{%
  \expandafter
  \let\csname enparen@symbol\the\numexpr#1L\endcsname\ltx@undefined
  \expandafter
  \let\csname enparen@symbol\the\numexpr#1R\endcsname\ltx@undefined
}
%    \end{macrocode}
%    \end{macro}
%
%    \begin{macrocode}
\enparenSetSymbols{0}{(}{)}
\enparenSetSymbols{1}{(}{)}
\enparenSetSymbols{2}{[}{]}
\enparenSetSymbols{3}{\{}{\}}
\enparenSetSymbols{4}{\ensuremath{\langle}}{\ensuremath{\rangle}}
%    \end{macrocode}
%
% \subsection{Main user macros}
%
%    \begin{macro}{\enparen}
%    \begin{macrocode}
\ProtectedDef{\enparen}[1]{%
  \enparenLeft#1\enparenRight
}
%    \end{macrocode}
%    \end{macro}
%
%    \begin{macro}{\enparenLeft}
%    \begin{macrocode}
\ProtectedDef*{\enparenLeft}[0]{%
  \UniqueCounterCall{enparen}\enparen@Left
}
%    \end{macrocode}
%    \end{macro}
%    \begin{macro}{\enparen@Left}
%    \begin{macrocode}
\def\enparen@Left#1{%
  \xdef\enparen@stack{%
    {#1}{1}%
    \expandafter\enparen@Inc\expandafter2\expandafter!%
    \enparen@stack\ltx@empty\ltx@empty
  }%
  \edef\enparen@tmp{\zref@extract{enparen#1}{enparen}}%
  \ltx@IfUndefined{enparen@symbol\enparen@tmp L}{%
    \PackageWarning{enparen}{%
      Undefined symbols for level \enparen@tmp
    }%
    \csname enparen@symbol0L\endcsname
  }{%
    \csname enparen@symbol\enparen@tmp L\endcsname
  }%
}
%    \end{macrocode}
%    \end{macro}
%    \begin{macro}{\enparen@Inc}
%    \begin{macrocode}
\def\enparen@Inc#1!#2#3{%
  \ifx\ltx@empty#2%
  \else
    \ifnum#3<#1 %
      {#2}{#1}%
      \expandafter\enparen@Inc
      \the\numexpr#1+1\expandafter\expandafter\expandafter!%
    \else
      {#2}{#3}%
    \fi
  \fi
}
%    \end{macrocode}
%    \end{macro}
%
%    \begin{macro}{\enparenRight}
%    \begin{macrocode}
\ProtectedDef*{\enparenRight}[0]{%
  \ifx\enparen@stack\ltx@empty
    \PackageWarning{enparen}{%
      Missing left symbol for right symbol%
    }%
    \csname enparen@symbol0R\endcsname
  \else
    \expandafter\enparen@Right\enparen@stack\@nil
  \fi
}
%    \end{macrocode}
%    \end{macro}
%
%    \begin{macro}{\enparen@Right}
%    \begin{macrocode}
\def\enparen@Right#1#2#3\@nil{%
  \ltx@IfUndefined{%
    enparen@symbol%
    \zref@extract{enparen#1}{enparen}%
    R%
  }{%
    \csname enparen@symbol0R\endcsname
  }{%
    \csname
      enparen@symbol%
      \zref@extract{enparen#1}{enparen}%
       R%
    \endcsname
  }%
  \zref@wrapper@mainaux{%
    \zref@setcurrent{enparen}{#2}%
    \zref@labelbyprops{enparen#1}{enparen}%
  }%
  \xdef\enparen@stack{#3}%
}
%    \end{macrocode}
%    \end{macro}
%
% \subsection{Options}
%
%    \begin{macrocode}
\SetupKeyvalOptions{%
  family=enparen,%
  prefix=enparen@,%
}
%    \end{macrocode}
%
%    \begin{macro}{\enparenSetup}
%    \begin{macrocode}
\ProtectedDef*{\enparenSetup}[0]{%
  \kvsetkeys{enparen}%
}
%    \end{macrocode}
%    \end{macro}
%
%    \begin{macrocode}
\DeclareBoolOption[true]{footnote}
\DeclareBoolOption[true]{caption}
\ProcessKeyvalOptions*
%    \end{macrocode}
%
% \subsection{Context settings}
%
%    \begin{macro}{\enparen@AtBegin}
%    \begin{macrocode}
\def\enparen@AtBegin{%
  \ifenparen@footnote
    \let\enparen@org@makefntext\@makefntext
    \long\def\@makefntext##1{%
      \enparen@org@makefntext{%
        \enparenBeginContext{footnote}%
        ##1%
        \enparenEndContext{footnote}%
      }%
    }%
  \fi
  \enparen@Disable{footnote}%
  \ifenparen@caption
    \let\enparen@org@makecaption\@makecaption
    \long\def\@makecaption##1##2{%
      \enparen@org@makecaption{##1}{%
        \enparenBeginContext{caption}%
        ##2%
        \enparenEndContext{caption}%
      }%
    }%
  \fi
  \enparen@Disable{caption}%
}
%    \end{macrocode}
%    \end{macro}
%
%    \begin{macro}{\enparen@Disable}
%    \begin{macrocode}
\def\enparen@Disable#1{%
  \DisableKeyvalOption[%
    action=warning,%
    package=enparen,%
  ]{enparen}{#1}%
}
%    \end{macrocode}
%    \end{macro}
%
%    \begin{macrocode}
\AtBeginDocument{\enparen@AtBegin}
%    \end{macrocode}
%
% \subsection{At end of document}
%
%    \begin{macro}{\enparen@AtEnd}
%    \begin{macrocode}
\def\enparen@AtEnd{%
  \enparenCheckEmptyStack
  \ifx\enparen@CtxStack\ltx@empty
  \else
    \PackageWarningNoLine{enparen}{%
      Context stack is not empty at end of document.\MessageBreak
      Current stack and contents of context stack:\MessageBreak
      [\enparen@ctx]:[\enparen@PrintStack]%
      \expandafter
      \enparen@PrintContextStack\enparen@CtxStack\relax\relax
    }%
  \fi
}
%    \end{macrocode}
%    \end{macro}
%    \begin{macro}{\enparen@PrintContextStack}
%    \begin{macrocode}
\def\enparen@PrintContextStack#1#2{%
  \ifx\relax#1\ltx@empty
  \else
    \MessageBreak
    [#1]:[\enparen@PrintStackAux#2\ltx@empty\ltx@empty]% hash-ok
    \expandafter\enparen@PrintContextStack
  \fi
}
%    \end{macrocode}
%    \end{macro}
%    \begin{macrocode}
\AtVeryEndDocument{\enparen@AtEnd}
%    \end{macrocode}
%
%    \begin{macrocode}
%</package>
%    \end{macrocode}
%
% \section{Installation}
%
% \subsection{Download}
%
% \paragraph{Package.} This package is available on
% CTAN\footnote{\url{ftp://ftp.ctan.org/tex-archive/}}:
% \begin{description}
% \item[\CTAN{macros/latex/contrib/oberdiek/enparen.dtx}] The source file.
% \item[\CTAN{macros/latex/contrib/oberdiek/enparen.pdf}] Documentation.
% \end{description}
%
%
% \paragraph{Bundle.} All the packages of the bundle `oberdiek'
% are also available in a TDS compliant ZIP archive. There
% the packages are already unpacked and the documentation files
% are generated. The files and directories obey the TDS standard.
% \begin{description}
% \item[\CTAN{install/macros/latex/contrib/oberdiek.tds.zip}]
% \end{description}
% \emph{TDS} refers to the standard ``A Directory Structure
% for \TeX\ Files'' (\CTAN{tds/tds.pdf}). Directories
% with \xfile{texmf} in their name are usually organized this way.
%
% \subsection{Bundle installation}
%
% \paragraph{Unpacking.} Unpack the \xfile{oberdiek.tds.zip} in the
% TDS tree (also known as \xfile{texmf} tree) of your choice.
% Example (linux):
% \begin{quote}
%   |unzip oberdiek.tds.zip -d ~/texmf|
% \end{quote}
%
% \paragraph{Script installation.}
% Check the directory \xfile{TDS:scripts/oberdiek/} for
% scripts that need further installation steps.
% Package \xpackage{attachfile2} comes with the Perl script
% \xfile{pdfatfi.pl} that should be installed in such a way
% that it can be called as \texttt{pdfatfi}.
% Example (linux):
% \begin{quote}
%   |chmod +x scripts/oberdiek/pdfatfi.pl|\\
%   |cp scripts/oberdiek/pdfatfi.pl /usr/local/bin/|
% \end{quote}
%
% \subsection{Package installation}
%
% \paragraph{Unpacking.} The \xfile{.dtx} file is a self-extracting
% \docstrip\ archive. The files are extracted by running the
% \xfile{.dtx} through \plainTeX:
% \begin{quote}
%   \verb|tex enparen.dtx|
% \end{quote}
%
% \paragraph{TDS.} Now the different files must be moved into
% the different directories in your installation TDS tree
% (also known as \xfile{texmf} tree):
% \begin{quote}
% \def\t{^^A
% \begin{tabular}{@{}>{\ttfamily}l@{ $\rightarrow$ }>{\ttfamily}l@{}}
%   enparen.sty & tex/latex/oberdiek/enparen.sty\\
%   enparen.pdf & doc/latex/oberdiek/enparen.pdf\\
%   enparen.dtx & source/latex/oberdiek/enparen.dtx\\
% \end{tabular}^^A
% }^^A
% \sbox0{\t}^^A
% \ifdim\wd0>\linewidth
%   \begingroup
%     \advance\linewidth by\leftmargin
%     \advance\linewidth by\rightmargin
%   \edef\x{\endgroup
%     \def\noexpand\lw{\the\linewidth}^^A
%   }\x
%   \def\lwbox{^^A
%     \leavevmode
%     \hbox to \linewidth{^^A
%       \kern-\leftmargin\relax
%       \hss
%       \usebox0
%       \hss
%       \kern-\rightmargin\relax
%     }^^A
%   }^^A
%   \ifdim\wd0>\lw
%     \sbox0{\small\t}^^A
%     \ifdim\wd0>\linewidth
%       \ifdim\wd0>\lw
%         \sbox0{\footnotesize\t}^^A
%         \ifdim\wd0>\linewidth
%           \ifdim\wd0>\lw
%             \sbox0{\scriptsize\t}^^A
%             \ifdim\wd0>\linewidth
%               \ifdim\wd0>\lw
%                 \sbox0{\tiny\t}^^A
%                 \ifdim\wd0>\linewidth
%                   \lwbox
%                 \else
%                   \usebox0
%                 \fi
%               \else
%                 \lwbox
%               \fi
%             \else
%               \usebox0
%             \fi
%           \else
%             \lwbox
%           \fi
%         \else
%           \usebox0
%         \fi
%       \else
%         \lwbox
%       \fi
%     \else
%       \usebox0
%     \fi
%   \else
%     \lwbox
%   \fi
% \else
%   \usebox0
% \fi
% \end{quote}
% If you have a \xfile{docstrip.cfg} that configures and enables \docstrip's
% TDS installing feature, then some files can already be in the right
% place, see the documentation of \docstrip.
%
% \subsection{Refresh file name databases}
%
% If your \TeX~distribution
% (\teTeX, \mikTeX, \dots) relies on file name databases, you must refresh
% these. For example, \teTeX\ users run \verb|texhash| or
% \verb|mktexlsr|.
%
% \subsection{Some details for the interested}
%
% \paragraph{Attached source.}
%
% The PDF documentation on CTAN also includes the
% \xfile{.dtx} source file. It can be extracted by
% AcrobatReader 6 or higher. Another option is \textsf{pdftk},
% e.g. unpack the file into the current directory:
% \begin{quote}
%   \verb|pdftk enparen.pdf unpack_files output .|
% \end{quote}
%
% \paragraph{Unpacking with \LaTeX.}
% The \xfile{.dtx} chooses its action depending on the format:
% \begin{description}
% \item[\plainTeX:] Run \docstrip\ and extract the files.
% \item[\LaTeX:] Generate the documentation.
% \end{description}
% If you insist on using \LaTeX\ for \docstrip\ (really,
% \docstrip\ does not need \LaTeX), then inform the autodetect routine
% about your intention:
% \begin{quote}
%   \verb|latex \let\install=y% \iffalse meta-comment
%
% File: enparen.dtx
% Version: 2012/01/07 v1.0
% Info: Parentheses nesting
%
% Copyright (C) 2012 by
%    Heiko Oberdiek <heiko.oberdiek at googlemail.com>
%
% This work may be distributed and/or modified under the
% conditions of the LaTeX Project Public License, either
% version 1.3c of this license or (at your option) any later
% version. This version of this license is in
%    http://www.latex-project.org/lppl/lppl-1-3c.txt
% and the latest version of this license is in
%    http://www.latex-project.org/lppl.txt
% and version 1.3 or later is part of all distributions of
% LaTeX version 2005/12/01 or later.
%
% This work has the LPPL maintenance status "maintained".
%
% This Current Maintainer of this work is Heiko Oberdiek.
%
% This work consists of the main source file enparen.dtx
% and the derived files
%    enparen.sty, enparen.pdf, enparen.ins, enparen.drv.
%
% Distribution:
%    CTAN:macros/latex/contrib/oberdiek/enparen.dtx
%    CTAN:macros/latex/contrib/oberdiek/enparen.pdf
%
% Unpacking:
%    (a) If enparen.ins is present:
%           tex enparen.ins
%    (b) Without enparen.ins:
%           tex enparen.dtx
%    (c) If you insist on using LaTeX
%           latex \let\install=y\input{enparen.dtx}
%        (quote the arguments according to the demands of your shell)
%
% Documentation:
%    (a) If enparen.drv is present:
%           latex enparen.drv
%    (b) Without enparen.drv:
%           latex enparen.dtx; ...
%    The class ltxdoc loads the configuration file ltxdoc.cfg
%    if available. Here you can specify further options, e.g.
%    use A4 as paper format:
%       \PassOptionsToClass{a4paper}{article}
%
%    Programm calls to get the documentation (example):
%       pdflatex enparen.dtx
%       makeindex -s gind.ist enparen.idx
%       pdflatex enparen.dtx
%       makeindex -s gind.ist enparen.idx
%       pdflatex enparen.dtx
%
% Installation:
%    TDS:tex/latex/oberdiek/enparen.sty
%    TDS:doc/latex/oberdiek/enparen.pdf
%    TDS:source/latex/oberdiek/enparen.dtx
%
%<*ignore>
\begingroup
  \catcode123=1 %
  \catcode125=2 %
  \def\x{LaTeX2e}%
\expandafter\endgroup
\ifcase 0\ifx\install y1\fi\expandafter
         \ifx\csname processbatchFile\endcsname\relax\else1\fi
         \ifx\fmtname\x\else 1\fi\relax
\else\csname fi\endcsname
%</ignore>
%<*install>
\input docstrip.tex
\Msg{************************************************************************}
\Msg{* Installation}
\Msg{* Package: enparen 2012/01/07 v1.0 Parentheses nesting (HO)}
\Msg{************************************************************************}

\keepsilent
\askforoverwritefalse

\let\MetaPrefix\relax
\preamble

This is a generated file.

Project: enparen
Version: 2012/01/07 v1.0

Copyright (C) 2012 by
   Heiko Oberdiek <heiko.oberdiek at googlemail.com>

This work may be distributed and/or modified under the
conditions of the LaTeX Project Public License, either
version 1.3c of this license or (at your option) any later
version. This version of this license is in
   http://www.latex-project.org/lppl/lppl-1-3c.txt
and the latest version of this license is in
   http://www.latex-project.org/lppl.txt
and version 1.3 or later is part of all distributions of
LaTeX version 2005/12/01 or later.

This work has the LPPL maintenance status "maintained".

This Current Maintainer of this work is Heiko Oberdiek.

This work consists of the main source file enparen.dtx
and the derived files
   enparen.sty, enparen.pdf, enparen.ins, enparen.drv.

\endpreamble
\let\MetaPrefix\DoubleperCent

\generate{%
  \file{enparen.ins}{\from{enparen.dtx}{install}}%
  \file{enparen.drv}{\from{enparen.dtx}{driver}}%
  \usedir{tex/latex/oberdiek}%
  \file{enparen.sty}{\from{enparen.dtx}{package}}%
}

\catcode32=13\relax% active space
\let =\space%
\Msg{************************************************************************}
\Msg{*}
\Msg{* To finish the installation you have to move the following}
\Msg{* file into a directory searched by TeX:}
\Msg{*}
\Msg{*     enparen.sty}
\Msg{*}
\Msg{* To produce the documentation run the file `enparen.drv'}
\Msg{* through LaTeX.}
\Msg{*}
\Msg{* Happy TeXing!}
\Msg{*}
\Msg{************************************************************************}

\endbatchfile
%</install>
%<*ignore>
\fi
%</ignore>
%<*driver>
\NeedsTeXFormat{LaTeX2e}
\ProvidesFile{enparen.drv}%
  [2012/01/07 v1.0 Parentheses nesting (HO)]%
\documentclass{ltxdoc}
\usepackage{holtxdoc}[2011/11/22]
\usepackage{enparen}[2012/01/07]
\begin{document}
  \DocInput{enparen.dtx}%
\end{document}
%</driver>
% \fi
%
% \CheckSum{339}
%
% \CharacterTable
%  {Upper-case    \A\B\C\D\E\F\G\H\I\J\K\L\M\N\O\P\Q\R\S\T\U\V\W\X\Y\Z
%   Lower-case    \a\b\c\d\e\f\g\h\i\j\k\l\m\n\o\p\q\r\s\t\u\v\w\x\y\z
%   Digits        \0\1\2\3\4\5\6\7\8\9
%   Exclamation   \!     Double quote  \"     Hash (number) \#
%   Dollar        \$     Percent       \%     Ampersand     \&
%   Acute accent  \'     Left paren    \(     Right paren   \)
%   Asterisk      \*     Plus          \+     Comma         \,
%   Minus         \-     Point         \.     Solidus       \/
%   Colon         \:     Semicolon     \;     Less than     \<
%   Equals        \=     Greater than  \>     Question mark \?
%   Commercial at \@     Left bracket  \[     Backslash     \\
%   Right bracket \]     Circumflex    \^     Underscore    \_
%   Grave accent  \`     Left brace    \{     Vertical bar  \|
%   Right brace   \}     Tilde         \~}
%
% \GetFileInfo{enparen.drv}
%
% \title{The \xpackage{enparen} package}
% \date{2012/01/07 v1.0}
% \author{Heiko Oberdiek\\\xemail{heiko.oberdiek at googlemail.com}}
%
% \maketitle
%
% \begin{abstract}
% The package defines macros to set parentheses that automatically
% change the symbols from inner to outer fences.
% \end{abstract}
%
% \tableofcontents
%
% \section{Documentation}
%
% The \hologo{LaTeX} package provides macros for automatically changed
% parentheses symbols depending on the fence order.
% The innermost parentheses are always using the same
% parentheses. The symbols changes for the outer fences.
%
% Example:
% \begin{quote}
%   \enparen{a \enparen{b} \enparen{c \enparen{d}} \enparen{e}}
% \end{quote}
% is generated by
%\begin{quote}
%\begin{verbatim}
%\documentclass{article}
%\usepackage{enparen}
%\begin{document}
%  \enparen{a \enparen{b} \enparen{c \enparen{d}} \enparen{e}}
%\end{document}
%\end{verbatim}
%\end{quote}
%
% The package is the result of a newsgroup thread. Dan gives
% the following specification \cite{dan}:
% \begin{quote}
% ``On the other hand, the rules for fences are usually the reverse:
% innermost fences are always |()|, next outer are |[]|, etc. This means
% the opening fence has to wait until all the fences between it and
% the matching close have been detected before it can decide whether
% to be |(| or |[| or |{|.''
% \end{quote}
%
% The fence level counting starts from innermost parentheses with one.
% For the next outer fences the level is increased by one.
% The example above with level indexes:
% \begin{quote}
%   \def\1{\ensuremath{_{1}}}
%   \def\2{\ensuremath{_{2}}}
%   \def\3{\ensuremath{_{3}}}
%   \enparen{\3a \enparen{\1b}\1 \enparen{\2c \enparen{\1d}\1}\2 \enparen{e}}\3
% \end{quote}
% The correct level is only known at the closing symbol. Therefore the
% correct value is remembered in the main \xext{aux} file and used
% in the second \hologo{LaTeX} run.
%
% \subsection{User macros}
%
% \begin{declcs}{enparen} \M{text}
% \end{declcs}
% The macro \cs{enparen} puts its argument \meta{text} in parentheses.
% If the macro is nested, the used fence symbols change for the
% outer fences.
%
% \begin{declcs}{enparenLeft}\\
%   \cs{enparenRight}
% \end{declcs}
% Instead of |\enparen{text}| the left and right symbol can be
% used separately, but in pairs:
% \begin{quote}
%   |\enparenLeft text\enparenRight|
% \end{quote}
% \cs{enparenLeft} and \cs{enparenRight} may be used at different
% group levels, but they must be properly nested.
%
% \begin{declcs}{enparenSetSymbols} \M{level} \M{opening symbol} \M{closing symbol}
% \end{declcs}
% Macro \cs{enparenSetSymbols} configures the \meta{opening symbol} and
% \meta{closing symbol} for the \meta{level}. The \meta{level}
% is a number and the counting starts with one. Level zero is used,
% if the correct level is not known (e.g. in the first \hologo{LaTeX} run).
% The package defines the following sets:
%\begin{quote}
%\begin{verbatim}
%\enparenSetSymbols{0}{(}{)}
%\enparenSetSymbols{1}{(}{)}
%\enparenSetSymbols{2}{[}{]}
%\enparenSetSymbols{3}{\{}{\}}
%\end{verbatim}
%\end{quote}
% Example for changing the third and adding a fourth level:
%\begin{quote}
%\begin{verbatim}
%\enparenSetSymbols{3}{\ensuremath{\langle}}{\ensuremath{\rangle}}
%\enparenSetSymbols{4}{\{}{\}}
%\end{verbatim}
%\end{quote}
%
% \begin{declcs}{enparenUnsetSymbols} \M{level}
% \end{declcs}
% The symbols for level \meta{level} are removed.
% Example scenario: Only two nesting levels must be used,
% the package defines more, then the third level can be
% disabled by |\enparenUnsetSymbols{3}| and the user
% gets warnings if parentheses at level 3 are needed.
%
% \subsection{Contexts}
%
% \begin{declcs}{enparenBeginContext} \M{name}\\
%   \cs{enparenEndContext} \M{name}
% \end{declcs}
% If the current text is interrupted by footnotes, floats
% with captions, then the parentheses inside the text of footnotes,
% captions, \dots should restart from scratch. This can be achieved
% by embedding the text inside macros \cs{enparenBeginContext}
% and \cs{enparenEndContext}. These macros must be properly nested.
% The \meta{name} for the begin and end macro must be the same.
% It is a help for debugging problems, because the warning messages
% show the context name. But it is not necessary that the
% begin\slash end pairs have different names. Example:
%\begin{quote}
%\begin{verbatim}
%\enparenLeft text before table ...
%\begin{table}
%  \caption{Table caption}
%  \enparenBegin{Context}{table}
%  Other text \enparen{foobar}.
%  \enparenEnd{Context}{table}
%\end{table}
%text after table ...
%\enparenRight
%\end{verbatim}
%\end{quote}
% The parentheses inside the table environment and context `table'
% are not nested inside other parentheses: \enparen{foobar}.
% In case of captions and footnotes the contexts are automatically
% added, see next section about options.
%
% \subsection{Options}
%
% \begin{declcs}{enparenSetup} \M{key value list}
% \end{declcs}
% Some options (currently all) can also be set after the
% package is loaded. They can be set in the argument
% \meta{key value list} of \cs{enparenSetup}. Options are disabled
% after they are used the last time. Currently all options are boolean
% options and are disabled in |\begin{document}|.
%
% \begin{description}
% \item[\xoption{caption}:] The caption text is put in a
%   context |caption|.
% \item[\xoption{footnote}:] The footnote text is put in a
%   context |footnote|.
% \end{description}
%
% Example for disabling the two options at different places:
%\begin{quote}
%\begin{verbatim}
%\usepackage[caption=false]{enparen}
%\enparenSetup{footnote=false}
%\end{verbatim}
%\end{quote}
%
% \subsection{Notes}
%
% \begin{description}
% \item[Implicite kerning:]
%   Unexpandable stuff might affect the implicite kerning.
%   The package cannot avoid this, because it need to define
%   and redefine macros at the occurence of each symbol.
%   This is done before the opening and after the closing
%   symbol, thus that the implicite kerning inside
%   is not affected.
% \end{description}
%
% \StopEventually{
% }
%
% \section{Implementation}
%
%    \begin{macrocode}
%<*package>
\NeedsTeXFormat{LaTeX2e}
\ProvidesPackage{enparen}
  [2012/01/07 v1.0 Parentheses nesting (HO)]%
%    \end{macrocode}
%
% \subsection{Resources}
%
%    \begin{macrocode}
\RequirePackage{ltxcmds}[2011/11/09]
%    \end{macrocode}
%
%    \begin{macrocode}
\ltx@IfUndefined{numexpr}{%
  \PackageError{enparen}{%
    Missing e-TeX's \ltx@backslashchar numexpr.\MessageBreak
    The package will continue with emergency definitions%
  }\@ehc
  \def\enparenLeft{(}%
  \def\enparenRight{)}%
  \long\def\enparen#1{\enparenLeft#1\enparenRight}%
  \let\enparenSetup\ltx@gobble
  \let\enparenSetSymbols\ltx@gobblethree
  \let\enparenUnsetSymbols\ltx@gobble
  \endinput
}{}
%    \end{macrocode}
%
%    \begin{macrocode}
\RequirePackage{protecteddef}[2011/01/31]
\RequirePackage{atveryend}[2011/06/30]
\RequirePackage{uniquecounter}[2011/01/30]
\RequirePackage{zref-base}[2011/03/18]
\RequirePackage{kvoptions}[2011/06/30]
\RequirePackage{kvsetkeys}[2011/10/18]
%    \end{macrocode}
%
%    \begin{macro}{\zref@wrapper@mainaux}
%    \begin{macrocode}
\providecommand{\zref@wrapper@mainaux}[1]{%
  \ifx\@auxout\@mainaux
    #1%
  \else
    \begingroup
      \let\@auxout\@mainaux
      #1%
    \endgroup
  \fi
}
%    \end{macrocode}
%    \end{macro}
%
%    \begin{macrocode}
\UniqueCounterNew{enparen}
%    \end{macrocode}
%
%    \begin{macrocode}
\zref@newprop{enparen}[0]{}
%    \end{macrocode}
%
% \subsection{Contexts}
%
%    \begin{macro}{\enparenContextDefault}
%    \begin{macrocode}
\def\enparenContextDefault{main}
%    \end{macrocode}
%    \end{macro}
%
%    \begin{macro}{\enparen@ctx}
%    \begin{macrocode}
\let\enparen@ctx\ltx@empty
%    \end{macrocode}
%    \end{macro}
%
%    \begin{macro}{\enparen@stack}
%    \begin{macrocode}
\let\enparen@stack\ltx@empty
%    \end{macrocode}
%    \end{macro}
%
% \subsubsection{Stack for contexts}
%
%    \begin{macro}{\enparen@CtxStack}
%    \begin{macrocode}
\global\let\enparen@CtxStack\ltx@empty
%    \end{macrocode}
%    \end{macro}
%
%    \begin{macro}{\enparen@CtxStackPush}
%    \begin{macrocode}
\def\enparen@CtxStackPush#1{%
  \xdef\enparen@CtxStack{%
    {\enparen@ctx}{\enparen@stack}%
    \enparen@CtxStack
  }%
  \xdef\enparen@ctx{#1}%
  \global\let\enparen@stack\ltx@empty
}
%    \end{macrocode}
%    \end{macro}
%
%    \begin{macro}{\enparen@CtxStackPop}
%    \begin{macrocode}
\def\enparen@CtxStackPop{%
  \ifx\enparen@CtxStack\ltx@empty
    \PackageWarning{enparen}{%
      Pop request for empty context stack%
    }%
    \global\let\enparen@ctx\enparenContextDefault
    \global\let\enparen@stack\ltx@empty
  \else
    \xdef\enparen@ctx{%
      \expandafter\ltx@car\enparen@CtxStack\@nil
    }%
    \xdef\enparen@stack{%
      \expandafter\ltx@carsecond\enparen@CtxStack\@nil
    }%
    \xdef\enparen@CtxStack{%
      \expandafter\ltx@cdrtwo\enparen@CtxStack\@nil
    }%
  \fi
}
%    \end{macrocode}
%    \end{macro}
%
% \subsubsection{Context user macros}
%
%    \begin{macro}{\enparenBeginContext}
%    \begin{macrocode}
\ProtectedDef*{\enparenBeginContext}[1]{%
  \enparen@CtxStackPush{#1}%
}
%    \end{macrocode}
%    \end{macro}
%    \begin{macro}{\enparenEndContext}
%    \begin{macrocode}
\ProtectedDef*{\enparenEndContext}[1]{%
  \edef\enparen@temp{#1}%
  \ifx\enparen@temp\enparen@ctx
  \else
    \PackageWarning{enparen}{%
      Context mismatch in end request.\MessageBreak
      `#1' should be ended, but current context\MessageBreak
      is `\enparen@ctx'%
    }%
  \fi
  \enparenCheckEmptyStack
  \enparen@CtxStackPop
}
%    \end{macrocode}
%    \end{macro}
%
%    \begin{macro}{\enparenCheckEmptyStack}
%    \begin{macrocode}
\ProtectedDef*{\enparenCheckEmptyStack}[0]{%
  \ifx\enparen@stack\ltx@empty
  \else
    \PackageWarning{enparen}{%
      Ending non-empty stack `\enparen@ctx':\MessageBreak
      \enparen@PrintStack\MessageBreak
    }%
  \fi
}
%    \end{macrocode}
%    \end{macro}
%
%    \begin{macro}{\enparen@PrintStack}
%    \begin{macrocode}
\def\enparen@PrintStack{%
  \expandafter\enparen@PrintStackAux
  \enparen@stack\ltx@empty\ltx@empty
}
%    \end{macrocode}
%    \end{macro}
%    \begin{macro}{\enparen@PrintStackAux}
%    \begin{macrocode}
\def\enparen@PrintStackAux#1#2{%
  \ifx\ltx@empty#1%
  \else
    {#1:#2}%
    \expandafter\enparen@PrintStackAux
  \fi
}
%    \end{macrocode}
%    \end{macro}
%
% \subsection{Symbols}
%
%    \begin{macro}{\enparenSetSymbols}
%    \begin{macrocode}
\ProtectedDef*{\enparenSetSymbols}[3]{%
  \expandafter
  \def\csname enparen@symbol\the\numexpr#1L\endcsname{#2}%
  \expandafter
  \def\csname enparen@symbol\the\numexpr#1R\endcsname{#3}%
}
%    \end{macrocode}
%    \end{macro}
%    \begin{macro}{\enparenUnsetSymbols}
%    \begin{macrocode}
\ProtectedDef*{\enparenUnsetSymbols}[1]{%
  \expandafter
  \let\csname enparen@symbol\the\numexpr#1L\endcsname\ltx@undefined
  \expandafter
  \let\csname enparen@symbol\the\numexpr#1R\endcsname\ltx@undefined
}
%    \end{macrocode}
%    \end{macro}
%
%    \begin{macrocode}
\enparenSetSymbols{0}{(}{)}
\enparenSetSymbols{1}{(}{)}
\enparenSetSymbols{2}{[}{]}
\enparenSetSymbols{3}{\{}{\}}
\enparenSetSymbols{4}{\ensuremath{\langle}}{\ensuremath{\rangle}}
%    \end{macrocode}
%
% \subsection{Main user macros}
%
%    \begin{macro}{\enparen}
%    \begin{macrocode}
\ProtectedDef{\enparen}[1]{%
  \enparenLeft#1\enparenRight
}
%    \end{macrocode}
%    \end{macro}
%
%    \begin{macro}{\enparenLeft}
%    \begin{macrocode}
\ProtectedDef*{\enparenLeft}[0]{%
  \UniqueCounterCall{enparen}\enparen@Left
}
%    \end{macrocode}
%    \end{macro}
%    \begin{macro}{\enparen@Left}
%    \begin{macrocode}
\def\enparen@Left#1{%
  \xdef\enparen@stack{%
    {#1}{1}%
    \expandafter\enparen@Inc\expandafter2\expandafter!%
    \enparen@stack\ltx@empty\ltx@empty
  }%
  \edef\enparen@tmp{\zref@extract{enparen#1}{enparen}}%
  \ltx@IfUndefined{enparen@symbol\enparen@tmp L}{%
    \PackageWarning{enparen}{%
      Undefined symbols for level \enparen@tmp
    }%
    \csname enparen@symbol0L\endcsname
  }{%
    \csname enparen@symbol\enparen@tmp L\endcsname
  }%
}
%    \end{macrocode}
%    \end{macro}
%    \begin{macro}{\enparen@Inc}
%    \begin{macrocode}
\def\enparen@Inc#1!#2#3{%
  \ifx\ltx@empty#2%
  \else
    \ifnum#3<#1 %
      {#2}{#1}%
      \expandafter\enparen@Inc
      \the\numexpr#1+1\expandafter\expandafter\expandafter!%
    \else
      {#2}{#3}%
    \fi
  \fi
}
%    \end{macrocode}
%    \end{macro}
%
%    \begin{macro}{\enparenRight}
%    \begin{macrocode}
\ProtectedDef*{\enparenRight}[0]{%
  \ifx\enparen@stack\ltx@empty
    \PackageWarning{enparen}{%
      Missing left symbol for right symbol%
    }%
    \csname enparen@symbol0R\endcsname
  \else
    \expandafter\enparen@Right\enparen@stack\@nil
  \fi
}
%    \end{macrocode}
%    \end{macro}
%
%    \begin{macro}{\enparen@Right}
%    \begin{macrocode}
\def\enparen@Right#1#2#3\@nil{%
  \ltx@IfUndefined{%
    enparen@symbol%
    \zref@extract{enparen#1}{enparen}%
    R%
  }{%
    \csname enparen@symbol0R\endcsname
  }{%
    \csname
      enparen@symbol%
      \zref@extract{enparen#1}{enparen}%
       R%
    \endcsname
  }%
  \zref@wrapper@mainaux{%
    \zref@setcurrent{enparen}{#2}%
    \zref@labelbyprops{enparen#1}{enparen}%
  }%
  \xdef\enparen@stack{#3}%
}
%    \end{macrocode}
%    \end{macro}
%
% \subsection{Options}
%
%    \begin{macrocode}
\SetupKeyvalOptions{%
  family=enparen,%
  prefix=enparen@,%
}
%    \end{macrocode}
%
%    \begin{macro}{\enparenSetup}
%    \begin{macrocode}
\ProtectedDef*{\enparenSetup}[0]{%
  \kvsetkeys{enparen}%
}
%    \end{macrocode}
%    \end{macro}
%
%    \begin{macrocode}
\DeclareBoolOption[true]{footnote}
\DeclareBoolOption[true]{caption}
\ProcessKeyvalOptions*
%    \end{macrocode}
%
% \subsection{Context settings}
%
%    \begin{macro}{\enparen@AtBegin}
%    \begin{macrocode}
\def\enparen@AtBegin{%
  \ifenparen@footnote
    \let\enparen@org@makefntext\@makefntext
    \long\def\@makefntext##1{%
      \enparen@org@makefntext{%
        \enparenBeginContext{footnote}%
        ##1%
        \enparenEndContext{footnote}%
      }%
    }%
  \fi
  \enparen@Disable{footnote}%
  \ifenparen@caption
    \let\enparen@org@makecaption\@makecaption
    \long\def\@makecaption##1##2{%
      \enparen@org@makecaption{##1}{%
        \enparenBeginContext{caption}%
        ##2%
        \enparenEndContext{caption}%
      }%
    }%
  \fi
  \enparen@Disable{caption}%
}
%    \end{macrocode}
%    \end{macro}
%
%    \begin{macro}{\enparen@Disable}
%    \begin{macrocode}
\def\enparen@Disable#1{%
  \DisableKeyvalOption[%
    action=warning,%
    package=enparen,%
  ]{enparen}{#1}%
}
%    \end{macrocode}
%    \end{macro}
%
%    \begin{macrocode}
\AtBeginDocument{\enparen@AtBegin}
%    \end{macrocode}
%
% \subsection{At end of document}
%
%    \begin{macro}{\enparen@AtEnd}
%    \begin{macrocode}
\def\enparen@AtEnd{%
  \enparenCheckEmptyStack
  \ifx\enparen@CtxStack\ltx@empty
  \else
    \PackageWarningNoLine{enparen}{%
      Context stack is not empty at end of document.\MessageBreak
      Current stack and contents of context stack:\MessageBreak
      [\enparen@ctx]:[\enparen@PrintStack]%
      \expandafter
      \enparen@PrintContextStack\enparen@CtxStack\relax\relax
    }%
  \fi
}
%    \end{macrocode}
%    \end{macro}
%    \begin{macro}{\enparen@PrintContextStack}
%    \begin{macrocode}
\def\enparen@PrintContextStack#1#2{%
  \ifx\relax#1\ltx@empty
  \else
    \MessageBreak
    [#1]:[\enparen@PrintStackAux#2\ltx@empty\ltx@empty]% hash-ok
    \expandafter\enparen@PrintContextStack
  \fi
}
%    \end{macrocode}
%    \end{macro}
%    \begin{macrocode}
\AtVeryEndDocument{\enparen@AtEnd}
%    \end{macrocode}
%
%    \begin{macrocode}
%</package>
%    \end{macrocode}
%
% \section{Installation}
%
% \subsection{Download}
%
% \paragraph{Package.} This package is available on
% CTAN\footnote{\url{ftp://ftp.ctan.org/tex-archive/}}:
% \begin{description}
% \item[\CTAN{macros/latex/contrib/oberdiek/enparen.dtx}] The source file.
% \item[\CTAN{macros/latex/contrib/oberdiek/enparen.pdf}] Documentation.
% \end{description}
%
%
% \paragraph{Bundle.} All the packages of the bundle `oberdiek'
% are also available in a TDS compliant ZIP archive. There
% the packages are already unpacked and the documentation files
% are generated. The files and directories obey the TDS standard.
% \begin{description}
% \item[\CTAN{install/macros/latex/contrib/oberdiek.tds.zip}]
% \end{description}
% \emph{TDS} refers to the standard ``A Directory Structure
% for \TeX\ Files'' (\CTAN{tds/tds.pdf}). Directories
% with \xfile{texmf} in their name are usually organized this way.
%
% \subsection{Bundle installation}
%
% \paragraph{Unpacking.} Unpack the \xfile{oberdiek.tds.zip} in the
% TDS tree (also known as \xfile{texmf} tree) of your choice.
% Example (linux):
% \begin{quote}
%   |unzip oberdiek.tds.zip -d ~/texmf|
% \end{quote}
%
% \paragraph{Script installation.}
% Check the directory \xfile{TDS:scripts/oberdiek/} for
% scripts that need further installation steps.
% Package \xpackage{attachfile2} comes with the Perl script
% \xfile{pdfatfi.pl} that should be installed in such a way
% that it can be called as \texttt{pdfatfi}.
% Example (linux):
% \begin{quote}
%   |chmod +x scripts/oberdiek/pdfatfi.pl|\\
%   |cp scripts/oberdiek/pdfatfi.pl /usr/local/bin/|
% \end{quote}
%
% \subsection{Package installation}
%
% \paragraph{Unpacking.} The \xfile{.dtx} file is a self-extracting
% \docstrip\ archive. The files are extracted by running the
% \xfile{.dtx} through \plainTeX:
% \begin{quote}
%   \verb|tex enparen.dtx|
% \end{quote}
%
% \paragraph{TDS.} Now the different files must be moved into
% the different directories in your installation TDS tree
% (also known as \xfile{texmf} tree):
% \begin{quote}
% \def\t{^^A
% \begin{tabular}{@{}>{\ttfamily}l@{ $\rightarrow$ }>{\ttfamily}l@{}}
%   enparen.sty & tex/latex/oberdiek/enparen.sty\\
%   enparen.pdf & doc/latex/oberdiek/enparen.pdf\\
%   enparen.dtx & source/latex/oberdiek/enparen.dtx\\
% \end{tabular}^^A
% }^^A
% \sbox0{\t}^^A
% \ifdim\wd0>\linewidth
%   \begingroup
%     \advance\linewidth by\leftmargin
%     \advance\linewidth by\rightmargin
%   \edef\x{\endgroup
%     \def\noexpand\lw{\the\linewidth}^^A
%   }\x
%   \def\lwbox{^^A
%     \leavevmode
%     \hbox to \linewidth{^^A
%       \kern-\leftmargin\relax
%       \hss
%       \usebox0
%       \hss
%       \kern-\rightmargin\relax
%     }^^A
%   }^^A
%   \ifdim\wd0>\lw
%     \sbox0{\small\t}^^A
%     \ifdim\wd0>\linewidth
%       \ifdim\wd0>\lw
%         \sbox0{\footnotesize\t}^^A
%         \ifdim\wd0>\linewidth
%           \ifdim\wd0>\lw
%             \sbox0{\scriptsize\t}^^A
%             \ifdim\wd0>\linewidth
%               \ifdim\wd0>\lw
%                 \sbox0{\tiny\t}^^A
%                 \ifdim\wd0>\linewidth
%                   \lwbox
%                 \else
%                   \usebox0
%                 \fi
%               \else
%                 \lwbox
%               \fi
%             \else
%               \usebox0
%             \fi
%           \else
%             \lwbox
%           \fi
%         \else
%           \usebox0
%         \fi
%       \else
%         \lwbox
%       \fi
%     \else
%       \usebox0
%     \fi
%   \else
%     \lwbox
%   \fi
% \else
%   \usebox0
% \fi
% \end{quote}
% If you have a \xfile{docstrip.cfg} that configures and enables \docstrip's
% TDS installing feature, then some files can already be in the right
% place, see the documentation of \docstrip.
%
% \subsection{Refresh file name databases}
%
% If your \TeX~distribution
% (\teTeX, \mikTeX, \dots) relies on file name databases, you must refresh
% these. For example, \teTeX\ users run \verb|texhash| or
% \verb|mktexlsr|.
%
% \subsection{Some details for the interested}
%
% \paragraph{Attached source.}
%
% The PDF documentation on CTAN also includes the
% \xfile{.dtx} source file. It can be extracted by
% AcrobatReader 6 or higher. Another option is \textsf{pdftk},
% e.g. unpack the file into the current directory:
% \begin{quote}
%   \verb|pdftk enparen.pdf unpack_files output .|
% \end{quote}
%
% \paragraph{Unpacking with \LaTeX.}
% The \xfile{.dtx} chooses its action depending on the format:
% \begin{description}
% \item[\plainTeX:] Run \docstrip\ and extract the files.
% \item[\LaTeX:] Generate the documentation.
% \end{description}
% If you insist on using \LaTeX\ for \docstrip\ (really,
% \docstrip\ does not need \LaTeX), then inform the autodetect routine
% about your intention:
% \begin{quote}
%   \verb|latex \let\install=y\input{enparen.dtx}|
% \end{quote}
% Do not forget to quote the argument according to the demands
% of your shell.
%
% \paragraph{Generating the documentation.}
% You can use both the \xfile{.dtx} or the \xfile{.drv} to generate
% the documentation. The process can be configured by the
% configuration file \xfile{ltxdoc.cfg}. For instance, put this
% line into this file, if you want to have A4 as paper format:
% \begin{quote}
%   \verb|\PassOptionsToClass{a4paper}{article}|
% \end{quote}
% An example follows how to generate the
% documentation with pdf\LaTeX:
% \begin{quote}
%\begin{verbatim}
%pdflatex enparen.dtx
%makeindex -s gind.ist enparen.idx
%pdflatex enparen.dtx
%makeindex -s gind.ist enparen.idx
%pdflatex enparen.dtx
%\end{verbatim}
% \end{quote}
%
% \begin{thebibliography}{9}
% \raggedright
% \bibitem{dan}
%   Dan Luecking: \textit{Re: bracket order};
%   newsgroup \xnewsgroup{comp.text.tex};
%   message id \texttt{\small$\langle$9b07c9c8-ff92-4cbf-b3a9-84efecfeb506@l24g2000yqm.googlegroups.com$\rangle$}
%   2012-01-05.
%   \url{http://groups.google.com/group/comp.text.tex/msg/8774519da31c2352}
% \end{thebibliography}
%
% \begin{History}
%   \begin{Version}{2012/01/07 v1.0}
%   \item
%     First version.
%   \end{Version}
% \end{History}
%
% \PrintIndex
%
% \Finale
\endinput
|
% \end{quote}
% Do not forget to quote the argument according to the demands
% of your shell.
%
% \paragraph{Generating the documentation.}
% You can use both the \xfile{.dtx} or the \xfile{.drv} to generate
% the documentation. The process can be configured by the
% configuration file \xfile{ltxdoc.cfg}. For instance, put this
% line into this file, if you want to have A4 as paper format:
% \begin{quote}
%   \verb|\PassOptionsToClass{a4paper}{article}|
% \end{quote}
% An example follows how to generate the
% documentation with pdf\LaTeX:
% \begin{quote}
%\begin{verbatim}
%pdflatex enparen.dtx
%makeindex -s gind.ist enparen.idx
%pdflatex enparen.dtx
%makeindex -s gind.ist enparen.idx
%pdflatex enparen.dtx
%\end{verbatim}
% \end{quote}
%
% \begin{thebibliography}{9}
% \raggedright
% \bibitem{dan}
%   Dan Luecking: \textit{Re: bracket order};
%   newsgroup \xnewsgroup{comp.text.tex};
%   message id \texttt{\small$\langle$9b07c9c8-ff92-4cbf-b3a9-84efecfeb506@l24g2000yqm.googlegroups.com$\rangle$}
%   2012-01-05.
%   \url{http://groups.google.com/group/comp.text.tex/msg/8774519da31c2352}
% \end{thebibliography}
%
% \begin{History}
%   \begin{Version}{2012/01/07 v1.0}
%   \item
%     First version.
%   \end{Version}
% \end{History}
%
% \PrintIndex
%
% \Finale
\endinput
|
% \end{quote}
% Do not forget to quote the argument according to the demands
% of your shell.
%
% \paragraph{Generating the documentation.}
% You can use both the \xfile{.dtx} or the \xfile{.drv} to generate
% the documentation. The process can be configured by the
% configuration file \xfile{ltxdoc.cfg}. For instance, put this
% line into this file, if you want to have A4 as paper format:
% \begin{quote}
%   \verb|\PassOptionsToClass{a4paper}{article}|
% \end{quote}
% An example follows how to generate the
% documentation with pdf\LaTeX:
% \begin{quote}
%\begin{verbatim}
%pdflatex enparen.dtx
%makeindex -s gind.ist enparen.idx
%pdflatex enparen.dtx
%makeindex -s gind.ist enparen.idx
%pdflatex enparen.dtx
%\end{verbatim}
% \end{quote}
%
% \begin{thebibliography}{9}
% \raggedright
% \bibitem{dan}
%   Dan Luecking: \textit{Re: bracket order};
%   newsgroup \xnewsgroup{comp.text.tex};
%   message id \texttt{\small$\langle$9b07c9c8-ff92-4cbf-b3a9-84efecfeb506@l24g2000yqm.googlegroups.com$\rangle$}
%   2012-01-05.
%   \url{http://groups.google.com/group/comp.text.tex/msg/8774519da31c2352}
% \end{thebibliography}
%
% \begin{History}
%   \begin{Version}{2012/01/07 v1.0}
%   \item
%     First version.
%   \end{Version}
% \end{History}
%
% \PrintIndex
%
% \Finale
\endinput

%        (quote the arguments according to the demands of your shell)
%
% Documentation:
%    (a) If enparen.drv is present:
%           latex enparen.drv
%    (b) Without enparen.drv:
%           latex enparen.dtx; ...
%    The class ltxdoc loads the configuration file ltxdoc.cfg
%    if available. Here you can specify further options, e.g.
%    use A4 as paper format:
%       \PassOptionsToClass{a4paper}{article}
%
%    Programm calls to get the documentation (example):
%       pdflatex enparen.dtx
%       makeindex -s gind.ist enparen.idx
%       pdflatex enparen.dtx
%       makeindex -s gind.ist enparen.idx
%       pdflatex enparen.dtx
%
% Installation:
%    TDS:tex/latex/oberdiek/enparen.sty
%    TDS:doc/latex/oberdiek/enparen.pdf
%    TDS:source/latex/oberdiek/enparen.dtx
%
%<*ignore>
\begingroup
  \catcode123=1 %
  \catcode125=2 %
  \def\x{LaTeX2e}%
\expandafter\endgroup
\ifcase 0\ifx\install y1\fi\expandafter
         \ifx\csname processbatchFile\endcsname\relax\else1\fi
         \ifx\fmtname\x\else 1\fi\relax
\else\csname fi\endcsname
%</ignore>
%<*install>
\input docstrip.tex
\Msg{************************************************************************}
\Msg{* Installation}
\Msg{* Package: enparen 2012/01/07 v1.0 Parentheses nesting (HO)}
\Msg{************************************************************************}

\keepsilent
\askforoverwritefalse

\let\MetaPrefix\relax
\preamble

This is a generated file.

Project: enparen
Version: 2012/01/07 v1.0

Copyright (C) 2012 by
   Heiko Oberdiek <heiko.oberdiek at googlemail.com>

This work may be distributed and/or modified under the
conditions of the LaTeX Project Public License, either
version 1.3c of this license or (at your option) any later
version. This version of this license is in
   http://www.latex-project.org/lppl/lppl-1-3c.txt
and the latest version of this license is in
   http://www.latex-project.org/lppl.txt
and version 1.3 or later is part of all distributions of
LaTeX version 2005/12/01 or later.

This work has the LPPL maintenance status "maintained".

This Current Maintainer of this work is Heiko Oberdiek.

This work consists of the main source file enparen.dtx
and the derived files
   enparen.sty, enparen.pdf, enparen.ins, enparen.drv.

\endpreamble
\let\MetaPrefix\DoubleperCent

\generate{%
  \file{enparen.ins}{\from{enparen.dtx}{install}}%
  \file{enparen.drv}{\from{enparen.dtx}{driver}}%
  \usedir{tex/latex/oberdiek}%
  \file{enparen.sty}{\from{enparen.dtx}{package}}%
}

\catcode32=13\relax% active space
\let =\space%
\Msg{************************************************************************}
\Msg{*}
\Msg{* To finish the installation you have to move the following}
\Msg{* file into a directory searched by TeX:}
\Msg{*}
\Msg{*     enparen.sty}
\Msg{*}
\Msg{* To produce the documentation run the file `enparen.drv'}
\Msg{* through LaTeX.}
\Msg{*}
\Msg{* Happy TeXing!}
\Msg{*}
\Msg{************************************************************************}

\endbatchfile
%</install>
%<*ignore>
\fi
%</ignore>
%<*driver>
\NeedsTeXFormat{LaTeX2e}
\ProvidesFile{enparen.drv}%
  [2012/01/07 v1.0 Parentheses nesting (HO)]%
\documentclass{ltxdoc}
\usepackage{holtxdoc}[2011/11/22]
\usepackage{enparen}[2012/01/07]
\begin{document}
  \DocInput{enparen.dtx}%
\end{document}
%</driver>
% \fi
%
% \CheckSum{339}
%
% \CharacterTable
%  {Upper-case    \A\B\C\D\E\F\G\H\I\J\K\L\M\N\O\P\Q\R\S\T\U\V\W\X\Y\Z
%   Lower-case    \a\b\c\d\e\f\g\h\i\j\k\l\m\n\o\p\q\r\s\t\u\v\w\x\y\z
%   Digits        \0\1\2\3\4\5\6\7\8\9
%   Exclamation   \!     Double quote  \"     Hash (number) \#
%   Dollar        \$     Percent       \%     Ampersand     \&
%   Acute accent  \'     Left paren    \(     Right paren   \)
%   Asterisk      \*     Plus          \+     Comma         \,
%   Minus         \-     Point         \.     Solidus       \/
%   Colon         \:     Semicolon     \;     Less than     \<
%   Equals        \=     Greater than  \>     Question mark \?
%   Commercial at \@     Left bracket  \[     Backslash     \\
%   Right bracket \]     Circumflex    \^     Underscore    \_
%   Grave accent  \`     Left brace    \{     Vertical bar  \|
%   Right brace   \}     Tilde         \~}
%
% \GetFileInfo{enparen.drv}
%
% \title{The \xpackage{enparen} package}
% \date{2012/01/07 v1.0}
% \author{Heiko Oberdiek\\\xemail{heiko.oberdiek at googlemail.com}}
%
% \maketitle
%
% \begin{abstract}
% The package defines macros to set parentheses that automatically
% change the symbols from inner to outer fences.
% \end{abstract}
%
% \tableofcontents
%
% \section{Documentation}
%
% The \hologo{LaTeX} package provides macros for automatically changed
% parentheses symbols depending on the fence order.
% The innermost parentheses are always using the same
% parentheses. The symbols changes for the outer fences.
%
% Example:
% \begin{quote}
%   \enparen{a \enparen{b} \enparen{c \enparen{d}} \enparen{e}}
% \end{quote}
% is generated by
%\begin{quote}
%\begin{verbatim}
%\documentclass{article}
%\usepackage{enparen}
%\begin{document}
%  \enparen{a \enparen{b} \enparen{c \enparen{d}} \enparen{e}}
%\end{document}
%\end{verbatim}
%\end{quote}
%
% The package is the result of a newsgroup thread. Dan gives
% the following specification \cite{dan}:
% \begin{quote}
% ``On the other hand, the rules for fences are usually the reverse:
% innermost fences are always |()|, next outer are |[]|, etc. This means
% the opening fence has to wait until all the fences between it and
% the matching close have been detected before it can decide whether
% to be |(| or |[| or |{|.''
% \end{quote}
%
% The fence level counting starts from innermost parentheses with one.
% For the next outer fences the level is increased by one.
% The example above with level indexes:
% \begin{quote}
%   \def\1{\ensuremath{_{1}}}
%   \def\2{\ensuremath{_{2}}}
%   \def\3{\ensuremath{_{3}}}
%   \enparen{\3a \enparen{\1b}\1 \enparen{\2c \enparen{\1d}\1}\2 \enparen{e}}\3
% \end{quote}
% The correct level is only known at the closing symbol. Therefore the
% correct value is remembered in the main \xext{aux} file and used
% in the second \hologo{LaTeX} run.
%
% \subsection{User macros}
%
% \begin{declcs}{enparen} \M{text}
% \end{declcs}
% The macro \cs{enparen} puts its argument \meta{text} in parentheses.
% If the macro is nested, the used fence symbols change for the
% outer fences.
%
% \begin{declcs}{enparenLeft}\\
%   \cs{enparenRight}
% \end{declcs}
% Instead of |\enparen{text}| the left and right symbol can be
% used separately, but in pairs:
% \begin{quote}
%   |\enparenLeft text\enparenRight|
% \end{quote}
% \cs{enparenLeft} and \cs{enparenRight} may be used at different
% group levels, but they must be properly nested.
%
% \begin{declcs}{enparenSetSymbols} \M{level} \M{opening symbol} \M{closing symbol}
% \end{declcs}
% Macro \cs{enparenSetSymbols} configures the \meta{opening symbol} and
% \meta{closing symbol} for the \meta{level}. The \meta{level}
% is a number and the counting starts with one. Level zero is used,
% if the correct level is not known (e.g. in the first \hologo{LaTeX} run).
% The package defines the following sets:
%\begin{quote}
%\begin{verbatim}
%\enparenSetSymbols{0}{(}{)}
%\enparenSetSymbols{1}{(}{)}
%\enparenSetSymbols{2}{[}{]}
%\enparenSetSymbols{3}{\{}{\}}
%\end{verbatim}
%\end{quote}
% Example for changing the third and adding a fourth level:
%\begin{quote}
%\begin{verbatim}
%\enparenSetSymbols{3}{\ensuremath{\langle}}{\ensuremath{\rangle}}
%\enparenSetSymbols{4}{\{}{\}}
%\end{verbatim}
%\end{quote}
%
% \begin{declcs}{enparenUnsetSymbols} \M{level}
% \end{declcs}
% The symbols for level \meta{level} are removed.
% Example scenario: Only two nesting levels must be used,
% the package defines more, then the third level can be
% disabled by |\enparenUnsetSymbols{3}| and the user
% gets warnings if parentheses at level 3 are needed.
%
% \subsection{Contexts}
%
% \begin{declcs}{enparenBeginContext} \M{name}\\
%   \cs{enparenEndContext} \M{name}
% \end{declcs}
% If the current text is interrupted by footnotes, floats
% with captions, then the parentheses inside the text of footnotes,
% captions, \dots should restart from scratch. This can be achieved
% by embedding the text inside macros \cs{enparenBeginContext}
% and \cs{enparenEndContext}. These macros must be properly nested.
% The \meta{name} for the begin and end macro must be the same.
% It is a help for debugging problems, because the warning messages
% show the context name. But it is not necessary that the
% begin\slash end pairs have different names. Example:
%\begin{quote}
%\begin{verbatim}
%\enparenLeft text before table ...
%\begin{table}
%  \caption{Table caption}
%  \enparenBegin{Context}{table}
%  Other text \enparen{foobar}.
%  \enparenEnd{Context}{table}
%\end{table}
%text after table ...
%\enparenRight
%\end{verbatim}
%\end{quote}
% The parentheses inside the table environment and context `table'
% are not nested inside other parentheses: \enparen{foobar}.
% In case of captions and footnotes the contexts are automatically
% added, see next section about options.
%
% \subsection{Options}
%
% \begin{declcs}{enparenSetup} \M{key value list}
% \end{declcs}
% Some options (currently all) can also be set after the
% package is loaded. They can be set in the argument
% \meta{key value list} of \cs{enparenSetup}. Options are disabled
% after they are used the last time. Currently all options are boolean
% options and are disabled in |\begin{document}|.
%
% \begin{description}
% \item[\xoption{caption}:] The caption text is put in a
%   context |caption|.
% \item[\xoption{footnote}:] The footnote text is put in a
%   context |footnote|.
% \end{description}
%
% Example for disabling the two options at different places:
%\begin{quote}
%\begin{verbatim}
%\usepackage[caption=false]{enparen}
%\enparenSetup{footnote=false}
%\end{verbatim}
%\end{quote}
%
% \subsection{Notes}
%
% \begin{description}
% \item[Implicite kerning:]
%   Unexpandable stuff might affect the implicite kerning.
%   The package cannot avoid this, because it need to define
%   and redefine macros at the occurence of each symbol.
%   This is done before the opening and after the closing
%   symbol, thus that the implicite kerning inside
%   is not affected.
% \end{description}
%
% \StopEventually{
% }
%
% \section{Implementation}
%
%    \begin{macrocode}
%<*package>
\NeedsTeXFormat{LaTeX2e}
\ProvidesPackage{enparen}
  [2012/01/07 v1.0 Parentheses nesting (HO)]%
%    \end{macrocode}
%
% \subsection{Resources}
%
%    \begin{macrocode}
\RequirePackage{ltxcmds}[2011/11/09]
%    \end{macrocode}
%
%    \begin{macrocode}
\ltx@IfUndefined{numexpr}{%
  \PackageError{enparen}{%
    Missing e-TeX's \ltx@backslashchar numexpr.\MessageBreak
    The package will continue with emergency definitions%
  }\@ehc
  \def\enparenLeft{(}%
  \def\enparenRight{)}%
  \long\def\enparen#1{\enparenLeft#1\enparenRight}%
  \let\enparenSetup\ltx@gobble
  \let\enparenSetSymbols\ltx@gobblethree
  \let\enparenUnsetSymbols\ltx@gobble
  \endinput
}{}
%    \end{macrocode}
%
%    \begin{macrocode}
\RequirePackage{protecteddef}[2011/01/31]
\RequirePackage{atveryend}[2011/06/30]
\RequirePackage{uniquecounter}[2011/01/30]
\RequirePackage{zref-base}[2011/03/18]
\RequirePackage{kvoptions}[2011/06/30]
\RequirePackage{kvsetkeys}[2011/10/18]
%    \end{macrocode}
%
%    \begin{macro}{\zref@wrapper@mainaux}
%    \begin{macrocode}
\providecommand{\zref@wrapper@mainaux}[1]{%
  \ifx\@auxout\@mainaux
    #1%
  \else
    \begingroup
      \let\@auxout\@mainaux
      #1%
    \endgroup
  \fi
}
%    \end{macrocode}
%    \end{macro}
%
%    \begin{macrocode}
\UniqueCounterNew{enparen}
%    \end{macrocode}
%
%    \begin{macrocode}
\zref@newprop{enparen}[0]{}
%    \end{macrocode}
%
% \subsection{Contexts}
%
%    \begin{macro}{\enparenContextDefault}
%    \begin{macrocode}
\def\enparenContextDefault{main}
%    \end{macrocode}
%    \end{macro}
%
%    \begin{macro}{\enparen@ctx}
%    \begin{macrocode}
\let\enparen@ctx\ltx@empty
%    \end{macrocode}
%    \end{macro}
%
%    \begin{macro}{\enparen@stack}
%    \begin{macrocode}
\let\enparen@stack\ltx@empty
%    \end{macrocode}
%    \end{macro}
%
% \subsubsection{Stack for contexts}
%
%    \begin{macro}{\enparen@CtxStack}
%    \begin{macrocode}
\global\let\enparen@CtxStack\ltx@empty
%    \end{macrocode}
%    \end{macro}
%
%    \begin{macro}{\enparen@CtxStackPush}
%    \begin{macrocode}
\def\enparen@CtxStackPush#1{%
  \xdef\enparen@CtxStack{%
    {\enparen@ctx}{\enparen@stack}%
    \enparen@CtxStack
  }%
  \xdef\enparen@ctx{#1}%
  \global\let\enparen@stack\ltx@empty
}
%    \end{macrocode}
%    \end{macro}
%
%    \begin{macro}{\enparen@CtxStackPop}
%    \begin{macrocode}
\def\enparen@CtxStackPop{%
  \ifx\enparen@CtxStack\ltx@empty
    \PackageWarning{enparen}{%
      Pop request for empty context stack%
    }%
    \global\let\enparen@ctx\enparenContextDefault
    \global\let\enparen@stack\ltx@empty
  \else
    \xdef\enparen@ctx{%
      \expandafter\ltx@car\enparen@CtxStack\@nil
    }%
    \xdef\enparen@stack{%
      \expandafter\ltx@carsecond\enparen@CtxStack\@nil
    }%
    \xdef\enparen@CtxStack{%
      \expandafter\ltx@cdrtwo\enparen@CtxStack\@nil
    }%
  \fi
}
%    \end{macrocode}
%    \end{macro}
%
% \subsubsection{Context user macros}
%
%    \begin{macro}{\enparenBeginContext}
%    \begin{macrocode}
\ProtectedDef*{\enparenBeginContext}[1]{%
  \enparen@CtxStackPush{#1}%
}
%    \end{macrocode}
%    \end{macro}
%    \begin{macro}{\enparenEndContext}
%    \begin{macrocode}
\ProtectedDef*{\enparenEndContext}[1]{%
  \edef\enparen@temp{#1}%
  \ifx\enparen@temp\enparen@ctx
  \else
    \PackageWarning{enparen}{%
      Context mismatch in end request.\MessageBreak
      `#1' should be ended, but current context\MessageBreak
      is `\enparen@ctx'%
    }%
  \fi
  \enparenCheckEmptyStack
  \enparen@CtxStackPop
}
%    \end{macrocode}
%    \end{macro}
%
%    \begin{macro}{\enparenCheckEmptyStack}
%    \begin{macrocode}
\ProtectedDef*{\enparenCheckEmptyStack}[0]{%
  \ifx\enparen@stack\ltx@empty
  \else
    \PackageWarning{enparen}{%
      Ending non-empty stack `\enparen@ctx':\MessageBreak
      \enparen@PrintStack\MessageBreak
    }%
  \fi
}
%    \end{macrocode}
%    \end{macro}
%
%    \begin{macro}{\enparen@PrintStack}
%    \begin{macrocode}
\def\enparen@PrintStack{%
  \expandafter\enparen@PrintStackAux
  \enparen@stack\ltx@empty\ltx@empty
}
%    \end{macrocode}
%    \end{macro}
%    \begin{macro}{\enparen@PrintStackAux}
%    \begin{macrocode}
\def\enparen@PrintStackAux#1#2{%
  \ifx\ltx@empty#1%
  \else
    {#1:#2}%
    \expandafter\enparen@PrintStackAux
  \fi
}
%    \end{macrocode}
%    \end{macro}
%
% \subsection{Symbols}
%
%    \begin{macro}{\enparenSetSymbols}
%    \begin{macrocode}
\ProtectedDef*{\enparenSetSymbols}[3]{%
  \expandafter
  \def\csname enparen@symbol\the\numexpr#1L\endcsname{#2}%
  \expandafter
  \def\csname enparen@symbol\the\numexpr#1R\endcsname{#3}%
}
%    \end{macrocode}
%    \end{macro}
%    \begin{macro}{\enparenUnsetSymbols}
%    \begin{macrocode}
\ProtectedDef*{\enparenUnsetSymbols}[1]{%
  \expandafter
  \let\csname enparen@symbol\the\numexpr#1L\endcsname\ltx@undefined
  \expandafter
  \let\csname enparen@symbol\the\numexpr#1R\endcsname\ltx@undefined
}
%    \end{macrocode}
%    \end{macro}
%
%    \begin{macrocode}
\enparenSetSymbols{0}{(}{)}
\enparenSetSymbols{1}{(}{)}
\enparenSetSymbols{2}{[}{]}
\enparenSetSymbols{3}{\{}{\}}
\enparenSetSymbols{4}{\ensuremath{\langle}}{\ensuremath{\rangle}}
%    \end{macrocode}
%
% \subsection{Main user macros}
%
%    \begin{macro}{\enparen}
%    \begin{macrocode}
\ProtectedDef{\enparen}[1]{%
  \enparenLeft#1\enparenRight
}
%    \end{macrocode}
%    \end{macro}
%
%    \begin{macro}{\enparenLeft}
%    \begin{macrocode}
\ProtectedDef*{\enparenLeft}[0]{%
  \UniqueCounterCall{enparen}\enparen@Left
}
%    \end{macrocode}
%    \end{macro}
%    \begin{macro}{\enparen@Left}
%    \begin{macrocode}
\def\enparen@Left#1{%
  \xdef\enparen@stack{%
    {#1}{1}%
    \expandafter\enparen@Inc\expandafter2\expandafter!%
    \enparen@stack\ltx@empty\ltx@empty
  }%
  \edef\enparen@tmp{\zref@extract{enparen#1}{enparen}}%
  \ltx@IfUndefined{enparen@symbol\enparen@tmp L}{%
    \PackageWarning{enparen}{%
      Undefined symbols for level \enparen@tmp
    }%
    \csname enparen@symbol0L\endcsname
  }{%
    \csname enparen@symbol\enparen@tmp L\endcsname
  }%
}
%    \end{macrocode}
%    \end{macro}
%    \begin{macro}{\enparen@Inc}
%    \begin{macrocode}
\def\enparen@Inc#1!#2#3{%
  \ifx\ltx@empty#2%
  \else
    \ifnum#3<#1 %
      {#2}{#1}%
      \expandafter\enparen@Inc
      \the\numexpr#1+1\expandafter\expandafter\expandafter!%
    \else
      {#2}{#3}%
    \fi
  \fi
}
%    \end{macrocode}
%    \end{macro}
%
%    \begin{macro}{\enparenRight}
%    \begin{macrocode}
\ProtectedDef*{\enparenRight}[0]{%
  \ifx\enparen@stack\ltx@empty
    \PackageWarning{enparen}{%
      Missing left symbol for right symbol%
    }%
    \csname enparen@symbol0R\endcsname
  \else
    \expandafter\enparen@Right\enparen@stack\@nil
  \fi
}
%    \end{macrocode}
%    \end{macro}
%
%    \begin{macro}{\enparen@Right}
%    \begin{macrocode}
\def\enparen@Right#1#2#3\@nil{%
  \ltx@IfUndefined{%
    enparen@symbol%
    \zref@extract{enparen#1}{enparen}%
    R%
  }{%
    \csname enparen@symbol0R\endcsname
  }{%
    \csname
      enparen@symbol%
      \zref@extract{enparen#1}{enparen}%
       R%
    \endcsname
  }%
  \zref@wrapper@mainaux{%
    \zref@setcurrent{enparen}{#2}%
    \zref@labelbyprops{enparen#1}{enparen}%
  }%
  \xdef\enparen@stack{#3}%
}
%    \end{macrocode}
%    \end{macro}
%
% \subsection{Options}
%
%    \begin{macrocode}
\SetupKeyvalOptions{%
  family=enparen,%
  prefix=enparen@,%
}
%    \end{macrocode}
%
%    \begin{macro}{\enparenSetup}
%    \begin{macrocode}
\ProtectedDef*{\enparenSetup}[0]{%
  \kvsetkeys{enparen}%
}
%    \end{macrocode}
%    \end{macro}
%
%    \begin{macrocode}
\DeclareBoolOption[true]{footnote}
\DeclareBoolOption[true]{caption}
\ProcessKeyvalOptions*
%    \end{macrocode}
%
% \subsection{Context settings}
%
%    \begin{macro}{\enparen@AtBegin}
%    \begin{macrocode}
\def\enparen@AtBegin{%
  \ifenparen@footnote
    \let\enparen@org@makefntext\@makefntext
    \long\def\@makefntext##1{%
      \enparen@org@makefntext{%
        \enparenBeginContext{footnote}%
        ##1%
        \enparenEndContext{footnote}%
      }%
    }%
  \fi
  \enparen@Disable{footnote}%
  \ifenparen@caption
    \let\enparen@org@makecaption\@makecaption
    \long\def\@makecaption##1##2{%
      \enparen@org@makecaption{##1}{%
        \enparenBeginContext{caption}%
        ##2%
        \enparenEndContext{caption}%
      }%
    }%
  \fi
  \enparen@Disable{caption}%
}
%    \end{macrocode}
%    \end{macro}
%
%    \begin{macro}{\enparen@Disable}
%    \begin{macrocode}
\def\enparen@Disable#1{%
  \DisableKeyvalOption[%
    action=warning,%
    package=enparen,%
  ]{enparen}{#1}%
}
%    \end{macrocode}
%    \end{macro}
%
%    \begin{macrocode}
\AtBeginDocument{\enparen@AtBegin}
%    \end{macrocode}
%
% \subsection{At end of document}
%
%    \begin{macro}{\enparen@AtEnd}
%    \begin{macrocode}
\def\enparen@AtEnd{%
  \enparenCheckEmptyStack
  \ifx\enparen@CtxStack\ltx@empty
  \else
    \PackageWarningNoLine{enparen}{%
      Context stack is not empty at end of document.\MessageBreak
      Current stack and contents of context stack:\MessageBreak
      [\enparen@ctx]:[\enparen@PrintStack]%
      \expandafter
      \enparen@PrintContextStack\enparen@CtxStack\relax\relax
    }%
  \fi
}
%    \end{macrocode}
%    \end{macro}
%    \begin{macro}{\enparen@PrintContextStack}
%    \begin{macrocode}
\def\enparen@PrintContextStack#1#2{%
  \ifx\relax#1\ltx@empty
  \else
    \MessageBreak
    [#1]:[\enparen@PrintStackAux#2\ltx@empty\ltx@empty]% hash-ok
    \expandafter\enparen@PrintContextStack
  \fi
}
%    \end{macrocode}
%    \end{macro}
%    \begin{macrocode}
\AtVeryEndDocument{\enparen@AtEnd}
%    \end{macrocode}
%
%    \begin{macrocode}
%</package>
%    \end{macrocode}
%
% \section{Installation}
%
% \subsection{Download}
%
% \paragraph{Package.} This package is available on
% CTAN\footnote{\url{ftp://ftp.ctan.org/tex-archive/}}:
% \begin{description}
% \item[\CTAN{macros/latex/contrib/oberdiek/enparen.dtx}] The source file.
% \item[\CTAN{macros/latex/contrib/oberdiek/enparen.pdf}] Documentation.
% \end{description}
%
%
% \paragraph{Bundle.} All the packages of the bundle `oberdiek'
% are also available in a TDS compliant ZIP archive. There
% the packages are already unpacked and the documentation files
% are generated. The files and directories obey the TDS standard.
% \begin{description}
% \item[\CTAN{install/macros/latex/contrib/oberdiek.tds.zip}]
% \end{description}
% \emph{TDS} refers to the standard ``A Directory Structure
% for \TeX\ Files'' (\CTAN{tds/tds.pdf}). Directories
% with \xfile{texmf} in their name are usually organized this way.
%
% \subsection{Bundle installation}
%
% \paragraph{Unpacking.} Unpack the \xfile{oberdiek.tds.zip} in the
% TDS tree (also known as \xfile{texmf} tree) of your choice.
% Example (linux):
% \begin{quote}
%   |unzip oberdiek.tds.zip -d ~/texmf|
% \end{quote}
%
% \paragraph{Script installation.}
% Check the directory \xfile{TDS:scripts/oberdiek/} for
% scripts that need further installation steps.
% Package \xpackage{attachfile2} comes with the Perl script
% \xfile{pdfatfi.pl} that should be installed in such a way
% that it can be called as \texttt{pdfatfi}.
% Example (linux):
% \begin{quote}
%   |chmod +x scripts/oberdiek/pdfatfi.pl|\\
%   |cp scripts/oberdiek/pdfatfi.pl /usr/local/bin/|
% \end{quote}
%
% \subsection{Package installation}
%
% \paragraph{Unpacking.} The \xfile{.dtx} file is a self-extracting
% \docstrip\ archive. The files are extracted by running the
% \xfile{.dtx} through \plainTeX:
% \begin{quote}
%   \verb|tex enparen.dtx|
% \end{quote}
%
% \paragraph{TDS.} Now the different files must be moved into
% the different directories in your installation TDS tree
% (also known as \xfile{texmf} tree):
% \begin{quote}
% \def\t{^^A
% \begin{tabular}{@{}>{\ttfamily}l@{ $\rightarrow$ }>{\ttfamily}l@{}}
%   enparen.sty & tex/latex/oberdiek/enparen.sty\\
%   enparen.pdf & doc/latex/oberdiek/enparen.pdf\\
%   enparen.dtx & source/latex/oberdiek/enparen.dtx\\
% \end{tabular}^^A
% }^^A
% \sbox0{\t}^^A
% \ifdim\wd0>\linewidth
%   \begingroup
%     \advance\linewidth by\leftmargin
%     \advance\linewidth by\rightmargin
%   \edef\x{\endgroup
%     \def\noexpand\lw{\the\linewidth}^^A
%   }\x
%   \def\lwbox{^^A
%     \leavevmode
%     \hbox to \linewidth{^^A
%       \kern-\leftmargin\relax
%       \hss
%       \usebox0
%       \hss
%       \kern-\rightmargin\relax
%     }^^A
%   }^^A
%   \ifdim\wd0>\lw
%     \sbox0{\small\t}^^A
%     \ifdim\wd0>\linewidth
%       \ifdim\wd0>\lw
%         \sbox0{\footnotesize\t}^^A
%         \ifdim\wd0>\linewidth
%           \ifdim\wd0>\lw
%             \sbox0{\scriptsize\t}^^A
%             \ifdim\wd0>\linewidth
%               \ifdim\wd0>\lw
%                 \sbox0{\tiny\t}^^A
%                 \ifdim\wd0>\linewidth
%                   \lwbox
%                 \else
%                   \usebox0
%                 \fi
%               \else
%                 \lwbox
%               \fi
%             \else
%               \usebox0
%             \fi
%           \else
%             \lwbox
%           \fi
%         \else
%           \usebox0
%         \fi
%       \else
%         \lwbox
%       \fi
%     \else
%       \usebox0
%     \fi
%   \else
%     \lwbox
%   \fi
% \else
%   \usebox0
% \fi
% \end{quote}
% If you have a \xfile{docstrip.cfg} that configures and enables \docstrip's
% TDS installing feature, then some files can already be in the right
% place, see the documentation of \docstrip.
%
% \subsection{Refresh file name databases}
%
% If your \TeX~distribution
% (\teTeX, \mikTeX, \dots) relies on file name databases, you must refresh
% these. For example, \teTeX\ users run \verb|texhash| or
% \verb|mktexlsr|.
%
% \subsection{Some details for the interested}
%
% \paragraph{Attached source.}
%
% The PDF documentation on CTAN also includes the
% \xfile{.dtx} source file. It can be extracted by
% AcrobatReader 6 or higher. Another option is \textsf{pdftk},
% e.g. unpack the file into the current directory:
% \begin{quote}
%   \verb|pdftk enparen.pdf unpack_files output .|
% \end{quote}
%
% \paragraph{Unpacking with \LaTeX.}
% The \xfile{.dtx} chooses its action depending on the format:
% \begin{description}
% \item[\plainTeX:] Run \docstrip\ and extract the files.
% \item[\LaTeX:] Generate the documentation.
% \end{description}
% If you insist on using \LaTeX\ for \docstrip\ (really,
% \docstrip\ does not need \LaTeX), then inform the autodetect routine
% about your intention:
% \begin{quote}
%   \verb|latex \let\install=y% \iffalse meta-comment
%
% File: enparen.dtx
% Version: 2012/01/07 v1.0
% Info: Parentheses nesting
%
% Copyright (C) 2012 by
%    Heiko Oberdiek <heiko.oberdiek at googlemail.com>
%
% This work may be distributed and/or modified under the
% conditions of the LaTeX Project Public License, either
% version 1.3c of this license or (at your option) any later
% version. This version of this license is in
%    http://www.latex-project.org/lppl/lppl-1-3c.txt
% and the latest version of this license is in
%    http://www.latex-project.org/lppl.txt
% and version 1.3 or later is part of all distributions of
% LaTeX version 2005/12/01 or later.
%
% This work has the LPPL maintenance status "maintained".
%
% This Current Maintainer of this work is Heiko Oberdiek.
%
% This work consists of the main source file enparen.dtx
% and the derived files
%    enparen.sty, enparen.pdf, enparen.ins, enparen.drv.
%
% Distribution:
%    CTAN:macros/latex/contrib/oberdiek/enparen.dtx
%    CTAN:macros/latex/contrib/oberdiek/enparen.pdf
%
% Unpacking:
%    (a) If enparen.ins is present:
%           tex enparen.ins
%    (b) Without enparen.ins:
%           tex enparen.dtx
%    (c) If you insist on using LaTeX
%           latex \let\install=y% \iffalse meta-comment
%
% File: enparen.dtx
% Version: 2012/01/07 v1.0
% Info: Parentheses nesting
%
% Copyright (C) 2012 by
%    Heiko Oberdiek <heiko.oberdiek at googlemail.com>
%
% This work may be distributed and/or modified under the
% conditions of the LaTeX Project Public License, either
% version 1.3c of this license or (at your option) any later
% version. This version of this license is in
%    http://www.latex-project.org/lppl/lppl-1-3c.txt
% and the latest version of this license is in
%    http://www.latex-project.org/lppl.txt
% and version 1.3 or later is part of all distributions of
% LaTeX version 2005/12/01 or later.
%
% This work has the LPPL maintenance status "maintained".
%
% This Current Maintainer of this work is Heiko Oberdiek.
%
% This work consists of the main source file enparen.dtx
% and the derived files
%    enparen.sty, enparen.pdf, enparen.ins, enparen.drv.
%
% Distribution:
%    CTAN:macros/latex/contrib/oberdiek/enparen.dtx
%    CTAN:macros/latex/contrib/oberdiek/enparen.pdf
%
% Unpacking:
%    (a) If enparen.ins is present:
%           tex enparen.ins
%    (b) Without enparen.ins:
%           tex enparen.dtx
%    (c) If you insist on using LaTeX
%           latex \let\install=y% \iffalse meta-comment
%
% File: enparen.dtx
% Version: 2012/01/07 v1.0
% Info: Parentheses nesting
%
% Copyright (C) 2012 by
%    Heiko Oberdiek <heiko.oberdiek at googlemail.com>
%
% This work may be distributed and/or modified under the
% conditions of the LaTeX Project Public License, either
% version 1.3c of this license or (at your option) any later
% version. This version of this license is in
%    http://www.latex-project.org/lppl/lppl-1-3c.txt
% and the latest version of this license is in
%    http://www.latex-project.org/lppl.txt
% and version 1.3 or later is part of all distributions of
% LaTeX version 2005/12/01 or later.
%
% This work has the LPPL maintenance status "maintained".
%
% This Current Maintainer of this work is Heiko Oberdiek.
%
% This work consists of the main source file enparen.dtx
% and the derived files
%    enparen.sty, enparen.pdf, enparen.ins, enparen.drv.
%
% Distribution:
%    CTAN:macros/latex/contrib/oberdiek/enparen.dtx
%    CTAN:macros/latex/contrib/oberdiek/enparen.pdf
%
% Unpacking:
%    (a) If enparen.ins is present:
%           tex enparen.ins
%    (b) Without enparen.ins:
%           tex enparen.dtx
%    (c) If you insist on using LaTeX
%           latex \let\install=y\input{enparen.dtx}
%        (quote the arguments according to the demands of your shell)
%
% Documentation:
%    (a) If enparen.drv is present:
%           latex enparen.drv
%    (b) Without enparen.drv:
%           latex enparen.dtx; ...
%    The class ltxdoc loads the configuration file ltxdoc.cfg
%    if available. Here you can specify further options, e.g.
%    use A4 as paper format:
%       \PassOptionsToClass{a4paper}{article}
%
%    Programm calls to get the documentation (example):
%       pdflatex enparen.dtx
%       makeindex -s gind.ist enparen.idx
%       pdflatex enparen.dtx
%       makeindex -s gind.ist enparen.idx
%       pdflatex enparen.dtx
%
% Installation:
%    TDS:tex/latex/oberdiek/enparen.sty
%    TDS:doc/latex/oberdiek/enparen.pdf
%    TDS:source/latex/oberdiek/enparen.dtx
%
%<*ignore>
\begingroup
  \catcode123=1 %
  \catcode125=2 %
  \def\x{LaTeX2e}%
\expandafter\endgroup
\ifcase 0\ifx\install y1\fi\expandafter
         \ifx\csname processbatchFile\endcsname\relax\else1\fi
         \ifx\fmtname\x\else 1\fi\relax
\else\csname fi\endcsname
%</ignore>
%<*install>
\input docstrip.tex
\Msg{************************************************************************}
\Msg{* Installation}
\Msg{* Package: enparen 2012/01/07 v1.0 Parentheses nesting (HO)}
\Msg{************************************************************************}

\keepsilent
\askforoverwritefalse

\let\MetaPrefix\relax
\preamble

This is a generated file.

Project: enparen
Version: 2012/01/07 v1.0

Copyright (C) 2012 by
   Heiko Oberdiek <heiko.oberdiek at googlemail.com>

This work may be distributed and/or modified under the
conditions of the LaTeX Project Public License, either
version 1.3c of this license or (at your option) any later
version. This version of this license is in
   http://www.latex-project.org/lppl/lppl-1-3c.txt
and the latest version of this license is in
   http://www.latex-project.org/lppl.txt
and version 1.3 or later is part of all distributions of
LaTeX version 2005/12/01 or later.

This work has the LPPL maintenance status "maintained".

This Current Maintainer of this work is Heiko Oberdiek.

This work consists of the main source file enparen.dtx
and the derived files
   enparen.sty, enparen.pdf, enparen.ins, enparen.drv.

\endpreamble
\let\MetaPrefix\DoubleperCent

\generate{%
  \file{enparen.ins}{\from{enparen.dtx}{install}}%
  \file{enparen.drv}{\from{enparen.dtx}{driver}}%
  \usedir{tex/latex/oberdiek}%
  \file{enparen.sty}{\from{enparen.dtx}{package}}%
}

\catcode32=13\relax% active space
\let =\space%
\Msg{************************************************************************}
\Msg{*}
\Msg{* To finish the installation you have to move the following}
\Msg{* file into a directory searched by TeX:}
\Msg{*}
\Msg{*     enparen.sty}
\Msg{*}
\Msg{* To produce the documentation run the file `enparen.drv'}
\Msg{* through LaTeX.}
\Msg{*}
\Msg{* Happy TeXing!}
\Msg{*}
\Msg{************************************************************************}

\endbatchfile
%</install>
%<*ignore>
\fi
%</ignore>
%<*driver>
\NeedsTeXFormat{LaTeX2e}
\ProvidesFile{enparen.drv}%
  [2012/01/07 v1.0 Parentheses nesting (HO)]%
\documentclass{ltxdoc}
\usepackage{holtxdoc}[2011/11/22]
\usepackage{enparen}[2012/01/07]
\begin{document}
  \DocInput{enparen.dtx}%
\end{document}
%</driver>
% \fi
%
% \CheckSum{339}
%
% \CharacterTable
%  {Upper-case    \A\B\C\D\E\F\G\H\I\J\K\L\M\N\O\P\Q\R\S\T\U\V\W\X\Y\Z
%   Lower-case    \a\b\c\d\e\f\g\h\i\j\k\l\m\n\o\p\q\r\s\t\u\v\w\x\y\z
%   Digits        \0\1\2\3\4\5\6\7\8\9
%   Exclamation   \!     Double quote  \"     Hash (number) \#
%   Dollar        \$     Percent       \%     Ampersand     \&
%   Acute accent  \'     Left paren    \(     Right paren   \)
%   Asterisk      \*     Plus          \+     Comma         \,
%   Minus         \-     Point         \.     Solidus       \/
%   Colon         \:     Semicolon     \;     Less than     \<
%   Equals        \=     Greater than  \>     Question mark \?
%   Commercial at \@     Left bracket  \[     Backslash     \\
%   Right bracket \]     Circumflex    \^     Underscore    \_
%   Grave accent  \`     Left brace    \{     Vertical bar  \|
%   Right brace   \}     Tilde         \~}
%
% \GetFileInfo{enparen.drv}
%
% \title{The \xpackage{enparen} package}
% \date{2012/01/07 v1.0}
% \author{Heiko Oberdiek\\\xemail{heiko.oberdiek at googlemail.com}}
%
% \maketitle
%
% \begin{abstract}
% The package defines macros to set parentheses that automatically
% change the symbols from inner to outer fences.
% \end{abstract}
%
% \tableofcontents
%
% \section{Documentation}
%
% The \hologo{LaTeX} package provides macros for automatically changed
% parentheses symbols depending on the fence order.
% The innermost parentheses are always using the same
% parentheses. The symbols changes for the outer fences.
%
% Example:
% \begin{quote}
%   \enparen{a \enparen{b} \enparen{c \enparen{d}} \enparen{e}}
% \end{quote}
% is generated by
%\begin{quote}
%\begin{verbatim}
%\documentclass{article}
%\usepackage{enparen}
%\begin{document}
%  \enparen{a \enparen{b} \enparen{c \enparen{d}} \enparen{e}}
%\end{document}
%\end{verbatim}
%\end{quote}
%
% The package is the result of a newsgroup thread. Dan gives
% the following specification \cite{dan}:
% \begin{quote}
% ``On the other hand, the rules for fences are usually the reverse:
% innermost fences are always |()|, next outer are |[]|, etc. This means
% the opening fence has to wait until all the fences between it and
% the matching close have been detected before it can decide whether
% to be |(| or |[| or |{|.''
% \end{quote}
%
% The fence level counting starts from innermost parentheses with one.
% For the next outer fences the level is increased by one.
% The example above with level indexes:
% \begin{quote}
%   \def\1{\ensuremath{_{1}}}
%   \def\2{\ensuremath{_{2}}}
%   \def\3{\ensuremath{_{3}}}
%   \enparen{\3a \enparen{\1b}\1 \enparen{\2c \enparen{\1d}\1}\2 \enparen{e}}\3
% \end{quote}
% The correct level is only known at the closing symbol. Therefore the
% correct value is remembered in the main \xext{aux} file and used
% in the second \hologo{LaTeX} run.
%
% \subsection{User macros}
%
% \begin{declcs}{enparen} \M{text}
% \end{declcs}
% The macro \cs{enparen} puts its argument \meta{text} in parentheses.
% If the macro is nested, the used fence symbols change for the
% outer fences.
%
% \begin{declcs}{enparenLeft}\\
%   \cs{enparenRight}
% \end{declcs}
% Instead of |\enparen{text}| the left and right symbol can be
% used separately, but in pairs:
% \begin{quote}
%   |\enparenLeft text\enparenRight|
% \end{quote}
% \cs{enparenLeft} and \cs{enparenRight} may be used at different
% group levels, but they must be properly nested.
%
% \begin{declcs}{enparenSetSymbols} \M{level} \M{opening symbol} \M{closing symbol}
% \end{declcs}
% Macro \cs{enparenSetSymbols} configures the \meta{opening symbol} and
% \meta{closing symbol} for the \meta{level}. The \meta{level}
% is a number and the counting starts with one. Level zero is used,
% if the correct level is not known (e.g. in the first \hologo{LaTeX} run).
% The package defines the following sets:
%\begin{quote}
%\begin{verbatim}
%\enparenSetSymbols{0}{(}{)}
%\enparenSetSymbols{1}{(}{)}
%\enparenSetSymbols{2}{[}{]}
%\enparenSetSymbols{3}{\{}{\}}
%\end{verbatim}
%\end{quote}
% Example for changing the third and adding a fourth level:
%\begin{quote}
%\begin{verbatim}
%\enparenSetSymbols{3}{\ensuremath{\langle}}{\ensuremath{\rangle}}
%\enparenSetSymbols{4}{\{}{\}}
%\end{verbatim}
%\end{quote}
%
% \begin{declcs}{enparenUnsetSymbols} \M{level}
% \end{declcs}
% The symbols for level \meta{level} are removed.
% Example scenario: Only two nesting levels must be used,
% the package defines more, then the third level can be
% disabled by |\enparenUnsetSymbols{3}| and the user
% gets warnings if parentheses at level 3 are needed.
%
% \subsection{Contexts}
%
% \begin{declcs}{enparenBeginContext} \M{name}\\
%   \cs{enparenEndContext} \M{name}
% \end{declcs}
% If the current text is interrupted by footnotes, floats
% with captions, then the parentheses inside the text of footnotes,
% captions, \dots should restart from scratch. This can be achieved
% by embedding the text inside macros \cs{enparenBeginContext}
% and \cs{enparenEndContext}. These macros must be properly nested.
% The \meta{name} for the begin and end macro must be the same.
% It is a help for debugging problems, because the warning messages
% show the context name. But it is not necessary that the
% begin\slash end pairs have different names. Example:
%\begin{quote}
%\begin{verbatim}
%\enparenLeft text before table ...
%\begin{table}
%  \caption{Table caption}
%  \enparenBegin{Context}{table}
%  Other text \enparen{foobar}.
%  \enparenEnd{Context}{table}
%\end{table}
%text after table ...
%\enparenRight
%\end{verbatim}
%\end{quote}
% The parentheses inside the table environment and context `table'
% are not nested inside other parentheses: \enparen{foobar}.
% In case of captions and footnotes the contexts are automatically
% added, see next section about options.
%
% \subsection{Options}
%
% \begin{declcs}{enparenSetup} \M{key value list}
% \end{declcs}
% Some options (currently all) can also be set after the
% package is loaded. They can be set in the argument
% \meta{key value list} of \cs{enparenSetup}. Options are disabled
% after they are used the last time. Currently all options are boolean
% options and are disabled in |\begin{document}|.
%
% \begin{description}
% \item[\xoption{caption}:] The caption text is put in a
%   context |caption|.
% \item[\xoption{footnote}:] The footnote text is put in a
%   context |footnote|.
% \end{description}
%
% Example for disabling the two options at different places:
%\begin{quote}
%\begin{verbatim}
%\usepackage[caption=false]{enparen}
%\enparenSetup{footnote=false}
%\end{verbatim}
%\end{quote}
%
% \subsection{Notes}
%
% \begin{description}
% \item[Implicite kerning:]
%   Unexpandable stuff might affect the implicite kerning.
%   The package cannot avoid this, because it need to define
%   and redefine macros at the occurence of each symbol.
%   This is done before the opening and after the closing
%   symbol, thus that the implicite kerning inside
%   is not affected.
% \end{description}
%
% \StopEventually{
% }
%
% \section{Implementation}
%
%    \begin{macrocode}
%<*package>
\NeedsTeXFormat{LaTeX2e}
\ProvidesPackage{enparen}
  [2012/01/07 v1.0 Parentheses nesting (HO)]%
%    \end{macrocode}
%
% \subsection{Resources}
%
%    \begin{macrocode}
\RequirePackage{ltxcmds}[2011/11/09]
%    \end{macrocode}
%
%    \begin{macrocode}
\ltx@IfUndefined{numexpr}{%
  \PackageError{enparen}{%
    Missing e-TeX's \ltx@backslashchar numexpr.\MessageBreak
    The package will continue with emergency definitions%
  }\@ehc
  \def\enparenLeft{(}%
  \def\enparenRight{)}%
  \long\def\enparen#1{\enparenLeft#1\enparenRight}%
  \let\enparenSetup\ltx@gobble
  \let\enparenSetSymbols\ltx@gobblethree
  \let\enparenUnsetSymbols\ltx@gobble
  \endinput
}{}
%    \end{macrocode}
%
%    \begin{macrocode}
\RequirePackage{protecteddef}[2011/01/31]
\RequirePackage{atveryend}[2011/06/30]
\RequirePackage{uniquecounter}[2011/01/30]
\RequirePackage{zref-base}[2011/03/18]
\RequirePackage{kvoptions}[2011/06/30]
\RequirePackage{kvsetkeys}[2011/10/18]
%    \end{macrocode}
%
%    \begin{macro}{\zref@wrapper@mainaux}
%    \begin{macrocode}
\providecommand{\zref@wrapper@mainaux}[1]{%
  \ifx\@auxout\@mainaux
    #1%
  \else
    \begingroup
      \let\@auxout\@mainaux
      #1%
    \endgroup
  \fi
}
%    \end{macrocode}
%    \end{macro}
%
%    \begin{macrocode}
\UniqueCounterNew{enparen}
%    \end{macrocode}
%
%    \begin{macrocode}
\zref@newprop{enparen}[0]{}
%    \end{macrocode}
%
% \subsection{Contexts}
%
%    \begin{macro}{\enparenContextDefault}
%    \begin{macrocode}
\def\enparenContextDefault{main}
%    \end{macrocode}
%    \end{macro}
%
%    \begin{macro}{\enparen@ctx}
%    \begin{macrocode}
\let\enparen@ctx\ltx@empty
%    \end{macrocode}
%    \end{macro}
%
%    \begin{macro}{\enparen@stack}
%    \begin{macrocode}
\let\enparen@stack\ltx@empty
%    \end{macrocode}
%    \end{macro}
%
% \subsubsection{Stack for contexts}
%
%    \begin{macro}{\enparen@CtxStack}
%    \begin{macrocode}
\global\let\enparen@CtxStack\ltx@empty
%    \end{macrocode}
%    \end{macro}
%
%    \begin{macro}{\enparen@CtxStackPush}
%    \begin{macrocode}
\def\enparen@CtxStackPush#1{%
  \xdef\enparen@CtxStack{%
    {\enparen@ctx}{\enparen@stack}%
    \enparen@CtxStack
  }%
  \xdef\enparen@ctx{#1}%
  \global\let\enparen@stack\ltx@empty
}
%    \end{macrocode}
%    \end{macro}
%
%    \begin{macro}{\enparen@CtxStackPop}
%    \begin{macrocode}
\def\enparen@CtxStackPop{%
  \ifx\enparen@CtxStack\ltx@empty
    \PackageWarning{enparen}{%
      Pop request for empty context stack%
    }%
    \global\let\enparen@ctx\enparenContextDefault
    \global\let\enparen@stack\ltx@empty
  \else
    \xdef\enparen@ctx{%
      \expandafter\ltx@car\enparen@CtxStack\@nil
    }%
    \xdef\enparen@stack{%
      \expandafter\ltx@carsecond\enparen@CtxStack\@nil
    }%
    \xdef\enparen@CtxStack{%
      \expandafter\ltx@cdrtwo\enparen@CtxStack\@nil
    }%
  \fi
}
%    \end{macrocode}
%    \end{macro}
%
% \subsubsection{Context user macros}
%
%    \begin{macro}{\enparenBeginContext}
%    \begin{macrocode}
\ProtectedDef*{\enparenBeginContext}[1]{%
  \enparen@CtxStackPush{#1}%
}
%    \end{macrocode}
%    \end{macro}
%    \begin{macro}{\enparenEndContext}
%    \begin{macrocode}
\ProtectedDef*{\enparenEndContext}[1]{%
  \edef\enparen@temp{#1}%
  \ifx\enparen@temp\enparen@ctx
  \else
    \PackageWarning{enparen}{%
      Context mismatch in end request.\MessageBreak
      `#1' should be ended, but current context\MessageBreak
      is `\enparen@ctx'%
    }%
  \fi
  \enparenCheckEmptyStack
  \enparen@CtxStackPop
}
%    \end{macrocode}
%    \end{macro}
%
%    \begin{macro}{\enparenCheckEmptyStack}
%    \begin{macrocode}
\ProtectedDef*{\enparenCheckEmptyStack}[0]{%
  \ifx\enparen@stack\ltx@empty
  \else
    \PackageWarning{enparen}{%
      Ending non-empty stack `\enparen@ctx':\MessageBreak
      \enparen@PrintStack\MessageBreak
    }%
  \fi
}
%    \end{macrocode}
%    \end{macro}
%
%    \begin{macro}{\enparen@PrintStack}
%    \begin{macrocode}
\def\enparen@PrintStack{%
  \expandafter\enparen@PrintStackAux
  \enparen@stack\ltx@empty\ltx@empty
}
%    \end{macrocode}
%    \end{macro}
%    \begin{macro}{\enparen@PrintStackAux}
%    \begin{macrocode}
\def\enparen@PrintStackAux#1#2{%
  \ifx\ltx@empty#1%
  \else
    {#1:#2}%
    \expandafter\enparen@PrintStackAux
  \fi
}
%    \end{macrocode}
%    \end{macro}
%
% \subsection{Symbols}
%
%    \begin{macro}{\enparenSetSymbols}
%    \begin{macrocode}
\ProtectedDef*{\enparenSetSymbols}[3]{%
  \expandafter
  \def\csname enparen@symbol\the\numexpr#1L\endcsname{#2}%
  \expandafter
  \def\csname enparen@symbol\the\numexpr#1R\endcsname{#3}%
}
%    \end{macrocode}
%    \end{macro}
%    \begin{macro}{\enparenUnsetSymbols}
%    \begin{macrocode}
\ProtectedDef*{\enparenUnsetSymbols}[1]{%
  \expandafter
  \let\csname enparen@symbol\the\numexpr#1L\endcsname\ltx@undefined
  \expandafter
  \let\csname enparen@symbol\the\numexpr#1R\endcsname\ltx@undefined
}
%    \end{macrocode}
%    \end{macro}
%
%    \begin{macrocode}
\enparenSetSymbols{0}{(}{)}
\enparenSetSymbols{1}{(}{)}
\enparenSetSymbols{2}{[}{]}
\enparenSetSymbols{3}{\{}{\}}
\enparenSetSymbols{4}{\ensuremath{\langle}}{\ensuremath{\rangle}}
%    \end{macrocode}
%
% \subsection{Main user macros}
%
%    \begin{macro}{\enparen}
%    \begin{macrocode}
\ProtectedDef{\enparen}[1]{%
  \enparenLeft#1\enparenRight
}
%    \end{macrocode}
%    \end{macro}
%
%    \begin{macro}{\enparenLeft}
%    \begin{macrocode}
\ProtectedDef*{\enparenLeft}[0]{%
  \UniqueCounterCall{enparen}\enparen@Left
}
%    \end{macrocode}
%    \end{macro}
%    \begin{macro}{\enparen@Left}
%    \begin{macrocode}
\def\enparen@Left#1{%
  \xdef\enparen@stack{%
    {#1}{1}%
    \expandafter\enparen@Inc\expandafter2\expandafter!%
    \enparen@stack\ltx@empty\ltx@empty
  }%
  \edef\enparen@tmp{\zref@extract{enparen#1}{enparen}}%
  \ltx@IfUndefined{enparen@symbol\enparen@tmp L}{%
    \PackageWarning{enparen}{%
      Undefined symbols for level \enparen@tmp
    }%
    \csname enparen@symbol0L\endcsname
  }{%
    \csname enparen@symbol\enparen@tmp L\endcsname
  }%
}
%    \end{macrocode}
%    \end{macro}
%    \begin{macro}{\enparen@Inc}
%    \begin{macrocode}
\def\enparen@Inc#1!#2#3{%
  \ifx\ltx@empty#2%
  \else
    \ifnum#3<#1 %
      {#2}{#1}%
      \expandafter\enparen@Inc
      \the\numexpr#1+1\expandafter\expandafter\expandafter!%
    \else
      {#2}{#3}%
    \fi
  \fi
}
%    \end{macrocode}
%    \end{macro}
%
%    \begin{macro}{\enparenRight}
%    \begin{macrocode}
\ProtectedDef*{\enparenRight}[0]{%
  \ifx\enparen@stack\ltx@empty
    \PackageWarning{enparen}{%
      Missing left symbol for right symbol%
    }%
    \csname enparen@symbol0R\endcsname
  \else
    \expandafter\enparen@Right\enparen@stack\@nil
  \fi
}
%    \end{macrocode}
%    \end{macro}
%
%    \begin{macro}{\enparen@Right}
%    \begin{macrocode}
\def\enparen@Right#1#2#3\@nil{%
  \ltx@IfUndefined{%
    enparen@symbol%
    \zref@extract{enparen#1}{enparen}%
    R%
  }{%
    \csname enparen@symbol0R\endcsname
  }{%
    \csname
      enparen@symbol%
      \zref@extract{enparen#1}{enparen}%
       R%
    \endcsname
  }%
  \zref@wrapper@mainaux{%
    \zref@setcurrent{enparen}{#2}%
    \zref@labelbyprops{enparen#1}{enparen}%
  }%
  \xdef\enparen@stack{#3}%
}
%    \end{macrocode}
%    \end{macro}
%
% \subsection{Options}
%
%    \begin{macrocode}
\SetupKeyvalOptions{%
  family=enparen,%
  prefix=enparen@,%
}
%    \end{macrocode}
%
%    \begin{macro}{\enparenSetup}
%    \begin{macrocode}
\ProtectedDef*{\enparenSetup}[0]{%
  \kvsetkeys{enparen}%
}
%    \end{macrocode}
%    \end{macro}
%
%    \begin{macrocode}
\DeclareBoolOption[true]{footnote}
\DeclareBoolOption[true]{caption}
\ProcessKeyvalOptions*
%    \end{macrocode}
%
% \subsection{Context settings}
%
%    \begin{macro}{\enparen@AtBegin}
%    \begin{macrocode}
\def\enparen@AtBegin{%
  \ifenparen@footnote
    \let\enparen@org@makefntext\@makefntext
    \long\def\@makefntext##1{%
      \enparen@org@makefntext{%
        \enparenBeginContext{footnote}%
        ##1%
        \enparenEndContext{footnote}%
      }%
    }%
  \fi
  \enparen@Disable{footnote}%
  \ifenparen@caption
    \let\enparen@org@makecaption\@makecaption
    \long\def\@makecaption##1##2{%
      \enparen@org@makecaption{##1}{%
        \enparenBeginContext{caption}%
        ##2%
        \enparenEndContext{caption}%
      }%
    }%
  \fi
  \enparen@Disable{caption}%
}
%    \end{macrocode}
%    \end{macro}
%
%    \begin{macro}{\enparen@Disable}
%    \begin{macrocode}
\def\enparen@Disable#1{%
  \DisableKeyvalOption[%
    action=warning,%
    package=enparen,%
  ]{enparen}{#1}%
}
%    \end{macrocode}
%    \end{macro}
%
%    \begin{macrocode}
\AtBeginDocument{\enparen@AtBegin}
%    \end{macrocode}
%
% \subsection{At end of document}
%
%    \begin{macro}{\enparen@AtEnd}
%    \begin{macrocode}
\def\enparen@AtEnd{%
  \enparenCheckEmptyStack
  \ifx\enparen@CtxStack\ltx@empty
  \else
    \PackageWarningNoLine{enparen}{%
      Context stack is not empty at end of document.\MessageBreak
      Current stack and contents of context stack:\MessageBreak
      [\enparen@ctx]:[\enparen@PrintStack]%
      \expandafter
      \enparen@PrintContextStack\enparen@CtxStack\relax\relax
    }%
  \fi
}
%    \end{macrocode}
%    \end{macro}
%    \begin{macro}{\enparen@PrintContextStack}
%    \begin{macrocode}
\def\enparen@PrintContextStack#1#2{%
  \ifx\relax#1\ltx@empty
  \else
    \MessageBreak
    [#1]:[\enparen@PrintStackAux#2\ltx@empty\ltx@empty]% hash-ok
    \expandafter\enparen@PrintContextStack
  \fi
}
%    \end{macrocode}
%    \end{macro}
%    \begin{macrocode}
\AtVeryEndDocument{\enparen@AtEnd}
%    \end{macrocode}
%
%    \begin{macrocode}
%</package>
%    \end{macrocode}
%
% \section{Installation}
%
% \subsection{Download}
%
% \paragraph{Package.} This package is available on
% CTAN\footnote{\url{ftp://ftp.ctan.org/tex-archive/}}:
% \begin{description}
% \item[\CTAN{macros/latex/contrib/oberdiek/enparen.dtx}] The source file.
% \item[\CTAN{macros/latex/contrib/oberdiek/enparen.pdf}] Documentation.
% \end{description}
%
%
% \paragraph{Bundle.} All the packages of the bundle `oberdiek'
% are also available in a TDS compliant ZIP archive. There
% the packages are already unpacked and the documentation files
% are generated. The files and directories obey the TDS standard.
% \begin{description}
% \item[\CTAN{install/macros/latex/contrib/oberdiek.tds.zip}]
% \end{description}
% \emph{TDS} refers to the standard ``A Directory Structure
% for \TeX\ Files'' (\CTAN{tds/tds.pdf}). Directories
% with \xfile{texmf} in their name are usually organized this way.
%
% \subsection{Bundle installation}
%
% \paragraph{Unpacking.} Unpack the \xfile{oberdiek.tds.zip} in the
% TDS tree (also known as \xfile{texmf} tree) of your choice.
% Example (linux):
% \begin{quote}
%   |unzip oberdiek.tds.zip -d ~/texmf|
% \end{quote}
%
% \paragraph{Script installation.}
% Check the directory \xfile{TDS:scripts/oberdiek/} for
% scripts that need further installation steps.
% Package \xpackage{attachfile2} comes with the Perl script
% \xfile{pdfatfi.pl} that should be installed in such a way
% that it can be called as \texttt{pdfatfi}.
% Example (linux):
% \begin{quote}
%   |chmod +x scripts/oberdiek/pdfatfi.pl|\\
%   |cp scripts/oberdiek/pdfatfi.pl /usr/local/bin/|
% \end{quote}
%
% \subsection{Package installation}
%
% \paragraph{Unpacking.} The \xfile{.dtx} file is a self-extracting
% \docstrip\ archive. The files are extracted by running the
% \xfile{.dtx} through \plainTeX:
% \begin{quote}
%   \verb|tex enparen.dtx|
% \end{quote}
%
% \paragraph{TDS.} Now the different files must be moved into
% the different directories in your installation TDS tree
% (also known as \xfile{texmf} tree):
% \begin{quote}
% \def\t{^^A
% \begin{tabular}{@{}>{\ttfamily}l@{ $\rightarrow$ }>{\ttfamily}l@{}}
%   enparen.sty & tex/latex/oberdiek/enparen.sty\\
%   enparen.pdf & doc/latex/oberdiek/enparen.pdf\\
%   enparen.dtx & source/latex/oberdiek/enparen.dtx\\
% \end{tabular}^^A
% }^^A
% \sbox0{\t}^^A
% \ifdim\wd0>\linewidth
%   \begingroup
%     \advance\linewidth by\leftmargin
%     \advance\linewidth by\rightmargin
%   \edef\x{\endgroup
%     \def\noexpand\lw{\the\linewidth}^^A
%   }\x
%   \def\lwbox{^^A
%     \leavevmode
%     \hbox to \linewidth{^^A
%       \kern-\leftmargin\relax
%       \hss
%       \usebox0
%       \hss
%       \kern-\rightmargin\relax
%     }^^A
%   }^^A
%   \ifdim\wd0>\lw
%     \sbox0{\small\t}^^A
%     \ifdim\wd0>\linewidth
%       \ifdim\wd0>\lw
%         \sbox0{\footnotesize\t}^^A
%         \ifdim\wd0>\linewidth
%           \ifdim\wd0>\lw
%             \sbox0{\scriptsize\t}^^A
%             \ifdim\wd0>\linewidth
%               \ifdim\wd0>\lw
%                 \sbox0{\tiny\t}^^A
%                 \ifdim\wd0>\linewidth
%                   \lwbox
%                 \else
%                   \usebox0
%                 \fi
%               \else
%                 \lwbox
%               \fi
%             \else
%               \usebox0
%             \fi
%           \else
%             \lwbox
%           \fi
%         \else
%           \usebox0
%         \fi
%       \else
%         \lwbox
%       \fi
%     \else
%       \usebox0
%     \fi
%   \else
%     \lwbox
%   \fi
% \else
%   \usebox0
% \fi
% \end{quote}
% If you have a \xfile{docstrip.cfg} that configures and enables \docstrip's
% TDS installing feature, then some files can already be in the right
% place, see the documentation of \docstrip.
%
% \subsection{Refresh file name databases}
%
% If your \TeX~distribution
% (\teTeX, \mikTeX, \dots) relies on file name databases, you must refresh
% these. For example, \teTeX\ users run \verb|texhash| or
% \verb|mktexlsr|.
%
% \subsection{Some details for the interested}
%
% \paragraph{Attached source.}
%
% The PDF documentation on CTAN also includes the
% \xfile{.dtx} source file. It can be extracted by
% AcrobatReader 6 or higher. Another option is \textsf{pdftk},
% e.g. unpack the file into the current directory:
% \begin{quote}
%   \verb|pdftk enparen.pdf unpack_files output .|
% \end{quote}
%
% \paragraph{Unpacking with \LaTeX.}
% The \xfile{.dtx} chooses its action depending on the format:
% \begin{description}
% \item[\plainTeX:] Run \docstrip\ and extract the files.
% \item[\LaTeX:] Generate the documentation.
% \end{description}
% If you insist on using \LaTeX\ for \docstrip\ (really,
% \docstrip\ does not need \LaTeX), then inform the autodetect routine
% about your intention:
% \begin{quote}
%   \verb|latex \let\install=y\input{enparen.dtx}|
% \end{quote}
% Do not forget to quote the argument according to the demands
% of your shell.
%
% \paragraph{Generating the documentation.}
% You can use both the \xfile{.dtx} or the \xfile{.drv} to generate
% the documentation. The process can be configured by the
% configuration file \xfile{ltxdoc.cfg}. For instance, put this
% line into this file, if you want to have A4 as paper format:
% \begin{quote}
%   \verb|\PassOptionsToClass{a4paper}{article}|
% \end{quote}
% An example follows how to generate the
% documentation with pdf\LaTeX:
% \begin{quote}
%\begin{verbatim}
%pdflatex enparen.dtx
%makeindex -s gind.ist enparen.idx
%pdflatex enparen.dtx
%makeindex -s gind.ist enparen.idx
%pdflatex enparen.dtx
%\end{verbatim}
% \end{quote}
%
% \begin{thebibliography}{9}
% \raggedright
% \bibitem{dan}
%   Dan Luecking: \textit{Re: bracket order};
%   newsgroup \xnewsgroup{comp.text.tex};
%   message id \texttt{\small$\langle$9b07c9c8-ff92-4cbf-b3a9-84efecfeb506@l24g2000yqm.googlegroups.com$\rangle$}
%   2012-01-05.
%   \url{http://groups.google.com/group/comp.text.tex/msg/8774519da31c2352}
% \end{thebibliography}
%
% \begin{History}
%   \begin{Version}{2012/01/07 v1.0}
%   \item
%     First version.
%   \end{Version}
% \end{History}
%
% \PrintIndex
%
% \Finale
\endinput

%        (quote the arguments according to the demands of your shell)
%
% Documentation:
%    (a) If enparen.drv is present:
%           latex enparen.drv
%    (b) Without enparen.drv:
%           latex enparen.dtx; ...
%    The class ltxdoc loads the configuration file ltxdoc.cfg
%    if available. Here you can specify further options, e.g.
%    use A4 as paper format:
%       \PassOptionsToClass{a4paper}{article}
%
%    Programm calls to get the documentation (example):
%       pdflatex enparen.dtx
%       makeindex -s gind.ist enparen.idx
%       pdflatex enparen.dtx
%       makeindex -s gind.ist enparen.idx
%       pdflatex enparen.dtx
%
% Installation:
%    TDS:tex/latex/oberdiek/enparen.sty
%    TDS:doc/latex/oberdiek/enparen.pdf
%    TDS:source/latex/oberdiek/enparen.dtx
%
%<*ignore>
\begingroup
  \catcode123=1 %
  \catcode125=2 %
  \def\x{LaTeX2e}%
\expandafter\endgroup
\ifcase 0\ifx\install y1\fi\expandafter
         \ifx\csname processbatchFile\endcsname\relax\else1\fi
         \ifx\fmtname\x\else 1\fi\relax
\else\csname fi\endcsname
%</ignore>
%<*install>
\input docstrip.tex
\Msg{************************************************************************}
\Msg{* Installation}
\Msg{* Package: enparen 2012/01/07 v1.0 Parentheses nesting (HO)}
\Msg{************************************************************************}

\keepsilent
\askforoverwritefalse

\let\MetaPrefix\relax
\preamble

This is a generated file.

Project: enparen
Version: 2012/01/07 v1.0

Copyright (C) 2012 by
   Heiko Oberdiek <heiko.oberdiek at googlemail.com>

This work may be distributed and/or modified under the
conditions of the LaTeX Project Public License, either
version 1.3c of this license or (at your option) any later
version. This version of this license is in
   http://www.latex-project.org/lppl/lppl-1-3c.txt
and the latest version of this license is in
   http://www.latex-project.org/lppl.txt
and version 1.3 or later is part of all distributions of
LaTeX version 2005/12/01 or later.

This work has the LPPL maintenance status "maintained".

This Current Maintainer of this work is Heiko Oberdiek.

This work consists of the main source file enparen.dtx
and the derived files
   enparen.sty, enparen.pdf, enparen.ins, enparen.drv.

\endpreamble
\let\MetaPrefix\DoubleperCent

\generate{%
  \file{enparen.ins}{\from{enparen.dtx}{install}}%
  \file{enparen.drv}{\from{enparen.dtx}{driver}}%
  \usedir{tex/latex/oberdiek}%
  \file{enparen.sty}{\from{enparen.dtx}{package}}%
}

\catcode32=13\relax% active space
\let =\space%
\Msg{************************************************************************}
\Msg{*}
\Msg{* To finish the installation you have to move the following}
\Msg{* file into a directory searched by TeX:}
\Msg{*}
\Msg{*     enparen.sty}
\Msg{*}
\Msg{* To produce the documentation run the file `enparen.drv'}
\Msg{* through LaTeX.}
\Msg{*}
\Msg{* Happy TeXing!}
\Msg{*}
\Msg{************************************************************************}

\endbatchfile
%</install>
%<*ignore>
\fi
%</ignore>
%<*driver>
\NeedsTeXFormat{LaTeX2e}
\ProvidesFile{enparen.drv}%
  [2012/01/07 v1.0 Parentheses nesting (HO)]%
\documentclass{ltxdoc}
\usepackage{holtxdoc}[2011/11/22]
\usepackage{enparen}[2012/01/07]
\begin{document}
  \DocInput{enparen.dtx}%
\end{document}
%</driver>
% \fi
%
% \CheckSum{339}
%
% \CharacterTable
%  {Upper-case    \A\B\C\D\E\F\G\H\I\J\K\L\M\N\O\P\Q\R\S\T\U\V\W\X\Y\Z
%   Lower-case    \a\b\c\d\e\f\g\h\i\j\k\l\m\n\o\p\q\r\s\t\u\v\w\x\y\z
%   Digits        \0\1\2\3\4\5\6\7\8\9
%   Exclamation   \!     Double quote  \"     Hash (number) \#
%   Dollar        \$     Percent       \%     Ampersand     \&
%   Acute accent  \'     Left paren    \(     Right paren   \)
%   Asterisk      \*     Plus          \+     Comma         \,
%   Minus         \-     Point         \.     Solidus       \/
%   Colon         \:     Semicolon     \;     Less than     \<
%   Equals        \=     Greater than  \>     Question mark \?
%   Commercial at \@     Left bracket  \[     Backslash     \\
%   Right bracket \]     Circumflex    \^     Underscore    \_
%   Grave accent  \`     Left brace    \{     Vertical bar  \|
%   Right brace   \}     Tilde         \~}
%
% \GetFileInfo{enparen.drv}
%
% \title{The \xpackage{enparen} package}
% \date{2012/01/07 v1.0}
% \author{Heiko Oberdiek\\\xemail{heiko.oberdiek at googlemail.com}}
%
% \maketitle
%
% \begin{abstract}
% The package defines macros to set parentheses that automatically
% change the symbols from inner to outer fences.
% \end{abstract}
%
% \tableofcontents
%
% \section{Documentation}
%
% The \hologo{LaTeX} package provides macros for automatically changed
% parentheses symbols depending on the fence order.
% The innermost parentheses are always using the same
% parentheses. The symbols changes for the outer fences.
%
% Example:
% \begin{quote}
%   \enparen{a \enparen{b} \enparen{c \enparen{d}} \enparen{e}}
% \end{quote}
% is generated by
%\begin{quote}
%\begin{verbatim}
%\documentclass{article}
%\usepackage{enparen}
%\begin{document}
%  \enparen{a \enparen{b} \enparen{c \enparen{d}} \enparen{e}}
%\end{document}
%\end{verbatim}
%\end{quote}
%
% The package is the result of a newsgroup thread. Dan gives
% the following specification \cite{dan}:
% \begin{quote}
% ``On the other hand, the rules for fences are usually the reverse:
% innermost fences are always |()|, next outer are |[]|, etc. This means
% the opening fence has to wait until all the fences between it and
% the matching close have been detected before it can decide whether
% to be |(| or |[| or |{|.''
% \end{quote}
%
% The fence level counting starts from innermost parentheses with one.
% For the next outer fences the level is increased by one.
% The example above with level indexes:
% \begin{quote}
%   \def\1{\ensuremath{_{1}}}
%   \def\2{\ensuremath{_{2}}}
%   \def\3{\ensuremath{_{3}}}
%   \enparen{\3a \enparen{\1b}\1 \enparen{\2c \enparen{\1d}\1}\2 \enparen{e}}\3
% \end{quote}
% The correct level is only known at the closing symbol. Therefore the
% correct value is remembered in the main \xext{aux} file and used
% in the second \hologo{LaTeX} run.
%
% \subsection{User macros}
%
% \begin{declcs}{enparen} \M{text}
% \end{declcs}
% The macro \cs{enparen} puts its argument \meta{text} in parentheses.
% If the macro is nested, the used fence symbols change for the
% outer fences.
%
% \begin{declcs}{enparenLeft}\\
%   \cs{enparenRight}
% \end{declcs}
% Instead of |\enparen{text}| the left and right symbol can be
% used separately, but in pairs:
% \begin{quote}
%   |\enparenLeft text\enparenRight|
% \end{quote}
% \cs{enparenLeft} and \cs{enparenRight} may be used at different
% group levels, but they must be properly nested.
%
% \begin{declcs}{enparenSetSymbols} \M{level} \M{opening symbol} \M{closing symbol}
% \end{declcs}
% Macro \cs{enparenSetSymbols} configures the \meta{opening symbol} and
% \meta{closing symbol} for the \meta{level}. The \meta{level}
% is a number and the counting starts with one. Level zero is used,
% if the correct level is not known (e.g. in the first \hologo{LaTeX} run).
% The package defines the following sets:
%\begin{quote}
%\begin{verbatim}
%\enparenSetSymbols{0}{(}{)}
%\enparenSetSymbols{1}{(}{)}
%\enparenSetSymbols{2}{[}{]}
%\enparenSetSymbols{3}{\{}{\}}
%\end{verbatim}
%\end{quote}
% Example for changing the third and adding a fourth level:
%\begin{quote}
%\begin{verbatim}
%\enparenSetSymbols{3}{\ensuremath{\langle}}{\ensuremath{\rangle}}
%\enparenSetSymbols{4}{\{}{\}}
%\end{verbatim}
%\end{quote}
%
% \begin{declcs}{enparenUnsetSymbols} \M{level}
% \end{declcs}
% The symbols for level \meta{level} are removed.
% Example scenario: Only two nesting levels must be used,
% the package defines more, then the third level can be
% disabled by |\enparenUnsetSymbols{3}| and the user
% gets warnings if parentheses at level 3 are needed.
%
% \subsection{Contexts}
%
% \begin{declcs}{enparenBeginContext} \M{name}\\
%   \cs{enparenEndContext} \M{name}
% \end{declcs}
% If the current text is interrupted by footnotes, floats
% with captions, then the parentheses inside the text of footnotes,
% captions, \dots should restart from scratch. This can be achieved
% by embedding the text inside macros \cs{enparenBeginContext}
% and \cs{enparenEndContext}. These macros must be properly nested.
% The \meta{name} for the begin and end macro must be the same.
% It is a help for debugging problems, because the warning messages
% show the context name. But it is not necessary that the
% begin\slash end pairs have different names. Example:
%\begin{quote}
%\begin{verbatim}
%\enparenLeft text before table ...
%\begin{table}
%  \caption{Table caption}
%  \enparenBegin{Context}{table}
%  Other text \enparen{foobar}.
%  \enparenEnd{Context}{table}
%\end{table}
%text after table ...
%\enparenRight
%\end{verbatim}
%\end{quote}
% The parentheses inside the table environment and context `table'
% are not nested inside other parentheses: \enparen{foobar}.
% In case of captions and footnotes the contexts are automatically
% added, see next section about options.
%
% \subsection{Options}
%
% \begin{declcs}{enparenSetup} \M{key value list}
% \end{declcs}
% Some options (currently all) can also be set after the
% package is loaded. They can be set in the argument
% \meta{key value list} of \cs{enparenSetup}. Options are disabled
% after they are used the last time. Currently all options are boolean
% options and are disabled in |\begin{document}|.
%
% \begin{description}
% \item[\xoption{caption}:] The caption text is put in a
%   context |caption|.
% \item[\xoption{footnote}:] The footnote text is put in a
%   context |footnote|.
% \end{description}
%
% Example for disabling the two options at different places:
%\begin{quote}
%\begin{verbatim}
%\usepackage[caption=false]{enparen}
%\enparenSetup{footnote=false}
%\end{verbatim}
%\end{quote}
%
% \subsection{Notes}
%
% \begin{description}
% \item[Implicite kerning:]
%   Unexpandable stuff might affect the implicite kerning.
%   The package cannot avoid this, because it need to define
%   and redefine macros at the occurence of each symbol.
%   This is done before the opening and after the closing
%   symbol, thus that the implicite kerning inside
%   is not affected.
% \end{description}
%
% \StopEventually{
% }
%
% \section{Implementation}
%
%    \begin{macrocode}
%<*package>
\NeedsTeXFormat{LaTeX2e}
\ProvidesPackage{enparen}
  [2012/01/07 v1.0 Parentheses nesting (HO)]%
%    \end{macrocode}
%
% \subsection{Resources}
%
%    \begin{macrocode}
\RequirePackage{ltxcmds}[2011/11/09]
%    \end{macrocode}
%
%    \begin{macrocode}
\ltx@IfUndefined{numexpr}{%
  \PackageError{enparen}{%
    Missing e-TeX's \ltx@backslashchar numexpr.\MessageBreak
    The package will continue with emergency definitions%
  }\@ehc
  \def\enparenLeft{(}%
  \def\enparenRight{)}%
  \long\def\enparen#1{\enparenLeft#1\enparenRight}%
  \let\enparenSetup\ltx@gobble
  \let\enparenSetSymbols\ltx@gobblethree
  \let\enparenUnsetSymbols\ltx@gobble
  \endinput
}{}
%    \end{macrocode}
%
%    \begin{macrocode}
\RequirePackage{protecteddef}[2011/01/31]
\RequirePackage{atveryend}[2011/06/30]
\RequirePackage{uniquecounter}[2011/01/30]
\RequirePackage{zref-base}[2011/03/18]
\RequirePackage{kvoptions}[2011/06/30]
\RequirePackage{kvsetkeys}[2011/10/18]
%    \end{macrocode}
%
%    \begin{macro}{\zref@wrapper@mainaux}
%    \begin{macrocode}
\providecommand{\zref@wrapper@mainaux}[1]{%
  \ifx\@auxout\@mainaux
    #1%
  \else
    \begingroup
      \let\@auxout\@mainaux
      #1%
    \endgroup
  \fi
}
%    \end{macrocode}
%    \end{macro}
%
%    \begin{macrocode}
\UniqueCounterNew{enparen}
%    \end{macrocode}
%
%    \begin{macrocode}
\zref@newprop{enparen}[0]{}
%    \end{macrocode}
%
% \subsection{Contexts}
%
%    \begin{macro}{\enparenContextDefault}
%    \begin{macrocode}
\def\enparenContextDefault{main}
%    \end{macrocode}
%    \end{macro}
%
%    \begin{macro}{\enparen@ctx}
%    \begin{macrocode}
\let\enparen@ctx\ltx@empty
%    \end{macrocode}
%    \end{macro}
%
%    \begin{macro}{\enparen@stack}
%    \begin{macrocode}
\let\enparen@stack\ltx@empty
%    \end{macrocode}
%    \end{macro}
%
% \subsubsection{Stack for contexts}
%
%    \begin{macro}{\enparen@CtxStack}
%    \begin{macrocode}
\global\let\enparen@CtxStack\ltx@empty
%    \end{macrocode}
%    \end{macro}
%
%    \begin{macro}{\enparen@CtxStackPush}
%    \begin{macrocode}
\def\enparen@CtxStackPush#1{%
  \xdef\enparen@CtxStack{%
    {\enparen@ctx}{\enparen@stack}%
    \enparen@CtxStack
  }%
  \xdef\enparen@ctx{#1}%
  \global\let\enparen@stack\ltx@empty
}
%    \end{macrocode}
%    \end{macro}
%
%    \begin{macro}{\enparen@CtxStackPop}
%    \begin{macrocode}
\def\enparen@CtxStackPop{%
  \ifx\enparen@CtxStack\ltx@empty
    \PackageWarning{enparen}{%
      Pop request for empty context stack%
    }%
    \global\let\enparen@ctx\enparenContextDefault
    \global\let\enparen@stack\ltx@empty
  \else
    \xdef\enparen@ctx{%
      \expandafter\ltx@car\enparen@CtxStack\@nil
    }%
    \xdef\enparen@stack{%
      \expandafter\ltx@carsecond\enparen@CtxStack\@nil
    }%
    \xdef\enparen@CtxStack{%
      \expandafter\ltx@cdrtwo\enparen@CtxStack\@nil
    }%
  \fi
}
%    \end{macrocode}
%    \end{macro}
%
% \subsubsection{Context user macros}
%
%    \begin{macro}{\enparenBeginContext}
%    \begin{macrocode}
\ProtectedDef*{\enparenBeginContext}[1]{%
  \enparen@CtxStackPush{#1}%
}
%    \end{macrocode}
%    \end{macro}
%    \begin{macro}{\enparenEndContext}
%    \begin{macrocode}
\ProtectedDef*{\enparenEndContext}[1]{%
  \edef\enparen@temp{#1}%
  \ifx\enparen@temp\enparen@ctx
  \else
    \PackageWarning{enparen}{%
      Context mismatch in end request.\MessageBreak
      `#1' should be ended, but current context\MessageBreak
      is `\enparen@ctx'%
    }%
  \fi
  \enparenCheckEmptyStack
  \enparen@CtxStackPop
}
%    \end{macrocode}
%    \end{macro}
%
%    \begin{macro}{\enparenCheckEmptyStack}
%    \begin{macrocode}
\ProtectedDef*{\enparenCheckEmptyStack}[0]{%
  \ifx\enparen@stack\ltx@empty
  \else
    \PackageWarning{enparen}{%
      Ending non-empty stack `\enparen@ctx':\MessageBreak
      \enparen@PrintStack\MessageBreak
    }%
  \fi
}
%    \end{macrocode}
%    \end{macro}
%
%    \begin{macro}{\enparen@PrintStack}
%    \begin{macrocode}
\def\enparen@PrintStack{%
  \expandafter\enparen@PrintStackAux
  \enparen@stack\ltx@empty\ltx@empty
}
%    \end{macrocode}
%    \end{macro}
%    \begin{macro}{\enparen@PrintStackAux}
%    \begin{macrocode}
\def\enparen@PrintStackAux#1#2{%
  \ifx\ltx@empty#1%
  \else
    {#1:#2}%
    \expandafter\enparen@PrintStackAux
  \fi
}
%    \end{macrocode}
%    \end{macro}
%
% \subsection{Symbols}
%
%    \begin{macro}{\enparenSetSymbols}
%    \begin{macrocode}
\ProtectedDef*{\enparenSetSymbols}[3]{%
  \expandafter
  \def\csname enparen@symbol\the\numexpr#1L\endcsname{#2}%
  \expandafter
  \def\csname enparen@symbol\the\numexpr#1R\endcsname{#3}%
}
%    \end{macrocode}
%    \end{macro}
%    \begin{macro}{\enparenUnsetSymbols}
%    \begin{macrocode}
\ProtectedDef*{\enparenUnsetSymbols}[1]{%
  \expandafter
  \let\csname enparen@symbol\the\numexpr#1L\endcsname\ltx@undefined
  \expandafter
  \let\csname enparen@symbol\the\numexpr#1R\endcsname\ltx@undefined
}
%    \end{macrocode}
%    \end{macro}
%
%    \begin{macrocode}
\enparenSetSymbols{0}{(}{)}
\enparenSetSymbols{1}{(}{)}
\enparenSetSymbols{2}{[}{]}
\enparenSetSymbols{3}{\{}{\}}
\enparenSetSymbols{4}{\ensuremath{\langle}}{\ensuremath{\rangle}}
%    \end{macrocode}
%
% \subsection{Main user macros}
%
%    \begin{macro}{\enparen}
%    \begin{macrocode}
\ProtectedDef{\enparen}[1]{%
  \enparenLeft#1\enparenRight
}
%    \end{macrocode}
%    \end{macro}
%
%    \begin{macro}{\enparenLeft}
%    \begin{macrocode}
\ProtectedDef*{\enparenLeft}[0]{%
  \UniqueCounterCall{enparen}\enparen@Left
}
%    \end{macrocode}
%    \end{macro}
%    \begin{macro}{\enparen@Left}
%    \begin{macrocode}
\def\enparen@Left#1{%
  \xdef\enparen@stack{%
    {#1}{1}%
    \expandafter\enparen@Inc\expandafter2\expandafter!%
    \enparen@stack\ltx@empty\ltx@empty
  }%
  \edef\enparen@tmp{\zref@extract{enparen#1}{enparen}}%
  \ltx@IfUndefined{enparen@symbol\enparen@tmp L}{%
    \PackageWarning{enparen}{%
      Undefined symbols for level \enparen@tmp
    }%
    \csname enparen@symbol0L\endcsname
  }{%
    \csname enparen@symbol\enparen@tmp L\endcsname
  }%
}
%    \end{macrocode}
%    \end{macro}
%    \begin{macro}{\enparen@Inc}
%    \begin{macrocode}
\def\enparen@Inc#1!#2#3{%
  \ifx\ltx@empty#2%
  \else
    \ifnum#3<#1 %
      {#2}{#1}%
      \expandafter\enparen@Inc
      \the\numexpr#1+1\expandafter\expandafter\expandafter!%
    \else
      {#2}{#3}%
    \fi
  \fi
}
%    \end{macrocode}
%    \end{macro}
%
%    \begin{macro}{\enparenRight}
%    \begin{macrocode}
\ProtectedDef*{\enparenRight}[0]{%
  \ifx\enparen@stack\ltx@empty
    \PackageWarning{enparen}{%
      Missing left symbol for right symbol%
    }%
    \csname enparen@symbol0R\endcsname
  \else
    \expandafter\enparen@Right\enparen@stack\@nil
  \fi
}
%    \end{macrocode}
%    \end{macro}
%
%    \begin{macro}{\enparen@Right}
%    \begin{macrocode}
\def\enparen@Right#1#2#3\@nil{%
  \ltx@IfUndefined{%
    enparen@symbol%
    \zref@extract{enparen#1}{enparen}%
    R%
  }{%
    \csname enparen@symbol0R\endcsname
  }{%
    \csname
      enparen@symbol%
      \zref@extract{enparen#1}{enparen}%
       R%
    \endcsname
  }%
  \zref@wrapper@mainaux{%
    \zref@setcurrent{enparen}{#2}%
    \zref@labelbyprops{enparen#1}{enparen}%
  }%
  \xdef\enparen@stack{#3}%
}
%    \end{macrocode}
%    \end{macro}
%
% \subsection{Options}
%
%    \begin{macrocode}
\SetupKeyvalOptions{%
  family=enparen,%
  prefix=enparen@,%
}
%    \end{macrocode}
%
%    \begin{macro}{\enparenSetup}
%    \begin{macrocode}
\ProtectedDef*{\enparenSetup}[0]{%
  \kvsetkeys{enparen}%
}
%    \end{macrocode}
%    \end{macro}
%
%    \begin{macrocode}
\DeclareBoolOption[true]{footnote}
\DeclareBoolOption[true]{caption}
\ProcessKeyvalOptions*
%    \end{macrocode}
%
% \subsection{Context settings}
%
%    \begin{macro}{\enparen@AtBegin}
%    \begin{macrocode}
\def\enparen@AtBegin{%
  \ifenparen@footnote
    \let\enparen@org@makefntext\@makefntext
    \long\def\@makefntext##1{%
      \enparen@org@makefntext{%
        \enparenBeginContext{footnote}%
        ##1%
        \enparenEndContext{footnote}%
      }%
    }%
  \fi
  \enparen@Disable{footnote}%
  \ifenparen@caption
    \let\enparen@org@makecaption\@makecaption
    \long\def\@makecaption##1##2{%
      \enparen@org@makecaption{##1}{%
        \enparenBeginContext{caption}%
        ##2%
        \enparenEndContext{caption}%
      }%
    }%
  \fi
  \enparen@Disable{caption}%
}
%    \end{macrocode}
%    \end{macro}
%
%    \begin{macro}{\enparen@Disable}
%    \begin{macrocode}
\def\enparen@Disable#1{%
  \DisableKeyvalOption[%
    action=warning,%
    package=enparen,%
  ]{enparen}{#1}%
}
%    \end{macrocode}
%    \end{macro}
%
%    \begin{macrocode}
\AtBeginDocument{\enparen@AtBegin}
%    \end{macrocode}
%
% \subsection{At end of document}
%
%    \begin{macro}{\enparen@AtEnd}
%    \begin{macrocode}
\def\enparen@AtEnd{%
  \enparenCheckEmptyStack
  \ifx\enparen@CtxStack\ltx@empty
  \else
    \PackageWarningNoLine{enparen}{%
      Context stack is not empty at end of document.\MessageBreak
      Current stack and contents of context stack:\MessageBreak
      [\enparen@ctx]:[\enparen@PrintStack]%
      \expandafter
      \enparen@PrintContextStack\enparen@CtxStack\relax\relax
    }%
  \fi
}
%    \end{macrocode}
%    \end{macro}
%    \begin{macro}{\enparen@PrintContextStack}
%    \begin{macrocode}
\def\enparen@PrintContextStack#1#2{%
  \ifx\relax#1\ltx@empty
  \else
    \MessageBreak
    [#1]:[\enparen@PrintStackAux#2\ltx@empty\ltx@empty]% hash-ok
    \expandafter\enparen@PrintContextStack
  \fi
}
%    \end{macrocode}
%    \end{macro}
%    \begin{macrocode}
\AtVeryEndDocument{\enparen@AtEnd}
%    \end{macrocode}
%
%    \begin{macrocode}
%</package>
%    \end{macrocode}
%
% \section{Installation}
%
% \subsection{Download}
%
% \paragraph{Package.} This package is available on
% CTAN\footnote{\url{ftp://ftp.ctan.org/tex-archive/}}:
% \begin{description}
% \item[\CTAN{macros/latex/contrib/oberdiek/enparen.dtx}] The source file.
% \item[\CTAN{macros/latex/contrib/oberdiek/enparen.pdf}] Documentation.
% \end{description}
%
%
% \paragraph{Bundle.} All the packages of the bundle `oberdiek'
% are also available in a TDS compliant ZIP archive. There
% the packages are already unpacked and the documentation files
% are generated. The files and directories obey the TDS standard.
% \begin{description}
% \item[\CTAN{install/macros/latex/contrib/oberdiek.tds.zip}]
% \end{description}
% \emph{TDS} refers to the standard ``A Directory Structure
% for \TeX\ Files'' (\CTAN{tds/tds.pdf}). Directories
% with \xfile{texmf} in their name are usually organized this way.
%
% \subsection{Bundle installation}
%
% \paragraph{Unpacking.} Unpack the \xfile{oberdiek.tds.zip} in the
% TDS tree (also known as \xfile{texmf} tree) of your choice.
% Example (linux):
% \begin{quote}
%   |unzip oberdiek.tds.zip -d ~/texmf|
% \end{quote}
%
% \paragraph{Script installation.}
% Check the directory \xfile{TDS:scripts/oberdiek/} for
% scripts that need further installation steps.
% Package \xpackage{attachfile2} comes with the Perl script
% \xfile{pdfatfi.pl} that should be installed in such a way
% that it can be called as \texttt{pdfatfi}.
% Example (linux):
% \begin{quote}
%   |chmod +x scripts/oberdiek/pdfatfi.pl|\\
%   |cp scripts/oberdiek/pdfatfi.pl /usr/local/bin/|
% \end{quote}
%
% \subsection{Package installation}
%
% \paragraph{Unpacking.} The \xfile{.dtx} file is a self-extracting
% \docstrip\ archive. The files are extracted by running the
% \xfile{.dtx} through \plainTeX:
% \begin{quote}
%   \verb|tex enparen.dtx|
% \end{quote}
%
% \paragraph{TDS.} Now the different files must be moved into
% the different directories in your installation TDS tree
% (also known as \xfile{texmf} tree):
% \begin{quote}
% \def\t{^^A
% \begin{tabular}{@{}>{\ttfamily}l@{ $\rightarrow$ }>{\ttfamily}l@{}}
%   enparen.sty & tex/latex/oberdiek/enparen.sty\\
%   enparen.pdf & doc/latex/oberdiek/enparen.pdf\\
%   enparen.dtx & source/latex/oberdiek/enparen.dtx\\
% \end{tabular}^^A
% }^^A
% \sbox0{\t}^^A
% \ifdim\wd0>\linewidth
%   \begingroup
%     \advance\linewidth by\leftmargin
%     \advance\linewidth by\rightmargin
%   \edef\x{\endgroup
%     \def\noexpand\lw{\the\linewidth}^^A
%   }\x
%   \def\lwbox{^^A
%     \leavevmode
%     \hbox to \linewidth{^^A
%       \kern-\leftmargin\relax
%       \hss
%       \usebox0
%       \hss
%       \kern-\rightmargin\relax
%     }^^A
%   }^^A
%   \ifdim\wd0>\lw
%     \sbox0{\small\t}^^A
%     \ifdim\wd0>\linewidth
%       \ifdim\wd0>\lw
%         \sbox0{\footnotesize\t}^^A
%         \ifdim\wd0>\linewidth
%           \ifdim\wd0>\lw
%             \sbox0{\scriptsize\t}^^A
%             \ifdim\wd0>\linewidth
%               \ifdim\wd0>\lw
%                 \sbox0{\tiny\t}^^A
%                 \ifdim\wd0>\linewidth
%                   \lwbox
%                 \else
%                   \usebox0
%                 \fi
%               \else
%                 \lwbox
%               \fi
%             \else
%               \usebox0
%             \fi
%           \else
%             \lwbox
%           \fi
%         \else
%           \usebox0
%         \fi
%       \else
%         \lwbox
%       \fi
%     \else
%       \usebox0
%     \fi
%   \else
%     \lwbox
%   \fi
% \else
%   \usebox0
% \fi
% \end{quote}
% If you have a \xfile{docstrip.cfg} that configures and enables \docstrip's
% TDS installing feature, then some files can already be in the right
% place, see the documentation of \docstrip.
%
% \subsection{Refresh file name databases}
%
% If your \TeX~distribution
% (\teTeX, \mikTeX, \dots) relies on file name databases, you must refresh
% these. For example, \teTeX\ users run \verb|texhash| or
% \verb|mktexlsr|.
%
% \subsection{Some details for the interested}
%
% \paragraph{Attached source.}
%
% The PDF documentation on CTAN also includes the
% \xfile{.dtx} source file. It can be extracted by
% AcrobatReader 6 or higher. Another option is \textsf{pdftk},
% e.g. unpack the file into the current directory:
% \begin{quote}
%   \verb|pdftk enparen.pdf unpack_files output .|
% \end{quote}
%
% \paragraph{Unpacking with \LaTeX.}
% The \xfile{.dtx} chooses its action depending on the format:
% \begin{description}
% \item[\plainTeX:] Run \docstrip\ and extract the files.
% \item[\LaTeX:] Generate the documentation.
% \end{description}
% If you insist on using \LaTeX\ for \docstrip\ (really,
% \docstrip\ does not need \LaTeX), then inform the autodetect routine
% about your intention:
% \begin{quote}
%   \verb|latex \let\install=y% \iffalse meta-comment
%
% File: enparen.dtx
% Version: 2012/01/07 v1.0
% Info: Parentheses nesting
%
% Copyright (C) 2012 by
%    Heiko Oberdiek <heiko.oberdiek at googlemail.com>
%
% This work may be distributed and/or modified under the
% conditions of the LaTeX Project Public License, either
% version 1.3c of this license or (at your option) any later
% version. This version of this license is in
%    http://www.latex-project.org/lppl/lppl-1-3c.txt
% and the latest version of this license is in
%    http://www.latex-project.org/lppl.txt
% and version 1.3 or later is part of all distributions of
% LaTeX version 2005/12/01 or later.
%
% This work has the LPPL maintenance status "maintained".
%
% This Current Maintainer of this work is Heiko Oberdiek.
%
% This work consists of the main source file enparen.dtx
% and the derived files
%    enparen.sty, enparen.pdf, enparen.ins, enparen.drv.
%
% Distribution:
%    CTAN:macros/latex/contrib/oberdiek/enparen.dtx
%    CTAN:macros/latex/contrib/oberdiek/enparen.pdf
%
% Unpacking:
%    (a) If enparen.ins is present:
%           tex enparen.ins
%    (b) Without enparen.ins:
%           tex enparen.dtx
%    (c) If you insist on using LaTeX
%           latex \let\install=y\input{enparen.dtx}
%        (quote the arguments according to the demands of your shell)
%
% Documentation:
%    (a) If enparen.drv is present:
%           latex enparen.drv
%    (b) Without enparen.drv:
%           latex enparen.dtx; ...
%    The class ltxdoc loads the configuration file ltxdoc.cfg
%    if available. Here you can specify further options, e.g.
%    use A4 as paper format:
%       \PassOptionsToClass{a4paper}{article}
%
%    Programm calls to get the documentation (example):
%       pdflatex enparen.dtx
%       makeindex -s gind.ist enparen.idx
%       pdflatex enparen.dtx
%       makeindex -s gind.ist enparen.idx
%       pdflatex enparen.dtx
%
% Installation:
%    TDS:tex/latex/oberdiek/enparen.sty
%    TDS:doc/latex/oberdiek/enparen.pdf
%    TDS:source/latex/oberdiek/enparen.dtx
%
%<*ignore>
\begingroup
  \catcode123=1 %
  \catcode125=2 %
  \def\x{LaTeX2e}%
\expandafter\endgroup
\ifcase 0\ifx\install y1\fi\expandafter
         \ifx\csname processbatchFile\endcsname\relax\else1\fi
         \ifx\fmtname\x\else 1\fi\relax
\else\csname fi\endcsname
%</ignore>
%<*install>
\input docstrip.tex
\Msg{************************************************************************}
\Msg{* Installation}
\Msg{* Package: enparen 2012/01/07 v1.0 Parentheses nesting (HO)}
\Msg{************************************************************************}

\keepsilent
\askforoverwritefalse

\let\MetaPrefix\relax
\preamble

This is a generated file.

Project: enparen
Version: 2012/01/07 v1.0

Copyright (C) 2012 by
   Heiko Oberdiek <heiko.oberdiek at googlemail.com>

This work may be distributed and/or modified under the
conditions of the LaTeX Project Public License, either
version 1.3c of this license or (at your option) any later
version. This version of this license is in
   http://www.latex-project.org/lppl/lppl-1-3c.txt
and the latest version of this license is in
   http://www.latex-project.org/lppl.txt
and version 1.3 or later is part of all distributions of
LaTeX version 2005/12/01 or later.

This work has the LPPL maintenance status "maintained".

This Current Maintainer of this work is Heiko Oberdiek.

This work consists of the main source file enparen.dtx
and the derived files
   enparen.sty, enparen.pdf, enparen.ins, enparen.drv.

\endpreamble
\let\MetaPrefix\DoubleperCent

\generate{%
  \file{enparen.ins}{\from{enparen.dtx}{install}}%
  \file{enparen.drv}{\from{enparen.dtx}{driver}}%
  \usedir{tex/latex/oberdiek}%
  \file{enparen.sty}{\from{enparen.dtx}{package}}%
}

\catcode32=13\relax% active space
\let =\space%
\Msg{************************************************************************}
\Msg{*}
\Msg{* To finish the installation you have to move the following}
\Msg{* file into a directory searched by TeX:}
\Msg{*}
\Msg{*     enparen.sty}
\Msg{*}
\Msg{* To produce the documentation run the file `enparen.drv'}
\Msg{* through LaTeX.}
\Msg{*}
\Msg{* Happy TeXing!}
\Msg{*}
\Msg{************************************************************************}

\endbatchfile
%</install>
%<*ignore>
\fi
%</ignore>
%<*driver>
\NeedsTeXFormat{LaTeX2e}
\ProvidesFile{enparen.drv}%
  [2012/01/07 v1.0 Parentheses nesting (HO)]%
\documentclass{ltxdoc}
\usepackage{holtxdoc}[2011/11/22]
\usepackage{enparen}[2012/01/07]
\begin{document}
  \DocInput{enparen.dtx}%
\end{document}
%</driver>
% \fi
%
% \CheckSum{339}
%
% \CharacterTable
%  {Upper-case    \A\B\C\D\E\F\G\H\I\J\K\L\M\N\O\P\Q\R\S\T\U\V\W\X\Y\Z
%   Lower-case    \a\b\c\d\e\f\g\h\i\j\k\l\m\n\o\p\q\r\s\t\u\v\w\x\y\z
%   Digits        \0\1\2\3\4\5\6\7\8\9
%   Exclamation   \!     Double quote  \"     Hash (number) \#
%   Dollar        \$     Percent       \%     Ampersand     \&
%   Acute accent  \'     Left paren    \(     Right paren   \)
%   Asterisk      \*     Plus          \+     Comma         \,
%   Minus         \-     Point         \.     Solidus       \/
%   Colon         \:     Semicolon     \;     Less than     \<
%   Equals        \=     Greater than  \>     Question mark \?
%   Commercial at \@     Left bracket  \[     Backslash     \\
%   Right bracket \]     Circumflex    \^     Underscore    \_
%   Grave accent  \`     Left brace    \{     Vertical bar  \|
%   Right brace   \}     Tilde         \~}
%
% \GetFileInfo{enparen.drv}
%
% \title{The \xpackage{enparen} package}
% \date{2012/01/07 v1.0}
% \author{Heiko Oberdiek\\\xemail{heiko.oberdiek at googlemail.com}}
%
% \maketitle
%
% \begin{abstract}
% The package defines macros to set parentheses that automatically
% change the symbols from inner to outer fences.
% \end{abstract}
%
% \tableofcontents
%
% \section{Documentation}
%
% The \hologo{LaTeX} package provides macros for automatically changed
% parentheses symbols depending on the fence order.
% The innermost parentheses are always using the same
% parentheses. The symbols changes for the outer fences.
%
% Example:
% \begin{quote}
%   \enparen{a \enparen{b} \enparen{c \enparen{d}} \enparen{e}}
% \end{quote}
% is generated by
%\begin{quote}
%\begin{verbatim}
%\documentclass{article}
%\usepackage{enparen}
%\begin{document}
%  \enparen{a \enparen{b} \enparen{c \enparen{d}} \enparen{e}}
%\end{document}
%\end{verbatim}
%\end{quote}
%
% The package is the result of a newsgroup thread. Dan gives
% the following specification \cite{dan}:
% \begin{quote}
% ``On the other hand, the rules for fences are usually the reverse:
% innermost fences are always |()|, next outer are |[]|, etc. This means
% the opening fence has to wait until all the fences between it and
% the matching close have been detected before it can decide whether
% to be |(| or |[| or |{|.''
% \end{quote}
%
% The fence level counting starts from innermost parentheses with one.
% For the next outer fences the level is increased by one.
% The example above with level indexes:
% \begin{quote}
%   \def\1{\ensuremath{_{1}}}
%   \def\2{\ensuremath{_{2}}}
%   \def\3{\ensuremath{_{3}}}
%   \enparen{\3a \enparen{\1b}\1 \enparen{\2c \enparen{\1d}\1}\2 \enparen{e}}\3
% \end{quote}
% The correct level is only known at the closing symbol. Therefore the
% correct value is remembered in the main \xext{aux} file and used
% in the second \hologo{LaTeX} run.
%
% \subsection{User macros}
%
% \begin{declcs}{enparen} \M{text}
% \end{declcs}
% The macro \cs{enparen} puts its argument \meta{text} in parentheses.
% If the macro is nested, the used fence symbols change for the
% outer fences.
%
% \begin{declcs}{enparenLeft}\\
%   \cs{enparenRight}
% \end{declcs}
% Instead of |\enparen{text}| the left and right symbol can be
% used separately, but in pairs:
% \begin{quote}
%   |\enparenLeft text\enparenRight|
% \end{quote}
% \cs{enparenLeft} and \cs{enparenRight} may be used at different
% group levels, but they must be properly nested.
%
% \begin{declcs}{enparenSetSymbols} \M{level} \M{opening symbol} \M{closing symbol}
% \end{declcs}
% Macro \cs{enparenSetSymbols} configures the \meta{opening symbol} and
% \meta{closing symbol} for the \meta{level}. The \meta{level}
% is a number and the counting starts with one. Level zero is used,
% if the correct level is not known (e.g. in the first \hologo{LaTeX} run).
% The package defines the following sets:
%\begin{quote}
%\begin{verbatim}
%\enparenSetSymbols{0}{(}{)}
%\enparenSetSymbols{1}{(}{)}
%\enparenSetSymbols{2}{[}{]}
%\enparenSetSymbols{3}{\{}{\}}
%\end{verbatim}
%\end{quote}
% Example for changing the third and adding a fourth level:
%\begin{quote}
%\begin{verbatim}
%\enparenSetSymbols{3}{\ensuremath{\langle}}{\ensuremath{\rangle}}
%\enparenSetSymbols{4}{\{}{\}}
%\end{verbatim}
%\end{quote}
%
% \begin{declcs}{enparenUnsetSymbols} \M{level}
% \end{declcs}
% The symbols for level \meta{level} are removed.
% Example scenario: Only two nesting levels must be used,
% the package defines more, then the third level can be
% disabled by |\enparenUnsetSymbols{3}| and the user
% gets warnings if parentheses at level 3 are needed.
%
% \subsection{Contexts}
%
% \begin{declcs}{enparenBeginContext} \M{name}\\
%   \cs{enparenEndContext} \M{name}
% \end{declcs}
% If the current text is interrupted by footnotes, floats
% with captions, then the parentheses inside the text of footnotes,
% captions, \dots should restart from scratch. This can be achieved
% by embedding the text inside macros \cs{enparenBeginContext}
% and \cs{enparenEndContext}. These macros must be properly nested.
% The \meta{name} for the begin and end macro must be the same.
% It is a help for debugging problems, because the warning messages
% show the context name. But it is not necessary that the
% begin\slash end pairs have different names. Example:
%\begin{quote}
%\begin{verbatim}
%\enparenLeft text before table ...
%\begin{table}
%  \caption{Table caption}
%  \enparenBegin{Context}{table}
%  Other text \enparen{foobar}.
%  \enparenEnd{Context}{table}
%\end{table}
%text after table ...
%\enparenRight
%\end{verbatim}
%\end{quote}
% The parentheses inside the table environment and context `table'
% are not nested inside other parentheses: \enparen{foobar}.
% In case of captions and footnotes the contexts are automatically
% added, see next section about options.
%
% \subsection{Options}
%
% \begin{declcs}{enparenSetup} \M{key value list}
% \end{declcs}
% Some options (currently all) can also be set after the
% package is loaded. They can be set in the argument
% \meta{key value list} of \cs{enparenSetup}. Options are disabled
% after they are used the last time. Currently all options are boolean
% options and are disabled in |\begin{document}|.
%
% \begin{description}
% \item[\xoption{caption}:] The caption text is put in a
%   context |caption|.
% \item[\xoption{footnote}:] The footnote text is put in a
%   context |footnote|.
% \end{description}
%
% Example for disabling the two options at different places:
%\begin{quote}
%\begin{verbatim}
%\usepackage[caption=false]{enparen}
%\enparenSetup{footnote=false}
%\end{verbatim}
%\end{quote}
%
% \subsection{Notes}
%
% \begin{description}
% \item[Implicite kerning:]
%   Unexpandable stuff might affect the implicite kerning.
%   The package cannot avoid this, because it need to define
%   and redefine macros at the occurence of each symbol.
%   This is done before the opening and after the closing
%   symbol, thus that the implicite kerning inside
%   is not affected.
% \end{description}
%
% \StopEventually{
% }
%
% \section{Implementation}
%
%    \begin{macrocode}
%<*package>
\NeedsTeXFormat{LaTeX2e}
\ProvidesPackage{enparen}
  [2012/01/07 v1.0 Parentheses nesting (HO)]%
%    \end{macrocode}
%
% \subsection{Resources}
%
%    \begin{macrocode}
\RequirePackage{ltxcmds}[2011/11/09]
%    \end{macrocode}
%
%    \begin{macrocode}
\ltx@IfUndefined{numexpr}{%
  \PackageError{enparen}{%
    Missing e-TeX's \ltx@backslashchar numexpr.\MessageBreak
    The package will continue with emergency definitions%
  }\@ehc
  \def\enparenLeft{(}%
  \def\enparenRight{)}%
  \long\def\enparen#1{\enparenLeft#1\enparenRight}%
  \let\enparenSetup\ltx@gobble
  \let\enparenSetSymbols\ltx@gobblethree
  \let\enparenUnsetSymbols\ltx@gobble
  \endinput
}{}
%    \end{macrocode}
%
%    \begin{macrocode}
\RequirePackage{protecteddef}[2011/01/31]
\RequirePackage{atveryend}[2011/06/30]
\RequirePackage{uniquecounter}[2011/01/30]
\RequirePackage{zref-base}[2011/03/18]
\RequirePackage{kvoptions}[2011/06/30]
\RequirePackage{kvsetkeys}[2011/10/18]
%    \end{macrocode}
%
%    \begin{macro}{\zref@wrapper@mainaux}
%    \begin{macrocode}
\providecommand{\zref@wrapper@mainaux}[1]{%
  \ifx\@auxout\@mainaux
    #1%
  \else
    \begingroup
      \let\@auxout\@mainaux
      #1%
    \endgroup
  \fi
}
%    \end{macrocode}
%    \end{macro}
%
%    \begin{macrocode}
\UniqueCounterNew{enparen}
%    \end{macrocode}
%
%    \begin{macrocode}
\zref@newprop{enparen}[0]{}
%    \end{macrocode}
%
% \subsection{Contexts}
%
%    \begin{macro}{\enparenContextDefault}
%    \begin{macrocode}
\def\enparenContextDefault{main}
%    \end{macrocode}
%    \end{macro}
%
%    \begin{macro}{\enparen@ctx}
%    \begin{macrocode}
\let\enparen@ctx\ltx@empty
%    \end{macrocode}
%    \end{macro}
%
%    \begin{macro}{\enparen@stack}
%    \begin{macrocode}
\let\enparen@stack\ltx@empty
%    \end{macrocode}
%    \end{macro}
%
% \subsubsection{Stack for contexts}
%
%    \begin{macro}{\enparen@CtxStack}
%    \begin{macrocode}
\global\let\enparen@CtxStack\ltx@empty
%    \end{macrocode}
%    \end{macro}
%
%    \begin{macro}{\enparen@CtxStackPush}
%    \begin{macrocode}
\def\enparen@CtxStackPush#1{%
  \xdef\enparen@CtxStack{%
    {\enparen@ctx}{\enparen@stack}%
    \enparen@CtxStack
  }%
  \xdef\enparen@ctx{#1}%
  \global\let\enparen@stack\ltx@empty
}
%    \end{macrocode}
%    \end{macro}
%
%    \begin{macro}{\enparen@CtxStackPop}
%    \begin{macrocode}
\def\enparen@CtxStackPop{%
  \ifx\enparen@CtxStack\ltx@empty
    \PackageWarning{enparen}{%
      Pop request for empty context stack%
    }%
    \global\let\enparen@ctx\enparenContextDefault
    \global\let\enparen@stack\ltx@empty
  \else
    \xdef\enparen@ctx{%
      \expandafter\ltx@car\enparen@CtxStack\@nil
    }%
    \xdef\enparen@stack{%
      \expandafter\ltx@carsecond\enparen@CtxStack\@nil
    }%
    \xdef\enparen@CtxStack{%
      \expandafter\ltx@cdrtwo\enparen@CtxStack\@nil
    }%
  \fi
}
%    \end{macrocode}
%    \end{macro}
%
% \subsubsection{Context user macros}
%
%    \begin{macro}{\enparenBeginContext}
%    \begin{macrocode}
\ProtectedDef*{\enparenBeginContext}[1]{%
  \enparen@CtxStackPush{#1}%
}
%    \end{macrocode}
%    \end{macro}
%    \begin{macro}{\enparenEndContext}
%    \begin{macrocode}
\ProtectedDef*{\enparenEndContext}[1]{%
  \edef\enparen@temp{#1}%
  \ifx\enparen@temp\enparen@ctx
  \else
    \PackageWarning{enparen}{%
      Context mismatch in end request.\MessageBreak
      `#1' should be ended, but current context\MessageBreak
      is `\enparen@ctx'%
    }%
  \fi
  \enparenCheckEmptyStack
  \enparen@CtxStackPop
}
%    \end{macrocode}
%    \end{macro}
%
%    \begin{macro}{\enparenCheckEmptyStack}
%    \begin{macrocode}
\ProtectedDef*{\enparenCheckEmptyStack}[0]{%
  \ifx\enparen@stack\ltx@empty
  \else
    \PackageWarning{enparen}{%
      Ending non-empty stack `\enparen@ctx':\MessageBreak
      \enparen@PrintStack\MessageBreak
    }%
  \fi
}
%    \end{macrocode}
%    \end{macro}
%
%    \begin{macro}{\enparen@PrintStack}
%    \begin{macrocode}
\def\enparen@PrintStack{%
  \expandafter\enparen@PrintStackAux
  \enparen@stack\ltx@empty\ltx@empty
}
%    \end{macrocode}
%    \end{macro}
%    \begin{macro}{\enparen@PrintStackAux}
%    \begin{macrocode}
\def\enparen@PrintStackAux#1#2{%
  \ifx\ltx@empty#1%
  \else
    {#1:#2}%
    \expandafter\enparen@PrintStackAux
  \fi
}
%    \end{macrocode}
%    \end{macro}
%
% \subsection{Symbols}
%
%    \begin{macro}{\enparenSetSymbols}
%    \begin{macrocode}
\ProtectedDef*{\enparenSetSymbols}[3]{%
  \expandafter
  \def\csname enparen@symbol\the\numexpr#1L\endcsname{#2}%
  \expandafter
  \def\csname enparen@symbol\the\numexpr#1R\endcsname{#3}%
}
%    \end{macrocode}
%    \end{macro}
%    \begin{macro}{\enparenUnsetSymbols}
%    \begin{macrocode}
\ProtectedDef*{\enparenUnsetSymbols}[1]{%
  \expandafter
  \let\csname enparen@symbol\the\numexpr#1L\endcsname\ltx@undefined
  \expandafter
  \let\csname enparen@symbol\the\numexpr#1R\endcsname\ltx@undefined
}
%    \end{macrocode}
%    \end{macro}
%
%    \begin{macrocode}
\enparenSetSymbols{0}{(}{)}
\enparenSetSymbols{1}{(}{)}
\enparenSetSymbols{2}{[}{]}
\enparenSetSymbols{3}{\{}{\}}
\enparenSetSymbols{4}{\ensuremath{\langle}}{\ensuremath{\rangle}}
%    \end{macrocode}
%
% \subsection{Main user macros}
%
%    \begin{macro}{\enparen}
%    \begin{macrocode}
\ProtectedDef{\enparen}[1]{%
  \enparenLeft#1\enparenRight
}
%    \end{macrocode}
%    \end{macro}
%
%    \begin{macro}{\enparenLeft}
%    \begin{macrocode}
\ProtectedDef*{\enparenLeft}[0]{%
  \UniqueCounterCall{enparen}\enparen@Left
}
%    \end{macrocode}
%    \end{macro}
%    \begin{macro}{\enparen@Left}
%    \begin{macrocode}
\def\enparen@Left#1{%
  \xdef\enparen@stack{%
    {#1}{1}%
    \expandafter\enparen@Inc\expandafter2\expandafter!%
    \enparen@stack\ltx@empty\ltx@empty
  }%
  \edef\enparen@tmp{\zref@extract{enparen#1}{enparen}}%
  \ltx@IfUndefined{enparen@symbol\enparen@tmp L}{%
    \PackageWarning{enparen}{%
      Undefined symbols for level \enparen@tmp
    }%
    \csname enparen@symbol0L\endcsname
  }{%
    \csname enparen@symbol\enparen@tmp L\endcsname
  }%
}
%    \end{macrocode}
%    \end{macro}
%    \begin{macro}{\enparen@Inc}
%    \begin{macrocode}
\def\enparen@Inc#1!#2#3{%
  \ifx\ltx@empty#2%
  \else
    \ifnum#3<#1 %
      {#2}{#1}%
      \expandafter\enparen@Inc
      \the\numexpr#1+1\expandafter\expandafter\expandafter!%
    \else
      {#2}{#3}%
    \fi
  \fi
}
%    \end{macrocode}
%    \end{macro}
%
%    \begin{macro}{\enparenRight}
%    \begin{macrocode}
\ProtectedDef*{\enparenRight}[0]{%
  \ifx\enparen@stack\ltx@empty
    \PackageWarning{enparen}{%
      Missing left symbol for right symbol%
    }%
    \csname enparen@symbol0R\endcsname
  \else
    \expandafter\enparen@Right\enparen@stack\@nil
  \fi
}
%    \end{macrocode}
%    \end{macro}
%
%    \begin{macro}{\enparen@Right}
%    \begin{macrocode}
\def\enparen@Right#1#2#3\@nil{%
  \ltx@IfUndefined{%
    enparen@symbol%
    \zref@extract{enparen#1}{enparen}%
    R%
  }{%
    \csname enparen@symbol0R\endcsname
  }{%
    \csname
      enparen@symbol%
      \zref@extract{enparen#1}{enparen}%
       R%
    \endcsname
  }%
  \zref@wrapper@mainaux{%
    \zref@setcurrent{enparen}{#2}%
    \zref@labelbyprops{enparen#1}{enparen}%
  }%
  \xdef\enparen@stack{#3}%
}
%    \end{macrocode}
%    \end{macro}
%
% \subsection{Options}
%
%    \begin{macrocode}
\SetupKeyvalOptions{%
  family=enparen,%
  prefix=enparen@,%
}
%    \end{macrocode}
%
%    \begin{macro}{\enparenSetup}
%    \begin{macrocode}
\ProtectedDef*{\enparenSetup}[0]{%
  \kvsetkeys{enparen}%
}
%    \end{macrocode}
%    \end{macro}
%
%    \begin{macrocode}
\DeclareBoolOption[true]{footnote}
\DeclareBoolOption[true]{caption}
\ProcessKeyvalOptions*
%    \end{macrocode}
%
% \subsection{Context settings}
%
%    \begin{macro}{\enparen@AtBegin}
%    \begin{macrocode}
\def\enparen@AtBegin{%
  \ifenparen@footnote
    \let\enparen@org@makefntext\@makefntext
    \long\def\@makefntext##1{%
      \enparen@org@makefntext{%
        \enparenBeginContext{footnote}%
        ##1%
        \enparenEndContext{footnote}%
      }%
    }%
  \fi
  \enparen@Disable{footnote}%
  \ifenparen@caption
    \let\enparen@org@makecaption\@makecaption
    \long\def\@makecaption##1##2{%
      \enparen@org@makecaption{##1}{%
        \enparenBeginContext{caption}%
        ##2%
        \enparenEndContext{caption}%
      }%
    }%
  \fi
  \enparen@Disable{caption}%
}
%    \end{macrocode}
%    \end{macro}
%
%    \begin{macro}{\enparen@Disable}
%    \begin{macrocode}
\def\enparen@Disable#1{%
  \DisableKeyvalOption[%
    action=warning,%
    package=enparen,%
  ]{enparen}{#1}%
}
%    \end{macrocode}
%    \end{macro}
%
%    \begin{macrocode}
\AtBeginDocument{\enparen@AtBegin}
%    \end{macrocode}
%
% \subsection{At end of document}
%
%    \begin{macro}{\enparen@AtEnd}
%    \begin{macrocode}
\def\enparen@AtEnd{%
  \enparenCheckEmptyStack
  \ifx\enparen@CtxStack\ltx@empty
  \else
    \PackageWarningNoLine{enparen}{%
      Context stack is not empty at end of document.\MessageBreak
      Current stack and contents of context stack:\MessageBreak
      [\enparen@ctx]:[\enparen@PrintStack]%
      \expandafter
      \enparen@PrintContextStack\enparen@CtxStack\relax\relax
    }%
  \fi
}
%    \end{macrocode}
%    \end{macro}
%    \begin{macro}{\enparen@PrintContextStack}
%    \begin{macrocode}
\def\enparen@PrintContextStack#1#2{%
  \ifx\relax#1\ltx@empty
  \else
    \MessageBreak
    [#1]:[\enparen@PrintStackAux#2\ltx@empty\ltx@empty]% hash-ok
    \expandafter\enparen@PrintContextStack
  \fi
}
%    \end{macrocode}
%    \end{macro}
%    \begin{macrocode}
\AtVeryEndDocument{\enparen@AtEnd}
%    \end{macrocode}
%
%    \begin{macrocode}
%</package>
%    \end{macrocode}
%
% \section{Installation}
%
% \subsection{Download}
%
% \paragraph{Package.} This package is available on
% CTAN\footnote{\url{ftp://ftp.ctan.org/tex-archive/}}:
% \begin{description}
% \item[\CTAN{macros/latex/contrib/oberdiek/enparen.dtx}] The source file.
% \item[\CTAN{macros/latex/contrib/oberdiek/enparen.pdf}] Documentation.
% \end{description}
%
%
% \paragraph{Bundle.} All the packages of the bundle `oberdiek'
% are also available in a TDS compliant ZIP archive. There
% the packages are already unpacked and the documentation files
% are generated. The files and directories obey the TDS standard.
% \begin{description}
% \item[\CTAN{install/macros/latex/contrib/oberdiek.tds.zip}]
% \end{description}
% \emph{TDS} refers to the standard ``A Directory Structure
% for \TeX\ Files'' (\CTAN{tds/tds.pdf}). Directories
% with \xfile{texmf} in their name are usually organized this way.
%
% \subsection{Bundle installation}
%
% \paragraph{Unpacking.} Unpack the \xfile{oberdiek.tds.zip} in the
% TDS tree (also known as \xfile{texmf} tree) of your choice.
% Example (linux):
% \begin{quote}
%   |unzip oberdiek.tds.zip -d ~/texmf|
% \end{quote}
%
% \paragraph{Script installation.}
% Check the directory \xfile{TDS:scripts/oberdiek/} for
% scripts that need further installation steps.
% Package \xpackage{attachfile2} comes with the Perl script
% \xfile{pdfatfi.pl} that should be installed in such a way
% that it can be called as \texttt{pdfatfi}.
% Example (linux):
% \begin{quote}
%   |chmod +x scripts/oberdiek/pdfatfi.pl|\\
%   |cp scripts/oberdiek/pdfatfi.pl /usr/local/bin/|
% \end{quote}
%
% \subsection{Package installation}
%
% \paragraph{Unpacking.} The \xfile{.dtx} file is a self-extracting
% \docstrip\ archive. The files are extracted by running the
% \xfile{.dtx} through \plainTeX:
% \begin{quote}
%   \verb|tex enparen.dtx|
% \end{quote}
%
% \paragraph{TDS.} Now the different files must be moved into
% the different directories in your installation TDS tree
% (also known as \xfile{texmf} tree):
% \begin{quote}
% \def\t{^^A
% \begin{tabular}{@{}>{\ttfamily}l@{ $\rightarrow$ }>{\ttfamily}l@{}}
%   enparen.sty & tex/latex/oberdiek/enparen.sty\\
%   enparen.pdf & doc/latex/oberdiek/enparen.pdf\\
%   enparen.dtx & source/latex/oberdiek/enparen.dtx\\
% \end{tabular}^^A
% }^^A
% \sbox0{\t}^^A
% \ifdim\wd0>\linewidth
%   \begingroup
%     \advance\linewidth by\leftmargin
%     \advance\linewidth by\rightmargin
%   \edef\x{\endgroup
%     \def\noexpand\lw{\the\linewidth}^^A
%   }\x
%   \def\lwbox{^^A
%     \leavevmode
%     \hbox to \linewidth{^^A
%       \kern-\leftmargin\relax
%       \hss
%       \usebox0
%       \hss
%       \kern-\rightmargin\relax
%     }^^A
%   }^^A
%   \ifdim\wd0>\lw
%     \sbox0{\small\t}^^A
%     \ifdim\wd0>\linewidth
%       \ifdim\wd0>\lw
%         \sbox0{\footnotesize\t}^^A
%         \ifdim\wd0>\linewidth
%           \ifdim\wd0>\lw
%             \sbox0{\scriptsize\t}^^A
%             \ifdim\wd0>\linewidth
%               \ifdim\wd0>\lw
%                 \sbox0{\tiny\t}^^A
%                 \ifdim\wd0>\linewidth
%                   \lwbox
%                 \else
%                   \usebox0
%                 \fi
%               \else
%                 \lwbox
%               \fi
%             \else
%               \usebox0
%             \fi
%           \else
%             \lwbox
%           \fi
%         \else
%           \usebox0
%         \fi
%       \else
%         \lwbox
%       \fi
%     \else
%       \usebox0
%     \fi
%   \else
%     \lwbox
%   \fi
% \else
%   \usebox0
% \fi
% \end{quote}
% If you have a \xfile{docstrip.cfg} that configures and enables \docstrip's
% TDS installing feature, then some files can already be in the right
% place, see the documentation of \docstrip.
%
% \subsection{Refresh file name databases}
%
% If your \TeX~distribution
% (\teTeX, \mikTeX, \dots) relies on file name databases, you must refresh
% these. For example, \teTeX\ users run \verb|texhash| or
% \verb|mktexlsr|.
%
% \subsection{Some details for the interested}
%
% \paragraph{Attached source.}
%
% The PDF documentation on CTAN also includes the
% \xfile{.dtx} source file. It can be extracted by
% AcrobatReader 6 or higher. Another option is \textsf{pdftk},
% e.g. unpack the file into the current directory:
% \begin{quote}
%   \verb|pdftk enparen.pdf unpack_files output .|
% \end{quote}
%
% \paragraph{Unpacking with \LaTeX.}
% The \xfile{.dtx} chooses its action depending on the format:
% \begin{description}
% \item[\plainTeX:] Run \docstrip\ and extract the files.
% \item[\LaTeX:] Generate the documentation.
% \end{description}
% If you insist on using \LaTeX\ for \docstrip\ (really,
% \docstrip\ does not need \LaTeX), then inform the autodetect routine
% about your intention:
% \begin{quote}
%   \verb|latex \let\install=y\input{enparen.dtx}|
% \end{quote}
% Do not forget to quote the argument according to the demands
% of your shell.
%
% \paragraph{Generating the documentation.}
% You can use both the \xfile{.dtx} or the \xfile{.drv} to generate
% the documentation. The process can be configured by the
% configuration file \xfile{ltxdoc.cfg}. For instance, put this
% line into this file, if you want to have A4 as paper format:
% \begin{quote}
%   \verb|\PassOptionsToClass{a4paper}{article}|
% \end{quote}
% An example follows how to generate the
% documentation with pdf\LaTeX:
% \begin{quote}
%\begin{verbatim}
%pdflatex enparen.dtx
%makeindex -s gind.ist enparen.idx
%pdflatex enparen.dtx
%makeindex -s gind.ist enparen.idx
%pdflatex enparen.dtx
%\end{verbatim}
% \end{quote}
%
% \begin{thebibliography}{9}
% \raggedright
% \bibitem{dan}
%   Dan Luecking: \textit{Re: bracket order};
%   newsgroup \xnewsgroup{comp.text.tex};
%   message id \texttt{\small$\langle$9b07c9c8-ff92-4cbf-b3a9-84efecfeb506@l24g2000yqm.googlegroups.com$\rangle$}
%   2012-01-05.
%   \url{http://groups.google.com/group/comp.text.tex/msg/8774519da31c2352}
% \end{thebibliography}
%
% \begin{History}
%   \begin{Version}{2012/01/07 v1.0}
%   \item
%     First version.
%   \end{Version}
% \end{History}
%
% \PrintIndex
%
% \Finale
\endinput
|
% \end{quote}
% Do not forget to quote the argument according to the demands
% of your shell.
%
% \paragraph{Generating the documentation.}
% You can use both the \xfile{.dtx} or the \xfile{.drv} to generate
% the documentation. The process can be configured by the
% configuration file \xfile{ltxdoc.cfg}. For instance, put this
% line into this file, if you want to have A4 as paper format:
% \begin{quote}
%   \verb|\PassOptionsToClass{a4paper}{article}|
% \end{quote}
% An example follows how to generate the
% documentation with pdf\LaTeX:
% \begin{quote}
%\begin{verbatim}
%pdflatex enparen.dtx
%makeindex -s gind.ist enparen.idx
%pdflatex enparen.dtx
%makeindex -s gind.ist enparen.idx
%pdflatex enparen.dtx
%\end{verbatim}
% \end{quote}
%
% \begin{thebibliography}{9}
% \raggedright
% \bibitem{dan}
%   Dan Luecking: \textit{Re: bracket order};
%   newsgroup \xnewsgroup{comp.text.tex};
%   message id \texttt{\small$\langle$9b07c9c8-ff92-4cbf-b3a9-84efecfeb506@l24g2000yqm.googlegroups.com$\rangle$}
%   2012-01-05.
%   \url{http://groups.google.com/group/comp.text.tex/msg/8774519da31c2352}
% \end{thebibliography}
%
% \begin{History}
%   \begin{Version}{2012/01/07 v1.0}
%   \item
%     First version.
%   \end{Version}
% \end{History}
%
% \PrintIndex
%
% \Finale
\endinput

%        (quote the arguments according to the demands of your shell)
%
% Documentation:
%    (a) If enparen.drv is present:
%           latex enparen.drv
%    (b) Without enparen.drv:
%           latex enparen.dtx; ...
%    The class ltxdoc loads the configuration file ltxdoc.cfg
%    if available. Here you can specify further options, e.g.
%    use A4 as paper format:
%       \PassOptionsToClass{a4paper}{article}
%
%    Programm calls to get the documentation (example):
%       pdflatex enparen.dtx
%       makeindex -s gind.ist enparen.idx
%       pdflatex enparen.dtx
%       makeindex -s gind.ist enparen.idx
%       pdflatex enparen.dtx
%
% Installation:
%    TDS:tex/latex/oberdiek/enparen.sty
%    TDS:doc/latex/oberdiek/enparen.pdf
%    TDS:source/latex/oberdiek/enparen.dtx
%
%<*ignore>
\begingroup
  \catcode123=1 %
  \catcode125=2 %
  \def\x{LaTeX2e}%
\expandafter\endgroup
\ifcase 0\ifx\install y1\fi\expandafter
         \ifx\csname processbatchFile\endcsname\relax\else1\fi
         \ifx\fmtname\x\else 1\fi\relax
\else\csname fi\endcsname
%</ignore>
%<*install>
\input docstrip.tex
\Msg{************************************************************************}
\Msg{* Installation}
\Msg{* Package: enparen 2012/01/07 v1.0 Parentheses nesting (HO)}
\Msg{************************************************************************}

\keepsilent
\askforoverwritefalse

\let\MetaPrefix\relax
\preamble

This is a generated file.

Project: enparen
Version: 2012/01/07 v1.0

Copyright (C) 2012 by
   Heiko Oberdiek <heiko.oberdiek at googlemail.com>

This work may be distributed and/or modified under the
conditions of the LaTeX Project Public License, either
version 1.3c of this license or (at your option) any later
version. This version of this license is in
   http://www.latex-project.org/lppl/lppl-1-3c.txt
and the latest version of this license is in
   http://www.latex-project.org/lppl.txt
and version 1.3 or later is part of all distributions of
LaTeX version 2005/12/01 or later.

This work has the LPPL maintenance status "maintained".

This Current Maintainer of this work is Heiko Oberdiek.

This work consists of the main source file enparen.dtx
and the derived files
   enparen.sty, enparen.pdf, enparen.ins, enparen.drv.

\endpreamble
\let\MetaPrefix\DoubleperCent

\generate{%
  \file{enparen.ins}{\from{enparen.dtx}{install}}%
  \file{enparen.drv}{\from{enparen.dtx}{driver}}%
  \usedir{tex/latex/oberdiek}%
  \file{enparen.sty}{\from{enparen.dtx}{package}}%
}

\catcode32=13\relax% active space
\let =\space%
\Msg{************************************************************************}
\Msg{*}
\Msg{* To finish the installation you have to move the following}
\Msg{* file into a directory searched by TeX:}
\Msg{*}
\Msg{*     enparen.sty}
\Msg{*}
\Msg{* To produce the documentation run the file `enparen.drv'}
\Msg{* through LaTeX.}
\Msg{*}
\Msg{* Happy TeXing!}
\Msg{*}
\Msg{************************************************************************}

\endbatchfile
%</install>
%<*ignore>
\fi
%</ignore>
%<*driver>
\NeedsTeXFormat{LaTeX2e}
\ProvidesFile{enparen.drv}%
  [2012/01/07 v1.0 Parentheses nesting (HO)]%
\documentclass{ltxdoc}
\usepackage{holtxdoc}[2011/11/22]
\usepackage{enparen}[2012/01/07]
\begin{document}
  \DocInput{enparen.dtx}%
\end{document}
%</driver>
% \fi
%
% \CheckSum{339}
%
% \CharacterTable
%  {Upper-case    \A\B\C\D\E\F\G\H\I\J\K\L\M\N\O\P\Q\R\S\T\U\V\W\X\Y\Z
%   Lower-case    \a\b\c\d\e\f\g\h\i\j\k\l\m\n\o\p\q\r\s\t\u\v\w\x\y\z
%   Digits        \0\1\2\3\4\5\6\7\8\9
%   Exclamation   \!     Double quote  \"     Hash (number) \#
%   Dollar        \$     Percent       \%     Ampersand     \&
%   Acute accent  \'     Left paren    \(     Right paren   \)
%   Asterisk      \*     Plus          \+     Comma         \,
%   Minus         \-     Point         \.     Solidus       \/
%   Colon         \:     Semicolon     \;     Less than     \<
%   Equals        \=     Greater than  \>     Question mark \?
%   Commercial at \@     Left bracket  \[     Backslash     \\
%   Right bracket \]     Circumflex    \^     Underscore    \_
%   Grave accent  \`     Left brace    \{     Vertical bar  \|
%   Right brace   \}     Tilde         \~}
%
% \GetFileInfo{enparen.drv}
%
% \title{The \xpackage{enparen} package}
% \date{2012/01/07 v1.0}
% \author{Heiko Oberdiek\\\xemail{heiko.oberdiek at googlemail.com}}
%
% \maketitle
%
% \begin{abstract}
% The package defines macros to set parentheses that automatically
% change the symbols from inner to outer fences.
% \end{abstract}
%
% \tableofcontents
%
% \section{Documentation}
%
% The \hologo{LaTeX} package provides macros for automatically changed
% parentheses symbols depending on the fence order.
% The innermost parentheses are always using the same
% parentheses. The symbols changes for the outer fences.
%
% Example:
% \begin{quote}
%   \enparen{a \enparen{b} \enparen{c \enparen{d}} \enparen{e}}
% \end{quote}
% is generated by
%\begin{quote}
%\begin{verbatim}
%\documentclass{article}
%\usepackage{enparen}
%\begin{document}
%  \enparen{a \enparen{b} \enparen{c \enparen{d}} \enparen{e}}
%\end{document}
%\end{verbatim}
%\end{quote}
%
% The package is the result of a newsgroup thread. Dan gives
% the following specification \cite{dan}:
% \begin{quote}
% ``On the other hand, the rules for fences are usually the reverse:
% innermost fences are always |()|, next outer are |[]|, etc. This means
% the opening fence has to wait until all the fences between it and
% the matching close have been detected before it can decide whether
% to be |(| or |[| or |{|.''
% \end{quote}
%
% The fence level counting starts from innermost parentheses with one.
% For the next outer fences the level is increased by one.
% The example above with level indexes:
% \begin{quote}
%   \def\1{\ensuremath{_{1}}}
%   \def\2{\ensuremath{_{2}}}
%   \def\3{\ensuremath{_{3}}}
%   \enparen{\3a \enparen{\1b}\1 \enparen{\2c \enparen{\1d}\1}\2 \enparen{e}}\3
% \end{quote}
% The correct level is only known at the closing symbol. Therefore the
% correct value is remembered in the main \xext{aux} file and used
% in the second \hologo{LaTeX} run.
%
% \subsection{User macros}
%
% \begin{declcs}{enparen} \M{text}
% \end{declcs}
% The macro \cs{enparen} puts its argument \meta{text} in parentheses.
% If the macro is nested, the used fence symbols change for the
% outer fences.
%
% \begin{declcs}{enparenLeft}\\
%   \cs{enparenRight}
% \end{declcs}
% Instead of |\enparen{text}| the left and right symbol can be
% used separately, but in pairs:
% \begin{quote}
%   |\enparenLeft text\enparenRight|
% \end{quote}
% \cs{enparenLeft} and \cs{enparenRight} may be used at different
% group levels, but they must be properly nested.
%
% \begin{declcs}{enparenSetSymbols} \M{level} \M{opening symbol} \M{closing symbol}
% \end{declcs}
% Macro \cs{enparenSetSymbols} configures the \meta{opening symbol} and
% \meta{closing symbol} for the \meta{level}. The \meta{level}
% is a number and the counting starts with one. Level zero is used,
% if the correct level is not known (e.g. in the first \hologo{LaTeX} run).
% The package defines the following sets:
%\begin{quote}
%\begin{verbatim}
%\enparenSetSymbols{0}{(}{)}
%\enparenSetSymbols{1}{(}{)}
%\enparenSetSymbols{2}{[}{]}
%\enparenSetSymbols{3}{\{}{\}}
%\end{verbatim}
%\end{quote}
% Example for changing the third and adding a fourth level:
%\begin{quote}
%\begin{verbatim}
%\enparenSetSymbols{3}{\ensuremath{\langle}}{\ensuremath{\rangle}}
%\enparenSetSymbols{4}{\{}{\}}
%\end{verbatim}
%\end{quote}
%
% \begin{declcs}{enparenUnsetSymbols} \M{level}
% \end{declcs}
% The symbols for level \meta{level} are removed.
% Example scenario: Only two nesting levels must be used,
% the package defines more, then the third level can be
% disabled by |\enparenUnsetSymbols{3}| and the user
% gets warnings if parentheses at level 3 are needed.
%
% \subsection{Contexts}
%
% \begin{declcs}{enparenBeginContext} \M{name}\\
%   \cs{enparenEndContext} \M{name}
% \end{declcs}
% If the current text is interrupted by footnotes, floats
% with captions, then the parentheses inside the text of footnotes,
% captions, \dots should restart from scratch. This can be achieved
% by embedding the text inside macros \cs{enparenBeginContext}
% and \cs{enparenEndContext}. These macros must be properly nested.
% The \meta{name} for the begin and end macro must be the same.
% It is a help for debugging problems, because the warning messages
% show the context name. But it is not necessary that the
% begin\slash end pairs have different names. Example:
%\begin{quote}
%\begin{verbatim}
%\enparenLeft text before table ...
%\begin{table}
%  \caption{Table caption}
%  \enparenBegin{Context}{table}
%  Other text \enparen{foobar}.
%  \enparenEnd{Context}{table}
%\end{table}
%text after table ...
%\enparenRight
%\end{verbatim}
%\end{quote}
% The parentheses inside the table environment and context `table'
% are not nested inside other parentheses: \enparen{foobar}.
% In case of captions and footnotes the contexts are automatically
% added, see next section about options.
%
% \subsection{Options}
%
% \begin{declcs}{enparenSetup} \M{key value list}
% \end{declcs}
% Some options (currently all) can also be set after the
% package is loaded. They can be set in the argument
% \meta{key value list} of \cs{enparenSetup}. Options are disabled
% after they are used the last time. Currently all options are boolean
% options and are disabled in |\begin{document}|.
%
% \begin{description}
% \item[\xoption{caption}:] The caption text is put in a
%   context |caption|.
% \item[\xoption{footnote}:] The footnote text is put in a
%   context |footnote|.
% \end{description}
%
% Example for disabling the two options at different places:
%\begin{quote}
%\begin{verbatim}
%\usepackage[caption=false]{enparen}
%\enparenSetup{footnote=false}
%\end{verbatim}
%\end{quote}
%
% \subsection{Notes}
%
% \begin{description}
% \item[Implicite kerning:]
%   Unexpandable stuff might affect the implicite kerning.
%   The package cannot avoid this, because it need to define
%   and redefine macros at the occurence of each symbol.
%   This is done before the opening and after the closing
%   symbol, thus that the implicite kerning inside
%   is not affected.
% \end{description}
%
% \StopEventually{
% }
%
% \section{Implementation}
%
%    \begin{macrocode}
%<*package>
\NeedsTeXFormat{LaTeX2e}
\ProvidesPackage{enparen}
  [2012/01/07 v1.0 Parentheses nesting (HO)]%
%    \end{macrocode}
%
% \subsection{Resources}
%
%    \begin{macrocode}
\RequirePackage{ltxcmds}[2011/11/09]
%    \end{macrocode}
%
%    \begin{macrocode}
\ltx@IfUndefined{numexpr}{%
  \PackageError{enparen}{%
    Missing e-TeX's \ltx@backslashchar numexpr.\MessageBreak
    The package will continue with emergency definitions%
  }\@ehc
  \def\enparenLeft{(}%
  \def\enparenRight{)}%
  \long\def\enparen#1{\enparenLeft#1\enparenRight}%
  \let\enparenSetup\ltx@gobble
  \let\enparenSetSymbols\ltx@gobblethree
  \let\enparenUnsetSymbols\ltx@gobble
  \endinput
}{}
%    \end{macrocode}
%
%    \begin{macrocode}
\RequirePackage{protecteddef}[2011/01/31]
\RequirePackage{atveryend}[2011/06/30]
\RequirePackage{uniquecounter}[2011/01/30]
\RequirePackage{zref-base}[2011/03/18]
\RequirePackage{kvoptions}[2011/06/30]
\RequirePackage{kvsetkeys}[2011/10/18]
%    \end{macrocode}
%
%    \begin{macro}{\zref@wrapper@mainaux}
%    \begin{macrocode}
\providecommand{\zref@wrapper@mainaux}[1]{%
  \ifx\@auxout\@mainaux
    #1%
  \else
    \begingroup
      \let\@auxout\@mainaux
      #1%
    \endgroup
  \fi
}
%    \end{macrocode}
%    \end{macro}
%
%    \begin{macrocode}
\UniqueCounterNew{enparen}
%    \end{macrocode}
%
%    \begin{macrocode}
\zref@newprop{enparen}[0]{}
%    \end{macrocode}
%
% \subsection{Contexts}
%
%    \begin{macro}{\enparenContextDefault}
%    \begin{macrocode}
\def\enparenContextDefault{main}
%    \end{macrocode}
%    \end{macro}
%
%    \begin{macro}{\enparen@ctx}
%    \begin{macrocode}
\let\enparen@ctx\ltx@empty
%    \end{macrocode}
%    \end{macro}
%
%    \begin{macro}{\enparen@stack}
%    \begin{macrocode}
\let\enparen@stack\ltx@empty
%    \end{macrocode}
%    \end{macro}
%
% \subsubsection{Stack for contexts}
%
%    \begin{macro}{\enparen@CtxStack}
%    \begin{macrocode}
\global\let\enparen@CtxStack\ltx@empty
%    \end{macrocode}
%    \end{macro}
%
%    \begin{macro}{\enparen@CtxStackPush}
%    \begin{macrocode}
\def\enparen@CtxStackPush#1{%
  \xdef\enparen@CtxStack{%
    {\enparen@ctx}{\enparen@stack}%
    \enparen@CtxStack
  }%
  \xdef\enparen@ctx{#1}%
  \global\let\enparen@stack\ltx@empty
}
%    \end{macrocode}
%    \end{macro}
%
%    \begin{macro}{\enparen@CtxStackPop}
%    \begin{macrocode}
\def\enparen@CtxStackPop{%
  \ifx\enparen@CtxStack\ltx@empty
    \PackageWarning{enparen}{%
      Pop request for empty context stack%
    }%
    \global\let\enparen@ctx\enparenContextDefault
    \global\let\enparen@stack\ltx@empty
  \else
    \xdef\enparen@ctx{%
      \expandafter\ltx@car\enparen@CtxStack\@nil
    }%
    \xdef\enparen@stack{%
      \expandafter\ltx@carsecond\enparen@CtxStack\@nil
    }%
    \xdef\enparen@CtxStack{%
      \expandafter\ltx@cdrtwo\enparen@CtxStack\@nil
    }%
  \fi
}
%    \end{macrocode}
%    \end{macro}
%
% \subsubsection{Context user macros}
%
%    \begin{macro}{\enparenBeginContext}
%    \begin{macrocode}
\ProtectedDef*{\enparenBeginContext}[1]{%
  \enparen@CtxStackPush{#1}%
}
%    \end{macrocode}
%    \end{macro}
%    \begin{macro}{\enparenEndContext}
%    \begin{macrocode}
\ProtectedDef*{\enparenEndContext}[1]{%
  \edef\enparen@temp{#1}%
  \ifx\enparen@temp\enparen@ctx
  \else
    \PackageWarning{enparen}{%
      Context mismatch in end request.\MessageBreak
      `#1' should be ended, but current context\MessageBreak
      is `\enparen@ctx'%
    }%
  \fi
  \enparenCheckEmptyStack
  \enparen@CtxStackPop
}
%    \end{macrocode}
%    \end{macro}
%
%    \begin{macro}{\enparenCheckEmptyStack}
%    \begin{macrocode}
\ProtectedDef*{\enparenCheckEmptyStack}[0]{%
  \ifx\enparen@stack\ltx@empty
  \else
    \PackageWarning{enparen}{%
      Ending non-empty stack `\enparen@ctx':\MessageBreak
      \enparen@PrintStack\MessageBreak
    }%
  \fi
}
%    \end{macrocode}
%    \end{macro}
%
%    \begin{macro}{\enparen@PrintStack}
%    \begin{macrocode}
\def\enparen@PrintStack{%
  \expandafter\enparen@PrintStackAux
  \enparen@stack\ltx@empty\ltx@empty
}
%    \end{macrocode}
%    \end{macro}
%    \begin{macro}{\enparen@PrintStackAux}
%    \begin{macrocode}
\def\enparen@PrintStackAux#1#2{%
  \ifx\ltx@empty#1%
  \else
    {#1:#2}%
    \expandafter\enparen@PrintStackAux
  \fi
}
%    \end{macrocode}
%    \end{macro}
%
% \subsection{Symbols}
%
%    \begin{macro}{\enparenSetSymbols}
%    \begin{macrocode}
\ProtectedDef*{\enparenSetSymbols}[3]{%
  \expandafter
  \def\csname enparen@symbol\the\numexpr#1L\endcsname{#2}%
  \expandafter
  \def\csname enparen@symbol\the\numexpr#1R\endcsname{#3}%
}
%    \end{macrocode}
%    \end{macro}
%    \begin{macro}{\enparenUnsetSymbols}
%    \begin{macrocode}
\ProtectedDef*{\enparenUnsetSymbols}[1]{%
  \expandafter
  \let\csname enparen@symbol\the\numexpr#1L\endcsname\ltx@undefined
  \expandafter
  \let\csname enparen@symbol\the\numexpr#1R\endcsname\ltx@undefined
}
%    \end{macrocode}
%    \end{macro}
%
%    \begin{macrocode}
\enparenSetSymbols{0}{(}{)}
\enparenSetSymbols{1}{(}{)}
\enparenSetSymbols{2}{[}{]}
\enparenSetSymbols{3}{\{}{\}}
\enparenSetSymbols{4}{\ensuremath{\langle}}{\ensuremath{\rangle}}
%    \end{macrocode}
%
% \subsection{Main user macros}
%
%    \begin{macro}{\enparen}
%    \begin{macrocode}
\ProtectedDef{\enparen}[1]{%
  \enparenLeft#1\enparenRight
}
%    \end{macrocode}
%    \end{macro}
%
%    \begin{macro}{\enparenLeft}
%    \begin{macrocode}
\ProtectedDef*{\enparenLeft}[0]{%
  \UniqueCounterCall{enparen}\enparen@Left
}
%    \end{macrocode}
%    \end{macro}
%    \begin{macro}{\enparen@Left}
%    \begin{macrocode}
\def\enparen@Left#1{%
  \xdef\enparen@stack{%
    {#1}{1}%
    \expandafter\enparen@Inc\expandafter2\expandafter!%
    \enparen@stack\ltx@empty\ltx@empty
  }%
  \edef\enparen@tmp{\zref@extract{enparen#1}{enparen}}%
  \ltx@IfUndefined{enparen@symbol\enparen@tmp L}{%
    \PackageWarning{enparen}{%
      Undefined symbols for level \enparen@tmp
    }%
    \csname enparen@symbol0L\endcsname
  }{%
    \csname enparen@symbol\enparen@tmp L\endcsname
  }%
}
%    \end{macrocode}
%    \end{macro}
%    \begin{macro}{\enparen@Inc}
%    \begin{macrocode}
\def\enparen@Inc#1!#2#3{%
  \ifx\ltx@empty#2%
  \else
    \ifnum#3<#1 %
      {#2}{#1}%
      \expandafter\enparen@Inc
      \the\numexpr#1+1\expandafter\expandafter\expandafter!%
    \else
      {#2}{#3}%
    \fi
  \fi
}
%    \end{macrocode}
%    \end{macro}
%
%    \begin{macro}{\enparenRight}
%    \begin{macrocode}
\ProtectedDef*{\enparenRight}[0]{%
  \ifx\enparen@stack\ltx@empty
    \PackageWarning{enparen}{%
      Missing left symbol for right symbol%
    }%
    \csname enparen@symbol0R\endcsname
  \else
    \expandafter\enparen@Right\enparen@stack\@nil
  \fi
}
%    \end{macrocode}
%    \end{macro}
%
%    \begin{macro}{\enparen@Right}
%    \begin{macrocode}
\def\enparen@Right#1#2#3\@nil{%
  \ltx@IfUndefined{%
    enparen@symbol%
    \zref@extract{enparen#1}{enparen}%
    R%
  }{%
    \csname enparen@symbol0R\endcsname
  }{%
    \csname
      enparen@symbol%
      \zref@extract{enparen#1}{enparen}%
       R%
    \endcsname
  }%
  \zref@wrapper@mainaux{%
    \zref@setcurrent{enparen}{#2}%
    \zref@labelbyprops{enparen#1}{enparen}%
  }%
  \xdef\enparen@stack{#3}%
}
%    \end{macrocode}
%    \end{macro}
%
% \subsection{Options}
%
%    \begin{macrocode}
\SetupKeyvalOptions{%
  family=enparen,%
  prefix=enparen@,%
}
%    \end{macrocode}
%
%    \begin{macro}{\enparenSetup}
%    \begin{macrocode}
\ProtectedDef*{\enparenSetup}[0]{%
  \kvsetkeys{enparen}%
}
%    \end{macrocode}
%    \end{macro}
%
%    \begin{macrocode}
\DeclareBoolOption[true]{footnote}
\DeclareBoolOption[true]{caption}
\ProcessKeyvalOptions*
%    \end{macrocode}
%
% \subsection{Context settings}
%
%    \begin{macro}{\enparen@AtBegin}
%    \begin{macrocode}
\def\enparen@AtBegin{%
  \ifenparen@footnote
    \let\enparen@org@makefntext\@makefntext
    \long\def\@makefntext##1{%
      \enparen@org@makefntext{%
        \enparenBeginContext{footnote}%
        ##1%
        \enparenEndContext{footnote}%
      }%
    }%
  \fi
  \enparen@Disable{footnote}%
  \ifenparen@caption
    \let\enparen@org@makecaption\@makecaption
    \long\def\@makecaption##1##2{%
      \enparen@org@makecaption{##1}{%
        \enparenBeginContext{caption}%
        ##2%
        \enparenEndContext{caption}%
      }%
    }%
  \fi
  \enparen@Disable{caption}%
}
%    \end{macrocode}
%    \end{macro}
%
%    \begin{macro}{\enparen@Disable}
%    \begin{macrocode}
\def\enparen@Disable#1{%
  \DisableKeyvalOption[%
    action=warning,%
    package=enparen,%
  ]{enparen}{#1}%
}
%    \end{macrocode}
%    \end{macro}
%
%    \begin{macrocode}
\AtBeginDocument{\enparen@AtBegin}
%    \end{macrocode}
%
% \subsection{At end of document}
%
%    \begin{macro}{\enparen@AtEnd}
%    \begin{macrocode}
\def\enparen@AtEnd{%
  \enparenCheckEmptyStack
  \ifx\enparen@CtxStack\ltx@empty
  \else
    \PackageWarningNoLine{enparen}{%
      Context stack is not empty at end of document.\MessageBreak
      Current stack and contents of context stack:\MessageBreak
      [\enparen@ctx]:[\enparen@PrintStack]%
      \expandafter
      \enparen@PrintContextStack\enparen@CtxStack\relax\relax
    }%
  \fi
}
%    \end{macrocode}
%    \end{macro}
%    \begin{macro}{\enparen@PrintContextStack}
%    \begin{macrocode}
\def\enparen@PrintContextStack#1#2{%
  \ifx\relax#1\ltx@empty
  \else
    \MessageBreak
    [#1]:[\enparen@PrintStackAux#2\ltx@empty\ltx@empty]% hash-ok
    \expandafter\enparen@PrintContextStack
  \fi
}
%    \end{macrocode}
%    \end{macro}
%    \begin{macrocode}
\AtVeryEndDocument{\enparen@AtEnd}
%    \end{macrocode}
%
%    \begin{macrocode}
%</package>
%    \end{macrocode}
%
% \section{Installation}
%
% \subsection{Download}
%
% \paragraph{Package.} This package is available on
% CTAN\footnote{\url{ftp://ftp.ctan.org/tex-archive/}}:
% \begin{description}
% \item[\CTAN{macros/latex/contrib/oberdiek/enparen.dtx}] The source file.
% \item[\CTAN{macros/latex/contrib/oberdiek/enparen.pdf}] Documentation.
% \end{description}
%
%
% \paragraph{Bundle.} All the packages of the bundle `oberdiek'
% are also available in a TDS compliant ZIP archive. There
% the packages are already unpacked and the documentation files
% are generated. The files and directories obey the TDS standard.
% \begin{description}
% \item[\CTAN{install/macros/latex/contrib/oberdiek.tds.zip}]
% \end{description}
% \emph{TDS} refers to the standard ``A Directory Structure
% for \TeX\ Files'' (\CTAN{tds/tds.pdf}). Directories
% with \xfile{texmf} in their name are usually organized this way.
%
% \subsection{Bundle installation}
%
% \paragraph{Unpacking.} Unpack the \xfile{oberdiek.tds.zip} in the
% TDS tree (also known as \xfile{texmf} tree) of your choice.
% Example (linux):
% \begin{quote}
%   |unzip oberdiek.tds.zip -d ~/texmf|
% \end{quote}
%
% \paragraph{Script installation.}
% Check the directory \xfile{TDS:scripts/oberdiek/} for
% scripts that need further installation steps.
% Package \xpackage{attachfile2} comes with the Perl script
% \xfile{pdfatfi.pl} that should be installed in such a way
% that it can be called as \texttt{pdfatfi}.
% Example (linux):
% \begin{quote}
%   |chmod +x scripts/oberdiek/pdfatfi.pl|\\
%   |cp scripts/oberdiek/pdfatfi.pl /usr/local/bin/|
% \end{quote}
%
% \subsection{Package installation}
%
% \paragraph{Unpacking.} The \xfile{.dtx} file is a self-extracting
% \docstrip\ archive. The files are extracted by running the
% \xfile{.dtx} through \plainTeX:
% \begin{quote}
%   \verb|tex enparen.dtx|
% \end{quote}
%
% \paragraph{TDS.} Now the different files must be moved into
% the different directories in your installation TDS tree
% (also known as \xfile{texmf} tree):
% \begin{quote}
% \def\t{^^A
% \begin{tabular}{@{}>{\ttfamily}l@{ $\rightarrow$ }>{\ttfamily}l@{}}
%   enparen.sty & tex/latex/oberdiek/enparen.sty\\
%   enparen.pdf & doc/latex/oberdiek/enparen.pdf\\
%   enparen.dtx & source/latex/oberdiek/enparen.dtx\\
% \end{tabular}^^A
% }^^A
% \sbox0{\t}^^A
% \ifdim\wd0>\linewidth
%   \begingroup
%     \advance\linewidth by\leftmargin
%     \advance\linewidth by\rightmargin
%   \edef\x{\endgroup
%     \def\noexpand\lw{\the\linewidth}^^A
%   }\x
%   \def\lwbox{^^A
%     \leavevmode
%     \hbox to \linewidth{^^A
%       \kern-\leftmargin\relax
%       \hss
%       \usebox0
%       \hss
%       \kern-\rightmargin\relax
%     }^^A
%   }^^A
%   \ifdim\wd0>\lw
%     \sbox0{\small\t}^^A
%     \ifdim\wd0>\linewidth
%       \ifdim\wd0>\lw
%         \sbox0{\footnotesize\t}^^A
%         \ifdim\wd0>\linewidth
%           \ifdim\wd0>\lw
%             \sbox0{\scriptsize\t}^^A
%             \ifdim\wd0>\linewidth
%               \ifdim\wd0>\lw
%                 \sbox0{\tiny\t}^^A
%                 \ifdim\wd0>\linewidth
%                   \lwbox
%                 \else
%                   \usebox0
%                 \fi
%               \else
%                 \lwbox
%               \fi
%             \else
%               \usebox0
%             \fi
%           \else
%             \lwbox
%           \fi
%         \else
%           \usebox0
%         \fi
%       \else
%         \lwbox
%       \fi
%     \else
%       \usebox0
%     \fi
%   \else
%     \lwbox
%   \fi
% \else
%   \usebox0
% \fi
% \end{quote}
% If you have a \xfile{docstrip.cfg} that configures and enables \docstrip's
% TDS installing feature, then some files can already be in the right
% place, see the documentation of \docstrip.
%
% \subsection{Refresh file name databases}
%
% If your \TeX~distribution
% (\teTeX, \mikTeX, \dots) relies on file name databases, you must refresh
% these. For example, \teTeX\ users run \verb|texhash| or
% \verb|mktexlsr|.
%
% \subsection{Some details for the interested}
%
% \paragraph{Attached source.}
%
% The PDF documentation on CTAN also includes the
% \xfile{.dtx} source file. It can be extracted by
% AcrobatReader 6 or higher. Another option is \textsf{pdftk},
% e.g. unpack the file into the current directory:
% \begin{quote}
%   \verb|pdftk enparen.pdf unpack_files output .|
% \end{quote}
%
% \paragraph{Unpacking with \LaTeX.}
% The \xfile{.dtx} chooses its action depending on the format:
% \begin{description}
% \item[\plainTeX:] Run \docstrip\ and extract the files.
% \item[\LaTeX:] Generate the documentation.
% \end{description}
% If you insist on using \LaTeX\ for \docstrip\ (really,
% \docstrip\ does not need \LaTeX), then inform the autodetect routine
% about your intention:
% \begin{quote}
%   \verb|latex \let\install=y% \iffalse meta-comment
%
% File: enparen.dtx
% Version: 2012/01/07 v1.0
% Info: Parentheses nesting
%
% Copyright (C) 2012 by
%    Heiko Oberdiek <heiko.oberdiek at googlemail.com>
%
% This work may be distributed and/or modified under the
% conditions of the LaTeX Project Public License, either
% version 1.3c of this license or (at your option) any later
% version. This version of this license is in
%    http://www.latex-project.org/lppl/lppl-1-3c.txt
% and the latest version of this license is in
%    http://www.latex-project.org/lppl.txt
% and version 1.3 or later is part of all distributions of
% LaTeX version 2005/12/01 or later.
%
% This work has the LPPL maintenance status "maintained".
%
% This Current Maintainer of this work is Heiko Oberdiek.
%
% This work consists of the main source file enparen.dtx
% and the derived files
%    enparen.sty, enparen.pdf, enparen.ins, enparen.drv.
%
% Distribution:
%    CTAN:macros/latex/contrib/oberdiek/enparen.dtx
%    CTAN:macros/latex/contrib/oberdiek/enparen.pdf
%
% Unpacking:
%    (a) If enparen.ins is present:
%           tex enparen.ins
%    (b) Without enparen.ins:
%           tex enparen.dtx
%    (c) If you insist on using LaTeX
%           latex \let\install=y% \iffalse meta-comment
%
% File: enparen.dtx
% Version: 2012/01/07 v1.0
% Info: Parentheses nesting
%
% Copyright (C) 2012 by
%    Heiko Oberdiek <heiko.oberdiek at googlemail.com>
%
% This work may be distributed and/or modified under the
% conditions of the LaTeX Project Public License, either
% version 1.3c of this license or (at your option) any later
% version. This version of this license is in
%    http://www.latex-project.org/lppl/lppl-1-3c.txt
% and the latest version of this license is in
%    http://www.latex-project.org/lppl.txt
% and version 1.3 or later is part of all distributions of
% LaTeX version 2005/12/01 or later.
%
% This work has the LPPL maintenance status "maintained".
%
% This Current Maintainer of this work is Heiko Oberdiek.
%
% This work consists of the main source file enparen.dtx
% and the derived files
%    enparen.sty, enparen.pdf, enparen.ins, enparen.drv.
%
% Distribution:
%    CTAN:macros/latex/contrib/oberdiek/enparen.dtx
%    CTAN:macros/latex/contrib/oberdiek/enparen.pdf
%
% Unpacking:
%    (a) If enparen.ins is present:
%           tex enparen.ins
%    (b) Without enparen.ins:
%           tex enparen.dtx
%    (c) If you insist on using LaTeX
%           latex \let\install=y\input{enparen.dtx}
%        (quote the arguments according to the demands of your shell)
%
% Documentation:
%    (a) If enparen.drv is present:
%           latex enparen.drv
%    (b) Without enparen.drv:
%           latex enparen.dtx; ...
%    The class ltxdoc loads the configuration file ltxdoc.cfg
%    if available. Here you can specify further options, e.g.
%    use A4 as paper format:
%       \PassOptionsToClass{a4paper}{article}
%
%    Programm calls to get the documentation (example):
%       pdflatex enparen.dtx
%       makeindex -s gind.ist enparen.idx
%       pdflatex enparen.dtx
%       makeindex -s gind.ist enparen.idx
%       pdflatex enparen.dtx
%
% Installation:
%    TDS:tex/latex/oberdiek/enparen.sty
%    TDS:doc/latex/oberdiek/enparen.pdf
%    TDS:source/latex/oberdiek/enparen.dtx
%
%<*ignore>
\begingroup
  \catcode123=1 %
  \catcode125=2 %
  \def\x{LaTeX2e}%
\expandafter\endgroup
\ifcase 0\ifx\install y1\fi\expandafter
         \ifx\csname processbatchFile\endcsname\relax\else1\fi
         \ifx\fmtname\x\else 1\fi\relax
\else\csname fi\endcsname
%</ignore>
%<*install>
\input docstrip.tex
\Msg{************************************************************************}
\Msg{* Installation}
\Msg{* Package: enparen 2012/01/07 v1.0 Parentheses nesting (HO)}
\Msg{************************************************************************}

\keepsilent
\askforoverwritefalse

\let\MetaPrefix\relax
\preamble

This is a generated file.

Project: enparen
Version: 2012/01/07 v1.0

Copyright (C) 2012 by
   Heiko Oberdiek <heiko.oberdiek at googlemail.com>

This work may be distributed and/or modified under the
conditions of the LaTeX Project Public License, either
version 1.3c of this license or (at your option) any later
version. This version of this license is in
   http://www.latex-project.org/lppl/lppl-1-3c.txt
and the latest version of this license is in
   http://www.latex-project.org/lppl.txt
and version 1.3 or later is part of all distributions of
LaTeX version 2005/12/01 or later.

This work has the LPPL maintenance status "maintained".

This Current Maintainer of this work is Heiko Oberdiek.

This work consists of the main source file enparen.dtx
and the derived files
   enparen.sty, enparen.pdf, enparen.ins, enparen.drv.

\endpreamble
\let\MetaPrefix\DoubleperCent

\generate{%
  \file{enparen.ins}{\from{enparen.dtx}{install}}%
  \file{enparen.drv}{\from{enparen.dtx}{driver}}%
  \usedir{tex/latex/oberdiek}%
  \file{enparen.sty}{\from{enparen.dtx}{package}}%
}

\catcode32=13\relax% active space
\let =\space%
\Msg{************************************************************************}
\Msg{*}
\Msg{* To finish the installation you have to move the following}
\Msg{* file into a directory searched by TeX:}
\Msg{*}
\Msg{*     enparen.sty}
\Msg{*}
\Msg{* To produce the documentation run the file `enparen.drv'}
\Msg{* through LaTeX.}
\Msg{*}
\Msg{* Happy TeXing!}
\Msg{*}
\Msg{************************************************************************}

\endbatchfile
%</install>
%<*ignore>
\fi
%</ignore>
%<*driver>
\NeedsTeXFormat{LaTeX2e}
\ProvidesFile{enparen.drv}%
  [2012/01/07 v1.0 Parentheses nesting (HO)]%
\documentclass{ltxdoc}
\usepackage{holtxdoc}[2011/11/22]
\usepackage{enparen}[2012/01/07]
\begin{document}
  \DocInput{enparen.dtx}%
\end{document}
%</driver>
% \fi
%
% \CheckSum{339}
%
% \CharacterTable
%  {Upper-case    \A\B\C\D\E\F\G\H\I\J\K\L\M\N\O\P\Q\R\S\T\U\V\W\X\Y\Z
%   Lower-case    \a\b\c\d\e\f\g\h\i\j\k\l\m\n\o\p\q\r\s\t\u\v\w\x\y\z
%   Digits        \0\1\2\3\4\5\6\7\8\9
%   Exclamation   \!     Double quote  \"     Hash (number) \#
%   Dollar        \$     Percent       \%     Ampersand     \&
%   Acute accent  \'     Left paren    \(     Right paren   \)
%   Asterisk      \*     Plus          \+     Comma         \,
%   Minus         \-     Point         \.     Solidus       \/
%   Colon         \:     Semicolon     \;     Less than     \<
%   Equals        \=     Greater than  \>     Question mark \?
%   Commercial at \@     Left bracket  \[     Backslash     \\
%   Right bracket \]     Circumflex    \^     Underscore    \_
%   Grave accent  \`     Left brace    \{     Vertical bar  \|
%   Right brace   \}     Tilde         \~}
%
% \GetFileInfo{enparen.drv}
%
% \title{The \xpackage{enparen} package}
% \date{2012/01/07 v1.0}
% \author{Heiko Oberdiek\\\xemail{heiko.oberdiek at googlemail.com}}
%
% \maketitle
%
% \begin{abstract}
% The package defines macros to set parentheses that automatically
% change the symbols from inner to outer fences.
% \end{abstract}
%
% \tableofcontents
%
% \section{Documentation}
%
% The \hologo{LaTeX} package provides macros for automatically changed
% parentheses symbols depending on the fence order.
% The innermost parentheses are always using the same
% parentheses. The symbols changes for the outer fences.
%
% Example:
% \begin{quote}
%   \enparen{a \enparen{b} \enparen{c \enparen{d}} \enparen{e}}
% \end{quote}
% is generated by
%\begin{quote}
%\begin{verbatim}
%\documentclass{article}
%\usepackage{enparen}
%\begin{document}
%  \enparen{a \enparen{b} \enparen{c \enparen{d}} \enparen{e}}
%\end{document}
%\end{verbatim}
%\end{quote}
%
% The package is the result of a newsgroup thread. Dan gives
% the following specification \cite{dan}:
% \begin{quote}
% ``On the other hand, the rules for fences are usually the reverse:
% innermost fences are always |()|, next outer are |[]|, etc. This means
% the opening fence has to wait until all the fences between it and
% the matching close have been detected before it can decide whether
% to be |(| or |[| or |{|.''
% \end{quote}
%
% The fence level counting starts from innermost parentheses with one.
% For the next outer fences the level is increased by one.
% The example above with level indexes:
% \begin{quote}
%   \def\1{\ensuremath{_{1}}}
%   \def\2{\ensuremath{_{2}}}
%   \def\3{\ensuremath{_{3}}}
%   \enparen{\3a \enparen{\1b}\1 \enparen{\2c \enparen{\1d}\1}\2 \enparen{e}}\3
% \end{quote}
% The correct level is only known at the closing symbol. Therefore the
% correct value is remembered in the main \xext{aux} file and used
% in the second \hologo{LaTeX} run.
%
% \subsection{User macros}
%
% \begin{declcs}{enparen} \M{text}
% \end{declcs}
% The macro \cs{enparen} puts its argument \meta{text} in parentheses.
% If the macro is nested, the used fence symbols change for the
% outer fences.
%
% \begin{declcs}{enparenLeft}\\
%   \cs{enparenRight}
% \end{declcs}
% Instead of |\enparen{text}| the left and right symbol can be
% used separately, but in pairs:
% \begin{quote}
%   |\enparenLeft text\enparenRight|
% \end{quote}
% \cs{enparenLeft} and \cs{enparenRight} may be used at different
% group levels, but they must be properly nested.
%
% \begin{declcs}{enparenSetSymbols} \M{level} \M{opening symbol} \M{closing symbol}
% \end{declcs}
% Macro \cs{enparenSetSymbols} configures the \meta{opening symbol} and
% \meta{closing symbol} for the \meta{level}. The \meta{level}
% is a number and the counting starts with one. Level zero is used,
% if the correct level is not known (e.g. in the first \hologo{LaTeX} run).
% The package defines the following sets:
%\begin{quote}
%\begin{verbatim}
%\enparenSetSymbols{0}{(}{)}
%\enparenSetSymbols{1}{(}{)}
%\enparenSetSymbols{2}{[}{]}
%\enparenSetSymbols{3}{\{}{\}}
%\end{verbatim}
%\end{quote}
% Example for changing the third and adding a fourth level:
%\begin{quote}
%\begin{verbatim}
%\enparenSetSymbols{3}{\ensuremath{\langle}}{\ensuremath{\rangle}}
%\enparenSetSymbols{4}{\{}{\}}
%\end{verbatim}
%\end{quote}
%
% \begin{declcs}{enparenUnsetSymbols} \M{level}
% \end{declcs}
% The symbols for level \meta{level} are removed.
% Example scenario: Only two nesting levels must be used,
% the package defines more, then the third level can be
% disabled by |\enparenUnsetSymbols{3}| and the user
% gets warnings if parentheses at level 3 are needed.
%
% \subsection{Contexts}
%
% \begin{declcs}{enparenBeginContext} \M{name}\\
%   \cs{enparenEndContext} \M{name}
% \end{declcs}
% If the current text is interrupted by footnotes, floats
% with captions, then the parentheses inside the text of footnotes,
% captions, \dots should restart from scratch. This can be achieved
% by embedding the text inside macros \cs{enparenBeginContext}
% and \cs{enparenEndContext}. These macros must be properly nested.
% The \meta{name} for the begin and end macro must be the same.
% It is a help for debugging problems, because the warning messages
% show the context name. But it is not necessary that the
% begin\slash end pairs have different names. Example:
%\begin{quote}
%\begin{verbatim}
%\enparenLeft text before table ...
%\begin{table}
%  \caption{Table caption}
%  \enparenBegin{Context}{table}
%  Other text \enparen{foobar}.
%  \enparenEnd{Context}{table}
%\end{table}
%text after table ...
%\enparenRight
%\end{verbatim}
%\end{quote}
% The parentheses inside the table environment and context `table'
% are not nested inside other parentheses: \enparen{foobar}.
% In case of captions and footnotes the contexts are automatically
% added, see next section about options.
%
% \subsection{Options}
%
% \begin{declcs}{enparenSetup} \M{key value list}
% \end{declcs}
% Some options (currently all) can also be set after the
% package is loaded. They can be set in the argument
% \meta{key value list} of \cs{enparenSetup}. Options are disabled
% after they are used the last time. Currently all options are boolean
% options and are disabled in |\begin{document}|.
%
% \begin{description}
% \item[\xoption{caption}:] The caption text is put in a
%   context |caption|.
% \item[\xoption{footnote}:] The footnote text is put in a
%   context |footnote|.
% \end{description}
%
% Example for disabling the two options at different places:
%\begin{quote}
%\begin{verbatim}
%\usepackage[caption=false]{enparen}
%\enparenSetup{footnote=false}
%\end{verbatim}
%\end{quote}
%
% \subsection{Notes}
%
% \begin{description}
% \item[Implicite kerning:]
%   Unexpandable stuff might affect the implicite kerning.
%   The package cannot avoid this, because it need to define
%   and redefine macros at the occurence of each symbol.
%   This is done before the opening and after the closing
%   symbol, thus that the implicite kerning inside
%   is not affected.
% \end{description}
%
% \StopEventually{
% }
%
% \section{Implementation}
%
%    \begin{macrocode}
%<*package>
\NeedsTeXFormat{LaTeX2e}
\ProvidesPackage{enparen}
  [2012/01/07 v1.0 Parentheses nesting (HO)]%
%    \end{macrocode}
%
% \subsection{Resources}
%
%    \begin{macrocode}
\RequirePackage{ltxcmds}[2011/11/09]
%    \end{macrocode}
%
%    \begin{macrocode}
\ltx@IfUndefined{numexpr}{%
  \PackageError{enparen}{%
    Missing e-TeX's \ltx@backslashchar numexpr.\MessageBreak
    The package will continue with emergency definitions%
  }\@ehc
  \def\enparenLeft{(}%
  \def\enparenRight{)}%
  \long\def\enparen#1{\enparenLeft#1\enparenRight}%
  \let\enparenSetup\ltx@gobble
  \let\enparenSetSymbols\ltx@gobblethree
  \let\enparenUnsetSymbols\ltx@gobble
  \endinput
}{}
%    \end{macrocode}
%
%    \begin{macrocode}
\RequirePackage{protecteddef}[2011/01/31]
\RequirePackage{atveryend}[2011/06/30]
\RequirePackage{uniquecounter}[2011/01/30]
\RequirePackage{zref-base}[2011/03/18]
\RequirePackage{kvoptions}[2011/06/30]
\RequirePackage{kvsetkeys}[2011/10/18]
%    \end{macrocode}
%
%    \begin{macro}{\zref@wrapper@mainaux}
%    \begin{macrocode}
\providecommand{\zref@wrapper@mainaux}[1]{%
  \ifx\@auxout\@mainaux
    #1%
  \else
    \begingroup
      \let\@auxout\@mainaux
      #1%
    \endgroup
  \fi
}
%    \end{macrocode}
%    \end{macro}
%
%    \begin{macrocode}
\UniqueCounterNew{enparen}
%    \end{macrocode}
%
%    \begin{macrocode}
\zref@newprop{enparen}[0]{}
%    \end{macrocode}
%
% \subsection{Contexts}
%
%    \begin{macro}{\enparenContextDefault}
%    \begin{macrocode}
\def\enparenContextDefault{main}
%    \end{macrocode}
%    \end{macro}
%
%    \begin{macro}{\enparen@ctx}
%    \begin{macrocode}
\let\enparen@ctx\ltx@empty
%    \end{macrocode}
%    \end{macro}
%
%    \begin{macro}{\enparen@stack}
%    \begin{macrocode}
\let\enparen@stack\ltx@empty
%    \end{macrocode}
%    \end{macro}
%
% \subsubsection{Stack for contexts}
%
%    \begin{macro}{\enparen@CtxStack}
%    \begin{macrocode}
\global\let\enparen@CtxStack\ltx@empty
%    \end{macrocode}
%    \end{macro}
%
%    \begin{macro}{\enparen@CtxStackPush}
%    \begin{macrocode}
\def\enparen@CtxStackPush#1{%
  \xdef\enparen@CtxStack{%
    {\enparen@ctx}{\enparen@stack}%
    \enparen@CtxStack
  }%
  \xdef\enparen@ctx{#1}%
  \global\let\enparen@stack\ltx@empty
}
%    \end{macrocode}
%    \end{macro}
%
%    \begin{macro}{\enparen@CtxStackPop}
%    \begin{macrocode}
\def\enparen@CtxStackPop{%
  \ifx\enparen@CtxStack\ltx@empty
    \PackageWarning{enparen}{%
      Pop request for empty context stack%
    }%
    \global\let\enparen@ctx\enparenContextDefault
    \global\let\enparen@stack\ltx@empty
  \else
    \xdef\enparen@ctx{%
      \expandafter\ltx@car\enparen@CtxStack\@nil
    }%
    \xdef\enparen@stack{%
      \expandafter\ltx@carsecond\enparen@CtxStack\@nil
    }%
    \xdef\enparen@CtxStack{%
      \expandafter\ltx@cdrtwo\enparen@CtxStack\@nil
    }%
  \fi
}
%    \end{macrocode}
%    \end{macro}
%
% \subsubsection{Context user macros}
%
%    \begin{macro}{\enparenBeginContext}
%    \begin{macrocode}
\ProtectedDef*{\enparenBeginContext}[1]{%
  \enparen@CtxStackPush{#1}%
}
%    \end{macrocode}
%    \end{macro}
%    \begin{macro}{\enparenEndContext}
%    \begin{macrocode}
\ProtectedDef*{\enparenEndContext}[1]{%
  \edef\enparen@temp{#1}%
  \ifx\enparen@temp\enparen@ctx
  \else
    \PackageWarning{enparen}{%
      Context mismatch in end request.\MessageBreak
      `#1' should be ended, but current context\MessageBreak
      is `\enparen@ctx'%
    }%
  \fi
  \enparenCheckEmptyStack
  \enparen@CtxStackPop
}
%    \end{macrocode}
%    \end{macro}
%
%    \begin{macro}{\enparenCheckEmptyStack}
%    \begin{macrocode}
\ProtectedDef*{\enparenCheckEmptyStack}[0]{%
  \ifx\enparen@stack\ltx@empty
  \else
    \PackageWarning{enparen}{%
      Ending non-empty stack `\enparen@ctx':\MessageBreak
      \enparen@PrintStack\MessageBreak
    }%
  \fi
}
%    \end{macrocode}
%    \end{macro}
%
%    \begin{macro}{\enparen@PrintStack}
%    \begin{macrocode}
\def\enparen@PrintStack{%
  \expandafter\enparen@PrintStackAux
  \enparen@stack\ltx@empty\ltx@empty
}
%    \end{macrocode}
%    \end{macro}
%    \begin{macro}{\enparen@PrintStackAux}
%    \begin{macrocode}
\def\enparen@PrintStackAux#1#2{%
  \ifx\ltx@empty#1%
  \else
    {#1:#2}%
    \expandafter\enparen@PrintStackAux
  \fi
}
%    \end{macrocode}
%    \end{macro}
%
% \subsection{Symbols}
%
%    \begin{macro}{\enparenSetSymbols}
%    \begin{macrocode}
\ProtectedDef*{\enparenSetSymbols}[3]{%
  \expandafter
  \def\csname enparen@symbol\the\numexpr#1L\endcsname{#2}%
  \expandafter
  \def\csname enparen@symbol\the\numexpr#1R\endcsname{#3}%
}
%    \end{macrocode}
%    \end{macro}
%    \begin{macro}{\enparenUnsetSymbols}
%    \begin{macrocode}
\ProtectedDef*{\enparenUnsetSymbols}[1]{%
  \expandafter
  \let\csname enparen@symbol\the\numexpr#1L\endcsname\ltx@undefined
  \expandafter
  \let\csname enparen@symbol\the\numexpr#1R\endcsname\ltx@undefined
}
%    \end{macrocode}
%    \end{macro}
%
%    \begin{macrocode}
\enparenSetSymbols{0}{(}{)}
\enparenSetSymbols{1}{(}{)}
\enparenSetSymbols{2}{[}{]}
\enparenSetSymbols{3}{\{}{\}}
\enparenSetSymbols{4}{\ensuremath{\langle}}{\ensuremath{\rangle}}
%    \end{macrocode}
%
% \subsection{Main user macros}
%
%    \begin{macro}{\enparen}
%    \begin{macrocode}
\ProtectedDef{\enparen}[1]{%
  \enparenLeft#1\enparenRight
}
%    \end{macrocode}
%    \end{macro}
%
%    \begin{macro}{\enparenLeft}
%    \begin{macrocode}
\ProtectedDef*{\enparenLeft}[0]{%
  \UniqueCounterCall{enparen}\enparen@Left
}
%    \end{macrocode}
%    \end{macro}
%    \begin{macro}{\enparen@Left}
%    \begin{macrocode}
\def\enparen@Left#1{%
  \xdef\enparen@stack{%
    {#1}{1}%
    \expandafter\enparen@Inc\expandafter2\expandafter!%
    \enparen@stack\ltx@empty\ltx@empty
  }%
  \edef\enparen@tmp{\zref@extract{enparen#1}{enparen}}%
  \ltx@IfUndefined{enparen@symbol\enparen@tmp L}{%
    \PackageWarning{enparen}{%
      Undefined symbols for level \enparen@tmp
    }%
    \csname enparen@symbol0L\endcsname
  }{%
    \csname enparen@symbol\enparen@tmp L\endcsname
  }%
}
%    \end{macrocode}
%    \end{macro}
%    \begin{macro}{\enparen@Inc}
%    \begin{macrocode}
\def\enparen@Inc#1!#2#3{%
  \ifx\ltx@empty#2%
  \else
    \ifnum#3<#1 %
      {#2}{#1}%
      \expandafter\enparen@Inc
      \the\numexpr#1+1\expandafter\expandafter\expandafter!%
    \else
      {#2}{#3}%
    \fi
  \fi
}
%    \end{macrocode}
%    \end{macro}
%
%    \begin{macro}{\enparenRight}
%    \begin{macrocode}
\ProtectedDef*{\enparenRight}[0]{%
  \ifx\enparen@stack\ltx@empty
    \PackageWarning{enparen}{%
      Missing left symbol for right symbol%
    }%
    \csname enparen@symbol0R\endcsname
  \else
    \expandafter\enparen@Right\enparen@stack\@nil
  \fi
}
%    \end{macrocode}
%    \end{macro}
%
%    \begin{macro}{\enparen@Right}
%    \begin{macrocode}
\def\enparen@Right#1#2#3\@nil{%
  \ltx@IfUndefined{%
    enparen@symbol%
    \zref@extract{enparen#1}{enparen}%
    R%
  }{%
    \csname enparen@symbol0R\endcsname
  }{%
    \csname
      enparen@symbol%
      \zref@extract{enparen#1}{enparen}%
       R%
    \endcsname
  }%
  \zref@wrapper@mainaux{%
    \zref@setcurrent{enparen}{#2}%
    \zref@labelbyprops{enparen#1}{enparen}%
  }%
  \xdef\enparen@stack{#3}%
}
%    \end{macrocode}
%    \end{macro}
%
% \subsection{Options}
%
%    \begin{macrocode}
\SetupKeyvalOptions{%
  family=enparen,%
  prefix=enparen@,%
}
%    \end{macrocode}
%
%    \begin{macro}{\enparenSetup}
%    \begin{macrocode}
\ProtectedDef*{\enparenSetup}[0]{%
  \kvsetkeys{enparen}%
}
%    \end{macrocode}
%    \end{macro}
%
%    \begin{macrocode}
\DeclareBoolOption[true]{footnote}
\DeclareBoolOption[true]{caption}
\ProcessKeyvalOptions*
%    \end{macrocode}
%
% \subsection{Context settings}
%
%    \begin{macro}{\enparen@AtBegin}
%    \begin{macrocode}
\def\enparen@AtBegin{%
  \ifenparen@footnote
    \let\enparen@org@makefntext\@makefntext
    \long\def\@makefntext##1{%
      \enparen@org@makefntext{%
        \enparenBeginContext{footnote}%
        ##1%
        \enparenEndContext{footnote}%
      }%
    }%
  \fi
  \enparen@Disable{footnote}%
  \ifenparen@caption
    \let\enparen@org@makecaption\@makecaption
    \long\def\@makecaption##1##2{%
      \enparen@org@makecaption{##1}{%
        \enparenBeginContext{caption}%
        ##2%
        \enparenEndContext{caption}%
      }%
    }%
  \fi
  \enparen@Disable{caption}%
}
%    \end{macrocode}
%    \end{macro}
%
%    \begin{macro}{\enparen@Disable}
%    \begin{macrocode}
\def\enparen@Disable#1{%
  \DisableKeyvalOption[%
    action=warning,%
    package=enparen,%
  ]{enparen}{#1}%
}
%    \end{macrocode}
%    \end{macro}
%
%    \begin{macrocode}
\AtBeginDocument{\enparen@AtBegin}
%    \end{macrocode}
%
% \subsection{At end of document}
%
%    \begin{macro}{\enparen@AtEnd}
%    \begin{macrocode}
\def\enparen@AtEnd{%
  \enparenCheckEmptyStack
  \ifx\enparen@CtxStack\ltx@empty
  \else
    \PackageWarningNoLine{enparen}{%
      Context stack is not empty at end of document.\MessageBreak
      Current stack and contents of context stack:\MessageBreak
      [\enparen@ctx]:[\enparen@PrintStack]%
      \expandafter
      \enparen@PrintContextStack\enparen@CtxStack\relax\relax
    }%
  \fi
}
%    \end{macrocode}
%    \end{macro}
%    \begin{macro}{\enparen@PrintContextStack}
%    \begin{macrocode}
\def\enparen@PrintContextStack#1#2{%
  \ifx\relax#1\ltx@empty
  \else
    \MessageBreak
    [#1]:[\enparen@PrintStackAux#2\ltx@empty\ltx@empty]% hash-ok
    \expandafter\enparen@PrintContextStack
  \fi
}
%    \end{macrocode}
%    \end{macro}
%    \begin{macrocode}
\AtVeryEndDocument{\enparen@AtEnd}
%    \end{macrocode}
%
%    \begin{macrocode}
%</package>
%    \end{macrocode}
%
% \section{Installation}
%
% \subsection{Download}
%
% \paragraph{Package.} This package is available on
% CTAN\footnote{\url{ftp://ftp.ctan.org/tex-archive/}}:
% \begin{description}
% \item[\CTAN{macros/latex/contrib/oberdiek/enparen.dtx}] The source file.
% \item[\CTAN{macros/latex/contrib/oberdiek/enparen.pdf}] Documentation.
% \end{description}
%
%
% \paragraph{Bundle.} All the packages of the bundle `oberdiek'
% are also available in a TDS compliant ZIP archive. There
% the packages are already unpacked and the documentation files
% are generated. The files and directories obey the TDS standard.
% \begin{description}
% \item[\CTAN{install/macros/latex/contrib/oberdiek.tds.zip}]
% \end{description}
% \emph{TDS} refers to the standard ``A Directory Structure
% for \TeX\ Files'' (\CTAN{tds/tds.pdf}). Directories
% with \xfile{texmf} in their name are usually organized this way.
%
% \subsection{Bundle installation}
%
% \paragraph{Unpacking.} Unpack the \xfile{oberdiek.tds.zip} in the
% TDS tree (also known as \xfile{texmf} tree) of your choice.
% Example (linux):
% \begin{quote}
%   |unzip oberdiek.tds.zip -d ~/texmf|
% \end{quote}
%
% \paragraph{Script installation.}
% Check the directory \xfile{TDS:scripts/oberdiek/} for
% scripts that need further installation steps.
% Package \xpackage{attachfile2} comes with the Perl script
% \xfile{pdfatfi.pl} that should be installed in such a way
% that it can be called as \texttt{pdfatfi}.
% Example (linux):
% \begin{quote}
%   |chmod +x scripts/oberdiek/pdfatfi.pl|\\
%   |cp scripts/oberdiek/pdfatfi.pl /usr/local/bin/|
% \end{quote}
%
% \subsection{Package installation}
%
% \paragraph{Unpacking.} The \xfile{.dtx} file is a self-extracting
% \docstrip\ archive. The files are extracted by running the
% \xfile{.dtx} through \plainTeX:
% \begin{quote}
%   \verb|tex enparen.dtx|
% \end{quote}
%
% \paragraph{TDS.} Now the different files must be moved into
% the different directories in your installation TDS tree
% (also known as \xfile{texmf} tree):
% \begin{quote}
% \def\t{^^A
% \begin{tabular}{@{}>{\ttfamily}l@{ $\rightarrow$ }>{\ttfamily}l@{}}
%   enparen.sty & tex/latex/oberdiek/enparen.sty\\
%   enparen.pdf & doc/latex/oberdiek/enparen.pdf\\
%   enparen.dtx & source/latex/oberdiek/enparen.dtx\\
% \end{tabular}^^A
% }^^A
% \sbox0{\t}^^A
% \ifdim\wd0>\linewidth
%   \begingroup
%     \advance\linewidth by\leftmargin
%     \advance\linewidth by\rightmargin
%   \edef\x{\endgroup
%     \def\noexpand\lw{\the\linewidth}^^A
%   }\x
%   \def\lwbox{^^A
%     \leavevmode
%     \hbox to \linewidth{^^A
%       \kern-\leftmargin\relax
%       \hss
%       \usebox0
%       \hss
%       \kern-\rightmargin\relax
%     }^^A
%   }^^A
%   \ifdim\wd0>\lw
%     \sbox0{\small\t}^^A
%     \ifdim\wd0>\linewidth
%       \ifdim\wd0>\lw
%         \sbox0{\footnotesize\t}^^A
%         \ifdim\wd0>\linewidth
%           \ifdim\wd0>\lw
%             \sbox0{\scriptsize\t}^^A
%             \ifdim\wd0>\linewidth
%               \ifdim\wd0>\lw
%                 \sbox0{\tiny\t}^^A
%                 \ifdim\wd0>\linewidth
%                   \lwbox
%                 \else
%                   \usebox0
%                 \fi
%               \else
%                 \lwbox
%               \fi
%             \else
%               \usebox0
%             \fi
%           \else
%             \lwbox
%           \fi
%         \else
%           \usebox0
%         \fi
%       \else
%         \lwbox
%       \fi
%     \else
%       \usebox0
%     \fi
%   \else
%     \lwbox
%   \fi
% \else
%   \usebox0
% \fi
% \end{quote}
% If you have a \xfile{docstrip.cfg} that configures and enables \docstrip's
% TDS installing feature, then some files can already be in the right
% place, see the documentation of \docstrip.
%
% \subsection{Refresh file name databases}
%
% If your \TeX~distribution
% (\teTeX, \mikTeX, \dots) relies on file name databases, you must refresh
% these. For example, \teTeX\ users run \verb|texhash| or
% \verb|mktexlsr|.
%
% \subsection{Some details for the interested}
%
% \paragraph{Attached source.}
%
% The PDF documentation on CTAN also includes the
% \xfile{.dtx} source file. It can be extracted by
% AcrobatReader 6 or higher. Another option is \textsf{pdftk},
% e.g. unpack the file into the current directory:
% \begin{quote}
%   \verb|pdftk enparen.pdf unpack_files output .|
% \end{quote}
%
% \paragraph{Unpacking with \LaTeX.}
% The \xfile{.dtx} chooses its action depending on the format:
% \begin{description}
% \item[\plainTeX:] Run \docstrip\ and extract the files.
% \item[\LaTeX:] Generate the documentation.
% \end{description}
% If you insist on using \LaTeX\ for \docstrip\ (really,
% \docstrip\ does not need \LaTeX), then inform the autodetect routine
% about your intention:
% \begin{quote}
%   \verb|latex \let\install=y\input{enparen.dtx}|
% \end{quote}
% Do not forget to quote the argument according to the demands
% of your shell.
%
% \paragraph{Generating the documentation.}
% You can use both the \xfile{.dtx} or the \xfile{.drv} to generate
% the documentation. The process can be configured by the
% configuration file \xfile{ltxdoc.cfg}. For instance, put this
% line into this file, if you want to have A4 as paper format:
% \begin{quote}
%   \verb|\PassOptionsToClass{a4paper}{article}|
% \end{quote}
% An example follows how to generate the
% documentation with pdf\LaTeX:
% \begin{quote}
%\begin{verbatim}
%pdflatex enparen.dtx
%makeindex -s gind.ist enparen.idx
%pdflatex enparen.dtx
%makeindex -s gind.ist enparen.idx
%pdflatex enparen.dtx
%\end{verbatim}
% \end{quote}
%
% \begin{thebibliography}{9}
% \raggedright
% \bibitem{dan}
%   Dan Luecking: \textit{Re: bracket order};
%   newsgroup \xnewsgroup{comp.text.tex};
%   message id \texttt{\small$\langle$9b07c9c8-ff92-4cbf-b3a9-84efecfeb506@l24g2000yqm.googlegroups.com$\rangle$}
%   2012-01-05.
%   \url{http://groups.google.com/group/comp.text.tex/msg/8774519da31c2352}
% \end{thebibliography}
%
% \begin{History}
%   \begin{Version}{2012/01/07 v1.0}
%   \item
%     First version.
%   \end{Version}
% \end{History}
%
% \PrintIndex
%
% \Finale
\endinput

%        (quote the arguments according to the demands of your shell)
%
% Documentation:
%    (a) If enparen.drv is present:
%           latex enparen.drv
%    (b) Without enparen.drv:
%           latex enparen.dtx; ...
%    The class ltxdoc loads the configuration file ltxdoc.cfg
%    if available. Here you can specify further options, e.g.
%    use A4 as paper format:
%       \PassOptionsToClass{a4paper}{article}
%
%    Programm calls to get the documentation (example):
%       pdflatex enparen.dtx
%       makeindex -s gind.ist enparen.idx
%       pdflatex enparen.dtx
%       makeindex -s gind.ist enparen.idx
%       pdflatex enparen.dtx
%
% Installation:
%    TDS:tex/latex/oberdiek/enparen.sty
%    TDS:doc/latex/oberdiek/enparen.pdf
%    TDS:source/latex/oberdiek/enparen.dtx
%
%<*ignore>
\begingroup
  \catcode123=1 %
  \catcode125=2 %
  \def\x{LaTeX2e}%
\expandafter\endgroup
\ifcase 0\ifx\install y1\fi\expandafter
         \ifx\csname processbatchFile\endcsname\relax\else1\fi
         \ifx\fmtname\x\else 1\fi\relax
\else\csname fi\endcsname
%</ignore>
%<*install>
\input docstrip.tex
\Msg{************************************************************************}
\Msg{* Installation}
\Msg{* Package: enparen 2012/01/07 v1.0 Parentheses nesting (HO)}
\Msg{************************************************************************}

\keepsilent
\askforoverwritefalse

\let\MetaPrefix\relax
\preamble

This is a generated file.

Project: enparen
Version: 2012/01/07 v1.0

Copyright (C) 2012 by
   Heiko Oberdiek <heiko.oberdiek at googlemail.com>

This work may be distributed and/or modified under the
conditions of the LaTeX Project Public License, either
version 1.3c of this license or (at your option) any later
version. This version of this license is in
   http://www.latex-project.org/lppl/lppl-1-3c.txt
and the latest version of this license is in
   http://www.latex-project.org/lppl.txt
and version 1.3 or later is part of all distributions of
LaTeX version 2005/12/01 or later.

This work has the LPPL maintenance status "maintained".

This Current Maintainer of this work is Heiko Oberdiek.

This work consists of the main source file enparen.dtx
and the derived files
   enparen.sty, enparen.pdf, enparen.ins, enparen.drv.

\endpreamble
\let\MetaPrefix\DoubleperCent

\generate{%
  \file{enparen.ins}{\from{enparen.dtx}{install}}%
  \file{enparen.drv}{\from{enparen.dtx}{driver}}%
  \usedir{tex/latex/oberdiek}%
  \file{enparen.sty}{\from{enparen.dtx}{package}}%
}

\catcode32=13\relax% active space
\let =\space%
\Msg{************************************************************************}
\Msg{*}
\Msg{* To finish the installation you have to move the following}
\Msg{* file into a directory searched by TeX:}
\Msg{*}
\Msg{*     enparen.sty}
\Msg{*}
\Msg{* To produce the documentation run the file `enparen.drv'}
\Msg{* through LaTeX.}
\Msg{*}
\Msg{* Happy TeXing!}
\Msg{*}
\Msg{************************************************************************}

\endbatchfile
%</install>
%<*ignore>
\fi
%</ignore>
%<*driver>
\NeedsTeXFormat{LaTeX2e}
\ProvidesFile{enparen.drv}%
  [2012/01/07 v1.0 Parentheses nesting (HO)]%
\documentclass{ltxdoc}
\usepackage{holtxdoc}[2011/11/22]
\usepackage{enparen}[2012/01/07]
\begin{document}
  \DocInput{enparen.dtx}%
\end{document}
%</driver>
% \fi
%
% \CheckSum{339}
%
% \CharacterTable
%  {Upper-case    \A\B\C\D\E\F\G\H\I\J\K\L\M\N\O\P\Q\R\S\T\U\V\W\X\Y\Z
%   Lower-case    \a\b\c\d\e\f\g\h\i\j\k\l\m\n\o\p\q\r\s\t\u\v\w\x\y\z
%   Digits        \0\1\2\3\4\5\6\7\8\9
%   Exclamation   \!     Double quote  \"     Hash (number) \#
%   Dollar        \$     Percent       \%     Ampersand     \&
%   Acute accent  \'     Left paren    \(     Right paren   \)
%   Asterisk      \*     Plus          \+     Comma         \,
%   Minus         \-     Point         \.     Solidus       \/
%   Colon         \:     Semicolon     \;     Less than     \<
%   Equals        \=     Greater than  \>     Question mark \?
%   Commercial at \@     Left bracket  \[     Backslash     \\
%   Right bracket \]     Circumflex    \^     Underscore    \_
%   Grave accent  \`     Left brace    \{     Vertical bar  \|
%   Right brace   \}     Tilde         \~}
%
% \GetFileInfo{enparen.drv}
%
% \title{The \xpackage{enparen} package}
% \date{2012/01/07 v1.0}
% \author{Heiko Oberdiek\\\xemail{heiko.oberdiek at googlemail.com}}
%
% \maketitle
%
% \begin{abstract}
% The package defines macros to set parentheses that automatically
% change the symbols from inner to outer fences.
% \end{abstract}
%
% \tableofcontents
%
% \section{Documentation}
%
% The \hologo{LaTeX} package provides macros for automatically changed
% parentheses symbols depending on the fence order.
% The innermost parentheses are always using the same
% parentheses. The symbols changes for the outer fences.
%
% Example:
% \begin{quote}
%   \enparen{a \enparen{b} \enparen{c \enparen{d}} \enparen{e}}
% \end{quote}
% is generated by
%\begin{quote}
%\begin{verbatim}
%\documentclass{article}
%\usepackage{enparen}
%\begin{document}
%  \enparen{a \enparen{b} \enparen{c \enparen{d}} \enparen{e}}
%\end{document}
%\end{verbatim}
%\end{quote}
%
% The package is the result of a newsgroup thread. Dan gives
% the following specification \cite{dan}:
% \begin{quote}
% ``On the other hand, the rules for fences are usually the reverse:
% innermost fences are always |()|, next outer are |[]|, etc. This means
% the opening fence has to wait until all the fences between it and
% the matching close have been detected before it can decide whether
% to be |(| or |[| or |{|.''
% \end{quote}
%
% The fence level counting starts from innermost parentheses with one.
% For the next outer fences the level is increased by one.
% The example above with level indexes:
% \begin{quote}
%   \def\1{\ensuremath{_{1}}}
%   \def\2{\ensuremath{_{2}}}
%   \def\3{\ensuremath{_{3}}}
%   \enparen{\3a \enparen{\1b}\1 \enparen{\2c \enparen{\1d}\1}\2 \enparen{e}}\3
% \end{quote}
% The correct level is only known at the closing symbol. Therefore the
% correct value is remembered in the main \xext{aux} file and used
% in the second \hologo{LaTeX} run.
%
% \subsection{User macros}
%
% \begin{declcs}{enparen} \M{text}
% \end{declcs}
% The macro \cs{enparen} puts its argument \meta{text} in parentheses.
% If the macro is nested, the used fence symbols change for the
% outer fences.
%
% \begin{declcs}{enparenLeft}\\
%   \cs{enparenRight}
% \end{declcs}
% Instead of |\enparen{text}| the left and right symbol can be
% used separately, but in pairs:
% \begin{quote}
%   |\enparenLeft text\enparenRight|
% \end{quote}
% \cs{enparenLeft} and \cs{enparenRight} may be used at different
% group levels, but they must be properly nested.
%
% \begin{declcs}{enparenSetSymbols} \M{level} \M{opening symbol} \M{closing symbol}
% \end{declcs}
% Macro \cs{enparenSetSymbols} configures the \meta{opening symbol} and
% \meta{closing symbol} for the \meta{level}. The \meta{level}
% is a number and the counting starts with one. Level zero is used,
% if the correct level is not known (e.g. in the first \hologo{LaTeX} run).
% The package defines the following sets:
%\begin{quote}
%\begin{verbatim}
%\enparenSetSymbols{0}{(}{)}
%\enparenSetSymbols{1}{(}{)}
%\enparenSetSymbols{2}{[}{]}
%\enparenSetSymbols{3}{\{}{\}}
%\end{verbatim}
%\end{quote}
% Example for changing the third and adding a fourth level:
%\begin{quote}
%\begin{verbatim}
%\enparenSetSymbols{3}{\ensuremath{\langle}}{\ensuremath{\rangle}}
%\enparenSetSymbols{4}{\{}{\}}
%\end{verbatim}
%\end{quote}
%
% \begin{declcs}{enparenUnsetSymbols} \M{level}
% \end{declcs}
% The symbols for level \meta{level} are removed.
% Example scenario: Only two nesting levels must be used,
% the package defines more, then the third level can be
% disabled by |\enparenUnsetSymbols{3}| and the user
% gets warnings if parentheses at level 3 are needed.
%
% \subsection{Contexts}
%
% \begin{declcs}{enparenBeginContext} \M{name}\\
%   \cs{enparenEndContext} \M{name}
% \end{declcs}
% If the current text is interrupted by footnotes, floats
% with captions, then the parentheses inside the text of footnotes,
% captions, \dots should restart from scratch. This can be achieved
% by embedding the text inside macros \cs{enparenBeginContext}
% and \cs{enparenEndContext}. These macros must be properly nested.
% The \meta{name} for the begin and end macro must be the same.
% It is a help for debugging problems, because the warning messages
% show the context name. But it is not necessary that the
% begin\slash end pairs have different names. Example:
%\begin{quote}
%\begin{verbatim}
%\enparenLeft text before table ...
%\begin{table}
%  \caption{Table caption}
%  \enparenBegin{Context}{table}
%  Other text \enparen{foobar}.
%  \enparenEnd{Context}{table}
%\end{table}
%text after table ...
%\enparenRight
%\end{verbatim}
%\end{quote}
% The parentheses inside the table environment and context `table'
% are not nested inside other parentheses: \enparen{foobar}.
% In case of captions and footnotes the contexts are automatically
% added, see next section about options.
%
% \subsection{Options}
%
% \begin{declcs}{enparenSetup} \M{key value list}
% \end{declcs}
% Some options (currently all) can also be set after the
% package is loaded. They can be set in the argument
% \meta{key value list} of \cs{enparenSetup}. Options are disabled
% after they are used the last time. Currently all options are boolean
% options and are disabled in |\begin{document}|.
%
% \begin{description}
% \item[\xoption{caption}:] The caption text is put in a
%   context |caption|.
% \item[\xoption{footnote}:] The footnote text is put in a
%   context |footnote|.
% \end{description}
%
% Example for disabling the two options at different places:
%\begin{quote}
%\begin{verbatim}
%\usepackage[caption=false]{enparen}
%\enparenSetup{footnote=false}
%\end{verbatim}
%\end{quote}
%
% \subsection{Notes}
%
% \begin{description}
% \item[Implicite kerning:]
%   Unexpandable stuff might affect the implicite kerning.
%   The package cannot avoid this, because it need to define
%   and redefine macros at the occurence of each symbol.
%   This is done before the opening and after the closing
%   symbol, thus that the implicite kerning inside
%   is not affected.
% \end{description}
%
% \StopEventually{
% }
%
% \section{Implementation}
%
%    \begin{macrocode}
%<*package>
\NeedsTeXFormat{LaTeX2e}
\ProvidesPackage{enparen}
  [2012/01/07 v1.0 Parentheses nesting (HO)]%
%    \end{macrocode}
%
% \subsection{Resources}
%
%    \begin{macrocode}
\RequirePackage{ltxcmds}[2011/11/09]
%    \end{macrocode}
%
%    \begin{macrocode}
\ltx@IfUndefined{numexpr}{%
  \PackageError{enparen}{%
    Missing e-TeX's \ltx@backslashchar numexpr.\MessageBreak
    The package will continue with emergency definitions%
  }\@ehc
  \def\enparenLeft{(}%
  \def\enparenRight{)}%
  \long\def\enparen#1{\enparenLeft#1\enparenRight}%
  \let\enparenSetup\ltx@gobble
  \let\enparenSetSymbols\ltx@gobblethree
  \let\enparenUnsetSymbols\ltx@gobble
  \endinput
}{}
%    \end{macrocode}
%
%    \begin{macrocode}
\RequirePackage{protecteddef}[2011/01/31]
\RequirePackage{atveryend}[2011/06/30]
\RequirePackage{uniquecounter}[2011/01/30]
\RequirePackage{zref-base}[2011/03/18]
\RequirePackage{kvoptions}[2011/06/30]
\RequirePackage{kvsetkeys}[2011/10/18]
%    \end{macrocode}
%
%    \begin{macro}{\zref@wrapper@mainaux}
%    \begin{macrocode}
\providecommand{\zref@wrapper@mainaux}[1]{%
  \ifx\@auxout\@mainaux
    #1%
  \else
    \begingroup
      \let\@auxout\@mainaux
      #1%
    \endgroup
  \fi
}
%    \end{macrocode}
%    \end{macro}
%
%    \begin{macrocode}
\UniqueCounterNew{enparen}
%    \end{macrocode}
%
%    \begin{macrocode}
\zref@newprop{enparen}[0]{}
%    \end{macrocode}
%
% \subsection{Contexts}
%
%    \begin{macro}{\enparenContextDefault}
%    \begin{macrocode}
\def\enparenContextDefault{main}
%    \end{macrocode}
%    \end{macro}
%
%    \begin{macro}{\enparen@ctx}
%    \begin{macrocode}
\let\enparen@ctx\ltx@empty
%    \end{macrocode}
%    \end{macro}
%
%    \begin{macro}{\enparen@stack}
%    \begin{macrocode}
\let\enparen@stack\ltx@empty
%    \end{macrocode}
%    \end{macro}
%
% \subsubsection{Stack for contexts}
%
%    \begin{macro}{\enparen@CtxStack}
%    \begin{macrocode}
\global\let\enparen@CtxStack\ltx@empty
%    \end{macrocode}
%    \end{macro}
%
%    \begin{macro}{\enparen@CtxStackPush}
%    \begin{macrocode}
\def\enparen@CtxStackPush#1{%
  \xdef\enparen@CtxStack{%
    {\enparen@ctx}{\enparen@stack}%
    \enparen@CtxStack
  }%
  \xdef\enparen@ctx{#1}%
  \global\let\enparen@stack\ltx@empty
}
%    \end{macrocode}
%    \end{macro}
%
%    \begin{macro}{\enparen@CtxStackPop}
%    \begin{macrocode}
\def\enparen@CtxStackPop{%
  \ifx\enparen@CtxStack\ltx@empty
    \PackageWarning{enparen}{%
      Pop request for empty context stack%
    }%
    \global\let\enparen@ctx\enparenContextDefault
    \global\let\enparen@stack\ltx@empty
  \else
    \xdef\enparen@ctx{%
      \expandafter\ltx@car\enparen@CtxStack\@nil
    }%
    \xdef\enparen@stack{%
      \expandafter\ltx@carsecond\enparen@CtxStack\@nil
    }%
    \xdef\enparen@CtxStack{%
      \expandafter\ltx@cdrtwo\enparen@CtxStack\@nil
    }%
  \fi
}
%    \end{macrocode}
%    \end{macro}
%
% \subsubsection{Context user macros}
%
%    \begin{macro}{\enparenBeginContext}
%    \begin{macrocode}
\ProtectedDef*{\enparenBeginContext}[1]{%
  \enparen@CtxStackPush{#1}%
}
%    \end{macrocode}
%    \end{macro}
%    \begin{macro}{\enparenEndContext}
%    \begin{macrocode}
\ProtectedDef*{\enparenEndContext}[1]{%
  \edef\enparen@temp{#1}%
  \ifx\enparen@temp\enparen@ctx
  \else
    \PackageWarning{enparen}{%
      Context mismatch in end request.\MessageBreak
      `#1' should be ended, but current context\MessageBreak
      is `\enparen@ctx'%
    }%
  \fi
  \enparenCheckEmptyStack
  \enparen@CtxStackPop
}
%    \end{macrocode}
%    \end{macro}
%
%    \begin{macro}{\enparenCheckEmptyStack}
%    \begin{macrocode}
\ProtectedDef*{\enparenCheckEmptyStack}[0]{%
  \ifx\enparen@stack\ltx@empty
  \else
    \PackageWarning{enparen}{%
      Ending non-empty stack `\enparen@ctx':\MessageBreak
      \enparen@PrintStack\MessageBreak
    }%
  \fi
}
%    \end{macrocode}
%    \end{macro}
%
%    \begin{macro}{\enparen@PrintStack}
%    \begin{macrocode}
\def\enparen@PrintStack{%
  \expandafter\enparen@PrintStackAux
  \enparen@stack\ltx@empty\ltx@empty
}
%    \end{macrocode}
%    \end{macro}
%    \begin{macro}{\enparen@PrintStackAux}
%    \begin{macrocode}
\def\enparen@PrintStackAux#1#2{%
  \ifx\ltx@empty#1%
  \else
    {#1:#2}%
    \expandafter\enparen@PrintStackAux
  \fi
}
%    \end{macrocode}
%    \end{macro}
%
% \subsection{Symbols}
%
%    \begin{macro}{\enparenSetSymbols}
%    \begin{macrocode}
\ProtectedDef*{\enparenSetSymbols}[3]{%
  \expandafter
  \def\csname enparen@symbol\the\numexpr#1L\endcsname{#2}%
  \expandafter
  \def\csname enparen@symbol\the\numexpr#1R\endcsname{#3}%
}
%    \end{macrocode}
%    \end{macro}
%    \begin{macro}{\enparenUnsetSymbols}
%    \begin{macrocode}
\ProtectedDef*{\enparenUnsetSymbols}[1]{%
  \expandafter
  \let\csname enparen@symbol\the\numexpr#1L\endcsname\ltx@undefined
  \expandafter
  \let\csname enparen@symbol\the\numexpr#1R\endcsname\ltx@undefined
}
%    \end{macrocode}
%    \end{macro}
%
%    \begin{macrocode}
\enparenSetSymbols{0}{(}{)}
\enparenSetSymbols{1}{(}{)}
\enparenSetSymbols{2}{[}{]}
\enparenSetSymbols{3}{\{}{\}}
\enparenSetSymbols{4}{\ensuremath{\langle}}{\ensuremath{\rangle}}
%    \end{macrocode}
%
% \subsection{Main user macros}
%
%    \begin{macro}{\enparen}
%    \begin{macrocode}
\ProtectedDef{\enparen}[1]{%
  \enparenLeft#1\enparenRight
}
%    \end{macrocode}
%    \end{macro}
%
%    \begin{macro}{\enparenLeft}
%    \begin{macrocode}
\ProtectedDef*{\enparenLeft}[0]{%
  \UniqueCounterCall{enparen}\enparen@Left
}
%    \end{macrocode}
%    \end{macro}
%    \begin{macro}{\enparen@Left}
%    \begin{macrocode}
\def\enparen@Left#1{%
  \xdef\enparen@stack{%
    {#1}{1}%
    \expandafter\enparen@Inc\expandafter2\expandafter!%
    \enparen@stack\ltx@empty\ltx@empty
  }%
  \edef\enparen@tmp{\zref@extract{enparen#1}{enparen}}%
  \ltx@IfUndefined{enparen@symbol\enparen@tmp L}{%
    \PackageWarning{enparen}{%
      Undefined symbols for level \enparen@tmp
    }%
    \csname enparen@symbol0L\endcsname
  }{%
    \csname enparen@symbol\enparen@tmp L\endcsname
  }%
}
%    \end{macrocode}
%    \end{macro}
%    \begin{macro}{\enparen@Inc}
%    \begin{macrocode}
\def\enparen@Inc#1!#2#3{%
  \ifx\ltx@empty#2%
  \else
    \ifnum#3<#1 %
      {#2}{#1}%
      \expandafter\enparen@Inc
      \the\numexpr#1+1\expandafter\expandafter\expandafter!%
    \else
      {#2}{#3}%
    \fi
  \fi
}
%    \end{macrocode}
%    \end{macro}
%
%    \begin{macro}{\enparenRight}
%    \begin{macrocode}
\ProtectedDef*{\enparenRight}[0]{%
  \ifx\enparen@stack\ltx@empty
    \PackageWarning{enparen}{%
      Missing left symbol for right symbol%
    }%
    \csname enparen@symbol0R\endcsname
  \else
    \expandafter\enparen@Right\enparen@stack\@nil
  \fi
}
%    \end{macrocode}
%    \end{macro}
%
%    \begin{macro}{\enparen@Right}
%    \begin{macrocode}
\def\enparen@Right#1#2#3\@nil{%
  \ltx@IfUndefined{%
    enparen@symbol%
    \zref@extract{enparen#1}{enparen}%
    R%
  }{%
    \csname enparen@symbol0R\endcsname
  }{%
    \csname
      enparen@symbol%
      \zref@extract{enparen#1}{enparen}%
       R%
    \endcsname
  }%
  \zref@wrapper@mainaux{%
    \zref@setcurrent{enparen}{#2}%
    \zref@labelbyprops{enparen#1}{enparen}%
  }%
  \xdef\enparen@stack{#3}%
}
%    \end{macrocode}
%    \end{macro}
%
% \subsection{Options}
%
%    \begin{macrocode}
\SetupKeyvalOptions{%
  family=enparen,%
  prefix=enparen@,%
}
%    \end{macrocode}
%
%    \begin{macro}{\enparenSetup}
%    \begin{macrocode}
\ProtectedDef*{\enparenSetup}[0]{%
  \kvsetkeys{enparen}%
}
%    \end{macrocode}
%    \end{macro}
%
%    \begin{macrocode}
\DeclareBoolOption[true]{footnote}
\DeclareBoolOption[true]{caption}
\ProcessKeyvalOptions*
%    \end{macrocode}
%
% \subsection{Context settings}
%
%    \begin{macro}{\enparen@AtBegin}
%    \begin{macrocode}
\def\enparen@AtBegin{%
  \ifenparen@footnote
    \let\enparen@org@makefntext\@makefntext
    \long\def\@makefntext##1{%
      \enparen@org@makefntext{%
        \enparenBeginContext{footnote}%
        ##1%
        \enparenEndContext{footnote}%
      }%
    }%
  \fi
  \enparen@Disable{footnote}%
  \ifenparen@caption
    \let\enparen@org@makecaption\@makecaption
    \long\def\@makecaption##1##2{%
      \enparen@org@makecaption{##1}{%
        \enparenBeginContext{caption}%
        ##2%
        \enparenEndContext{caption}%
      }%
    }%
  \fi
  \enparen@Disable{caption}%
}
%    \end{macrocode}
%    \end{macro}
%
%    \begin{macro}{\enparen@Disable}
%    \begin{macrocode}
\def\enparen@Disable#1{%
  \DisableKeyvalOption[%
    action=warning,%
    package=enparen,%
  ]{enparen}{#1}%
}
%    \end{macrocode}
%    \end{macro}
%
%    \begin{macrocode}
\AtBeginDocument{\enparen@AtBegin}
%    \end{macrocode}
%
% \subsection{At end of document}
%
%    \begin{macro}{\enparen@AtEnd}
%    \begin{macrocode}
\def\enparen@AtEnd{%
  \enparenCheckEmptyStack
  \ifx\enparen@CtxStack\ltx@empty
  \else
    \PackageWarningNoLine{enparen}{%
      Context stack is not empty at end of document.\MessageBreak
      Current stack and contents of context stack:\MessageBreak
      [\enparen@ctx]:[\enparen@PrintStack]%
      \expandafter
      \enparen@PrintContextStack\enparen@CtxStack\relax\relax
    }%
  \fi
}
%    \end{macrocode}
%    \end{macro}
%    \begin{macro}{\enparen@PrintContextStack}
%    \begin{macrocode}
\def\enparen@PrintContextStack#1#2{%
  \ifx\relax#1\ltx@empty
  \else
    \MessageBreak
    [#1]:[\enparen@PrintStackAux#2\ltx@empty\ltx@empty]% hash-ok
    \expandafter\enparen@PrintContextStack
  \fi
}
%    \end{macrocode}
%    \end{macro}
%    \begin{macrocode}
\AtVeryEndDocument{\enparen@AtEnd}
%    \end{macrocode}
%
%    \begin{macrocode}
%</package>
%    \end{macrocode}
%
% \section{Installation}
%
% \subsection{Download}
%
% \paragraph{Package.} This package is available on
% CTAN\footnote{\url{ftp://ftp.ctan.org/tex-archive/}}:
% \begin{description}
% \item[\CTAN{macros/latex/contrib/oberdiek/enparen.dtx}] The source file.
% \item[\CTAN{macros/latex/contrib/oberdiek/enparen.pdf}] Documentation.
% \end{description}
%
%
% \paragraph{Bundle.} All the packages of the bundle `oberdiek'
% are also available in a TDS compliant ZIP archive. There
% the packages are already unpacked and the documentation files
% are generated. The files and directories obey the TDS standard.
% \begin{description}
% \item[\CTAN{install/macros/latex/contrib/oberdiek.tds.zip}]
% \end{description}
% \emph{TDS} refers to the standard ``A Directory Structure
% for \TeX\ Files'' (\CTAN{tds/tds.pdf}). Directories
% with \xfile{texmf} in their name are usually organized this way.
%
% \subsection{Bundle installation}
%
% \paragraph{Unpacking.} Unpack the \xfile{oberdiek.tds.zip} in the
% TDS tree (also known as \xfile{texmf} tree) of your choice.
% Example (linux):
% \begin{quote}
%   |unzip oberdiek.tds.zip -d ~/texmf|
% \end{quote}
%
% \paragraph{Script installation.}
% Check the directory \xfile{TDS:scripts/oberdiek/} for
% scripts that need further installation steps.
% Package \xpackage{attachfile2} comes with the Perl script
% \xfile{pdfatfi.pl} that should be installed in such a way
% that it can be called as \texttt{pdfatfi}.
% Example (linux):
% \begin{quote}
%   |chmod +x scripts/oberdiek/pdfatfi.pl|\\
%   |cp scripts/oberdiek/pdfatfi.pl /usr/local/bin/|
% \end{quote}
%
% \subsection{Package installation}
%
% \paragraph{Unpacking.} The \xfile{.dtx} file is a self-extracting
% \docstrip\ archive. The files are extracted by running the
% \xfile{.dtx} through \plainTeX:
% \begin{quote}
%   \verb|tex enparen.dtx|
% \end{quote}
%
% \paragraph{TDS.} Now the different files must be moved into
% the different directories in your installation TDS tree
% (also known as \xfile{texmf} tree):
% \begin{quote}
% \def\t{^^A
% \begin{tabular}{@{}>{\ttfamily}l@{ $\rightarrow$ }>{\ttfamily}l@{}}
%   enparen.sty & tex/latex/oberdiek/enparen.sty\\
%   enparen.pdf & doc/latex/oberdiek/enparen.pdf\\
%   enparen.dtx & source/latex/oberdiek/enparen.dtx\\
% \end{tabular}^^A
% }^^A
% \sbox0{\t}^^A
% \ifdim\wd0>\linewidth
%   \begingroup
%     \advance\linewidth by\leftmargin
%     \advance\linewidth by\rightmargin
%   \edef\x{\endgroup
%     \def\noexpand\lw{\the\linewidth}^^A
%   }\x
%   \def\lwbox{^^A
%     \leavevmode
%     \hbox to \linewidth{^^A
%       \kern-\leftmargin\relax
%       \hss
%       \usebox0
%       \hss
%       \kern-\rightmargin\relax
%     }^^A
%   }^^A
%   \ifdim\wd0>\lw
%     \sbox0{\small\t}^^A
%     \ifdim\wd0>\linewidth
%       \ifdim\wd0>\lw
%         \sbox0{\footnotesize\t}^^A
%         \ifdim\wd0>\linewidth
%           \ifdim\wd0>\lw
%             \sbox0{\scriptsize\t}^^A
%             \ifdim\wd0>\linewidth
%               \ifdim\wd0>\lw
%                 \sbox0{\tiny\t}^^A
%                 \ifdim\wd0>\linewidth
%                   \lwbox
%                 \else
%                   \usebox0
%                 \fi
%               \else
%                 \lwbox
%               \fi
%             \else
%               \usebox0
%             \fi
%           \else
%             \lwbox
%           \fi
%         \else
%           \usebox0
%         \fi
%       \else
%         \lwbox
%       \fi
%     \else
%       \usebox0
%     \fi
%   \else
%     \lwbox
%   \fi
% \else
%   \usebox0
% \fi
% \end{quote}
% If you have a \xfile{docstrip.cfg} that configures and enables \docstrip's
% TDS installing feature, then some files can already be in the right
% place, see the documentation of \docstrip.
%
% \subsection{Refresh file name databases}
%
% If your \TeX~distribution
% (\teTeX, \mikTeX, \dots) relies on file name databases, you must refresh
% these. For example, \teTeX\ users run \verb|texhash| or
% \verb|mktexlsr|.
%
% \subsection{Some details for the interested}
%
% \paragraph{Attached source.}
%
% The PDF documentation on CTAN also includes the
% \xfile{.dtx} source file. It can be extracted by
% AcrobatReader 6 or higher. Another option is \textsf{pdftk},
% e.g. unpack the file into the current directory:
% \begin{quote}
%   \verb|pdftk enparen.pdf unpack_files output .|
% \end{quote}
%
% \paragraph{Unpacking with \LaTeX.}
% The \xfile{.dtx} chooses its action depending on the format:
% \begin{description}
% \item[\plainTeX:] Run \docstrip\ and extract the files.
% \item[\LaTeX:] Generate the documentation.
% \end{description}
% If you insist on using \LaTeX\ for \docstrip\ (really,
% \docstrip\ does not need \LaTeX), then inform the autodetect routine
% about your intention:
% \begin{quote}
%   \verb|latex \let\install=y% \iffalse meta-comment
%
% File: enparen.dtx
% Version: 2012/01/07 v1.0
% Info: Parentheses nesting
%
% Copyright (C) 2012 by
%    Heiko Oberdiek <heiko.oberdiek at googlemail.com>
%
% This work may be distributed and/or modified under the
% conditions of the LaTeX Project Public License, either
% version 1.3c of this license or (at your option) any later
% version. This version of this license is in
%    http://www.latex-project.org/lppl/lppl-1-3c.txt
% and the latest version of this license is in
%    http://www.latex-project.org/lppl.txt
% and version 1.3 or later is part of all distributions of
% LaTeX version 2005/12/01 or later.
%
% This work has the LPPL maintenance status "maintained".
%
% This Current Maintainer of this work is Heiko Oberdiek.
%
% This work consists of the main source file enparen.dtx
% and the derived files
%    enparen.sty, enparen.pdf, enparen.ins, enparen.drv.
%
% Distribution:
%    CTAN:macros/latex/contrib/oberdiek/enparen.dtx
%    CTAN:macros/latex/contrib/oberdiek/enparen.pdf
%
% Unpacking:
%    (a) If enparen.ins is present:
%           tex enparen.ins
%    (b) Without enparen.ins:
%           tex enparen.dtx
%    (c) If you insist on using LaTeX
%           latex \let\install=y\input{enparen.dtx}
%        (quote the arguments according to the demands of your shell)
%
% Documentation:
%    (a) If enparen.drv is present:
%           latex enparen.drv
%    (b) Without enparen.drv:
%           latex enparen.dtx; ...
%    The class ltxdoc loads the configuration file ltxdoc.cfg
%    if available. Here you can specify further options, e.g.
%    use A4 as paper format:
%       \PassOptionsToClass{a4paper}{article}
%
%    Programm calls to get the documentation (example):
%       pdflatex enparen.dtx
%       makeindex -s gind.ist enparen.idx
%       pdflatex enparen.dtx
%       makeindex -s gind.ist enparen.idx
%       pdflatex enparen.dtx
%
% Installation:
%    TDS:tex/latex/oberdiek/enparen.sty
%    TDS:doc/latex/oberdiek/enparen.pdf
%    TDS:source/latex/oberdiek/enparen.dtx
%
%<*ignore>
\begingroup
  \catcode123=1 %
  \catcode125=2 %
  \def\x{LaTeX2e}%
\expandafter\endgroup
\ifcase 0\ifx\install y1\fi\expandafter
         \ifx\csname processbatchFile\endcsname\relax\else1\fi
         \ifx\fmtname\x\else 1\fi\relax
\else\csname fi\endcsname
%</ignore>
%<*install>
\input docstrip.tex
\Msg{************************************************************************}
\Msg{* Installation}
\Msg{* Package: enparen 2012/01/07 v1.0 Parentheses nesting (HO)}
\Msg{************************************************************************}

\keepsilent
\askforoverwritefalse

\let\MetaPrefix\relax
\preamble

This is a generated file.

Project: enparen
Version: 2012/01/07 v1.0

Copyright (C) 2012 by
   Heiko Oberdiek <heiko.oberdiek at googlemail.com>

This work may be distributed and/or modified under the
conditions of the LaTeX Project Public License, either
version 1.3c of this license or (at your option) any later
version. This version of this license is in
   http://www.latex-project.org/lppl/lppl-1-3c.txt
and the latest version of this license is in
   http://www.latex-project.org/lppl.txt
and version 1.3 or later is part of all distributions of
LaTeX version 2005/12/01 or later.

This work has the LPPL maintenance status "maintained".

This Current Maintainer of this work is Heiko Oberdiek.

This work consists of the main source file enparen.dtx
and the derived files
   enparen.sty, enparen.pdf, enparen.ins, enparen.drv.

\endpreamble
\let\MetaPrefix\DoubleperCent

\generate{%
  \file{enparen.ins}{\from{enparen.dtx}{install}}%
  \file{enparen.drv}{\from{enparen.dtx}{driver}}%
  \usedir{tex/latex/oberdiek}%
  \file{enparen.sty}{\from{enparen.dtx}{package}}%
}

\catcode32=13\relax% active space
\let =\space%
\Msg{************************************************************************}
\Msg{*}
\Msg{* To finish the installation you have to move the following}
\Msg{* file into a directory searched by TeX:}
\Msg{*}
\Msg{*     enparen.sty}
\Msg{*}
\Msg{* To produce the documentation run the file `enparen.drv'}
\Msg{* through LaTeX.}
\Msg{*}
\Msg{* Happy TeXing!}
\Msg{*}
\Msg{************************************************************************}

\endbatchfile
%</install>
%<*ignore>
\fi
%</ignore>
%<*driver>
\NeedsTeXFormat{LaTeX2e}
\ProvidesFile{enparen.drv}%
  [2012/01/07 v1.0 Parentheses nesting (HO)]%
\documentclass{ltxdoc}
\usepackage{holtxdoc}[2011/11/22]
\usepackage{enparen}[2012/01/07]
\begin{document}
  \DocInput{enparen.dtx}%
\end{document}
%</driver>
% \fi
%
% \CheckSum{339}
%
% \CharacterTable
%  {Upper-case    \A\B\C\D\E\F\G\H\I\J\K\L\M\N\O\P\Q\R\S\T\U\V\W\X\Y\Z
%   Lower-case    \a\b\c\d\e\f\g\h\i\j\k\l\m\n\o\p\q\r\s\t\u\v\w\x\y\z
%   Digits        \0\1\2\3\4\5\6\7\8\9
%   Exclamation   \!     Double quote  \"     Hash (number) \#
%   Dollar        \$     Percent       \%     Ampersand     \&
%   Acute accent  \'     Left paren    \(     Right paren   \)
%   Asterisk      \*     Plus          \+     Comma         \,
%   Minus         \-     Point         \.     Solidus       \/
%   Colon         \:     Semicolon     \;     Less than     \<
%   Equals        \=     Greater than  \>     Question mark \?
%   Commercial at \@     Left bracket  \[     Backslash     \\
%   Right bracket \]     Circumflex    \^     Underscore    \_
%   Grave accent  \`     Left brace    \{     Vertical bar  \|
%   Right brace   \}     Tilde         \~}
%
% \GetFileInfo{enparen.drv}
%
% \title{The \xpackage{enparen} package}
% \date{2012/01/07 v1.0}
% \author{Heiko Oberdiek\\\xemail{heiko.oberdiek at googlemail.com}}
%
% \maketitle
%
% \begin{abstract}
% The package defines macros to set parentheses that automatically
% change the symbols from inner to outer fences.
% \end{abstract}
%
% \tableofcontents
%
% \section{Documentation}
%
% The \hologo{LaTeX} package provides macros for automatically changed
% parentheses symbols depending on the fence order.
% The innermost parentheses are always using the same
% parentheses. The symbols changes for the outer fences.
%
% Example:
% \begin{quote}
%   \enparen{a \enparen{b} \enparen{c \enparen{d}} \enparen{e}}
% \end{quote}
% is generated by
%\begin{quote}
%\begin{verbatim}
%\documentclass{article}
%\usepackage{enparen}
%\begin{document}
%  \enparen{a \enparen{b} \enparen{c \enparen{d}} \enparen{e}}
%\end{document}
%\end{verbatim}
%\end{quote}
%
% The package is the result of a newsgroup thread. Dan gives
% the following specification \cite{dan}:
% \begin{quote}
% ``On the other hand, the rules for fences are usually the reverse:
% innermost fences are always |()|, next outer are |[]|, etc. This means
% the opening fence has to wait until all the fences between it and
% the matching close have been detected before it can decide whether
% to be |(| or |[| or |{|.''
% \end{quote}
%
% The fence level counting starts from innermost parentheses with one.
% For the next outer fences the level is increased by one.
% The example above with level indexes:
% \begin{quote}
%   \def\1{\ensuremath{_{1}}}
%   \def\2{\ensuremath{_{2}}}
%   \def\3{\ensuremath{_{3}}}
%   \enparen{\3a \enparen{\1b}\1 \enparen{\2c \enparen{\1d}\1}\2 \enparen{e}}\3
% \end{quote}
% The correct level is only known at the closing symbol. Therefore the
% correct value is remembered in the main \xext{aux} file and used
% in the second \hologo{LaTeX} run.
%
% \subsection{User macros}
%
% \begin{declcs}{enparen} \M{text}
% \end{declcs}
% The macro \cs{enparen} puts its argument \meta{text} in parentheses.
% If the macro is nested, the used fence symbols change for the
% outer fences.
%
% \begin{declcs}{enparenLeft}\\
%   \cs{enparenRight}
% \end{declcs}
% Instead of |\enparen{text}| the left and right symbol can be
% used separately, but in pairs:
% \begin{quote}
%   |\enparenLeft text\enparenRight|
% \end{quote}
% \cs{enparenLeft} and \cs{enparenRight} may be used at different
% group levels, but they must be properly nested.
%
% \begin{declcs}{enparenSetSymbols} \M{level} \M{opening symbol} \M{closing symbol}
% \end{declcs}
% Macro \cs{enparenSetSymbols} configures the \meta{opening symbol} and
% \meta{closing symbol} for the \meta{level}. The \meta{level}
% is a number and the counting starts with one. Level zero is used,
% if the correct level is not known (e.g. in the first \hologo{LaTeX} run).
% The package defines the following sets:
%\begin{quote}
%\begin{verbatim}
%\enparenSetSymbols{0}{(}{)}
%\enparenSetSymbols{1}{(}{)}
%\enparenSetSymbols{2}{[}{]}
%\enparenSetSymbols{3}{\{}{\}}
%\end{verbatim}
%\end{quote}
% Example for changing the third and adding a fourth level:
%\begin{quote}
%\begin{verbatim}
%\enparenSetSymbols{3}{\ensuremath{\langle}}{\ensuremath{\rangle}}
%\enparenSetSymbols{4}{\{}{\}}
%\end{verbatim}
%\end{quote}
%
% \begin{declcs}{enparenUnsetSymbols} \M{level}
% \end{declcs}
% The symbols for level \meta{level} are removed.
% Example scenario: Only two nesting levels must be used,
% the package defines more, then the third level can be
% disabled by |\enparenUnsetSymbols{3}| and the user
% gets warnings if parentheses at level 3 are needed.
%
% \subsection{Contexts}
%
% \begin{declcs}{enparenBeginContext} \M{name}\\
%   \cs{enparenEndContext} \M{name}
% \end{declcs}
% If the current text is interrupted by footnotes, floats
% with captions, then the parentheses inside the text of footnotes,
% captions, \dots should restart from scratch. This can be achieved
% by embedding the text inside macros \cs{enparenBeginContext}
% and \cs{enparenEndContext}. These macros must be properly nested.
% The \meta{name} for the begin and end macro must be the same.
% It is a help for debugging problems, because the warning messages
% show the context name. But it is not necessary that the
% begin\slash end pairs have different names. Example:
%\begin{quote}
%\begin{verbatim}
%\enparenLeft text before table ...
%\begin{table}
%  \caption{Table caption}
%  \enparenBegin{Context}{table}
%  Other text \enparen{foobar}.
%  \enparenEnd{Context}{table}
%\end{table}
%text after table ...
%\enparenRight
%\end{verbatim}
%\end{quote}
% The parentheses inside the table environment and context `table'
% are not nested inside other parentheses: \enparen{foobar}.
% In case of captions and footnotes the contexts are automatically
% added, see next section about options.
%
% \subsection{Options}
%
% \begin{declcs}{enparenSetup} \M{key value list}
% \end{declcs}
% Some options (currently all) can also be set after the
% package is loaded. They can be set in the argument
% \meta{key value list} of \cs{enparenSetup}. Options are disabled
% after they are used the last time. Currently all options are boolean
% options and are disabled in |\begin{document}|.
%
% \begin{description}
% \item[\xoption{caption}:] The caption text is put in a
%   context |caption|.
% \item[\xoption{footnote}:] The footnote text is put in a
%   context |footnote|.
% \end{description}
%
% Example for disabling the two options at different places:
%\begin{quote}
%\begin{verbatim}
%\usepackage[caption=false]{enparen}
%\enparenSetup{footnote=false}
%\end{verbatim}
%\end{quote}
%
% \subsection{Notes}
%
% \begin{description}
% \item[Implicite kerning:]
%   Unexpandable stuff might affect the implicite kerning.
%   The package cannot avoid this, because it need to define
%   and redefine macros at the occurence of each symbol.
%   This is done before the opening and after the closing
%   symbol, thus that the implicite kerning inside
%   is not affected.
% \end{description}
%
% \StopEventually{
% }
%
% \section{Implementation}
%
%    \begin{macrocode}
%<*package>
\NeedsTeXFormat{LaTeX2e}
\ProvidesPackage{enparen}
  [2012/01/07 v1.0 Parentheses nesting (HO)]%
%    \end{macrocode}
%
% \subsection{Resources}
%
%    \begin{macrocode}
\RequirePackage{ltxcmds}[2011/11/09]
%    \end{macrocode}
%
%    \begin{macrocode}
\ltx@IfUndefined{numexpr}{%
  \PackageError{enparen}{%
    Missing e-TeX's \ltx@backslashchar numexpr.\MessageBreak
    The package will continue with emergency definitions%
  }\@ehc
  \def\enparenLeft{(}%
  \def\enparenRight{)}%
  \long\def\enparen#1{\enparenLeft#1\enparenRight}%
  \let\enparenSetup\ltx@gobble
  \let\enparenSetSymbols\ltx@gobblethree
  \let\enparenUnsetSymbols\ltx@gobble
  \endinput
}{}
%    \end{macrocode}
%
%    \begin{macrocode}
\RequirePackage{protecteddef}[2011/01/31]
\RequirePackage{atveryend}[2011/06/30]
\RequirePackage{uniquecounter}[2011/01/30]
\RequirePackage{zref-base}[2011/03/18]
\RequirePackage{kvoptions}[2011/06/30]
\RequirePackage{kvsetkeys}[2011/10/18]
%    \end{macrocode}
%
%    \begin{macro}{\zref@wrapper@mainaux}
%    \begin{macrocode}
\providecommand{\zref@wrapper@mainaux}[1]{%
  \ifx\@auxout\@mainaux
    #1%
  \else
    \begingroup
      \let\@auxout\@mainaux
      #1%
    \endgroup
  \fi
}
%    \end{macrocode}
%    \end{macro}
%
%    \begin{macrocode}
\UniqueCounterNew{enparen}
%    \end{macrocode}
%
%    \begin{macrocode}
\zref@newprop{enparen}[0]{}
%    \end{macrocode}
%
% \subsection{Contexts}
%
%    \begin{macro}{\enparenContextDefault}
%    \begin{macrocode}
\def\enparenContextDefault{main}
%    \end{macrocode}
%    \end{macro}
%
%    \begin{macro}{\enparen@ctx}
%    \begin{macrocode}
\let\enparen@ctx\ltx@empty
%    \end{macrocode}
%    \end{macro}
%
%    \begin{macro}{\enparen@stack}
%    \begin{macrocode}
\let\enparen@stack\ltx@empty
%    \end{macrocode}
%    \end{macro}
%
% \subsubsection{Stack for contexts}
%
%    \begin{macro}{\enparen@CtxStack}
%    \begin{macrocode}
\global\let\enparen@CtxStack\ltx@empty
%    \end{macrocode}
%    \end{macro}
%
%    \begin{macro}{\enparen@CtxStackPush}
%    \begin{macrocode}
\def\enparen@CtxStackPush#1{%
  \xdef\enparen@CtxStack{%
    {\enparen@ctx}{\enparen@stack}%
    \enparen@CtxStack
  }%
  \xdef\enparen@ctx{#1}%
  \global\let\enparen@stack\ltx@empty
}
%    \end{macrocode}
%    \end{macro}
%
%    \begin{macro}{\enparen@CtxStackPop}
%    \begin{macrocode}
\def\enparen@CtxStackPop{%
  \ifx\enparen@CtxStack\ltx@empty
    \PackageWarning{enparen}{%
      Pop request for empty context stack%
    }%
    \global\let\enparen@ctx\enparenContextDefault
    \global\let\enparen@stack\ltx@empty
  \else
    \xdef\enparen@ctx{%
      \expandafter\ltx@car\enparen@CtxStack\@nil
    }%
    \xdef\enparen@stack{%
      \expandafter\ltx@carsecond\enparen@CtxStack\@nil
    }%
    \xdef\enparen@CtxStack{%
      \expandafter\ltx@cdrtwo\enparen@CtxStack\@nil
    }%
  \fi
}
%    \end{macrocode}
%    \end{macro}
%
% \subsubsection{Context user macros}
%
%    \begin{macro}{\enparenBeginContext}
%    \begin{macrocode}
\ProtectedDef*{\enparenBeginContext}[1]{%
  \enparen@CtxStackPush{#1}%
}
%    \end{macrocode}
%    \end{macro}
%    \begin{macro}{\enparenEndContext}
%    \begin{macrocode}
\ProtectedDef*{\enparenEndContext}[1]{%
  \edef\enparen@temp{#1}%
  \ifx\enparen@temp\enparen@ctx
  \else
    \PackageWarning{enparen}{%
      Context mismatch in end request.\MessageBreak
      `#1' should be ended, but current context\MessageBreak
      is `\enparen@ctx'%
    }%
  \fi
  \enparenCheckEmptyStack
  \enparen@CtxStackPop
}
%    \end{macrocode}
%    \end{macro}
%
%    \begin{macro}{\enparenCheckEmptyStack}
%    \begin{macrocode}
\ProtectedDef*{\enparenCheckEmptyStack}[0]{%
  \ifx\enparen@stack\ltx@empty
  \else
    \PackageWarning{enparen}{%
      Ending non-empty stack `\enparen@ctx':\MessageBreak
      \enparen@PrintStack\MessageBreak
    }%
  \fi
}
%    \end{macrocode}
%    \end{macro}
%
%    \begin{macro}{\enparen@PrintStack}
%    \begin{macrocode}
\def\enparen@PrintStack{%
  \expandafter\enparen@PrintStackAux
  \enparen@stack\ltx@empty\ltx@empty
}
%    \end{macrocode}
%    \end{macro}
%    \begin{macro}{\enparen@PrintStackAux}
%    \begin{macrocode}
\def\enparen@PrintStackAux#1#2{%
  \ifx\ltx@empty#1%
  \else
    {#1:#2}%
    \expandafter\enparen@PrintStackAux
  \fi
}
%    \end{macrocode}
%    \end{macro}
%
% \subsection{Symbols}
%
%    \begin{macro}{\enparenSetSymbols}
%    \begin{macrocode}
\ProtectedDef*{\enparenSetSymbols}[3]{%
  \expandafter
  \def\csname enparen@symbol\the\numexpr#1L\endcsname{#2}%
  \expandafter
  \def\csname enparen@symbol\the\numexpr#1R\endcsname{#3}%
}
%    \end{macrocode}
%    \end{macro}
%    \begin{macro}{\enparenUnsetSymbols}
%    \begin{macrocode}
\ProtectedDef*{\enparenUnsetSymbols}[1]{%
  \expandafter
  \let\csname enparen@symbol\the\numexpr#1L\endcsname\ltx@undefined
  \expandafter
  \let\csname enparen@symbol\the\numexpr#1R\endcsname\ltx@undefined
}
%    \end{macrocode}
%    \end{macro}
%
%    \begin{macrocode}
\enparenSetSymbols{0}{(}{)}
\enparenSetSymbols{1}{(}{)}
\enparenSetSymbols{2}{[}{]}
\enparenSetSymbols{3}{\{}{\}}
\enparenSetSymbols{4}{\ensuremath{\langle}}{\ensuremath{\rangle}}
%    \end{macrocode}
%
% \subsection{Main user macros}
%
%    \begin{macro}{\enparen}
%    \begin{macrocode}
\ProtectedDef{\enparen}[1]{%
  \enparenLeft#1\enparenRight
}
%    \end{macrocode}
%    \end{macro}
%
%    \begin{macro}{\enparenLeft}
%    \begin{macrocode}
\ProtectedDef*{\enparenLeft}[0]{%
  \UniqueCounterCall{enparen}\enparen@Left
}
%    \end{macrocode}
%    \end{macro}
%    \begin{macro}{\enparen@Left}
%    \begin{macrocode}
\def\enparen@Left#1{%
  \xdef\enparen@stack{%
    {#1}{1}%
    \expandafter\enparen@Inc\expandafter2\expandafter!%
    \enparen@stack\ltx@empty\ltx@empty
  }%
  \edef\enparen@tmp{\zref@extract{enparen#1}{enparen}}%
  \ltx@IfUndefined{enparen@symbol\enparen@tmp L}{%
    \PackageWarning{enparen}{%
      Undefined symbols for level \enparen@tmp
    }%
    \csname enparen@symbol0L\endcsname
  }{%
    \csname enparen@symbol\enparen@tmp L\endcsname
  }%
}
%    \end{macrocode}
%    \end{macro}
%    \begin{macro}{\enparen@Inc}
%    \begin{macrocode}
\def\enparen@Inc#1!#2#3{%
  \ifx\ltx@empty#2%
  \else
    \ifnum#3<#1 %
      {#2}{#1}%
      \expandafter\enparen@Inc
      \the\numexpr#1+1\expandafter\expandafter\expandafter!%
    \else
      {#2}{#3}%
    \fi
  \fi
}
%    \end{macrocode}
%    \end{macro}
%
%    \begin{macro}{\enparenRight}
%    \begin{macrocode}
\ProtectedDef*{\enparenRight}[0]{%
  \ifx\enparen@stack\ltx@empty
    \PackageWarning{enparen}{%
      Missing left symbol for right symbol%
    }%
    \csname enparen@symbol0R\endcsname
  \else
    \expandafter\enparen@Right\enparen@stack\@nil
  \fi
}
%    \end{macrocode}
%    \end{macro}
%
%    \begin{macro}{\enparen@Right}
%    \begin{macrocode}
\def\enparen@Right#1#2#3\@nil{%
  \ltx@IfUndefined{%
    enparen@symbol%
    \zref@extract{enparen#1}{enparen}%
    R%
  }{%
    \csname enparen@symbol0R\endcsname
  }{%
    \csname
      enparen@symbol%
      \zref@extract{enparen#1}{enparen}%
       R%
    \endcsname
  }%
  \zref@wrapper@mainaux{%
    \zref@setcurrent{enparen}{#2}%
    \zref@labelbyprops{enparen#1}{enparen}%
  }%
  \xdef\enparen@stack{#3}%
}
%    \end{macrocode}
%    \end{macro}
%
% \subsection{Options}
%
%    \begin{macrocode}
\SetupKeyvalOptions{%
  family=enparen,%
  prefix=enparen@,%
}
%    \end{macrocode}
%
%    \begin{macro}{\enparenSetup}
%    \begin{macrocode}
\ProtectedDef*{\enparenSetup}[0]{%
  \kvsetkeys{enparen}%
}
%    \end{macrocode}
%    \end{macro}
%
%    \begin{macrocode}
\DeclareBoolOption[true]{footnote}
\DeclareBoolOption[true]{caption}
\ProcessKeyvalOptions*
%    \end{macrocode}
%
% \subsection{Context settings}
%
%    \begin{macro}{\enparen@AtBegin}
%    \begin{macrocode}
\def\enparen@AtBegin{%
  \ifenparen@footnote
    \let\enparen@org@makefntext\@makefntext
    \long\def\@makefntext##1{%
      \enparen@org@makefntext{%
        \enparenBeginContext{footnote}%
        ##1%
        \enparenEndContext{footnote}%
      }%
    }%
  \fi
  \enparen@Disable{footnote}%
  \ifenparen@caption
    \let\enparen@org@makecaption\@makecaption
    \long\def\@makecaption##1##2{%
      \enparen@org@makecaption{##1}{%
        \enparenBeginContext{caption}%
        ##2%
        \enparenEndContext{caption}%
      }%
    }%
  \fi
  \enparen@Disable{caption}%
}
%    \end{macrocode}
%    \end{macro}
%
%    \begin{macro}{\enparen@Disable}
%    \begin{macrocode}
\def\enparen@Disable#1{%
  \DisableKeyvalOption[%
    action=warning,%
    package=enparen,%
  ]{enparen}{#1}%
}
%    \end{macrocode}
%    \end{macro}
%
%    \begin{macrocode}
\AtBeginDocument{\enparen@AtBegin}
%    \end{macrocode}
%
% \subsection{At end of document}
%
%    \begin{macro}{\enparen@AtEnd}
%    \begin{macrocode}
\def\enparen@AtEnd{%
  \enparenCheckEmptyStack
  \ifx\enparen@CtxStack\ltx@empty
  \else
    \PackageWarningNoLine{enparen}{%
      Context stack is not empty at end of document.\MessageBreak
      Current stack and contents of context stack:\MessageBreak
      [\enparen@ctx]:[\enparen@PrintStack]%
      \expandafter
      \enparen@PrintContextStack\enparen@CtxStack\relax\relax
    }%
  \fi
}
%    \end{macrocode}
%    \end{macro}
%    \begin{macro}{\enparen@PrintContextStack}
%    \begin{macrocode}
\def\enparen@PrintContextStack#1#2{%
  \ifx\relax#1\ltx@empty
  \else
    \MessageBreak
    [#1]:[\enparen@PrintStackAux#2\ltx@empty\ltx@empty]% hash-ok
    \expandafter\enparen@PrintContextStack
  \fi
}
%    \end{macrocode}
%    \end{macro}
%    \begin{macrocode}
\AtVeryEndDocument{\enparen@AtEnd}
%    \end{macrocode}
%
%    \begin{macrocode}
%</package>
%    \end{macrocode}
%
% \section{Installation}
%
% \subsection{Download}
%
% \paragraph{Package.} This package is available on
% CTAN\footnote{\url{ftp://ftp.ctan.org/tex-archive/}}:
% \begin{description}
% \item[\CTAN{macros/latex/contrib/oberdiek/enparen.dtx}] The source file.
% \item[\CTAN{macros/latex/contrib/oberdiek/enparen.pdf}] Documentation.
% \end{description}
%
%
% \paragraph{Bundle.} All the packages of the bundle `oberdiek'
% are also available in a TDS compliant ZIP archive. There
% the packages are already unpacked and the documentation files
% are generated. The files and directories obey the TDS standard.
% \begin{description}
% \item[\CTAN{install/macros/latex/contrib/oberdiek.tds.zip}]
% \end{description}
% \emph{TDS} refers to the standard ``A Directory Structure
% for \TeX\ Files'' (\CTAN{tds/tds.pdf}). Directories
% with \xfile{texmf} in their name are usually organized this way.
%
% \subsection{Bundle installation}
%
% \paragraph{Unpacking.} Unpack the \xfile{oberdiek.tds.zip} in the
% TDS tree (also known as \xfile{texmf} tree) of your choice.
% Example (linux):
% \begin{quote}
%   |unzip oberdiek.tds.zip -d ~/texmf|
% \end{quote}
%
% \paragraph{Script installation.}
% Check the directory \xfile{TDS:scripts/oberdiek/} for
% scripts that need further installation steps.
% Package \xpackage{attachfile2} comes with the Perl script
% \xfile{pdfatfi.pl} that should be installed in such a way
% that it can be called as \texttt{pdfatfi}.
% Example (linux):
% \begin{quote}
%   |chmod +x scripts/oberdiek/pdfatfi.pl|\\
%   |cp scripts/oberdiek/pdfatfi.pl /usr/local/bin/|
% \end{quote}
%
% \subsection{Package installation}
%
% \paragraph{Unpacking.} The \xfile{.dtx} file is a self-extracting
% \docstrip\ archive. The files are extracted by running the
% \xfile{.dtx} through \plainTeX:
% \begin{quote}
%   \verb|tex enparen.dtx|
% \end{quote}
%
% \paragraph{TDS.} Now the different files must be moved into
% the different directories in your installation TDS tree
% (also known as \xfile{texmf} tree):
% \begin{quote}
% \def\t{^^A
% \begin{tabular}{@{}>{\ttfamily}l@{ $\rightarrow$ }>{\ttfamily}l@{}}
%   enparen.sty & tex/latex/oberdiek/enparen.sty\\
%   enparen.pdf & doc/latex/oberdiek/enparen.pdf\\
%   enparen.dtx & source/latex/oberdiek/enparen.dtx\\
% \end{tabular}^^A
% }^^A
% \sbox0{\t}^^A
% \ifdim\wd0>\linewidth
%   \begingroup
%     \advance\linewidth by\leftmargin
%     \advance\linewidth by\rightmargin
%   \edef\x{\endgroup
%     \def\noexpand\lw{\the\linewidth}^^A
%   }\x
%   \def\lwbox{^^A
%     \leavevmode
%     \hbox to \linewidth{^^A
%       \kern-\leftmargin\relax
%       \hss
%       \usebox0
%       \hss
%       \kern-\rightmargin\relax
%     }^^A
%   }^^A
%   \ifdim\wd0>\lw
%     \sbox0{\small\t}^^A
%     \ifdim\wd0>\linewidth
%       \ifdim\wd0>\lw
%         \sbox0{\footnotesize\t}^^A
%         \ifdim\wd0>\linewidth
%           \ifdim\wd0>\lw
%             \sbox0{\scriptsize\t}^^A
%             \ifdim\wd0>\linewidth
%               \ifdim\wd0>\lw
%                 \sbox0{\tiny\t}^^A
%                 \ifdim\wd0>\linewidth
%                   \lwbox
%                 \else
%                   \usebox0
%                 \fi
%               \else
%                 \lwbox
%               \fi
%             \else
%               \usebox0
%             \fi
%           \else
%             \lwbox
%           \fi
%         \else
%           \usebox0
%         \fi
%       \else
%         \lwbox
%       \fi
%     \else
%       \usebox0
%     \fi
%   \else
%     \lwbox
%   \fi
% \else
%   \usebox0
% \fi
% \end{quote}
% If you have a \xfile{docstrip.cfg} that configures and enables \docstrip's
% TDS installing feature, then some files can already be in the right
% place, see the documentation of \docstrip.
%
% \subsection{Refresh file name databases}
%
% If your \TeX~distribution
% (\teTeX, \mikTeX, \dots) relies on file name databases, you must refresh
% these. For example, \teTeX\ users run \verb|texhash| or
% \verb|mktexlsr|.
%
% \subsection{Some details for the interested}
%
% \paragraph{Attached source.}
%
% The PDF documentation on CTAN also includes the
% \xfile{.dtx} source file. It can be extracted by
% AcrobatReader 6 or higher. Another option is \textsf{pdftk},
% e.g. unpack the file into the current directory:
% \begin{quote}
%   \verb|pdftk enparen.pdf unpack_files output .|
% \end{quote}
%
% \paragraph{Unpacking with \LaTeX.}
% The \xfile{.dtx} chooses its action depending on the format:
% \begin{description}
% \item[\plainTeX:] Run \docstrip\ and extract the files.
% \item[\LaTeX:] Generate the documentation.
% \end{description}
% If you insist on using \LaTeX\ for \docstrip\ (really,
% \docstrip\ does not need \LaTeX), then inform the autodetect routine
% about your intention:
% \begin{quote}
%   \verb|latex \let\install=y\input{enparen.dtx}|
% \end{quote}
% Do not forget to quote the argument according to the demands
% of your shell.
%
% \paragraph{Generating the documentation.}
% You can use both the \xfile{.dtx} or the \xfile{.drv} to generate
% the documentation. The process can be configured by the
% configuration file \xfile{ltxdoc.cfg}. For instance, put this
% line into this file, if you want to have A4 as paper format:
% \begin{quote}
%   \verb|\PassOptionsToClass{a4paper}{article}|
% \end{quote}
% An example follows how to generate the
% documentation with pdf\LaTeX:
% \begin{quote}
%\begin{verbatim}
%pdflatex enparen.dtx
%makeindex -s gind.ist enparen.idx
%pdflatex enparen.dtx
%makeindex -s gind.ist enparen.idx
%pdflatex enparen.dtx
%\end{verbatim}
% \end{quote}
%
% \begin{thebibliography}{9}
% \raggedright
% \bibitem{dan}
%   Dan Luecking: \textit{Re: bracket order};
%   newsgroup \xnewsgroup{comp.text.tex};
%   message id \texttt{\small$\langle$9b07c9c8-ff92-4cbf-b3a9-84efecfeb506@l24g2000yqm.googlegroups.com$\rangle$}
%   2012-01-05.
%   \url{http://groups.google.com/group/comp.text.tex/msg/8774519da31c2352}
% \end{thebibliography}
%
% \begin{History}
%   \begin{Version}{2012/01/07 v1.0}
%   \item
%     First version.
%   \end{Version}
% \end{History}
%
% \PrintIndex
%
% \Finale
\endinput
|
% \end{quote}
% Do not forget to quote the argument according to the demands
% of your shell.
%
% \paragraph{Generating the documentation.}
% You can use both the \xfile{.dtx} or the \xfile{.drv} to generate
% the documentation. The process can be configured by the
% configuration file \xfile{ltxdoc.cfg}. For instance, put this
% line into this file, if you want to have A4 as paper format:
% \begin{quote}
%   \verb|\PassOptionsToClass{a4paper}{article}|
% \end{quote}
% An example follows how to generate the
% documentation with pdf\LaTeX:
% \begin{quote}
%\begin{verbatim}
%pdflatex enparen.dtx
%makeindex -s gind.ist enparen.idx
%pdflatex enparen.dtx
%makeindex -s gind.ist enparen.idx
%pdflatex enparen.dtx
%\end{verbatim}
% \end{quote}
%
% \begin{thebibliography}{9}
% \raggedright
% \bibitem{dan}
%   Dan Luecking: \textit{Re: bracket order};
%   newsgroup \xnewsgroup{comp.text.tex};
%   message id \texttt{\small$\langle$9b07c9c8-ff92-4cbf-b3a9-84efecfeb506@l24g2000yqm.googlegroups.com$\rangle$}
%   2012-01-05.
%   \url{http://groups.google.com/group/comp.text.tex/msg/8774519da31c2352}
% \end{thebibliography}
%
% \begin{History}
%   \begin{Version}{2012/01/07 v1.0}
%   \item
%     First version.
%   \end{Version}
% \end{History}
%
% \PrintIndex
%
% \Finale
\endinput
|
% \end{quote}
% Do not forget to quote the argument according to the demands
% of your shell.
%
% \paragraph{Generating the documentation.}
% You can use both the \xfile{.dtx} or the \xfile{.drv} to generate
% the documentation. The process can be configured by the
% configuration file \xfile{ltxdoc.cfg}. For instance, put this
% line into this file, if you want to have A4 as paper format:
% \begin{quote}
%   \verb|\PassOptionsToClass{a4paper}{article}|
% \end{quote}
% An example follows how to generate the
% documentation with pdf\LaTeX:
% \begin{quote}
%\begin{verbatim}
%pdflatex enparen.dtx
%makeindex -s gind.ist enparen.idx
%pdflatex enparen.dtx
%makeindex -s gind.ist enparen.idx
%pdflatex enparen.dtx
%\end{verbatim}
% \end{quote}
%
% \begin{thebibliography}{9}
% \raggedright
% \bibitem{dan}
%   Dan Luecking: \textit{Re: bracket order};
%   newsgroup \xnewsgroup{comp.text.tex};
%   message id \texttt{\small$\langle$9b07c9c8-ff92-4cbf-b3a9-84efecfeb506@l24g2000yqm.googlegroups.com$\rangle$}
%   2012-01-05.
%   \url{http://groups.google.com/group/comp.text.tex/msg/8774519da31c2352}
% \end{thebibliography}
%
% \begin{History}
%   \begin{Version}{2012/01/07 v1.0}
%   \item
%     First version.
%   \end{Version}
% \end{History}
%
% \PrintIndex
%
% \Finale
\endinput
|
% \end{quote}
% Do not forget to quote the argument according to the demands
% of your shell.
%
% \paragraph{Generating the documentation.}
% You can use both the \xfile{.dtx} or the \xfile{.drv} to generate
% the documentation. The process can be configured by the
% configuration file \xfile{ltxdoc.cfg}. For instance, put this
% line into this file, if you want to have A4 as paper format:
% \begin{quote}
%   \verb|\PassOptionsToClass{a4paper}{article}|
% \end{quote}
% An example follows how to generate the
% documentation with pdf\LaTeX:
% \begin{quote}
%\begin{verbatim}
%pdflatex enparen.dtx
%makeindex -s gind.ist enparen.idx
%pdflatex enparen.dtx
%makeindex -s gind.ist enparen.idx
%pdflatex enparen.dtx
%\end{verbatim}
% \end{quote}
%
% \begin{thebibliography}{9}
% \raggedright
% \bibitem{dan}
%   Dan Luecking: \textit{Re: bracket order};
%   newsgroup \xnewsgroup{comp.text.tex};
%   message id \texttt{\small$\langle$9b07c9c8-ff92-4cbf-b3a9-84efecfeb506@l24g2000yqm.googlegroups.com$\rangle$}
%   2012-01-05.
%   \url{http://groups.google.com/group/comp.text.tex/msg/8774519da31c2352}
% \end{thebibliography}
%
% \begin{History}
%   \begin{Version}{2012/01/07 v1.0}
%   \item
%     First version.
%   \end{Version}
% \end{History}
%
% \PrintIndex
%
% \Finale
\endinput
